\documentclass{article}
\usepackage[backend=biber,natbib=true,style=alphabetic,maxbibnames=10]{biblatex}
\addbibresource{/home/nqbh/reference/bib.bib}
\usepackage[utf8]{vietnam}
\usepackage{tocloft}
\renewcommand{\cftsecleader}{\cftdotfill{\cftdotsep}}
\usepackage[colorlinks=true,linkcolor=blue,urlcolor=red,citecolor=magenta]{hyperref}
\usepackage{amsmath,amssymb,amsthm,float,graphicx,mathtools}
\allowdisplaybreaks
\newtheorem{assumption}{Assumption}
\newtheorem{baitoan}{Bài toán}
\newtheorem{cauhoi}{Câu hỏi}
\newtheorem{conjecture}{Conjecture}
\newtheorem{corollary}{Corollary}
\newtheorem{dangtoan}{Dạng toán}
\newtheorem{definition}{Definition}
\newtheorem{dinhly}{Định lý}
\newtheorem{dinhnghia}{Định nghĩa}
\newtheorem{example}{Example}
\newtheorem{ghichu}{Ghi chú}
\newtheorem{hequa}{Hệ quả}
\newtheorem{hypothesis}{Hypothesis}
\newtheorem{lemma}{Lemma}
\newtheorem{luuy}{Lưu ý}
\newtheorem{nhanxet}{Nhận xét}
\newtheorem{notation}{Notation}
\newtheorem{note}{Note}
\newtheorem{principle}{Principle}
\newtheorem{problem}{Problem}
\newtheorem{proposition}{Proposition}
\newtheorem{question}{Question}
\newtheorem{remark}{Remark}
\newtheorem{theorem}{Theorem}
\newtheorem{vidu}{Ví dụ}
\usepackage[left=1cm,right=1cm,top=5mm,bottom=5mm,footskip=4mm]{geometry}
\def\labelitemii{$\circ$}
\DeclareRobustCommand{\divby}{%
	\mathrel{\vbox{\baselineskip.65ex\lineskiplimit0pt\hbox{.}\hbox{.}\hbox{.}}}%
}

\title{Problems in Elementary Computer Science -- Bài Tập Tin Học Sơ Cấp}
\author{Nguyễn Quản Bá Hồng\footnote{Independent Researcher, Ben Tre City, Vietnam\\e-mail: \texttt{nguyenquanbahong@gmail.com}; website: \url{https://nqbh.github.io}.}}
\date{\today}

\begin{document}
\maketitle

%------------------------------------------------------------------------------%

\begin{baitoan}[\cite{Trung_HSG_THPT_Tin}, 1., p. 13, HSG Lớp 10 Vĩnh Phúc 2020--2021, Square -- Hình vuông]
	Cho $n$ điểm có tọa độ là các số nguyên trên hệ trục tọa độ $Oxy$. Tìm diện tích hình vuông nhỏ nhất có các cạnh song song với các trục tọa độ sao cho tất cả các điểm đã cho đều thuộc hình vuông đó (điểm nằm trên cạnh hình vuông cũng được coi là thuộc hình vuông đó).
	\begin{itemize}
		\item {\sf Input.} Dòng 1: chứa số nguyên dương $n\in\mathbb{N}^\star$, $2\le n\le20$, là số lượng điểm có tọa độ là các số nguyên. $n$ dòng tiếp theo, mỗi dòng ghi 2 số nguyên $x,y\in\mathbb{Z}$, $1\le x,y\le100$, là tọa độ của mỗi điểm.
		\item {\sf Output.} Ghi diện tích hình vuông nhỏ nhất tìm được.
		\item {\sf Sample.}
		\begin{table}[H]
			\centering
			\begin{tabular}{|l|l|}
				\hline
				{\tt square.inp} & {\tt square.out} \\
				\hline
				3 & 16 \\
				3 4 &  \\
				5 7 &  \\
				4 3 &  \\
				\hline
			\end{tabular}
		\end{table}
	\end{itemize}
	Mở rộng bài toán từ `hình vuông' sang `hình chữ nhật'.
\end{baitoan}

\begin{baitoan}[\cite{Trung_HSG_THPT_Tin}, 2., pp. 13--14, HSG Lớp 10 Vĩnh Phúc 2020--2021, Divisible by 3 -- Chia hết cho 3]
	Cho dãy $a$ gồm $n$ số nguyên dương. Cho biết có bao nhiêu cặp số trong dãy có tổng chia hết cho $3$, i.e., đếm xem có bao nhiêu cặp chỉ số $i,j$, $1\le i < j\le n$, sao cho $a_i + a_j\divby3$.
	\begin{itemize}
		\item {\sf Input.} Dòng 1: 1 số nguyên duy nhất $n$, $1\le n\le10^5$. Dòng 2: Ghi $n$ số nguyên dương $a_1,a_2,\ldots,a_n$, $1\le a_i\le10^5$, $\forall i = 1,2,\ldots,n$, là các phần tử của dãy.
		\item {\sf Output.} 1 dòng duy nhất ghi số lượng cặp số của dãy $a$ có tổng chia hết cho $3$.
		\item {\sf Sample.}
		\begin{table}[H]
			\centering
			\begin{tabular}{|l|l|l|}
				\hline
				{\tt div3.inp} & {\tt div3.out} & Giải thích \\
				\hline
				5 & 3 & 3 cặp số tìm được có chỉ số: $(1,4),(2,3),(3,5)$. \\
				3 6 9 12 & & \\
				\hline
				4 & 6 & 6 cặp số tìm được có chỉ số: $(1,2),(1,3),(1,4),(2,3),(2,4),(3,4)$. \\
				3 6 9 12 & & \\
				\hline
			\end{tabular}
		\end{table}
	\end{itemize}
\end{baitoan}

\begin{baitoan}[\cite{Trung_HSG_THPT_Tin}, 3., p. 14, HSG Lớp 10 Vĩnh Phúc 2020--2021, Delete element -- Xóa phần tử]
	Cho dãy gồm $n$ số nguyên $a_1,a_2,\ldots,a_n$ với $1\le a_i\le3$, $\forall i = 1,2,\ldots,n$. Có bao nhiêu cách để xóa đi 1 số phần tử của dãy (không xóa phần tử nào cũng được coi là 1 cách) mà vẫn giữ nguyên thứ tự ban đầu để được 1 dãy mới thỏa mãn 2 yêu cầu sau: (i) Dãy còn ít nhất 3 phần  tử. (ii) Phần tử đầu tiên của dãy có giá trị $1$, tiếp theo là 1 số phần tử có giá trị là $2$ (ít nhất có 1 số $2$), \& kết thúc bằng đúng 1 phần tử có giá trị là $3$. E.g., các dãy $1,2,2,3$ \& $1,2,3$ thỏa mãn yêu cầu, các dãy $1,2,3,3$ \& $1,1,2,3$ không thỏa mãn yêu cầu.
	\begin{itemize}
		\item {\sf Input.} Dòng 1: 1 số nguyên dương $n\in\mathbb{N}^\star$, $n\le10^6$, là số lượng phần tử của dãy. Dòng 2: Ghi $n$ số nguyên dương $a_1,a_2,\ldots,a_n$ là giá trị của các phần tử của dãy ban đầu.
		\item {\sf Output.} Gồm 1 dòng duy nhất là số cách xóa để được dãy mới thỏa mãn yêu cầu của đề bài. Do số lượng cách xóa phần tử có thể rất lớn nên chỉ cần ghi ra số lượng cách xóa sau khi chia lấy dư cho $10^9 + 7$.
		\item {\sf Sample.}
		\begin{table}[H]
			\centering
			\begin{tabular}{|l|l|}
				\hline
				\verb|delete_element.inp| & \verb|delete_element.out| \\
				\hline
				8 & 15 \\
				1 2 1 2 3 1 2 3 &  \\
				\hline
			\end{tabular}
		\end{table}
	\end{itemize}
\end{baitoan}

\begin{baitoan}[\cite{Trung_HSG_THPT_Tin}, 1., p. 15, HSG Lớp 11 Vĩnh Phúc 2020--2021, Game button -- Trò chơi bấm nút]
	Người chơi đang tham gia 1 trò chơi như sau: Có 2 nút bấm A, B, trên nút A có ghi số $m_A$, trên nút B có ghi số $m_B$. Ở mỗi lượt chơi, người chơi phải chọn bấm 1 trong 2 nút \& sẽ nhận được số điểm thưởng bằng với số ghi trên nút đó, sau đó số trên nút bấm giảm đi $1$ đơn vị. Hỏi sau 2 lượt chơi, số điểm thưởng lớn nhất mà người chơi có thể nhận được là bao nhiêu?
	\begin{itemize}
		\item {\sf Input.} 1 dòng duy nhất ghi 2 số nguyên dương $m_A,m_B$ với $3\le A,B\le20$, tương ứng với 2 số ghi trên 2 nút A \& B.
		\item {\sf Output.} Ghi số điểm thưởng lớn nhất mà người chơi có thể nhận được sau 2 lượt chơi.
		\item {\sf Sample.}
		\begin{table}[H]
			\centering
			\begin{tabular}{|l|l|l|}
				\hline
				\verb|game_button.inp| & \verb|game_button.out| & Giải thích\\
				\hline
				5 3 & 9 & Bấm 2 lần nút A \& sẽ có tổng điểm thưởng: $5 + 4 = 9$. \\
				\hline
			\end{tabular}
		\end{table}
	\end{itemize}
\end{baitoan}

\begin{baitoan}[\cite{Trung_HSG_THPT_Tin}, 2., p. 15, HSG Lớp 11 Vĩnh Phúc 2020--2021, Count number -- Đếm số]
	Cho 4 số nguyên dương $a,b,c,d$. Đếm xem có bao nhiêu số nguyên dương $x\in\mathbb{N}^\star$ thỏa mãn các điều kiện sau: (i) $a\le x\le b$. (ii) $x\not{\divby}c$. (iii) $x\not{\divby}d$.
	\begin{itemize}
		\item {\sf Input.} 1 dòng duy nhất ghi 4 số $a,b,c,d$, $1\le a,b\le10^{18}$, $1\le c,d\le10^9$.
		\item {\sf Output.} 1 dòng duy nhất ghi số lượng số nguyên dương $x\in\mathbb{N}^\star$ thỏa mãn điều kiện đề bài.
		\item {\sf Sample.}
		\begin{table}[H]
			\centering
			\begin{tabular}{|l|l|l|}
				\hline
				\verb|count_number.inp| & \verb|count_number.out| & Giải thích \\
				\hline
				4 9 2 3 & 2 & Chỉ có số 5 \& 7 thỏa mãn điều kiện đề bài. \\
				\hline
			\end{tabular}
		\end{table}
	\end{itemize}
\end{baitoan}

\begin{baitoan}[\cite{Trung_HSG_THPT_Tin}, 3., p. 16, HSG Lớp 11 Vĩnh Phúc 2020--2021, Reverse \& reverse -- Lật qua lật lại]
	Cho dãy $a$ gồm $n\in\mathbb{N}^\star$ phần tử $1,2,\ldots,n$. Người ta thực hiện trên dãy số này đúng $k$ lần 2 thao tác sau: (i) Đầu tiên, đảo ngược thứ tự (lật đối xứng) đoạn phần tử có chỉ số từ $u$ đến $v$. (ii) Tiếp theo, đảo ngược thứ tự (lật đối xứng) đoạn phần tử có chỉ số từ $l$ đến $r$. Với $u,v,l,r$ là các hằng số cho trước. Đưa ra dãy $a$ sau khi thực hiện $k$ lần 2 thao tác nói trên.
	\begin{itemize}
		\item {\sf Input.} Dòng 1: 2 số nguyên dương $n,k\in\mathbb{N}^\star$, $1\le n\le100$, $1\le k\le10^9$. Dòng 2: gồm 2 số nguyên dương $u,v$, $1\le u < v\le n$. Dòng 3: gồm 2 số nguyên dương $l,r$, $1\le l < r\le n$.
		\item {\sf Output.} Ghi trên $n$ dòng, dòng thứ $i$ ghi giá trị của phần tử thứ $i$ của dãy $a$ sau khi thực hiện $k$ lần 2 thao tác nói trên, $\forall i = 1,2,\ldots,n$.
		\item {\sf Sample.}
		\begin{table}[H]
			\centering
			\begin{tabular}{|l|l|l|}
				\hline
				\verb|reverse_reverse.inp| & \verb|reverse_reverse.out| & Giải thích\\
				\hline
				7 2 & 1 & Dãy ban đầu: \\
				2 5 & 2 & 1 2 3 4 5 6 7 \\
				3 7 & 4 & Lần 1: \\
				& 3 & 1 \textit{5 4 3 2} 6 7 \\
				& 5 & 1 5 \textit{7 6 2 3 4} \\
				& 7 & Lần 2: \\
				& 6 & 1 \textit{2 6 7 5} 3 4 \\
				& & 1 2 4 3 5 7 6 \\
				\hline
			\end{tabular}
		\end{table}
	\end{itemize}
\end{baitoan}
%------------------------------------------------------------------------------%

\printbibliography[heading=bibintoc]

\end{document}