\documentclass{article}
\usepackage[backend=biber,natbib=true,style=alphabetic,maxbibnames=50]{biblatex}
\addbibresource{/home/nqbh/reference/bib.bib}
\usepackage[utf8]{vietnam}
\usepackage{tocloft}
\renewcommand{\cftsecleader}{\cftdotfill{\cftdotsep}}
\usepackage[colorlinks=true,linkcolor=blue,urlcolor=red,citecolor=magenta]{hyperref}
\usepackage{amsmath,amssymb,amsthm,float,graphicx,mathtools,tikz}
\usetikzlibrary{angles,calc,intersections,matrix,patterns,quotes,shadings}
\allowdisplaybreaks
\newtheorem{assumption}{Assumption}
\newtheorem{baitoan}{}
\newtheorem{cauhoi}{Câu hỏi}
\newtheorem{conjecture}{Conjecture}
\newtheorem{corollary}{Corollary}
\newtheorem{dangtoan}{Dạng toán}
\newtheorem{definition}{Definition}
\newtheorem{dinhly}{Định lý}
\newtheorem{dinhnghia}{Định nghĩa}
\newtheorem{example}{Example}
\newtheorem{ghichu}{Ghi chú}
\newtheorem{hequa}{Hệ quả}
\newtheorem{hypothesis}{Hypothesis}
\newtheorem{lemma}{Lemma}
\newtheorem{luuy}{Lưu ý}
\newtheorem{nhanxet}{Nhận xét}
\newtheorem{notation}{Notation}
\newtheorem{note}{Note}
\newtheorem{principle}{Principle}
\newtheorem{problem}{Problem}
\newtheorem{proposition}{Proposition}
\newtheorem{question}{Question}
\newtheorem{remark}{Remark}
\newtheorem{theorem}{Theorem}
\newtheorem{vidu}{Ví dụ}
\usepackage[left=1cm,right=1cm,top=5mm,bottom=5mm,footskip=4mm]{geometry}
\def\labelitemii{$\circ$}
\DeclareRobustCommand{\divby}{%
	\mathrel{\vbox{\baselineskip.65ex\lineskiplimit0pt\hbox{.}\hbox{.}\hbox{.}}}%
}

\title{Problem: Natural Science 6 -- Bài Tập: Khoa Học Tự Nhiên 6}
\author{Nguyễn Quản Bá Hồng\footnote{Independent Researcher, Ben Tre City, Vietnam\\e-mail: \texttt{nguyenquanbahong@gmail.com}; website: \url{https://nqbh.github.io}.}}
\date{\today}

\begin{document}
\maketitle
\tableofcontents

%------------------------------------------------------------------------------%

\section{Mở Đầu về Khoa Học Tự Nhiên}

\begin{baitoan}[\cite{ncpt_KHTN_6_tap_1}, 1., p. 10]
	Khoa học tự nhiên: {\sf A.} Nghiên cứu về các sự vật, hiện tượng tự nhiên, tìm ra quy luật chi phối chúng. {\sf B.} Nghiên cứu về các sự vật, hiện tượng xã hội, các ảnh hưởng của chúng đến đời sống con người \& môi trường. {\sf C.} Phát minh ra các giống vật nuôi \& cây trồng mới. {\sf D.} Cải tiến các phương tiện giao thông vận tải \& thông tin liên lạc.
\end{baitoan}

\begin{baitoan}[\cite{ncpt_KHTN_6_tap_1}, 2., p. 10]
	Trường hợp nào sau đây không thuộc đối tượng nghiên cứu của Khoa học tự nhiên? {\sf A.} Quy luật chuyển động của Mặt Trời \& các hành tinh. {\sf B.} Sự phát triển của các loại cây. {\sf C.} Trào lưu của tuổi học trò trong từng giai đoạn. {\sf D.} Điều chế vaccin trong phòng bệnh.
\end{baitoan}

\begin{baitoan}[\cite{ncpt_KHTN_6_tap_1}, 3., p. 10]
	Khoa học tự nhiên có vai trò gì đối với đời sống con người? Chọn các đáp án đúng. (a) Cung cấp thông tin cho con người về thế giới tự nhiên. (b) Cung cấp thông tin cho con người về quy luật hình thành \& phát triển của xã hội \& con người. (c) Góp phần vào công tác bảo vệ môi trường \& ứng phó với biến đổi khí hậu. (d) Giúp đưa ứng dụng công nghệ nhằm mở rộng sản xuất \& phát triển kinh tế. (e) Cung cấp thông tin cho con người về hoạt động văn hóa, tín ngưỡng.
\end{baitoan}

\begin{baitoan}[\cite{ncpt_KHTN_6_tap_1}, 4., p. 11]
	Nối đáp án: 1. Khoa học tự nhiên. 2. Khoa học vật chất. 3. Khoa học đời sống. 4. Hóa học. 5. Vật lý học. 6. Thiên văn học. 7. Khoa học Trái Đất. (a) nghiên cứu về quy luật vận động \& sự biến đổi của các vật thể trên bầu trời (các hành tin, sao, $\ldots$). (b) nghiên cứu về các sinh vật \& sự sống (con người, động vật, thực vật, $\ldots$), mối quan hệ giữa chúng với nhau \& với môi trường. (c) nghiên cứu về Trái Đất \& bầu khí quyển của nó. (d) bao gồm vật lý, hóa học, thiên văn học, khoa học Trái Đất, $\ldots$ (e) nghiên cứu cấu tạo, các phản ứng hóa học, cấu trúc, các tính chất của vật chất \& các biến đổi lý hóa mà chúng trải qua. (f) nghiên cứu về vật chất, quy luật vận động, lực, năng lượng \& sự biến đổi năng lượng, $\ldots$ (g) bao gồm khoa học đời sống \& khoa học vật chất.
\end{baitoan}

\begin{baitoan}[\cite{ncpt_KHTN_6_tap_1}, 6., p. 12]
	Quan sát các hiện tượng sau trong đời sống. Có quy luật{\tt/}chu trình nào đã được rút ra từ các hiện tượng đó? (a) Các mùa trong năm \& hiện tượng Mặt Trời mọc trong ngày. (b) Tung vật lên cao, vật sẽ rơi về phía Trái Đất. (c) Sự thay đổi trạng thái của hạt đậu xanh khi vùi trong đất ẩm. (d) Sự dịch chuyển của nam châm khi đặt 2 nam châm gần nhau.
\end{baitoan}

\begin{baitoan}[\cite{ncpt_KHTN_6_tap_1}, 7., p. 12]
	Vật sống (vật hữu sinh) không có các đặc điểm nào? {\sf A.} Có sự trao đổi chất với môi trường. {\sf B.} Có khả năng sinh trưởng, phát triển. {\sf C.} Có khả năng sinh sản. {\sf D.} Không có sự trao đổi chất với môi trường.
\end{baitoan}

\begin{baitoan}[\cite{ncpt_KHTN_6_tap_1}, 8., p. 13]
	Vật không sống (vật vô sinh) không có đặc điểm nào? {\sf A.} Không trao đổi chất với môi trường. {\sf B.} Không có khả năng sinh trưởng, phát triển. {\sf C.} Không có khả năng sinh sản. {\sf D.} Không có màu sắc.
\end{baitoan}

\begin{baitoan}[\cite{ncpt_KHTN_6_tap_1}, 9., p. 13]
	Phân biệt vật sống \& vật không sống: vi khuẩn, máy tính, cây xanh, cái bàn, robot lau nhà, con mèo, hòn đá, quyển sách.
\end{baitoan}

\begin{baitoan}[\cite{ncpt_KHTN_6_tap_1}, 10., p. 13]
	Phân biệt khoa học về vật chất \& khoa học về sự sống.
\end{baitoan}

\begin{baitoan}[\cite{ncpt_KHTN_6_tap_1}, 11., p. 13]
	Lợi ích chính của việc chấp hành các quy định an toàn khi học trong phòng thực hành? {\sf A.} Giúp tiết kiệm thời gian. {\sf B.} Giúp tiết kiệm chi phí. {\sf C.} Giúp tránh phải các tình huống gây nguy hiểm. {\sf D.} Giúp ổn định trật tự lớp học.
\end{baitoan}

\begin{baitoan}[\cite{ncpt_KHTN_6_tap_1}, 12., p. 13]
	Học sinh phải tuân thủ yêu cầu gì khi làm thực hành? {\sf A.} Sử dụng găng tay, khẩu trang, kính bảo vệ mắt \& các thiết bị bảo vệ khác (nếu cần). {\sf B.} Chỉ được tiến hành thí nghiệm khi có người hướng dẫn. {\sf C.} Ngửi, nếm hóa chất để phát hiện ra chất an toàn \& không an toàn. {\sf D.} Sau khi thí nghiệm xong, thu dọn sạch sẽ, để dụng cụ vào đúng nơi quy định.
\end{baitoan}

\begin{baitoan}[\cite{ncpt_KHTN_6_tap_1}, 15., p. 14]
	Sử dụng kính lúp trong trường hợp nào? {\sf A.} Quan sát vật không màu. {\sf B.} Quan sát vật có kích thước nhỏ. {\sf C.} Quan sát vật có kích thước vô cùng nhỏ mà mắt thường không thể thấy được. {\sf D.} Quan sát các vật ở rất xa.
\end{baitoan}

\begin{baitoan}[\cite{ncpt_KHTN_6_tap_1}, 16., p. 14]
	Tìm lưu ý sai khi sử dụng kính lúp. {\sf A.} Ban đầu đặt kính gần vật cần quan sát. {\sf B.} Điều chỉnh kính sao cho kích thước kính lớn nhất. {\sf C.} Chọn vật cần quan sát có kích thước đủ nhỏ. {\sf D.} Từ từ dịch chuyển kính ra xa vật cần quan sát.
\end{baitoan}

\begin{baitoan}[\cite{ncpt_KHTN_6_tap_1}, 17., pp. 14--15]
	Trường hợp sử dụng kính hiển vi: {\sf A.} Quan sát vật không màu. {\sf B.} Quan sát vật có kích thước nhỏ. {\sf C.} Quan sát vật có kích thước vô cùng nhỏ mà mắt thường không thể thấy được. {\sf D.} Cả 3 trường hợp.
\end{baitoan}

\begin{baitoan}[\cite{ncpt_KHTN_6_tap_1}, 18., p. 15]
	Tìm lưu ý sai khi sử dụng kính hiển vi: {\sf A.} Chọn kính có vật kính thích hợp. {\sf B.} Điều chỉnh kính sao cho có thể quan sát được vật. {\sf C.} Tiêu bản cần được đặt trên bàn kính. {\sf D.} Vật kính có thể chọn tùy ý.
\end{baitoan}

\begin{baitoan}[\cite{ncpt_KHTN_6_tap_1}, 19., p. 15]
	Tác dụng của kính lúp \& kính hiển vi giống \& khác nhau ở điểm nào?
\end{baitoan}

\begin{baitoan}[\cite{ncpt_KHTN_6_tap_1}, 20., p. 15]
	Xác định {\rm GHĐ} \& {\rm ĐCNN} của cây thước học sinh.
\end{baitoan}

\begin{baitoan}[\cite{ncpt_KHTN_6_tap_1}, 21., p. 15]
	Để đo kích thước của chiếc bàn trong phòng, nên chọn thước nào trong các thước sau? {\sf A.} Thước thẳng có {\rm GHĐ 20 cm}. {\sf B.} Thước kẹp có {\rm GHĐ 10 cm}. {\sf C.} Thước dây có {\rm GHĐ 2 m}. {\sf D.} Thước thẳng có {\rm GHĐ 30 cm}.
\end{baitoan}

\begin{baitoan}[\cite{ncpt_KHTN_6_tap_1}, 22., p. 15]
	1 bạn dùng thước có {\rm ĐCNN} là {\rm0.2 cm} để đo chiều dài của cuốn sách. Cách ghi sai? {\sf A.} {\rm33.4 cm}. {\sf B.} {\rm334 mm}. {\sf C.} {\rm334 m}. {\sf D.} {\rm0.334 m}.
\end{baitoan}

\begin{baitoan}[\cite{ncpt_KHTN_6_tap_1}, 23., p. 15]
	Dùng thước có {\rm ĐCNN 1 cm} đo chiều cao của cửa sổ. Kết quả đúng: {\sf A.} {\rm2 m}. {\sf B.} {\rm195 cm}. {\sf C.} {\rm19.7 cm}. {\sf D.} {\rm19.2 m}.
\end{baitoan}

\begin{baitoan}[\cite{ncpt_KHTN_6_tap_1}, 25., p. 16]
	Thước thẳng có thể đo chiều dài các vật thế nào? {\sf A.} Chỉ đo được vật có hình dạng là các đoạn thẳng. {\sf B.} Vật có hình dạng bất kỳ (cần thêm dụng cụ hỗ trợ). {\sf C.} Tùy thuộc vào {\rm GHĐ} của thước. {\sf D.} Tùy thuộc vào {\rm ĐCNN} của thước.
\end{baitoan}

\begin{baitoan}[\cite{ncpt_KHTN_6_tap_1}, 26., p. 16]
	Có 1 cái thước thẳng, {\rm GHĐ 30 cm}. Tìm cách dùng thước ấy để đo chu vi bánh xe đạp.
\end{baitoan}

\begin{baitoan}[\cite{ncpt_KHTN_6_tap_1}, 27., p. 16]
	Làm thế nào để có thể đo được đường kính của 1 sợi dây đồng mảnh khi chỉ có dụng cụ đo là 1 chiếc thước kẹp có {\rm ĐCNN 1 mm}? Nêu phương án để ít sai số nhất có thể.
\end{baitoan}

\begin{baitoan}[\cite{ncpt_KHTN_6_tap_1}, 28., p. 16]
	Chỉ bằng 1 cái thước thẳng, làm thế nào để đo chu vi của mặt bàn có hình dạng xù xì?
\end{baitoan}

\begin{baitoan}[\cite{ncpt_KHTN_6_tap_1}, 29., p. 16]
	Chỉ dùng 1 chiếc thước đo góc \& 1 thước dây. Không trèo lên cây, làm thế nào để đo được gần đúng chiều cao của 1 cây cổ thụ?
\end{baitoan}

\begin{baitoan}[\cite{ncpt_KHTN_6_tap_1}, .30, p. 16]
	Cần lấy {\rm200 mL} nước để pha sữa thì nên dùng dụng cụ nào? {\sf A.} Bình chia độ. {\sf B.} Ca đong. {\sf C.} Bình tràn. {\sf D.} Cốc uống nước thông thường.
\end{baitoan}

\begin{baitoan}[\cite{ncpt_KHTN_6_tap_1}, 33., p. 17]
	Nêu phương án để đo thể tích của 1 vật rắn thấm nước có hình dạng bất kỳ. Giả sử vật đó bỏ lọt bình chia độ.
\end{baitoan}

\begin{baitoan}[\cite{ncpt_KHTN_6_tap_1}, 34., p. 17]
	Chỉ với 2 can {\rm3 L, 5 L}. Nêu các cách để lấy ra được {\rm4 L} nước. Cách nào ưu điểm hơn? Vì sao?
\end{baitoan}

\begin{baitoan}[\cite{ncpt_KHTN_6_tap_1}, 35., p. 17]
	1 bạn dùng cân với bộ quả cân {\rm1 kg, 0.5 kg, 0.2 kg, 100 g} (mỗi loại quả cân có 2 quả) để cân 1 vật. Kết quả thu được là {\rm3.2 kg}. Bạn đó đã dùng các quả cân nào?
\end{baitoan}

\begin{baitoan}[\cite{ncpt_KHTN_6_tap_1}, 36., p. 17]
	Khi cân 1 vật, 1 người đã dùng các quả cân {\rm0.5 kg, 0.2 kg, 100 g, 50 g}. Khối lượng vật: {\sf A.} {\rm150.7 kg}. {\sf B.} {\rm850 g}. {\sf C.} {\rm0.8 kg}. {\sf D.} Không xác định được.
\end{baitoan}

\begin{baitoan}[\cite{ncpt_KHTN_6_tap_1}, 37., p. 17]
	Có 8 viên bi với hình dạng \& kích thước giống hệt nhau. Dùng cân Roberval, nêu phương án để chỉ với 2 lần cân, có thể tìm ra 1 viên bi nhẹ hơn.
\end{baitoan}

\begin{baitoan}[\cite{ncpt_KHTN_6_tap_1}, 39., p. 18]
	Tại sao ở các cửa hàng vàng bạc, người ta thường dùng cân tiểu ly (cân điện tử)? {\sf A.} Vì cân tiểu ly nhỏ gọn. {\sf B.} Vì cân tiểu ly có {\rm ĐCNN} nhỏ nên có tính chính xác cao. {\sf C.} Vì cân tiểu ly có {\rm GHĐ} nhỏ nên có tính chính xác cao. {\sf D.} Vì cả 3 lý do.
\end{baitoan}

\begin{baitoan}[\cite{ncpt_KHTN_6_tap_1}, 40., p. 18]
	Chọn loại cân phù hợp: cân đòn, cân tạ, cân tiểu ly, cân y tế, máy đo chiều cao cân nặng, cân Roberval, để cân: vàng, bạc, quả bí ngô, người trưởng thành, bao gạo, vật cần xác định khối lượng trong phòng thực hành, trẻ sơ sinh.
\end{baitoan}

\begin{baitoan}[\cite{ncpt_KHTN_6_tap_1}, 41., p. 19]
	Để đo nhiệt độ cơ thể người, sử dụng loại nhiệt kế nào không phù hợp? {\sf A.} Nhiệt kế thủy ngân. {\sf B.} Nhiệt kế y tế. {\sf C.} Nhiệt kế rượu. {\sf D.} Nhiệt kế hồng ngoại.
\end{baitoan}

\begin{baitoan}[\cite{ncpt_KHTN_6_tap_1}, 42., p. 19]
	Để đo thời gian chạy của vận động viên, nên dùng loại đồng hồ nào? {\sf A.} Đồng hồ bấm giây. {\sf B.} Đồng hồ treo tường. {\sf C.} Đồng hồ quả lắc. {\sf D.} Có thể dùng bất cứ đồng hồ nào.
\end{baitoan}

%------------------------------------------------------------------------------%

\section{Các Thể của Chất}

\begin{baitoan}[\cite{ncpt_KHTN_6_tap_1}, 1., p. 23]
	Đếm số vật thể chứa nước trong các vật thể: cây kem, cốc sữa, quả bóng bay, cái chai, lọ mực, quả táo, con gà.
\end{baitoan}

\begin{baitoan}[\cite{ncpt_KHTN_6_tap_1}, 2., p. 23]
	Ghép nội dung: 1. Chất lỏng đun nóng. 2. Chất rắn. 3. Chất. 4. Sữa. 5. Chất khí. (a) được tạo thành từ các hạt vô cùng nhỏ bé. (b) ở thể lỏng. (c) dễ dàng bị nén. (d) bị hóa hơi. (e) không thể chảy. (f) bị thăng hoa.
\end{baitoan}

\begin{baitoan}[\cite{ncpt_KHTN_6_tap_1}, 3., p. 23]
	Các hạt đang sắp xếp có trật tự, chỉ dao động quanh vị trí cố định trở nên di chuyển tự do, cách xa nhau. Quá trình đó gọi là: {\sf A.} nóng chảy. {\sf B.} thăng hoa. {\sf C.} sôi. {\sf D.} ngưng kết.
\end{baitoan}

\begin{baitoan}[\cite{ncpt_KHTN_6_tap_1}, 4., p. 23]
	Các hạt đang di chuyển tự do, trượt lên nhau trở nên sắp xếp có trật tự, chỉ dao động quanh vị trí cố định. Quá trình đó gọi là: {\sf A.} nóng chảy. {\sf B.} đông đặc. {\sf C.} hóa hơi. {\sf D.} sôi.
\end{baitoan}

\begin{baitoan}[\cite{ncpt_KHTN_6_tap_1}, 5., p. 23]
	{\rm Đ{\tt/}S?} Nếu sai, sửa cho đúng. (a) Sự nóng chảy luôn làm tăng thể tích. (b) Sự hóa hơi luôn làm tăng thể tích. (c) Khi đun nóng chất rắn, có thể xảy ra sự sôi hoặc sự thăng hoa. (d) Mật ong ở nhiệt độ cao chảy nhanh hơn ở nhiệt độ thấp. (e) Khi làm lạnh, chất khí sẽ luôn hóa lỏng.
\end{baitoan}

\begin{baitoan}[\cite{ncpt_KHTN_6_tap_1}, 6., p. 23]
	{\rm Đ{\tt/}S?} Nếu sai, sửa cho đúng. (a) Khi rót vào bình kín, chất lỏng, \& chất khí đều chiếm 1 phần không gian của bình chứa. (b) Chất khí \& chất lỏng đều có khối lượng. (c) Chỉ chất rắn mới có hình dạng xác định. (d) Các chất khí đều không màu, không mùi nên không thể cảm nhận bằng giác quan.
\end{baitoan}

\begin{baitoan}[\cite{ncpt_KHTN_6_tap_1}, 7., p. 24]
	Khi xảy ra sự ngưng tụ thì: {\sf A.} khoảng cách giữa các hạt tăng lên. {\sf B.} khối lượng riêng của chất tăng lên. {\sf C.} tốc độ chuyển động của hạt tăng lên. {\sf D.} lực tương tác giữa các hạt tăng lên.
\end{baitoan}

\begin{baitoan}[\cite{ncpt_KHTN_6_tap_1}, 8., p. 24]
	Chỉ ra: (a) 1 số vật thể có chứa chất sắt. (b) 1 số vật thể có chứa chất khí. (c) 1 số vật thể có chứa chất đường.
\end{baitoan}

\begin{baitoan}[\cite{ncpt_KHTN_6_tap_1}, 9., p. 24]
	Từ cấu tạo hạt của chất, giải thích tại sao thể rắn rất khó nén.
\end{baitoan}

\begin{baitoan}[\cite{ncpt_KHTN_6_tap_1}, 10., p. 24]
	Miếng nút có hình dạng cố định \& ở thể rắn. Vậy tại sao ta có thể nén miếng mút được?
\end{baitoan}

\begin{baitoan}[\cite{ncpt_KHTN_6_tap_1}, 11., p. 24]
	Dựa vào thuyết cấu tạo hạt của chất, giải thích hiện tượng: (a) Khi nhiệt độ tăng, thể tích chất lỏng tăng lên. (b) Khi hòa tan đường vào nước, thu được 1 dung dịch đồng nhất, trong suốt, \& có vị ngọt.
\end{baitoan}

\begin{baitoan}[\cite{ncpt_KHTN_6_tap_1}, 12., p. 24]
	Làm thí nghiệm: lấy 1 chén mật ong \& 1 cốc nước. Cho cả 2 vào trong 1 cốc lớn, khuấy đều cho mật ong tan. Sau đó, rót hỗn hợp trong cốc lớn vào lại cốc nước \& chén mật ong ban đầu. Thể tích hỗn hợp thu được có bằng tổng thể tích mật ong \& nước ban đầu không. Bằng kiến thức về cấu tạo hạt của chất, giải thích kết quả thí nghiệm.
\end{baitoan}

\begin{baitoan}[\cite{ncpt_KHTN_6_tap_1}, 13., p. 24]
	Mô tả sự thăng hoa dựa vào cấu tạo hạt của chất.
\end{baitoan}

\begin{baitoan}[\cite{ncpt_KHTN_6_tap_1}, 14., pp. 24--25]
	Cho nước đá vào 1 cốc. Cắm nhiệt kế vào \& ghi lại nhiệt độ sau mỗi phút, thu được bảng số liệu:
	\begin{table}[H]
		\centering
		\begin{tabular}{|c|c|c|}
			\hline
			$t$ (phút) & Nhiệt độ (${}^\circ$C) & Thể \\
			\hline
			0 & $-3.0$ &  \\
			\hline
			1 & $-0.5$ & rắn \\
			\hline
			2 & 0.0 & rắn $+$ lỏng \\
			\hline
			3 & 0.0 &  \\
			\hline
			4 & 0.0 &  \\
			\hline
			5 & 0.0 &  \\
			\hline
			6 & 0.0 &  \\
			\hline
			7 & 0.5 &  \\
			\hline
			8 & 1.5 &  \\
			\hline
			9 & 3.5 &  \\
			\hline
			10 & 6.0 &  \\
			\hline
		\end{tabular}
	\end{table}
	\noindent(a) Điều gì xảy ra từ phút thứ 2 đến phút thứ 6? (b) Điều gì xayr a từ phút thứ 7? (c) Ghi thể của nước tại các nhiệt độ.
\end{baitoan}

\begin{baitoan}[\cite{ncpt_KHTN_6_tap_1}, 15., p. 25]
	Trộn muối vào nước đá đập nhỏ, sẽ thu được hỗn hợp làm lạnh. Đặt 1 ống nghiệm chứa nước trong hỗn hợp này, cắm nhiệt kế vào ống nghiệm. Ghi lại nhiệt độ sau mỗi phút, thu được bảng số liệu:
	\begin{table}[H]
		\centering
		\begin{tabular}{|c|c|c|}
			\hline
			$t$ (phút) & Nhiệt độ (${}^\circ$C) & Thể \\
			\hline
			0 & 10.0 &  \\
			\hline
			1 & 5.0 & rắn \\
			\hline
			2 & 2.2 & rắn $+$ lỏng \\
			\hline
			3 & 1.0 &  \\
			\hline
			4 & 0.3 &  \\
			\hline
			5 & 0.0 &  \\
			\hline
			6 & 0.0 &  \\
			\hline
			7 & 0.0 &  \\
			\hline
			8 & $-0.2$ &  \\
			\hline
			9 & $-1.7$ &  \\
			\hline
			10 & $-4.0$ &  \\
			\hline
		\end{tabular}
	\end{table}
	\noindent(a) Từ thời diểm đầu đến phút thứ 4, nước ở thể gì? (b) Điều gì xảy ra từ phút thứ 5 đến phút thứ 7? (c) Ở nhiệt nào nước tồn tại ở 2 thể? Ghi thể của nước tại các nhiệt độ này. (d) Giải thích tại sao nhiệt độ nóng chảy bằng nhiệt độ đông đặc?
\end{baitoan}

\begin{baitoan}[\cite{ncpt_KHTN_6_tap_1}, 16., p. 26]
	Cho 1 ít nến (parafin) \& lưu huỳnh vào 2 ống nghiệm riêng biệt. Đặt 2 ống nghiệm \& nhiệt kế vào cốc nước chịu nhiệt, sau đó đun đến khi nước sôi thì dừng đun. Dự đoán thể của lưu huỳnh \& nến khi đó. Biết nhiệt độ nóng chảy của nến \& lưu huỳnh lần lượt là $80^\circ${\rm C} \& $115^\circ${\rm C}.
\end{baitoan}

\begin{baitoan}[\cite{ncpt_KHTN_6_tap_1}, 17., p. 26]
	Tiến hành thí nghiệm: Cho vài {\rm g} bột băng phiến vào ống nghiệm. Cắm nhiệt kế vào giữa khối bột. Cho ống nghiệm vào cốc nước chứa khoảng {\rm250 mL} nước, đun nóng từ từ đến khi nước sôi thì dừng đun. Ghi nhiệt độ của chất sau mỗi phút, thu được kết quả:
	\begin{table}[H]
		\centering
		\begin{tabular}{|c|c|c|}
			\hline
			$t$ (phút) & Nhiệt độ (${}^\circ$C) & Thể \\
			\hline
			0 & 61 & rắn \\
			\hline
			1 & 68 &  \\
			\hline
			2 & 74 &  \\
			\hline
			3 & 80 &  \\
			\hline
			4 & 80 &  \\
			\hline
			5 & 80 &  \\
			\hline
			6 & 80 &  \\
			\hline
			7 & 85 &  \\
			\hline
		\end{tabular}
	\end{table}
	\noindent(a) Cho biết nhiệt độ nóng chảy của băng phiến. (b) Ghi thể của băng phiến tại các nhiệt độ trên. (c) So sánh nhiệt độ nóng chảy của băng phiến với nhiệt độ sôi của nước. (d) Tại sao để băng phiến nóng chảy cần phải đun nóng, còn để nước đá nóng chảy chỉ cần để nước đá ở nhiệt độ phòng?
\end{baitoan}

\begin{baitoan}[\cite{ncpt_KHTN_6_tap_1}, 18., p. 27]
	Đổ cồn ra 2 cốc. Đặt 1 cốc vào chậu đựng nước nóng. (a) So sánh tốc độ bay hơi giữa 2 cốc. (b) Để giảm tốc độ bay hơi của cồn ta làm thế nào?
\end{baitoan}

\begin{baitoan}[\cite{ncpt_KHTN_6_tap_1}, 19., p. 27]
	Quan sát 1 cây nến. Khi đốt thì xung quanh chỗ cháy chảy lỏng. Giải thích.
\end{baitoan}

\begin{baitoan}[\cite{ncpt_KHTN_6_tap_1}, 20., p. 27]
	Khi tưới cây ta thường tưới vào buổi chiều tối bởi vì tưới vào chiều tối sẽ đỡ tốn nước hơn vào ban ngày. Giải thích.
\end{baitoan}

\begin{baitoan}[\cite{ncpt_KHTN_6_tap_1}, 21., p. 27]
	Ở $37^\circ${\rm C, 1 L} nước có khối lượng là {\rm0.9933 kg}. Nếu làm lạnh lượng nước này xuống $4^\circ${\rm C}, thể tích nước giảm xuống còn {\rm0.9933 L}. Ở $0^\circ${\rm C}, nước lỏng đông đặc, thể tích đá nước là {\rm0.9935 L}. (a) Tính khối lượng riêng của nước lỏng ở $37^\circ${\rm C} \& $4^\circ${\rm C}. Nhận xét: khi nhiệt độ giảm đến $4^\circ${\rm C}, khối lượng riêng của nước giảm hay tăng? (b) Tính khối lượng riêng của đá nước. So sánh khối lượng riêng của đá nước với nước lỏng ở $4^\circ${\rm C}. (c) Từ khối lượng riêng của đá nước \& nước lỏng ở $4^\circ${\rm C}, giải thích tại sao viên đá nước nổi trên mặt nước. (d) Để có 1 chai nước đá, Nam cho đầy nước vào chai, đậy chặt nắp, \& cho vào ngăn đá tủ lạnh. Có nên làm vậy không? Dự đoán điều gì sẽ xảy ra khi lấy chai nước ra khỏi tủ lạnh.
\end{baitoan}

%------------------------------------------------------------------------------%

\section{Oxygen \& Không Khí}

\begin{baitoan}[\cite{ncpt_KHTN_6_tap_1}, 1., p. 32]
	Mưa sao băng xảy ra ở tầng khí quyển nào? {\sf A.} Tầng bình lưu. {\sf B.} Tầng đối lưu. {\sf C.} Tầng trung lưu. {\sf D.} Tầng nhiệt.
\end{baitoan}

\begin{baitoan}[\cite{ncpt_KHTN_6_tap_1}, 2., p. 32]
	Tần khí quyển nào có khí ozone $\rm O_3$? {\sf A.} Tầng bình lưu. {\sf B.} Tầng đối lưu. {\sf C.} Tầng trung lưu. {\sf D.} Tầng nhiệt.
\end{baitoan}

\begin{baitoan}[\cite{ncpt_KHTN_6_tap_1}, 3., p. 32]
	Tác hại khi tầng ôzne bị suy giảm: {\sf A.} làm lượng tia cực tím chiếu xuống Trái Đất tăng lên. {\sf B.} hình thành sấm sét, mưa bão. {\sf C.} gây cháy rừng. {\sf D.} gây hiệu ứng nhà kính.
\end{baitoan}

\begin{baitoan}[\cite{ncpt_KHTN_6_tap_1}, 4., p. 32]
	Cho tàn đóm đỏ vào 1 bình chứa khí X \& thấy que đóm bùng cháy. Khí X? {\sf A.} Nitrogen $\rm N_2$. {\sf B.} Carbon dioxide $\rm CO_2$. {\sf C.} Helium {\rm He}. {\sf D.} Oxygen $\rm O_2$.
\end{baitoan}

\begin{baitoan}[\cite{ncpt_KHTN_6_tap_1}, 5., p. 32]
	Để tách được {\rm1 L} khí argon cần {\rm106 L} không khí. Hàm lượng khí argon trong không khí: {\sf A.} $1.06\%$. {\sf B.} $0.94\%$. {\sf C.} $0.106\%$. {\sf D.} $9.4\%$.
\end{baitoan}

\begin{baitoan}[\cite{ncpt_KHTN_6_tap_1}, 6., p. 33]
	Cho khí nào vào khinh khí cầu để khinh khí cầu bay lên? {\sf A.} Oxygen $\rm O_2$. {\sf B.} Không khí. {\sf C.} Carbon dioxide $\rm CO_2$. {\sf D.} Helium {\rm He}.
\end{baitoan}

\begin{baitoan}[\cite{ncpt_KHTN_6_tap_1}, 7., p. 33]
	Khi đi máy bay, lúc cất cánh \& hạ cánh, hành khách thường cảm thấy sóc. Giải thích.
\end{baitoan}

\begin{baitoan}[\cite{ncpt_KHTN_6_tap_1}, 8., p. 33]
	Bơm căng 2 quả bóng bay cùng kích thước, dán 1 miếng băng dính nhỏ lên 2 quả bóng. Treo cẩn thận 2 quả bóng lên thành treo \& điều chỉnh cho thanh treo thăng bằng. Dùng kim châm vào vị trí có băng dính của 1 quả bóng để làm bóng xì hơi. Quan sát thấy thanh treo bị lệch về 1 phía. (a) Thanh treo sẽ lệch về bên nào? (b) Vì sao thanh treo bị lệch?
\end{baitoan}

\begin{baitoan}[\cite{ncpt_KHTN_6_tap_1}, 9., p. 33]
	Treo cẩn thận 2 túi giấy bóng nhẹ cùng kích thước lên thanh treo \& điều chỉnh cho thanh treo thăng bằng. Đặt phía dưới 1 túi giấy bóng 1 bóng đèn để làm nóng. Ta sẽ thấy thanh treo bị lệch. (a) Thanh treo sẽ lệch về bên nào? (b) Vì sao thanh treo bị lệch?
\end{baitoan}

\begin{baitoan}[\cite{ncpt_KHTN_6_tap_1}, 10., p. 33]
	Trong khinh khí cầu có chứa khí helium. Bên dưới khí cầu có 1 bệ đốt. Khi đốt nóng, khinh khí cầu sẽ bay lên. Khi dừng đốt, khinh khí cầu sẽ hạ xuống mặt đất. Giải thích cách hoạt động này của khinh khí cầu.
\end{baitoan}

\begin{baitoan}[\cite{ncpt_KHTN_6_tap_1}, 11., p. 33]
	Helium được tách từ không khí. Để thu được 1 bình chứa {\rm120 g} helium lỏng thì cần bao nhiêu $\rm m^3$ không khí? Biết trong không khí có chứa $0.0005\%$ thể tích là khí helium, khối lượng riêng của helium là {\rm0.16 kg{\tt/}$\rm m^3$}.
\end{baitoan}

\begin{baitoan}[\cite{ncpt_KHTN_6_tap_1}, 12., p. 34]
	Trong {\rm100 L} không khí, người ta đo được: {\rm21 L} oxygen có khối lượng {\rm27.5 g, \rm78 L} nitrogen có khối lượng {\rm89.4 g, 1L} carbon dioxide có khối lượng {\rm1.8 g} (các khí khác có thành phần \& khối lượng không đáng kể). (a) Tính khối lượng riêng của khí oxygen, nitrogen, \& carbon dioxide. (b) Tính khối lượng riêng của không khí. (c) Khí nào nặng hơn không khí? Khí nào nhẹ hơn không khí? (d) Khí carbon dioxide thường tích tụ trong đáy giếng hoặc trên nền hang sâu. Giải thích.
\end{baitoan}

\begin{baitoan}[\cite{ncpt_KHTN_6_tap_1}, 13., p. 34]
	Biết khối lượng của không khí là {\rm1.18 kg{\tt/}$\rm m^3$}. Tính khối lượng không khí chứa trong căn phòng hình hộp chữ nhật có diện tích $\rm90\ m^2$, chiều cao {\rm3 m}.
\end{baitoan}

\begin{baitoan}[\cite{ncpt_KHTN_6_tap_1}, 14., p. 34]
	Bơm 1 chất khí vào quả bóng thể tích {\rm1.5 L}, rồi cân thấy khối lượng là {\rm1.6 g} (vỏ quá bóng làm bằng chất rất nhẹ, khối lượng của nó có thể bỏ qua). (a) Tính khối lượng riêng của chất khí này. (b) Sau khi bơm khí, quả bóng sẽ có xu hướng bay lên trên hay nằm trên mặt đất? Biết khối lượng riêng của không khí là {\rm1.18 kg{\tt/}$\rm m^3$}.
\end{baitoan}

\begin{baitoan}[\cite{ncpt_KHTN_6_tap_1}, 15., pp. 34--35]
	Cho 3 chiếc đinh sắt vào 3 ống nghiệm có nút (hoặc 3 lọ thủy tinh có nắp kín). Lọ 1: đổ nước ngập nửa chiếc đinh. Lọ 2: cho thêm gói hút ẩm vào. Lọ 3: đổ từ từ nước đun sôi để nguội cho ngập chiếc đinh, rồi đổ lên trên nước 1 lớp dầu ăn. Sau 1 thời gian khoảng 3--4 ngày, nhận thấy chiếc đinh ở lọ 1 đã bị gỉ, ở lọ 2 \& 3 không bị gỉ \& vẫn sáng bóng. (a) Cho biết: Cho gói hút ẩm vào lọ 2 để làm gì? Cho 1 lớp dầu vào lọ 3 để làm gì? (b) Chỉ ra chiếc đinh sắt trong lọ nào: (i) Chỉ tiếp xúc với nước. (ii) Chỉ tiếp xúc với không khí khô. (iii) Tiếp xúc cả với nước \& không khí. (c) Chỉ ra khi nào chiếc đinh bị gỉ mạnh nhất. (d) Đề xuất phương pháp để bảo quản các cổ vật bằng sắt.
\end{baitoan}

\begin{baitoan}[\cite{ncpt_KHTN_6_tap_1}, 16., p. 35]
	Để đốt cháy {\rm12 g} than cần {\rm24.4 L} khí oxygen, sinh ra {\rm44 g} khí carbon dioxide. (a) Để đốt cháy hết {\rm1 kg} than cần bao nhiêu {\rm L} khí oxygen? Khối lượng khí carbon dioxide sinh ra là bao nhiêu? (b) Để đốt cháy hết {\rm12 g} than cần bao nhiêu {\rm L} không khí? (Coi oxygen chiếm $20\%$ thể tích không khí).
\end{baitoan}

\begin{baitoan}[\cite{ncpt_KHTN_6_tap_1}, 17., p. 35]
	1 số khí có thể được thu bằng phương pháp ``dời chỗ khí.'' Cho khí sinh ra đi vào bình, khí đó sẽ đẩy không khí ra \& chiếm chỗ trong bình. Sau 1 thời gian, đậy chặt nút bình lại, ta thu được bình chứa khí. Trong số các khí: sulfur dioxide, ammonia, hydrogen chloride, hydrogen sulfide, cho biết: Khí nào có thể thu được theo cách dẫn vào bình đựng ngửa? Khí nào có thể thu được theo cách dẫn vào bình đựng úp? Cho biết khối lượng riêng của sulfur dioxide, ammonia, hydrogen chloride, \& hydrogen sulfide lần lượt là $2.62,0.7,1.49,1.39$.
\end{baitoan}

%------------------------------------------------------------------------------%

\section{1 Số Vật Liệu, Nhiên Liệu, Nguyên Liệu, Lương Thực, Thực Phẩm Thông Dụng}

\begin{baitoan}[\cite{ncpt_KHTN_6_tap_1}, 1., p. 47]
	Hợp kim nào sau đây nhẹ nhất? {\sf A.} Đồng thanh. {\sf B.} Thép. {\sf C.} Duralumin. {\sf D.} Đồng nhôm.
\end{baitoan}

\begin{baitoan}[\cite{ncpt_KHTN_6_tap_1}, 2., p. 47]
	Hợp kim của thiếc thường được dùng để hàn vì: {\sf A.} thiếc dẫn nhiệt rất tốt. {\sf B.} thiếc là kim loại nặng. {\sf C.} nhiệt độ nóng chảy của thiếc thấp. {\sf D.} thiếc mềm, dẻo, dễ uốn.
\end{baitoan}

\begin{baitoan}[\cite{ncpt_KHTN_6_tap_1}, 3., p. 47]
	Hợp kim của sắt thường dùng để làm khung nhà, làm cầu vì sắt: {\sf A.} có nhiệt độ nóng chảy rất cao. {\sf B.} cứng chắc \& dẻo. {\sf C.} khối lượng riêng rất lớn. {\sf D.} dễ bị han gỉ.
\end{baitoan}

\begin{baitoan}[\cite{ncpt_KHTN_6_tap_1}, 4., p. 47]
	Khả năng dẫn điện của bạc, nhôm, \& đồng tăng dần theo thứ tự: {\sf A.} nhôm, đồng, bạc. {\sf B.} đồng, bạc, nhôm. {\sf C.} bạc, đồng, nhôm. {\sf D.} nhôm, bạc, đồng.
\end{baitoan}

\begin{baitoan}[\cite{ncpt_KHTN_6_tap_1}, 5., p. 47]
	Cho 4 kim loại: vàng, bạc, thiếc, sắt. Kim loại dễ phản ứng với oxygen \& hơi nước trong không khí nhất: {\sf A.} Vàng. {\sf B.} Bạc. {\sf C.} Thiếc. {\sf D.} Sắt.
\end{baitoan}

\begin{baitoan}[\cite{ncpt_KHTN_6_tap_1}, 6., p. 48]
	{\rm Đ{\tt/}S?} (a) Thủy ngân dễ nóng chảy nên thường được dùng để hàn. (b) Đồng là kim loại nhẹ nên thường được dùng để chế tạo đồ gia dụng. (c) Sắt dẫn điện tốt nên thường được dùng làm dây dẫn điện. (d) Nhôm dẫn nhiệt tốt nên thường dùng làm xoong, chảo.
\end{baitoan}

\begin{baitoan}[\cite{ncpt_KHTN_6_tap_1}, 7., p. 48]
	{\rm Đ{\tt/}S?} Dựa vào nhiệt độ nóng chảy của các kim loại: (a) Khi làm nguội hỗn hợp lỏng gồm đồng \& nhôm, nhôm sẽ đông đặc trước. (b) Khi thiếc bắt đầu nóng chảy thì đồng \& sắt vẫn ở thể rắn. (c) Để nấu chảy đồng có thể đun trong các nồi bằng thép. (d) Thủy ngân là kim loại duy nhất ở thể lỏng ở nhiệt độ phòng $25^\circ${\rm C}.
\end{baitoan}

\begin{baitoan}[\cite{ncpt_KHTN_6_tap_1}, 8., p. 48]
	Ghép nội dung sao cho hợp lý để giải thích về ứng dụng của nhôm: 1. Nhôm \& hợp kim nhôm--đồng thường dùng làm dây dẫn điện. 2. Hợp kim của nhôm dùng làm thân vỏ máy bay, tàu vũ trụ. 3. Nhôm dùng làm màng bọc thực phẩm. 4. Hợp kim nhôm thường được dùng làm xoong, nồi, chảo, $\ldots$ (a) vì dẻo. (b) vì dẫn nhiệt tốt. (c) vì dẫn điện tốt. (d) vì khối lượng riêng nhỏ. (e) vì có ánh kim.
\end{baitoan}

\begin{baitoan}[\cite{ncpt_KHTN_6_tap_1}, 9., p. 48]
	Điền từ thích hợp: Gốm sứ, chất dẻo, kim loại đều là các vật liệu được sử dụng trong ngành công nghiệp điện. Sứ cách điện dùng để cố định khoảng cách giữa các đường dây điện cao thế có (a) $\ldots$ rất cao \& khả năng (b) $\ldots$ tốt. Trong khi đó, (c) $\ldots$ có khả năng cách điện tốt nhưng dễ bị chảy mềm khi gặp nhiệt độ cao nên thường được dùng làm vỏ dây điện thông thường. Ngược lại, kim loại do có khả năng (d) $\ldots$ tốt nên được dùng làm dây dẫn. Bạc \& vàng là các kim loại dẫn điện tốt nhất, nhưng không dùng làm dây dẫn vì giá thành cao. Các loại dây dẫn thường làm bằng (e) $\ldots$ hoặc hợp kim của nó với nhôm.
\end{baitoan}

\begin{baitoan}[\cite{ncpt_KHTN_6_tap_1}, 10., p. 48]
	{\rm Đ{\tt/}S?} (a) Nhựa nhiệt dẻo có thể tái chế được. (b) Đun nóng nhựa nhiệt rắn sẽ xảy ra phản ứng hóa học. (c) Sau khi nhựa nhiệt dẻo chảy mềm, để nguội sẽ thành nhựa nhiệt rắn. (d) Nhựa nhiệt dẻo \& nhựa nhiệt rắn đều là các chất cách điện.
\end{baitoan}

\begin{baitoan}[\cite{ncpt_KHTN_6_tap_1}, 11., p. 49]
	Xăng, dầu hỏa, \& dầu diesel có các đặc điểm chung: (a) có nguồn gốc dầu mỏ. (b) chứa các hợp chất hydrocarbon. (c) chất lỏng sánh, nhớt. (d) nhẹ hơn nước.
\end{baitoan}

\begin{baitoan}[\cite{ncpt_KHTN_6_tap_1}, 12., p. 49]
	Hỗn hợp nào sau đây có thể dùng làm diesel sinh học? {\sf A.} Dầu hướng dương. {\sf B.} Mủ cao su. {\sf C.} Rỉ đường mía. {\sf D.} Sữa.
\end{baitoan}

\begin{baitoan}[\cite{ncpt_KHTN_6_tap_1}, 13., p. 49]
	Dầu diesel \& diesel sinh học có đặc điểm chung: {\sf A.} Là nhiên liệu hóa thạch. {\sf B.} Là chất lỏng, nhớt. {\sf C.} Tan được trong nước nóng. {\sf D.} Dễ bay hơi.
\end{baitoan}

\begin{baitoan}[\cite{ncpt_KHTN_6_tap_1}, 14., p. 49]
	{\rm Đ{\tt/}S?} (a) Xăng không tan trong nước, cồn tan được trong nước. (b) Xăng \& cồn đều là các chất lỏng, nhẹ hơn nước. (c) Loại xăng có pha cồn dễ hút hơi nước trong không khí. (d) Xăng dễ bay hơi, cồn hầu như không bay hơi.
\end{baitoan}

\begin{baitoan}[\cite{ncpt_KHTN_6_tap_1}, 15., p. 49]
	{\rm Đ{\tt/}S?} Dầu diesel sinh học: (a) Tan được trong nước. (b) Tan được trong xăng. (c) Là nhiên liệu không có nguồn gốc dầu mỏ. (d) Khi cháy tiêu thụ khí oxygen.
\end{baitoan}

\begin{baitoan}[\cite{ncpt_KHTN_6_tap_1}, 16., pp. 49--50]
	{\rm Đ{\tt/}S?} Khí thiên nhiên: (a) Thuộc loại nhiên liệu hóa thạch. (b) Hòa tan trong nước biển. (c) Có thể sản xuất bằng cách ủ men các phế thải nông nghiệp. (d) Có nhiệt độ sôi cao hơn dầu mỏ.
\end{baitoan}

\begin{baitoan}[\cite{ncpt_KHTN_6_tap_1}, 17., p. 50]
	{\rm Đ{\tt/}S?} Khí sinh học: (a) Khi cháy tỏa nhiều nhiệt. (b) Không tan trong nước. (c) Khi cháy không cần oxygen. (d) Thường được sản xuất bằng cách lên men tinh bột.
\end{baitoan}

\begin{baitoan}[\cite{ncpt_KHTN_6_tap_1}, 18., p. 50]
	Ứng dụng không phải là của khí sinh học? {\sf A.} Nấu ăn. {\sf B.} Sưởi ấm. {\sf C.} Sục vào bể cá. {\sf D.} Thắp sáng.
\end{baitoan}

\begin{baitoan}[\cite{ncpt_KHTN_6_tap_1}, 19., p. 50]
	Vật liệu nào sau đây được sản xuất từ nguyên liệu dầu mỏ? {\sf A.} Chất dẻo. {\sf B.} Sành. {\sf C.} Xi măng. {\sf D.} Giấy.
\end{baitoan}

\begin{baitoan}[\cite{ncpt_KHTN_6_tap_1}, 20., p. 50]
	Nguyên liệu nào dùng để sản xuất phân bón? {\sf A.} Magnetite. {\sf B.} Apatite. {\sf C.} Hematite. {\sf D.} Bauxite.
\end{baitoan}

\begin{baitoan}[\cite{ncpt_KHTN_6_tap_1}, 21., p. 50]
	Vùng đất đỏ Tây Nguyên có quặng X với trữ lượng lớn. Từ quặng X, tinh chế để thu 1 kim loại được dùng nhiều trong xây dựng, hàng không. Quặng X? {\sf A.} Magnetite. {\sf B.} Apatite. {\sf C.} Hematite. {\sf D.} Bauxite.
\end{baitoan}

\begin{baitoan}[\cite{ncpt_KHTN_6_tap_1}, 22., p. 50]
	Có thể phân biệt quặng hematite \& magnetite dựa vào: {\sf A.} Khối lượng riêng (1 loại nặng, 1 loại nhẹ). {\sf B.} Độ rắn chưacs (1 loại rắn chắc, 1 loại xốp mềm). {\sf C.} Từ tính (1 loại có từ tính, 1 loại không có từ tính). {\sf D.} Tính tan (1 loại tan trong nước, 1 loại không tan).
\end{baitoan}

\begin{baitoan}[\cite{ncpt_KHTN_6_tap_1}, 23., p. 50]
	Ứng dụng không phải của đá vôi? {\sf A.} Khử chua đất. {\sf B.} Sản xuất xi măng. {\sf C.} Sản xuất gốm sứ. {\sf D.} Làm vật liệu xây dựng.
\end{baitoan}

\begin{baitoan}[\cite{ncpt_KHTN_6_tap_1}, 24., p. 50]
	Tính chất không phải của quặng apatite? {\sf A.} Tan tốt trong nước. {\sf B.} Cứng. {\sf C.} Dễ bị vỡ vụn. {\sf D.} Nặng hơn nước.
\end{baitoan}

\begin{baitoan}[\cite{ncpt_KHTN_6_tap_1}, 25., p. 51]
	Chất nào là nguồn cung cấp năng lượng chính cho cơ thể? {\sf A.} Chất bột đường. {\sf B.} Chất béo. {\sf C.} Chất đạm. {\sf D.} Chất xơ.
\end{baitoan}

\begin{baitoan}[\cite{ncpt_KHTN_6_tap_1}, 26., p. 51]
	Cho các loại lương thực--thực phẩm: thịt, cá, đường mía, khoai tây, ngô, bột mì, dầu đậu nành, dầu lạc. Số lương thực--thực phẩm có thành phần chủ yếu là tinh bột?
\end{baitoan}

\begin{baitoan}[\cite{ncpt_KHTN_6_tap_1}, 27., p. 51]
	{\rm Đ{\tt/}S?} (a) Con người không tự tổng hợp được vitamin mà phải đưa vào cơ thể qua ăn uống. (b) Các vitamin khó tan trong nước. (c) Khi vitamin trong cơ thể bị thừa, nó sẽ tích trữ dưới dạng chất béo. (d) Khi trong cơ thể thiếu vitamin, chất khoáng sẽ chuyển hóa thành vitamin.
\end{baitoan}

\begin{baitoan}[\cite{ncpt_KHTN_6_tap_1}, 28., p. 51]
	{\rm Đ{\tt/}S?} (a) Chất béo no ở thể rắn còn chất béo không no ở thể lỏng. (b) Các chất béo no có nguồn gốc động vật còn béo không no có nguồn gốc thực vật. (c) Chất béo không no tốt cho cơ thể hơn chất béo no. (d) Dư thừa chất béo no hay chất béo không no đều gây béo phì.
\end{baitoan}

\begin{baitoan}[\cite{ncpt_KHTN_6_tap_1}, 29., p. 51]
	{\rm Đ{\tt/}S?} (a) Vai trò của chất khoáng là cung cấp năng lượng cho cơ thể hoạt động bình thường. (b) Khi cơ thể thừa chất khoáng sẽ gây rối loạn tiêu hóa, hại thận, $\ldots$. (c) Khi cơ thể thiếu chất khoáng, vitamin sẽ được chuyển hóa thành chất khoáng. (d) Cơ thể chỉ hấp thu chất khoáng trong thức ăn khi nó được nấu chín.
\end{baitoan}

\begin{baitoan}[\cite{ncpt_KHTN_6_tap_1}, 30., p. 51]
	{\rm Đ{\tt/}S?} (a) Chất đạm có nhiều trong thịt, cá, trứng, sữa. (b) Thiếu chất đạm sẽ làm cơ thể giảm sức đề kháng. (c) Thừa chất đạm sẽ gây hiện tượng béo phì. (d) Chất đạm trong thực vật không có lợi cho cơ thể.
\end{baitoan}

\begin{baitoan}[\cite{ncpt_KHTN_6_tap_1}, 31., pp. 51--52]
	Rừng là tài nguyên quan trọng của quốc gia. Rừng cung cấp gỗ để làm nhà, làm đồ gia dụng. Rừng cũng cung cấp các sản vật phong phú làm thức ăn cho con người. Cây rừng có tác dụng hút nước ngầm vào đất \& lớp mùn, nhờ đó rừng có khả năng ngăn lũ lụt. Khi gặp mưa, lá cây rừng sẽ ngăn mưa rơi trực tiếp xuống đất. Điều này sẽ giúp lớp đất mùn giàu dinh dưỡng không bị xói mòn, rửa trôi. Cây rừng sẽ hấp thụ khí carbon dioxide \& nhả khí oxygen vào ban ngày, giữ cho hàm lượng carbon dioxide ổn định trong khí quyển. Nhờ đó, rừng giúp hạn chế sự ấm lên của toàn cầu. (a) Tại sao nói sự tồn tại của rừng là điều kiện tiên quyết để bảo tồn các loài động thực vật quý hiếm khỏi tuyệt chủng? (b) 1 số tác hại xảy ra khi rừng bị chặt phá. (c) {\rm Đ{\tt/}S?} (i) Rừng cung cấp các nguyên liệu quan trọng cho đời sống con người. (ii) Rừng có khả năng giữa nước trong lớp đất mùn. (iii) Rừng giúp ngăn hiện tượng xói mòn, sạt lở đất. (iv) Rừng giúp làm giảm dần hiện tượng khí carbon dioxide trong khí quyển.
\end{baitoan}

\begin{baitoan}[\cite{ncpt_KHTN_6_tap_1}, 32., p. 52]
	Cho ví dụ: (a) Loại lương thực--thực phẩm có nguồn gốc khoáng vật. (b) Loại lương thực--thực phẩm có nguồn gốc động vật \& thực vật. (c) Loại vật liệu có nguồn gốc từ khoáng vật. (d) Loại vật liệu có nguồn gốc từ động vật \& thực vật. (e) Loại vật liệu có nguồn gốc từ đất đá. (f) Loại vật liệu có nguồn gốc từ dầu mỏ. (g) Loại nguyên liệu có nguồn gốc từ động vật \& thực vật. (h) Loại nguyên liệu có nguồn gốc từ đất đá. (i) Loại nguyên liệu có nguồn gốc từ dầu mỏ. (j) Loại nhiên liệu có nguồn gốc từ động vật \& thực vật. (k) Loại nhiên liệu có nguồn gốc từ dầu mỏ.
\end{baitoan}

\begin{baitoan}[\cite{ncpt_KHTN_6_tap_1}, 33., p. 52]
	(a) Nếu đặt 1 đồng xu bằng thép lên thủy ngân, nó sẽ nổi hay chìm? Vì sao? (b) Nếu cho thủy ngân vào nước, thủy ngân sẽ nổi hay chìm? Vì sao? (c) Trong phòng thí nghiệm thường có 1 dụng cụ thủy tinh hình chữ U trong có chứa thủy ngân. Biết hơi thủy ngân rất độc. Có biện pháp nào để tránh thủy ngân bay hơi?
\end{baitoan}

\begin{baitoan}[\cite{ncpt_KHTN_6_tap_1}, 34., p. 53]
	Phương pháp mà Acsimet đã dùng để xác định 1 chiếc vương miện được làm bằng vàng nguyên chất không: 1 thỏi vàng nguyên chất \& 1 thỏi hợp kim vàng--bạc có khối lượng đều là {\rm50 g}. Nhúng 2 thỏi ngập trong 2 ống đong giống hệt nhau đều chứa $\rm10.0\ cm^3$ nước. kết quả cho thấy ống A: nước trong ống dâng lên đến vạch $\rm13.3\ cm^3$, ống B: nước trong ống dâng lên đến vạch $\rm12.6\ cm^3$. (a) Tính khối lượng riêng của mỗi thỏi. (b) Dựa vào giá trị khối lượng riêng của vàng, cho biết thỏi vàng nguyên chất đã được nhúng vào ống nào? 
\end{baitoan}

\begin{baitoan}[\cite{ncpt_KHTN_6_tap_1}, 35., p. 53]
	(a) Khi mở nắp bình xăng ta ngửi thấy mùi xăng. Điều đó cho biết tính chất gì của xăng? (b) Không để 1 bình xăng gần ngọn lửa vì rất dễ bị bắt cháy. Giải thích tại sao có hiện tượng bắt cháy?
\end{baitoan}

\begin{baitoan}[\cite{ncpt_KHTN_6_tap_1}, 36., p. 53]
	(a) Có 2 cốc đựng 2 loại chất lỏng không màu là xăng \& cồn. Trình bày 1 số cách để phân loại 2 chất lỏng này. (b) Có 2 cốc đựng 2 loại chất lỏng không màu là nước \& cồn. Trình bày 1 số cách để phân biệt 2 chất lỏng này. (c) Có 2 cốc đựng 2 loại chất lỏng không màu là diesel sinh học \& cồn. Trình bày 1 số cách để phân biệt 2 chất lỏng này.
\end{baitoan}

\begin{baitoan}[\cite{ncpt_KHTN_6_tap_1}, 37., p. 53]
	Xăng E5 là loại xăng có chứa $95\%$ thể tích là xăng \& $5\%$ thể tích là cồn. Xăng E10 có chứa $90\%$ thể tích là xăng \& $10\%$ thể tích là cồn. Tính toán cho thấy 1 động cơ dùng {\rm1 L} xăng thì đi được quãng đường {\rm14 km}, nếu dùng {\rm1 L} cồn thì đi được quãng đường {\rm10 km}. (a) Nếu dùng {\rm1 L} xăng E5 thì đi được quãng đường bao nhiêu {\rm km}? (b) Nếu dùng {\rm1 L} xăng E10 thì đi được quãng đường bao nhiêu {\rm km}?
\end{baitoan}

\begin{baitoan}[\cite{ncpt_KHTN_6_tap_1}, 38., pp. 53--54]
	{\rm Năng lượng tái tạo} (còn gọi là {\rm năng lượng sạch}) thu được từ các nguồn tài nguyên vô hạn, hoặc các nguồn tài nguyên có thể được bổ sung, tái tạo trong 1 thời gian ngắn. Loại năng lượng này trái ngược với năng lượng thu được từ nhiên liệu hóa thạch, đang được sử dụng nhanh hơn nhiều so với việc được bổ sung. Năng lượng tái tạo thường được cung cấp cho các lĩnh vực quan trọng, e.g., phát điện, làm mát không khí \& nước, sưởi ấm, chạy động cơ, $\ldots$ (a) Loại năng lượng nào trong: điện gió, điện mặt trời, năng lượng sóng biển, nhiệt điện, thủy điện, điện hạt nhân, điện từ khí thiên nhiên, là năng lượng tái tạo? (b) {\rm Đ{\tt/}S?} (i) Năng lượng thu được từ khí sinh học được coi là năng lượng tái tạo. (ii) Năng lượng thu được từ cồn không phải năng lượng tái tạo vì cồn được tạo ra từ lên men tinh bột. (iii) Năng lượng thu được từ đốt than không phải là năng lượng tái tạo. (iv) Diesel sinh học chế tạo từ dầu dừa là 1 loại năng lượng tái tạo.
\end{baitoan}

\begin{baitoan}[\cite{ncpt_KHTN_6_tap_1}, 39., p. 54]
	Khi khai thác gỗ trên rừng, có thể vận chuyển bằng cách thả gỗ trôi từ đầu nguồn nước để nó trôi về phía cuối nguồn. (a) Tại sao gỗ lại nổi trên mặt nước? (b) Nếu sau 1 thời gian, khúc gỗ nổi bị mục \& ngấm nước, nó chìm hay nổi?
\end{baitoan}

\begin{baitoan}[\cite{ncpt_KHTN_6_tap_1}, 40., p. 54]
	Quy trình sản xuất đồ sứ: quặng kaonilite (cao lanh) được nghiền mịn, trộn với nước thành dạng bột nhão \& tạo hình, trang trí \& tráng men. Sau đó, nung với nhiều giai đoạn khác nhau, thu được sứ cứng, bền, không thấm nước. {\rm Đ{\tt/}S?} (a) Quặng kaonilite tan được trong nước. (b) Trong quá trình nung, nước bay hơi khỏi kaonilite. (c) Trong quá trình nung, quặng kaonilite ban đầu đã bị biến đổi hóa học.
\end{baitoan}

\begin{baitoan}[\cite{ncpt_KHTN_6_tap_1}, 41., p. 54]
	Dầu đậu nành, dầu lạc, \& dầu dừa có nhiệt độ nóng chảy lần lượt là $\rm-16^\circ C,3^\circ C,25^\circ C$. (a) 1 hỗn hợp chứa 3 loại dầu trên. Khi làm lạnh hỗn hợp này đến $0^\circ${\rm C} thì: {\sf A.} hỗn hợp bị đông đặc hoàn toàn. {\sf B.} có 2 chất bị đông đặc. {\sf C.} có 1 chất bị đông đặc. {\sf D.} hỗn hợp vẫn ở thể lỏng. (b) Phân biệt 3 loại dầu này khi chúng được đựng trong 3 lọ khác nhau.
\end{baitoan}

\begin{baitoan}[\cite{ncpt_KHTN_6_tap_1}, 42., p. 54]
	(a) Tiến hành thí nghiệm: vắt nửa quả chanh vào {\rm20 mL} sữa tươi. Nêu hiện tượng quan sát được. (b) Tại sao không nên uống nước chanh hoặc ăn cam, bưởi, $\ldots$ ngay sau khi uống sữa?
\end{baitoan}

%------------------------------------------------------------------------------%

\section{Chất Tinh Khiết, Hỗn Hợp, Dung Dịch}

\begin{baitoan}[\cite{ncpt_KHTN_6_tap_1}, 1., p. 57]
	Nêu nhận xét về thành phần chất tinh khiết \& hỗn hợp.
\end{baitoan}

\begin{baitoan}[\cite{ncpt_KHTN_6_tap_1}, 2., p. 57]
	Trộn đều muối với đường. Nếm thử, thấy không còn vị ngọt, mặn ban đầu. Trong hỗn hợp có chất mới tạo ra không?
\end{baitoan}

\begin{baitoan}[\cite{ncpt_KHTN_6_tap_1}, 3., p. 57]
	Khi hòa tan muối ăn vào nước có thu được dung dịch không? Chất tan \& dung môi của dung dịch đó là các chất nào?
\end{baitoan}

\begin{baitoan}[\cite{ncpt_KHTN_6_tap_1}, 4., p. 57]
	Phân biệt huyền phù \& nhũ tương với dung dịch bằng cách nào?
\end{baitoan}

\begin{baitoan}[\cite{ncpt_KHTN_6_tap_1}, 5., p. 57]
	Cho 1 thìa nhỏ copper sulfate (màu trắng) vào cốc nước, khuấy nhẹ, cốc nước chuyển màu xanh, trong suốt. Nhỏ 1 giọt hỗn hợp lên tấm kính, đem phơi khô. Thấy có lớp bột rắn màu trắng trên tấm kính. Giải thích.
\end{baitoan}

\begin{baitoan}[\cite{ncpt_KHTN_6_tap_1}, 6., p. 57]
	1 chất tan có thể có nhiều độ hòa tan trong cùng 1 dung môi không?
\end{baitoan}

\begin{baitoan}[\cite{ncpt_KHTN_6_tap_1}, 7., p. 58]
	Chất tan khí có độ hòa tan không? Có thể đo độ hòa tan của chất khí theo thể tích không?
\end{baitoan}

\begin{baitoan}[\cite{ncpt_KHTN_6_tap_1}, 8., p. 58]
	Sắp xếp độ hòa tan của đường, muối ăn, \& thạch cao (dạng bột) ở nhiệt độ phòng.
\end{baitoan}

\begin{baitoan}[\cite{ncpt_KHTN_6_tap_1}, 9., p. 58]
	Hiện tượng nào cho thấy có chất khí tan trong suối nước khoáng, trong nước ngọt có gas.
\end{baitoan}

\begin{baitoan}[\cite{ncpt_KHTN_6_tap_1}, 10., p. 58]
	Để hòa tan được nhiều muối ăn, ta phải pha muối vào nước nóng hay lạnh? Vì sao?
\end{baitoan}

\begin{baitoan}[\cite{ncpt_KHTN_6_tap_1}, 11., p. 58]
	Tại sao bia, nước giải khát lại phải bảo quản trong tủ lạnh?
\end{baitoan}

\begin{baitoan}[\cite{ncpt_KHTN_6_tap_1}, 12., p. 58]
	{\rm Đ{\tt/}S?} Nhà An có 1 bể cá cảnh. An nghĩ phải đun sôi nước, để nguội rồi đổ vào bể cá, giúp môi trường sống của cá trong sạch. 
\end{baitoan}

\begin{baitoan}[\cite{ncpt_KHTN_6_tap_1}, 13., p. 58]
	Vật thể chỉ chứa chất tinh khiết? {\sf A.} Áo sơ mi. {\sf B.} Bút chì. {\sf C.} Giày da. {\sf D.} Viên kim cương.
\end{baitoan}

\begin{baitoan}[\cite{ncpt_KHTN_6_tap_1}, 14., p. 58]
	Cho các hỗn hợp: dầu ăn, nước tương, rượu, tương ớt. Hỗn hợp nào là dung dịch, huyền phù, nhũ tương?
\end{baitoan}

\begin{baitoan}[\cite{ncpt_KHTN_6_tap_1}, 15., p. 58]
	Pha {\rm10 g} muối vào {\rm50 g} nước, khuấy đều cho muối tan hết. Khối lượng dung dịch thu được? {\sf A.} {\rm60 g}. {\sf B.} {\rm50 g}. {\sf C.} {\rm40 g}. {\sf D.} {\rm10 g}.
\end{baitoan}

\begin{baitoan}[\cite{ncpt_KHTN_6_tap_1}, 16., p. 58]
	Pha {\rm10 g} muối vào {\rm30 g} nước, khuấy kỹ thì còn {\rm1 g} muối chưa tan. Khối lượng của dung dịch thu được? {\sf A.} {\rm40 g}. {\sf B.} {\rm39 g}. {\sf C.} {\rm31 g}. {\sf D.} {\rm30 g}.
\end{baitoan}

\begin{baitoan}[\cite{ncpt_KHTN_6_tap_1}, 17., p. 58]
	Pha {\rm10 g} đường vào cốc lớn đựng {\rm100 g} nước, khuấy kỹ cho tan hết rồi rót đều ra 2 cốc nhỏ bằng nhau. (a) Số {\rm g} đường hòa tan trong mỗi cốc nhỏ: {\sf A.} {\rm10 g}. {\sf B.} {\rm5 g}. {\sf C.} {\rm3 g}. {\sf D.} {\rm7 g}. (b) Khối lượng dung dịch trong mỗi cốc nhỏ: {\sf A.} {\rm50 g}. {\sf B.} {\rm55 g}. {\sf C.} {\rm45 g}. {\sf D.} {\rm5 g}.
\end{baitoan}

\begin{baitoan}[\cite{ncpt_KHTN_6_tap_1}, 18., p. 58]
	Cho 4 cốc chất lỏng có màu sắc khác nhau: không màu, màu vàng, màu xanh, màu đỏ. (a) Cho biết mỗi cốc đó đựng chất lỏng nào: nước cất, nước chè (trà), nước cam, nước rau muống, nước rau dền. (b) Cốc nào đựng chất tinh khiết, cốc nào đựng hỗn hợp?
\end{baitoan}

\begin{baitoan}[\cite{ncpt_KHTN_6_tap_1}, 19., p. 59]
	Cho 4 cốc chất lỏng có vị khác nhau: không vị, mặn, ngọt, chua. (a) Cho biết mỗi cốc đó đựng chất lỏng nào: giấm, nước cất, nước chè (trà), nước muối, nước đường. (b) Cốc nào đựng chất tinh khiết, cốc nào đựng hỗn hợp?
\end{baitoan}

\begin{baitoan}[\cite{ncpt_KHTN_6_tap_1}, 20., p. 59]
	Cho 4 cốc chất lỏng có mùi khác nhau: không mùi, mùi thơm, mùi tanh, mùi hôi. (a) Cho biết mỗi cốc đựng chất lỏng nào: nước gạo rang, nước cất, nước cống, nước xả vải, nước bể cá. (b) Cốc nào đựng chất tinh khiết, cốc nào đựng hỗn hợp?
\end{baitoan}

\begin{baitoan}[\cite{ncpt_KHTN_6_tap_1}, 21., p. 59]
	Cho 4 cốc chất lỏng với các mô tả: Cốc 1: trong suốt, không màu, khi đun nóng không còn lại gì trong cốc. Cốc 2: trong suốt, không màu, khi đun nóng còn lại bột rắn màu trắng trong cốc. Cốc 3: trắng đục, sau 1 thời gian lắng đọng bột màu trắng trong cốc. Cốc 4: trắng đục, sau 1 thời gian tách thành 2 lớp chất lỏng trong suốt. (a) Cho biết mỗi cốc đó đựng chất lỏng nào: nước pha bột sắn, nước muối, cồn tuyệt đối, nước trộn dầu ăn, nước mắm. (b) Cốc nào đựng chất tinh khiết, cốc nào đựng hỗn hợp?
\end{baitoan}

\begin{baitoan}[\cite{ncpt_KHTN_6_tap_1}, 22., p. 59]
	Cho bảng khối lượng riêng của nước muối với các thành phần khác nhau ở $20^\circ${\rm C}:
	\begin{table}[H]
		\centering
		\begin{tabular}{|c|c|c|c|c|c|c|c|c|}
			\hline
			Thành phần muối (\% khối lượng) & 0 & 1 & 2 & 4 & 10 & 16 & 20 & 26 \\
			\hline
			Khối lượng riêng (g{\tt/}mL) & 0.9982 & 1.0053 & 1.0125 & 1.0268 & 1.0707 & 1.1162 & 1.1478 & 1.1972 \\
			\hline
		\end{tabular}
	\end{table}
	\noindent Khối lượng riêng của nước muối có phụ thuộc vào thành phần muối trong đó không? Khi thành phần muối tăng lên thì khối lượng riêng của nước muối thay đổi thế nào?
\end{baitoan}

\begin{baitoan}[\cite{ncpt_KHTN_6_tap_1}, 23., pp. 59--60]
	Cho bảng ghi nhiệt độ sôi của nước muối với các thành phần khác nhau ở áp suất khí quyển:
	\begin{table}[H]
		\centering
		\begin{tabular}{|c|c|c|c|c|c|c|c|c|}
			\hline
			Thành phần muối (\% khối lượng) & 0 & 1 & 2 & 4 & 10 & 16 & 20 & 26 \\
			\hline
			Nhiệt độ sôi (${}^\circ$C) & 100 & 100.178 & 100.357 & 100.730 & 101.118 & 101.486 & 101.851 & 102.242 \\
			\hline
		\end{tabular}
	\end{table}
	\noindent(a) Nhiệt độ sôi của nước muối có phụ thuộc vào thành phần nước muối không? Khi thành phần muối tăng lên thì nhiệt độ sôi của nước muối thay đổi thế nào? (b) Đưa cho 1 cốc nước, làm thế nào biết được đó là nước tinh khiết hay nước muối? Nếu đó là nước muối, có biết được thành phần của nó không?
\end{baitoan}

\begin{baitoan}[\cite{ncpt_KHTN_6_tap_1}, 24., p. 60]
	Ở $20^\circ${\rm C, 100 g} nước hòa tan tối đa {\rm36 g} muối ăn. Cho 1 cốc chứa {\rm50 g} nước. Lấy thìa xúc {\rm25 g} muối ăn cho vào cốc \& khuấy đều thật kỹ. (a) Có thể khuấy cho tan hết muối không? (b) Tính khối lượng dung dịch nước muối thu được.
\end{baitoan}

\begin{baitoan}[\cite{ncpt_KHTN_6_tap_1}, 25., p. 60]
	Lấy {\rm100 g} dung dịch sodium chloride (muối ăn) bão hòa ở $20^\circ${\rm C} vào cốc chịu nhiệt đem đun nóng trên ngọn lửa đèn cồn. Tính khối lượng muối còn lại trong cốc sau khi cô cạn dung dịch.
\end{baitoan}

\begin{baitoan}[\cite{ncpt_KHTN_6_tap_1}, 26., p. 60]
	Cho bảng độ hòa tan (số g chất tan{\tt/}100 g nước) các chất:
	\begin{table}[H]
		\centering
		\begin{tabular}{|c|c|c|c|c|c|}
			\hline
			Chất tan & Potassium chloride & Sodium chloride & Sodium bicarbonate & Potassium chlorate & Calcium carbonate \\
			\hline
			Độ hòa tan ở $20^\circ$C & 34 & 36 & 9.6 & 7 & 0.0015 \\
			\hline
		\end{tabular}
	\end{table}
	\noindent(a) Sắp xếp các chất theo chiều tăng dần của độ hòa tan. (b) Ở $20^\circ${\rm C}, độ hòa tan của sodium chloride lớn hơn độ hòa tan của calcium carbonate bao nhiêu lần? (c) Để hòa tan {\rm1 g} calcium carbonate (thành phần chính của đá vôi) ở $20^\circ${\rm C} cần ít nhất bao nhiêu {\rm kg} nước?
\end{baitoan}

\begin{baitoan}[\cite{ncpt_KHTN_6_tap_1}, 27., p. 60]
	Ở 1 vùng biển, trung bình cứ {\rm100 g} nước biển có {\rm3.5 g} muối ăn tan. 1 ngày nắng nóng $40^\circ${\rm C}, người làm muối lấy nước biển vào ruộng cát, phơi cho nước bay hơi để lấy nước muối đặc, chuyển sang sân muối. Hỏi từ $1$ tấn nước biển sẽ thu được bao nhiêu {\rm kg} nước muối bão hòa? Biết ở nhiệt độ này độ hòa tan của muối ăn là {\rm37 g{\tt/}100 g} nước.
\end{baitoan}

\begin{baitoan}[\cite{ncpt_KHTN_6_tap_1}, 28., p. 60]
	Ở áp suất {\rm1 atm}, độ hòa tan của khí carbon dioxide tại các nhiệt độ $20^\circ${\rm C} \& $60^\circ${\rm C} lần lượt là {\rm1.73 g{\tt/}100 g} nước \& {\rm0.07 g{\tt/}100 g} nước. 1 cốc nước đựng {\rm250 g} nước hòa tan bão hòa khí carbonic ở $20^\circ${\rm C}, khi làm nóng lên đến $60^\circ${\rm C} sẽ giải phóng bao nhiêu {\rm g} khí carbon dioxide?
\end{baitoan}

%------------------------------------------------------------------------------%

\section{Tách Chất Khỏi Hỗn Hợp}

\begin{baitoan}[\cite{ncpt_KHTN_6_tap_1}, 1., p. 64]
	Để loại bỏ đất đá trong quặng thiếc, khuấy quặng đó trong nước, kim loại nổi lên trên được thu lại, đất đá lắng phía dưới bị loại bỏ. Phương pháp nào sau đây đã được dùng để làm sạch quặng? {\sf A.} Chưng cất. {\sf B.} Chiết. {\sf C.} Lắng, gạn. {\sf D.} Lọc.
\end{baitoan}

\begin{baitoan}[\cite{ncpt_KHTN_6_tap_1}, 2., p. 64]
	Dẫn nước bẩn qua hỗn hợp cát vàng \& than củi, thu được nước sạch. Phương pháp nào đã được sử dụng để loại bỏ chất bẩn trong nước? {\sf A.} Cô cạn. {\sf B.} Chiết. {\sf C.} Lắng, gạn. {\sf D.} Lọc.
\end{baitoan}

\begin{baitoan}[\cite{ncpt_KHTN_6_tap_1}, 3., p. 64]
	Tinh dầu bưởi có thể thu được theo cách: Đun sôi vỏ bưởi trong nước, thu lấy hơi. Sau đó làm lạnh hơi, thu được tinh dầu bưởi. Cách nào đã được dùng để tách tinh dầu bưởi từ vỏ? {\sf A.} Chưng cất. {\sf B.} Kết tinh. {\sf C.} Lọc. {\sf D.} Lắng, gạn.
\end{baitoan}

\begin{baitoan}[\cite{ncpt_KHTN_6_tap_1}, 4., p. 65]
	Gạo sau khi xay thường có lẫn vỏ trấu. Để loại bỏ vỏ trấu ra khỏi gạo, có thể làm theo cách nào? {\sf A.} Dùng quạt thổi cho vỏ trấu bay đi. {\sf B.} Dùng lưới để lọc gạo. {\sf C.} Phơi nắng cho vỏ trấu bay hơi. {\sf D.} Chờ cho hạt gạo nặng lắng xuống dưới, tách bỏ lớp vỏ trấu ở phía trên.
\end{baitoan}

\begin{baitoan}[\cite{ncpt_KHTN_6_tap_1}, 5., p. 65]
	Nước bị đục thường là do lẫn các hạt đất lơ lửng. Để nước nhanh trong, có thể dùng cách ``đánh phèn''. Cho chất phèn chua vào nước \& khuấy đều. Phèn chua sẽ kết hợp với hạt đất để tạo thành các hạt có kích thước lớn \& chìm xuống dưới. {\rm Đ{\tt/}S?} (a) Các hạt đất lơ lửng rất khó sa lắng. (b) Hạt đất kết hợp với phèn chua sẽ tạo thành loại hạt mới, dễ dàng sa lắng. (c) Phèn chua có tác dụng làm nước nhẹ hơn, nổi lên trên. (d) Nước sau khi đánh phèn được làm sạch bằng phương pháp lắng gạn.
\end{baitoan}

\begin{baitoan}[\cite{ncpt_KHTN_6_tap_1}, 6., p. 65]
	{\rm Đ{\tt/}S?} Về phương pháp lắng: (a) Chất rắn có thể được tách khỏi chất khí nhờ phương pháp lắng. (b) Nên khuấy đều hỗn hợp rắn--lỏng để tăng tốc độ lắng. (c) Hạt rắn có khối lượng \& kích thước càng lớn thì bị lắng càng nhanh. (d) Sau khi chất rắn lắng, có thể tách chất rắn khỏi chất lỏng bằng cách gạn.
\end{baitoan}

\begin{baitoan}[\cite{ncpt_KHTN_6_tap_1}, 7., p. 65]
	{\rm Đ{\tt/}S?} Về phương pháp sắc ký? (a) Chất lỏng dùng trong sắc ký phải hòa tan được chất màu trong hỗn hợp. (b) Các chất màu khác nhau có độ tan trong nước khác nhau. (c) Khi dung môi đi qua vị trí giọt mực, các chất màu chảy loang xung quanh giấy. (d) Trong hỗn hợp màu có bao nhiêu chất thì sẽ thấy bấy nhiêu điểm trên giấy sắc ký.
\end{baitoan}

\begin{baitoan}[\cite{ncpt_KHTN_6_tap_1}, 8., p. 65]
	{\rm Đ{\tt/}S?} Về phương pháp chưng cất phân đoạn dầu mỏ? (a) Phần đỉnh tháp chưng cất chứa các chất khó sôi nhất. (b) Khi càng lên cao, nhiệt độ tháp càng giảm dần. (c) Xăng dùng cho máy bay khó sôi hơn xăng dùng cho ôtô, xe máy. (d) Chất lấy ra từ giữa thân tháp có nhiệt độ sôi thấp hơn chất ở đáy tháp.
\end{baitoan}

\begin{baitoan}[\cite{ncpt_KHTN_6_tap_1}, 9., p. 66]
	Ghép hỗn hợp với tên phương pháp tách có thể dùng để tách hỗn hợp đó: 1. Nước \& cát. 2. Nước \& muối. 3. Bột sắt \& cát. 4. Dầu ăn \& nước. 5. Các loại màu thực phẩm. (a) Sắc ký. (b) Kết tinh. (c) Chưng cất. (d) Lắng, gạn. (e) Dùng nam châm. (f) Dùng phễu chiết.
\end{baitoan}

\begin{baitoan}[\cite{ncpt_KHTN_6_tap_1}, 10., p. 66]
	Điền từ: (a) Khi tách chất bằng cách $\ldots$, cần chờ cho chất rắn lơ lửng tụ lại ở đáy bình. (b) Quá trình $\ldots$ cho phép tách các thành phần trong 1 hỗn hợp lỏng phân lớp. (c) Quá trình $\ldots$ cho phép tách các thành phần trong 1 hỗn hợp lỏng đồng nhất. (d) Dùng phương pháp $\ldots$ có thể tách các chất trong 1 hỗn hợp màu.
\end{baitoan}

\begin{baitoan}[\cite{ncpt_KHTN_6_tap_1}, 11., p. 66]
	Đường bị lẫn 1 ít cát. Trình bày cách để thu được đường sạch.
\end{baitoan}

\begin{baitoan}[\cite{ncpt_KHTN_6_tap_1}, 12., p. 66]
	Trong canh có rất nhiều mỡ. Trình bày 1 cách để tách bớt mỡ trong canh.
\end{baitoan}

\begin{baitoan}[\cite{ncpt_KHTN_6_tap_1}, 13., p. 66]
	Khi phơi nước biển để làm muối, thường chọn ngày nắng \& có nhiều gió. Giải thích.
\end{baitoan}

\begin{baitoan}[\cite{ncpt_KHTN_6_tap_1}, 14., p. 66]
	1 dung dịch chứa chất tan rắng là sodium sulfate. Nêu các cách có thể dùng để tách sodium sulfate ra khỏi dung dịch.
\end{baitoan}

\begin{baitoan}[\cite{ncpt_KHTN_6_tap_1}, 15., p. 66]
	Trong 1 thí nghiệm, dùng giấy lọc để lọc các chất bẩn có trong nước. Nhưng sau khi tiến hành cẩn thận, vẫn thấy nước lọc đục chứ không trong suốt như nước máy. Giải thích.
\end{baitoan}

%------------------------------------------------------------------------------%

\section{Tế Bào -- Đơn Vị Cơ Sở của Sự Sống}

%------------------------------------------------------------------------------%

\section{Từ Tế Bào Đến Cơ Thể}

%------------------------------------------------------------------------------%

\section{Miscellaneous}

%------------------------------------------------------------------------------%

\printbibliography[heading=bibintoc]
	
\end{document}