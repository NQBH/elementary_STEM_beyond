\documentclass{article}
\usepackage[backend=biber,natbib=true,style=alphabetic,maxbibnames=50]{biblatex}
\addbibresource{/home/nqbh/reference/bib.bib}
\usepackage[utf8]{vietnam}
\usepackage{tocloft}
\renewcommand{\cftsecleader}{\cftdotfill{\cftdotsep}}
\usepackage[colorlinks=true,linkcolor=blue,urlcolor=red,citecolor=magenta]{hyperref}
\usepackage{amsmath,amssymb,amsthm,float,graphicx,mathtools,tikz}
\usetikzlibrary{angles,calc,intersections,matrix,patterns,quotes,shadings}
\allowdisplaybreaks
\newtheorem{assumption}{Assumption}
\newtheorem{baitoan}{Bài toán}
\newtheorem{cauhoi}{Câu hỏi}
\newtheorem{conjecture}{Conjecture}
\newtheorem{corollary}{Corollary}
\newtheorem{dangtoan}{Dạng toán}
\newtheorem{definition}{Definition}
\newtheorem{dinhly}{Định lý}
\newtheorem{dinhnghia}{Định nghĩa}
\newtheorem{example}{Example}
\newtheorem{ghichu}{Ghi chú}
\newtheorem{hequa}{Hệ quả}
\newtheorem{hypothesis}{Hypothesis}
\newtheorem{lemma}{Lemma}
\newtheorem{luuy}{Lưu ý}
\newtheorem{nhanxet}{Nhận xét}
\newtheorem{notation}{Notation}
\newtheorem{note}{Note}
\newtheorem{principle}{Principle}
\newtheorem{problem}{Problem}
\newtheorem{proposition}{Proposition}
\newtheorem{question}{Question}
\newtheorem{remark}{Remark}
\newtheorem{theorem}{Theorem}
\newtheorem{vidu}{Ví dụ}
\usepackage[left=1cm,right=1cm,top=5mm,bottom=5mm,footskip=4mm]{geometry}
\def\labelitemii{$\circ$}
\DeclareRobustCommand{\divby}{%
	\mathrel{\vbox{\baselineskip.65ex\lineskiplimit0pt\hbox{.}\hbox{.}\hbox{.}}}%
}

\title{Problem: Circle -- Bài Tập: Đường Tròn}
\author{Nguyễn Quản Bá Hồng\footnote{Independent Researcher, Ben Tre City, Vietnam\\e-mail: \texttt{nguyenquanbahong@gmail.com}; website: \url{https://nqbh.github.io}.}}
\date{\today}

\begin{document}
\maketitle
\begin{abstract}
	Last updated version: \href{https://github.com/NQBH/elementary_STEM_beyond/blob/main/elementary_mathematics/grade_9/circle/problem/NQBH_circle_problem.pdf}{GitHub{\tt/}NQBH{\tt/}elementary STEM \& beyond{\tt/}elementary mathematics{\tt/}grade 9{\tt/}circle{\tt/}problem: set $\mathbb{Q}$ of circles [pdf]}.\footnote{\textsc{url}: \url{https://github.com/NQBH/elementary_STEM_beyond/blob/main/elementary_mathematics/grade_9/circle/problem/NQBH_circle_problem.pdf}.} [\href{https://github.com/NQBH/elementary_STEM_beyond/blob/main/elementary_mathematics/grade_9/circle/problem/NQBH_circle_problem.tex}{\TeX}]\footnote{\textsc{url}: \url{https://github.com/NQBH/elementary_STEM_beyond/blob/main/elementary_mathematics/grade_9/rational/problem/NQBH_circle_problem.tex}.}. 
\end{abstract}
\tableofcontents

%------------------------------------------------------------------------------%

\section{Sự Xác Định Đường Tròn. Tính Chất Đối Xứng của Đường Tròn}

\begin{baitoan}[\cite{Binh_Toan_9_tap_1}, Ví dụ 8, p. 95]
	Cho hình thang cân ABCD. Chứng minh tồn tại 1 đường tròn đi qua cả 4 đỉnh của hình thang.
\end{baitoan}

\begin{baitoan}[\cite{Binh_Toan_9_tap_1}, 50., p. 95]
	Cho $\Delta ABC$ cân tại A nội tiếp đường tròn $(O)$, $AC = 40$ {\rm cm}, $BC = 48$ {\rm cm}. Tính khoảng cách từ O đến BC. 
\end{baitoan}

\begin{baitoan}[\cite{Binh_Toan_9_tap_1}, 51., p. 96]
	Cho $\Delta ABC$ cân tại A nội tiếp đường tròn $(O)$, cạnh bên bằng $b$, đường cao $AH = h$. Tính bán kính của đường tròn $(O)$.
\end{baitoan}

\begin{baitoan}[\cite{Binh_Toan_9_tap_1}, 52., p. 96]
	Cho $\Delta ABC$ nhọn nội tiếp đường tròn $(O;R)$. Gọi M là trung điểm BC. Giả sử O nằm trong $\Delta AMC$ hoặc O nằm giữa A \& M. Gọi I là trung điểm của AC. Chứng minh: (a) Chu vi $\Delta IMC$ lớn hơn $2R$. (b) Chu vi $\Delta ABC$ lớn hơn $4R$.
\end{baitoan}

\begin{baitoan}[\cite{Binh_Toan_9_tap_1}, 53., p. 96]
	Cho $\Delta ABC$ nội tiếp đường tròn $(O)$. Gọi D, E, F lần lượt là trung điểm của BC, CA, AB. Kẻ 3 đường thẳng $DD',EE',FF'$ sao cho $DD'\parallel OA,EE'\parallel OB,FF'\parallel OC$. Chứng minh 3 đường thẳng $DD',EE',FF'$ đồng quy.
\end{baitoan}

\begin{baitoan}[\cite{Binh_Toan_9_tap_1}, 54., p. 96]
	Cho 3 điểm A, B, C bất kỳ \& đường tròn $(O;1)$. Chứng minh tồn tại 1 điểm M nằm trên đường tròn $(O)$ sao cho $MA + MB + MC\ge3$.
\end{baitoan}

%------------------------------------------------------------------------------%

\section{Đường Kính \& Dây của Đường Tròn. Liên Hệ Giữa Dây \& Khoảng Cách Từ Tâm Đến Dây}

\begin{baitoan}[\cite{Binh_Toan_9_tap_1}, Ví dụ 9, p. 96]
	Cho $\Delta ABC$ nhọn nội tiếp đường tròn $(O)$. M là điểm bất kỳ thuộc cung BC không chứa A. Gọi D, E theo thứ tự là các điểm đối xứng với M qua AB, AC. Tìm vị trí của M để DE có độ dài lớn nhất.
\end{baitoan}

\begin{baitoan}[\cite{Binh_Toan_9_tap_1}, Ví dụ 10, p. 97]
	Cho $(O)$ bán kính $OA = 11$ {\rm cm}. Điểm M thuộc bán kính OA \& cách O {\rm7 cm}. Qua M kẻ dây CD có độ dài {\rm18 cm}. Tính MC, MD với $MC < MD$.
\end{baitoan}

\begin{baitoan}[\cite{Binh_Toan_9_tap_1}, Ví dụ 11, p. 97]
	Cho $(O)$ bán kính {\rm15 cm}, điểm M cách O {\rm9 cm}. (a) Dựng dây AB đi qua M \& có độ dài {\rm26 cm}. (b) Có bao nhiêu dây đi qua M \& có độ dài là 1 số nguyên {\rm cm}?
\end{baitoan}

\begin{baitoan}[\cite{Binh_Toan_9_tap_1}, 55., p. 98]
	Tứ giác ABCD có $\widehat{A} = \widehat{C} = 90^\circ$. (a) Chứng minh $AC\le BD$. (b) Trong trường hợp nào thì $AC = BD$?
\end{baitoan}

\begin{baitoan}[\cite{Binh_Toan_9_tap_1}, 56., p. 98]
	Cho $(O)$ đường kính AB, 2 dây AC, AD. Gọi E là điểm bất kỳ trên đường tròn, H, K lần lượt là hình chiếu của E trên AC, AD. Chứng minh $HK\le AB$.
\end{baitoan}

\begin{baitoan}[\cite{Binh_Toan_9_tap_1}, 57., p. 98]
	Cho $(O)$, dây $AB = 24$ {\rm cm}, dây $AC = 20$ {\rm cm} ($\widehat{BAC} < 90^\circ$ \& điểm O nằm trong $\widehat{BAC}$). Gọi M là trung điểm của AC. Khoảng cách từ M đến AB bằng {\rm8 cm}. (a) Chứng minh $\Delta ABC$ cân tại C. (b) Tính bán kính đường tròn.
\end{baitoan}

\begin{baitoan}[\cite{Binh_Toan_9_tap_1}, 58., p. 98]
	Cho $(O)$ bán kính {\rm5 cm}, 2 dây AB \& CD song song với nhau có độ dài theo thứ tự bằng {\rm8 cm} \& {\rm6 cm}. Tính khoảng cách giữa 2 dây.
\end{baitoan}

\begin{baitoan}[\cite{Binh_Toan_9_tap_1}, 59., p. 98]
	Cho $(O)$, đường kính $AB = 13$ {\rm cm}. Dây CD có độ dài {\rm 12 cm} vuông góc với AB tại H. (a) Tính AH, BH. (b) Gọi M, N lần lượt là hình chiếu của H trên AC, BC. Tính diện tích tứ giác CMHN.
\end{baitoan}

\begin{baitoan}[\cite{Binh_Toan_9_tap_1}, 60., p. 99]
	Cho nửa đường tròn tâm O đường kính AB, dây CD. Gọi H, K lần lượt là chân 2 đường vuông góc kẻ từ A, B đến CD. (a) Chứng minh $CH = DK$. (b) Chứng minh $S_{AHKB} = S_{ABC} + S_{ABD}$. (c) Tính diện tích lớn nhất của tứ giác AHKB, biết $AB = 30$ {\rm cm}, $CD = 18$ {\rm cm}.
\end{baitoan}

\begin{baitoan}[\cite{Binh_Toan_9_tap_1}, 61., p. 99]
	Cho $\Delta ABC$, 3 đường cao $AD,BE,CF$. Đường tròn đi qua D, E, F cắt BC, CA, AB lần lượt tại M, N, P. Chứng minh 3 đường thẳng kẻ từ M vuông góc với BC, kẻ từ N vuông góc với AC, kẻ từ P vuông góc với AB đồng quy.
\end{baitoan}

\begin{baitoan}[\cite{Binh_Toan_9_tap_1}, 62., p. 99]
	$\Delta ABC$ cân tại A nội tiếp $(O)$. Gọi D là trung điểm của AB, E lfa trọng tâm của $\Delta ACD$. Chứng minh $OE\bot CD$.
\end{baitoan}

%------------------------------------------------------------------------------%

\section{Vị Trí Tương Đối của Đường Thẳng \& Đường Tròn. Dấu Hiệu Nhận Biết Tiếp Tuyến của Đường Tròn}

\begin{baitoan}[\cite{Binh_Toan_9_tap_1}, Ví dụ 12, p. 99]
	Cho $\Delta ABC$ vuông tại A, $AB < AC$, đường cao $AH$. Gọi E là điểm đối xứng với B qua H. Đường tròn có đường kính EC cắt AC ở K. Chứng minh HK là tiếp tuyến của đường tròn.
\end{baitoan}

\begin{baitoan}[\cite{Binh_Toan_9_tap_1}, Ví dụ 13, p. 100]
	Cho 1 hình vuông $8\times9$ gồm $64$ ô vuông nhỏ. Đặt 1 tấm bìa hình tròn có đường kính $8$ sao cho tâm O của hình tròn trùng với tâm của hình vuông. (a) Chứng minh hình tròn tiếp xúc với 4 cạnh của hình vuông. (b) Có bao nhiêu ô vuông nhỏ bị tấm bìa che lấp hoàn toàn? (c) Có bao nhiêu ô vuông nhỏ bị tấm bìa che lấp (cả che lấp 1 phần \& che lấp hoàn toàn)?
\end{baitoan}

\begin{baitoan}[\cite{Binh_Toan_9_tap_1}, 63., pp. 100--101]
	Cho nửa đường tròn tâm O đường kính AB, M là 1 điểm thuộc nửa đường tròn. Qua M vẽ tiếp tuyến với nửa đường tròn. Gọi D, C lần lượt là hình chiếu của A, B trên tiếp tuyến ấy. (a) Chứng minh M là trung điểm của CD. (b) Chứng minh $AB = BC + AD$. (c) Giả sử $ \widehat{AOM}\ge\widehat{BOM}$, gọi E là giao điểm của AD với nửa đường tròn. Xác định dạng của tứ giác BCDE. (d) Xác định vị trí của điểm M trên nửa đường tròn sao cho tứ giác ABCD có diện tích lớn nhất. Tính diện tích đó theo bán kính $R$ của nửa đường tròn đã cho.
\end{baitoan}

\begin{baitoan}[\cite{Binh_Toan_9_tap_1}, 64., p. 101]
	Cho $\Delta ABC$ cân tại A, I là giao điểm của 3 đường phân giác. (a) Xác định vị trí tương đối của đường thẳng AC với đường tròn $(O)$ ngoại tiếp $\Delta BIC$. (b) Gọi H là trung điểm của BC, IK là đường kính của đường tròn $(O)$. Chứng minh $\dfrac{AI}{AK} = \dfrac{HI}{HK}$.
\end{baitoan}

\begin{baitoan}[\cite{Binh_Toan_9_tap_1}, 65., p. 101]
	Cho nửa đường tròn tâm O đường kính AB, Ax là tiếp tuyến của nửa đường tròn (Ax \& nửa đường tròn nằm cùng phía đối với AB), C là 1 điểm thuộc nửa đường tròn, H là hình chiếu của C trên AB. Đường thẳng qua O \& vuông góc với AC cắt Ax tại M. Gọi I là giao điểm của MB \& CH. Chứng minh $IC = IH$.
\end{baitoan}

\begin{baitoan}[\cite{Binh_Toan_9_tap_1}, 66., p. 101]
	Cho hình thang vuông ABCD, $\widehat{A} = \widehat{D} = 90^\circ$, có $\widehat{BMC} = 90^\circ$ với M là trung điểm của AD. Chứng minh: (a) AD là tiếp tuyến của đường tròn có đường kính BC. (b) BC là tiếp tuyến của đường tròn có đường kính AD.
\end{baitoan}

\begin{baitoan}[\cite{Binh_Toan_9_tap_1}, 67., p. 101]
	Cho nửa đường tròn tâm O đường kính AB, C là 1 điểm thuộc nửa đường tròn, H là hình chiếu của C trên AB. Qua trung điểm M của CH, kẻ đường vuông góc với OC, cắt nửa đường tròn tại D \& E. Chứng minh AB là tiếp tuyến của $(C;CD)$.
\end{baitoan}

\begin{baitoan}[\cite{Binh_Toan_9_tap_1}, 68., p. 101]
	Cho đường tròn tâm O đường kính AB. Gọi $d,d'$ lần lượt là 2 tiếp tuyến tại A, B của đường tròn, $C\in d$ bất kỳ. Đường vuông góc với OC tại O cắt $d'$ tại D. Chứng minh CD là tiếp tuyến của $(O)$.
\end{baitoan}

\begin{baitoan}[\cite{Binh_Toan_9_tap_1}, 69., p. 101]
	Cho nửa đường tròn tâm O đường kính AB, C là 1 điểm thuộc nửa đường tròn. Qua C kẻ tiếp tuyến $d$ với nửa đường tròn. Kẻ 2 tia Ax, By song song với nhau, cắt $d$ theo thứ tự tại D, E. Chứng minh AB là tiếp tuyến của đường tròn đường kính DE.
\end{baitoan}

\begin{baitoan}[\cite{Binh_Toan_9_tap_1}, 70., pp. 101--102]
	Cho đường tròn tâm O có đường kính $AB = 2R$. Gọi $d$ là tiếp tuyến của đường tròn, A là tiếp điểm. Gọi M là điểm bất kỳ thuộc $d$. Qua O kẻ đường thẳng vuông góc với BM, cắt $d$ tại N. (a) Chứng minh tích $AM\cdot AN$ không đổi khi điểm M chuyển động trên đường thẳng $d$. (b) Tìm {\rm GTNN} của MN.
\end{baitoan}

\begin{baitoan}[\cite{Binh_Toan_9_tap_1}, 71., p. 102]
	Cho $\Delta ABC$ cân tại A có $\widehat{A} = \alpha$, đường cao $AH = h$. Vẽ đường tròn tâm A bán kính $h$. 1 tiếp tuyến bất kỳ ($\ne BC$) của đường tròn $(A)$ cắt 2 tia AB, AC theo thứ tự tại $B',C'$. (a) Chứng minh $S_{ABC} = S_{AB'C'}$. (b) Trong các $\Delta ABC$ có $\widehat{A} = \alpha$ \& đường cao $AH = h$, tam giác nào có diện tích nhỏ nhất?
\end{baitoan}

%------------------------------------------------------------------------------%

\section{Tính Chất của 2 Tiếp Tuyến Cắt Nhau}

%------------------------------------------------------------------------------%

\section{Miscellaneous}

%------------------------------------------------------------------------------%

\printbibliography[heading=bibintoc]

\end{document}