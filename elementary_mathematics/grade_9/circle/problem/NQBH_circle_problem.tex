\documentclass{article}
\usepackage[backend=biber,natbib=true,style=alphabetic,maxbibnames=50]{biblatex}
\addbibresource{/home/nqbh/reference/bib.bib}
\usepackage[utf8]{vietnam}
\usepackage{tocloft}
\renewcommand{\cftsecleader}{\cftdotfill{\cftdotsep}}
\usepackage[colorlinks=true,linkcolor=blue,urlcolor=red,citecolor=magenta]{hyperref}
\usepackage{amsmath,amssymb,amsthm,float,graphicx,mathtools,tikz}
\usetikzlibrary{angles,calc,intersections,matrix,patterns,quotes,shadings}
\allowdisplaybreaks
\newtheorem{assumption}{Assumption}
\newtheorem{baitoan}{Bài toán}
\newtheorem{cauhoi}{Câu hỏi}
\newtheorem{conjecture}{Conjecture}
\newtheorem{corollary}{Corollary}
\newtheorem{dangtoan}{Dạng toán}
\newtheorem{definition}{Definition}
\newtheorem{dinhly}{Định lý}
\newtheorem{dinhnghia}{Định nghĩa}
\newtheorem{example}{Example}
\newtheorem{ghichu}{Ghi chú}
\newtheorem{hequa}{Hệ quả}
\newtheorem{hypothesis}{Hypothesis}
\newtheorem{lemma}{Lemma}
\newtheorem{luuy}{Lưu ý}
\newtheorem{nhanxet}{Nhận xét}
\newtheorem{notation}{Notation}
\newtheorem{note}{Note}
\newtheorem{principle}{Principle}
\newtheorem{problem}{Problem}
\newtheorem{proposition}{Proposition}
\newtheorem{question}{Question}
\newtheorem{remark}{Remark}
\newtheorem{theorem}{Theorem}
\newtheorem{vidu}{Ví dụ}
\usepackage[left=1cm,right=1cm,top=5mm,bottom=5mm,footskip=4mm]{geometry}
\def\labelitemii{$\circ$}
\DeclareRobustCommand{\divby}{%
	\mathrel{\vbox{\baselineskip.65ex\lineskiplimit0pt\hbox{.}\hbox{.}\hbox{.}}}%
}

\title{Problem: Circle -- Bài Tập: Đường Tròn}
\author{Nguyễn Quản Bá Hồng\footnote{Independent Researcher, Ben Tre City, Vietnam\\e-mail: \texttt{nguyenquanbahong@gmail.com}; website: \url{https://nqbh.github.io}.}}
\date{\today}

\begin{document}
\maketitle
\begin{abstract}
	Last updated version: \href{https://github.com/NQBH/elementary_STEM_beyond/blob/main/elementary_mathematics/grade_9/circle/problem/NQBH_circle_problem.pdf}{GitHub{\tt/}NQBH{\tt/}elementary STEM \& beyond{\tt/}elementary mathematics{\tt/}grade 9{\tt/}circle{\tt/}problem: set $\mathbb{Q}$ of circles [pdf]}.\footnote{\textsc{url}: \url{https://github.com/NQBH/elementary_STEM_beyond/blob/main/elementary_mathematics/grade_9/circle/problem/NQBH_circle_problem.pdf}.} [\href{https://github.com/NQBH/elementary_STEM_beyond/blob/main/elementary_mathematics/grade_9/circle/problem/NQBH_circle_problem.tex}{\TeX}]\footnote{\textsc{url}: \url{https://github.com/NQBH/elementary_STEM_beyond/blob/main/elementary_mathematics/grade_9/rational/problem/NQBH_circle_problem.tex}.}. 
\end{abstract}
\tableofcontents

%------------------------------------------------------------------------------%

\section{Sự Xác Định Đường Tròn. Tính Chất Đối Xứng của Đường Tròn}

\begin{baitoan}[\cite{Binh_Toan_9_tap_1}, Ví dụ 8, p. 95]
	Cho hình thang cân ABCD. Chứng minh tồn tại 1 đường tròn đi qua cả 4 đỉnh của hình thang.
\end{baitoan}

\begin{baitoan}[\cite{Binh_Toan_9_tap_1}, 50., p. 95]
	Cho $\Delta ABC$ cân tại A nội tiếp đường tròn $(O)$, $AC = 40$ {\rm cm}, $BC = 48$ {\rm cm}. Tính khoảng cách từ O đến BC. 
\end{baitoan}

\begin{baitoan}[\cite{Binh_Toan_9_tap_1}, 51., p. 96]
	Cho $\Delta ABC$ cân tại A nội tiếp đường tròn $(O)$, cạnh bên bằng $b$, đường cao $AH = h$. Tính bán kính của đường tròn $(O)$.
\end{baitoan}

\begin{baitoan}[\cite{Binh_Toan_9_tap_1}, 52., p. 96]
	Cho $\Delta ABC$ nhọn nội tiếp đường tròn $(O;R)$. Gọi M là trung điểm BC. Giả sử O nằm trong $\Delta AMC$ hoặc O nằm giữa A \& M. Gọi I là trung điểm của AC. Chứng minh: (a) Chu vi $\Delta IMC$ lớn hơn $2R$. (b) Chu vi $\Delta ABC$ lớn hơn $4R$.
\end{baitoan}

\begin{baitoan}[\cite{Binh_Toan_9_tap_1}, 53., p. 96]
	Cho $\Delta ABC$ nội tiếp đường tròn $(O)$. Gọi D, E, F lần lượt là trung điểm của BC, CA, AB. Kẻ 3 đường thẳng $DD',EE',FF'$ sao cho $DD'\parallel OA,EE'\parallel OB,FF'\parallel OC$. Chứng minh 3 đường thẳng $DD',EE',FF'$ đồng quy.
\end{baitoan}

\begin{baitoan}[\cite{Binh_Toan_9_tap_1}, 54., p. 96]
	Cho 3 điểm A, B, C bất kỳ \& đường tròn $(O;1)$. Chứng minh tồn tại 1 điểm M nằm trên đường tròn $(O)$ sao cho $MA + MB + MC\ge3$.
\end{baitoan}

%------------------------------------------------------------------------------%

\section{Đường Kính \& Dây của Đường Tròn. Liên Hệ Giữa Dây \& Khoảng Cách Từ Tâm Đến Dây}

\begin{baitoan}[\cite{Binh_Toan_9_tap_1}, Ví dụ 9, p. 96]
	Cho $\Delta ABC$ nhọn nội tiếp đường tròn $(O)$. M là điểm bất kỳ thuộc cung BC không chứa A. Gọi D, E theo thứ tự là các điểm đối xứng với M qua AB, AC. Tìm vị trí của M để DE có độ dài lớn nhất.
\end{baitoan}

\begin{baitoan}[\cite{Binh_Toan_9_tap_1}, Ví dụ 10, p. 97]
	Cho $(O)$ bán kính $OA = 11$ {\rm cm}. Điểm M thuộc bán kính OA \& cách O {\rm7 cm}. Qua M kẻ dây CD có độ dài {\rm18 cm}. Tính MC, MD với $MC < MD$.
\end{baitoan}

\begin{baitoan}[\cite{Binh_Toan_9_tap_1}, Ví dụ 11, p. 97]
	Cho $(O)$ bán kính {\rm15 cm}, điểm M cách O {\rm9 cm}. (a) Dựng dây AB đi qua M \& có độ dài {\rm26 cm}. (b) Có bao nhiêu dây đi qua M \& có độ dài là 1 số nguyên {\rm cm}?
\end{baitoan}

\begin{baitoan}[\cite{Binh_Toan_9_tap_1}, 55., p. 98]
	Tứ giác ABCD có $\widehat{A} = \widehat{C} = 90^\circ$. (a) Chứng minh $AC\le BD$. (b) Trong trường hợp nào thì $AC = BD$?
\end{baitoan}

\begin{baitoan}[\cite{Binh_Toan_9_tap_1}, 56., p. 98]
	Cho $(O)$ đường kính AB, 2 dây AC, AD. Gọi E là điểm bất kỳ trên đường tròn, H, K lần lượt là hình chiếu của E trên AC, AD. Chứng minh $HK\le AB$.
\end{baitoan}

\begin{baitoan}[\cite{Binh_Toan_9_tap_1}, 57., p. 98]
	Cho $(O)$, dây $AB = 24$ {\rm cm}, dây $AC = 20$ {\rm cm} ($\widehat{BAC} < 90^\circ$ \& điểm O nằm trong $\widehat{BAC}$). Gọi M là trung điểm của AC. Khoảng cách từ M đến AB bằng {\rm8 cm}. (a) Chứng minh $\Delta ABC$ cân tại C. (b) Tính bán kính đường tròn.
\end{baitoan}

\begin{baitoan}[\cite{Binh_Toan_9_tap_1}, 58., p. 98]
	Cho $(O)$ bán kính {\rm5 cm}, 2 dây AB \& CD song song với nhau có độ dài theo thứ tự bằng {\rm8 cm} \& {\rm6 cm}. Tính khoảng cách giữa 2 dây.
\end{baitoan}

\begin{baitoan}[\cite{Binh_Toan_9_tap_1}, 59., p. 98]
	Cho $(O)$, đường kính $AB = 13$ {\rm cm}. Dây CD có độ dài {\rm 12 cm} vuông góc với AB tại H. (a) Tính AH, BH. (b) Gọi M, N lần lượt là hình chiếu của H trên AC, BC. Tính diện tích tứ giác CMHN.
\end{baitoan}

\begin{baitoan}[\cite{Binh_Toan_9_tap_1}, 60., p. 99]
	Cho nửa đường tròn tâm O đường kính AB, dây CD. Gọi H, K lần lượt là chân 2 đường vuông góc kẻ từ A, B đến CD. (a) Chứng minh $CH = DK$. (b) Chứng minh $S_{AHKB} = S_{ABC} + S_{ABD}$. (c) Tính diện tích lớn nhất của tứ giác AHKB, biết $AB = 30$ {\rm cm}, $CD = 18$ {\rm cm}.
\end{baitoan}

\begin{baitoan}[\cite{Binh_Toan_9_tap_1}, 61., p. 99]
	Cho $\Delta ABC$, 3 đường cao $AD,BE,CF$. Đường tròn đi qua D, E, F cắt BC, CA, AB lần lượt tại M, N, P. Chứng minh 3 đường thẳng kẻ từ M vuông góc với BC, kẻ từ N vuông góc với AC, kẻ từ P vuông góc với AB đồng quy.
\end{baitoan}

\begin{baitoan}[\cite{Binh_Toan_9_tap_1}, 62., p. 99]
	$\Delta ABC$ cân tại A nội tiếp $(O)$. Gọi D là trung điểm của AB, E lfa trọng tâm của $\Delta ACD$. Chứng minh $OE\bot CD$.
\end{baitoan}

%------------------------------------------------------------------------------%

\section{Vị Trí Tương Đối của Đường Thẳng \& Đường Tròn. Dấu Hiệu Nhận Biết Tiếp Tuyến của Đường Tròn}

%------------------------------------------------------------------------------%

\section{Miscellaneous}

%------------------------------------------------------------------------------%

\printbibliography[heading=bibintoc]

\end{document}