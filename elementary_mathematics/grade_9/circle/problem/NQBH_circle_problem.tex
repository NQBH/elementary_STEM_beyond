\documentclass{article}
\usepackage[backend=biber,natbib=true,style=alphabetic,maxbibnames=50]{biblatex}
\addbibresource{/home/nqbh/reference/bib.bib}
\usepackage[utf8]{vietnam}
\usepackage{tocloft}
\renewcommand{\cftsecleader}{\cftdotfill{\cftdotsep}}
\usepackage[colorlinks=true,linkcolor=blue,urlcolor=red,citecolor=magenta]{hyperref}
\usepackage{amsmath,amssymb,amsthm,float,graphicx,mathtools,tikz}
\usetikzlibrary{angles,calc,intersections,matrix,patterns,quotes,shadings}
\makeatletter
\DeclareFontFamily{U}{tipa}{}
\DeclareFontShape{U}{tipa}{m}{n}{<->tipa10}{}
\newcommand{\arc@char}{{\usefont{U}{tipa}{m}{n}\symbol{62}}}%

\newcommand{\arc}[1]{\mathpalette\arc@arc{#1}}

\newcommand{\arc@arc}[2]{%
	\sbox0{$\m@th#1#2$}%
	\vbox{
		\hbox{\resizebox{\wd0}{\height}{\arc@char}}
		\nointerlineskip
		\box0
	}%
}
\makeatother
\allowdisplaybreaks
\newtheorem{assumption}{Assumption}
\newtheorem{baitoan}{}
\newtheorem{cauhoi}{Câu hỏi}
\newtheorem{conjecture}{Conjecture}
\newtheorem{corollary}{Corollary}
\newtheorem{dangtoan}{Dạng toán}
\newtheorem{definition}{Definition}
\newtheorem{dinhly}{Định lý}
\newtheorem{dinhnghia}{Định nghĩa}
\newtheorem{example}{Example}
\newtheorem{ghichu}{Ghi chú}
\newtheorem{hequa}{Hệ quả}
\newtheorem{hypothesis}{Hypothesis}
\newtheorem{lemma}{Lemma}
\newtheorem{luuy}{Lưu ý}
\newtheorem{nhanxet}{Nhận xét}
\newtheorem{notation}{Notation}
\newtheorem{note}{Note}
\newtheorem{principle}{Principle}
\newtheorem{problem}{Problem}
\newtheorem{proposition}{Proposition}
\newtheorem{question}{Question}
\newtheorem{remark}{Remark}
\newtheorem{theorem}{Theorem}
\newtheorem{vidu}{Ví dụ}
\usepackage[left=1cm,right=1cm,top=5mm,bottom=5mm,footskip=4mm]{geometry}
\def\labelitemii{$\circ$}
\DeclareRobustCommand{\divby}{%
	\mathrel{\vbox{\baselineskip.65ex\lineskiplimit0pt\hbox{.}\hbox{.}\hbox{.}}}%
}

\title{Problem: Circle -- Bài Tập: Đường Tròn}
\author{Nguyễn Quản Bá Hồng\footnote{Independent Researcher, Ben Tre City, Vietnam\\e-mail: \texttt{nguyenquanbahong@gmail.com}; website: \url{https://nqbh.github.io}.}}
\date{\today}

\begin{document}
\maketitle
\begin{abstract}
	Last updated version: \href{https://github.com/NQBH/elementary_STEM_beyond/blob/main/elementary_mathematics/grade_9/circle/problem/NQBH_circle_problem.pdf}{GitHub{\tt/}NQBH{\tt/}elementary STEM \& beyond{\tt/}elementary mathematics{\tt/}grade 9{\tt/}circle{\tt/}problem: set $\mathbb{Q}$ of circles [pdf]}.\footnote{\textsc{url}: \url{https://github.com/NQBH/elementary_STEM_beyond/blob/main/elementary_mathematics/grade_9/circle/problem/NQBH_circle_problem.pdf}.} [\href{https://github.com/NQBH/elementary_STEM_beyond/blob/main/elementary_mathematics/grade_9/circle/problem/NQBH_circle_problem.tex}{\TeX}]\footnote{\textsc{url}: \url{https://github.com/NQBH/elementary_STEM_beyond/blob/main/elementary_mathematics/grade_9/rational/problem/NQBH_circle_problem.tex}.}. 
\end{abstract}
\tableofcontents

%------------------------------------------------------------------------------%

\section{Sự Xác Định Đường Tròn. Tính Chất Đối Xứng của Đường Tròn}

\begin{baitoan}[\cite{Binh_boi_duong_Toan_9_tap_1}, p. 99]
	Tại sao các nan hoa của bánh xe đạp dài bằng nhau?
\end{baitoan}

\begin{baitoan}[\cite{Binh_boi_duong_Toan_9_tap_1}, H1, p. 101]
	Có bao nhiêu đường tròn bán kính $R$ đi qua 1 điểm cho trước? Tâm các đường tròn đó nằm ở đâu?
\end{baitoan}

\begin{baitoan}[\cite{Binh_boi_duong_Toan_9_tap_1}, H2, p. 101]
	Qua 3 điểm bất kỳ có luôn vẽ được 1 đường tròn?
\end{baitoan}

\begin{baitoan}[\cite{Binh_boi_duong_Toan_9_tap_1}, H3, p. 101]
	Vẽ đường tròn nhận đoạn thẳng AB cho trước làm đường kính.
\end{baitoan}

\begin{baitoan}[\cite{Binh_boi_duong_Toan_9_tap_1}, H4, p. 101]
	Tính đường kính của các đường tròn $(O;2R),(O;aR)$, $\forall a\in\mathbb{R}$, $a > 0$.
\end{baitoan}

\begin{baitoan}[\cite{Binh_boi_duong_Toan_9_tap_1}, H5, p. 101]
	{\rm Đ{\tt/}S?} (a) Dây vuông góc với đường kính thì bị đường kính chia làm đôi. (b) Dây vuông góc với đường kính thì chia đôi đường kính. (c) Đường kính đi qua trung điểm 1 dây thì vuông góc với dây ấy. (d) Đường trung trực của 1 dây là trục đối xứng của đường tròn.
\end{baitoan}

\begin{baitoan}[\cite{Binh_boi_duong_Toan_9_tap_1}, VD1, p. 101]
	Chứng minh: (a) Tâm của đường tròn ngoại tiếp tam giác vuông là trung điểm cạnh huyền. (b) Nếu 1 tam giác có 1 cạnh là đường kính của đường tròn ngoại tiếp thì tam giác đó là tam giác vuông (đường kính là cạnh huyền). (c) Các đỉnh góc vuông của các tam giác vuông có chung cạnh huyền cùng thuộc 1 đường tròn đường kính là cạnh huyền chung đó. (d) Mọi hình chữ nhật đều nội tiếp được trong đường tròn.
\end{baitoan}

\begin{baitoan}[\cite{Binh_boi_duong_Toan_9_tap_1}, VD2, p. 102]
	Khi nào thì tâm của đường tròn ngoại tiếp tam giác nằm: (a) trong tam giác? (b) ngoài tam giác?
\end{baitoan}

\begin{baitoan}[\cite{Binh_boi_duong_Toan_9_tap_1}, VD3, p. 102]
	Cho $\Delta ABC$ có $AB = 13$ {\rm cm}, $BC = 5$ {\rm cm}, $CA = 12$ {\rm cm}. Xác định tâm \& tính bán kính đường tròn ngoại tiếp $\Delta ABC$.
\end{baitoan}

\begin{baitoan}[\cite{Binh_boi_duong_Toan_9_tap_1}, VD4, p. 103]
	Cho đường tròn đường kính AB, M là 1 điểm bất kỳ. Chứng minh M nằm trong đường tròn khi \& chỉ khi $\widehat{AMB} > 90^\circ$.
\end{baitoan}

\begin{baitoan}[\cite{Binh_boi_duong_Toan_9_tap_1}, VD5, p. 103]
	Cho đường tròn $(O,R)$ \& 2 điểm $A,B$ nằm trong đường tròn. Chứng minh tồn tại 1 đường tròn $(C)$ đi qua 2 điểm $A,B$ \& nằm hoàn toàn bên trong $(O)$.
\end{baitoan}

\begin{baitoan}[\cite{Binh_boi_duong_Toan_9_tap_1}, VD6, p. 103]
	Có 1 miếng bìa hình tròn bị khoét đi 1 lỗ thủng cũng hình tròn. Dùng kéo cắt (theo 1 đường thẳng) để chia đôi miếng bìa đó.
\end{baitoan}

\begin{baitoan}[\cite{Binh_boi_duong_Toan_9_tap_1}, VD7, p. 104]
	Cho đoạn thẳng AB, điểm M thuộc đoạn AB. Dựng 2 đường tròn đường kính AB \& đường kính BM. 1 đường thẳng $d$ vuông góc với AB tại N cắt đường tròn đường kính AB tại $E,F$, cắt đường tròn đường kính BM tại $P,Q$. Chứng minh: (a) $PE = QF$. (b) $\widehat{PMB} > \widehat{EAB}$.
\end{baitoan}

\begin{baitoan}[\cite{Binh_boi_duong_Toan_9_tap_1}, VD8, p. 104]
	Cho đường tròn $(O,R)$ \& điểm A nằm ngoài đường tròn. Dựng qua A cát tuyến cắt đường tròn tại $B,C$ sao cho B là trung điểm AC.
\end{baitoan}

\begin{baitoan}[\cite{Binh_boi_duong_Toan_9_tap_1}, VD9, p. 105]
	Cho đường tròn $(O,6{\rm cm})$, 2 dây $AB\parallel CD$. (a) Chứng minh $AC = BD,AD = BC$. (b) Tính khoảng cách từ O đến AC, biết khoảng cách tư O đến AB là {\rm2 cm}, khoảng cách từ O đến CD là {\rm4 cm}.
\end{baitoan}

\begin{baitoan}[\cite{Binh_boi_duong_Toan_9_tap_1}, 4.1., p. 106]
	Cho $\Delta ABC$ vuông tại A, đường trung tuyến AM, $AB = 6$ {\rm cm}, $AC = 8$ {\rm cm}. Trên tia AM lấy 3 điểm $D,E,F$ sao cho $AD = 9$ {\rm cm}, $AE = 11$ {\rm cm}, $AF = 10$ {\rm cm}. Xác định vị trí của mỗi điểm $D,E,F$ đối với đường tròn ngoại tiếp $\Delta ABC$.
\end{baitoan}

\begin{baitoan}[\cite{Binh_boi_duong_Toan_9_tap_1}, 4.2., p. 106]
	Cho $\Delta ABC$ vuông tại A, đường cao AH. Từ điểm M bất kỳ trên cạnh BC kẻ $MD\bot AB,ME\bot AC$. Chứng minh 5 điểm $A,D,M,H,E$ cùng nằm trên 1 đường tròn.
\end{baitoan}

\begin{baitoan}[\cite{Binh_boi_duong_Toan_9_tap_1}, 4.3., p. 106]
	Tứ giác ABCD có $\widehat{A} = \widehat{C} = 90^\circ$. So sánh $AC,BD$.
\end{baitoan}

\begin{baitoan}[\cite{Binh_boi_duong_Toan_9_tap_1}, 4.4., p. 106]
	Cho đường tròn đường kính AB, $C,D$ là 2 điểm khác nhau thuộc đường tròn, $C,D$ không trùng với $A,B$. 2 điểm $E,F$ thuộc đường tròn sao cho $CE\bot AB,DF\bot AB$. Chứng minh $CF,ED,AB$ đồng quy.
\end{baitoan}

\begin{baitoan}[\cite{Binh_boi_duong_Toan_9_tap_1}, 4.5., p. 106]
	Cho đường tròn $(O,R)$ \& dây $AB = 2a$, $a < R$. Từ O kẻ đường thẳng vuông góc với AB cắt đường tròn tại D. Tính độ dài AD theo $a,R$.
\end{baitoan}

\begin{baitoan}[\cite{Binh_boi_duong_Toan_9_tap_1}, 4.6., p. 106]
	Cho tứ giác ABCD có $\widehat{C} + \widehat{D} = 90^\circ$. Gọi $M,N,P,Q$ lần lượt là trung điểm của $AB,BD,DC,CA$. Chứng minh 4 điểm $M,N,P,Q$ cùng thuộc 1 đường tròn.
\end{baitoan}

\begin{baitoan}[\cite{Binh_boi_duong_Toan_9_tap_1}, 4.7., p. 106]
	Cho $\Delta ABC$ cân tại A, nội tiếp đường tròn $(O)$. Đường cao AH cắt $(O)$ ở D. Biết $BC = 24,AC = 20$. Tính chiều cao AH \& bán kính $(O)$.
\end{baitoan}

\begin{baitoan}[\cite{Binh_boi_duong_Toan_9_tap_1}, 4.8., p. 106]
	Cho đường tròn $(O,R)$ \& dây AB. Kéo dài AB về phía B lấy điểm C sao cho $BC = R$. Chứng minh $\widehat{AOC} = 180^\circ - 3\widehat{ACO}$.
\end{baitoan}

\begin{baitoan}[\cite{Binh_boi_duong_Toan_9_tap_1}, 4.9., p. 106]
	Cho đường tròn $(O,R)$ \& điểm A nằm ngoài đường tròn. Xác định vị trí của điểm M trên đường tròn sao cho đoạn MA là ngắn nhất, dài nhất.
\end{baitoan}

\begin{baitoan}[\cite{Binh_boi_duong_Toan_9_tap_1}, 4.10., p. 107]
	Cho đường tròn $(O,R)$ \& điểm P nằm bên trong nó. 2 dây $AB,CD$ thay đổi luôn đi qua P \& vuông góc với nhau. Chứng minh $AB^2 + CD^2$ là đại lượng không đổi.
\end{baitoan}

\begin{baitoan}[\cite{Binh_boi_duong_Toan_9_tap_1}, 4.11., p. 107]
	Cho đường tròn $(O,R)$, đường kính AB, E là điểm nằm trong đường tròn, AE cắt đường tròn tại C, BE cắt đường tròn tại D. Chứng minh $AE\cdot AC + BE\cdot BD = 4R^2$.
\end{baitoan}

\begin{baitoan}[\cite{Binh_boi_duong_Toan_9_tap_1}, 4.12., p. 107]
	Cho tứ giác ABCD. Chứng minh 4 hình tròn có đường kính $AB,BC,CD,DA$ phủ kín miền tứ giác ABCD.
\end{baitoan}

\begin{baitoan}[\cite{Binh_boi_duong_Toan_9_tap_1}, 4.13., p. 107]
	Cho nửa đường tròn đường kính AB \& điểm M nằm trong nửa đường tròn. Chỉ bằng thước kẻ, dựng qua M đường thẳng vuông góc với AB.
\end{baitoan}

\begin{baitoan}[\cite{Tuyen_Toan_9_old}, Thí dụ 5, pp. 113--114]
	Trên đường tròn $(O,R)$ đường kính $AB$ lấy 1 điểm $C$. Trên tia $AC$ lấy điểm $M$ sao cho $C$ là trung điểm $AM$. (a) Xác định vị trí của điểm $C$ để $AM$ có độ dài lớn nhất. (b) Xác định vị trí của điểm $C$ để $AM = 2R\sqrt{3}$. (c) Chứng minh khi $C$ di động trên đường tròn $(O)$ thì điểm $M$ di động trên 1 đường tròn cố định.
\end{baitoan}

\begin{baitoan}[\cite{Tuyen_Toan_9_old}, 36., p. 114]
	Cho $\Delta ABC$ cân tại $A$, đường cao $AH = BC = a$. Tính bán kính của đường tròn ngoại tiếp $\Delta ABC$.
\end{baitoan}

\begin{baitoan}[\cite{Tuyen_Toan_9_old}, 37., p. 114]
	Cho $\Delta ABC$. Gọi $D,E,F$ lần lượt là trung điểm $BC,CA,AB$. Chứng minh: các đường tròn $(AFE),(BFD),(CDE)$ bằng nhau \& cùng đi qua 1 điểm. Xác định điểm chung đó.
\end{baitoan}

\begin{baitoan}[\cite{Tuyen_Toan_9_old}, 38., p. 114]
	Cho hình thoi $ABCD$ cạnh $1$, 2 đường chéo cắt nhau tại $O$. Gọi $R_1$ \& $R_2$ lần lượt là bán kính các đường tròn ngoại tiếp các $\Delta ABC,\Delta ABD$. Chứng minh: $\dfrac{1}{R_1^2} + \dfrac{1}{R_2^2} = 4$.
\end{baitoan}

\begin{baitoan}[\cite{Tuyen_Toan_9_old}, 39., p. 115]
	Cho hình bình hành $ABCD$, cạnh $AB$ cố định, đường chéo $AC = 2$ \emph{cm}. Chứng minh điểm $D$ di động trên 1 đường tròn cố định.
\end{baitoan}

\begin{baitoan}[\cite{Tuyen_Toan_9_old}, 40., p. 115]
	Cho đường tròn $(O,R)$ \& 1 dây $BC$ cố định. Trên đường tròn lấy 1 điểm $A$ ($A\not\equiv B$, $A\not\equiv C$). Gọi $G$ là trọng tâm của $\Delta ABC$. Chứng minh khi $A$ di động trên đường tròn $(O)$ thì điểm $G$ di động trên 1 đường tròn cố định.
\end{baitoan}

\begin{baitoan}[\cite{Tuyen_Toan_9_old}, 41., p. 115]
	Trong mặt phẳng cho $2n + 1$ điểm, $n\in\mathbb{N}$, sao cho $3$ điểm bất kỳ nào cũng tồn tại $2$ điểm có khoảng cách nhỏ hơn $1$. Chứng minh: trong các điểm này có ít nhất $n + 1$ điểm nằm trong 1 đường tròn có bán kính bằng $1$.
\end{baitoan}

\begin{baitoan}[\cite{Tuyen_Toan_9_old}, 42., p. 115]
	Cho hình bình hành $ABCD$, 2 đường chéo cắt nhau tại $O$. Vẽ đường tròn tâm $O$ cắt các đường thẳng $AB,BC,CD,DA$ lần lượt tại $M,N,P,Q$. Xác định dạng của tứ giác $MNPQ$.
\end{baitoan}

\begin{baitoan}[\cite{Tuyen_Toan_9_old}, 43., p. 115]
	2 người chơi 1 trò chơi như sau: Mỗi người lần lượt đặt lên 1 chiếc bàn hình tròn 1 cái cốc. Ai là người cuối cùng đặt được cốc lên bàn thì người đó thắng cuộc. Muốn chắc thắng thì phải chơi theo ``chiến thuật'' nào? (các chiếc cốc đều như nhau).
\end{baitoan}

\begin{baitoan}[\cite{Binh_Toan_9_tap_1}, Ví dụ 8, p. 95]
	Cho hình thang cân ABCD. Chứng minh tồn tại 1 đường tròn đi qua cả 4 đỉnh của hình thang.
\end{baitoan}

\begin{baitoan}[\cite{Binh_Toan_9_tap_1}, 50., p. 95]
	Cho $\Delta ABC$ cân tại A nội tiếp đường tròn $(O)$, $AC = 40$ {\rm cm}, $BC = 48$ {\rm cm}. Tính khoảng cách từ O đến BC. 
\end{baitoan}

\begin{baitoan}[\cite{Binh_Toan_9_tap_1}, 51., p. 96]
	Cho $\Delta ABC$ cân tại A nội tiếp đường tròn $(O)$, cạnh bên bằng $b$, đường cao $AH = h$. Tính bán kính của đường tròn $(O)$.
\end{baitoan}

\begin{baitoan}[\cite{Binh_Toan_9_tap_1}, 52., p. 96]
	Cho $\Delta ABC$ nhọn nội tiếp đường tròn $(O,R)$. Gọi M là trung điểm BC. Giả sử O nằm trong $\Delta AMC$ hoặc O nằm giữa A \& M. Gọi I là trung điểm AC. Chứng minh: (a) Chu vi $\Delta IMC$ lớn hơn $2R$. (b) Chu vi $\Delta ABC$ lớn hơn $4R$.
\end{baitoan}

\begin{baitoan}[\cite{Binh_Toan_9_tap_1}, 53., p. 96]
	Cho $\Delta ABC$ nội tiếp đường tròn $(O)$. Gọi D, E, F lần lượt là trung điểm BC, CA, AB. Kẻ 3 đường thẳng $DD',EE',FF'$ sao cho $DD'\parallel OA,EE'\parallel OB,FF'\parallel OC$. Chứng minh 3 đường thẳng $DD',EE',FF'$ đồng quy.
\end{baitoan}

\begin{baitoan}[\cite{Binh_Toan_9_tap_1}, 54., p. 96]
	Cho 3 điểm A, B, C bất kỳ \& đường tròn $(O;1)$. Chứng minh tồn tại 1 điểm M nằm trên đường tròn $(O)$ sao cho $MA + MB + MC\ge3$.
\end{baitoan}

\begin{baitoan}[\cite{TLCT_THCS_Toan_9_hinh_hoc}, VD1, p. 20]
	Cho đường tròn $(O)$, đường kính AB, 2 dây $AC,BD$. Chứng minh $AC\parallel BD\Leftrightarrow CD$ là đường kính.
\end{baitoan}

\begin{baitoan}[\cite{TLCT_THCS_Toan_9_hinh_hoc}, VD2, p. 20]
	Cho đường tròn $(O)$, 2 dây $AB,CD$ song song với nhau. Gọi $E,F$ là trung điểm $AB,CD$. Chứng minh $E,F,O$ thẳng hàng.
\end{baitoan}

\begin{baitoan}[\cite{TLCT_THCS_Toan_9_hinh_hoc}, VD3, p. 20]
	Dựng 1 đường tròn nhận đoạn thẳng AB cho trước làm dây cung có bán kính $r$ cho trước.
\end{baitoan}

\begin{baitoan}[\cite{TLCT_THCS_Toan_9_hinh_hoc}, VD4, p. 21]
	Cho đường tròn $(O,R)$ \& dây AB. Kéo dài AB về phía B lấy điểm C sao cho $BC = R$. Chứng minh $\widehat{AOC} = 180^\circ - 3\widehat{ACO}$.
\end{baitoan}

\begin{baitoan}[\cite{TLCT_THCS_Toan_9_hinh_hoc}, VD5, p. 21]
	Cho $\Delta ABC$. Từ trung điểm 3 cạnh kẻ các đường vuông góc với 2 cạnh kia tạo thành 1 lục giác. Chứng minh diện tích $\Delta ABC$ gấp 2 lần diện tích lục giác.
\end{baitoan}

\begin{baitoan}[\cite{TLCT_THCS_Toan_9_hinh_hoc}, VD6, p. 21]
	Cho đường tròn $(O)$, 2 dây $AB,CD$ kéo dài cắt nhau tại điểm M ở ngoài đường tròn. Gọi $H,E$ là trung điểm $AB,CD$. Chứng minh $AB < CD\Leftrightarrow MH < ME$.
\end{baitoan}

\begin{baitoan}[\cite{TLCT_THCS_Toan_9_hinh_hoc}, VD7, p. 22]
	Cho đường tròn $(O)$ \& điểm A nằm trong đường tròn, $A\ne O$. Tìm trên đường tròn điểm M sao cho $\widehat{OMA}$ lớn nhất.
\end{baitoan}

\begin{baitoan}[\cite{TLCT_THCS_Toan_9_hinh_hoc}, VD8, p. 22]
	Cho đường tròn $(O)$, $A,B,C$ là 3 điểm trên đường tròn sao cho $AB = AC$. Gọi I là trung điểm AC, G là trọng tâm của $\Delta ABI$. Chứng minh $OG\bot BI$.
\end{baitoan}

\begin{baitoan}[\cite{TLCT_THCS_Toan_9_hinh_hoc}, VD9, p. 23]
	Dựng $\Delta ABC$. Biết $\widehat{A} = \alpha < 90^\circ$, đường cao $BH = h$ \& trung tuyến $CM = m$.
\end{baitoan}

\begin{baitoan}[\cite{TLCT_THCS_Toan_9_hinh_hoc}, VD10, p. 23]
	Cho $\Delta ABC$ nhọn, nội tiếp đường tròn $(O,r)$, $AB = r\sqrt{3}$, $AC = r\sqrt{2}$. Giải $\Delta ABC$.
\end{baitoan}

\begin{baitoan}[\cite{TLCT_THCS_Toan_9_hinh_hoc}, VD11, p. 23]
	Cho đoạn thẳng BC cố định, I là trung điểm BC, điểm A trên mặt phẳng sao cho $AB = BC$. Gọi H là trung điểm AC, đường thẳng AI cắt đường thẳng BH tại M. Chứng minh M nằm trên 1 đường tròn cố định khi A thay đổi.
\end{baitoan}

\begin{baitoan}[\cite{TLCT_THCS_Toan_9_hinh_hoc}, VD12, p. 24]
	Cho hình chữ nhật ABCD, kẻ $BH\bot AC$. Trên cạnh $AC,CD$ lấy 2 điểm $M,N$ sao cho $\dfrac{AM}{AH} = \dfrac{DN}{CD}$. Chứng minh 4 điểm $B,C,M,N$ nằm trên 1 đường tròn.
\end{baitoan}

\begin{baitoan}[\cite{TLCT_THCS_Toan_9_hinh_hoc}, VD13, p. 24]
	Cho đường tròn $(O,R)$, dây $AB = 2a$, $a < R$. Từ O kẻ đường thẳng vuông góc với AB cắt đường tròn tại D. Tính độ dài AD theo $a,R$.
\end{baitoan}

\begin{baitoan}[\cite{TLCT_THCS_Toan_9_hinh_hoc}, VD14, p. 25]
	Cho đường tròn $(O,R)$, đường kính AB, điểm E nằm trong đường tròn, AE cắt đường tròn tại C, BE cắt đường tròn tại D. Chứng minh $AE\cot AC + BE\cdot BD$ không phụ thuộc vào vị trí của điểm E.
\end{baitoan}

\begin{baitoan}[\cite{TLCT_THCS_Toan_9_hinh_hoc}, VD15, p. 25]
	Cho tứ giác lồi ABCD. Chứng minh 4 hình tròn có đường kính $AB,BC,CD,DA$ phủ kín miền tứ giác ABCD.
\end{baitoan}

\begin{baitoan}[\cite{TLCT_THCS_Toan_9_hinh_hoc}, 4.1., p. 26]
	Tính độ dài cạnh của tam giác đều, bát giác đều, $n$-giác đều nội tiếp đường tròn $(O,R)$.
\end{baitoan}

\begin{baitoan}[\cite{TLCT_THCS_Toan_9_hinh_hoc}, 4.2., p. 26]
	Cho đường tròn $(O)$, điểm P ở trong đường tròn. Xác định dây lớn nhất \& dây ngắn nhất đi qua điểm P.
\end{baitoan}

\begin{baitoan}[\cite{TLCT_THCS_Toan_9_hinh_hoc}, 4.3., p. 26]
	Cho đường tròn $(O)$, 2 bán kính $OA,OB$ vuông góc với nhau. Kẻ tia phân giác của $\widehat{AOB}$, cắt đường tròn ở D, M là điểm chuyển động trên cung nhỏ AB, từ M kẻ $MH\bot OB$ cắt OD tại K. Chứng minh $MH^2 + KH^2$ có giá trị không phụ thuộc vào vị trí điểm M.
\end{baitoan}

\begin{baitoan}[\cite{TLCT_THCS_Toan_9_hinh_hoc}, 4.4., p. 26]
	Chứng minh bao giờ cũng chia được 1 tam giác bất kỳ thành 7 tam giác cân, trong đó có 3 tam giác bằng nhau.
\end{baitoan}

\begin{baitoan}[\cite{TLCT_THCS_Toan_9_hinh_hoc}, 4.5., p. 26]
	Cho đường tròn $(O)$, 1 dây cung EF có khoảng cách từ tâm O đến dây là $d$. Dựng 2 hình vuông nội tiếp trong mỗi phần đó, sao cho mỗi hình vuông có 2 đỉnh nằm trên đường tròn, 2 đỉnh còn lại nằm trên dây EF. Tính hiệu của 2 cạnh hình vuông đó theo $d$.
\end{baitoan}

\begin{baitoan}[\cite{TLCT_THCS_Toan_9_hinh_hoc}, 4.6., p. 26]
	Cho 2 đường tròn đồng tâm. Dựng 1 dây cắt 2 đường tròn theo thứ tự tại $A,B,C,D$ sao cho $AB = BC = CD$.
\end{baitoan}

\begin{baitoan}[\cite{TLCT_THCS_Toan_9_hinh_hoc}, 4.7., p. 26]
	Cho $\Delta ABC$ nội tiếp đường tròn $(O,R)$, $AB = R\sqrt{2 - \sqrt{3}}$, $AC = R\sqrt{2 + \sqrt{3}}$. Giải $\Delta ABC$.
\end{baitoan}

\begin{baitoan}[\cite{TLCT_THCS_Toan_9_hinh_hoc}, 4.8., p. 26]
	Cho hình thoi ABCD. Gọi $R_1$ là bán kính đường tròn ngoại tiếp $\Delta ABC$, $R_2$ là bán kính đường tròn ngoại tiếp $\Delta ABD$. Tính cạnh của hình thoi ABCD theo $R_1,R_2$.
\end{baitoan}

\begin{baitoan}[\cite{TLCT_THCS_Toan_9_hinh_hoc}, 4.9., p. 26]
	Mỗi điểm trên mặt phẳng được tô bởi 1 trong 3 màu xanh, đỏ, vàng. Chứng minh tồn tại ít nhất 2 điểm được tô cùng 1 màu mà khoảng cách giữa 2 điểm đó bằng $1$.
\end{baitoan}

\begin{baitoan}[\cite{TLCT_THCS_Toan_9_hinh_hoc}, 4.10., p. 26]
	Cho đường tròn $(O,R)$ \& dây AB cố định. Từ điểm C thay đổi trên đường tròn dựng hình bình hành CABD. Chứng minh giao điểm 2 đường chéo của hình bình hành CABD nằm trên 1 đường tròn cố định.
\end{baitoan}

%------------------------------------------------------------------------------%

\section{Đường Kính \& Dây của Đường Tròn. Liên Hệ Giữa Dây \& Khoảng Cách Từ Tâm Đến Dây}

\begin{baitoan}[\cite{Binh_Toan_9_tap_1}, Ví dụ 9, p. 96]
	Cho $\Delta ABC$ nhọn nội tiếp đường tròn $(O)$. Điểm M bất kỳ thuộc cung BC không chứa A. Gọi D, E theo thứ tự là các điểm đối xứng với M qua AB, AC. Tìm vị trí của M để DE có độ dài lớn nhất.
\end{baitoan}

\begin{baitoan}[\cite{Binh_Toan_9_tap_1}, Ví dụ 10, p. 97]
	Cho $(O)$ bán kính $OA = 11$ {\rm cm}. Điểm M thuộc bán kính OA \& cách O {\rm7 cm}. Qua M kẻ dây CD có độ dài {\rm18 cm}. Tính MC, MD với $MC < MD$.
\end{baitoan}

\begin{baitoan}[\cite{Binh_Toan_9_tap_1}, Ví dụ 11, p. 97]
	Cho $(O)$ bán kính {\rm15 cm}, điểm M cách O {\rm9 cm}. (a) Dựng dây AB đi qua M \& có độ dài {\rm26 cm}. (b) Có bao nhiêu dây đi qua M \& có độ dài là 1 số nguyên {\rm cm}?
\end{baitoan}

\begin{baitoan}[\cite{Binh_Toan_9_tap_1}, 55., p. 98]
	Tứ giác ABCD có $\widehat{A} = \widehat{C} = 90^\circ$. (a) Chứng minh $AC\le BD$. (b) Trong trường hợp nào thì $AC = BD$?
\end{baitoan}

\begin{baitoan}[\cite{Binh_Toan_9_tap_1}, 56., p. 98]
	Cho $(O)$ đường kính AB, 2 dây AC, AD. Điểm E bất kỳ trên đường tròn, H, K lần lượt là hình chiếu của E trên AC, AD. Chứng minh $HK\le AB$.
\end{baitoan}

\begin{baitoan}[\cite{Binh_Toan_9_tap_1}, 57., p. 98]
	Cho $(O)$, dây $AB = 24$ {\rm cm}, dây $AC = 20$ {\rm cm} ($\widehat{BAC} < 90^\circ$ \& điểm O nằm trong $\widehat{BAC}$). Gọi M là trung điểm AC. Khoảng cách từ M đến AB bằng {\rm8 cm}. (a) Chứng minh $\Delta ABC$ cân tại C. (b) Tính bán kính đường tròn.
\end{baitoan}

\begin{baitoan}[\cite{Binh_Toan_9_tap_1}, 58., p. 98]
	Cho $(O)$ bán kính {\rm5 cm}, 2 dây AB \& CD song song với nhau có độ dài theo thứ tự bằng {\rm8 cm} \& {\rm6 cm}. Tính khoảng cách giữa 2 dây.
\end{baitoan}

\begin{baitoan}[\cite{Binh_Toan_9_tap_1}, 59., p. 98]
	Cho $(O)$, đường kính $AB = 13$ {\rm cm}. Dây CD có độ dài {\rm 12 cm} vuông góc với AB tại H. (a) Tính AH, BH. (b) Gọi M, N lần lượt là hình chiếu của H trên AC, BC. Tính diện tích tứ giác CMHN.
\end{baitoan}

\begin{baitoan}[\cite{Binh_Toan_9_tap_1}, 60., p. 99]
	Cho nửa đường tròn tâm O đường kính AB, dây CD. Gọi H, K lần lượt là chân 2 đường vuông góc kẻ từ A, B đến CD. (a) Chứng minh $CH = DK$. (b) Chứng minh $S_{AHKB} = S_{ABC} + S_{ABD}$. (c) Tính diện tích lớn nhất của tứ giác AHKB, biết $AB = 30$ {\rm cm}, $CD = 18$ {\rm cm}.
\end{baitoan}

\begin{baitoan}[\cite{Binh_Toan_9_tap_1}, 61., p. 99]
	Cho $\Delta ABC$, 3 đường cao $AD,BE,CF$. Đường tròn đi qua D, E, F cắt BC, CA, AB lần lượt tại M, N, P. Chứng minh 3 đường thẳng kẻ từ M vuông góc với BC, kẻ từ N vuông góc với AC, kẻ từ P vuông góc với AB đồng quy.
\end{baitoan}

\begin{baitoan}[\cite{Binh_Toan_9_tap_1}, 62., p. 99]
	$\Delta ABC$ cân tại A nội tiếp $(O)$. Gọi D là trung điểm AB, E lfa trọng tâm của $\Delta ACD$. Chứng minh $OE\bot CD$.
\end{baitoan}

%------------------------------------------------------------------------------%

\section{Vị Trí Tương Đối của Đường Thẳng \& Đường Tròn. Dấu Hiệu Nhận Biết Tiếp Tuyến của Đường Tròn}

\begin{baitoan}[\cite{Binh_Toan_9_tap_1}, Ví dụ 12, p. 99]
	Cho $\Delta ABC$ vuông tại A, $AB < AC$, đường cao $AH$. Điểm E đối xứng với B qua H. Đường tròn có đường kính EC cắt AC ở K. Chứng minh HK là tiếp tuyến của đường tròn.
\end{baitoan}

\begin{baitoan}[\cite{Binh_Toan_9_tap_1}, Ví dụ 13, p. 100]
	Cho 1 hình vuông $8\times9$ gồm $64$ ô vuông nhỏ. Đặt 1 tấm bìa hình tròn có đường kính $8$ sao cho tâm O của hình tròn trùng với tâm của hình vuông. (a) Chứng minh hình tròn tiếp xúc với 4 cạnh của hình vuông. (b) Có bao nhiêu ô vuông nhỏ bị tấm bìa che lấp hoàn toàn? (c) Có bao nhiêu ô vuông nhỏ bị tấm bìa che lấp (cả che lấp 1 phần \& che lấp hoàn toàn)?
\end{baitoan}

\begin{baitoan}[\cite{Binh_Toan_9_tap_1}, 63., pp. 100--101]
	Cho nửa đường tròn tâm O đường kính AB, M là 1 điểm thuộc nửa đường tròn. Qua M vẽ tiếp tuyến với nửa đường tròn. Gọi D, C lần lượt là hình chiếu của A, B trên tiếp tuyến ấy. (a) Chứng minh M là trung điểm CD. (b) Chứng minh $AB = BC + AD$. (c) Giả sử $ \widehat{AOM}\ge\widehat{BOM}$, gọi E là giao điểm của AD với nửa đường tròn. Xác định dạng của tứ giác BCDE. (d) Xác định vị trí của điểm M trên nửa đường tròn sao cho tứ giác ABCD có diện tích lớn nhất. Tính diện tích đó theo bán kính $R$ của nửa đường tròn đã cho.
\end{baitoan}

\begin{baitoan}[\cite{Binh_Toan_9_tap_1}, 64., p. 101]
	Cho $\Delta ABC$ cân tại A, I là giao điểm của 3 đường phân giác. (a) Xác định vị trí tương đối của đường thẳng AC với đường tròn $(O)$ ngoại tiếp $\Delta BIC$. (b) Gọi H là trung điểm BC, IK là đường kính của đường tròn $(O)$. Chứng minh $\dfrac{AI}{AK} = \dfrac{HI}{HK}$.
\end{baitoan}

\begin{baitoan}[\cite{Binh_Toan_9_tap_1}, 65., p. 101]
	Cho nửa đường tròn tâm O đường kính AB, Ax là tiếp tuyến của nửa đường tròn (Ax \& nửa đường tròn nằm cùng phía đối với AB), C là 1 điểm thuộc nửa đường tròn, H là hình chiếu của C trên AB. Đường thẳng qua O \& vuông góc với AC cắt Ax tại M. Gọi I là giao điểm của MB \& CH. Chứng minh $IC = IH$.
\end{baitoan}

\begin{baitoan}[\cite{Binh_Toan_9_tap_1}, 66., p. 101]
	Cho hình thang vuông ABCD, $\widehat{A} = \widehat{D} = 90^\circ$, có $\widehat{BMC} = 90^\circ$ với M là trung điểm AD. Chứng minh: (a) AD là tiếp tuyến của đường tròn có đường kính BC. (b) BC là tiếp tuyến của đường tròn có đường kính AD.
\end{baitoan}

\begin{baitoan}[\cite{Binh_Toan_9_tap_1}, 67., p. 101]
	Cho nửa đường tròn tâm O đường kính AB, C là 1 điểm thuộc nửa đường tròn, H là hình chiếu của C trên AB. Qua trung điểm M của CH, kẻ đường vuông góc với OC, cắt nửa đường tròn tại D \& E. Chứng minh AB là tiếp tuyến của $(C;CD)$.
\end{baitoan}

\begin{baitoan}[\cite{Binh_Toan_9_tap_1}, 68., p. 101]
	Cho đường tròn tâm O đường kính AB. Gọi $d,d'$ lần lượt là 2 tiếp tuyến tại A, B của đường tròn, $C\in d$ bất kỳ. Đường vuông góc với OC tại O cắt $d'$ tại D. Chứng minh CD là tiếp tuyến của $(O)$.
\end{baitoan}

\begin{baitoan}[\cite{Binh_Toan_9_tap_1}, 69., p. 101]
	Cho nửa đường tròn tâm O đường kính AB, C là 1 điểm thuộc nửa đường tròn. Qua C kẻ tiếp tuyến $d$ với nửa đường tròn. Kẻ 2 tia Ax, By song song với nhau, cắt $d$ theo thứ tự tại D, E. Chứng minh AB là tiếp tuyến của đường tròn đường kính DE.
\end{baitoan}

\begin{baitoan}[\cite{Binh_Toan_9_tap_1}, 70., pp. 101--102]
	Cho đường tròn tâm O có đường kính $AB = 2R$. Gọi $d$ là tiếp tuyến của đường tròn, A là tiếp điểm. Điểm M bất kỳ thuộc $d$. Qua O kẻ đường thẳng vuông góc với BM, cắt $d$ tại N. (a) Chứng minh tích $AM\cdot AN$ không đổi khi điểm M chuyển động trên đường thẳng $d$. (b) Tìm {\rm GTNN} của MN.
\end{baitoan}

\begin{baitoan}[\cite{Binh_Toan_9_tap_1}, 71., p. 102]
	Cho $\Delta ABC$ cân tại A có $\widehat{A} = \alpha$, đường cao $AH = h$. Vẽ đường tròn tâm A bán kính $h$. 1 tiếp tuyến bất kỳ ($\ne BC$) của đường tròn $(A)$ cắt 2 tia AB, AC theo thứ tự tại $B',C'$. (a) Chứng minh $S_{ABC} = S_{AB'C'}$. (b) Trong các $\Delta ABC$ có $\widehat{A} = \alpha$ \& đường cao $AH = h$, tam giác nào có diện tích nhỏ nhất?
\end{baitoan}

\begin{baitoan}[\cite{TLCT_THCS_Toan_9_hinh_hoc}, 1, p. 28]
	Chứng minh: Nếu I là tâm đường tròn nội tiếp $\Delta ABC$ thì $\widehat{BIC} = 90^\circ + \dfrac{\widehat{A}}{2}$.
\end{baitoan}

\begin{baitoan}[\cite{TLCT_THCS_Toan_9_hinh_hoc}, 2, p. 28]
	Chứng minh: Nếu I nằm trong $\Delta ABC$ \& $\widehat{BIC} = 90^\circ + \dfrac{\widehat{A}}{2}$, $\widehat{AIC} = 90^\circ + \dfrac{\widehat{B}}{2}$ thì I là tâm đường tròn nội tiếp $\Delta ABC$.
\end{baitoan}

\begin{baitoan}[\cite{TLCT_THCS_Toan_9_hinh_hoc}, 3, p. 28]
	Chứng minh: Nếu J là tâm đường tròn bàng tiếp $\widehat{A}$ của $\Delta ABC$ thì $\widehat{BJC} = 90^\circ - \dfrac{\widehat{A}}{2}$.
\end{baitoan}

\begin{baitoan}[\cite{TLCT_THCS_Toan_9_hinh_hoc}, 4, p. 28]
	Cho $\Delta ABC$, đặt $BC = a,CA = b,AB = c$, $a + b + c = 2p$, $r$ là bán kính đường tròn nội tiếp, $S$ là diện tích $\Delta ABC$. Chứng minh: $r = (p - a)\tan\dfrac{A}{2} = (p - b)\tan\dfrac{B}{2} = (p - c)\tan\dfrac{C}{2}$, $S = pr$.
\end{baitoan}

\begin{baitoan}[\cite{TLCT_THCS_Toan_9_hinh_hoc}, 5, p. 28]
	Đường tròn nội tiếp $\Delta ABC$ tiếp xúc với $AB,AC$ tại $F,E$. Chứng minh: $AE = AF = \frac{1}{2}(AB + AC - BC)$.
\end{baitoan}

\begin{baitoan}[\cite{TLCT_THCS_Toan_9_hinh_hoc}, VD1, p. 29]
	Cho $\widehat{xOy} = 90^\circ$, đường tròn $(I)$ tiếp xúc với 2 cạnh $Ox,Oy$ tại $A,B$. 1 tiếp tuyến của đường tròn $(I)$ tại điểm E cắt $Ox,Oy$ tại $C,D$.
\end{baitoan}

\begin{baitoan}[\cite{TLCT_THCS_Toan_9_hinh_hoc}, VD2, p. 29]
	Cho $\widehat{xOy}$, 2 điểm $A,B$ lần lượt chuyển động trên Ox \& Oy sao cho chu vi $\Delta OAB$ không đổi. Chứng minh AB luôn tiếp xúc với đường tròn cố định.
\end{baitoan}

\begin{baitoan}[\cite{TLCT_THCS_Toan_9_hinh_hoc}, VD3, p. 29]
	Cho hình vuông $ABCD$, lấy điểm E trên cạnh BC \& điểm F trên cạnh CD sao cho $AB = 3BE = 2DF$. Chứng minh EF tiếp xúc với cung tròn tâm A, bán kính AB.
\end{baitoan}

\begin{baitoan}[\cite{TLCT_THCS_Toan_9_hinh_hoc}, VD4, p. 30]
	Cho đường tròn $(O,R)$, \& đường thẳng $a$ cắt đường tròn tại $A,B$. Gọi M là điểm trên $a$ \& nằm ngoài đường tròn, qua M kẻ 2 tiếp tuyển $MC,MD$. Chứng minh khi M thay đổi trên $a$, đường thẳng CD luôn đi qua 1 điểm cố định.
\end{baitoan}

\begin{baitoan}[\cite{TLCT_THCS_Toan_9_hinh_hoc}, VD5, p. 31]
	Cho $\Delta ABC$, gọi I là tâm đường tròn nội tiếp tam giác. Qua I dựng đường thẳng vuông góc với IA cắt $AB,AC$ tại $M,N$. Chứng minh: (a) $\dfrac{BM}{CN} = \dfrac{BI^2}{CI^2}$. (b) $BM\cdot AC + CN\cdot AB + AI^2 = AB\cdot AC$.
\end{baitoan}

\begin{baitoan}[\cite{TLCT_THCS_Toan_9_hinh_hoc}, VD6, p. 31]
	Cho $\Delta ABC$, $D,E,F$ theo thứ tự là 3 tiếp điểm của đường tròn nội tiếp $\Delta ABC$ với 3 cạnh $BC,CA,AB$, H là hình chiếu của D trên EF. Chứng minh DH là tia phân giác của $\widehat{BHC}$.
\end{baitoan}

\begin{baitoan}[\cite{TLCT_THCS_Toan_9_hinh_hoc}, VD7, p. 32]
	Gọi I là tâm đường tròn nội tiếp $\Delta ABC$. $D,E$ lần lượt là giao điểm của đường thẳng $BI,CI$ với cạnh $AC,AB$. Chứng minh $\Delta ABC$ vuông tại A $\Leftrightarrow BI\cdot CI = \frac{1}{2}BD\cdot CF$.
\end{baitoan}

\begin{baitoan}[\cite{TLCT_THCS_Toan_9_hinh_hoc}, VD8, p. 32]
	Cho đường tròn $(O,R)$ \& điểm M cách tâm O 1 khoảng bằng $3R$. Từ M kẻ 2 đường thẳng tiếp xúc với đường tròn $(O,R)$ tại $A,B$, gọi $I,E$ lần lượt là trung điểm của $MA,MB$. Tính khoảng cách từ O đến IE.
\end{baitoan}

\begin{baitoan}[\cite{TLCT_THCS_Toan_9_hinh_hoc}, VD9, p. 33]
	Cho $\Delta ABC$ cân tại A. Gọi O là trung điểm BC, dựng đường tròn $(O)$ tiếp xúc với $AB,AC$ tại $D,E$. M là điểm chuyển động trên cung nhỏ $\arc{DE}$, tiếp tuyến với đường tròn $(O)$ tại M cắt 2 cạnh $AB,AC$ lần lượt tại $P,Q$. Chứng minh: (a) $BC^2 = 4BP\cdot CQ$. Từ đó xác định vị trí của M để diện tích $\Delta APQ$ đạt {\rm GTLN}. (b) Nếu $BC^2 = 4BP\cdot CQ$ thì $PQ$ là tiếp tuyến.
\end{baitoan}

\begin{baitoan}[\cite{TLCT_THCS_Toan_9_hinh_hoc}, VD10, p. 34]
	Cho đường tròn $(O)$, điểm M ở ngoài đường tròn. Qua M kẻ 2 tiếp tuyến cắt đường tròn tại $A,B$, $MA > MB$, gọi CD là đường kính vuông góc với AB, đường thẳng $MC,MD$ cắt đường tròn tại $E,K$, giao điểm của $DE,CK$ là H, I là trung điểm MH. Chứng minh $IE,IK$ là 2 tiếp tuyến của đường tròn $(O)$.
\end{baitoan}

\begin{baitoan}[\cite{TLCT_THCS_Toan_9_hinh_hoc}, VD11, p. 34]
	Cho $\Delta ABC$, đường cao AH. Gọi $AD,AE$ là đường phân giác của 2 góc $\widehat{BAH},\widehat{CAH}$. Chứng minh tâm đường tròn nội tiếp $\Delta ABC$ trùng với tâm đường tròn ngoại tiếp $\Delta ADE$.
\end{baitoan}

\begin{baitoan}[\cite{TLCT_THCS_Toan_9_hinh_hoc}, VD12, p. 35]
	Cho $\Delta ABC$ vuông tại A. Gọi I là tâm đường tròn nội tiếp $\Delta ABC$, 3 tiếp điểm trên $BC,CA,AB$ lần lượt là $D,E,F$. Gọi M là trung điểm AC, đường thẳng MI cắt cạnh AB tại N, đường thẳng DF cắt đường cao AH của $\Delta ABC$ tại P. Chứng minh $\Delta ANP$ cân.
\end{baitoan}

\begin{baitoan}[\cite{TLCT_THCS_Toan_9_hinh_hoc}, VD13, p. 36]
	Tính $\widehat{A}$ của $\Delta ABC$, biết đỉnh B cách đều tâm 2 đường tròn bàng tiếp của $\widehat{A},\widehat{B}$ của $\Delta ABC$.
\end{baitoan}

\begin{baitoan}[\cite{TLCT_THCS_Toan_9_hinh_hoc}, VD14, p. 36]
	Cho $\Delta ABC$ có $AB = 2AC$ \& đường phân giác AD. Gọi $r,r_1,r_2$ lần lượt là bán kính đường tròn nội tiếp $\Delta ABC,\Delta ACD,\Delta ABD$. Chứng minh $AD = \dfrac{pr}{3}\left(\dfrac{1}{r_1} + \dfrac{2}{r_2}\right) - p$ với $p$ là nửa chu vi $\Delta ABC$.
\end{baitoan}

\begin{baitoan}[\cite{TLCT_THCS_Toan_9_hinh_hoc}, VD15, p. 37]
	Cho đường tròn $(O)$ \& điểm A cố định nằm ngoài đường tròn. Kẻ tiếp tuyến AB \& cát tuyến qua A cắt đường tròn tại $C,D$, $AC < AD$. Hỏi trọng tâm $\Delta BCD$ chạy trên đường nào khi cát tuyến ACD thay đổi?
\end{baitoan}

\begin{baitoan}[\cite{TLCT_THCS_Toan_9_hinh_hoc}, 5.1., p. 38]
	Cho nửa đường tròn bán kính $AB = 2R$. C là điểm trên nửa đường tròn, khoảng cách từ C đến AB là $h$. Tính bán kính đường tròn nội tiếp $\Delta ABC$ theo $R,h$.
\end{baitoan}

\begin{baitoan}[\cite{TLCT_THCS_Toan_9_hinh_hoc}, 5.2., p. 38]
	Cho $\Delta ABC$, D là điểm trên BC. Đường tròn nội tiếp $\Delta ABD$ tiếp xúc với cạnh BC tại E, đường tròn nội tiếp $\Delta ADC$ tiếp xúc với cạnh BC tại F, đồng thời 2 đường tròn này cùng tiếp xúc với đường thẳng $d\ne BC$, đường thẳng $d$ cắt AD tại I. Chứng minh $AI = \frac{1}{2}(AB + AC - BC)$.
\end{baitoan}

\begin{baitoan}[\cite{TLCT_THCS_Toan_9_hinh_hoc}, 5.3., p. 38]
	Cho $\Delta ABC$ vuông tại A, đường cao AH. Đường tròn đường kính BH cắt cạnh AB tại M, đường tròn đường kính HC cắt cạnh AC tại N. Chứng minh MN là tiếp tuyến chung của 2 đường tròn đường kính $BH,CH$.
\end{baitoan}

\begin{baitoan}[\cite{TLCT_THCS_Toan_9_hinh_hoc}, 5.4., p. 38]
	Cho $\Delta ABC$ cân tại A, đường cao AK. Gọi H là trực tâm $\Delta ABC$, đường tròn đường kính AH cắt 2 cạnh $AB,AC$ tại $D,E$. Chứng minh $KD,KE$ là 2 tiếp tuyến của đường tròn đường kính AH.
\end{baitoan}

\begin{baitoan}[\cite{TLCT_THCS_Toan_9_hinh_hoc}, 5.5., p. 38]
	Cho đường tròn $(O)$ \& điểm M ở ngoài đường tròn. Từ M kẻ tiếp tuyến $MA,MB$ với đường tròn, $A,B$ là 2 tiếp điểm, tia OM cắt đường tròn tại C, tiếp tuyến tại C cắt tiếp tuyến $MA,MB$ tại $P,Q$. Chứng minh diện tích $\Delta MPQ$ lớn hơn $\frac{1}{2}$ diện tích $\Delta ABC$.
\end{baitoan}

\begin{baitoan}[\cite{TLCT_THCS_Toan_9_hinh_hoc}, 5.6., p. 38]
	Trong tất cả các tam giác có cùng cạnh $a$, đường cao kẻ từ đỉnh đối diện với cạnh $a$ bằng $h$, xác định tam giác có bán kính đường tròn nội tiếp lớn nhất.
\end{baitoan}

\begin{baitoan}[\cite{TLCT_THCS_Toan_9_hinh_hoc}, 5.7., p. 38]
	Cho $\Delta ABC$, I là tâm đường tròn nội tiếp tam giác. Qua I kẻ đường thẳng vuông góc với IA cắt 2 cạnh $AB,AC$ tại $D,E$. Chứng minh $\dfrac{BD}{CE} = \left(\dfrac{IB}{IC}\right)^2$.
\end{baitoan}

\begin{baitoan}[\cite{TLCT_THCS_Toan_9_hinh_hoc}, 5.8., p. 38]
	Cho 3 điểm $A,B,C$ cố định nằm trên 1 đường thẳng theo thứ tự đó. Đường tròn $(O)$ thay đổi luôn đi qua $B,C$. Từ A kẻ 2 tiếp tuyến $AM,AN$ với đường tròn $(O)$, $M,N$ là 2 tiếp điểm. Đường thẳng MN cắt AO tại H, gọi E là trung điểm BC. Chứng minh khi đường tròn $(O)$ thay đổi tâm của đường tròn ngoại tiếp $\Delta OHE$ nằm trên 1 đường thẳng cố định.
\end{baitoan}

\begin{baitoan}[\cite{TLCT_THCS_Toan_9_hinh_hoc}, 5.9., p. 39]
	Cho $\Delta ABC$, $\widehat{A} = 30^\circ$, BC là cạnh nhỏ nhất. Trên AB lấy điểm D, trên AC lấy điểm E sao cho $BD = CE = BC$. Gọi $O,I$ là tâm đường tròn ngoại, nội tiếp $\Delta ABC$. Chứng minh $OI = DE$ \& $OI\bot DE$.
\end{baitoan}

\begin{baitoan}[\cite{TLCT_THCS_Toan_9_hinh_hoc}, 5.10., p. 39]
	Cho $\Delta ABC$ ngoại tiếp đường tròn $(I,r)$, kẻ các tiếp tuyến với đường tròn \& song song với 3 cạnh $\Delta ABC$. Các tiếp tuyến này tạo với 3 cạnh $\Delta ABC$ thành 3 tam giác nhỏ, gọi diện tích 3 tam giác nhỏ là $S_1,S_2,S_3$ \& diện tích $\Delta ABC$ là $S$. Tìm {\rm GTNN} của biểu thức $\dfrac{S_1 + S_2 + S_3}{S}$.
\end{baitoan}

\begin{baitoan}[\cite{TLCT_THCS_Toan_9_hinh_hoc}, 5.11., p. 39]
	Cho $\Delta ABC$, gọi I là tâm đường tròn nội tiếp, $I_A$ là tâm đường tròn bàng tiếp $\widehat{A}$ \& M là trung điểm BC. Gọi $H,D$ là hình chiếu của $I,I_A$ trên cạnh BC. Chứng minh M là trung điểm của DH, từ đó suy ra đường thẳng MI đi qua trung điểm AH.
\end{baitoan}

\begin{baitoan}[\cite{TLCT_THCS_Toan_9_hinh_hoc}, 5.12., p. 39]
	Cho đường tròn $(O,r)$ \& điểm A cố định trên đường tròn. Qua A dựng tiếp tuyến $d$ với đường tròn $(O,r)$. M là điểm chuyển động trên $d$, từ M kẻ tiếp tuyến đến đường tròn $(O,r)$ có tiếp điểm là $B\ne A$. Tâm của đường tròn ngoại tiếp \& trực tâm của $\Delta AMB$ chạy trên đường nào?
\end{baitoan}

\begin{baitoan}[\cite{TLCT_THCS_Toan_9_hinh_hoc}, 5.13., p. 39]
	Cho nửa đường tròn đường kính AB, từ điểm M trên đường tròn kẻ tiếp tuyến $d$. Gọi $H,K$ là hình chiếu của $A,B$ trên $d$. Chứng minh $AH + BK$ không đổi từ đó suy ra đường tròn đường kính HK luôn tiếp xúc với $AH,BK,AB$.
\end{baitoan}

\begin{baitoan}[\cite{TLCT_THCS_Toan_9_hinh_hoc}, 5.14., p. 39]
	Cho $\Delta ABC$, điểm M trong tam giác, gọi $H,D,E$ là hình chiếu của M thứ tự trên $BC,CA,AB$. Xác định vị trí của M sao cho giá trị của biểu thức $\dfrac{BC}{MH} + \dfrac{CA}{MD} + \dfrac{AB}{ME}$ đạt {\rm GTNN}.
\end{baitoan}

\begin{baitoan}[\cite{TLCT_THCS_Toan_9_hinh_hoc}, 5.15., p. 39]
	Cho $\Delta ABC$ vuông tại A. Gọi $O,I$ là tâm đường tròn ngoại \& nội tiếp $\Delta ABC$. Biết $\Delta BIO$ vuông tại I. Chứng minh $\dfrac{BC}{5} = \dfrac{CA}{4} = \dfrac{AB}{3}$.
\end{baitoan}

%------------------------------------------------------------------------------%

\section{Vị Trí Tương Đối của 2 Đường Tròn}

\begin{baitoan}[\cite{TLCT_THCS_Toan_9_hinh_hoc}, VD1, p. 42]
	Cho 2 đường tròn $(O),(O')$ cắt nhau tại $A,B$. Qua A kẻ cát tuyến CAD \& EAF, $C,E\in(O)$, $D,F\in(O')$, sao cho AB là phân giác của $\widehat{CAF}$. Chứng minh $CD = EF$.
\end{baitoan}

\begin{baitoan}[\cite{TLCT_THCS_Toan_9_hinh_hoc}, VD2, pp. 42--43]
	Cho hình chữ nhật ABCD \& 4 đường tròn $(A,R_A),(B,R_B),(C,R_C),(D,R_D)$ sao cho $R_A + R_C = R_B + R_D < AC$. Gọi $d_1,d_3$ là 2 tiếp tuyến chung ngoài của $(A,R_A),(C,R_C)$, $d_2,d_4$ là 2 tiếp tuyến chung ngoài của $(B,R_B),(D,R_D)$. Chứng minh tồn tại 1 đường tròn tiếp xúc với cả 4 đường thẳng $d_1,d_2,d_3,d_4$.
\end{baitoan}

\begin{baitoan}[\cite{TLCT_THCS_Toan_9_hinh_hoc}, VD3, p. 43]
	Cho 2 đường tròn $(O),(O')$ ngoài nhau, $AB,CD$ là 2 tiếp tuyến chung ngoài của 2 đường tròn, đường thẳng AD cắt đường tròn $(O)$ tại M, cắt đường tròn $(O')$ tại N. Chứng minh $AM = DN$.
\end{baitoan}

\begin{baitoan}[\cite{TLCT_THCS_Toan_9_hinh_hoc}, VD4, p. 44]
	Cho 3 đường tròn $(O_1),(O_2),(O_3)$ tiếp xúc ngoài với nhau từng đôi một. Gọi các tiếp điểm của $(O_1),(O_2)$ là A, của $(O_2),(O_3)$ là B, của $(O_3),(O_1)$ là C. $AB,AC$ kéo dài cắt đường tròn $(O_3)$ tại $Q,P$. Chứng minh $P,O_3,Q$ thẳng hàng.
\end{baitoan}

\begin{baitoan}[\cite{TLCT_THCS_Toan_9_hinh_hoc}, VD5, p. 44]
	Cho 2 đường tròn $(O,R),(O',R')$ tiếp xúc ngoài, tiếp tuyến chung ngoài AB, $A\in(O),B\in(O')$. Đường tròn $(I,r)$ tiếp xúc với AB \& 2 đường tròn $(O),(O')$. Chứng minh: (a) $AB = 2\sqrt{RR'}$. (b) $\dfrac{1}{\sqrt{r}} = \dfrac{1}{\sqrt{R}} + \dfrac{1}{\sqrt{R'}}$.
\end{baitoan}

\begin{baitoan}[\cite{TLCT_THCS_Toan_9_hinh_hoc}, VD6, p. 45]
	Cho 3 đường tròn $(A,a),(B,b),(C,c)$ tiếp xúc với nhau từng đôi một. Tại tiếp điểm D của đường tròn $(A,a),(B,b)$, kẻ tiếp tuyến chung cắt đường tròn $(C,c)$ tại $M,N$. Tính MN theo $a,b,c$.
\end{baitoan}

\begin{baitoan}[\cite{TLCT_THCS_Toan_9_hinh_hoc}, VD7, p. 45]
	Cho 2 đường tròn $(O),(O')$ có bán kính bằng nhau, cắt nhau tại $A,B$. Trong nửa mặt phẳng bờ $OO'$ có chứa điểm B, kẻ 2 bán kính $OC\parallel O'D$. Chứng minh B là trực tâm của $\Delta ACD$.
\end{baitoan}

\begin{baitoan}[\cite{TLCT_THCS_Toan_9_hinh_hoc}, VD8, p. 46]
	Cho 2 đường tròn $(O,R),(O',R')$ tiếp xúc ngoài tại A, $\widehat{xOy} = 90^\circ$  thay đổi luôn đi qua A, cắt đường tròn $(O,R),(O',R')$ tại $B,C$. Gọi H là hình chiếu của A trên BC. Xác định vị trí của $B,C$ để AH có độ dài lớn nhất.
\end{baitoan}

\begin{baitoan}[\cite{TLCT_THCS_Toan_9_hinh_hoc}, VD9, p. 47]
	Cho 2 đường tròn $(O,R),(O',R')$, $R > R'$ cắt nhau tại $A,B$. Kẻ đường kính AC \& đường kính AD. Tính độ dài $BC,BD$ biết $CD = a$.
\end{baitoan}

\begin{baitoan}[\cite{TLCT_THCS_Toan_9_hinh_hoc}, VD10, p. 47]
	Cho $\Delta ABC$. Tìm điểm M sao cho $\Delta MAB,\Delta MBC,\Delta MCA$ có chu vi bằng nhau.
\end{baitoan}

\begin{baitoan}[\cite{TLCT_THCS_Toan_9_hinh_hoc}, VD11, p. 48]
	Cho đường tròn $(O)$ \& dây cung AB. M là điểm trên AB. Dựng đường tròn $(O_1)$ qua $A,M$ \& tiếp xúc với $(O)$, đường tròn $(O_2)$ qua $B,M$ \& tiếp xúc với $(O)$, 2 đường tròn này cắt nhau tại điểm thứ 2 là N. Chứng minh $\widehat{MNO} = 90^\circ$.
\end{baitoan}

\begin{baitoan}[\cite{TLCT_THCS_Toan_9_hinh_hoc}, VD12, p. 48]
	Cho 2 đường tròn $(O),(O')$ ngoài nhau, tiếp tuyến chung trong CD \& tiếp tuyến chung ngoài AB, $A,C\in(O)$, $B,D\in(O')$. Chứng minh $AC,BD,OO'$ đồng quy.
\end{baitoan}

\begin{baitoan}[\cite{TLCT_THCS_Toan_9_hinh_hoc}, VD13, p. 49]
	Dựng 2 đường tròn tiếp xúc ngoài với nhau có tâm là 2 điểm $A,B$ cho trước, sao cho 1 trong 2 tiếp tuyến chung ngoài đi qua điểm M cho trước.
\end{baitoan}

\begin{baitoan}[\cite{TLCT_THCS_Toan_9_hinh_hoc}, 6.1., p. 50]
	Cho đường tròn $(O,R)$ ngoại tiếp $\Delta ABC$ đều. Đường tròn $(O')$ tiếp xúc với 2 cạnh $AB,AC$ \& đường tròn $(O,R)$. Tính khoảng cách từ $O'$ đến B theo $R$.
\end{baitoan}

\begin{baitoan}[\cite{TLCT_THCS_Toan_9_hinh_hoc}, 6.2., p. 50]
	Cho nửa đường tròn đường kính AB, điểm C trên nửa đường tròn sao cho $CA < CB$, H là hình chiếu của C trên AB. Gọi I là trung điểm của CH, đường tròn $(I,CH/2)$ cắt nửa đường tròn tại D \& cắt 2 cạnh $CA,CB$ thứ tự tại $M,N$, đường thẳng CD cắt AB tại E. Chứng minh: (a) CMHN là hình chữ nhật. (b) $E,I,M,N$ thẳng hàng.
\end{baitoan}

\begin{baitoan}[\cite{TLCT_THCS_Toan_9_hinh_hoc}, 6.3., p. 50]
	Cho 3 đường tròn $O_1,O_2,O_3$ có cùng bán kính $R$ cắt nhau tại điểm O cho trước. $A,B,C$ là 3 giao điểm còn lại của 3 đường tròn. Chứng minh đường tròn ngoại tiếp $\Delta ABC$ có bán kính $R$.
\end{baitoan}

\begin{baitoan}[\cite{TLCT_THCS_Toan_9_hinh_hoc}, 6.4., p. 50]
	3 đường tròn có bán kính bằng nhau cùng đi qua điểm O, từng đôi cắt nhau tại điểm thứ 2 là $A,B,C$. Chứng minh O là trực tâm $\Delta ABC$.
\end{baitoan}

\begin{baitoan}[\cite{TLCT_THCS_Toan_9_hinh_hoc}, 6.5., p. 50]
	Cho 2 đường tròn $(O_1),(O_2)$ cắt nhau tại $A,B$, kẻ dây AM của đường tròn $(O_1)$ tiếp xúc với đường tròn $(O_2)$ tại A, kẻ dây AN của $(O_2)$ tiếp xúc với đường tròn $(O_1)$ tại A. Trên đường thẳng AB lấy điểm D sao cho $BD = AB$. Chứng minh 4 điểm $A,M,N,D$ nằm trên 1 đường tròn.
\end{baitoan}

\begin{baitoan}[\cite{TLCT_THCS_Toan_9_hinh_hoc}, 6.6., p. 50]
	Cho đường tròn $(O,R)$, 1 điểm A trên đường tròn \& đường thẳng $d$ không đi qua A. Dựng đường tròn tiếp xúc với $(O,R)$ tại A \& tiếp xúc với đường thẳng $d$.
\end{baitoan}

\begin{baitoan}[\cite{TLCT_THCS_Toan_9_hinh_hoc}, 6.7., p. 51]
	Cho 2 đường tròn $(O),(O')$ có cùng bán kính $R$ sao cho tâm của đường tròn này nằm trên đường tròn kia, chúng cắt nhau tại $A,B$. Tính bán kính của đường tròn tâm I tiếp xúc với 2 cung nhỏ $\arc{AO},\arc{AO'}$ đồng thời tiếp xúc với $OO'$.
\end{baitoan}

\begin{baitoan}[\cite{TLCT_THCS_Toan_9_hinh_hoc}, 6.8., p. 51]
	Cho đường tròn $(O)$ \& dây AB cố định, điểm M tùy ý thay đổi trên đoạn thẳng AB. Qua $A,M$ dựng đường tròn tâm I tiếp xúc với đường tròn $(O)$ tại A. Qua $B,M$ dựng đường tròn tâm J tiếp xúc với $(O)$ tại B. 2 đường tròn tâm $I,J$ cắt nhau tại điểm thứ 2 N. Chứng minh MN luôn đi qua 1 điểm cố định.
\end{baitoan}

\begin{baitoan}[\cite{TLCT_THCS_Toan_9_hinh_hoc}, 6.9., p. 51]
	Cho đoạn thẳng AB có độ dài bằng $a$ cho trước \& 2 tia $Ax,By$ vuông góc với AB, nằm về cùng 1 phía đối với AB. Gọi $(O),(O')$ là 2 đường tròn thay đổi thỏa mãn đồng thời: (a) $(O)$ tiếp xúc với $(O')$. (b) Đường tròn $(O)$ tiếp xúc với $Ax,AB$. (c) Đường tròn $(O')$ tiếp xúc với By \& tiếp xúc với BA. Tính {\rm GTLN} của diện tích hình thang $HOO'E$, trong đó $H,E$ là hình chiếu của $O,O'$ trên AB.
\end{baitoan}

\begin{baitoan}[\cite{TLCT_THCS_Toan_9_hinh_hoc}, 6.10., p. 51]
	Cho 2 đường tròn $(O_1,R_1),(O_2,R_2)$ tiếp xúc ngoài tại A. 1 đường tròn $(O)$ thay đổi tiếp xúc ngoài với 2 đường tròn $(O_1,R_1),(O_2,R_2)$. Giả sử MN là đường kính của đường tròn $(O)$ sao cho $MN\parallel OO'$. Gọi H là giao điểm của $MO_2,NO_1$. Chứng minh điểm H thuộc 1 đường thẳng cố định.
\end{baitoan}

%------------------------------------------------------------------------------%

\section{Tính Chất của 2 Tiếp Tuyến Cắt Nhau}

\begin{baitoan}[\cite{Binh_Toan_9_tap_1}, Ví dụ 14, p. 102]
	Cho đoạn thẳng AB. Trên cùng 1 nửa mặt phẳng bờ AB, vẽ nửa đường tròn $(O)$ đường kính AB \& 2 tiếp tuyến Ax, By. Qua điểm M thuộc nửa đường tròn này, kẻ tiếp tuyến cắt Ax, By lần lượt tại C, D. Gọi N là giao điểm của AD \& BC. Chứng minh $MN\bot AB$.
\end{baitoan}

\begin{baitoan}[\cite{Binh_Toan_9_tap_1}, Ví dụ 15, p. 103]
	Cho $(O)$, điểm K nằm bên ngoài đường tròn. Kẻ 2 tiếp tuyến KA, KB với đường tròn (A, B là 2 tiếp điểm). Kẻ đường kính AOC. Tiếp tuyến của đường tròn $(O)$ tại C cắt AB tại E. Chứng minh: (a) $\Delta KBC\backsim\Delta OBE$. (b) $CK\bot OE$.
\end{baitoan}

\begin{baitoan}[\cite{Binh_Toan_9_tap_1}, 72., p. 103]
	
\end{baitoan}

\begin{baitoan}[\cite{Binh_Toan_9_tap_1}, 73., p. 104]
	
\end{baitoan}

\begin{baitoan}[\cite{Binh_Toan_9_tap_1}, 74., p. 104]
	
\end{baitoan}

\begin{baitoan}[\cite{Binh_Toan_9_tap_1}, 75., p. 104]
	
\end{baitoan}

\begin{baitoan}[\cite{Binh_Toan_9_tap_1}, 76., p. 104]
	
\end{baitoan}

\begin{baitoan}[\cite{Binh_Toan_9_tap_1}, 77., p. 104]
	
\end{baitoan}

\begin{baitoan}[\cite{Binh_Toan_9_tap_1}, 78., p. 105]
	
\end{baitoan}

\begin{baitoan}[\cite{Binh_Toan_9_tap_1}, 79., p. 105]
	
\end{baitoan}

\begin{baitoan}[\cite{Binh_Toan_9_tap_1}, 80., p. 105]
	
\end{baitoan}

%------------------------------------------------------------------------------%

\section{Đường Tròn Nội Tiếp Tam Giác}

%------------------------------------------------------------------------------%

\section{Miscellaneous}

%------------------------------------------------------------------------------%

\printbibliography[heading=bibintoc]

\end{document}