\documentclass{article}
\usepackage[backend=biber,natbib=true,style=alphabetic,maxbibnames=50]{biblatex}
\addbibresource{/home/nqbh/reference/bib.bib}
\usepackage[utf8]{vietnam}
\usepackage{tocloft}
\renewcommand{\cftsecleader}{\cftdotfill{\cftdotsep}}
\usepackage[colorlinks=true,linkcolor=blue,urlcolor=red,citecolor=magenta]{hyperref}
\usepackage{amsmath,amssymb,amsthm,float,graphicx,mathtools,tikz}
\usetikzlibrary{angles,calc,intersections,matrix,patterns,quotes,shadings}
\makeatletter
\DeclareFontFamily{U}{tipa}{}
\DeclareFontShape{U}{tipa}{m}{n}{<->tipa10}{}
\newcommand{\arc@char}{{\usefont{U}{tipa}{m}{n}\symbol{62}}}%

\newcommand{\arc}[1]{\mathpalette\arc@arc{#1}}

\newcommand{\arc@arc}[2]{%
	\sbox0{$\m@th#1#2$}%
	\vbox{
		\hbox{\resizebox{\wd0}{\height}{\arc@char}}
		\nointerlineskip
		\box0
	}%
}
\makeatother
\allowdisplaybreaks
\newtheorem{assumption}{Assumption}
\newtheorem{baitoan}{}
\newtheorem{cauhoi}{Câu hỏi}
\newtheorem{conjecture}{Conjecture}
\newtheorem{corollary}{Corollary}
\newtheorem{dangtoan}{Dạng toán}
\newtheorem{definition}{Definition}
\newtheorem{dinhly}{Định lý}
\newtheorem{dinhnghia}{Định nghĩa}
\newtheorem{example}{Example}
\newtheorem{ghichu}{Ghi chú}
\newtheorem{hequa}{Hệ quả}
\newtheorem{hypothesis}{Hypothesis}
\newtheorem{lemma}{Lemma}
\newtheorem{luuy}{Lưu ý}
\newtheorem{nhanxet}{Nhận xét}
\newtheorem{notation}{Notation}
\newtheorem{note}{Note}
\newtheorem{principle}{Principle}
\newtheorem{problem}{Problem}
\newtheorem{proposition}{Proposition}
\newtheorem{question}{Question}
\newtheorem{remark}{Remark}
\newtheorem{theorem}{Theorem}
\newtheorem{vidu}{Ví dụ}
\usepackage[left=1cm,right=1cm,top=5mm,bottom=5mm,footskip=4mm]{geometry}
\def\labelitemii{$\circ$}
\DeclareRobustCommand{\divby}{%
	\mathrel{\vbox{\baselineskip.65ex\lineskiplimit0pt\hbox{.}\hbox{.}\hbox{.}}}%
}

\title{Cheatsheet Mathematics 9}
\author{Nguyễn Quản Bá Hồng\footnote{Ben Tre City, Vietnam. e-mail: \texttt{nguyenquanbahong@gmail.com}; website: \url{https://nqbh.github.io}.}}
\date{\today}

\begin{document}
\maketitle
\tableofcontents

%------------------------------------------------------------------------------%

\section{Square, Cube, \& $n$th Roots -- Căn Bậc 2, 3, $n$}

%------------------------------------------------------------------------------%

\section{1st-Order Function -- Hàm Số Bậc Nhất $y = ax + b$}

%------------------------------------------------------------------------------%

\section{System of 1st-Order Equations -- Hệ Phương Trình Bậc Nhất 2 Ẩn}

\subsection{1st-order equations of 2 unknowns -- Phương trình bậc nhất 2 ẩn $ax + by = c$}
\fbox{1} Phương trình bậc nhất 2 ẩn: \fbox{$ax + by = c$} (1), $a,b,c\in\mathbb{R},(a,b)\ne(0,0)$. \fbox{2} $(x_0,y_0)\in\mathbb{R}^2$ là nghiệm của (1) $\Leftrightarrow(x_0,y_0)\in S\Leftrightarrow ax_0 + by_0 = c$. \fbox{3} Tập nghiệm $S$ biểu diễn bởi đường thẳng $(d):ax + by = c$, i.e., $S = \{(x,y)\in\mathbb{R}^2|ax + by = c\} = (d)$. \fbox{4} Nếu $ab\ne0$ thì $(d):ax + by = c\Leftrightarrow y = -\dfrac{a}{b}x + \dfrac{c}{b}$ (hàm số bậc nhất) là đường thẳng cắt cả 2 trục tọa độ $Ox,Oy$ lần lượt tại 2 điểm $\left(\dfrac{c}{a},0\right),\left(0,\dfrac{c}{b}\right)$. \fbox{5} Nếu $a\ne0,b = 0$ thì $(d):ax + 0y = c\Leftrightarrow x = \dfrac{c}{a}$ là đường thẳng song song hoặc trùng với trục tung $Oy$. \fbox{6} Nếu $a = 0,b\ne0$ thì $(d):0x + by = c\Leftrightarrow y = \dfrac{c}{b}$ là đường thẳng song song hoặc trùng với trục hoành $Ox$.

%------------------------------------------------------------------------------%

\section{2nd-Order Function. Quadratic Equation -- Hàm Số $y = ax^2,a\ne0$. Phương Trình Bậc 2 1 Ẩn $ax^2 + bx + c = 0,a\ne0$}

\subsection{Hàm số $y = ax^2,a\ne0$}

\subsection{Đồ thị của hàm số $y = ax^2,a\ne0$}

\subsection{Quadratic equation -- Phương trình bậc 2 1 ẩn $ax^2 + bx + c = 0,a\ne0$}

\subsection{Vi\`ete theorem -- Định lý Vi\`ete}

%------------------------------------------------------------------------------%

\section{Trigonometry in Right Triangles -- Hệ Thức Lượng Trong Tam Giác Vuông}

%------------------------------------------------------------------------------%

\section{Circle -- Đường Tròn}

\subsection{Góc ở tâm. Số đo cung}
\fbox{1} Cho đường tròn $(O;R)$, $\widehat{AOB} = \alpha\in[0^\circ,180^\circ]$: góc ở tâm. Nếu $0^\circ < \alpha < 180^\circ$, cung nhỏ $\arc{AmB}$ có số đo cung $\mbox{\rm sđ}\arc{AmB} = \alpha$, cung lớn $\arc{AnB}$ có số đo cung $\mbox{\rm sđ}\arc{AnB} = 360^\circ - \alpha$. Nếu $\alpha = 0^\circ$, cung không có số đo $0^\circ$ \& cung cả đường tròn có số đo $360^\circ$. Nếu $\alpha = 180^\circ$, 2 cung $\arc{AmB},\arc{AnB}$ là 2 nửa đường tròn với $\mbox{\rm sđ}\arc{AmB} = \mbox{\rm sđ}\arc{AnB} = 180^\circ$. \fbox{2} Trên cùng 1 đường tròn $(O;R)$ hoặc trên 2 đường tròn bằng nhau $(O;R),(O';R),O\ne O'$, $\mbox{\rm sđ}\arc{AB} = \mbox{\rm sđ}\arc{CD}\Leftrightarrow\arc{AB} = \arc{CD}\Leftrightarrow AB = CD$, $\mbox{\rm sđ}\arc{AB} < \mbox{\rm sđ}\arc{CD}\Leftrightarrow\arc{AB} < \arc{CD}\Leftrightarrow AB < CD$. Tính chất này không còn đúng khi xét trên 2 đường tròn không bằng nhau $(O;R),(O',R')$ với $R\ne R'$. \fbox{3} $B\in\arc{AC}\Rightarrow\mbox{\rm sđ}\arc{AB} + \mbox{\rm sđ}\arc{BC} = \mbox{\rm sđ}\arc{AC}$.

\subsection{Liên hệ giữa cung \& dây}
\fbox{1} 2 cung chắn giữa 2 dây song song thì bằng nhau.

\subsection{Góc nội tiếp}
\fbox{1} Cho đường tròn $(O;R)$, $\angle BAC$: góc nội tiếp chắn cung $\arc{BC}$ thì $\widehat{BAC} = \frac{1}{2}\mbox{\rm sđ}\arc{BC} = \frac{1}{2}\widehat{BOC}$. \fbox{2} Các góc nội tiếp bằng nhau chắn các cung bằng nhau. \fbox{3} Các góc nội tiếp cùng chắn 1 cung hoặc các cung bằng nhau thì bằng nhau. \fbox{4} Góc nội tiếp $\le90^\circ$ có số đo bằng nữa số đo góc ở tâm cùng chắn 1 cung. \fbox{5} Góc nội tiếp chắn nửa đường tròn là góc vuông.

\subsection{Góc tạo bởi tia tiếp tuyến \& dây cung}
\fbox{1} Cho đường tròn $(O;R)$, $Ax$: tia tiếp tuyến, $AB$: dây cung, $\widehat{BAx} = \frac{1}{2}\mbox{\rm sđ}\arc{AB}$. \fbox{2} Trong 1 đường tròn, góc tạo bởi tia tiếp tuyến \& dây cung \& góc nội tiếp cùng chắn 1 cung thì bằng nhau.

\subsection{Góc có đỉnh ở bên trong{\tt/}ngoài đường tròn}
\fbox{1} $\widehat{BEC}$: góc có đỉnh ở bên trong đường tròn $(O;R)$ chắn 2 cung $\arc{DmA},\arc{BnC}$: $\widehat{BEC} = \frac{1}{2}(\mbox{\rm sđ}\arc{DmA} + \mbox{\rm sđ}\arc{BnC})$. \fbox{2} $\widehat{BEC}$: góc có đỉnh ở bên ngoài đường tròn $(O;R)$ chắn 2 cung nhỏ $\arc{AB},\arc{CD}$: $\widehat{BEC} = \frac{1}{2}|\mbox{\rm sđ}\arc{AB} - \mbox{\rm sđ}\arc{CD}|$.

\subsection{Cung chứa góc}
\fbox{1} $A,B$ cố định, $\widehat{AMB} = \alpha\in(0^\circ,180^\circ)\Rightarrow$ Quỹ tích điểm $M$ là 2 cung $\arc{AmB},\arc{Am'B}$ chứa góc $\alpha$ dựng trên đoạn $AB$. Nếu $\alpha = 90^\circ$, quỹ tích điểm $M$ là đường tròn đường kính $AB$. \fbox{2} {\sf Bài toán quỹ tích}: \textit{Phần thuận}: Mọi điểm có tính chất $\mathcal{T}$ đều thuộc hình $\mathcal{H}$. \textit{Phần đảo}: Mọi điểm thuộc hình $\mathcal{H}$ đều có tính chất $\mathcal{T}$. \textit{Kết luận}: Quỹ tích các điểm $M$ có tính chất $\mathcal{T}$ là hình $\mathcal{H}$.

\subsection{Tứ giác nội tiếp}
\fbox{1} $A,B,C,D\in(O)$ (theo thứ tự đó) $\Leftrightarrow ABCD$: tứ giác nội tiếp $\Leftrightarrow\widehat{BAD} + \widehat{BCD} = 180^\circ\Leftrightarrow\widehat{BAC} = \widehat{BDC}$. \fbox{2} Tứ giác nội tiếp có tổng 2 góc đối diện bằng $180^\circ$.

\subsection{Đường tròn ngoại tiếp. Đường tròn nội tiếp}
$\forall n\in\mathbb{N},n\ge3$: \fbox{1} Đa giác $A_1A_2\ldots A_n$ nội tiếp đường tròn $(O;R)\Leftrightarrow(O;R)$ ngoại tiếp đa giác $A_1A_2\ldots A_n$. \fbox{2} Đa giác $A_1A_2\ldots A_n$ ngoại tiếp đường tròn $(O;R)\Leftrightarrow(O;R)$ nội tiếp đa giác $A_1A_2\ldots A_n$. \fbox{3} Mọi đa giác đều đều có đường tròn ngoại tiếp \& đường tròn nội tiếp. Tâm 2 đường tròn ngoại tiếp \& nội tiếp là tâm đa giác đều. \fbox{4} Tam giác bất kỳ (không nhất thiết phải đều) luôn có đường tròn ngoại tiếp \& đường tròn nội tiếp nhưng đa giác với $n\ge4$ cạnh chưa chắc có đường tròn ngoại tiếp hay đường tròn nội tiếp. Đa giác với $n\ge4$ cạnh phải thỏa 1 số điều kiện nhất định thì mới có đường tròn nội tiếp hoặc đường tròn nội tiếp hoặc cả 2.

\subsection{Độ dài đường tròn, cung tròn. Diện tích hình tròn, hình quạt tròn}
\fbox{1} Chu vi{\tt/}độ dài đường tròn $(O;R)$: $C = 2\pi R = \pi d$ với $d = 2R$: đường kính. Độ dài cung tròn $n^\circ\in[0^\circ,360^\circ]$: $l = \dfrac{\pi Rn}{180}$. \fbox{2} Diện tích hình tròn $S = \pi R^2$. Diện tích hình quạt tròn $n^\circ$: $S_{\rm q} = \dfrac{\pi R^2n}{360} = \dfrac{lR}{2}$. \fbox{3} Diện tích hình vành khăn $S = \pi(R^2 - r^2)$.

%------------------------------------------------------------------------------%

\section{Cylinder. Cone. Sphere -- Hình Trụ. Hình Nón. Hình Cầu}

\subsection{Cylinder -- Hình trụ}
\fbox{1} Quay hình chữ nhật $ABCD$ 1 vòng quanh cạnh $CD$ cố định (trục quay) được 1 hình trụ với 2 đáy: 2 hình tròn $(C;R),(D;R)$ với $R = AD = BC$, mặt xung quanh, đường sinh $AB$, chiều cao $AB = h$. \fbox{2} {\sf Thiết diện}: \textit{Mặt cắt song song với đáy}: Thiết diện là 1 hình tròn bằng đáy. \textit{Mặt cắt song song với trục}: Thiết diện là 1 hình chữ nhật. \fbox{3} Hình trụ có diện tích xung quanh $S_{\rm xq} = 2\pi Rh$, diện tích toàn phần $S_{\rm tp} = 2\pi Rh + 2\pi R^2$, thể tích $V = S_{\scriptsize\mbox{\rm đ}}h = \pi R^2h$.

\subsection{Cone. Chopped Cone -- Hình nón. Hình nón cụt}
\fbox{1} Quy $\Delta AOB$ vuông tại $O$ 1 vòng quành cạnh góc vuông $OA$ cố định được 1 hình nón có đáy: hình tròn $(O;R)$, đỉnh $A$, mặt xung quanh, đường sinh $AB = l$, chiều cao $AO = h$. \fbox{2} Hình nón có diện tích xung quanh $S_{xq} = \pi Rl$, diện tích toàn phần $S_{\rm tp} = \pi Rl + \pi R^2$, thể tích $V = \frac{1}{3}S_{\scriptsize\mbox{\rm đ}}h = \frac{1}{3}\pi R^2h$. \fbox{3} Hình nón cụt với 2 đáy $(O';r),(O;R)$ có diện tích xung quanh $S_{\rm xq} = \pi(R + r)l$, diện tích toàn phần $S_{\rm tp} = \pi(R + r)l + \pi(R^2 + r^2)$, thể tích $V = \frac{1}{3}\pi h(R^2 + Rr + r^2)$.

\subsection{Sphere -- Hình cầu}
\fbox{1} Quay nửa hình tròn tâm $O$ 1 vòng quanh đường kính $AB$ cố định ta được 1 hình cầu. \fbox{2} {\sf Thiết diện}: Cắt hình cầu (mặt cầu) bán kính $R$ bởi 1 mặt phẳng ta được 1 hình tròn (đường tròn) bán kính $r$: Bán kính đường tròn lớn $r = R$ nếu mặt phẳng cắt đi qua tâm. Bán kính đường tròn $r < R$ nếu mặt phẳng cắt không đi qua tâm. \fbox{3} Hình cầu có diện tích $S = 4\pi R^2 = \pi d^2$, thể tích $V = \frac{4}{3}\pi R^3$. \fbox{4} Hình cầu nội tiếp hình trụ thì bán kính hình cầu bằng bán kính đáy hình trụ, chiều cao hình trụ bằng đường kính hình cầu, $V_{\rm c} = \frac{2}{3}V_{\rm tr}$. \fbox{5} Công thức tính thể tích các vật thể có 2 đáy song song: $V = \dfrac{h}{6}(B_1 + 4B_2 + B_3)$ với $h$: chiều cao của vật thể, $B_1$: diện tích đáy dưới, $B_2$: diện tích thiết diện trung bình (thiết diện qua trung điểm của chiều cao), $B_3$: diện tích đáy trên.

%------------------------------------------------------------------------------%

\printbibliography[heading=bibintoc]
	
\end{document}