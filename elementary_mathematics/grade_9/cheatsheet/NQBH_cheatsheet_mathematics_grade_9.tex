\documentclass{article}
\usepackage[backend=biber,natbib=true,style=alphabetic,maxbibnames=50]{biblatex}
\addbibresource{/home/nqbh/reference/bib.bib}
\usepackage[utf8]{vietnam}
\usepackage{tocloft}
\renewcommand{\cftsecleader}{\cftdotfill{\cftdotsep}}
\usepackage[colorlinks=true,linkcolor=blue,urlcolor=red,citecolor=magenta]{hyperref}
\usepackage{amsmath,amssymb,amsthm,float,graphicx,mathtools,tikz}
\usetikzlibrary{angles,calc,intersections,matrix,patterns,quotes,shadings}
\allowdisplaybreaks
\newtheorem{assumption}{Assumption}
\newtheorem{baitoan}{}
\newtheorem{cauhoi}{Câu hỏi}
\newtheorem{conjecture}{Conjecture}
\newtheorem{corollary}{Corollary}
\newtheorem{dangtoan}{Dạng toán}
\newtheorem{definition}{Definition}
\newtheorem{dinhly}{Định lý}
\newtheorem{dinhnghia}{Định nghĩa}
\newtheorem{example}{Example}
\newtheorem{ghichu}{Ghi chú}
\newtheorem{hequa}{Hệ quả}
\newtheorem{hypothesis}{Hypothesis}
\newtheorem{lemma}{Lemma}
\newtheorem{luuy}{Lưu ý}
\newtheorem{nhanxet}{Nhận xét}
\newtheorem{notation}{Notation}
\newtheorem{note}{Note}
\newtheorem{principle}{Principle}
\newtheorem{problem}{Problem}
\newtheorem{proposition}{Proposition}
\newtheorem{question}{Question}
\newtheorem{remark}{Remark}
\newtheorem{theorem}{Theorem}
\newtheorem{vidu}{Ví dụ}
\usepackage[left=1cm,right=1cm,top=5mm,bottom=5mm,footskip=4mm]{geometry}
\def\labelitemii{$\circ$}
\DeclareRobustCommand{\divby}{%
	\mathrel{\vbox{\baselineskip.65ex\lineskiplimit0pt\hbox{.}\hbox{.}\hbox{.}}}%
}

\title{Cheatsheet Mathematics 9}
\author{Nguyễn Quản Bá Hồng\footnote{Ben Tre City, Vietnam. e-mail: \texttt{nguyenquanbahong@gmail.com}; website: \url{https://nqbh.github.io}.}}
\date{\today}

\begin{document}
\maketitle
\tableofcontents

%------------------------------------------------------------------------------%

\section{Square, Cube, \& $n$th Roots -- Căn Bậc 2, 3, $n$}

%------------------------------------------------------------------------------%

\section{1st-Order Function -- Hàm Số Bậc Nhất $y = ax + b$}

%------------------------------------------------------------------------------%

\section{System of 1st-Order Equations -- Hệ Phương Trình Bậc Nhất 2 Ẩn}

\subsection{1st-Order Equations of 2 Unknowns -- Phương Trình Bậc Nhất 2 Ẩn}
\fbox{1} Phương trình bậc nhất 2 ẩn: \fbox{$ax + by = c$} (1), $a,b,c\in\mathbb{R},(a,b)\ne(0,0)$. \fbox{2} $(x_0,y_0)\in\mathbb{R}^2$ là nghiệm của (1) $\Leftrightarrow(x_0,y_0)\in S\Leftrightarrow ax_0 + by_0 = c$. \fbox{3} Tập nghiệm $S$ biểu diễn bởi đường thẳng $(d):ax + by = c$, i.e., $S = \{(x,y)\in\mathbb{R}^2|ax + by = c\} = (d)$. \fbox{4} Nếu $ab\ne0$ thì $(d):ax + by = c\Leftrightarrow y = -\dfrac{a}{b}x + \dfrac{c}{b}$ (hàm số bậc nhất) là đường thẳng cắt cả 2 trục tọa độ $Ox,Oy$ lần lượt tại 2 điểm $\left(\dfrac{c}{a},0\right),\left(0,\dfrac{c}{b}\right)$. \fbox{5} Nếu $a\ne0,b = 0$ thì $(d):ax + 0y = c\Leftrightarrow x = \dfrac{c}{a}$ là đường thẳng song song hoặc trùng với trục tung $Oy$. \fbox{6} Nếu $a = 0,b\ne0$ thì $(d):0x + by = c\Leftrightarrow y = \dfrac{c}{b}$ là đường thẳng song song hoặc trùng với trục hoành $Ox$.

%------------------------------------------------------------------------------%

\section{2nd-Order Function. Quadratic Equation -- Hàm Số $y = ax^2,a\ne0$. Phương Trình Bậc 2 1 Ẩn $ax^2 + bx + c = 0,a\ne0$}

%------------------------------------------------------------------------------%

\section{Trigonometry in Right Triangles -- Hệ Thức Lượng Trong Tam Giác Vuông}

%------------------------------------------------------------------------------%

\section{Circle -- Đường Tròn}

%------------------------------------------------------------------------------%

\section{Cylinder. Cone. Sphere -- Hình Trụ. Hình Nón. Hình Cầu}

%------------------------------------------------------------------------------%

\section{Miscellaneous}

%------------------------------------------------------------------------------%

\printbibliography[heading=bibintoc]
	
\end{document}