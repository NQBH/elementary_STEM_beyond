\documentclass{article}
\usepackage[backend=biber,natbib=true,style=alphabetic,maxbibnames=50]{biblatex}
\addbibresource{/home/nqbh/reference/bib.bib}
\usepackage[utf8]{vietnam}
\usepackage{tocloft}
\renewcommand{\cftsecleader}{\cftdotfill{\cftdotsep}}
\usepackage[colorlinks=true,linkcolor=blue,urlcolor=red,citecolor=magenta]{hyperref}
\usepackage{amsmath,amssymb,amsthm,float,graphicx,mathtools,tikz}
\usetikzlibrary{angles,calc,intersections,matrix,patterns,quotes,shadings}
\allowdisplaybreaks
\newtheorem{assumption}{Assumption}
\newtheorem{baitoan}{Bài toán}
\newtheorem{cauhoi}{Câu hỏi}
\newtheorem{conjecture}{Conjecture}
\newtheorem{corollary}{Corollary}
\newtheorem{dangtoan}{Dạng toán}
\newtheorem{definition}{Definition}
\newtheorem{dinhly}{Định lý}
\newtheorem{dinhnghia}{Định nghĩa}
\newtheorem{example}{Example}
\newtheorem{ghichu}{Ghi chú}
\newtheorem{hequa}{Hệ quả}
\newtheorem{hypothesis}{Hypothesis}
\newtheorem{lemma}{Lemma}
\newtheorem{luuy}{Lưu ý}
\newtheorem{nhanxet}{Nhận xét}
\newtheorem{notation}{Notation}
\newtheorem{note}{Note}
\newtheorem{principle}{Principle}
\newtheorem{problem}{Problem}
\newtheorem{proposition}{Proposition}
\newtheorem{question}{Question}
\newtheorem{remark}{Remark}
\newtheorem{theorem}{Theorem}
\newtheorem{vidu}{Ví dụ}
\usepackage[left=1cm,right=1cm,top=5mm,bottom=5mm,footskip=4mm]{geometry}
\def\labelitemii{$\circ$}
\DeclareRobustCommand{\divby}{%
	\mathrel{\vbox{\baselineskip.65ex\lineskiplimit0pt\hbox{.}\hbox{.}\hbox{.}}}%
}

\title{Problem: Trigonometry In Triangles\\Bài Tập: Hệ Thức Lượng Trong Tam Giác}
\author{Nguyễn Quản Bá Hồng\footnote{Independent Researcher, Ben Tre City, Vietnam\\e-mail: \texttt{nguyenquanbahong@gmail.com}; website: \url{https://nqbh.github.io}.}}
\date{\today}

\begin{document}
\maketitle
\begin{abstract}
	Last updated version: \href{https://github.com/NQBH/elementary_STEM_beyond/blob/main/elementary_mathematics/grade_9/trigonometry/problem/NQBH_trigonometry_problem.pdf}{GitHub{\tt/}NQBH{\tt/}elementary STEM \& beyond{\tt/}elementary mathematics{\tt/}grade 9{\tt/}trigonometry{\tt/}problem: set $\mathbb{Q}$ of trigonometrys [pdf]}.\footnote{\textsc{url}: \url{https://github.com/NQBH/elementary_STEM_beyond/blob/main/elementary_mathematics/grade_9/trigonometry/problem/NQBH_trigonometry_problem.pdf}.} [\href{https://github.com/NQBH/elementary_STEM_beyond/blob/main/elementary_mathematics/grade_9/trigonometry/problem/NQBH_trigonometry_problem.tex}{\TeX}]\footnote{\textsc{url}: \url{https://github.com/NQBH/elementary_STEM_beyond/blob/main/elementary_mathematics/grade_9/rational/problem/NQBH_trigonometry_problem.tex}.}. 
\end{abstract}
\tableofcontents

%------------------------------------------------------------------------------%

\section{1 Số Hệ Thức Lượng về Cạnh \& Đường Cao Trong Tam Giác Vuông}
\textbf{\textsf{Ký hiệu.}} $\Delta ABC$ vuông tại $A$: $a\coloneqq BC$, $b\coloneqq CA$, $c\coloneqq AB$, $b'\coloneqq CH$, $c'\coloneqq BH$, $h\coloneqq AH$.\\\textbf{\textsf{Tính chất.}} \fbox{1} $b^2 = ab'$, $c^2 = ac'$. \fbox{2} \textit{Định lý Pythagore thuận \& đảo}: $\Delta ABC$ vuông tại $A\Leftrightarrow a^2 = b^2 + c^2$. \fbox{3} $h^2 = b'c'$. \fbox{4} $ah = bc = 2S_{ABC}$. \fbox{5} $\dfrac{1}{h^2} = \dfrac{1}{b^2} + \dfrac{1}{c^2}$.

\begin{baitoan}[\cite{Binh_Toan_9_tap_1}, Ví dụ 1, p. 84]
	Tính diện tích hình thang $ABCD$ có đường cao bằng {\rm12 cm}, 2 đường chéo $AC,BD$ vuông góc với nhau, $BD = 15$ {\rm cm}.
\end{baitoan}

\begin{baitoan}[\cite{Binh_Toan_9_tap_1}, Ví dụ 2, p. 85]
	Hình thang cân ABCD có đáy lớn $CD = 10$ {\rm cm}, đáy nhỏ bằng đường cao, đường chéo vuông góc với cạnh bên. Tính đường cao của hình thang.
\end{baitoan}

\begin{baitoan}[\cite{Binh_Toan_9_tap_1}, Ví dụ 3, p. 85]
	Tính diện tích 1 tam giác vuông có chu vi {\rm72 cm}, hiệu giữa đường trung tuyến \& đường cao ứng với cạnh huyền bằng {\rm7 cm}.
\end{baitoan}

%------------------------------------------------------------------------------%

\section{Tỷ Số Lượng Giác của Góc Nhọn}

%------------------------------------------------------------------------------%

\section{Miscellaneous}

%------------------------------------------------------------------------------%

\printbibliography[heading=bibintoc]

\end{document}