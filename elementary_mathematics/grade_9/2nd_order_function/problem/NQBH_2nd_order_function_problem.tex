\documentclass{article}
\usepackage[backend=biber,natbib=true,style=alphabetic,maxbibnames=50]{biblatex}
\addbibresource{/home/nqbh/reference/bib.bib}
\usepackage[utf8]{vietnam}
\usepackage{tocloft}
\renewcommand{\cftsecleader}{\cftdotfill{\cftdotsep}}
\usepackage[colorlinks=true,linkcolor=blue,urlcolor=red,citecolor=magenta]{hyperref}
\usepackage{amsmath,amssymb,amsthm,float,graphicx,mathtools,tikz}
\usetikzlibrary{angles,calc,intersections,matrix,patterns,quotes,shadings}
\allowdisplaybreaks
\newtheorem{assumption}{Assumption}
\newtheorem{baitoan}{}
\newtheorem{cauhoi}{Câu hỏi}
\newtheorem{conjecture}{Conjecture}
\newtheorem{corollary}{Corollary}
\newtheorem{dangtoan}{Dạng toán}
\newtheorem{definition}{Definition}
\newtheorem{dinhly}{Định lý}
\newtheorem{dinhnghia}{Định nghĩa}
\newtheorem{example}{Example}
\newtheorem{ghichu}{Ghi chú}
\newtheorem{hequa}{Hệ quả}
\newtheorem{hypothesis}{Hypothesis}
\newtheorem{lemma}{Lemma}
\newtheorem{luuy}{Lưu ý}
\newtheorem{nhanxet}{Nhận xét}
\newtheorem{notation}{Notation}
\newtheorem{note}{Note}
\newtheorem{principle}{Principle}
\newtheorem{problem}{Problem}
\newtheorem{proposition}{Proposition}
\newtheorem{question}{Question}
\newtheorem{remark}{Remark}
\newtheorem{theorem}{Theorem}
\newtheorem{vidu}{Ví dụ}
\usepackage[left=1cm,right=1cm,top=5mm,bottom=5mm,footskip=4mm]{geometry}
\def\labelitemii{$\circ$}
\DeclareRobustCommand{\divby}{%
	\mathrel{\vbox{\baselineskip.65ex\lineskiplimit0pt\hbox{.}\hbox{.}\hbox{.}}}%
}

\title{Problem: 2nd-Order Function. Quadratic Equation\\Bài Tập: Hàm Số Bậc 2 $y = ax^2$. Phương Trình Bậc 2 1 Ẩn $ax^2 + bx + c = 0$}
\author{Nguyễn Quản Bá Hồng\footnote{Ben Tre City, Vietnam. e-mail: \texttt{nguyenquanbahong@gmail.com}; website: \url{https://nqbh.github.io}.}}
\date{\today}

\begin{document}
\maketitle
\tableofcontents

%------------------------------------------------------------------------------%

\section{2nd-Order Function -- Hàm Số $y = ax^2,a\ne0$}
\fbox{1} Hàm số bậc 2 $y = f(x) = ax^2$, $a\ne0$ có tập xác định TXĐ: $D_f = \mathbb{R}$. Nếu $a > 0$, hàm số $y = ax^2$ nghịch biến khi $x < 0$, đồng biến khi $x > 0$. Nếu $a < 0$, hàm số $y = ax^2$ nghịch biến khi $x > 0$, đồng biến khi $x < 0$. \fbox{2} Đồ thị hàm số $y = ax^2,a\ne0$ là 1 parabol đi qua gốc tọa độ $O$, nhận trục $Oy$ là trục đối xứng, $O$ là đỉnh của parabol. Nếu $a > 0$, đồ thị nằm phía trên trục hoành, $O$ là điểm thấp nhất của đồ thị. $\min_{x\in\mathbb{R}} y = 0\Leftrightarrow x = 0$. Nếu $a < 0$, đồ thị nằm phía dưới trục hoành, $O$ là điểm cao nhất của đồ thị. $\max_{x\in\mathbb{R}} y = 0\Leftrightarrow x = 0$.\\

\noindent\cite[Chap. VII, \S1, pp. 46--51]{SGK_Toan_9_Canh_Dieu_tap_2}: HD1. LT1. LT2. HD2. LT3. HD3. 1. 2. 3. 4. 5.

\begin{baitoan}[\cite{Binh_boi_duong_Toan_9_tap_2}, VD1, p. 35]
	Cho hàm số $y = -(n^2 + 4n + 5)x^2$. (a) Chứng minh hàm số nghịch biến với $x > 0$ \& đồng biến với $x < 0$. (b) Biết khi $x = \pm2$ thì $y = -24$, tìm $n$. (c) Mở rộng.
\end{baitoan}

\begin{baitoan}[\cite{Binh_boi_duong_Toan_9_tap_2}, VD2, p. 36]
	Cho hàm số $y = 1.5x^2$. (a) Lập bảng tính giá trị của $y$ ứng với giá trị của $x$ lần lượt bằng $-2,-1,0,1,2$. (b) Vẽ đồ thị hàm số. (c) Trong các điểm $A(3,13.5),B(-3,-13.5),C\left(-\dfrac{5}{2},\dfrac{75}{8}\right),D(\sqrt{3},-4.5),E(\sqrt{2},3)$, điểm nào thuộc đồ thị?
\end{baitoan}

\begin{baitoan}[\cite{Binh_boi_duong_Toan_9_tap_2}, VD3, p. 36]
	Cho 2 hàm số $y = \dfrac{1}{2}x^2,y = -\dfrac{1}{2}x + 3$. (a) Vẽ đồ thị 2 hàm số trên cùng 1 mặt phẳng tọa độ. (b) Tìm tọa độ các giao điểm của 2 đồ thị.
\end{baitoan}

\begin{baitoan}[\cite{Binh_boi_duong_Toan_9_tap_2}, VD4, p. 37]
	Cho hàm số $y = (\sqrt{m - 3} - 1)x^2$. (a) Tìm $m\in\mathbb{R}$ để hàm số nghịch biến với $x > 0$ \& đồng biến với $x < 0$. (b) Biết đồ thị hàm số đi qua điểm $A(-3,18)$. (c) Vẽ đồ thị hàm số với $m = 7$.
\end{baitoan}

\begin{dinhnghia}[Even{\tt/}odd function -- Hàm số chẵn{\tt/}lẻ]
	Hàm số $f:D\to\mathbb{R}$ là {\rm hàm số chẵn} nếu $f(x) = f(-x)$, $\forall x\in D$, là là {\rm hàm số lẻ} nếu $f(x) = -f(-x)$ hay $f(x) + f(-x) = 0$, $\forall x\in D$.
\end{dinhnghia}

\begin{baitoan}[\cite{Binh_boi_duong_Toan_9_tap_2}, VD5, p. 38]
	Cho hàm số $y = f(x) = 3(m^2 - 1)x^2$ với $m\ne\pm1$. (a) Chứng minh $y$ là hàm số chẵn. (b) Tìm $a\in\mathbb{R}$ để $f(a - 1) = 27(m^2 - 1)$. (c) Tìm $m\in\mathbb{R}$ để hàm số đồng biến khi $x > 0$ \& nghịch biến khi $x < 0$.
\end{baitoan}

\begin{baitoan}[\cite{Binh_boi_duong_Toan_9_tap_2}, VD6, p. 38]
	Cho 2 hàm số $y = -\dfrac{m}{4}x^2,y = -\dfrac{3x - m}{2}$ với $m\in\mathbb{R}$. (a) Xét tính chất biến thiên \& tìm {\rm GTLN, GTNN} của hàm số thứ nhất. (a) Vẽ đồ thị $(P)$ của hàm số $y = -\dfrac{m}{4}x^2$ \& đồ thị $(d)$ của hàm số $y = -\dfrac{3x - m}{2}$ với $m = 2$ trên cùng 1 mặt phẳng tọa độ. (c) Dùng đồ thị của 2 hàm số trên với $m = 2$ để giải phương trình $x^2 - 3x + 2 = 0$ \& kiểm tra lại bằng tính toán.
\end{baitoan}

\begin{baitoan}[\cite{Binh_boi_duong_Toan_9_tap_2}, 4.1., p. 39]
	Cho 2 hàm số $y = \pm\dfrac{3}{4}x^2$. Tính các giá trị tương ứng của $y$ tại các giá trị $x = 0,\pm1,\pm2,\pm3$. Nhận xét.
\end{baitoan}

\begin{baitoan}[\cite{Binh_boi_duong_Toan_9_tap_2}, 4.2., p. 39]
	Thả 1 vật nặng hình cầu lăn từ trên đỉnh dốc xuống chân dốc dài {\rm50 m}. Quan hệ giữa quãng đường $y$ {\rm m} \& thời gian lăn $x$ {\rm s} được thể hiện bởi công thức $y = (a - 1)x^2$. (a) Biết hết giấy thứ 4, vật nặng lăn được {\rm8 m}. Tìm $a$. (b) Hỏi hết giây thứ $2,5,8$ thì vật nặng đã lăn được bao nhiêu {\rm m}? (c) Khi vật nặng còn cách chân dốc {\rm32 m} thì nó đã lăn trong thời gian bao lâu? (d) Sao bao lâu thì vật nặng lăn xuống đến chân dốc? (e) Vẽ đồ thị chuyển động của vật nặng với 1 đơn vị trên trục tung ứng với {\rm5 m} \& 1 đơn vị trên trục hoành ứng với {\rm1 s}.
\end{baitoan}

\begin{baitoan}[\cite{Binh_boi_duong_Toan_9_tap_2}, 4.3., pp. 39--40]
	Trên mặt phẳng tọa độ có 1 điểm $A(2,-6)$ thuộc đồ thì hàm số $y = ax^2$. Biết $a = -2m + 2.5$. (a) Tìm $a,m$. (b) Vẽ đồ thị $(P_1)$ của hàm số $y = mx^2$ \& đồ thị $(P_2)$ của hàm số $y = ax^2$ với $a,m$ tìm được trên cùng 1 mặt phẳng tọa độ. (c) Điểm $B(8,-96),D(-12,288)$ thuộc đồ thị hàm số nào? (d) Biết điểm $E(h,18)$ nằm trên $(P_1)$ \& điểm $F(-9,n)$ nằm trên $(P_2)$. Tìm $h,n$.
\end{baitoan}

\begin{baitoan}[\cite{Binh_boi_duong_Toan_9_tap_2}, 4.4., p. 40]
	Cho hàm số $y = (m^2 - 2m + 9)x^2$. (a) Xét tính biến thiên của hàm số. (b) Biết khi $x = \pm2$ thì $y = 96$, tìm $m$.
\end{baitoan}

\begin{baitoan}[\cite{Binh_boi_duong_Toan_9_tap_2}, 4.5., p. 40]
	Cho hàm số $y = (\sqrt{2m - 5} - 3)x^2$. Tìm $m\in\mathbb{R}$ để: (a) Hàm số nghịch biến với $x > 0$. (b) Đồ thị hàm số đi qua điểm $A(4,-32)$.
\end{baitoan}

\begin{baitoan}[\cite{Binh_boi_duong_Toan_9_tap_2}, 4.6., p. 40]
	Cho hàm số $y = f(x) = -0.75x^2$. Tìm giá trị của $m,n$ để: (a) $f(m)\le-1.5$. (b) $f(n - 1)\ge-6.75$.
\end{baitoan}

\begin{baitoan}[\cite{Binh_boi_duong_Toan_9_tap_2}, 4.7., p. 40]
	Cho 2 hàm số $y = f(x) = x^2,y = g(x) = ax + 3$. (a) Tìm $a\in\mathbb{R}$ biết $f(a + 2) - f(a - 5) = 7$. (b) Vẽ 2 đồ thị 2 hàm số trên cùng 1 mặt phẳng tọa độ. (c) Tìm giao điểm 2 đồ thị, kiểm tra lại bằng tính toán.
\end{baitoan}

\begin{baitoan}[\cite{Binh_boi_duong_Toan_9_tap_2}, 4.8., p. 40]
	Cho hàm số $y = f(x) = x^2$ \& $a,b,c$ là 3 giá trị phân biệt của $x$. Biết $f(a) + b = f(b) + c = f(c) + a$. Tính giá trị biểu thức $A = (a + b - 1)(b + c - 1)(c + a - 1)$.
\end{baitoan}

\begin{baitoan}[\cite{Binh_boi_duong_Toan_9_tap_2}, p. 41, bài toán thả vật nặng rơi từ trên cao]
	Galilei là người phát hiện ra quãng đường chuyển động của vật rơi tự do tỷ lệ thuận với bình phương của thời gian. Quan hệ giữa quãng đường chuyển động $y$ {\rm m} \& thời gian chuyển động $x$ {\rm s} được biểu diễn gần đúng bởi công thức $y = 5x^2$. Thả 1 vật nặng từ độ cao {\rm55 m} trên tháp nghiêng Pisa xuống đất (sức cản không khí không đáng kể), (a) Cho biết sau {\rm1 s, 1.5 s, 2 s, 2.5 s, 3 s} thì vật nặng còn cách đất bao nhiêu {\rm m}? (b) Khi vật nặng còn cách đất {\rm25 m} thì nó đã rơi được thời gian bao lâu? (c) Sau bao lâu thì vật nặng chạm đất?
\end{baitoan}

\begin{baitoan}[\cite{Binh_boi_duong_Toan_9_tap_2}, p. 42]
	Trong mặt phẳng tọa độ $(Oxy)$ 1 điểm A chạy trên parabol $(P):y = ax^2,a\ne0$ thì trung điểm $I$ của $OM$ chạy trên đường nào?
\end{baitoan}

\begin{baitoan}[\cite{Tuyen_Toan_9_old}, VD33, p. 70]
	Cho parabol $(P):y = ax^2$. (a) Tìm $a\in\mathbb{R}$ để $(P)$ đi qua điểm $M(-4,4)$. Vẽ $(P)$ ứng với giá trị vừa tìm được của $a$. (b) Lấy điểm $A(0,3)$ \& điểm B thuộc đồ thị vừa vẽ. Tìm độ dài nhỏ nhất của $AB$.
\end{baitoan}

\begin{baitoan}[\cite{Tuyen_Toan_9_old}, 184., p. 71]
	Cho hàm số $y = f(x) = ax^2$. Chứng minh: (a) $f(3) + f(4) = f(5)$. (b) $f(x) + f(y) = f(z)$ với $x,y,z$ là độ dài 3 cạnh 1 tam giác vuông, $z$ là độ dài cạnh huyền.
\end{baitoan}

\begin{baitoan}[\cite{Tuyen_Toan_9_old}, 185., p. 71]
	Cho hàm số $y = 3x^2$. Tìm {\rm GTNN, GTLN} của $y$ nếu $m\le x\le n$ với $mm < 0 < n$.
\end{baitoan}

\begin{baitoan}[\cite{Tuyen_Toan_9_old}, 186., p. 71]
	Cho hàm số $y = f(x) = (m^2 - m + 1)x^2$. Chứng minh: (a) Hàm số $y = f(x)$ luôn đồng biến $\forall m,x\in\mathbb{R},x > 0$. (b) $f(\sqrt{3} - \sqrt{2}) < f(\sqrt{2} - 1)$.
\end{baitoan}

\begin{baitoan}[\cite{Tuyen_Toan_9_old}, 187., p. 71]
	Cho hàm số $y = f(x) = (\sqrt{m - 5} - 2)x^2$. Tìm $m\in\mathbb{R}$ để: (a) Hàm số đồng biến với $x < 0$. (b) Đồ thị hàm số đi qua điểm $A(-2,12)$.
\end{baitoan}

\begin{baitoan}[\cite{Tuyen_Toan_9_old}, 188., p. 71]
	Cho hàm số $y = (\lfloor m\rfloor - 3)x^2$. Tìm $m\in\mathbb{R}$ để hàm số: (a) nghịch biến với $x < 0$. (b) Có đồ thị hàm số đi qua điểm $A(-2,m + 3)$.
\end{baitoan}

\begin{baitoan}[\cite{Tuyen_Toan_9_old}, 189., p. 72]
	Cho hàm số $y = f(x) = -\frac{2}{3}x^2$. Tìm $m\in\mathbb{R}$ để: (a) $f(m)\ge-6$. (b) $f(m + 2) - f(m - 1) = 6$.
\end{baitoan}

\begin{baitoan}[\cite{Tuyen_Toan_9_old}, 190., p. 72]
	Vẽ đồ thị hàm số $y = \dfrac{1}{4}x|x| + \dfrac{x^3}{4|x|}$.
\end{baitoan}

\begin{baitoan}[\cite{Tuyen_Toan_9_old}, 191., p. 72]
	Cho parrabol $(P):y = -x^2$. Đường thẳng $y = m$ cắt $(P)$ tại 2 điểm $A,B$. Tìm $m\in\mathbb{R}$ để $\Delta AOB$ đều. Tính diện tích tam giác đều đó.
\end{baitoan}

\begin{baitoan}[\cite{Tuyen_Toan_9_old}, 192., p. 72]
	Dùng đồ thị để giải phương trình \& bất phương trình: (a) $x^2 - x - 2 = 0$. (b) $x^2 - x - 2 < 0$.
\end{baitoan}

\begin{baitoan}[\cite{Binh_Toan_9_tap_2}, VD74, p. 18]
	(a) Cho parabol $y = \frac{1}{4}x^2$, điểm $A(0,1)$ \& đường thẳng $d:y = -1$. Gọi M là 1 điểm bất kỳ thuộc parabol. Chứng minh MA bằng khoảng cách MH từ điểm M đến $d$. (b) Cho điểm $A(0,a)$, $d:y = -a$. Chứng minh quỹ tích của điểm $M(x,y)$ sao cho khoảng cách MH từ M tới $d$ bằng MA là 1 parabol.
\end{baitoan}
\noindent\cite[235., p. 19, 236., p. 20]{Binh_Toan_9_tap_2}.

\begin{baitoan}[\cite{Binh_Toan_9_tap_2}, 237., p. 20]
	(a) Tìm hệ số $a$ của parabol $y = ax^2$, biết parabol đi qua điểm $A(-2,-2)$. (b) Tìm tọa độ của điểm M thuộc parabol này, biết khoảng cách từ M đến trục hoành gấp đôi khoảng cách từ M đến trục tung.
\end{baitoan}

\begin{baitoan}[\cite{Binh_Toan_9_tap_2}, 238., p. 20]
	Vẽ đồ thị hàm số $y = \frac{1}{3}x|x|$.
\end{baitoan}

\begin{baitoan}[\cite{Binh_Toan_9_tap_2}, 239., p. 20]
	(a) Vẽ đồ thị hàm số $y = -\frac{1}{2}x^2$. (b) Gọi C là 1 điểm tùy ý nằm trên parabol $y = -\frac{1}{2}x^2$. Gọi K là trung điểm OC. Khi điểm C di chuyển trên parabol đó thì điểm K di chuyển trên đường nào?
\end{baitoan}

\begin{baitoan}[\cite{TLCT_THCS_Toan_9_dai_so}, VD12.1, p. 65]
	Trong mặt phẳng tọa độ $(Oxy)$ cho đường thẳng $(d)$: $y = -1$ \& điểm $F(0,1)$. Tìm tập hợp tất cả những điểm $I$ sao cho khoảng cách từ $I$ đến $(d)$ bằng $IF$.
\end{baitoan}

\begin{proof}[Giải]
	Giả sử điểm $I(x,y)$. Khi đó khoảng cách từ $I$ đến $(d)$ bằng $|y + 1|$ \& $IF = \sqrt{x^2 + (y - 1)^2}$, nên $d_{I,(d)} = IF\Leftrightarrow|y + 1| = \sqrt{x^2 + (y - 1)^2}\Leftrightarrow(y + 1)^2 = x^2 + (y - 1)^2\Leftrightarrow y = \frac{1}{4}x^2$. Suy ra tập hợp tất cả những điểm $I$ sao cho khoảng cách từ $I$ đến $(d)$ bằng $IF$ là đường parabol $(P_1)$: $y = \frac{1}{4}x^2$, i.e., $\{I\in\mathbb{R}^2|d_{I,(d)} = IF\} = \{(x,y)\in\mathbb{R}^2|y = \frac{1}{4}x^2\}$.
\end{proof}

\begin{baitoan}
	Trong mặt phẳng tọa độ $(Oxy)$ cho đường thẳng $(d)$: $y = ax + b$ \& điểm $F(c,d)$. Tìm tập hợp tất cả những điểm $I$ sao cho khoảng cách từ $I$ đến $(d)$ bằng $IF$.
\end{baitoan}

\begin{baitoan}[\cite{TLCT_THCS_Toan_9_dai_so}, VD12.2, p. 66]
	Xác định điểm $M$ thuộc parabol $(P)$: $y = x^2$ sao cho độ dài đoạn $IM$ là nhỏ nhất, trong đó $I(0,1)$.
\end{baitoan}

\begin{baitoan}[\cite{TLCT_THCS_Toan_9_dai_so}, VD12.3, p. 66]
	Trong mặt phẳng $(Oxy)$, giả sử điểm $A$ chạy trên parabol $(P)$: $y = x^2$. Tìm tập hợp trung điểm $I$ của đoạn thẳng $OA$.
\end{baitoan}

\begin{baitoan}[\cite{TLCT_THCS_Toan_9_dai_so}, VD12.4, p. 66]
	Trong mặt phẳng $(Oxy)$, giả sử 2 điểm $A,B$ chạy trên parabol $(P)$: $y = x^2$ sao cho $A,B\ne O(0,0)$ \& $OA\bot OB$. Giả sử $I$ là trung điểm của đoạn thẳng $AB$. (a) Chứng minh tọa độ của điểm $I$ thỏa mãn phương trình $y = 2x^2 + 1$. (b) Chứng minh đường thẳng $(AB)$ luôn đi qua 1 điểm cố định. (c) Xác định tọa độ của các điểm $A,B$ sao cho độ dài $AB$ nhỏ nhất.
\end{baitoan}

\begin{baitoan}[\cite{TLCT_THCS_Toan_9_dai_so}, VD12.5, p. 67]
	Trên parabol $(P)$: $y = x^2$ ta lấy 2 điểm $A(-1,1)$ \& $B(3,9)$. Xác định điểm $C$ thuộc cung nhỏ $AB$ của $(P)$ sao cho diện tích $\Delta ABC$ lớn nhất.
\end{baitoan}

\begin{baitoan}[\cite{TLCT_THCS_Toan_9_dai_so}, VD12.6, p. 68]
	Trên parabol $(P):y = x^2$ lấy 6 điểm phân biệt $A_1,A_2,A_3,A_4,A_5,A_6$. Chứng minh nếu $A_1A_2\parallel A_4A_5,A_2A_3\parallel A_5A_6$ thì $A_3A_4\parallel A_6A_1$.
\end{baitoan}

\begin{baitoan}[\cite{TLCT_THCS_Toan_9_dai_so}, VD12.7, p. 68]
	Trên parabol $(P):y = x^2$ lấy 6 điểm phân biệt $A_i(a_i,a_i^2),i = 1,2,3,4,5,6$. Giả sử $A_1A_2\bot A_4A_5,A_2A_3\bot A_5A_6$. Chứng minh $A_3A_4,A_1A_6$ không thể vuông góc với nhau nếu $(a_1 + a_2)(a_2 + a_3)(a_3 + a_4)(a_4 + a_5)(a_5 + a_6)(a_6 + a_1)\ne-1$.
\end{baitoan}

\begin{baitoan}[\cite{TLCT_THCS_Toan_9_dai_so}, VD12.8, p. 69]
	Trên parabol $(P):y = x^2$ lấy 3 điểm phân biệt $A(a,a^2),B(b,b^2),C(c,c^2)$ thỏa mãn $a^2 - b = b^2 - c = c^2 - a$. Tính $A =  (a + b + 1)(b + c + 1)(c + a + 1)$.
\end{baitoan}

\begin{baitoan}[\cite{TLCT_THCS_Toan_9_dai_so}, 12.1., p. 69]
	Trên parabol $(P):y = x^2$. Tính khoảng cách giữa 2 điểm A,B nằm trên parabol có hoành độ lần lượt bằng: (a) $-1,2$. (b) $a,b\in\mathbb{R}$.
\end{baitoan}

\begin{baitoan}[\cite{TLCT_THCS_Toan_9_dai_so}, 12.2., p. 69]
	Trong mặt phẳng $(Oxy)$ cho đường thẳng $(d):y = kx + 1$ \& parabol $(P):y = x^2$. Chứng minh: (a) Đường thẳng $(d)$ luôn cắt parabol $(P)$ tại 2 điểm phân biệt A,B. (b) Có đúng 1 điểm M thuộc đường thẳng $(d'):y = -1$ để $MA\bot MB$ \& đường thẳng MA tiếp xúc với $(P)$.
\end{baitoan}

\begin{baitoan}[\cite{TLCT_THCS_Toan_9_dai_so}, 12.3., p. 69]
	Trong mặt phẳng $(Oxy)$ cho đường thẳng $(d):y = kx + \dfrac{1}{2}$ \& parabol $(P):x = \dfrac{1}{2}x^2$. Giả sử đường thẳng $(d)$ cắt parabol $(P)$ tại 2 điểm phân biệt A,B. Chứng minh tọa độ trung điểm của đoạn thẳng AB thỏa mãn phương trình $y = x^2 + \dfrac{1}{2}$.
\end{baitoan}

\begin{baitoan}[\cite{TLCT_THCS_Toan_9_dai_so}, 12.4., p. 69]
	Trong mặt phẳng $(Oxy)$ cho parabol $(P): y = x^2$ \& 2 điểm $I(0,1),J(1,0)$. Tìm 2 điểm $M,N\in(P)$ sao cho IM,JN ngắn nhất.
\end{baitoan}

\begin{baitoan}[\cite{TLCT_THCS_Toan_9_dai_so}, 12.5., p. 69]
	Trong mặt phẳng $(Oxy)$ cho parabol $(P): y = x^2$. Có thể tìm được hay không 3 điểm $A,B,C\in(P)$ sao cho $\Delta ABC$ đều?
\end{baitoan}

\begin{baitoan}[\cite{TLCT_THCS_Toan_9_dai_so}, 12.6., p. 70]
	Trong mặt phẳng $(Oxy)$ cho parabol $(P): y = x^2$. Có thể tìm được hay không 4 điểm $A,B,C\in(P)$ sao cho tứ giác ABCD là 1 hình vuông?
\end{baitoan}

\begin{baitoan}[\cite{TLCT_THCS_Toan_9_dai_so}, 12.7., p. 70]
	Trên parabol $(P): y = x^2$, lấy 4 điểm phân biệt $A(a,a^2),B(b,b^2),C(c,c^2),D(d,d^2)$ thỏa mãn $a^2 - b = b^2 - c = c^2 - d = d^2 - a$. Chứng minh
	\begin{align*}
		(a + b + 1)(b + c + 1)(c + d + 1)(d + a + 1) = \dfrac{(a - c)^2(b - d)^2}{(a - b)(b - c)(c - d)(d - a)}.
	\end{align*}
\end{baitoan}

%------------------------------------------------------------------------------%

\section{Quadratic Equation -- Phương Trình Bậc 2 1 Ẩn $ax^2 + bx + c = 0$, $a\ne0$}
\fbox{1} Phương trình bậc 2 1 ẩn $ax^2 + bx + c = 0$ (1), $a,b,c\in\mathbb{R},a\ne0$. Biệt số $\Delta\coloneqq b^2 - 4ac$. Biệt số rút gọn $\Delta'\coloneqq b'^2 - ac$ với $b = 2b'$. \fbox{2} Công thức nghiệm: $\Delta > 0$, (1) có 2 nghiệm thực phân biệt $x_{1,2} = \dfrac{-b\pm\sqrt{\Delta}}{2a}\in\mathbb{R}$. $\Delta = 0$, (1) có nghiệm kép $x_1 = x_2 = -\dfrac{b}{2a}$. $\Delta < 0$, (1) vô nghiệm thực, nhưng có 2 nghiệm phức $x_{1,2} = \dfrac{-b\pm i\sqrt{-\Delta}}{2a}\in\mathbb{C}\backslash\mathbb{R}$. \fbox{3} Công thức nghiệm thu gọn: $\Delta' > 0$, (1) có 2 nghiệm thực phân biệt $x_{1,2} = \dfrac{-b'\pm\sqrt{\Delta'}}{a}\in\mathbb{R}$. $\Delta = 0$, (1) có nghiệm kép $x_1 = x_2 = -\dfrac{b'}{a}$. $\Delta < 0$, (1) vô nghiệm thực, nhưng có 2 nghiệm phức $x_{1,2} = \dfrac{-b'\pm i\sqrt{-\Delta'}}{a}\in\mathbb{C}\backslash\mathbb{R}$. \fbox{4} Phương trình bậc 2 $ax^2 + bx + c = 0$ với $a,c$ trái dấu, i.e., $ac < 0$ bao giờ cùng có 2 nghiệm thực phân biệt vì $\Delta = b^2 - 4ac\ge-4ac > 0$.\\

\noindent\cite[Chap. VII, \S2, pp. 52--60]{SGK_Toan_9_Canh_Dieu_tap_2}: HD1. LT1. HD2. LT2. HD3. LT3. HD4. LT4. LT5. LT6. 1. 2. 3. 4. 5. 6.

\begin{baitoan}[{\sf Program}: Solve quadratic equation]
	Viết chương trình {\sf Pascal, Python, C{\tt/}C++} để giải phương trình bậc 2 1 ẩn $ax^2 + bx + c = 0$.
	\begin{itemize}
		\item {\sf Input}: 3 hệ số $a,b,c\in\mathbb{R}$ được nhập từ bàn phím.
		\item {\sf Output}: Số nghiệm của phương trình \& liệt kê các nghiệm đó, {\rm GTNN, GTLN}, các khoảng đồng biến, nghịch biến.
	\end{itemize}
\end{baitoan}

\begin{baitoan}[\cite{Binh_boi_duong_Toan_9_tap_2}, VD1, p. 45]
	Giải phương trình: (a) $4x^2 - 25 = 0$. (b) $3x^2 + 8 = 0$.
\end{baitoan}

\begin{baitoan}
	Biện luận theo tham số $a,b\in\mathbb{R}$ để giải phương trình: (a) $ax^2 + b = 0$. (b) $ax^2 + bx = 0$.
\end{baitoan}

\begin{baitoan}[\cite{Binh_boi_duong_Toan_9_tap_2}, VD2, p. 45]
	Giải phương trình bằng cách cộng vào 2 vế mỗi phương trình cùng 1 số hoặc biểu thức thích hợp để được 1 phương trình mà vế trái là 1 bình phương, vế phải là 1 số: (a) $x^2 - 4x + 2$. (b) $2x^2 + 5x = 1$. (c) $ax^2 + bx + c$ với $a,b,c\in\mathbb{R},a\ne0$.
\end{baitoan}

\begin{baitoan}[\cite{Binh_boi_duong_Toan_9_tap_2}, VD3, p. 46]
	Dùng công thức nghiệm hoặc công thức nghiệm thu gọn để giải phương trình: (a) $6x^2 - x - 7 = 0$. (b) $6x^2 - x + 7 = 0$. (c) $2x^2 - 2\sqrt{10}x - \sqrt{5} = 0$.
\end{baitoan}

\begin{baitoan}[\cite{Binh_boi_duong_Toan_9_tap_2}, VD4, p. 46]
	Cho phương trình bậc 2: $x^2 - 2(m + 2)x + m^2 - 5 = 0$. (a) Giải phương trình với $m = 2$. (b) Biện luận theo $m\in\mathbb{R}$ số nghiệm của phương trình. (c) Tính hiệu của nghiệm lớn \& nghiệm nhỏ trong trường hợp phương trình có 2 nghiệm.
\end{baitoan}

\begin{baitoan}[\cite{Binh_boi_duong_Toan_9_tap_2}, VD5, p. 47]
	Cho $a,b,c$ là độ dài 3 cạnh 1 tam giác. Chứng minh phương trình $(a^2 + b^2 - c^2)x^2 + 4abx + a^2 + b^2 - c^2 = 0$ luôn có 2 nghiệm phân biệt.
\end{baitoan}

\begin{baitoan}[\cite{Binh_boi_duong_Toan_9_tap_2}, 5.1., p. 47]
	Giải phương trình: (a) $9x^2 - 27 = 0$. (b) $5x^2 + 19 = 0$. (c) $1.2x^2 - 6x = 0$. (d) $x(x - 2) + 4x - 8 = 0$.
\end{baitoan}

\begin{baitoan}[\cite{Binh_boi_duong_Toan_9_tap_2}, 5.2., p. 47]
	Cho phương trình bậc 2 $x^2 + bx + 8 = 0$. (a) Tìm hệ số $b\in\mathbb{R}$ để phương trình có 1 nghiệm là $8$. (b) Giải phương trình bằng cách biến đổi thành phương trình tích.
\end{baitoan}

\begin{baitoan}[\cite{Binh_boi_duong_Toan_9_tap_2}, 5.3., p. 47]
	Giải phương trình bằng cách biến đổi vế trái thành 1 bình phương, vế phải là 1 số: (a) $x^2 - 6x + 4 = 0$. (b) $(x - 3)(x + 1) = 12$. (c) $4x^2 + 4x = 1$. (d) $-5x^2 + 6 = 10x$.
\end{baitoan}

\begin{baitoan}[\cite{Binh_boi_duong_Toan_9_tap_2}, 5.4., p. 47]
	Dùng công thức nghiệm hoặc công thức nghiệm thu gọn để giải phương trình bậc 2: (a) $5x^2 + 3x - 8 = 0$. (b) $5x^2 - 3x + 8 = 0$. (c) $3y^2 - 2\sqrt{15}y - 3\sqrt{5} = 0$.
\end{baitoan}

\begin{baitoan}[\cite{Binh_boi_duong_Toan_9_tap_2}, 5.5., p. 47]
	Cho phương trình bậc 2 $2x^2 - 2mx + m + 4 = 0$. Tìm $m\in\mathbb{R}$ để phương trình có nghiệm kép rồi tìm nghiệm kép đó.
\end{baitoan}

\begin{baitoan}[\cite{Binh_boi_duong_Toan_9_tap_2}, 5.6., p. 48]
	Cho phương trình bậc 2 $2x^2 + 2mx + m - 2018 = 0$. (a) Chứng minh phương trình luôn có 2 nghiệm $\forall m\in\mathbb{R}$. (b) Giải phương trình với $m = 672$.
\end{baitoan}

\begin{baitoan}[\cite{Binh_boi_duong_Toan_9_tap_2}, 5.7., p. 48]
	Chứng minh phương trình $2x^2 + 2(a + b + c)x + ab + bc + ca = 0$ luôn có nghiệm $\forall a,b,c\in\mathbb{R}$.
\end{baitoan}

\begin{baitoan}[\cite{Binh_boi_duong_Toan_9_tap_2}, 5.8., p. 48]
	Cho $a,b,c$ là độ dài 3 cạnh 1 tam giác. Chứng minh: (a) Phương trình $b^2x^2 + (b^2 + c^2 - a^2)x + c^2 = 0$ vô nghiệm. (b) Phương trình $f(x) + g(x) = 0$ có nghiệm kép khi \& chỉ khi tam giác là đều, với $f(x) = (x - 2a)(x - b) + (x - 2b)(x - c) + (x - 2c)(x - a),g(x) = (a + b + c)x - (ab + bc + ca)$.
\end{baitoan}

\begin{baitoan}[\cite{Binh_boi_duong_Toan_9_tap_2}, VD, p. 49]
	Tìm {\rm GTNN, GTLN} của biểu thức: (a) $A = \dfrac{4x + 3}{x^2 + 1}$. (b) $B = \dfrac{x^2 + x + 1}{x + 1}$.
\end{baitoan}

\begin{baitoan}[\cite{Tuyen_Toan_9_old}, VD34, p. 73]
	Cho phương trình $mx^2 - (2m + 1)x + m + 1 = 0$. (a) Giải phương trình với $m = -\frac{3}{5}$. (b) Chứng minh phương trình luôn có nghiệm $\forall m\in\mathbb{R}$. (c) Tìm $m\in\mathbb{R}$ để phương trình có 1 nghiệm lớn hơn $2$.
\end{baitoan}

\begin{baitoan}[\cite{Tuyen_Toan_9_old}, VD35, p. 75]
	Tìm {\rm GTNN} của biểu thức $A = 5x^2 - 4x + 1$.
\end{baitoan}

\begin{baitoan}
	Tìm {\rm GTNN, GTLN} của hàm số $y = ax^2 + bx + c$ với $a,b,c\in\mathbb{R},a\ne0$.
\end{baitoan}

\begin{baitoan}[\cite{Tuyen_Toan_9_old}, 193., p. 75]
	Chứng minh định lý về dấu của tam thức bậc 2.
\end{baitoan}

\begin{baitoan}[\cite{Tuyen_Toan_9_old}, 194., p. 75]
	Tính hiệu giữa nghiệm lớn \& nghiệm nhỏ của phương trình $(9 - 4\sqrt{5})x^2 + (\sqrt{5} - 2)x - 6 = 0$.
\end{baitoan}

\begin{baitoan}[\cite{Tuyen_Toan_9_old}, 195., p. 76]
	Cho biết tổng của $n$ số nguyên dương đầu tiên nhỏ hơn tổng của $2n$ số nguyên dương đầu tiên là $155$. Tính tổng của $3n$ số nguyên dương đầu tiên.
\end{baitoan}

\begin{baitoan}[\cite{Tuyen_Toan_9_old}, 196., p. 76]
	Cho 2 phương trình $x^2 + 5x + c 0,x^2 - 5x - c = 0$. Biết 2 phương trình này có 1 nghiệm đối nhau, chứngm inh nghiệm còn lại của 2 phương trình này cũng đối nhau.
\end{baitoan}

\begin{baitoan}[\cite{Tuyen_Toan_9_old}, 197., p. 76]
	Tìm nghiệm nguyên dương của hệ phương trình
	\begin{equation*}
		\left\{\begin{split}
			xy + yz &= 36,\\
			xz + yz &= 19.
		\end{split}\right.
	\end{equation*}
\end{baitoan}

\begin{baitoan}[\cite{Tuyen_Toan_9_old}, 198., p. 76]
	Cho phương trình $4x^2 + 4mx + m + 6 = 0$. Tìm $m\in\mathbb{R}$ để phương trình có nghiệm kép.
\end{baitoan}

\begin{baitoan}[\cite{Tuyen_Toan_9_old}, 199., p. 76]
	Cho phương trình $(x - a)(x - b) + (x - b)(x - c) + (x - c)(x - a) = 0$. Tìm điều kiện của $a,b,c\in\mathbb{R}$ để phương trình có nghiệm kép.
\end{baitoan}

\begin{baitoan}[\cite{Tuyen_Toan_9_old}, 200., p. 76]
	Tìm $m\in\mathbb{Z}$ nhỏ nhất sao cho phương trình $x^2 - 4x(mx - 5) - 8 = 0$ vô nghiệm.
\end{baitoan}

\begin{baitoan}[\cite{Tuyen_Toan_9_old}, 201., p. 76]
	Cho phương trình $x^2 - px - 228p = 0$ với $p$ là số nguyên tố. Tìm $p$ để phương trình có 2 nghiệm nguyên.
\end{baitoan}

\begin{baitoan}[\cite{Tuyen_Toan_9_old}, 202., p. 76]
	Cho phương trình $ax^2 + bx + c = 0$ với $a > 0$. Chứng minh nếu $b > a + c$ thì phương trình luôn có 2 nghiệm phân biệt.
\end{baitoan}

\begin{baitoan}[\cite{Tuyen_Toan_9_old}, 203., p. 76]
	Chứng minh phương trình $x^2 - 2mx + 2010\cdot2011 = 0$ không có nghiệm nguyên $\forall m\in\mathbb{Z}$.
\end{baitoan}

\begin{baitoan}[\cite{Tuyen_Toan_9_old}, 204., p. 77]
	Cho $a,b,c$ là độ dài 3 cạnh của 1 tam giác. Chứng minh phương trình vô nghiệm: (a) $x^2 + (a + b + c)x + ab + bc + ca = 0$. (b) $a^2x^2 + (a^2 + b^2 - c^2)x + b^2 = 0$.
\end{baitoan}

\begin{baitoan}[\cite{Tuyen_Toan_9_old}, 205., p. 77]
	Cho 2 phương trình $x^2 + bx + c = 0,x^2 + cx + b = 0$ với $b,c\in\mathbb{R},\dfrac{1}{b} + \dfrac{1}{c} = \dfrac{1}{2}$. Chứng minh có ít nhất 1 tron 2 phương trình có nghiệm.
\end{baitoan}

\begin{baitoan}[\cite{Tuyen_Toan_9_old}, 206., p. 77]
	Cho 3 phương trình $ax^2 + 2bx + c = 0,bx^2 + 2cx + a = 0,cx^2 + 2ax + b = 0$ với $a,b,c\in\mathbb{R}^\star$. Chứng minh có ít nhất 1 trong 3 phương trình có nghiệm.
\end{baitoan}

\begin{baitoan}[\cite{Tuyen_Toan_9_old}, 207., p. 77]
	Tìm $m\in\mathbb{R}$ để 2 phương trình $x^2 + mx + 1 = 0,x^2 - (m + 1)x - 2m = 0$ có ít nhất 1 nghiệm chung.
\end{baitoan}

\begin{baitoan}[\cite{Tuyen_Toan_9_old}, 208., p. 77]
	Tìm $m\in\mathbb{R}$ để 2 phương trình $x^2 + (2m - 1)x - 10 = 0,3x^2 + (4m - 2)x - 22 = 0$ có ít nhất 1 nghiệm chung.
\end{baitoan}

\begin{baitoan}[\cite{Tuyen_Toan_9_old}, 209., p. 77]
	Tìm $(x,y)\in\mathbb{R}^2$ thỏa phương trình $3x^2 - 6x + y - 2 = 0$ sao cho $y$ đạt {\rm GTLN}.
\end{baitoan}

\begin{baitoan}[\cite{Tuyen_Toan_9_old}, 210., p. 77]
	Tìm nghiệm nguyên của phương trình $x^2 - xy + y^2 = x - y$.
\end{baitoan}

\begin{baitoan}[\cite{Tuyen_Toan_9_old}, 211., p. 77]
	Tìm {\rm GTNN} của biểu thức $A = \dfrac{x^2 - x + 1}{x^2 - 2x + 1}$.
\end{baitoan}

\begin{baitoan}[\cite{Tuyen_Toan_9_old}, 212., p. 78]
	Tìm {\rm GTNN, GTLN} của biểu thức: (a) $A = \dfrac{4x - 3}{x^2 + 1}$. (b) $B = \dfrac{x^2 + 2x + 1}{x^2 - 2x + 3}$.
\end{baitoan}

\begin{baitoan}
	Biện luận theo các tham số để tìm {\rm GTNN, GTLN} của biểu thức: (a) $A = \dfrac{mx + n}{ax^2 + bx + c}$. (b) $B = \dfrac{a_1x^2 + b_1x + c_1}{a_2x^2 + b_2x + c_2}$.
\end{baitoan}

\begin{baitoan}[\cite{Tuyen_Toan_9_old}, 213., p. 78]
	Cho parabol $(P):y = -\frac{3}{4}x^2$, đường thẳng $(d):y = (m - 2)x + 3$. Tìm $m\in\mathbb{R}$ để $(P),(d)$ tiếp xúc nhau. Tìm tọa độ của tiếp điểm.
\end{baitoan}

\begin{baitoan}[\cite{Tuyen_Toan_9_old}, 214., p. 78]
	Trong mặt phẳng tọa độ $Oxy$ cho điểm $M(0,2)$. Cho parabol $(P):y = \dfrac{x^2}{4}$, đường thẳng $(d):ax + by = -2$. Biết $(d)$ đi qua M. (a) Chứng minh khi $a$ thay đổi thì $(d)$ luôn cắt $(P)$ tại 2 điểm phân biệt $A,B$. (b) Tìm $a\in\mathbb{R}$ để AB có độ dài ngắn nhất.
\end{baitoan}

\begin{baitoan}[\cite{Binh_boi_duong_Toan_9_tap_2}, VD, p. 49]
	Giải phương trình: (a) $x^2 + y^2 - 6x - 8y + 25 = 0$. (b) $x^2 - 10x + y - 6\sqrt{y} + 34 = 0$. (c) $2y^2x(x - 1) + y(y - 2) + 2 = 0$.
\end{baitoan}

\begin{baitoan}
	Giải \& biện luận phương trình $x^2 + y^2 + ax + by + c = 0$ với 3 tham số $a,b,c\in\mathbb{R}$.
\end{baitoan}

\begin{baitoan}[\cite{Binh_Toan_9_tap_2}, VD75, p. 20]
	Cho phương trình $(m^2 - m - 2)x^2 + 2(m + 1)x + 1 = 0$ với tham số $m$. (a) Giải phương trình khi $m = 1$. (b) Tìm các giá trị của $m\in\mathbb{R}$ để phương trình có 2 nghiệm phân biệt. (c) Tìm các giá trị của $m\in\mathbb{R}$ để tập nghiệm của phương trình chỉ có 1 phần tử.
\end{baitoan}

\begin{baitoan}[\cite{Binh_Toan_9_tap_2}, VD76, p. 21]
	Chứng minh phương trình $(a + 1)x^2 - 2(a + b)x + b - 1 = 0$ có nghiệm $\forall a,b\in\mathbb{R}$.
\end{baitoan}

\begin{baitoan}[\cite{Binh_Toan_9_tap_2}, VD77, p. 22]
	Chứng minh phương trình $x^2 - (3m^2 - 5m + 1)x - (m^2 - 4m + 5) = 0$ có nghiệm $\forall a,b\in\mathbb{R}$.
\end{baitoan}

\begin{baitoan}[\cite{Binh_Toan_9_tap_2}, VD78, p. 22]
	Cho phương trình $x^2 + mx + n = 0$ với $m,n\in\mathbb{Z}$. (a) Chứng minh nếu phương trình có nghiệm hữu tỷ thì nghiệm đó là số nguyên. (b) Tìm nghiệm hữu tỷ của phương trình với $n = 3$.
\end{baitoan}

\begin{baitoan}[\cite{Binh_Toan_9_tap_2}, VD79, p. 20]
	Tìm $n\in\mathbb{Z}$ để các nghiệm của phương trình $x^2 - (4 + n)x + 2n = 0$ là các số nguyên.
\end{baitoan}

\begin{baitoan}[\cite{Binh_Toan_9_tap_2}, VD80, p. 20]
	Tìm các giá trị của $a$ để 2 phương trình $x^2 + ax + 8 = 0,x^2 + x + a = 0$ có ít nhất 1 nghiệm chung.
\end{baitoan}

\begin{baitoan}[\cite{Binh_Toan_9_tap_2}, 240., p. 25]
	Cho phương trình $mx^2 + 6(m - 2)x + 4m - 7 = 0$. Tìm các giá trị của $m\in\mathbb{R}$ để phương trình: (a) Có nghiệm kép. (b) Có 2 nghiệm phân biệt. (c) Vô nghiệm.
\end{baitoan}

\begin{baitoan}[\cite{Binh_Toan_9_tap_2}, 241., p. 25]
	Giải phương trình với tham số $m$: (a) $x^2 - mx - 3(m + 3) = 0$. (b) $mx^2 - 4x + 4 = 0$.
\end{baitoan}

\begin{baitoan}[\cite{Binh_Toan_9_tap_2}, 242., p. 25]
	Tìm các giá trị của $m\in\mathbb{R}$ biết phương trình $x^2 + mx + 12 = 0$ có hiệu 2 nghiệm bằng $1$.
\end{baitoan}

\begin{baitoan}[\cite{Binh_Toan_9_tap_2}, 243., p. 25]
	Cho 2 số thực dương $a,b$ thỏa $a + b = 4\sqrt{ab}$. Tính tỷ số $\dfrac{a}{b}$.
\end{baitoan}

\begin{baitoan}[\cite{Binh_Toan_9_tap_2}, 244., p. 25]
	Tìm $x,y\in\mathbb{Z}$ biết $2(x^2 + 1) + y^2 = 2y(x + 1)$.
\end{baitoan}

\begin{baitoan}[\cite{Binh_Toan_9_tap_2}, 245., p. 26]
	Tìm các giá trị của $m\in\mathbb{R}$ để phương trình có nghiệm: (a) $(m^2 - m)x^2 + 2mx + 1 = 0$. (b) $(m + 1)x^2 - 2x + (m - 1) = 0$.
\end{baitoan}

\begin{baitoan}[\cite{Binh_Toan_9_tap_2}, 246., p. 26]
	Chứng minh phương trình có nghiệm $\forall a,b\in\mathbb{R}$: (a) $x(x - a) + x(x - b) + (x - a)(x - b) = 0$. (b) $x^2 + (a + b)x - 2(a^2 - ab + b^2) = 0$.
\end{baitoan}

\begin{baitoan}[\cite{Binh_Toan_9_tap_2}, 247., p. 26]
	Chứng minh phương trình có nghiệm $\forall a,b,c\in\mathbb{R}$: (a) $3x^2 - 2(a + b + c)x + (ab + bc + ca) = 0$. (b) $(x - a)(x - b) + (x - b)(x - c) + (x - c)(x - a) = 0$.
\end{baitoan}

\begin{baitoan}[\cite{Binh_Toan_9_tap_2}, 248., p. 26]
	Chứng minh nếu $a,b,c\in\mathbb{R}^\star$ thì tồn tại 1 trong 3 phương trình bậc 2 $ax^2 + 2bx + c = 0,bx^2 + 2cx + a = 0,cx^2 + 2ax + b = 0$ có nghiệm.
\end{baitoan}

\begin{baitoan}[\cite{Binh_Toan_9_tap_2}, 249., p. 26]
	Chứng minh phương trình $ax^2 + bx + c = 0,a\ne0$, có nghiệm, biết $5a + 2c = b$.
\end{baitoan}

\begin{baitoan}[\cite{Binh_Toan_9_tap_2}, 250., p. 26]
	Cho $a,b,c$ là độ dài 3 cạnh 1 tam giác. Chứng minh phương trình $(a^2 + b^2 - c^2)x^2 - 4abx + a^2 + b^2 - c^2 = 0$ có nghiệm.
\end{baitoan}

\begin{baitoan}[\cite{Binh_Toan_9_tap_2}, 251., p. 26]
	Chứng minh phương trình $ax^2 + bx + c = 0,a\ne0$, có nghiệm nếu $\dfrac{2b}{a}\ge\dfrac{c}{a} + 4$.
\end{baitoan}

\begin{baitoan}[\cite{Binh_Toan_9_tap_2}, 252., p. 26]
	Chứng minh nếu $bm = 2(c + n)$ thì ít nhất 1 trong 2 phương trình $x^2 + bx + c = 0,x^2 + mx + n = 0$ có nghiệm.
\end{baitoan}

\begin{baitoan}[\cite{Binh_Toan_9_tap_2}, 253., p. 26]
	Cho $a,b,c\in\mathbb{Q},a\ne0,|b| = |a + c|$. Chứng minh các nghiệm của phương trình $ax^2 + bx + c = 0$ là các số hữu tỷ.
\end{baitoan}

\begin{baitoan}[\cite{Binh_Toan_9_tap_2}, 254., p. 26]
	Chứng minh phương trình $ax^2 + bx + c = 0$ không có nghiệm hữu tỷ nếu $a,b,c$ là 3 số nguyên lẻ.
\end{baitoan}

\begin{baitoan}[\cite{Binh_Toan_9_tap_2}, 255., p. 26]
	Chứng minh nếu $\overline{abc}$ là số nguyên tố thì phương trình $ax^2 + bx + c = 0$ không có nghiệm hữu tỷ.
\end{baitoan}

\begin{baitoan}[\cite{Binh_Toan_9_tap_2}, 256., p. 27]
	Tìm các giá trị nguyên của $m$ để nghiệm của phương trình $mx^2 - 2(m - 1)x + m - 4 = 0$ là số hữu tỷ.
\end{baitoan}

\begin{baitoan}[\cite{Binh_Toan_9_tap_2}, 257., p. 27]
	Tìm $n\in\mathbb{Z}$ để các nghiệm của phương trình $x^2 - (n + 4)x + 4n - 25 = 0$ là các số nguyên.
\end{baitoan}

\begin{baitoan}[\cite{Binh_Toan_9_tap_2}, 258., p. 27]
	Tìm số nguyên tố $p$ biết phương trình $x^2 + px - 12p = 0$ có 2 nghiệm đều là các số nguyên.
\end{baitoan}

\begin{baitoan}[\cite{Binh_Toan_9_tap_2}, 259., p. 27]
	Tìm các giá trị của $m\in\mathbb{R}$ để 2 phương trình có ít nhất 1 nghiệm chung: (a) $x^2 + 2x + m = 0,x^2 + mx + 2 = 0$. (b) $x^2 + mx + 1 = 0,x^2 - x - m = 0$.
\end{baitoan}

\begin{baitoan}[\cite{Binh_Toan_9_tap_2}, 260., p. 27]
	Tìm các giá trị của $m\in\mathbb{R}$ để 2 phương trình có ít nhất 1 nghiệm chung: (a) $x^2 + (m - 2)x + 3 = 0,2x^2 + mx + m + 2 = 0$. (b) $2x^2 + (3m - 5)x - 9 = 0,6x^2 + (7m - 15)x - 19 = 0$.
\end{baitoan}

\begin{baitoan}[\cite{Binh_Toan_9_tap_2}, 261., p. 27]
	Tìm các giá trị của $m\in\mathbb{R}$ để 1 nghiệm của phương trình $2x^2 - 13x + 2m = 0$ gấp đôi 1 nghiệm của phương trình $x^2 - 4x + m = 0$.
\end{baitoan}

\begin{baitoan}[\cite{Binh_Toan_9_tap_2}, 262., p. 27]
	Cho 2 phương trình $ax^2 + bx + c = 0,cx^2 + bx + a = 0$. Biết phương trình thứ nhất có nghiệm dương $m$, chứng minh phương trình thứ 2 có nghiệm $n$ sao cho $m + n\ge2$.
\end{baitoan}

\begin{baitoan}[\cite{TLCT_THCS_Toan_9_dai_so}, VD13.1, p. 71]
	Cho phương trình $mx^2 - 2(m + 2)x + 9 = 0$. Tìm $m\in\mathbb{R}$ để phương trình có nghiệm kép \& tìm nghiệm đó.
\end{baitoan}

\begin{baitoan}[\cite{TLCT_THCS_Toan_9_dai_so}, VD13.2, p. 71]
	Giả sử $x_1,x_2$ là 2 nghiệm của phương trình $x^2 + ax + b = 0$. (a) Lập phương trình bậc 2 nhận $x_1^2,x_2^2$ làm nghiệm. (b) Lập phương trình bậc 2 nhận $\dfrac{x_1}{x_2},\dfrac{x_2}{x_1}$ làm nghiệm khi $b\ne0$.
\end{baitoan}

\begin{baitoan}[\cite{TLCT_THCS_Toan_9_dai_so}, VD13.3, p. 72]
	Cho hệ phương trình
	\begin{equation*}
		\left\{\begin{split}
			2x + y &= 4 + a,\\
			3x - 4y &= -5 + 7a.
		\end{split}\right.
	\end{equation*}
	(a) Tìm $a\in\mathbb{R}$ để hệ có 1 nghiệm $(x,y)$ thỏa $x^2 + y^2 = 185$. (b) Tìm $a\in\mathbb{R}$ để hệ có 1 nghiệm $(x,y)$ với $P = xy$ lớn nhất.
\end{baitoan}

\begin{baitoan}[\cite{TLCT_THCS_Toan_9_dai_so}, VD13.4, p. 72]
	Chứng minh phương trình có vô số nghiệm trong $\mathbb{Q}$: (a) $x^2 + xy + y^2 = 1$. (b) $x^4 + y^4 + z^4 = 2$.
\end{baitoan}

\begin{baitoan}[\cite{TLCT_THCS_Toan_9_dai_so}, VD13.5, p. 73]
	Chứng minh phương trình $x^4 + 16y^4 + z^4 = 2$ có vô số nghiệm trong $\mathbb{Q}$.
\end{baitoan}

\begin{baitoan}[\cite{TLCT_THCS_Toan_9_dai_so}, VD13.6, p. 73]
	Trong mặt phẳng $(Oxy)$ cho parabol $(P):y = \dfrac{1}{4}x^2$. Giả sử 2 đường thẳng đi qua $I(0,1)$ cắt $(P)$ lần lượt tại $A_1,B_1,A_2,B_2$. Chứng minh: (a) $\dfrac{1}{IA_1} + \dfrac{1}{IB_1} = \dfrac{1}{IA_2} + \dfrac{1}{IB_2} = 1$. (b) $\dfrac{1}{\sqrt{IA_1\cdot IA_2}} + \dfrac{1}{\sqrt{IB_1\cdot IB_2}}\le1$.
\end{baitoan}

\begin{baitoan}[\cite{TLCT_THCS_Toan_9_dai_so}, VD13.7, p. 74]
	Trong mặt phẳng $(Oxy)$ cho parabol $(P):y = x^2$. Giả sử góc vuông $\widehat{uIv}$ thay đổi, nhưng 2 cạnh của nó luon tiếp xúc với $(P)$. Chứng minh đỉnh I chạy trên 1 đường thẳng cố định.
\end{baitoan}

\begin{baitoan}[\cite{TLCT_THCS_Toan_9_dai_so}, VD13.8, p. 75]
	Trong mặt phẳng $(Oxy)$ cho điểm $A(0,1)$. Giả sử điểm $B(2b,-1)$ thay đổi. (a) Viết phương trình đường trung trực $(d)$ của đoạn AB. (b) Chứng minh $(d)$ tiếp xúc với parabol $(P):y = \dfrac{1}{4}x^2$.
\end{baitoan}

\begin{baitoan}[\cite{TLCT_THCS_Toan_9_dai_so}, VD13.9, p. 76]
	Trong mặt phẳng $(Oxy)$ cho đường thẳng $(d):y = kx + \dfrac{1}{2}$ \& parabol $(P):y = \dfrac{1}{2}x^2$. Chứng minh: (a) $(d)$ đi qua 1 điểm cố định \& $(d)$ cắt parabol $(P)$ tại 2 điểm phân biệt A,B. (b) Có đúng 1 điểm M thuộc đường thẳng $(d'):y = -\dfrac{1}{2}$ để $MA\bot MB$.
\end{baitoan}

\begin{baitoan}[\cite{TLCT_THCS_Toan_9_dai_so}, VD13.10, p. 77]
	Trong mặt phẳng $(Oxy)$ cho đường thẳng $(d):y = x + 8$ \& parabol $(P):y = x^2$. Tính độ dài cạnh hình vuông ABCD biết $A,B\in(d)$ còn $C,D\in(P)$.
\end{baitoan}

\begin{baitoan}[\cite{TLCT_THCS_Toan_9_dai_so}, VD13.11, p. 78]
	Trong mặt phẳng $(Oxy)$ cho $(P):y = x^2$ \& 2 điểm $M(-1,1),N(1,1)\in(P)$. Giả sử 2 dây cung bất kỳ AB,CD đều khác MN, đi qua trung điểm I của MN. Gọi giao điểm của AC,BD với MN lần lượt là P,Q. Chứng minh $IP = IQ$. 
\end{baitoan}

\begin{baitoan}[\cite{TLCT_THCS_Toan_9_dai_so}, 13.1., p. 80]
	Giải phương trình $(8x - 7)^2 - (3x + 1)^2 = 0$.
\end{baitoan}

\begin{baitoan}[\cite{TLCT_THCS_Toan_9_dai_so}, 13.2., p. 80]
	Giải phương trình $(x + 1)(x + 2)(x + 3) = x^3 - 1$.
\end{baitoan}

\begin{baitoan}[\cite{TLCT_THCS_Toan_9_dai_so}, 13.3., p. 80]
	Giải \& biện luận phương trình $2ax^2 - 5x + 3 = 0$.
\end{baitoan}

\begin{baitoan}[\cite{TLCT_THCS_Toan_9_dai_so}, 13.4., p. 80]
	Tìm $m\in\mathbb{R}$ để phương trình $x^2 + (3m - 7)x + m = 0$ có 2 nghiệm $x_1,x_2$ thỏa $x_2 = 3x_1$.
\end{baitoan}

\begin{baitoan}[\cite{TLCT_THCS_Toan_9_dai_so}, 13.5., p. 80]
	Tìm $m\in\mathbb{R}$ để phương trình $x^2 - 5|x| + m = 0$ có đúng 1 nghiệm.
\end{baitoan}

\begin{baitoan}[\cite{TLCT_THCS_Toan_9_dai_so}, 13.6., p. 80]
	Giải \& biện luận phương trình $ax^2 - 2(3a + 1)x + a - 2 = 0$.
\end{baitoan}

\begin{baitoan}[\cite{TLCT_THCS_Toan_9_dai_so}, 13.7., p. 80]
	Tính $a = \sqrt{52 + 14\sqrt{3}} + \sqrt{52 - 14\sqrt{3}}$ bằng 2 cách khác nhau.
\end{baitoan}

\begin{baitoan}[\cite{TLCT_THCS_Toan_9_dai_so}, 13.8., p. 80]
	Giả sử $x_1,x_2$ là 2 nghiệm của phương trình $x^2 - ax + b = 0$. Lập phương trình bậc 2 nhận $3x_1 + 2x_2,3x_2 + 2x_1$ làm nghiệm.
\end{baitoan}

\begin{baitoan}[\cite{TLCT_THCS_Toan_9_dai_so}, 13.9., p. 80]
	Cho $x_1,x_2$ là 2 nghiệm của $x^2 + ax + 1 = 0$. Tìm {\rm GTNN} của $A = x_1^4 + x_2^4$.
\end{baitoan}

\begin{baitoan}[\cite{TLCT_THCS_Toan_9_dai_so}, 13.10., p. 80]
	Giả sử $x_1,x_2$ là 2 nghiệm của phương trình $x^2 + ax + 1 = 0$ \& $x_3,x_4$ là 2 nghiệm của phương trình $x^2 + bx + 1 = 0$. Tính $A = (x_1 + x_2)(x_2 + x_3)(x_3 + x_4)(x_4 + x_1),B = (x_1 + x_3)(x_2 + x_3) + (x_1 + x_4)(x_2 + x_4)$.
\end{baitoan}

\begin{baitoan}[\cite{TLCT_THCS_Toan_9_dai_so}, 13.11., p. 80]
	Cho $a,b,c,d\in\mathbb{R}$ thỏa $ab + 2(b + c + d) = c(a + b)$. Chứng minh có ít nhất 1 trong 3 phương trình $x^2 - ax + b = 0,x^2 - bx + c = 0,x^2 - cx + d = 0$ có nghiệm.
\end{baitoan}

\begin{baitoan}[\cite{TLCT_THCS_Toan_9_dai_so}, 13.12., pp. 80--81]
	Cho 4 phương trình ẩn $x$: $x^2 - 2ax + 4 = 0,x^2 + 2x + 4a^2 = 0,x^2 + 4ax + 1 = 0,x^2 - 4x + a^2 = 0$. Chứng minh có ít nhất 2 phương trình có nghiệm, $\forall a\in\mathbb{R}$.
\end{baitoan}

\begin{baitoan}[\cite{TLCT_THCS_Toan_9_dai_so}, 13.13., p. 81]
	Giả sử $x_1,x_2$ là 2 nghiệm của phương trình $ax^2 + 6x - 8 = 0$. Chứng minh luôn lập được 1 hệ thức giữa $x_1,x_2$ không phụ thuộc vào a.
\end{baitoan}

%------------------------------------------------------------------------------%

\section{Vi\`ete Theorem -- Định Lý Vi\`ete}
\fbox{1} Định lý Vi\`ete: Nếu $x_1,x_2$ là 2 nghiệm của phương trình $ax^2 + bx + c = 0,a\ne0$ (1) thì $x_1 + x_2 = -\dfrac{b}{a},x_1x_2 = \dfrac{c}{a}$. \fbox{2} Nếu $a + b + c = 0$ thì phương trình (1) có 2 nghiệm $x_1 = 1,x_2 = \dfrac{c}{a}$. Nếu $a - b + c = 0$ thì phương trình (1) có 2 nghiệm $x_1 = -1,x_2 = -\dfrac{c}{a}$. \fbox{3} Nếu 2 số có tổng bằng $S$ \& tích bằng $P$ với $S^2\ge4P$ thì 2 số đó là 2 nghiệm của phương trình bậc 2 $x^2 - Sx + P = 0$. \fbox{4} Nếu (1) có 2 nghiệm $x_1,x_2$ thì $ax^2 + bx + c = a(x - x_1)(x - x_2)$. \fbox{5} Đặt $S\coloneqq x_1 + x_2,P\coloneqq x_1x_2$. Điều kiện để (1): Có 2 nghiệm trái dấu là $P < 0$. Có 2 nghiệm cùng dấu là $\Delta\ge0,P > 0$. Có 2 nghiệm dương là $\Delta\ge0,P > 0,S > 0$. Có 2 nghiệm âm là $\Delta\ge0,P > 0,S < 0$.\\

\noindent\cite[Chap. VII, \S3, pp. 61--65]{SGK_Toan_9_Canh_Dieu_tap_2}: HD1. LT1. LT2. LT3. HD2. LT4. 1. 2. 3. 4. 5. 6. 7.

\begin{baitoan}[\cite{Binh_boi_duong_Toan_9_tap_2}, VD1, p. 51]
	Tìm điều kiện của $n\in\mathbb{R}$ để phương trình có nghiệm rồi tính tổng \& tích của các nghiệm theo $n$: (a) $x^2 - 2(n + 1)x + (n - 3)(n + 8) = 0$. (b) $(n - 5)x^2 + (2n - 1)x + n + 2 = 0$.
\end{baitoan}

\begin{baitoan}[\cite{Binh_boi_duong_Toan_9_tap_2}, VD2, p. 52]
	Tìm $u,v\in\mathbb{R}$ thỏa: (a) $u + v = 14,uv = 45$. (b) $u + v = 30,uv = 230$. (c) $u - v = 7,uv = 120$.
\end{baitoan}

\begin{baitoan}[\cite{Binh_boi_duong_Toan_9_tap_2}, VD3, p. 52]
	Phân tích đa thức thành nhân tử: (a) $f(x) = 2x^2 - 3x - 5$. (b) $g(x) = x^2 + (\sqrt{3} - \sqrt{2})x - \sqrt{6}$.
\end{baitoan}

\begin{baitoan}[\cite{Binh_boi_duong_Toan_9_tap_2}, VD4, p. 53]
	Lập 1 phương trình bậc 2 có 2 nghiệm: (a) $5,8$. (b) $1\pm\sqrt{3}$. (c) $(m\pm1)^2$.
\end{baitoan}

\begin{baitoan}[\cite{Binh_boi_duong_Toan_9_tap_2}, VD5, p. 53]
	Tìm $m\in\mathbb{R}$ để phương trình $x^2 - 2(m - 2)x + m + 4 = 0$: (a) Có 2 nghiệm trái dấu. (b) Có 2 nghiệm dương phân biệt.
\end{baitoan}

\begin{baitoan}[\cite{Binh_boi_duong_Toan_9_tap_2}, VD6, p. 53]
	Cho phương trình $x^2 + 5x + m = 0$. (a) Tìm $m\in\mathbb{R}$ để phương trình có 2 nghiệm $x_1,x_2$ thỏa mãn $9x_1 + 2x_2 = 18$. (b) Tính theo $m$ giá trị biểu thức $x_1^2 + x_2^2,x_1^3 + x_2^3,\left(\dfrac{x_1}{x_2}\right)^2 + \left(\dfrac{x_2}{x_1}\right)^2$.
\end{baitoan}

\begin{baitoan}[\cite{Binh_boi_duong_Toan_9_tap_2}, 6.1., p. 54]
	Nhẩm nghiệm phương trình: (a) $9x^2 + 300x - 309 = 0$. (b) $61x^2 - 1954x - 2015 = 0$. (c) $x^2 - 8x + 12 = 0$. (d) $x^2 + 2x - 35 = 0$. (e) $(m - 2)x^2 - (4m + 3)x + 3m + 5 = 0$.
\end{baitoan}

\begin{baitoan}[\cite{Binh_boi_duong_Toan_9_tap_2}, 6.2., p. 54]
	(a) Phương trình $5x^2 + 9x + n = 0$ có nghiệm $x_1 = 1$. Tìm $n$ \& tìm nghiệm còn lại. (b) Phương trình $0.2x^2 - x + n = 0$ có nghiệm $x_1 = -1$. Tìm $n$ \& tìm nghiệm còn lại.
\end{baitoan}

\begin{baitoan}[\cite{Binh_boi_duong_Toan_9_tap_2}, 6.3., p. 54]
	Tìm giá trị của $n\in\mathbb{R}$ để phương trình có 2 nghiệm rồi tính tổng, tích 2 nghiệm: (a) $x^2 - 3x + n = 0$. (b) $nx^2 - 2(n - 1)x + n - 1 = 0$.
\end{baitoan}

\begin{baitoan}[\cite{Binh_boi_duong_Toan_9_tap_2}, 6.4., p. 54]
	Tìm $u,v$ thỏa: (a) $u + v = 25,uv = 84$. (b) $u - 2v = -17,uv = 240$.
\end{baitoan}

\begin{baitoan}[\cite{Binh_boi_duong_Toan_9_tap_2}, 6.5., p. 54]
	Lập phương trình bậc 2 có 2 nghiệm: (a) $20,21$. (b) $2\pm\sqrt{3}$. (c) $\dfrac{3m + 2}{3m - 2},\dfrac{3m - 2}{3m + 2}$ với $m\ne\pm\dfrac{2}{3}$.
\end{baitoan}

\begin{baitoan}[\cite{Binh_boi_duong_Toan_9_tap_2}, 6.6., p. 55]
	Viết phương trình bậc 2 có 2 nghiệm $x_1,x_2$ thỏa mãn các điều kiện:
	\begin{equation*}
		\left\{\begin{split}
			2x_1x_2 - 5(x_1 + x_2) &= 12,\\
			(x_1 - 3)(x_2 - 3) &= 15 - k.
		\end{split}\right.
	\end{equation*}
\end{baitoan}

\begin{baitoan}[\cite{Binh_boi_duong_Toan_9_tap_2}, 6.7., p. 55]
	Cho tam thức bậc 2 $f(x) = x^2 - 2(m - 4)x + m - 6$. (a) Phân tích $f(x)$ với $m = 3$ thành nhân tử. (b) Chứng minh phương trình $f(x) = 0$ luôn có 2 nghiệm $x_1,x_2$ phân biệt $\forall m\in\mathbb{R}$. (c) Tìm $m\in\mathbb{R}$ để 2 nghiệm đó trái dấu. (d) Tính theo $m$ giá trị biểu thức $A = \dfrac{1}{x_1} + \dfrac{1}{x_2}$ \& tìm $m\in\mathbb{Z}$ để $A\in\mathbb{Z}$. (e) Tìm {\rm GTNN} của $x_1^2 + x_2^2$.
\end{baitoan}

\begin{baitoan}[\cite{Binh_boi_duong_Toan_9_tap_2}, 6.8., p. 55]
	Cho phương trình $x^2 + (2m - 5)x + n = 0$. (a) Phân tích vế trái thành nhân tử với $m = 1,n = -4$. (b) Tìm $m,n\in\mathbb{R}$ để phương trình có 2 nghiệm là $2,-3$. (c) Cho $m = 5$, tìm số nguyên $n$ lớn nhất để phương trình có nghiệm dương rồi tìm nghiệm của phương trình.
\end{baitoan}

\begin{baitoan}[\cite{Binh_boi_duong_Toan_9_tap_2}, p. 56, định lý Vi\`ete cho phương trình bậc 3]
	Chứng minh nếu $x_1,x_2,x_3$ là 3 nghiệm của phương trình bậc 3 $ax^3 + bx^2 + cx + d = 0,a,b,c,d\in\mathbb{R},a\ne0$ thì
	\begin{equation*}
		\left\{\begin{split}
			x_1 + x_2 + x_3 &= -\frac{b}{a},\\
			x_1x_2 + x_2x_3 + x_3x_1 &= \frac{c}{a},\\
			x_1x_2x_3 &= -\frac{d}{a}.
		\end{split}\right.
	\end{equation*}
\end{baitoan}

\begin{baitoan}[\cite{Binh_boi_duong_Toan_9_tap_2}, p. 5, định lý Vi\`ete cho phương trình bậc 4]
	Chứng minh nếu $x_1,x_2,x_3,x_4$ là 4 nghiệm của phương trình bậc 4 $ax^4 + bx^3 + cx^2 + dx + e = 0,a,b,c,d,e\in\mathbb{R},a\ne0$ thì
	\begin{equation*}
		\left\{\begin{split}
			x_1 + x_2 + x_3 + x_4 &= -\frac{b}{a},\\
			x_1x_2 + x_1x_3 + x_1x_4 + x_2x_3 + x_2x_4 + x_3x_4 &= \frac{c}{a},\\
			x_1x_2x_3 + x_1x_2x_4 + x_1x_3x_4 + x_2x_3x_4 &= -\frac{d}{a},\\
			x_1x_2x_3x_4 &= \frac{e}{a}.
		\end{split}\right.
	\end{equation*}
\end{baitoan}

\begin{baitoan}
	Phát biểu \& chứng minh định lý Vi\`ete cho phương trình bậc $n$.
\end{baitoan}
Hệ thức Vi\`ete còn được sử dụng trong việc tìm hệ thức giữa 2 nghiệm $x_1,x_2$ của phương trình bậc 2 không phụ thuộc tham số $m$: Tìm điều kiện để phương trình bậc 2 có nghiệm: $\Delta\ge0$. Từ hệ thức Vi\`ete tìm $S,P$ theo tham số $m$ rồi khử tham số $m$ từ $S,P$ để có hệ thức chỉ còn $x_1,x_2$.

\begin{baitoan}[\cite{Binh_boi_duong_Toan_9_tap_2}, VD, p. 56]
	Giả sử $x_1,x_2$ là nghiệm của phương trình $x^2 - (1 - 2m)x - 2 + 2m = 0$. Tìm hệ thức giữa $x_1,x_2$ không phụ thuộc vào $m$.
\end{baitoan}

\begin{dinhnghia}
	Hệ phương trình 2 ẩn $x,y$ gọi là {đối xứng loại I} nếu hệ không thay đổi khi thay $x$ bởi $y$, $y$ bởi $x$ trong mỗi phương trình.
\end{dinhnghia}
Khi giải hệ phương trình đối xứng, đặt $S\coloneqq x + y,P\coloneqq xy$ đưa hệ về hệ mới với 2 ẩn $S,P$. Khi đó, $x,y$ là nghiệm của phương trình $X^2 - SX + P = 0$. Trong hệ đối xứng loại I, nếu $(x_0,y_0)$ là nghiệm  thì $(y_0,x_0)$ cũng là nghiệm.

\begin{baitoan}[\cite{Binh_boi_duong_Toan_9_tap_2}, VD, p. 57]
	Giải hệ phương trình:
	\begin{equation*}
		\left\{\begin{split}
			x^2 + y^2 &= 29,\\
			x + xy + y &= 17.
		\end{split}\right.
	\end{equation*}
\end{baitoan}

\begin{baitoan}[\cite{Binh_boi_duong_Toan_9_tap_2}, p. 57]
	Giải hệ phương trình với $x\ne-y$:
	\begin{equation*}
		\left\{\begin{split}
			\frac{x^3 + y^3}{x + y} &= 7,\\
			x^2 + y^2 &= 13.
		\end{split}\right.
	\end{equation*}
\end{baitoan}

\begin{baitoan}[\cite{Tuyen_Toan_9_old}, VD36, p. 79]
	Cho 2 phương trình $x^2 + ax + bc = 0,x^2 + bx + ca = 0$ với $a,b,c\in\mathbb{R},ac\ne bc$. Giả sử $x_1,x_2$ là nghiệm của phương trình thứ nhất, $x_2,x_3$ là nghiệm của phương trình thứ 2. Viết 1 phương trình bậc 2 có nghiệm là $x_1,x_3$.
\end{baitoan}

\begin{baitoan}[\cite{Tuyen_Toan_9_old}, VD37, p. 80]
	Cho phương trình $x^2 - 2(m + 3)x + 4m - 1 = 0$. (a) Tìm $m\in\mathbb{R}$ để phương trình có 2 nghiệm dương. (b) Tìm 1 hệ thức liên hệ giữa 2 nghiệm không phụ thuộc vào $m$.
\end{baitoan}

\begin{baitoan}[\cite{Tuyen_Toan_9_old}, 215., p. 81]
	Cho phương trình $ax^2 + bx + c = 0$, $a,b,c\in\mathbb{R},ac\ne0$. Biết $x_1,x_2$ là 2 nghiệm. Tính theo $a,b,c$ giá trị biểu thức: (a) $A = \dfrac{1}{x_1} + \dfrac{1}{x_2}$. (b) $B = x_1^2 + x_2^2$. (c) $C = x_1^3 + x_2^3$.
\end{baitoan}

\begin{baitoan}[\cite{Tuyen_Toan_9_old}, 216., pp. 81--82]
	Không giải phương trình $2x^2 - 4x - 1 = 0$. Tính: (a) Hiệu 2 nghiệm. (b) Hiệu các bình phương của 2 nghiệm. (c) Hiệu các lập phương của 2 nghiệm.
\end{baitoan}

\begin{baitoan}[\cite{Tuyen_Toan_9_old}, 217., p. 82]
	Gọi $x_1,x_2\in\mathbb{R}$ là 2 nghiệm phân biệt của phương trình $x^2 + mx + 25 = 0$. Chứng minh $|x_1 + x_2| > 10$.
\end{baitoan}

\begin{baitoan}[\cite{Tuyen_Toan_9_old}, 218., p. 82]
	Cho phương trình $x^2 + mx + n = 0$, $m,n\in\mathbb{R}^\star$. Biết phương trình có 2 nghiệm là $m,n$. Chứng minh $|x_1x_2| = 2$.
\end{baitoan}

\begin{baitoan}[\cite{Tuyen_Toan_9_old}, 219., p. 82]
	Cho phương trình $x^2 + mx - 5 = 0$. Tìm $m\in\mathbb{R}$ để tổng bình phương 2 nghiệm bằng $11$.
\end{baitoan}

\begin{baitoan}[\cite{Tuyen_Toan_9_old}, 220., p. 82]
	Cho phương trình $x^2 + (4m + 1) + 2(m - 4) = 0$. (a) Tìm $m\in\mathbb{R}$ để phương trình có 2 nghiệm $x_1,x_2$ thỏa $x_2 - x_1 = 17$. (b) Tìm $m\in\mathbb{R}$ để biểu thức $A = (x_1 - x_2)^2$ có {\rm GTNN}. (c) Tìm hệ thức liên hệ giữa 2 nghiệm không phụ thuộc vào $m$.
\end{baitoan}

\begin{baitoan}[\cite{Tuyen_Toan_9_old}, 221., p. 82]
	Cho phương trình $x^2 - px + q = 0$ với $p,q$ là 2 số nguyên tố. Biết phương trình có 2 nhgiệm nguyên dương phân biệt, chứng minh $p^2 + q^2$ là 1 số nguyên tố.
\end{baitoan}

\begin{baitoan}[\cite{Tuyen_Toan_9_old}, 222., p. 82]
	Giả sử phương trình $x^2 + mx + n + 1 = 0$ có 2 nghiệm $x_1,x_2\in\mathbb{Z}^\star$. Chứng minh $m^2 + n^2$ là 1 hợp số.
\end{baitoan}

\begin{baitoan}[\cite{Tuyen_Toan_9_old}, 223., p. 82]
	Tìm $m\in\mathbb{R}$ để phương trình: (a) $2x^2 - 3(m + 1)x + m^2 - m - 2 = 0$ có 2 nghiệm trái dấu. (b) $mx^2 - 2(m - 2)x + 3(m - 2) = 0$ có 2 nghiệm cùng dấu.
\end{baitoan}

\begin{baitoan}[\cite{Tuyen_Toan_9_old}, 224., p. 82]
	Cho phương trình $3mx^2 + 2(2m + 1)x + m = 0$. Tìm $m\in\mathbb{R}$ để phương trình có 2 nghiệm âm.
\end{baitoan}

\begin{baitoan}[\cite{Tuyen_Toan_9_old}, 225., p. 82]
	Tìm $m\in\mathbb{R}$ để phương trình $(m - 1)x^2 + 2x + m = 0$ có ít nhất 1 nghiệm không âm.
\end{baitoan}

\begin{baitoan}[\cite{Tuyen_Toan_9_old}, 226., p. 83]
	Tìm $m\in\mathbb{R}$ để phương trình $2x^2 - 4x + 5(m - 1) = 0$ có 2 nghiệm phân biệt nhỏ hơn $3$.
\end{baitoan}

\begin{baitoan}[\cite{Tuyen_Toan_9_old}, 227., p. 83]
	Tìm $m\in\mathbb{R}$ để phương trình $x^2 + mx + m - 1 = 0$ có 2 nghiệm lớn $m$.
\end{baitoan}

\begin{baitoan}[\cite{Tuyen_Toan_9_old}, 228., p. 83]
	Viết phương trình bậc 2 có 2 nghiệm $x_1,x_2$ thỏa
	\begin{equation*}
		\left\{\begin{split}
			2x_1x_2 - 3(x_1 + x_2) &= 2,\\
			(x_1 - 2)(x_2 - 2) &= k + 5.
		\end{split}\right.
	\end{equation*}
\end{baitoan}

\begin{baitoan}[\cite{Tuyen_Toan_9_old}, 229., p. 83]
	Cho 2 phương trình $x^2 + x + m = 0,x^2 + ax + b = 0$. Tìm $a,b\in\mathbb{Z}$ để các nghiệm của phương trình thứ nhất tương ứng là lập phương các nghiệm của phương trình thứ 2.
\end{baitoan}

\begin{baitoan}[\cite{Tuyen_Toan_9_old}, 230., p. 83]
	Cho 2 phương trình $x^2 + bx + c = 0,x^2 + mx + n = 0$ với $b,c,m,n\in\mathbb{R}^\star$. Biết $b,c$ là 2 nghiệm của phương trình thứ 2 \& $m,n$ là 2 nghiệm của phương trình thứ nhất. Chứng minh $b^2 + c^2 + m^2 + n^2 = 10$.
\end{baitoan}

\begin{baitoan}[\cite{Tuyen_Toan_9_old}, 231., p. 83]
	Cho phương trình $ax^2 + bx + c = 0$, $a,bc\in\mathbb{R},ac\ne0$, có nghiệm $x_1 > 0$. Chứng minh phương trình $cx^2 + bx + a = 0$ có nghiệm $x_2 > 0$ \& $x_1 + x_2 + x_1x_2\ge3$.
\end{baitoan}

\begin{baitoan}[\cite{Tuyen_Toan_9_old}, 232., p. 83]
	Cho phương trình $2x^2 - 2(m - 1)x + m^2 - 4m + 3 = 0$. (a) Tìm $m\in\mathbb{Z}$ để phương trình có nghiệm. (b) Xác định dấu của 2 nghiệm $x_1,x_2,x_1\le x_2$, với các giá trị vừa tìm được của $m$.
\end{baitoan}

\begin{baitoan}[\cite{Tuyen_Toan_9_old}, 233., p. 84]
	Giải hệ phương trình:
	\begin{equation*}
		\left\{\begin{split}
			xy + x + y &= 11,\\
			x^2 + y^2 &= 13.
		\end{split}\right.
	\end{equation*}
\end{baitoan}

\begin{baitoan}[\cite{Binh_Toan_9_tap_2}, VD81, p. 28]
	Cho phương trình $mx^2 - 2(m + 1)x + m - 4 = 0$ với tham số $m$. (a) Tìm $m$ để phương trình có nghiệm. (b) Tìm $m$ để phương trình có 2 nghiệm trái dấu. Khi đó trong 2 nghiệm, nghiệm nào có giá trị tuyệt đối lớn hơn? (c) Tìm $m$ để 2 nghiệm $x_1,x_2$ của phương trình thỏa mãn $x_1 + 4x_2 = 3$. (d) Tìm 1 hệ thức giữa $x_1,x_2$ không phụ thuộc vào $m$.
\end{baitoan}

\begin{baitoan}[\cite{Binh_Toan_9_tap_2}, VD82, p. 30]
	Cho phương trình $mx^2 - 2(m - 2)x + m - 3 = 0$. Tìm các giá trị của $m\in\mathbb{R}$ để 2 nghiệm $x_1,x_2$ của phương trình thỏa $x_1^2 + x_2^2 = 1$.
\end{baitoan}

\begin{baitoan}[\cite{Binh_Toan_9_tap_2}, VD83, p. 30]
	Cho phương trình $x^2 + ax + b = 0$ có 2 nghiệm $c,d$, phương trình $x^2 + cx + d = 0$ có 2 nghiệm $a,b$. Tính $a,b,c,d$ biết chúng đều khác $0$.
\end{baitoan}

\begin{baitoan}[\cite{Binh_Toan_9_tap_2}, VD84, p. 31]
	Cho phương trình $x^2 + 5x - 1 = 0$. Không giải phương trình, lập 1 phương trình bậc 2 có 2 nghiệm là lũy thừa bậc $4$ của 2 nghiệm của phương trình ban đầu.
\end{baitoan}

\begin{baitoan}[\cite{Binh_Toan_9_tap_2}, 263., p. 31]
	Tính nhẩm nghiệm của phương trình: (a) $mx^2 - 2(m - 1)x + m - 2 = 0$. (b) $(m - 1)x^2 + (m + 1)x + 2 = 0$.
\end{baitoan}

\begin{baitoan}[\cite{Binh_Toan_9_tap_2}, 264., p. 31]
	Không giải phương trình, xét dấu các nghiệm của phương trình (nếu có): (a) $3x^2 - 7x + 2 = 0$. (b) $5x^2 + 3x - 1 = 0$. (c) $2x^2 + 13x + 8 = 0$. (d) $4x^2 - 11x + 8 = 0$.
\end{baitoan}

\begin{baitoan}[\cite{Binh_Toan_9_tap_2}, 265., p. 32]
	Tìm giá trị của $m$ để phương trình $(m - 1)x^2 - 2x + 3 = 0$ có 2 nghiệm phân biệt cùng dấu.
\end{baitoan}

\begin{baitoan}[\cite{Binh_Toan_9_tap_2}, 266., p. 32]
	Giải phương trình $x^2 - mx + n = 0$ biết phương trình có 2 nghiệm nguyên dương phân biệt \& $m,n$ là 2 số nguyên tố.
\end{baitoan}

\begin{baitoan}[\cite{Binh_Toan_9_tap_2}, 267., p. 32]
	Gọi $x_1,x_2$ là 2 nghiệm của phương trình $2x^2 - 3x - 5 = 0$. Không giải phương trình, tính: (a) $\dfrac{1}{x_1} + \dfrac{1}{x_2}$. (b) $(x_1 - x_2)^2$. (c) $x_1^3 + x_2^3$.
\end{baitoan}

\begin{baitoan}[\cite{Binh_Toan_9_tap_2}, 268., p. 32]
	Cho phương trình $x^2 - 2(m - 2)x + m^2 + 2m - 3 = 0$. Tìm các giá trị của $m\in\mathbb{R}$ để phương trình có 2 nghiệm $x_1,x_2$ phân biệt thỏa $\dfrac{1}{x_1} + \dfrac{1}{x_2} = \dfrac{x_1 + x_2}{5}$.
\end{baitoan}

\begin{baitoan}[\cite{Binh_Toan_9_tap_2}, 269., p. 32]
	Cho phương trình $x^2 + mx + n = 0$ có $3m^2 = 16n$. Chứng minh trong 2 nghiệm của phương trình, có 1 nghiệm gấp 3 lần nghiệm kia.
\end{baitoan}

\begin{baitoan}[\cite{Binh_Toan_9_tap_2}, 270., p. 32]
	Cho biết phương trình $x^2 - (m + 2)x + 2m - 1 = 0$ có 2 nghiệm $x_1,x_2$. Lập 1 hệ thức giữa $x_1,x_2$ độc lập đối với $m$.
\end{baitoan}

\begin{baitoan}[\cite{Binh_Toan_9_tap_2}, 271., p. 32]
	Tìm 2 số biết: (a) Tổng của chúng bằng $2$, tích của chúng bằng $-1$. (b) Tổng của chúng bằng $1$, tích của chúng bằng $5$.
\end{baitoan}

\begin{baitoan}[\cite{Binh_Toan_9_tap_2}, 272., p. 32]
	Lập phương trình bậc 2 có 2 nghiệm bằng: (a) $\sqrt{3},2\sqrt{3}$. (b) $2\pm\sqrt{3}$.
\end{baitoan}

\begin{baitoan}[\cite{Binh_Toan_9_tap_2}, 273., p. 32]
	Chứng minh tồn tại 1 phương trình có các hệ số hữu tỷ nhận 1 trong các nghiệm là: (a) $\dfrac{\sqrt{3} - \sqrt{5}}{\sqrt{3} + \sqrt{5}}$. (b) $\dfrac{\sqrt{2} + \sqrt{3}}{\sqrt{2} - \sqrt{3}}$. (c) $\sqrt{2} + \sqrt{3}$.
\end{baitoan}

\begin{baitoan}[\cite{Binh_Toan_9_tap_2}, 274., p. 32]
	Lập phương trình bậc 2 có 2 nghiệm bằng: (a) Bình phương của 2 nghiệm của phương trình $x^2 - 2x - 1 = 0$. (b) Nghịch đảo của 2 nghiệm của phương trình $x^2 + mx - 2 = 0$.
\end{baitoan}

\begin{baitoan}[\cite{Binh_Toan_9_tap_2}, 275., p. 33]
	Tìm $m,n$ sao cho 2 nghiệm của phương trình $x^2 + mx + n = 0$ cũng là $m,n$.
\end{baitoan}

\begin{baitoan}[\cite{Binh_Toan_9_tap_2}, 276., p. 33]
	Cho $a,b,c\in\mathbb{R}$ khác nhau đôi một, $c\ne0$. Biết 2 phương trình $x^2 + ax + bc = 0,x^2 + bx + ca = 0$ có ít nhất 1 nghiệm chung. (a) Tìm các nghiệm còn lại của 2 phương trình. (b) Chứng minh các nghiệm còn lại đó là nghiệm của phương trình $x^2 + cx + ab = 0$.
\end{baitoan}

\begin{baitoan}[\cite{Binh_Toan_9_tap_2}, 277., p. 33]
	Cho 2 phương trình $ax^2 + bx + c = 0,cx^2 + dx + a = 0$. Biết phương trình thứ nhất có 2 nghiệm $m,n$, phương trình thứ 2 có 2 nghiệm $p,q$. Chứng minh $m^2 + n^2 + p^2 + q^2\ge4$.
\end{baitoan}

\begin{baitoan}[\cite{Binh_Toan_9_tap_2}, 278., p. 33]
	Cho 2 phương trình $ax^2 + bx + c = 0,cx^2 + bx + a = 0$. Tìm 1 hệ thức giữa 3 hệ số $a,b,c$, biết 2 nghiệm $x_1,x_2$ của phương trình thứ nhất \& 2 nghiệm $x_3,x_4$ của phương trình thứ 2 thỏa mãn đẳng thức $x_1^2 + x_2^2 + x_3^2 + x_4^2 = 4$.
\end{baitoan}

\begin{baitoan}[\cite{Binh_Toan_9_tap_2}, 279., p. 33]
	Cho phương trình $x^2 + bx + c = 0$ có 2 nghiệm $x_1,x_2$, phương trình $x^2 - b^2x + bc = 0$ có 2 nghiệm $x_3,x_4$. Biết $x_3 - x_1 = x_4 - x_2 = 1$. Tìm $b,c$.
\end{baitoan}

\begin{baitoan}[\cite{Binh_Toan_9_tap_2}, 280., p. 33]
	Tìm $a,b\in\mathbb{R}$ sao cho 2 phương trình $x^2 + ax + 6 = 0,x^2 + bx + 12 = 0$ có ít nhất 1 nghiệm chung \& $|a| + |b|$ nhỏ nhất.
\end{baitoan}

\begin{baitoan}[\cite{Binh_Toan_9_tap_2}, 281., pp. 33--34]
	Gọi $x_1,x_2$ là 2 nghiệm của phương trình $x^2 - 6x + 1 = 0$. Ký hiệu $s_n = x_1^n + x_2^n$, $\forall n\in\mathbb{N}^\star$. (a) Tính $s_1,s_2,s_3$. (b) Tìm 1 hệ thức giữa $s_n,s_{n+1},s_{n+2}$. (c) Chứng minh $s_n\in\mathbb{Z}$, $\forall n\in\mathbb{N}^\star$. (d) Tìm số dư khi chia $s_{50}$ cho $5$.
\end{baitoan}

%------------------------------------------------------------------------------%

\section{Phương Trình Quy Về Phương Trình Bậc 2}
\fbox{1} \textit{Phương trình trùng phương}: $ax^4 + bx^2 + c = 0,a,b,c\in\mathbb{R},a\ne0$ (1). Đặt $t\coloneqq x^2\ge0$ được phương trình bậc 2 (trung gian) $at^2 + bt + c = 0$. \fbox{2} \textit{Phương trình chứa ẩn ở mẫu}: Tìm ĐKXĐ của phương trình. Quy đồng mẫu thức 2 vế rồi khử mẫu thức. Giải phương trình vừa nhận được. Trong các giá trị tìm được của ẩn: Loại các giá trị không thỏa mãn ĐKXĐ. Các giá trị thỏa mãn ĐKXĐ là nghiệm của phương trình. \fbox{3} \textit{Phương trình tích}: $\prod_{i=1}^n f_i(x) = f_1(x)f_2(x)\cdots f_n(x) = 0\Leftrightarrow f_1(x) = 0$ or $f_2(x) = 0$ or $\ldots$ or $f_n(x) = 0$.

\subsection{Phương trình trùng phương}

\begin{baitoan}[{\sf Program}: Solve biquadratic equation]
	Viết chương trình {\sf Pascal, Python, C{\tt/}C++} để giải phương trình bậc 2 1 ẩn $ax^4 + bx^2 + c = 0$.
	\begin{itemize}
		\item {\sf Input}: 3 hệ số $a,b,c\in\mathbb{R}$ được nhập từ bàn phím.
		\item {\sf Output}: Số nghiệm của phương trình \& liệt kê các nghiệm đó, {\rm GTNN, GTLN}, các khoảng đồng biến, nghịch biến.
	\end{itemize}
\end{baitoan}

\begin{baitoan}[{\sf Program}: Solve multi-quadratic equation]
	Viết chương trình {\sf Pascal, Python, C{\tt/}C++} để giải phương trình bậc 2 1 ẩn $ax^{2n} + bx^n + c = 0$.
	\begin{itemize}
		\item {\sf Input}: $n\in\mathbb{N}^\star$, 3 hệ số $a,b,c\in\mathbb{R}$ được nhập từ bàn phím.
		\item {\sf Output}: Số nghiệm của phương trình \& liệt kê các nghiệm đó, {\rm GTNN, GTLN}, các khoảng đồng biến, nghịch biến.
	\end{itemize}
\end{baitoan}

\begin{baitoan}[Mở rộng phương trình trùng phương]
	Biện luận theo 3 tham số $a,b,c\in\mathbb{R},a\ne0$ để giải phương trình: (a) $ax^6 + bx^3 + c = 0$. (b) $ax^8 + bx^4 + c = 0$. (c) $ax^{2n} + bx^n + c = 0$ với $n\in\mathbb{Z}$.
\end{baitoan}
Giải phương trình:

\begin{baitoan}[\cite{Binh_boi_duong_Toan_9_tap_2}, H1, p. 59]
	$x^4 - 5x^2 + 4 = 0$.
\end{baitoan}

\begin{baitoan}[\cite{Binh_boi_duong_Toan_9_tap_2}, H2, p. 59]
	$2x^2(x^2 - 1) = 0$.
\end{baitoan}

\begin{baitoan}[\cite{Binh_boi_duong_Toan_9_tap_2}, H3, p. 59]
	$\dfrac{(x - 9)^2}{x(x^2 - 4)} = \dfrac{5}{x^2 + 1}$.
\end{baitoan}

\begin{baitoan}[\cite{Binh_boi_duong_Toan_9_tap_2}, H4, p. 59]
	Nhẩm nhanh nghiệm: (a) $x^4 - 2x^2 + 1 = 0$. (b) $x^4 - 10x^2 + 9 = 0$. (c) $(x^2 - 7)(x^2 + 9) = 0$.
\end{baitoan}
Giải phương trình:

\begin{baitoan}[\cite{Binh_boi_duong_Toan_9_tap_2}, VD1, p. 59]
	(a) $25x^4 - 26x^2 + 1 = 0$. (b) $9x^4 + 5x^2 = 4$. (c) $0.1x^4 + 0.2x^2 + 0.3 = 0$.
\end{baitoan}

\begin{baitoan}[\cite{Binh_boi_duong_Toan_9_tap_2}, VD2, p. 60]
	(a) $\dfrac{x + 2}{x - 3} - 3 = \dfrac{5}{2 - x}$. (b) $\dfrac{x^4 - 8}{x^4 - 4} + \dfrac{1}{x^2 - 2} = \dfrac{2}{x^2 + 2}$.
\end{baitoan}

\begin{baitoan}[\cite{Binh_boi_duong_Toan_9_tap_2}, VD3, p. 61]
	(a) $(4x^2 - 9)(3x^2 - 5x - 8) = 0$. (b) $(3x^2 - x - 5)^2 - (4x - 1)^2 = 0$. (c) $x^4 - 5x^3 = 4x^2 - 20x$.
\end{baitoan}

\begin{baitoan}[\cite{Binh_boi_duong_Toan_9_tap_2}, VD4, p. 61]
	Cho phương trình $x^4 - (2m - 1)x^2 + 2\sqrt{3} = 0$. (a) Giải phương trình với $m = 2.5$. (b) Tìm giá trị của $m\in\mathbb{R}$ để phương trình có đúng 2 nghiệm.
\end{baitoan}
Giải phương trình:

\begin{baitoan}[\cite{Binh_boi_duong_Toan_9_tap_2}, VD5, p. 62]
	(a) $(x^2 + 2x)(x^2 + 2x - 8) + 15 = 0$. (b) $x^2 - 4\sqrt{x^2 - 5x + 3} = 5x - 6$.
\end{baitoan}

\begin{baitoan}[\cite{Binh_boi_duong_Toan_9_tap_2}, 7.1., p. 62]
	(a) $x^2(x^2 - 0.8) = 0.9(1 + x^4)$. (b) $5x^4 + 9x^2 + 2015 = 0$. (c) $x^4 - 2x^2 + 3 = 2\sqrt{2}(x^2 - 1)$. (d) $(x^2 - 3)^2 + 3x^2 = 77 - (x^2 + 4)^2$.
\end{baitoan}

\begin{baitoan}[\cite{Binh_boi_duong_Toan_9_tap_2}, 7.2., p. 62]
	(a) $\dfrac{x^2}{x - 2} - \dfrac{x^2}{x + 2} = \dfrac{8x}{x^2 - 4}$. (b) $\dfrac{18}{x^4 - 16} + 2 = \dfrac{4}{x^2 - 4}$. (c) $\dfrac{x - 1}{x + 2} + \dfrac{x + 2}{x - 1} = \dfrac{6}{x^2 + x - 2}$.
\end{baitoan}

\begin{baitoan}[\cite{Binh_boi_duong_Toan_9_tap_2}, 7.3., p. 62]
	(a) $x^2(2x^2 - 5x + 1) - 18x^2 + 45x = 9$. (b) $(3x^2 - x + 2)^2 = 4x^2 - 12x + 9$. (c) $2x^3 - x^2 + 4x(2x - 1) = 2x - 1$.
\end{baitoan}

\begin{baitoan}[\cite{Binh_boi_duong_Toan_9_tap_2}, 7.4., p. 62]
	(a) $(x^2 + x + 1)^2 - 5(x^2 + x) + 1 = 0$. (b) $2\left(x^2 + \dfrac{1}{x^2}\right)^2 - 5\left(x^2 + \dfrac{1}{x^2}\right) + 2 = 0$. (c) $x^2 - 3x - 7\sqrt{x^2 - 3x - 10} = 0$. (d) $(x - 1)(x - 3)(x - 5)(x - 7) = -15$.
\end{baitoan}

\begin{baitoan}[\cite{Binh_boi_duong_Toan_9_tap_2}, 7.5., p. 63]
	Cho phương trình $x^4 + 2(m - 2)x^2 + m^2 = 0$. (a) Giải phương trình khi $m = -3$. (b) Tìm $m\in\mathbb{R}$ để phương trình có 4 nghiệm phân biệt. (c) Biện luận theo $m$ số nghiệm của phương trình.
\end{baitoan}
Giải phương trình:

\begin{baitoan}[\cite{Binh_boi_duong_Toan_9_tap_2}, 7.6., p. 63]
	(a) $x^4 - 8x^3 + 16x^2 = 9$. (b) $x^4 - 4x^2 + 20x = 25$.
\end{baitoan}

\begin{baitoan}[\cite{Binh_boi_duong_Toan_9_tap_2}, 7.7., p. 63]
	(a) $(x^2 - 4x + 3)(x^2 + 10x + 24) = -24$. (b) $(x - 3)(x^3 - 7x^2 + 19x - 18) = 10$. (c) $(x - 2)(x - 4)(x^2 - 4x + 8) = 15x^2$. (d) $(x^2 - 2x + 3)^2 - 9x^2(x^2 - 2x + 3) + 20x^4 = 0$.
\end{baitoan}

\begin{baitoan}[\cite{Binh_boi_duong_Toan_9_tap_2}, 7.8., p. 63]
	(a) $\dfrac{x^2 - 11}{2} + \dfrac{x^2 - 12}{3} + \dfrac{x^2 - 13}{4} + \dfrac{x^2 - 14}{5} = -4$. (b) $\dfrac{x^4 + x^2 - 1}{5} + \dfrac{x^4 + x^2 - 2}{4} + \dfrac{x^4 + x^2 - 3}{3} = \dfrac{137}{30}$. (c) $\dfrac{3}{x^2 + 2} + \dfrac{5}{x^2 + 4} + \dfrac{7}{x^2 + 6} = 3$.
\end{baitoan}

\begin{baitoan}[\cite{Binh_boi_duong_Toan_9_tap_2}, 7.9., p. 63]
	Cho $f_1(x) = x^2 + 3x + 2,f_2(x) = x^2 + 5x + 6,f_3(x) = x^2 + 7x + 12,\ldots,f_{99}(x) = x^2 + 199x + 9900$. Giải phương trình $\sum_{i=1}^{99} \dfrac{1}{f_i(x)} = \dfrac{1}{f_1(x)} + \dfrac{1}{f_2(x)} + \cdots + \dfrac{1}{f_{99}(x)} = \dfrac{99}{100}$.
\end{baitoan}
Tartaglia đã giải được phương trình bậc 3 có dạng $x^3 + px + q = 0$ với công thức nghiệm:
\begin{align*}
	x = \sqrt[3]{-\frac{q}{2} + \sqrt{\frac{q^2}{4} + \frac{p^3}{27}}} + \sqrt[3]{-\frac{q}{2} - \sqrt{\frac{q^2}{4} + \frac{p^3}{27}}}.
\end{align*}

\subsection{Phương trình bậc 4 có hệ số đối xứng}

\begin{baitoan}[\cite{Binh_boi_duong_Toan_9_tap_2}, p. 64, phương trình bậc 4 có hệ số đối xứng]
	Giải phương trình bằng cách chia 2 vế cho $x^2\ne0$ \& đặt ẩn phụ $y\coloneqq x + \dfrac{1}{x}$: (a) $2x^4 - 5x^3 + 4x^2 - 5x + 2 = 0$. (b) $ax^4 + bx^3 + cx^2 + bx + a = 0$, $a,b,c\in\mathbb{R},a\ne0$
\end{baitoan}

\begin{baitoan}[\cite{Tuyen_Toan_9_old}, VD40, p. 89]
	Giải phương trình $10x^4 - 27x^3 - 110x^2 - 27x + 10 = 0$.
\end{baitoan}

\subsection{Phương trình bậc 5 có hệ số đối xứng}

\begin{baitoan}[\cite{Binh_boi_duong_Toan_9_tap_2}, p. 65, phương trình bậc 5 có hệ số đối xứng]
	Biện luận theo 3 tham số $a,b,c\in\mathbb{R},a\ne0$ để giải phương trình bậc 5 có hệ số đối xứng $ax^5 + bx^4 + cx^3 + cx^2 + bx + a = 0$.
\end{baitoan}

\subsection{Phương trình hồi quy}

\begin{baitoan}[\cite{Binh_boi_duong_Toan_9_tap_2}, p. 65, phương trình hồi quy]
	(a) Biện luận theo 5 tham số $a,b,c,m,n\in\mathbb{R},a\ne0$ để giải phương trình hồi quy $ax^4 + bx^3 + cx^2 + mx + n = 0$ với $\dfrac{n}{a} = \left(\dfrac{m}{b}\right)^2$ bằng cách chia 2 vế cho $x^2\ne0$ rồi đặt ẩn phụ $y\coloneqq bx + \dfrac{m}{x}$. (b) Giải phương trình: $x^4 - 3x^3 - 8x^2 + 6x + 4 = 0$.
\end{baitoan}

\begin{baitoan}[\cite{Tuyen_Toan_9_old}, VD41, p. 91]
	Giải phương trình $x^4 - 4x^3 - 9x^2 + 8x + 4 = 0$.
\end{baitoan}

\subsection{Phương trình $(x + a)(x + b)(x + c)(x + d) = m$ với $a + d = b + c$}

\begin{baitoan}[\cite{Tuyen_Toan_9_old}, VD42, p. 92]
	Giải phương trình $(x + 1)(x + 3)(x + 5)(x + 7) = -15$.
\end{baitoan}

\subsection{Phương trình $(x + a)(x + b)(x + c)(x + d) = mx^2$ với $ad = bc$}

\begin{baitoan}[\cite{Tuyen_Toan_9_old}, VD43, p. 93]
	Giải phương trình $(x - 4)(x - 5)9x - 8)(x - 10) = 72x^2$.
\end{baitoan}

\subsection{Phương trình $(x + a)^4 + (x + b)^4 = c$}

\begin{baitoan}[\cite{Tuyen_Toan_9_old}, VD44, p. 94]
	Giải phương trình $(x + 3)^4 + (x - 1)^4 = 626$.
\end{baitoan}

\begin{baitoan}
	Giải \& biện luận phương trình $(x + a)^4 + (x + b)^4 = c$ với $a,b,c\in\mathbb{R}$ bằng cách đặt ẩn phụ $y = x + \dfrac{a + b}{2}$.
\end{baitoan}

\subsection{Miscellaneous}

\begin{baitoan}[\cite{Tuyen_Toan_9_old}, VD38, p. 85]
	Cho phương trình $x^4 - (3m - 2)x^2 + 1 = 0$. (a) Giải phương trình với $m = 2$. (b) Tìm $m\in\mathbb{R}$ để phương trình có đúng 2 nghiệm.
\end{baitoan}

\begin{baitoan}[\cite{Tuyen_Toan_9_old}, VD39, p. 86]
	Giải phương trình $(x^2 + 16x + 60)(x^2 + 17x + 60) = 6x^2$.
\end{baitoan}

\begin{baitoan}[\cite{Tuyen_Toan_9_old}, 234., p. 87]
	Giải phương trình: (a) $x^3 + x^2 - 8x - 6 = 0$. (b) $x^6 + 61x^3 - 8000 = 0$.
\end{baitoan}

\begin{baitoan}[\cite{Tuyen_Toan_9_old}, 235., p. 87]
	Cho phương trình $mx^4 + 2(m - 2)x^2 + m = 0$.  Tìm $m\in\mathbb{R}$ để phương trình có: (a) $4$ nghiệm. (b) $2$ nghiệm. (c) Biện luận số nghiệm của phương trình theo $m$.
\end{baitoan}

\begin{baitoan}[\cite{Tuyen_Toan_9_old}, 236., pp. 87--88]
	Cho phương trình $x^4 - 2(m + 1)x^2 + 2m + 1 = 0$. Tìm $m\in\mathbb{R}$ để phương trình có 4 nghiệm $x_1,x_2,x_3,x_4$ sao cho khi biểu diễn 4 nghiệm đó trên trục số thì 4 điểm đó chắn trên trục hoành thành 3 đoạn thẳng bằng nhau.
\end{baitoan}
Giải phương trình:

\begin{baitoan}[\cite{Tuyen_Toan_9_old}, 237., p. 88]
	(a) $(x^2 + 5x + 8)(x^2 + 6x + 8) = 2x^2$. (b) $(x - 9)(x - 10)(x - 11) - 8x = 0$.
\end{baitoan}

\begin{baitoan}[\cite{Tuyen_Toan_9_old}, 238., p. 88]
	$\dfrac{1}{x^2 + 4x + 3} + \dfrac{1}{x^2 + 8x + 15} + \dfrac{1}{x^2 + 12x + 35} + \dfrac{1}{x^2 + 16x + 63}  = \dfrac{1}{5}$.
\end{baitoan}

\begin{baitoan}[\cite{Tuyen_Toan_9_old}, 239., p. 88]
	$\dfrac{x + 5}{3} - \dfrac{x - 3}{5} = \dfrac{5}{x - 3} - \dfrac{3}{x + 5}$.
\end{baitoan}

\begin{baitoan}[\cite{Tuyen_Toan_9_old}, 240., p. 88]
	$x^2 + \dfrac{9x^2}{(x + 3)^2} = 40$.
\end{baitoan}

\begin{baitoan}[\cite{Tuyen_Toan_9_old}, 241., p. 88]
	$\dfrac{x^2}{2} + \dfrac{18}{x^2} = 13\left(\dfrac{x}{2} - \dfrac{3}{x}\right)$.
\end{baitoan}

\begin{baitoan}[\cite{Tuyen_Toan_9_old}, 242., p. 88]
	$\sqrt{13 - \sqrt{13 + x}} = x$.
\end{baitoan}

\begin{baitoan}[\cite{Tuyen_Toan_9_old}, 243., p. 88]
	$\sqrt{8 + \sqrt{x - 3}} + \sqrt{5 - \sqrt{x - 3}} = 5$.
\end{baitoan}

\begin{baitoan}[\cite{Tuyen_Toan_9_old}, 244., p. 88]
	$\sqrt{x - 1 + 4\sqrt{x - 5}} + \sqrt{x - 1 - 4\sqrt{x - 5}} = 2(x - 17)$.
\end{baitoan}

\begin{baitoan}[\cite{Tuyen_Toan_9_old}, 245., p. 88]
	$8\sqrt[3]{(x - 1)^2} - \sqrt[3]{(x + 1)^2} = 2\sqrt[3]{(x^2 - 1)}$.
\end{baitoan}

\begin{baitoan}[\cite{Tuyen_Toan_9_old}, 246., p. 88]
	Cho $x,y > 0$. Biết tổng của chúng bằng $6$ lần trung bình nhân của chúng. Tính $\dfrac{x}{y}$.
\end{baitoan}

\begin{baitoan}[\cite{Tuyen_Toan_9_old}, 247., p. 88]
	Tìm $m\in\mathbb{R}$ để phương trình $\sqrt{x - 6\sqrt{x - 9}} + x + \sqrt{x - 9} = m$ có nghiệm.
\end{baitoan}

\begin{baitoan}[\cite{Tuyen_Toan_9_old}, 248., p. 94]
	(a) $2x^4 - 13x^3 + 24x^2 - 13x + 2 = 0$. (b) $x^4 - 10x^3 + 11x^2 - 10x + 1 = 0$. (c) $3x^5 - 10x^4 + 3x^3 + 3x^2 - 10x + 3 = 0$.
\end{baitoan}

\begin{baitoan}[\cite{Tuyen_Toan_9_old}, 249., p. 94]
	(a) $x^4 + 5x^3 + 10x^2 + 15x + 9 = 0$. (b) $x^4 + 5x^3 - 14x^2 - 20x + 16 = 0$.
\end{baitoan}

\begin{baitoan}[\cite{Tuyen_Toan_9_old}, 250., pp. 94--95]
	Cho phương trình $3x^4 - mx^3 - 16x^2 + mx + 3 = 0$. (a) Chứng minh phương trình luôn có nghiệm. (b) Giải phương trình với $m = 7$.
\end{baitoan}

\begin{baitoan}[\cite{Tuyen_Toan_9_old}, 251., p. 95]
	(a) $(x + 2)(x + 5)(x - 6)(x - 9) = 280$. (b) $(x^2 + 7x + 12)(x^2 - 15x + 56) = 180$.
\end{baitoan}

\begin{baitoan}[\cite{Tuyen_Toan_9_old}, 252., p. 95]
	$(x + 10)(x + 12)(x + 15)(x + 18) = 2x^2$. (b) $(x - 90)(x - 35)(x + 18)(x + 7) = -1080x^2$.
\end{baitoan}

\begin{baitoan}[\cite{Tuyen_Toan_9_old}, 253., p. 95]
	(a) $(x - 2.5)^4 + (x - 1.5)^4 = 17$. (b) $(x + 5)^4 - (x + 1)^4 = 80$.
\end{baitoan}

\begin{baitoan}[\cite{Binh_Toan_9_tap_2}, VD85, p. 34]
	$x^3 + 2x^2 + 2\sqrt{2}x + 2\sqrt{2} = 0$.
\end{baitoan}

\begin{baitoan}[\cite{Binh_Toan_9_tap_2}, VD86, p. 35]
	$\sqrt{2}x^3 + 3x^2 - 2 = 0$.
\end{baitoan}

\begin{baitoan}[\cite{Binh_Toan_9_tap_2}, VD87, p. 35]
	$(x + 1)^4 = 2(x^4 + 1)$.
\end{baitoan}

\begin{baitoan}[\cite{Binh_Toan_9_tap_2}, VD88, p. 36]
	$4(x + 5)(x + 6)(x + 10)(x + 12) = 3x^2$.
\end{baitoan}

\begin{baitoan}[\cite{Binh_Toan_9_tap_2}, VD89, p. 37]
	$x^4 = 24x + 32$.
\end{baitoan}

\begin{baitoan}[\cite{Binh_Toan_9_tap_2}, VD90, p. 37]
	$x^3 + 3x^2 - 3x + 1 = 0$.
\end{baitoan}

\begin{baitoan}[\cite{Binh_Toan_9_tap_2}, VD91, p. 38]
	$|x - 8|^5 + |x - 9|^6 = 1$.
\end{baitoan}

\begin{baitoan}[\cite{Binh_Toan_9_tap_2}, VD92, p. 38]
	$|x^2 - x + 1| + |x^2 - x - 2| = 3$.
\end{baitoan}

\begin{baitoan}[\cite{Binh_Toan_9_tap_2}, VD93, p. 39]
	$\dfrac{2x}{3x^2 - x + 2} - \dfrac{7x}{3x^2 + 5x + 2} = 1$.
\end{baitoan}

\begin{baitoan}[\cite{Binh_Toan_9_tap_2}, VD94, p. 40]
	$x^2 + \dfrac{4x^2}{(x + 2)^2} = 12$.
\end{baitoan}

\begin{baitoan}[\cite{Binh_Toan_9_tap_2}, VD95, p. 40]
	$20\left(\dfrac{x - 2}{x + 1}\right)^2 - 5\left(\dfrac{x + 2}{x - 1}\right)^2 + 48\cdot\dfrac{x^2 - 4}{x^2 - 1} = 0$.
\end{baitoan}

\begin{baitoan}[\cite{Binh_Toan_9_tap_2}, VD96, p. 41]
	$\dfrac{x}{\sqrt{4x - 1}} + \dfrac{\sqrt{4x - 1}}{x} = 2$.
\end{baitoan}

\begin{baitoan}[\cite{Binh_Toan_9_tap_2}, VD97, p. 41]
	$x + \sqrt{x + \dfrac{1}{2} + \sqrt{x + \dfrac{1}{4}}} = 2$.
\end{baitoan}

\begin{baitoan}[\cite{Binh_Toan_9_tap_2}, VD98, p. 42]
	$\dfrac{1}{x} + \dfrac{1}{\sqrt{2 - x^2}} = 2$.
\end{baitoan}

\begin{baitoan}[\cite{Binh_Toan_9_tap_2}, VD99, p. 42]
	Giải \& biện luận phương trình $a\sqrt{a - \sqrt{a + x}} = x$ với tham số $a$.
\end{baitoan}

\begin{baitoan}[\cite{Binh_Toan_9_tap_2}, VD100, p. 43]
	Tìm các giá trị của $m\in\mathbb{R}$ để tồn tại 2 số $x,y\in\mathbb{R}$ thỏa $4x - 3y = 7,2x^2 + 5y^2 = m$.
\end{baitoan}
Giải hệ phương trình:

\begin{baitoan}[\cite{Binh_Toan_9_tap_2}, VD101, p. 44]	
	$x^2 + y^2 = 11,x + xy + y = 3 + 4\sqrt{2}$.
\end{baitoan}

\begin{baitoan}[\cite{Binh_Toan_9_tap_2}, VD102, p. 44]
	$x^2 + y + \frac{1}{4} = 0,x + y^2 + \frac{1}{4} = 0$.
\end{baitoan}

\begin{baitoan}[\cite{Binh_Toan_9_tap_2}, VD103, p. 45]
	$x^2 - xy + y^2 = 1,2x^2 - 3xy + 4y^2 = 3$.
\end{baitoan}

\begin{baitoan}[\cite{Binh_Toan_9_tap_2}, VD104, p. 46]
	$x + y + z = 9,x^2 + y^2 + z^2 = 27$.
\end{baitoan}

\begin{baitoan}[\cite{Binh_Toan_9_tap_2}, VD105, p. 46]
	$x + y + z = a,x^2 + y^2 + z^2 = a^2,x^3 + y^3 + z^3 = a^3$.
\end{baitoan}

\begin{baitoan}[\cite{Binh_Toan_9_tap_2}, VD106, p. 47]
	$x + \dfrac{1}{y} = 2,y + \dfrac{1}{z} = 2,z + \dfrac{1}{z} = 2$.
\end{baitoan}

\subsection{Phương trình đại số bậc cao}
Giải phương trình:

\begin{baitoan}[\cite{Binh_Toan_9_tap_2}, 282., p. 47]
	(a) $x^3 - 3x^2 + x + 1 = 0$. (b) $x^3 - 5x^2 + x + 7 = 0$. (c) $x^3 + 2x - 5\sqrt{3} = 0$. (d) $x^3 - x - \sqrt{2} = 0$. (e) $(x - 2)^2 + (x + 1)^3 = 8x^3 - 1$.
\end{baitoan}

\begin{baitoan}[\cite{Binh_Toan_9_tap_2}, 283., p. 48]
	(a) $x^4 - 2x^3 - x^2 - 2x + 1 = 0$. (b) $x^4 - 10x^3 + 26x^2 - 10x + 1 = 0$. (c) $x^4 + x^3 - 4x^2 + x + 1 = 0$. (d) $2x^4 + x^3 - 11x^2 + x + 2 = 0$. (e) $x^4 - 7x^3 + 14x^2 - 7x + 1 = 0$. (f) $x^4 + x^3 - 10x^2 + x + 1 = 0$.
\end{baitoan}

\begin{baitoan}[\cite{Binh_Toan_9_tap_2}, 284., p. 48]
	(a) $x^4 - 3x^3 - 6x^2 + 3x + 1 = 0$. (b) $x^4 - 3x^3 + 3x + 1 = 0$. (c) $x^4 + 3x^3 - 14x^2 - 6x + 4 = 0$.
\end{baitoan}

\begin{baitoan}[\cite{Binh_Toan_9_tap_2}, 285., p. 48]
	$6x^5 - 11x^4 - 11x + 6 = 0$.
\end{baitoan}

\begin{baitoan}[\cite{Binh_Toan_9_tap_2}, 286., p. 48]
	(a) $x^4 + 9 = 5x(x^2 - 3)$. (b) $(x^2 - 6x - 9)^2 = x(x^2 - 4x - 9)$.
\end{baitoan}

\begin{baitoan}[\cite{Binh_Toan_9_tap_2}, 287., p. 48]
	(a) $(x^2 - 2x + 4)(x^2 + 3x + 4) = 14x^2$. (b) $(2x^2 - 3x + 1)(2x^2 + 5x + 1) = 9x^2$.
\end{baitoan}

\begin{baitoan}[\cite{Binh_Toan_9_tap_2}, 288., p. 48]
	(a) $4\sqrt{2}x^3 - 22x^2 + 17\sqrt{2}x - 6 = 0$. (b) $x^4 - 12x^2 + 16\sqrt{2}x - 12 = 0$.
\end{baitoan}

\begin{baitoan}[\cite{Binh_Toan_9_tap_2}, 289., p. 48]
	(a) $x(x + 1)(x + 2)(x + 3) = 8$. (b) $x(x - 1)(x + 1)(x + 2) = 3$. (c) $(x + 2)(x + 3)(x - 7)(x - 8) = 144$. (d) $(x + 5)(x + 6)(x + 8)(x + 9) = 40$. (e) $(4x + 3)^2(x + 1)(2x + 1) = 810$. (f) $(6x + 5)^2(3x + 2)(x + 1) = 35$.
\end{baitoan}

\begin{baitoan}[\cite{Binh_Toan_9_tap_2}, 290., p. 48]
	(a) $4(x^2 - x + 1)^3 = 27(x^2 - x)^2$. (b) $3(x + 5)(x + 6)(x + 7) = 8x$.
\end{baitoan}

\begin{baitoan}[\cite{Binh_Toan_9_tap_2}, 291., p. 48]
	(a) $(x - 2)^3 + (x - 4)^3 = 8$. (b) $(x + 2)^4 + (x + 4)^4 = 82$. (c) $(x + 2)^4 + (x + 8)^4 = 272$. (d) $(x - 2)^6 + (x - 4)^6 = 64$.
\end{baitoan}

\begin{baitoan}[\cite{Binh_Toan_9_tap_2}, 292., p. 48]
	(a) $(x^2 - 6x)^2 - 2(x - 3)^2 = 81$. (b) $x^4 + (x - 1)(3x^2 + 2x - 2) = 0$. (c) $x^4 + (x + 1)(5x^2 - 6x - 6) = 0$. (d) $(x^2 + 1)^2 + (x + 2)(3x^2 - 4x - 5) = 0$. (e) $x^2(x - 1)^2 + x(x^2 - 1) = 2(x + 1)^2$.
\end{baitoan}

\begin{baitoan}[\cite{Binh_Toan_9_tap_2}, 293., p. 48]
	$x^5 + x^2 + 2x + 2 = 0$.
\end{baitoan}

\begin{baitoan}[\cite{Binh_Toan_9_tap_2}, 294., p. 49]
	(a) $x^4 - x^2 + 2x - 1 = 0$. (b) $x^4 - 9x^2 + 24x - 16 = 0$. (c) $x^4 = 2x^2 + 8x = 3$. (d) $(x^2 - 16)^2 = 16x + 1$. (e) $(x^2 - a^2)^2 = 4ax + 1$. (f) $x^4 = 4x - 3$. (g) $x^4 = 2x^2 - 12x + 8$.
\end{baitoan}

\begin{baitoan}[\cite{Binh_Toan_9_tap_2}, 295., p. 49]
	(a) $x^4 = 4x + 1$. (b) $x^4 = 8x + 7$. (c) $x^3 - 3x^2 + 9x - 9 = 0$. (d) $x^3 - x^2 - x = \dfrac{1}{3}$.
\end{baitoan}

\begin{baitoan}[\cite{Binh_Toan_9_tap_2}, 296., p. 49]
	$(x + 2)^2 + (x + 3)^3 + (x + 4)^4 = 2$.
\end{baitoan}

\begin{baitoan}[\cite{Binh_Toan_9_tap_2}, 297., p. 49]
	$(x - \sqrt{2})^3 + (x + \sqrt{3})^3 + (\sqrt{2} - \sqrt{3} - 2x)^3 = 0$.
\end{baitoan}

\begin{baitoan}[\cite{Binh_Toan_9_tap_2}, 298., p. 49]
	$x^3 - 3abx + a^3 + b^3 = 0$ với 2 tham số $a,b$.
\end{baitoan}

\begin{baitoan}[\cite{Binh_Toan_9_tap_2}, 299., p. 49]
	$(a + b + x)^3 - 4(a^3 + b^3 + x^3) - 12abx = 0$ với 2 tham số $a,b$.
\end{baitoan}

\begin{baitoan}[\cite{Binh_Toan_9_tap_2}, 300., p. 49]
	Giải phương trình $x^3 - (m^2 - m + 7)x - 3(m^2 - m - 2) = 0$ biết $-1$ là 1 nghiệm của phương trình.
\end{baitoan}

\begin{baitoan}[\cite{Binh_Toan_9_tap_2}, 301., p. 49]
	Giải phương trình $x^3 + ax^2 + bx + c = 0$ biết $a,b\in\mathbb{Q}$, $\sqrt{2}$ là 1 nghiệm của phương trình.
\end{baitoan}

\begin{baitoan}[\cite{Binh_Toan_9_tap_2}, 302., p. 49]
	Giải phương trình $x^5 + ax^3 + bx^2 + 5x + 2 = 0$ biết $a,b\in\mathbb{Q}$, $1 + \sqrt{2}$ là 1 nghiệm của phương trình.
\end{baitoan}

\begin{baitoan}[\cite{Binh_Toan_9_tap_2}, 303., p. 49]
	Giải phương trình $4x^4 - 11x^2 + 9x + m = 0$ biết tồn tại 2 nghiệm $x_1,x_2$ của phương trình thỏa $x_1 + x_2 = -1$, $x_1 > x_2$.
\end{baitoan}

\begin{baitoan}[\cite{Binh_Toan_9_tap_2}, 304., p. 49]
	(a) Chứng minh nếu $x = \dfrac{1}{2}\left(a - \dfrac{1}{a}\right)$ thì $4x^3 + 3x = \dfrac{1}{2}\left(a^3 - \dfrac{1}{a^3}\right)$. (b) Giải phương trình $4x^3 + 3x = \dfrac{3}{4}$. (c) Giải phương trình $4x^3 + 3x = m\in\mathbb{R}$.
\end{baitoan}

\begin{baitoan}[\cite{Binh_Toan_9_tap_2}, 305., pp. 49--50, định lý Vi\`ete cho phương trình bậc 3]
	Chứng minh: (a) Nếu phương trình bậc 3 $ax^3 + bx^2 + cx + d = 0,a\ne0$ có 3 nghiệm thực $x_1,x_2,x_3$ thì:
	\begin{equation*}
		\left\{\begin{split}
			x_1 + x_2 + x_3 &= -\frac{b}{a},\\
			x_1x_2 + x_2x_3 + x_3x_1 &= \frac{c}{a},\\
			x_1x_2x_3 &= -\dfrac{d}{a}.
		\end{split}\right.
	\end{equation*}
	(b) Tìm các nghiệm của phương trình $x^3 - 9x^2 + 26x - 24 = 0$ rồi kiểm nghiệm lại chúng thỏa mãn định lý Vi\`ete cho phương trình bậc 3.
\end{baitoan}

\subsection{Phương trình chứa ẩn ở mẫu thức}
Giải phương trình:

\begin{baitoan}[\cite{Binh_Toan_9_tap_2}, 306., p. 50]
	(a) $\dfrac{1}{x^2 - 3x + 3} + \dfrac{2}{x^2 - 3x + 4} = \dfrac{6}{x^2 - 3x + 5}$. (b) $\dfrac{1}{x^2 - 2x + 2} + \dfrac{1}{x^2 - 2x + 3} = \dfrac{9}{2(x^2 - 2x + 4)}$. (c) $\dfrac{6}{(x + 1)(x + 2)} + \dfrac{8}{(x - 1)(x + 4)} = 1$. (d) $\dfrac{x^2 + 2x + 1}{x^2 + 2x + 2} + \dfrac{x^2 + 2x + 2 }{x^2 + 2x + 3} = \dfrac{7}{6}$.
\end{baitoan}

\begin{baitoan}[\cite{Binh_Toan_9_tap_2}, 307., p. 50]
	(a) $x^2 + \dfrac{81x^2}{(x + 9)^2} = 40$. (b) $x^2 + \dfrac{x^2}{(x + 1)^2} = 15$. (c) $\dfrac{x^4}{2x^2 + 1} + \dfrac{2x^2 + 1}{x^4} = 2$.
\end{baitoan}

\begin{baitoan}[\cite{Binh_Toan_9_tap_2}, 308., p. 50]
	(a) $4\left(x^3 + \dfrac{1}{x^3}\right) = 13\left(x + \dfrac{1}{x}\right)$. (b) $x^3 + \dfrac{1}{x^3} = 13\left(x + \dfrac{1}{x}\right)$.
\end{baitoan}

\begin{baitoan}[\cite{Binh_Toan_9_tap_2}, 309., p. 50]
	(a) $\dfrac{4x}{4x^2 - 8x + 7} + \dfrac{3x}{4x^2 - 10x + 7} = 1$. (b) $\dfrac{2x}{2x^2 - 5x + 3} + \dfrac{13x}{2x^2 + x + 3} = 6$. (b) $\dfrac{2x}{2x^2 - 5x + 3} + \dfrac{13x}{2x^2 + x + 3} = 6$. (c) $\dfrac{3x}{x^2 - 3x + 1} + \dfrac{7x}{x^2 + x + 1} = -4$.
\end{baitoan}

\begin{baitoan}[\cite{Binh_Toan_9_tap_2}, 310., p. 51]
	(a) $\dfrac{x^2 - 10x + 15}{x^2 - 6x + 15} = \dfrac{4x}{x^2 - 12x + 15}$. (b) $\dfrac{x^2 - 3x + 5}{x^2 - 4x + 5} - \dfrac{x^2 - 5x + 5}{x^2 - 6x + 5} = -\dfrac{1}{4}$.
\end{baitoan}

\begin{baitoan}[\cite{Binh_Toan_9_tap_2}, 311., p. 51]
	(a) $x^2 + \dfrac{4x^2}{(x + 2)^2} = 5$. (b) $x^2 + \dfrac{25x^2}{(x + 5)^2} = 11$.
\end{baitoan}

\begin{baitoan}[\cite{Binh_Toan_9_tap_2}, 312., p. 51]
	(a) $\left(\dfrac{x - 1}{x}\right)^2 + \left(\dfrac{x - 1}{x - 2}\right)^2 = \dfrac{40}{9}$. (b) $\left(\dfrac{x + 2}{x + 1}\right)^2 + \left(\dfrac{x - 2}{x - 1}\right)^2 - \dfrac{5}{2}\cdot\dfrac{x^2 - 4}{x^2 - 1} = 0$.
\end{baitoan}

\begin{baitoan}[\cite{Binh_Toan_9_tap_2}, 313., p. 51]
	(a) $\dfrac{x(3 - x)}{x + 1}\left(x + \dfrac{3 - x}{x + 1}\right) = 2$. (b) $\dfrac{x(5 - x)}{x + 1}\left(x + \dfrac{5 - x}{x + 1}\right) = 6$. (c) $x\cdot\dfrac{8 - x}{x - 1}\left(x - \dfrac{8 - x}{x - 1}\right) = 15$.
\end{baitoan}

\begin{baitoan}[\cite{Binh_Toan_9_tap_2}, 314., p. 51]
	$\dfrac{1}{x} + \dfrac{1}{x + 2} + \dfrac{1}{x + 5} + \dfrac{1}{x + 7} = \dfrac{1}{x + 1} + \dfrac{1}{x + 3} + \dfrac{1}{x + 4} + \dfrac{1}{x + 6}$.
\end{baitoan}

\begin{baitoan}[\cite{Binh_Toan_9_tap_2}, 315., p. 51]
	$\dfrac{(1995 - x)^2 + (1995 - x)(x - 1996) + (x - 1996)^2}{(1995 - x)^2 - (1995 - x)(x - 1996) + (x - 1996)^2} = \dfrac{19}{49}$.
\end{baitoan}

\subsection{Phương trình vô tỷ}
Giải phương trình:

\begin{baitoan}[\cite{Binh_Toan_9_tap_2}, 316., p. 51]
	(a) $x^2 - 4x = 8\sqrt{x - 1}$. (b) $x^2 + \sqrt{x + 72} = 72$.
\end{baitoan}

\begin{baitoan}[\cite{Binh_Toan_9_tap_2}, 317., p. 51]
	(a) $2\sqrt[3]{2x - 1} = x^3 + 1$. (b) $5\sqrt{x^3 + 1} = 2(x^2 + 2)$.
\end{baitoan}

\begin{baitoan}[\cite{Binh_Toan_9_tap_2}, 318., p. 51]
	(a) $\sqrt{x^2 - \dfrac{7}{x^2}} + \sqrt{x - \dfrac{7}{x^2}} = x$. (b) $\sqrt{x - \dfrac{1}{x}} + \sqrt{1 - \dfrac{1}{x}} = x$.
\end{baitoan}

\begin{baitoan}[\cite{Binh_Toan_9_tap_2}, 319., p. 51]
	$(x - 1)(x + 3) + 2(x - 1)\sqrt{\dfrac{x + 3}{x - 1}} = 8$.
\end{baitoan}

\begin{baitoan}[\cite{Binh_Toan_9_tap_2}, 320., p. 51]
	$\sqrt{x - 2a + 16} - 2\sqrt{x - a + 4} + \sqrt{x} = 0$ với tham số $a$.
\end{baitoan}

\begin{baitoan}[\cite{Binh_Toan_9_tap_2}, 321., p. 52]
	$\dfrac{\sqrt{1 + x} + \sqrt{1 - x}}{\sqrt{1 + x} - \sqrt{1 - x}} = \sqrt{a}$ với tham số $a$.
\end{baitoan}

\begin{baitoan}[\cite{Binh_Toan_9_tap_2}, 322., p. 52]
	Tìm $x,y\in\mathbb{Q},x > y\ge0$ thỏa mãn phương trình $\sqrt{x} - \sqrt{y} = \sqrt{2 - \sqrt{3}}$.
\end{baitoan}

\begin{baitoan}[\cite{Binh_Toan_9_tap_2}, 323., p. 52]
	Cho $\left(x + \sqrt{x^2 + 3}\right)\left(y + \sqrt{y^2 + 3}\right) = 3$. Tính giá trị biểu thức $A = x + y$.
\end{baitoan}

\subsection{Miscellaneous}
Giải hệ phương trình:

\begin{baitoan}[\cite{Binh_Toan_9_tap_2}, 324., p. 52]
	(a) $x^2 + 4y^2 + x = 4xy + 2y + 2,4x^2 + 4xy + y^2 = 2x + y + 56$. (b) $\dfrac{x + y}{x - y} + \dfrac{x - y}{x + y} = \dfrac{26}{5}$, $xy = 6$. (c) $3x + 5y = 9 + 2xy,2x + 3y = 10 - xy$. (d) $x^2 - 4xy + y^2 = 1,y^2 - 3xy = 4$.
\end{baitoan}

\begin{baitoan}[\cite{Binh_Toan_9_tap_2}, 325., p. 52]
	(a) $x + y = 1,x^2 + y^2 = 41$. (b) $x - y = 1,x^2 - xy + y^2 = 7$. (c) $x - y = 3,x^2 + xy + y^2 = 21$. (d) $x - y = 2,x^3 - y^3 = 26$. (e) $x - y = a,x^3 - y^3 = 19a^3$ với $a > 0$.
\end{baitoan}

\begin{baitoan}[\cite{Binh_Toan_9_tap_2}, 326., p. 52]
	(a) $x^2 + 2y + 1 = 0,y^2 + 2x + 1 = 0$. (b) $x^2 - 3x = 2y,y^2 - 3y = 2x$. (c) $2x = y(1 - x^2),2y = x(1 - y^2)$.
\end{baitoan}

\begin{baitoan}[\cite{Binh_Toan_9_tap_2}, 327., p. 52]
	(a) $x^2 + (x + y)^2 = 17,y^2 + (x + y)^2 = 25$. (b) $x^2 + 2xy - 2y^2 = 1,2x^2 - xy + 3y^2 = 4$.
\end{baitoan}

\begin{baitoan}[\cite{Binh_Toan_9_tap_2}, 328., p. 52]
	(a) $2x^2 - y^2 = 1,xy + x^2 = 2$. (b) $x^2 + y^2 = 5,x + y - xy = 1$.
\end{baitoan}

\begin{baitoan}[\cite{Binh_Toan_9_tap_2}, 329., p. 52]
	(a) $x + y = 4,x^4 + y^4 = 82$. (b) $x + y + xy = 8,x^4 + y^4 = 32$.
\end{baitoan}

\begin{baitoan}[\cite{Binh_Toan_9_tap_2}, 330., p. 52]
	(a) $x^2 + y^2 + z^2 = 12,xy + yz + zx = 12$. (b) $x^2 + y^2 + z^2 = 3,x + y + z = 3$. (c) $x^2 + y^2 + z^2 = 1,x^3 + y^3 + z^3 = 1$.
\end{baitoan}

\begin{baitoan}[\cite{Binh_Toan_9_tap_2}, 331., p. 53]
	(a) $\dfrac{1}{x} + \dfrac{1}{y + z} = \dfrac{1}{3},\dfrac{1}{y} + \dfrac{1}{z + x} = \dfrac{1}{4},\dfrac{1}{z} + \dfrac{1}{x + y} = \dfrac{1}{5}$. (b) $x + y + z = 8,xy + yz + zx = 20,xyz = 16$. (c) $x + xy + y = 1,y + yz + z = 3,z + zx + x = 7$. (d) $x - \dfrac{1}{y} = 1.y - \dfrac{1}{z} = 1,z - \dfrac{1}{x} = 1$.
\end{baitoan}

\begin{baitoan}[\cite{Binh_Toan_9_tap_2}, 332., p. 53]
	$\dfrac{4x^2}{1 + 4x^2} = y,\dfrac{4y^2}{1 + 4y^2} = z,\dfrac{4z^2}{1 + 4z^2} = x$.
\end{baitoan}

\begin{baitoan}[\cite{Binh_Toan_9_tap_2}, 333., p. 53]
	$\sqrt{x}(1 + y) = 2y,\sqrt{y}(1 + z) = 2z,\sqrt{z}(1 + x) = 2x$.
\end{baitoan}

\begin{baitoan}[\cite{Binh_Toan_9_tap_2}, 334., p. 53]
	Cho $x,y,z\in\mathbb{R}$ thỏa $x^3 - y^2 - y = \dfrac{1}{3},y^3 - z^2 - z = \dfrac{1}{3},z^2 - x^2 - x = \dfrac{1}{3}$. (a) Chứng minh $x,y,z > 0$. (b) Chứng minh $x = y = z$. (c) Giải hệ phương trình.
\end{baitoan}

\begin{baitoan}[\cite{Binh_Toan_9_tap_2}, 335., p. 53]
	Cho $x,y,z\in\mathbb{R}$ thỏa $x^2 = y + 1,y^2 = z + 1,z^2 = x + 1$. (a) Chứng minh $xyz\ne0$. (b) Chứng minh $x,y,z$ cùng dấu. (c) Chứng minh $x = y = z$. (d) Giải hệ phương trình.
\end{baitoan}

\begin{baitoan}[\cite{Binh_Toan_9_tap_2}, 336., p. 53]
	Tìm 4 số thực dương sao cho mỗi số bằng bình phương của tổng 3 số còn lại.
\end{baitoan}

\begin{baitoan}[\cite{Binh_Toan_9_tap_2}, 337., p. 53]
	Tìm 4 số biết nếu cộng tích của 3 số bất kỳ với số còn lại thì mỗi kết quả đều bằng $2$.
\end{baitoan}

\begin{baitoan}[\cite{TLCT_THCS_Toan_9_dai_so}, VD14.1, p. 82]
	Giải phương trình $x^4 - 4x^3 - 1 = 0$.
\end{baitoan}

\begin{baitoan}[\cite{TLCT_THCS_Toan_9_dai_so}, VD14.2, p. 82]
	Tìm $m\in\mathbb{R}$ để phương trình $x^4 - 2(m + 1)x^2 + 2m + 1 = 0$ có 4 nghiệm phân biệt $x_1,x_2,x_3,x_4$ thỏa $x_2 - x_1 = x_3 - x_2 = x_4 - x_3 > 0$.
\end{baitoan}

\begin{baitoan}[\cite{TLCT_THCS_Toan_9_dai_so}, VD14.3, p. 83]
	Giải phương trình $x^4 + 3x^3 - 2x^2 - 6x + 4 = 0$.
\end{baitoan}

\begin{baitoan}[\cite{TLCT_THCS_Toan_9_dai_so}, VD14.4, p. 83]
	Chứng minh nếu phương trình $x^4 + ax^3 + bx^2 + ax + 1 = 0$ có nghiệm thực thì $a^2 + b^2\ge\dfrac{4}{5},2a^2 + b^2\ge\dfrac{4}{3}$.
\end{baitoan}
Giải phương trình:

\begin{baitoan}[\cite{TLCT_THCS_Toan_9_dai_so}, VD14.5, p. 84]
	$\sqrt{2x^2 + 15x - 17} = x + 3$.
\end{baitoan}

\begin{baitoan}[\cite{TLCT_THCS_Toan_9_dai_so}, VD14.6, p. 85]
	$\sqrt{3x^2 - 2x + 15} + \sqrt{3x^2 - 2x + 8} = 7$.
\end{baitoan}

\begin{baitoan}[\cite{TLCT_THCS_Toan_9_dai_so}, VD14.7, p. 85]
	$x^4 + 4x^3 + 6x^2 + 4x + \sqrt{x^2 + 2x + 17} = 3$.
\end{baitoan}

\begin{baitoan}[\cite{TLCT_THCS_Toan_9_dai_so}, VD14.8, p. 86]
	$3x - 2|x - 2| = 3\sqrt{3x + 18} - 2|\sqrt{3x + 18} - 2|.$
 \end{baitoan}

\begin{baitoan}[\cite{TLCT_THCS_Toan_9_dai_so}, VD14.9, p. 86]
	$\sqrt[3]{25x(2x^2 + 9)} = 4x + \dfrac{3}{x}$.
\end{baitoan}

\begin{baitoan}[\cite{TLCT_THCS_Toan_9_dai_so}, VD14.1, p. 86]
	Giải phương trình $x(x + 1)(x + 2)(x + 3) = 24$.
\end{baitoan}

\begin{baitoan}[\cite{TLCT_THCS_Toan_9_dai_so}, VD14.2, p. 86]
	Giải phương trình $(x - 1)^5 + (x + 3)^4 = 32$.
\end{baitoan}

\begin{baitoan}[\cite{TLCT_THCS_Toan_9_dai_so}, VD14.3, p. 86]
	Tìm quan hệ giữa $a,b,c\in\mathbb{R}$ để phương trình $(x + a)^4 + (x + b)^4 = c$ có nghiệm.
\end{baitoan}

\begin{baitoan}[\cite{TLCT_THCS_Toan_9_dai_so}, VD14.4, p. 86]
	Chứng minh nếu phương trình $x^4 + ax^3 + bx^2 + ax + 1 = 0$ có nghiệm thì $a^2 + 8\ge4b$.
\end{baitoan}

\begin{baitoan}[\cite{TLCT_THCS_Toan_9_dai_so}, VD14.5, p. 86]
	Giải \& biện luận phương trình $mx^4 + 5x^2 - 1 = 0$.
\end{baitoan}

\begin{baitoan}[\cite{TLCT_THCS_Toan_9_dai_so}, VD14.6, p. 87]
	Tìm $m\in\mathbb{R}$ để phương trình $x^4 - (3m + 4)x^2 + 12m = 0$ có 4 nghiệm phân biệt $x_1,x_2,x_3,x_4$ thỏa mãn $x_2 - x_1 = x_3 - x_2 = x_4 - x_3 > 0$.
\end{baitoan}

\begin{baitoan}[\cite{TLCT_THCS_Toan_9_dai_so}, VD14.7, p. 87]
	Giải phương trình $x^4 + 2x^3 - 8x^2 + 2x + 1 = 0$.
\end{baitoan}

\begin{baitoan}[\cite{TLCT_THCS_Toan_9_dai_so}, VD14.8, p. 87]
	Giải phương trình $x^4 + 4x^3 - 7x^2 - 20x + 25 = 0$.
\end{baitoan}

\begin{baitoan}[\cite{TLCT_THCS_Toan_9_dai_so}, VD14.9, p. 87]
	Chứng minh nếu phương trình $x^4 + ax^3 + bx^2 + ax + 1 = 0$ có nghiệm thực thì $a^2 + (b - 2)^2\ge\dfrac{16}{5}$.
\end{baitoan}
Giải phương trình:
	
\begin{baitoan}[\cite{TLCT_THCS_Toan_9_dai_so}, VD14.10, p. 87]
	$\sqrt{x^2 + x + 1} + \sqrt{x^2 + x + 4} = \sqrt{2x^2 + 2x + 9}$.
\end{baitoan}

\begin{baitoan}[\cite{TLCT_THCS_Toan_9_dai_so}, VD14.11, p. 87]
	$\sqrt{\dfrac{x^2 + 3x + 5}{x^2 - 4x + 5}} + \sqrt{\dfrac{x^2 - 4x + 5}{x^2 + 3x + 5}} = \dfrac{10}{3}$.
\end{baitoan}

\begin{baitoan}[\cite{TLCT_THCS_Toan_9_dai_so}, VD14.12, p. 87]
	$\sqrt[3]{10 - \sqrt{x^2 + 1}} + \sqrt[3]{6 + \sqrt{x^2 + 1}} = 4$.
\end{baitoan}

\begin{baitoan}[\cite{TLCT_THCS_Toan_9_dai_so}, VD14.13, p. 87]
	$\sqrt{x^2 + x - 1} + \sqrt{1 + x - x^2} = x^2 - x + 2$.
\end{baitoan}

%------------------------------------------------------------------------------%

\section{Giải Bài Toán Bằng Cách Lập Phương Trình}
\fbox{1} {\sf Giải bài toán bằng cách lập phương trình bậc 2}: \textit{Bước 1}: Lập phương trình: Chọn 1 đại lượng chưa biết làm ẩn, đặt đơn vị \& điều kiện thích hợp cho ẩn. Biểu diễn các đại lượng chưa biết khác trong bài toán theo ẩn \& các đại lượng đã biết. Lập phương trình biểu thị sự tương quan giữa các đại lượng trong bài toán. \textit{Bước 2}: Giải phương trình vừa lập được. \textit{Bước 3}: Chọn kết quả thích hợp \& kết luận. \fbox{2} Các dạng toán: Toán chuyển động đều, toán năng suất lao động, toán về quan hệ giữa các số, $\ldots$

\begin{baitoan}
	So sánh cách giải bài toán bằng cách lập phương trình bậc nhất 1 ẩn, bậc 2 1 ẩn, hệ phương trình bậc nhất 2 ẩn.
\end{baitoan}

\begin{baitoan}[\cite{Binh_boi_duong_Toan_9_tap_2}, VD1, p. 67]
	Lúc {\rm6:00} 1 ôtô xuất phát từ A đến B cách nhau {\rm100 km} với vận tốc đã định. Đến B ôtô nghỉ lại {\rm30 ph} rồi quay trở lại A với vận tốc lớn hơn vận tốc lúc đi là {\rm10 km{\tt/}h}. Ôtô về A lúc {\rm11:00}. Tính vận tốc lúc đi của ôtô.
\end{baitoan}

\begin{baitoan}[\cite{Binh_boi_duong_Toan_9_tap_2}, VD2, p. 68]
	2 vòi nước cùng chảy vào 1 cái bể cạn nước thì sau {\rm8 h} bể sẽ đầy. Nếu vòi thứ nhất chảy 1 mình đầy $75\%$ bể rồi vòi thứ 2 chảy tiếp 1 mình thì bể sẽ đầy sau {\rm15 h}. Nếu mỗi vòi chảy 1 mình thì sau bao lâu đầy bể?
\end{baitoan}

\begin{baitoan}[\cite{Binh_boi_duong_Toan_9_tap_2}, VD3, p. 68]
	Người ta dự định xây dựng 1 hội trường có $600$ chỗ ngồi với 1 số hàng ghế, mỗi hàng ghế có số chỗ ngồi như nhau. Thực tế hội trường có $650$ chỗ ngồi. Do mỗi hàng ghế giảm đi $5$ chỗ ngồi nên số hàng ghế phải tăng thêm $6$ hàng. Tính số hàng ghế \& số chỗ ngồi mỗi hàng dự định ban đầu.
\end{baitoan}

\begin{baitoan}[\cite{Binh_boi_duong_Toan_9_tap_2}, VD4, p. 69]
	Trong phong trào trồng cây gây rừng, 1 lớp học tham gia 3 đợt trồng cây trong năm. Số cây mỗi em trong lớp trồng trong mỗi đợt là như nhau. Đợt 1 lớp vắng $5$ em, trồng được $120$ cây. Đợt 2 lớp vắng $3$ em, trồng được $160$ cây. Đợt 3 lớp không vắng em nào, trồng được $315$ cây. Biết 1 học sinh có mặt cả 3 đợt trồng cây, có số cây trồng được đợt thứ 3 bằng tổng số cây trồng được của cả 2 đợt trước. Tính số hoc sinh của lớp.
\end{baitoan}

\begin{baitoan}[\cite{Binh_boi_duong_Toan_9_tap_2}, 8.1., p. 70]
	1 ôtô khách lúc {\rm7:00} khởi hành từ A đến B dài {\rm150 km} với vận tốc dự định. Đi được nửa quãng đường thì ôtô phải dừng lại {\rm15 ph} nên để đi đến B đúng thời gian dự định, ôtô phải tăng vận tốc thêm {\rm10 km{\tt/}h}. Tính vận tốc dự định của ôtô. Ôtô đến B lúc mấy giờ? Giải bằng 2 cách chọn ẩn số.
\end{baitoan}

\begin{baitoan}[\cite{Binh_boi_duong_Toan_9_tap_2}, 8.2., p. 70]
	Trên 1 dòng sông 1 chiếc canô xuôi dòng {\rm80 km} rồi ngược dòng {\rm32 km} hết {\rm6 h}. Tính vận tốc riêng của canô, biết 1 đám bèo trôi trên đoạn sông đó trong {\rm2 h} được {\rm4 km}.
\end{baitoan}

\begin{baitoan}[\cite{Binh_boi_duong_Toan_9_tap_2}, 8.3., p. 70]
	1 đoàn ôtô tải dự định điều 1 số xe cùng loại để chở $60$ tấn hàng. Khi sắp khởi hành thì được giao thêm $24$ tấn hàng nữa. Do chỉ được bổ sung thêm 2 xe cùng loại nên mỗi xe đều phải chở thêm $1$ tấn so với dự định ban đầu. Tìm số xe dự định ban đầu.
\end{baitoan}

\begin{baitoan}[\cite{Binh_boi_duong_Toan_9_tap_2}, 8.4., p. 70]
	1 tổ công nhân theo kế hoạch phải làm $500$ sản phẩm trong 1 thời gian nhất định. Do cải tiến kỹ thuật, mỗi ngày tổ làm thêm được $5$ sản phẩm so với dự định nên chẳng những tổ hoàn thành công việc trước $3$ ngày mà còn làm thêm được $10$ sản phẩm nữa. Hỏi theo kế hoạch mỗi ngày tổ phải làm bao nhiêu sản phẩm?
\end{baitoan}

\begin{baitoan}[\cite{Binh_boi_duong_Toan_9_tap_2}, 8.5., p. 70]
	Tìm 2 số có hiệu là $6$ \& tổng các bình phương của chúng là $146$.
\end{baitoan}

\begin{baitoan}[\cite{Binh_boi_duong_Toan_9_tap_2}, 8.6., p. 70]
	Trộn {\rm30 g} chất lỏng loại I với {\rm12 g} chất lỏng loại II được 1 hỗn hợp có khối lượng riêng {\rm1.4 g{\tt/}$\rm cm^3$}. Tính khối lượng riêng của từng loại chất lỏng, biết khối lượng riêng chất lỏng loại I hơn khối lượng riêng chất lỏng loại II là {\rm0.3 g{\tt/}$\rm cm^3$}.
\end{baitoan}

\begin{baitoan}[\cite{Binh_boi_duong_Toan_9_tap_2}, 8.7., p. 70]
	(a) Chứng minh đa giác lồi $n$ cạnh có $\dfrac{n(n - 3)}{2}$ đường chéo. (b) 1 đa giác đều có $740$ đường chéo. Tính số đo mỗi góc ở đỉnh của đa giác.
\end{baitoan}

\begin{baitoan}[\cite{Binh_boi_duong_Toan_9_tap_2}, 8.8., p. 70]
	Đoạn đường AB dài {\rm100 km}. Lúc {\rm7:00} 1 ôtô \& 1 xe đạp cùng xuất phát từ A để đi đến B, vận tốc ôtô hơn vận tốc xe đạp là {\rm25 km{\tt/}h}. Ôtô đi đến B nghỉ {\rm30 ph} rồi quay trở lại A với vận tốc như lúc đi \& gặp xe đạp tại C cách B {\rm40 km}. Hỏi ôtô \& xe đạp gặp nhau lúc mấy giờ?
\end{baitoan}

\begin{baitoan}[\cite{Binh_boi_duong_Toan_9_tap_2}, 8.9., p. 70]
	Tìm 2 số biết 2 lần số thứ nhất hơn $3$ lần số thứ 2 là $1$ đơn vị, còn hiệu các bình phương của số thứ nhất \& số thứ 2 bằng $16$.
\end{baitoan}

\begin{baitoan}[\cite{Binh_boi_duong_Toan_9_tap_2}, 8.., p. 71, bài toán cổ Ấn Độ về đàn ong]
	1 số ong bằng căn bậc 2 của 1 nửa toàn bộ đàn ong đậu trên bụi hoa nhài, đằng sau nó là $\dfrac{8}{9}$ đàn ong. Chỉ có 1 chú ong cùng tổ lượn vòng quanh 1 đóa hoa sen, bị cuốn hút bởi tiếng vo vo của người tình, đã cùng rơi vào cạm bẫy của bông hoa thơm ngát. Tính số con ong trong đàn.
\end{baitoan}

\begin{baitoan}[\cite{Binh_boi_duong_Toan_9_tap_2}, p. 72]
	1 kho hàng xi măng: Ngày thứ nhất xuất kho $10$ tấn \& $10\%$ số còn lại. Ngày thứ 2 xuất kho $20$ tấn \& $10\%$ số còn lại. Ngày thứ 3 xuất kho $30$ tấn \& $10\%$ số còn lại. $\ldots$ Cứ như thế đến ngày cuối cùng thì vừa hết. Biết số xi măng xuất kho mỗi ngày là bằng nhau. Tính số ngày xuất kho \& số tấn xi măng lúc đầu có trong kho.
\end{baitoan}

\begin{baitoan}[\cite{Binh_boi_duong_Toan_9_tap_2}, p. 72]
	Trong dịp tết trồng cây, lớp 9A tham gia trồng 1 số cây. Tổ 1 trồng $5$ cây \& $20\%$ số cây còn lại. Sau đó tổ 2 trồng $10$ cây \& $20\%$ số cây còn lại. Tiếp theo tổ 3 trồng $15$ cây \& $20\%$ số cây còn lại. Cứ như thế cho đến tổ cuối cùng thì vừa hết. Biết số cây mỗi tổ trồng được như nhau, số học sinh mỗi tổ đều bằng $10$ \& mỗi em trồng được số cây bằng nhau. Tính số học sinh của lớp 9A, số cây mỗi học sinh trồng được.
\end{baitoan}

\begin{baitoan}[\cite{Kien_dai_so_9}, VD1, p. 122]
	1 người đi xe đạp từ A đến B cách nhau {\rm24 km}. Khi đi từ B trở về A người đó tăng vận tốc thêm {\rm4 km{\tt/}h} so với lúc đi, nên thời gian về ít hơn thời gian đi là {\rm30 ph}. Tính vận tốc của xe đạp khi đi từ A đến B.
\end{baitoan}

\begin{baitoan}[\cite{Kien_dai_so_9}, VD2, p. 122, TS2015 10 Chuyên ĐHSP Hà Nội]
	Quãng đường AB dài {\rm120 km}. Lúc {\rm7:00} 1 xe máy đi từ A đến B. Đi được $\frac{3}{4}$ xe bị hỏng phải dừng lại {\rm10 ph} để sửa rồi đi tiếp với vận tốc kém vận tốc lúc đầu {\rm10 km{\tt/}h}. Biết xe máy đến B lúc {\rm11:40} trưa cùng ngày. Giả sử vận tốc xe máy trên $\frac{3}{4}$ quãng đường đầu không đổi \& vận tốc xe máy trên $\frac{1}{4}$ quãng đường sau cũng không đổi. Xe máy bị hỏng lúc mấy giờ?
\end{baitoan}

\begin{baitoan}[\cite{Kien_dai_so_9}, VD1, p. 125, TS2015 Quãng Ngãi]
	1 công ty dự định điều động 1 số xe để chuyển $180$ tấn hàng từ cảng Dung Quất vào Tp. Hồ Chí Minh, mỗi xe chở khối lượng hàng như nhau. Nhưng do nhu cầu thực tế cần chuyển thêm $28$ tấn hàng nên công ty đó phải điều động thêm 1 xe cùng loại \& mỗi xe bây giờ phải chở thêm $1$ tấn hàng mới đáp ứng được nhu cầu đặt ra. Theo dự định, tính số xe công ty đó cần điều động, biết mỗi xe không chở quá $15$ tấn.
\end{baitoan}

\begin{baitoan}[\cite{Kien_dai_so_9}, VD2, p. 125, TS2015 Bà Rịa Vũng Tàu]
	Hưởng ứng phong trào ``Vì biển đảo Trường Sa'', 1 đội tàu dự định chở $280$ tấn hàng ra đảo. Nhưng khi chuẩn bị khởi hành thì số hàng hóa đã tăng thêm $6$ tấn so với dự định, vì vậy đội tàu phải bổ sung thêm $1$ tàu \& mỗi tàu chở ít hơn dự định $2$ tấn hàng. Tính số chiếc tàu dự định của đội tàu, biết các tàu chở số tấn hàng bằng nhau.
\end{baitoan}

\begin{baitoan}[\cite{Kien_dai_so_9}, VD3, p. 126]
	1 công nhân theo kế hoạch phải làm thêm $85$ sản phẩm trong 1 khoảng thời gian dự định. Nhưng do yêu cầu đột xuất, người công nhân đó phải làm $96$ sản phẩm. Do người công nhân mỗi giờ đã làm tăng thêm $3$ sản phẩm nên người đó đã hoàn thành công việc sớm hơn so với thời gian dự định {\rm20 ph}. Tính xem theo dự định mỗi giờ người đó phải làm bao nhiêu sản phẩm, biết mỗi giờ chỉ làm được không quá $20$ sản phẩm.
\end{baitoan}

\begin{baitoan}[\cite{Kien_dai_so_9}, 2., p. 129]
	1 phòng họp có $180$ ghế được chia thành các dãy ghế có số ghế ở mỗi dãy bằng nhau. Nếu kê thêm mỗi dãy $5$ ghế \& bớt đi $3$ dãy thì số ghế trong phòng không thay đổi. Tính số dãy ban đầu phòng họp được chia thành.
\end{baitoan}

\begin{baitoan}[\cite{Kien_dai_so_9}, 3., p. 130]
	Khu du lịch sinh thái Đồng Mô thuộc thị xã Tây Sơn là 1 địa điểm thu hút khách du lịch của Tp. Hà Nội. Nhà An ở cách địa điểm du lịch {\rm1800 m}. Lúc đi từ nhà đến địa điểm du lịch An đi bộ. Lúc về An đi bằng xe điện với vận tốc lớn hơn lúc đi là {\rm120 m{\tt/}ph} nên thời gian về ít hơn thời gian đi {\rm20 ph}. Tính vận tốc xe điện.
\end{baitoan}

\begin{baitoan}[\cite{Kien_dai_so_9}, 4., p. 130]
	1 ôtô dự định đi từ A đến B cách nhau {\rm148 km} trong 1 thời gian đã định. Sau khi đi được {\rm1 h} thì ôtô bị chắn bởi tàu hỏa trong {\rm5 ph}, vì vậy để đến B đúng giờ ôtô phải chạy với vận tốc tăng thêm {\rm2 km{\tt/}h} so với lúc đầu. Tính vận tốc ôtô trong {\rm1 h} đầu.
\end{baitoan}

\begin{baitoan}[\cite{Kien_dai_so_9}, 5., p. 131]
	Khoảng cách giữa 2 tỉnh A,B là {\rm60 km}. 2 người đi xe đạp cùng khởi hành 1 lúc đi từ A đến B với vận tốc bằng nhau. Sau khi đi được {\rm1 h} thì xe của người thứ nhất bị hỏng nên phải đứng lại sửa xe {\rm20 ph} còn người thứ 2 tiếp tục đi với vận tốc ban đầu. Sau khi sửa xe xong, người thứ nhất tiếp tục đi với vận tốc lớn hơn lúc đầu {\rm4 km{\tt/}h} nên đến B cùng lúc với người thứ 2. Tính vận tốc 2 người lúc đầu.
\end{baitoan}

\begin{baitoan}[\cite{Kien_dai_so_9}, 6., p. 131]
	Trong giờ thể dục 2 bạn An \& Bình chạy bền trên cùng 1 quãng đường dài {\rm2 km} \& xuất phát tại cùng 1 thời điểm. Biết An chạy bền với vận tốc trung bình lớn hơn vận tốc trung bình của Bình là {\rm2 km{\tt/}h} \& về đích sớm hơn Bình {\rm5 ph}. Tính thời gian chạy hết quãng đường của mỗi bạn, giả sử vận tốc của mỗi bạn không đổi trong suốt quãng đường.
\end{baitoan}

\begin{baitoan}[\cite{Kien_dai_so_9}, 7., p. 131]
	1 đoàn xe vận tải nhận chuyên chở $22$ tấn hàng. Khi sắp khởi hành thì 1 xe phải điều đi làm công việc khác nên mỗi xe còn lại chở nhiều hơn $0.2$ tấn hàng so với dự định. Tính số xe đã tham gia vận chuyển trong thực tế. Biết khối lượng hàng mỗi xe vận chuyển là như nhau.
\end{baitoan}

\begin{baitoan}[\cite{Kien_dai_so_9}, 9., p. 132, dự bị TS2017 Hà Nội]
	1 đội xe dự định chở $24$ tấn hàng. Thực tế khi chở được bổ sung thêm $4$ xe nữa nên mỗi xe chở ít hơn thực tế $1$ tấn hàng. Tính số xe ban đầu, biết khối lượng hàng chở trên mỗi xe là như nhau.
\end{baitoan}

\begin{baitoan}[\cite{Kien_dai_so_9}, 10., p. 132]
	1 người đi xe đạp từ A đến B cách nhau {\rm24 km} với vận tốc dự định. Khi đi từ B trở về A người đó tăng vận tốc trung bình thêm {\rm4 km{\tt/}h} so với lúc đi, nên thời gian về ít hơn thời gian đi là {\rm30 ph}. Tính vận tốc trung bình dự định của xe đạp khi đi từ A đến B.
\end{baitoan}

\begin{baitoan}[\cite{Kien_dai_so_9}, 11., p. 133]
	2 tỉnh A,B cách nhau {\rm60 km}. Có 1 xe đạp đi từ A đến B. Khi xe đạp bắt đầu khởi hành thì có 1 xe máy ở cách A {\rm40 km} đi đến A rồi trở về B ngay. Tính vận tốc mỗi xe biết xe gắn máy về B trước xe đạp {\rm40 ph} \& vận tốc xe gắn máy hơn vận tốc xe đạp {\rm15 km{\tt/}h}.
\end{baitoan}

\begin{baitoan}[\cite{Tuyen_Toan_9_old}, VD45, p. 95]
	Quãng đường AB dài {\rm200 km}. Cùng 1 lúc, 1 xe tải khởi hành từ A đi về B \& 1 xe con khởi hành từ B đi về A. Sau khi 2 xe gặp nhau, xe tải phải đi thêm {\rm3 h} nữa mới tới B. Biết vận tốc xe tải kém vận tốc xe con là {\rm20 km{\tt/}h}, tính vận tốc mỗi xe.
\end{baitoan}

\begin{baitoan}[\cite{Tuyen_Toan_9_old}, 254., p. 96]
	1 canô xuôi dòng {\rm80 km} \& ngược dòng {\rm64 km} hết {\rm8 h} với vận tốc riêng không đổi. Biết vận tốc xuôi dòng hơn vận tốc ngược dòng là {\rm4 km{\tt/}h}. Tính vận tốc riêng của canô.
\end{baitoan}

\begin{baitoan}[\cite{Tuyen_Toan_9_old}, 255., p. 96]
	2 ôtô khởi hành cùng 1 lúc tại 2 tỉnh $A,B$, đi ngược chiều nhau với vận tốc không đổi. Xe I đi từ A đến B rồi trở về A còn xe II đi từ B đến A rồi trở về B. 2 xe gặp nhau lần đầu tại 1 điểm cách A là {\rm40 km} \& gặp nhau lần thứ 2 tại 1 điểm cách B là {\rm10 km}. Tính khoảng cách AB biết 2 xe gặp nhau khi di chuyển ngược chiều nhau.
\end{baitoan}

\begin{baitoan}[\cite{Tuyen_Toan_9_old}, 256., p. 96]
	1 buổi tổng kết thi đua có $55$ đại biểu tham dự. Lúc đầu ác đại biểu được chia ngồi đều trên các ghế đài (mỗi ghế có số người ngồi như nhau). Về sau, có thêm $3$ ghế dài nên bây giờ mỗi ghế ngồi bớt đi $1$ đại biểu \& chiếc ghế cuối cùng chỉ có $3$ đại biểu. Tính số ghế dài ban đầu.
\end{baitoan}

\begin{baitoan}[\cite{Tuyen_Toan_9_old}, 257., p. 96]
	1 xí nghiệp đặt kế hoạch sản xuất $3000$ sản phẩm trong 1 thời gian. Trong $6$ ngày đầu, họ thực hiện đúng tiến độ, các ngày sau đó mỗi ngày vượt $10$ sản phẩm nên chẳng những hoàn thành sớm được $1$ ngày mà còn vượt mức $60$ sản phẩm nữa. Tính năng suất dự kiến theo kế hoạch.
\end{baitoan}

\begin{baitoan}[\cite{Tuyen_Toan_9_old}, 258., p. 96]
	2 đội công nhân cùng làm chung 1 công việc. Thời gian để đội I làm 1 mình xong công việc ít hơn thời gian để đội II làm 1 mình xong công việc đó là {\rm4 h}. Tổng2 thời gian này gấp $4.5$ lần thời gian 2 đội cùng làm chung để xong công việc đó. Mỗi đội nếu làm 1 mình thì phải bao lâu mới xong?
\end{baitoan}

\begin{baitoan}[\cite{Binh_Toan_9_tap_2}, VD107, p. 54, Euler's]
	2 bà ra chợ bán tổng cộng $100$ quả trứng. Số trứng của 2 người không bằng nhau, nhưng số tiền thu được lại bằng nhau. Bà thứ nhất nói với bà thứ 2: Nếu tôi có số trứng như của bà, tôi sẽ thu được $15$ đồng. Bà thứ 2: Nếu số trứng của tôi bằng số trứng của bà, tôi chỉ bán được $6\dfrac{2}{3}$ đồng. Hỏi mỗi bà có bao nhiêu quả trứng mang đi bán?
\end{baitoan}

\begin{baitoan}[\cite{Binh_Toan_9_tap_2}, 338., p. 55]
	Trong cùng 1 thời gian như nhau, đội I phải đào $V_1 = 810\ {\rm m}^3$ đất, đội II phải đào $V_2 = 900\ {\rm m}^3$ đất. Kết quả đội I đã hoàn thành trước thời hạn $3$ ngày, đội II hoàn thành trước thời hạn $6$ ngày. Tính số đất mỗi đội đã đào trong 1 ngày, biết mỗi ngày đội II đã đào nhiều hơn đội I là $\rm4\ m^3$.
\end{baitoan}

\begin{baitoan}[\cite{Binh_Toan_9_tap_2}, 339., p. 55]
	1 người thả hòn đá rơi xuống giếng, sau $1.5$ giây thì nghe thấy tiếng đá chạm đáy giếng. Tìm thời gian rơi của đá \& chiều sâu của giếng (làm tròn đến {\rm m}), biết quãng đường $s$ {\rm m} của 1 vật rơi tự do không có vận tốc ban đầu sau $t$ giây được tính theo công thức $s = 5t^2$ \& vận tốc của âm thanh là $340$ {\rm m{\tt/}s}.
\end{baitoan}

\begin{baitoan}[\cite{Binh_Toan_9_tap_2}, 340., p. 55]
	Có 2 loại quặng sắt: quặng loại I \& quặng loại II, khối lượng tổng cộng là $10$ tấn. Khối lượng sắt nguyên chất trong quặng loại I là $0.8$ tấn, trong quặng loại II là $0.6$ tấn. Biết tỷ lệ sắt nguyên chất trong quặng loại I nhiều hơn tỷ lệ sắt nguyên chất trong quặng loại II là $10\%$. Tính khối lượng của mỗi loại quặng.
\end{baitoan}

\begin{baitoan}[\cite{Binh_Toan_9_tap_2}, 341., p. 55]
	Nếu đường kính của 1 hình tròn tăng {\rm3 m} thì diện tích của nó tăng gấp đôi. Tính độ dài đường kính lúc đầu.
\end{baitoan}

\begin{baitoan}[\cite{Binh_Toan_9_tap_2}, 342., p. 55]
	2 người đi xe đạp cùng khởi hành 1 lúc ở cùng 1 chỗ, người thứ nhất đi về phía bắt, người thứ 2 đi về phía đông. Sau $2$ giờ họ cách nhau {\rm60 km} theo đường chim bay. Biết vận tốc của người thứ nhất lớn hơn vận tốc của người thứ 2 là {\rm6 km{\tt/}h}. Tính vận tốc mỗi người.
\end{baitoan}

\begin{baitoan}[\cite{Binh_Toan_9_tap_2}, 343., p. 55]
	2 vòi nước cùng chảy vào 1 bể thì sau $6$ giờ đầy bể. Nếu chảy 1 mình cho đầy bể thì vòi I cần nhiều hơn vòi II là $5$ giờ. Hỏi mỗi vòi chảy 1 mình trong bao lâu thì đầy bể?
\end{baitoan}

\begin{baitoan}[\cite{Binh_Toan_9_tap_2}, 344., pp. 55--56]
	Trên quãng đường AB dài {\rm60 km}, người I đi từ A đến B, người II đi từ B đến A. Họ khởi hành cùng 1 lúc \& gặp nhau tại C sau khi khởi hành {\rm1h12ph}. Từ C, người I đi tiếp đến B với vận tốc giảm hơn trước {\rm6 km{\tt/}h}, người thứ II đi tiếp đến A với vận tốc như cũ. Kết quả người I đến nơi sớm hơn người II $48$ phút. Tính vận tốc ban đầu của mỗi người.
\end{baitoan}

\begin{baitoan}[\cite{Binh_Toan_9_tap_2}, 345., p. 56]
	1 cửa hàng mua $x$ chiếc áo hết $d$ nghìn đồng. Cửa hàng bán $2$ chiếc với giá bằng $\frac{1}{2}$ giá mua, bán các chiếc còn lại lãi được $8000$ đồng{\tt/}chiếc. Tiền lãi tổng cộng là $72000$ đồng. (a) Tính $x$ biết $d = 480$. (b) Tìm {\rm GTNN} của $x$ biết $d\in\mathbb{N}$.
\end{baitoan}

\begin{baitoan}[\cite{Binh_Toan_9_tap_2}, 346., p. 56]
	Long có đồ chơi là 1 chiếc thuyền buồm. Khi Long cho thuyền chạy từ A đến B thì thời gian thuyền đi nhiều hơn so với khi có gió thổi thuận chiều là $9$ giây. Khi bị gió thổi ngược chiều thì thời gian thuyền đi từ A đến B là $84$ giây. Tính thời gian thuyền đi từ A đến B khi không có gió thổi.
\end{baitoan}

\begin{baitoan}[\cite{Binh_Toan_9_tap_2}, 347., p. 56]
	3 công nhân cùng làm 1 công việc thì làm xong sớm hơn $18$ giờ so với khi người III làm 1 mình, sớm hơn $3$ giờ so với khi người II làm 1 mình \& bằng nửa thời gian so với khi người I làm 1 mình công việc đó. Tính thời gian của mỗi công nhận làm 1 mình xong công việc đó.
\end{baitoan}

\begin{baitoan}[\cite{Binh_Toan_9_tap_2}, 348., p. 56, Sam Loyd's]
	1 điền chủ muốn cắt ra từ 1 mảnh đất hình chữ nhật 1 dải đất có chiều rộng không đổi dọc theo 4 bờ của mảnh đất sao cho diện tích phần cắt ra bằng diện tích phần còn lại. Trong cuốn sách của mình, Sam Loyd đưa ra cách làm của người điền chủ nhưng không chứng minh: Lấy nửa chu vi hình chữ nhật ban đầu trừ đi đường chéo của nó, rồi chia cho $4$, đó chính là chiều rộng của dải đất được cắt ra. Chứng minh cách làm này đúng.
\end{baitoan}

%------------------------------------------------------------------------------%

\section{Relation Between Parabol \& Line -- Quan Hệ Giữa Parabol $y = ax^2$ \& Đường Thẳng $y = mx + n$}

\begin{baitoan}[\cite{Binh_Toan_9_tap_2}, VD108, p. 57]
	Cho parabol $y = \dfrac{1}{2}x^2$, đường thẳng $y = \dfrac{1}{2}x + 3$. (a) Tìm tọa độ 2 giao điểm $A,B$ của parabol \& đường thẳng. (b) Tìm tọa độ điểm C thuộc cung $AB$ của parabol đó sao cho $\Delta ABC$ có diện tích lớn nhất.
\end{baitoan}

\begin{baitoan}[\cite{Binh_Toan_9_tap_2}, 349., p. 58]
	Cho parabol $y = \dfrac{1}{2}x^2$, đường thẳng $(d):y = mx + n$. Tìm 2 hệ số $m,n\in\mathbb{R}$ để đường thẳng $d$ đi qua điểm $A(1,0)$ \& tiếp xúc với parabol. Tìm tọa độ tiếp điểm.
\end{baitoan}

\begin{baitoan}[\cite{Binh_Toan_9_tap_2}, 350., p. 58]
	Cho parabol $y = x^2$. Tìm điểm A thuộc parabol sao cho tiếp tuyến của parabol tại A song song với đường thẳng $y = 4x + 5$.
\end{baitoan}

\begin{baitoan}[\cite{Binh_Toan_9_tap_2}, 351., p. 58]
	Cho parabol $y = x^2$, 2 điểm $A,B$ thuộc parabol với hoành độ tương ứng là $-1,2$. Tìm điểm M trên cùng $AB$ của parabol sao cho $\Delta ABM$ có diện tích lớn nhất.
\end{baitoan}

\begin{baitoan}[\cite{Binh_Toan_9_tap_2}, 352., p. 58]
	Cho parabol $y = x^2$. chứng minh với mọi điểm M thuộc đường thẳng $y = -\dfrac{1}{4}$, 2 tiếp tuyến kẻ từ M với parabol vuông góc với nhau.
\end{baitoan}

\begin{baitoan}[\cite{Binh_Toan_9_tap_2}, 353., p. 58]
	Cho parabol $y = x^2$. Gọi $A,B$ là 2 giao điểm của đường thẳng $y = mx + 2$ với parabol với tham số $m\in\mathbb{R}$. Tìm giá trị của $m$ để đoạn thẳng $AB$ có độ dài nhỏ nhất.
\end{baitoan}

%------------------------------------------------------------------------------%

\section{Conditions on Roots of Equation -- Điều Kiện Về Nghiệm của 1 Phương Trình}

\begin{baitoan}[\cite{Binh_Toan_9_tap_2}, VD109, p. 59]
	Tìm $m\in\mathbb{R}$ để phương trình $x^2 + mx + 2m - 4 = 0$ có ít nhất 1 nghiệm không âm.
\end{baitoan}

\begin{baitoan}[\cite{Binh_Toan_9_tap_2}, VD110, p. 60]
	Tìm $m\in\mathbb{R}$ để phương trình $x^2 + mx - 1 = 0$ có ít nhất 1 nghiệm $\ge2$.
\end{baitoan}

\begin{baitoan}[\cite{Binh_Toan_9_tap_2}, VD111, p. 61]
	Tìm $m\in\mathbb{R}$ để phương trình $3x^2 - 4x + 2(m - 1) = 0$ có 2 nghiệm phân biệt nhỏ hơn $2$.
\end{baitoan}

\begin{baitoan}[\cite{Binh_Toan_9_tap_2}, VD112, p. 62]
	Tìm $m\in\mathbb{R}$ để phương trình $x^4 + mx^2 + 2m - 4 = 0$ có nghiệm.
\end{baitoan}

\begin{baitoan}[\cite{Binh_Toan_9_tap_2}, VD113, p. 62]
	Tìm $m\in\mathbb{R}$ để phương trình $\sqrt{2x^2 + (m - 4)x + 3} = x - 2$ có nghiệm.
\end{baitoan}

\begin{baitoan}[\cite{Binh_Toan_9_tap_2}, VD114, p. 63]
	Tìm $m\in\mathbb{R}$ để phương trình $x^3 - m(x + 1) + 1 = 0$ có đúng 2 nghiệm phân biệt.
\end{baitoan}

\begin{baitoan}[\cite{Binh_Toan_9_tap_2}, VD115, p. 62]
	Tìm $m\in\mathbb{R}$ để tập nghiệm của phương trình $x - \sqrt{1 - x^2} = m$ chỉ có 1 phần tử.
\end{baitoan}

\begin{baitoan}[\cite{Binh_Toan_9_tap_2}, VD116, p. 64]
	Tìm $m\in\mathbb{R}$ để phương trình $x(x - 2)(x + 2)(x + 4) = m$ có 4 nghiệm phân biệt.
\end{baitoan}

\begin{baitoan}[\cite{Binh_Toan_9_tap_2}, VD117, p. 65]
	Cho phương trình $x^4 - 2(m + 1)x^2 + 2m + 1 = 0$. Tìm $m\in\mathbb{R}$ để phương trình có 4 nghiệm $x_1 < x_2 < x_3 < x_4$ thỏa $x_4 - x_3 = x_3 - x_2 = x_2 - x_1$.
\end{baitoan}

\begin{baitoan}[\cite{Binh_Toan_9_tap_2}, VD118, p. 65]
	Tìm $m\in\mathbb{R}$ để tập nghiệm của phương trình $\sqrt{x - 5} + \sqrt{9 - x} = m$ chỉ có 1 phần tử.
\end{baitoan}

\begin{baitoan}[\cite{Binh_Toan_9_tap_2}, 354., p. 66]
	Tìm $m\in\mathbb{R}$ để phương trình $x^2 - 2x + m - 2 = 0$ có nghiệm không âm.
\end{baitoan}

\begin{baitoan}[\cite{Binh_Toan_9_tap_2}, 355., p. 66]
	Tìm $m\in\mathbb{R}$ để phương trình $x^2 + 2m|x - 2| - 4x + m^2 + 3 = 0$ có nghiệm.
\end{baitoan}

\begin{baitoan}[\cite{Binh_Toan_9_tap_2}, 356., p. 67]
	Tìm $m\in\mathbb{R}$ để phương trình $(m - 1)x^2 - (m - 5)x + m - 1 = 0$ có 2 nghiệm phân biệt $> -1$.
\end{baitoan}

\begin{baitoan}[\cite{Binh_Toan_9_tap_2}, 357., p. 67]
	Tìm $m\in\mathbb{R}$ để 2 nghiệm của phương trình $x^2 + x + m = 0$ đều lớn hơn $m$.
\end{baitoan}

\begin{baitoan}[\cite{Binh_Toan_9_tap_2}, 358., p. 67]
	Tìm $m\in\mathbb{R}$ để phương trình $x^2 + mx - 1 = 0$ có ít nhất 1 nghiệm $\le-2$.
\end{baitoan}

\begin{baitoan}[\cite{Binh_Toan_9_tap_2}, 359., p. 67]
	Tìm $m\in\mathbb{R}$ để phương trình $x^3 - (m + 1)x^2 + (m^2 + m - 3)x - m^2 + 3 = 0$.
\end{baitoan}

\begin{baitoan}[\cite{Binh_Toan_9_tap_2}, 360., p. 67]
	Tìm $m\in\mathbb{R}$ để phương trình có nghiệm: (a) $(m - 3)x^4 - 2mx^2 + 6m = 0$. (b) $x^4 - 2mx^2 + m + 2 = 0$.
\end{baitoan}

\begin{baitoan}[\cite{Binh_Toan_9_tap_2}, 361., p. 67]
	Cho phương trình $x^4 - 2(m - 1)x^2 - (m - 3) = 0$. Tìm $m\in\mathbb{R}$ để tập nghiệm của phương trình có: (a) $4$ phần tử. (b) $3$ phần tử. (c) $2$ phần tử. (d) Không có phần tử nào.
\end{baitoan}

\begin{baitoan}[\cite{Binh_Toan_9_tap_2}, 362., p. 67]
	Tìm $m\in\mathbb{R}$ để phương trình có 4 nghiệm $x_1 < x_2 < x_3 < x_4$ thỏa $x_4 - x_3 = x_3 - x_2 = x_2 - x_1$: (a) $mx^4 - 10mx^2 + m + 8 = 0$. (b) $x^4 - (m + 7)x^2 + 3m = 0$.
\end{baitoan}

\begin{baitoan}[\cite{Binh_Toan_9_tap_2}, 363., p. 67]
	Tìm $m\in\mathbb{R}$ để phương trình $x^4 - 40x^2 + 6m = 0$ có 4 nghiệm, \& khi biểu diễn 4 nghiệm đó từ nhỏ đến lớn trên trục số bởi 4 điểm $A,B,C,D$ thì $AB = BC = CD$.
\end{baitoan}

\begin{baitoan}[\cite{Binh_Toan_9_tap_2}, 364., p. 67]
	Chứng minh phương trình $(x + 1)^4 - (m - 1)(x + 1)^2 - (m^2 - m + 1) = 0$ có 2 nghiệm $\forall m\in\mathbb{R}$.
\end{baitoan}

\begin{baitoan}[\cite{Binh_Toan_9_tap_2}, 365., p. 67]
	Tìm $m\in\mathbb{R}$ để phương trình $x^4 + mx^3 + x^2 + mx + 1 = 0$ có nghiệm.
\end{baitoan}

\begin{baitoan}[\cite{Binh_Toan_9_tap_2}, 366., p. 67]
	Tìm $m\in\mathbb{R}$ để phương trình $2x - 4 = 3\sqrt{x - m}$ có nghiệm.
\end{baitoan}

\begin{baitoan}[\cite{Binh_Toan_9_tap_2}, 367., p. 68]
	Tìm $m\in\mathbb{R}$ để tập nghiệm của phương trình chỉ có 1 phần tử: (a) $\sqrt{4 + x^2} + \sqrt{4 - x^2} = m$. (b) $\sqrt{6 - x} + \sqrt{x + 2} = m$.
\end{baitoan}

%------------------------------------------------------------------------------%

\section{System of 2nd-Order Equations of 2 Unknowns -- Hệ Phương Trình Bậc 2 2 Ẩn}
\fbox{1} Phương pháp giải: phương pháp biến thành phương trình tích, phương pháp thế, phương pháp đánh giá. \fbox{2} Các dạng hệ phương trình bậc 2 2 ẩn: hệ phương trình đối xứng, hệ phương trình phản xứng, hệ phương trình đẳng cấp bậc 2.

Giải hệ phương trình:

\begin{baitoan}[\cite{TLCT_THCS_Toan_9_dai_so}, VD15.1, p. 88]
	\begin{equation*}
		\left\{\begin{split}
			4x + 4y + xy &= 14,\\
			x^2 + y^2 + xy &= 7.
		\end{split}\right.
	\end{equation*}
\end{baitoan}

\begin{baitoan}[\cite{TLCT_THCS_Toan_9_dai_so}, VD15.2, p. 88]
	\begin{equation*}
		\left\{\begin{split}
			\dfrac{x^2}{y^2} + \dfrac{16y^2}{x^2} &= 8,\\
			x + y + xy &= 5.
		\end{split}\right.
	\end{equation*}
\end{baitoan}

\begin{baitoan}[\cite{TLCT_THCS_Toan_9_dai_so}, VD15.3, p. 89]
	Tìm $a,b\in\mathbb{R}$ để hệ phương trình
	\begin{equation*}
		\left\{\begin{split}
			4x^2 + y^2 + axy - 1 &= 0,\\
			bx(2x - y) + (y - 1)(2x - y) &= bx + y - 1,
		\end{split}\right.
	\end{equation*}
	có không ít hơn 5 nghiệm.
\end{baitoan}

\begin{baitoan}[\cite{TLCT_THCS_Toan_9_dai_so}, VD15.4, p. 90]
	Tìm $a\in\mathbb{R}$ để hệ phương trình
	\begin{equation*}
		\left\{\begin{split}
			4x^2 - 12xy + 9y^2 + 2x - 6y &= 0,\\
			5x^2 - 16xy + 13x^2 - 6x + 10y + 2ax - 4ay + a^2 - 2a - 5 &= 0,
		\end{split}\right.
	\end{equation*}
	có không ít hơn 1 nghiệm.
\end{baitoan}

\begin{baitoan}[\cite{TLCT_THCS_Toan_9_dai_so}, VD15.5, p. 91]
	\begin{equation*}
		\left\{\begin{split}
			(1 + x)(1 + 2x)(1 + 3x) &= (1 + 3y)(1 + 3y + 2x^2),\\
			2x + 3y &= 30.
		\end{split}\right.
	\end{equation*}
\end{baitoan}

\begin{baitoan}[\cite{TLCT_THCS_Toan_9_dai_so}, VD15.6, p. 91]
	\begin{equation*}
		\left\{\begin{split}
			\dfrac{x^4}{y^2} + xy &= 72,\\
			\dfrac{y^4}{x^2} + xy &= 9.
		\end{split}\right.
	\end{equation*}
\end{baitoan}

\begin{baitoan}[\cite{TLCT_THCS_Toan_9_dai_so}, VD15.7, p. 92]
	\begin{equation*}
		\left\{\begin{split}
			\dfrac{1}{x^2 + y^2} + 2xy &= \dfrac{21}{5},\\
			\dfrac{1}{2xy} + x^2 + y^2 &= \dfrac{21}{4}.
		\end{split}\right.
	\end{equation*}
\end{baitoan}

\begin{baitoan}[\cite{TLCT_THCS_Toan_9_dai_so}, VD15.8, p. 93]
	\begin{equation*}
		\left\{\begin{split}
			x + y &= 2,\\
			x^2y^2(x^2 + y^2) &= 2,\\
			x,y&\ge0.
		\end{split}\right.
	\end{equation*}
\end{baitoan}

\begin{baitoan}[\cite{TLCT_THCS_Toan_9_dai_so}, VD15.9, p. 94]
	\begin{equation*}
		\left\{\begin{split}
			x^2 + (y + 3)^2 &= 1,\\
			x^3 + (y + 3)^3 &= 1.
		\end{split}\right.
	\end{equation*}
\end{baitoan}

\begin{baitoan}[\cite{TLCT_THCS_Toan_9_dai_so}, VD15.10, p. 94]
	Chứng minh hệ phương trình
	\begin{equation*}
		\left\{\begin{split}
			2x^2 + xy + y^2 &= 32,\\
			2y^2 + yz + z^2 &= 25,\\
			2z^2 + zx + x^2 &= 86,
		\end{split}\right.
	\end{equation*}
	không có nghiệm dương.
\end{baitoan}

\begin{baitoan}[\cite{TLCT_THCS_Toan_9_dai_so}, VD15.11, p. 95]
	\begin{equation*}
		\left\{\begin{split}
			2y^2 + 9y &= 21 + x^2,\\
			3x^2 + 2y^2 - 4xy + 3y &= 7.
		\end{split}\right.
	\end{equation*}
\end{baitoan}

\begin{baitoan}[\cite{TLCT_THCS_Toan_9_dai_so}, VD15.12, p. 96]
	(a) Giải \& biện luận hệ phương trình
	\begin{equation*}
		\left\{\begin{split}
			x + y &= 2a,\\
			3x^2 + 2y^2 - 4xy + 3y &= 7.
		\end{split}\right.
	\end{equation*}
	(b) Tìm $a\in\mathbb{R}$ để hệ có đúng 1 nghiệm.
\end{baitoan}

\begin{baitoan}[\cite{TLCT_THCS_Toan_9_dai_so}, VD15.13, p. 97]
	\begin{equation*}
		\left\{\begin{split}
			x + y - a(1 + xy) &= 0,\\
			xy + 2x + 2y + 5 &= 0.
		\end{split}\right.
	\end{equation*}
	(a) Giải hệ khi $a = 1$. (b) Tìm $a\in\mathbb{R}$ để hệ có đúng 1 nghiệm.
\end{baitoan}
Tìm $a\in\mathbb{R}$ để hệ có đúng 1 nghiệm:

\begin{baitoan}[\cite{TLCT_THCS_Toan_9_dai_so}, VD15.14, p. 97]
	\begin{equation*}
		\left\{\begin{split}
			x^2 + y^2 - a(x + y) &= a,\\
			xy + a(x + y) + 4 &= 0.
		\end{split}\right.
	\end{equation*}
\end{baitoan}

\begin{baitoan}[\cite{TLCT_THCS_Toan_9_dai_so}, VD15.15, p. 98]
	\begin{equation*}
		\left\{\begin{split}
			x + y &= 2a,\\
			x^4 + y^4 - x^2y^2 &= a.
		\end{split}\right.
	\end{equation*}
\end{baitoan}

\begin{baitoan}[\cite{TLCT_THCS_Toan_9_dai_so}, VD15.16, p. 99]
	\begin{equation*}
		\left\{\begin{split}
			x^2 + (1 + y)^2 &= a,\\
			y^2 + (1 + x)^2 &= a.
		\end{split}\right.
	\end{equation*}
\end{baitoan}

\begin{baitoan}[\cite{TLCT_THCS_Toan_9_dai_so}, VD15.17, p. 99]
	\begin{equation*}
		\left\{\begin{split}
			x^2 + 6x + a &= 2y,\\
			y^2 + 6y + a &= 2x.
		\end{split}\right.
	\end{equation*}
\end{baitoan}

\begin{baitoan}[\cite{TLCT_THCS_Toan_9_dai_so}, VD15.18, p. 100]
	\begin{equation*}
		\left\{\begin{split}
			x^2 + ay &= 4x,\\
			y^2 + ax &= 4x.
		\end{split}\right.
	\end{equation*}
\end{baitoan}

\begin{baitoan}[\cite{TLCT_THCS_Toan_9_dai_so}, VD15.19, p. 101]
	\begin{equation*}
		\left\{\begin{split}
			(x + y)^2 + a &= x + 4y,\\
			(x - y)^2 + a &= 4y - x.
		\end{split}\right.
	\end{equation*}
\end{baitoan}

\begin{baitoan}[\cite{TLCT_THCS_Toan_9_dai_so}, VD15.20, p. 101]
	\begin{equation*}
		\left\{\begin{split}
			x + y + \dfrac{1}{x - y} &= a,\\
			x - y + \dfrac{1}{x + y} &= a.
		\end{split}\right.
	\end{equation*}
\end{baitoan}

\begin{baitoan}[\cite{TLCT_THCS_Toan_9_dai_so}, VD15.21, p. 102]
	\begin{equation*}
		\left\{\begin{split}
			(x + y)^2 + 2x + 2y &= a,\\
			(x - y)^2 - 2x + 2y &= a.
		\end{split}\right.
	\end{equation*}
\end{baitoan}

\begin{baitoan}[\cite{TLCT_THCS_Toan_9_dai_so}, VD15.22, p. 103]
	Giải hệ phương trình:
	\begin{equation*}
		\left\{\begin{split}
			x^2 + 6y^2 &= 33,\\
			x^2 - 2xy &= -3.
		\end{split}\right.
	\end{equation*}
\end{baitoan}

\begin{baitoan}[\cite{TLCT_THCS_Toan_9_dai_so}, VD15.23, p. 104]
	Giải hệ phương trình:
	\begin{equation*}
		\left\{\begin{split}
			x^2 - 3xy + y^2 &= 3,\\
			x^2 + 2xy - 2y^2 &= 6.
		\end{split}\right.
	\end{equation*}
\end{baitoan}

\begin{baitoan}[\cite{TLCT_THCS_Toan_9_dai_so}, 15.1., p. 104]
	Giải \& biện luận hệ phương trình:
	\begin{equation*}
		\left\{\begin{split}
			x^2 - 3y &= a,\\
			y^2 - 3x &= a.
		\end{split}\right.
	\end{equation*}
\end{baitoan}

\begin{baitoan}[\cite{TLCT_THCS_Toan_9_dai_so}, 15.2., p. 105]
	\begin{equation*}
		\left\{\begin{split}
			(x + y)^2 &= 16,\\
			x^2 + y^2 &= a.
		\end{split}\right.
	\end{equation*}
	(a) Giải hệ khi $a = 10$. (b) Tìm $a\in\mathbb{R}$ để hệ có nghiệm. (c) Tìm $a\in\mathbb{R}$ để hệ có đúng 2 nghiệm.
\end{baitoan}

\begin{baitoan}[\cite{TLCT_THCS_Toan_9_dai_so}, 15.3., p. 105]
	\begin{equation*}
		\left\{\begin{split}
			x + |y| &= 2,\\
			x^2 + y^2 &= a.
		\end{split}\right.
	\end{equation*}
	(a) Giải hệ khi $a = 3$. (b) Tìm $a\in\mathbb{R}$ để hệ có nghiệm. (c) Tìm $a\in\mathbb{R}$ để hệ có đúng 2 nghiệm. (d) Tìm $a\in\mathbb{R}$ để hệ có đúng 1 nghiệm.
\end{baitoan}

\begin{baitoan}[\cite{TLCT_THCS_Toan_9_dai_so}, 15.4., p. 105]
	\begin{equation*}
		\left\{\begin{split}
			|x| + |y| &= a - 1,\\
			x^2 + y^2 &= a^2 - 1.
		\end{split}\right.
	\end{equation*}
	(a) Tìm $a\in\mathbb{R}$ để hệ có nghiệm. (b) Tìm $a\in\mathbb{R}$ để hệ có đúng 2 nghiệm.
\end{baitoan}
Giải hệ phương trình:

\begin{baitoan}[\cite{TLCT_THCS_Toan_9_dai_so}, 15.5., p. 105]
	\begin{equation*}
		\left\{\begin{split}
			x + y + 3xy &= 21,\\
			x^2 + y^2 - xy &= -15.
		\end{split}\right.
	\end{equation*}
\end{baitoan}

\begin{baitoan}[\cite{TLCT_THCS_Toan_9_dai_so}, 15.6., p. 105]
	\begin{equation*}
		\left\{\begin{split}
			x^2 + 4xy + 7y^2 &= 28,\\
			x^2 + 3xy + 2y^2 &= 10.
		\end{split}\right.
	\end{equation*}
\end{baitoan}

\begin{baitoan}[\cite{TLCT_THCS_Toan_9_dai_so}, 15.7., p. 105]
	\begin{equation*}
		\left\{\begin{split}
			x^2 + 4xy + 3y^2 &= 8,\\
			x^2 - 9xy - 2y^2 &= -10.
		\end{split}\right.
	\end{equation*}
\end{baitoan}

\begin{baitoan}[\cite{TLCT_THCS_Toan_9_dai_so}, 15.8., p. 105]
	\begin{equation*}
		\left\{\begin{split}
			x^2 + y^2 - 2x + 3y &= 9,\\
			2x^2 + 2y^2 + x - 5y &= 1.
		\end{split}\right.
	\end{equation*}
\end{baitoan}

\begin{baitoan}[\cite{TLCT_THCS_Toan_9_dai_so}, 15.9., p. 105]
	\begin{equation*}
		\left\{\begin{split}
			15x^2 - 11xy + 2y^2 &= -7,\\
			2a^2x + 3ay &< 0,\\
			x &< y.
		\end{split}\right.
	\end{equation*}
\end{baitoan}

\begin{baitoan}[\cite{TLCT_THCS_Toan_9_dai_so}, 15.10., p. 105]
	\begin{equation*}
		\left\{\begin{split}
			x^4 + 6x^2y^2 + y^4 &= 136,\\
			x^3y + xy^3 &= 30.
		\end{split}\right.
	\end{equation*}
\end{baitoan}

\begin{baitoan}[\cite{TLCT_THCS_Toan_9_dai_so}, 15.11., p. 106]
	\begin{equation*}
		\left\{\begin{split}
			x^3 + 3xy^2 &= 158,\\
			3x^2y + y^3 &= -185.
		\end{split}\right.
	\end{equation*}
\end{baitoan}

\begin{baitoan}[\cite{TLCT_THCS_Toan_9_dai_so}, 15.12., p. 106]
	\begin{equation*}
		\left\{\begin{split}
			x^4 + x^2y^2 + y^4 &= 91,\\
			x^2 + xy + y^2 &= 13.
		\end{split}\right.
	\end{equation*}
\end{baitoan}

\begin{baitoan}[\cite{TLCT_THCS_Toan_9_dai_so}, 15.13., p. 106]
	\begin{equation*}
		\left\{\begin{split}
			x^3 + x^3y^3 + y^3 &= 17,\\
			x + xy + y &= 5.
		\end{split}\right.
	\end{equation*}
\end{baitoan}

%------------------------------------------------------------------------------%

\section{Đa Thức Bậc 2 Với Bất Đẳng Thức \& Toán Cực Trị}

\begin{baitoan}[\cite{Binh_Toan_9_tap_2}, VD119, p. 68]
	Cho đa thức bậc 2 $f(x) = ax^2 + bx + c,a\ne0$. Chứng minh: (a) Nếu $\Delta < 0$ thì $f(x)$ cùng dấu với $a$, $\forall x\in\mathbb{R}$. (b) Nếu $\Delta = 0$ thì $f(x)$ cùng dấu với $a$ với mọi giá trị của $x$ khác $-\dfrac{b}{2a}$. (c) Nếu $\Delta > 0$ thì $f(x)$ trái dấu với $a$ với mọi giá trị của $x$ nằm trong khoảng 2 nghiệm, $f(x)$ cùng dấu với $a$ với mọi giá trị của $x$ nằm ngoài khoảng 2 nghiệm.
\end{baitoan}

\begin{baitoan}[\cite{Binh_Toan_9_tap_2}, VD120, p. 69]
	Giải bất phương trình bậc 2 $x^2 - 2x - 1 > 0$.
\end{baitoan}

\begin{baitoan}[\cite{Binh_Toan_9_tap_2}, VD121, p. 69]
	Tìm $m\in\mathbb{R}$ để phương trình $mx^2 + (m - 2)x + 3 = 0$ có nghiệm.
\end{baitoan}

\begin{baitoan}[\cite{Binh_Toan_9_tap_2}, VD122, p. 70]
	Cho đẳng thức $x^2 - x + y^2 - y = xy$ {\rm(1)}. (a) Chứng minh $(x - 1)^2\le\dfrac{4}{3},(y - 1)^2\le\dfrac{4}{3}$. (b) Tìm $(x,y)\in\mathbb{Z}^2$ thỏa mãn đẳng thức {\rm(1)}.
\end{baitoan}

\begin{baitoan}[\cite{Binh_Toan_9_tap_2}, VD123, p. 70]
	Tìm {\rm GTNN, GTLN} của $A = \dfrac{x^2 - x + 1}{x^2 + x + 1}$.
\end{baitoan}

\begin{baitoan}[\cite{Binh_Toan_9_tap_2}, VD124, p. 72]
	Cho $A = \dfrac{x^2 + mx + n}{x^2 + 2x + 4}$. Tìm $m,n\in\mathbb{R}$ để $A$ có {\rm GTNN} bằng $\dfrac{1}{3}$, {\rm GTLN} bằng $3$.
\end{baitoan}

\begin{baitoan}[\cite{Binh_Toan_9_tap_2}, VD125, p. 73]
	Tìm {\rm GTNN} của biểu thức $A = (2x - 3)^3 - 7$ với $x\le-1$ hoặc $x\ge3$.
\end{baitoan}

\begin{baitoan}[\cite{Binh_Toan_9_tap_2}, VD126, p. 73]
	Gọi $x_1,x_2$ là 2 nghiệm của phương trình sau, tìm $m\in\mathbb{R}$ để $x_1^2 + x_2^2$ có {\rm GTNN}: (a) $x^2 - (2m - 1)x + m - 2 = 0$. (b) $x^2 + 2(m - 2)x - (2m - 7) = 0$.
\end{baitoan}

\begin{baitoan}[\cite{Binh_Toan_9_tap_2}, VD127, p. 74]
	Tìm {\rm GTNN} của $A = 3\left(\dfrac{a^2}{b^2} + \dfrac{b^2}{a^2}\right) - 8\left(\dfrac{b}{a} + \dfrac{a}{b}\right)$.
\end{baitoan}

\begin{baitoan}[\cite{Binh_Toan_9_tap_2}, 368., p. 75]
	Chứng minh nếu $x,y\in\mathbb{R}$ thỏa đẳng thức $x + y + xy = (x + y)^2$ thì $-\dfrac{1}{3}\le x,y\le1$.
\end{baitoan}

\begin{baitoan}[\cite{Binh_Toan_9_tap_2}, 369., p. 76]
	Chứng minh nếu $x,y\in\mathbb{R}$ thỏa đẳng thức $x^2 = 3(xy + y - y^2)$ thì $0\le y\le4$.
\end{baitoan}

\begin{baitoan}[\cite{Binh_Toan_9_tap_2}, 370., p. 76]
	Chứng minh nếu $a,b,c\in\mathbb{R}$ thỏa $a + b + c = 5,ab + bc + ca = 8$ thì $1\le a,b,c\le\dfrac{7}{3}$.
\end{baitoan}

\begin{baitoan}[\cite{Binh_Toan_9_tap_2}, 371., p. 76]
	Tìm {\rm GTNN} của $A = x^4 - 4x^3 + 8x + 20$.
\end{baitoan}

\begin{baitoan}[\cite{Binh_Toan_9_tap_2}, 372., p. 76]
	Tìm {\rm GTNN, GTLN}: (a) $A = \dfrac{x}{x^2 + 1}$. (b) $B = \dfrac{2x^2 + 4x + 5}{x^2 + 1}$. (c) $C = \dfrac{x^2 - 2x + 2}{x^2 + 2x + 2}$. (d) $D = \dfrac{x^2 + 2x + 2}{x^2 + 1}$.
\end{baitoan}

\begin{baitoan}[\cite{Binh_Toan_9_tap_2}, 373., p. 76]
	Tìm {\rm GTNN, GTLN} của $A = \dfrac{x^2 - xy + y^2}{x^2 + xy + y^2}$.
\end{baitoan}

\begin{baitoan}[\cite{Binh_Toan_9_tap_2}, 374., p. 76]
	Tìm {\rm GTNN} của $A = \left(\dfrac{a^2}{b^2} + \dfrac{b^2}{a^2}\right) - 3\left(\dfrac{a}{b} + \dfrac{b}{a}\right)$.
\end{baitoan}

\begin{baitoan}[\cite{Binh_Toan_9_tap_2}, 375., p. 76]
	Tìm {\rm GTNN} của $A = x + \sqrt{x^2 + \dfrac{1}{x}}$ với $x > 0$.
\end{baitoan}

\begin{baitoan}[\cite{Binh_Toan_9_tap_2}, 376., p. 76]
	Tìm $m,n\in\mathbb{R}$ để biểu thức $A = \dfrac{2x^2 + mx + n}{x^2 + 1}$ nhận {\rm GTNN} bằng $1$, {\rm GTLN} bằng $6$.
\end{baitoan}

\begin{baitoan}[\cite{Binh_Toan_9_tap_2}, 377., p. 76]
	Tìm {\rm GTNN} của $A = \left(xy + \dfrac{1}{xy}\right)^2$ với $x + y = 1$.
\end{baitoan}

\begin{baitoan}[\cite{Binh_Toan_9_tap_2}, 378., p. 76]
	Tìm {\rm GTNN} của $A = \dfrac{x}{1 - x} + \dfrac{5}{x}$ với $0 < x < 1$.
\end{baitoan}

\begin{baitoan}[\cite{Binh_Toan_9_tap_2}, 379., p. 76]
	Cho phương trình $x^4 + 2x^2 + 2ax + (a + 1)^2 = 0$. Tìm $a\in\mathbb{R}$ để nghiệm của phương trình: (a) Đạt {\rm GTNN}. (b) Đạt {\rm GTLN}.
\end{baitoan}

\begin{baitoan}[\cite{Binh_Toan_9_tap_2}, 380., p. 76]
	Cho phương trình $x^2 + ax + a - 5 = 0$ với $a\ge-1$. Tìm {\rm GTLN} mà nghiệm của phương trình có thể đạt được.
\end{baitoan}

%------------------------------------------------------------------------------%

\section{Phương Trình Đại Số Bậc Cao}
Giải phương trình:

\begin{baitoan}[\cite{Binh_Toan_9_tap_2}, VD1--4, pp. 78--80]
	(a) $x^3 + 3x^2 + 12x - 16 = 0$. (b) $x^3 - 3x - 2 = 0$. (c) $x^3 - 6x + 4 = 0$. (d) $x^4 + 8x^3 + 15x^2 - 4x - 2 = 0$.
\end{baitoan}

%------------------------------------------------------------------------------%

\section{Miscellaneous}
\cite[BTCCVII, \S3, pp. 66--67]{SGK_Toan_9_Canh_Dieu_tap_2}: 1. 2. 3. 4. 5. 6. 7. 8. 9. 10. 11.

\begin{baitoan}[\cite{Tuyen_Toan_9_old}, VD46, p. 97]
	Cho phương trình $(m - 2)x^2 - 2mx + m + 2 = 0$. (a) Giải phương trình với $m = -5$. (b) Tìm $m\in\mathbb{R}$ để phương trình có nghiệm duy nhất. (c) Tìm $m\in\mathbb{R}$ để phương trình có 2 nghiệm phân biệt. (d) Giả sử $x_1,x_2$ là 2 nghiệm của phương trình, tìm {\rm GTNN} của biểu thức $x_1^2 + x_2^2$. (e) Viết hệ thức liên hệ giữa 2 nghiệm không phụ thuộc vào $m$.
\end{baitoan}

\begin{baitoan}[\cite{Tuyen_Toan_9_old}, 259., p. 99]
	Cho 2 phương trình $x^2 + 2bx + c = 0,x^2 + 2cx + b = 0$. Chứng minh nếu $b + c\ge2$ thì ít nhất 1 trong 2 phương trình phải có nghiệm.
\end{baitoan}

\begin{baitoan}[\cite{Tuyen_Toan_9_old}, 260., p. 99]
	Cho phương trình $x^2 + 2x^2 - 2mx + (m - 1)^2 = 0$. Tìm $m\in\mathbb{R}$ để phương trình này có nghiệm lớn nhất, nhỏ nhất.
\end{baitoan}

\begin{baitoan}[\cite{Tuyen_Toan_9_old}, 261., p. 99]
	Giải phương trình: (a) $\dfrac{x^2 + 6x}{x + 1}\left(x - \dfrac{x + 6}{x + 1}\right) = 27$. (b) $\dfrac{3x}{x^2 - x + 3} - \dfrac{2x}{x^2 - 3x + 3} = -1$.
\end{baitoan}

\begin{baitoan}[\cite{Tuyen_Toan_9_old}, 262., p. 99]
	Cho 2 phương trình bậc 2 $ax^2 + bx + c = 0,ay^2 + by - c = 0$. (a) Chứng minh ít nhất 1 trong 2 phương trình phải có nghiệm. (b) Tìm điều kiện để 2 phương trình cùng có nghiệm. (c) Giả sử $x_1,x_2,y_1,y_2$ lần lượt là các nghiệm của 2 phương trình, chứng minh $(y_1 - y_2)^2 - (x_1 - x_2)^2 = 8x_1x_2$.
\end{baitoan}

\begin{baitoan}[\cite{Tuyen_Toan_9_old}, 263., p. 99]
	Cho phương trình $x^2 + 2(a + b)x + 4ab = 0$. (a) Chứng minh phương trình luôn có nghiệm $x_1,x_2$, $\forall a,b\in\mathbb{R}$. (b) Tính $x_1^2 + x_2^2$. (c) Tìm $a,b\in\mathbb{R}$ để phương trình có ít nhất 1 nghiệm không âm.
\end{baitoan}

\begin{baitoan}[\cite{Tuyen_Toan_9_old}, 264., p. 99]
	Cho phương trình $x^2 - mx - \dfrac{1}{2m^2} = 0$, $m\in\mathbb{R}^\star$. (a) Chứng minh phương trình luôn có nghiệm. (b) Tìm {\rm GTNN} của biểu thức $x_1^4 + x_2^4$.
\end{baitoan}

\begin{baitoan}[\cite{Tuyen_Toan_9_old}, 265., p. 100]
	Cho phương trình $mx^2 + 2(m - 2)x + m - 3 = 0$. (a) Tìm $m\in\mathbb{R}$ để phương trình có 2 nghiệm trái dấu. (b) Tìm $m\in\mathbb{R}$ để phương trình có 2 nghiệm trái dấu\& nghiệm âm có giá trị tuyệt đối lớn hơn. (c) Gọi $x_1,x_2$ là 2 nghiệm của phương trình. Viết 1 hệ thức liên hệ giữa 2 nghiệm không phụ thuộc $m$. (d) Tìm {\rm GTNN} của biểu thức $x_1^2 + x_2^2$.
\end{baitoan}

\begin{baitoan}[\cite{Tuyen_Toan_9_old}, 266., p. 100]
	Rút gọn biểu thức $A = \left(\dfrac{\sqrt{1 + x}}{\sqrt{1 + x} - \sqrt{1 - x}} - \dfrac{1 - x}{\sqrt{1 - x^2} - 1 + x}\right)$$\left(\dfrac{x - 1}{x} + \sqrt{\dfrac{1}{x^2}}\right)$ với $0 < x < 1$.
\end{baitoan}

\begin{baitoan}[\cite{Tuyen_Toan_9_old}, 267., p. 100]
	Cho biểu thức $A = \left(1 - \dfrac{\sqrt{x}}{\sqrt{x} + 1}\right):\left(\dfrac{\sqrt{x} + 3}{\sqrt{x} - 2} + \dfrac{\sqrt{x} + 2}{3 - \sqrt{x}} + \dfrac{\sqrt{x} + 2}{x - 5\sqrt{x} + 6}\right)$. (a) Rút gọn A. (b) Tìm $x\in\mathbb{R}$ để $A > 0$. (c) Tìm $m\in\mathbb{R}$ để có các giá trị của $x$ thỏa mãn $A(\sqrt{x} + 1) = m(x + 1) - 2$.
\end{baitoan}

\begin{baitoan}[\cite{Tuyen_Toan_9_old}, 268., p. 100]
	Biện luận theo tham số $m\in\mathbb{R}$ số nghiệm của hệ phương trình:
	\begin{equation*}
		\left\{\begin{split}
			x + my &= m - 1,\\
			mx + y &= m^2 + 1.
		\end{split}\right.
	\end{equation*}
\end{baitoan}

\begin{baitoan}[\cite{Tuyen_Toan_9_old}, 269., p. 100]
	Cho đường thẳng $(d):y = mx + b$ đi qua 2 điểm $A(m,4),B(1,m)$ với $m < 0$. (a) Chứng minh $(d)$ đi qua gốc tọa độ. (b) Tính góc $\alpha$ tạo bởi $(d)$ \& $Ox$.
\end{baitoan}

\begin{baitoan}[\cite{Tuyen_Toan_9_old}, 270., pp. 100--101]
	Cho parabol $(P):y = x^2$, đường thẳng $(d)$ có hệ số góc $k\in\mathbb{R}$ đi qua điểm $M(0,1)$. (a) Chứng minh $\forall k\in\mathbb{R}$, $(d)$ luôn cắt $(P)$ tại 2 điểm phân biệt $A,B$. (b) Gọi hoành độ của $A,B$ lần lượt là $x_1,x_2$. Chứng minh $|x_1 - x_2|\ge2$. (c) Chứng minh $\Delta OAB$ vuông.
\end{baitoan}

\begin{baitoan}[\cite{Tuyen_Toan_9_old}, 271., p. 101]
	Cho parabol $(P):y = \frac{1}{2}x^2$, đường thẳng $(d):mx + y = 2$. (a) Chứng minh khi $m$ thay đổi thì $(d)$ luôn đi qua 1 điểm cố định C. (b) Chứng minh $(d)$ luôn cắt $(P)$ tại 2 điểm phân biệt $A,B$. (c) Tìm $m\in\mathbb{R}$ để $AB$ ngắn nhất, khi đó, tính diện tích $\Delta AOB$. (d) Chứng minh trung điểm I của AB khi $m$ thay đổi luôn nằm trên 1 parabol cố định.
\end{baitoan}

\begin{baitoan}[\cite{Tuyen_Toan_9_old}, 273., p. 101]
	2 ôtô khởi hành cùng 1 lúc từ 2 địa điểm $A,B$ cách nhau {\rm165 km}, đi ngược chiều nhau, sau {\rm1 h 30 ph} thì gặp nhau. Tính vận tốc mỗi xe biết thời gian xe 1 chạy hết quãng đường AB nhiều hơn thời gian xe 2 chạy hết quãng đường ấy là {\rm33 ph}.
\end{baitoan}

\begin{baitoan}[\cite{Tuyen_Toan_9_old}, 274., p. 101]
	2 địa điểm $A,B$ cách nhau {\rm150 km}. Xe 1 khởi hành từ A đi về B, sau đó {\rm40 ph} xe 2 khởi hành từ B đi về A với vận tốc nhỏ hơn vận tốc xe 1 là {\rm10 km{\tt/}h}. Biết 2 xe gặp nhau khi xe I đã đi được 1 quãng đường gấp đôi quãng đường xe 2 đã đi. Tính vận tốc mỗi xe biết vận tốc của chúng không nhỏ hơn {\rm30 km{\tt/}h}.
\end{baitoan}

%------------------------------------------------------------------------------%

\printbibliography[heading=bibintoc]
	
\end{document}