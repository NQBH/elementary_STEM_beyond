\documentclass{article}
\usepackage[backend=biber,natbib=true,style=alphabetic,maxbibnames=50]{biblatex}
\addbibresource{/home/nqbh/reference/bib.bib}
\usepackage[utf8]{vietnam}
\usepackage{tocloft}
\renewcommand{\cftsecleader}{\cftdotfill{\cftdotsep}}
\usepackage[colorlinks=true,linkcolor=blue,urlcolor=red,citecolor=magenta]{hyperref}
\usepackage{amsmath,amssymb,amsthm,float,graphicx,mathtools,tikz}
\usetikzlibrary{angles,calc,intersections,matrix,patterns,quotes,shadings}
\allowdisplaybreaks
\newtheorem{assumption}{Assumption}
\newtheorem{baitoan}{}
\newtheorem{cauhoi}{Câu hỏi}
\newtheorem{conjecture}{Conjecture}
\newtheorem{corollary}{Corollary}
\newtheorem{dangtoan}{Dạng toán}
\newtheorem{definition}{Definition}
\newtheorem{dinhly}{Định lý}
\newtheorem{dinhnghia}{Định nghĩa}
\newtheorem{example}{Example}
\newtheorem{ghichu}{Ghi chú}
\newtheorem{hequa}{Hệ quả}
\newtheorem{hypothesis}{Hypothesis}
\newtheorem{lemma}{Lemma}
\newtheorem{luuy}{Lưu ý}
\newtheorem{nhanxet}{Nhận xét}
\newtheorem{notation}{Notation}
\newtheorem{note}{Note}
\newtheorem{principle}{Principle}
\newtheorem{problem}{Problem}
\newtheorem{proposition}{Proposition}
\newtheorem{question}{Question}
\newtheorem{remark}{Remark}
\newtheorem{theorem}{Theorem}
\newtheorem{vidu}{Ví dụ}
\usepackage[left=1cm,right=1cm,top=5mm,bottom=5mm,footskip=4mm]{geometry}
\def\labelitemii{$\circ$}
\DeclareRobustCommand{\divby}{%
	\mathrel{\vbox{\baselineskip.65ex\lineskiplimit0pt\hbox{.}\hbox{.}\hbox{.}}}%
}

\title{Problem: System of 1st-Order Equations -- Bài Tập: Hệ Phương Trình Bậc Nhất}
\author{Nguyễn Quản Bá Hồng\footnote{Independent Researcher, Ben Tre City, Vietnam\\e-mail: \texttt{nguyenquanbahong@gmail.com}; website: \url{https://nqbh.github.io}.}}
\date{\today}

\begin{document}
\maketitle
\tableofcontents

%------------------------------------------------------------------------------%

\section{Phương Trình Quy Về Phương Trình Bậc Nhất 1 Ẩn}
\cite[\S1, pp. 5--11]{SGK_Toan_9_Canh_Dieu_tap_1}: HD1. LT1. LT2. HD2. LT3. HD3. LT4. LT5. 1. 2. 3. 4. 5. 6.

See also NQBH's \textit{Problem: 1st-Order Function -- Bài Tập: Hàm Số Bậc Nhất $y = ax + b,a\ne0$}\footnote{\textsc{url}: \url{https://github.com/NQBH/elementary_STEM_beyond/blob/main/elementary_mathematics/grade_8/1st_order_function/problem/NQBH_1st_order_function_problem.pdf}.}.

%------------------------------------------------------------------------------%

\section{1st-Order Equations of 2 Unknowns -- Phương Trình Bậc Nhất 2 Ẩn}
\fbox{1} Phương trình bậc nhất 2 ẩn: $ax + by = c$ (1), $a,b,c\in\mathbb{R},(a,b)\ne(0,0)$. \fbox{2} $(x_0,y_0)\in\mathbb{R}^2$ là nghiệm của (1) $\Leftrightarrow(x_0,y_0)\in S\Leftrightarrow ax_0 + by_0 = c$. \fbox{3} Tập nghiệm $S$ biểu diễn bởi đường thẳng $(d):ax + by = c$, i.e., $S = \{(x,y)\in\mathbb{R}^2|ax + by = c\} = (d)$. \fbox{4} Nếu $ab\ne0$ thì $(d):ax + by = c\Leftrightarrow y = -\dfrac{a}{b}x + \dfrac{c}{b}$ (hàm số bậc nhất) là đường thẳng cắt cả 2 trục tọa độ $Ox,Oy$ lần lượt tại 2 điểm $\left(\dfrac{c}{a},0\right),\left(0,\dfrac{c}{b}\right)$. \fbox{5} Nếu $a\ne0,b = 0$ thì $(d):ax + 0y = c\Leftrightarrow x = \dfrac{c}{a}$ là đường thẳng song song hoặc trùng với trục tung $Oy$. \fbox{6} Nếu $a = 0,b\ne0$ thì $(d):0x + by = c\Leftrightarrow y = \dfrac{c}{b}$ là đường thẳng song song hoặc trùng với trục hoành $Ox$.

\begin{baitoan}[\cite{Binh_boi_duong_Toan_9_tap_2}, VD1, p. 8]
	Cho phương trình $2x - y = 6$. (a) Tìm công thức nghiệm tổng quát của phương trình. (b) Vẽ đường thẳng biểu diễn tập nghiệm.
\end{baitoan}

\begin{baitoan}[\cite{Binh_boi_duong_Toan_9_tap_2}, VD2, p. 9]
	(a) Cho phương trình $ax + 2y = 4$. Tìm $a\in\mathbb{R}$ để đường thẳng $(d)$ biểu diễn tập nghiệm của phương trình đi qua điểm $A(1,1.5)$. (b) Vẽ 2 đường thẳng $(d),(t):-2x + y = -3$ trên cùng 1 hệ trục tọa độ. Tìm tọa độ giao điểm của 2 đường thẳng.
\end{baitoan}

\begin{baitoan}[\cite{Binh_boi_duong_Toan_9_tap_2}, VD3, p. 9]
	(a) Tìm nghiệm nguyên của phương trình $2x - 3y = 6$. (b) Tìm nghiệm nguyên dương của phương trình $2x + 5y = 9$.
\end{baitoan}

\begin{baitoan}[\cite{Binh_boi_duong_Toan_9_tap_2}, 1.1., p. 10]
	Trong các cặp số $(0,3),(2,1),(1.5,-2),(4,-6),(-2,0)$, cặp số nào là nghiệm của phương trình: (a) $2x - 5y = -1$. (b) $3x + 4y = -6$.
\end{baitoan}

\begin{baitoan}[\cite{Binh_boi_duong_Toan_9_tap_2}, 1.2., p. 10]
	Tìm $a,b\in\mathbb{R}$ để: (a) Điểm $A(0,-3)$ thuộc đường thẳng $2x + by = -6$. (b) Điểm $B(-2,1)$ thuộc đường thẳng $ax + 4y = 8$. (c) Điểm $C(2.5,0)$ thuộc đường thẳng $ax - 5y = 7.5$. (d) Điểm $D(2,-4)$ thuộc đường thẳng $5x + by = -4$.
\end{baitoan}

\begin{baitoan}[\cite{Binh_boi_duong_Toan_9_tap_2}, 1.3., p. 10]
	Tìm nghiệm tổng quát của phương trình: (a) $3x - 5y = 15$. (b) $5x + 0y = -4$. (c) $0x + 9y = 27$. Vẽ 3 đường thẳng biểu diễn 3 tập nghiệm \& nhận xét.
\end{baitoan}

\begin{baitoan}[\cite{Binh_boi_duong_Toan_9_tap_2}, 1.4., p. 10]
	Tìm nghiệm tổng quát của phương trình: (a) $2x + y = -6$. (b) $x - 2y = 2$. Vẽ 2 đường thẳng biểu diễn tập nghiệm của 2 phương trình trên cùng 1 hệ trục tọa độ. Tìm tọa độ giao điểm của 2 đường thẳng.
\end{baitoan}

\begin{baitoan}[\cite{Binh_boi_duong_Toan_9_tap_2}, 1.5., p. 11]
	Tìm nghiệm nguyên của phương trình: (a) $x + 2y = 0$. (b) $3x - y = 0$. (c) $2x - 5y = 10$. (d) $7x + 4y = 13$.
\end{baitoan}

\begin{baitoan}[\cite{Binh_boi_duong_Toan_9_tap_2}, 1.6., p. 11]
	Tìm nghiệm nguyên dương của phương trình: (a) $5x + 3y - 13 = 0$. (b) $28x + 31y = 273$.
\end{baitoan}

\begin{baitoan}[\cite{Binh_boi_duong_Toan_9_tap_2}, 1.7., p. 11]
	Cho đường thẳng $(d):(a - 2)x + (2a + 3)y - a - 5 = 0$. (a) Tìm $a$ để $(d)$ song song với trục hoành. (b) Tìm $a$ để $(d)$ song song với trục tung. (c) Tìm $a$ để $(d)$ đi qua điểm $A(-3,5)$. (d) Tìm điểm cố định mà $(d)$ luôn đi qua $\forall a\in\mathbb{R}$.
\end{baitoan}

\begin{baitoan}[\cite{Binh_boi_duong_Toan_9_tap_2}, 1.8., p. 11]
	Cho 2 đường thẳng $(d_1):ax - 4y = 8,(d_2):5x - 10y = b - 5$. (a) Tính $2016a + 100b$ khi $(d_1),(d_2)$ cắt nhau tại $A(2,-3)$. (b) Tính $10a - 3b$ khi 2 đường thẳng $(d_1),(d_2)$ có vô số điểm chung.
\end{baitoan}

\begin{baitoan}[\cite{Binh_boi_duong_Toan_9_tap_2}, 1.9., p. 11]
	Trên 1 đoạn đường phố thẳng dài {\rm100 m} 1 đội công nhân lắp đường ống dẫn nước. Có 2 loại ống, 1 loại dài {\rm3 m}, 1 loại dài {\rm5 m}. Hỏi có bao nhiêu các lắp các ống nước trên đoạn đường đó (các mối nối không đáng kể)?
\end{baitoan}

\begin{baitoan}[\cite{Binh_boi_duong_Toan_9_tap_2}, p. 12]
	1 người bán hàng phải trả lại cho khách $45000$ đồng. Người đó chỉ có 1 tờ $10000$ đồng đã trả lại cho khách hàng, còn lại chỉ có 2 loại tiền lẻ là $2000$ đồng \& $5000$ đồng. Cho biết người bán hàng sẽ có các cách nào trả lại tiếp cho khách hàng đúng số tiền trên?
\end{baitoan}

\begin{baitoan}[\cite{Binh_boi_duong_Toan_9_tap_2}, p. 12]
	Chứng minh khoảng cách $h$ từ gốc tọa độ $O$ đến đường thẳng $ax + by = c$ được cho bởi công thức $h = OH = \dfrac{|c|}{\sqrt{a^2 + b^2}}$ với $H$ là hình chiếu của $O$ lên đường thẳng.
\end{baitoan}

\begin{baitoan}[\cite{Binh_boi_duong_Toan_9_tap_2}, p. 12, Điều kiện để phương trình bậc nhất 2 ẩn có nghiệm nguyên]
	Chứng minh phương trình $ax + by = c$, $a,b,c\in\mathbb{Z},(a,b)\ne(0,0)$, có nghiệm nguyên khi \& chỉ khi $c\divby\mbox{\rm ƯCLN}(a,b)$.
\end{baitoan}

\begin{dinhnghia}[Điểm nguyên]
	Trong mặt phẳng tọa độ, điểm $A(x,y)\in\mathbb{R}^2$ được gọi là {\rm điểm nguyên} nếu $x,y\in\mathbb{Z}$.
\end{dinhnghia}

\begin{baitoan}[\cite{Binh_boi_duong_Toan_9_tap_2}, p. 12]
	Tìm tất cả các điểm nguyên trên đường thẳng $4x - 5y + 6 = 0$, nằm giữa 2 đường thẳng: (a) $x = -10,x = 20$. (b) $x = a,x = b$ với $a,b\in\mathbb{R}$.
\end{baitoan}

\begin{baitoan}[\cite{Tuyen_Toan_9_old}, VD20, p. 44]
	Cho đường thẳng $(d):(m + 1)x + (m - 4)y = 6$. (a) Khi $m = 2$, vẽ đồ thị $(d)$. (b)  Viết công thức tổng quát nghiệm của phương trình \& tìm nghiệm nguyên của nó.
\end{baitoan}

\begin{baitoan}[\cite{Tuyen_Toan_9_old}, 126., p. 46]
	Cho 2 đường thẳng $(d_1):3x - 2y = m + 3,(d_2):(m - 5)x + 3y = 6$. (a) Tìm $m\in\mathbb{R}$ để $(d_1)$ cắt $(d_2)$. (b) Chứng minh $\forall m\in\mathbb{R}$ thì $(d_1),(d_2)$ là 2 đường thẳng phân biệt.
\end{baitoan}

\begin{baitoan}[\cite{Tuyen_Toan_9_old}, 127., p. 46]
	Cho đường thẳng $(d):(2m - 1)x + (m - 2)y = m^2 - 3$. Tìm $m\in\mathbb{R}$ để: (a) $(d)$ đi qua gốc tọa độ. (b) $(d)$ đi qua điểm $A(3,5)$. (c) $(d)$ cắt mỗi trục tọa độ tại 1 điểm khác gốc. (d) $(d)$ song song với 1 trong 2 trục tọa độ.
\end{baitoan}

\begin{baitoan}[\cite{Tuyen_Toan_9_old}, 128., p. 46]
	Cho đường tròn có tâm là gốc tọa độ O bán kính $1$ \& đường thẳng $(d)$ có phương trình $3x - 4y = m^2 - m + 3$. (a) Tìm $m\in\mathbb{R}$ để đường thẳng tiếp xúc với đường tròn. (b) Tính {\rm GTNN} của khoảng cách từ $O$ đến đường thẳng $(d)$.
\end{baitoan}

\begin{baitoan}[\cite{Tuyen_Toan_9_old}, 129., p. 47]
	Cho 2 đường thẳng $(d_1):(4m - 3)x + (m - 2)y = m + 1,m\ne 2,(d_2):-2x + 3y = 5$. Tìm $m\in\mathbb{R}$ để $(d_1)\bot(d_2)$.
\end{baitoan}

\begin{baitoan}[\cite{Tuyen_Toan_9_old}, 130., p. 47]
	Cho các đường thẳng $(d_1):mx + 9y = -3,(d_2):4x + my = n$. Đếm số cặp $(m,n)\in\mathbb{R}^2$ để $(d_1)$ trùng $(d_2)$.
\end{baitoan}

\begin{baitoan}[\cite{Tuyen_Toan_9_old}, 131., p. 47]
	Dùng đồ thị để biện luận số nghiệm của phương trình $|x + 1| + |x - 1| = m$.
\end{baitoan}

\begin{baitoan}[\cite{Tuyen_Toan_9_old}, 132., p. 47]
	Cho đường thẳng $(d):3x + 4y = 21$. (a) Viết công thức tổng quát nghiệm của phương trình. (b) Tìm các điểm trên đường thẳng $(d)$ có tọa độ nguyên \& nằm trong góc phần tư thứ (I).
\end{baitoan}

\begin{baitoan}[\cite{Binh_Toan_9_tap_2}, VD66, p. 5]
	Cho đường thẳng: $d:(m - 2)x + (m - 1)y = 1$ với tham số $m$. (a) Chứng minh đường thẳng $d$ luôn đi qua 1 điểm cố định với mọi giá trị của $m$. (b) Tìm giá trị của $m$ để khoảng cách từ gốc tọa độ O đến $d$ lớn nhất.
\end{baitoan}

\begin{baitoan}[\cite{Binh_Toan_9_tap_2}, VD67, p. 6]
	Tìm các điểm thuộc đường thẳng $3x - 5y = 8$ có tọa độ là các số nguyên \& nằm trên dải song song tạo bởi 2 đường thẳng $y = 10,y = 20$.
\end{baitoan}

\begin{baitoan}[\cite{Binh_Toan_9_tap_2}, 198., p. 8]
	Xét các đường thẳng $d$ có phương trình: $(2m + 3)x + (m + 5)y + 4m - 1 = 0$ với tham số $m$. (a) Vẽ đường thẳng $d$ ứng với $m = -1$. (b) Tìm điểm cố định mà mọi đường thẳng $d$ đều đi qua.
\end{baitoan}

\begin{baitoan}[\cite{Binh_Toan_9_tap_2}, 199., p. 8]
	Tìm các giá trị của $b,c$ để các đường thẳng $4x + by + c = 0,cx - 3y + 9 = 0$ trùng nhau.
\end{baitoan}

\begin{baitoan}[\cite{Binh_Toan_9_tap_2}, 200., p. 8]
	Vẽ đồ thị biểu diễn tập nghiệm của phương trình $x^2 - 2xy + y^2 = 1$.
\end{baitoan}

\begin{baitoan}[\cite{Binh_Toan_9_tap_2}, 201., p. 8]
	Đường thẳng $ax + by = 6$ với $a > 0,b > 0$, tạo với 2 trục tọa độ 1 tam giác có diện tích bằng $9$. Tính $ab$.
\end{baitoan}

\begin{baitoan}[\cite{Binh_Toan_9_tap_2}, 202., p. 8]
	Cho đường thẳng $d:(m + 2)x - my = -1$ với tham số $m$. (a) Tìm điểm cố định mà $d$ luôn đi qua. (b) Tìm giá trị của $m$ để khoảng cách từ gốc tọa độ O đến $d$ lớn nhất.
\end{baitoan}

\begin{baitoan}[\cite{Binh_Toan_9_tap_2}, 203., p. 8]
	Trong hệ trục tọa độ $Oxy$, $A(1,1),B(9,1)$. Viết phương trình của đường thẳng $d\bot AB$ \& chia $\Delta OAB$ thành 2 phần có diện tích bằng nhau.
\end{baitoan}

\begin{baitoan}[\cite{Binh_Toan_9_tap_2}, 204., p. 8]
	Tìm các điểm nằm trên đường thẳng $8x + 9y = -79$, có hoành độ \& tung độ là các số nguyên \& nằm bên trong góc vuông phần tư {\rm III}.
\end{baitoan}

\begin{baitoan}[\cite{Binh_Toan_9_tap_2}, 205., p. 8]
	Cho 2 điểm $A(3,17),B(33,193)$. (a) Viết phương trình của đường thẳng AB. (b) Có bao nhiêu điểm thuộc đoạn thẳng AB \& có hoành độ \& tung độ là các số nguyên?
\end{baitoan}

\begin{baitoan}[\cite{Binh_Toan_9_tap_2}, 206., p. 8]
	(a) Vẽ đồ thị hàm số $(d):y = \dfrac{3}{2}x + \dfrac{7}{4}$. (b) Có bao nhiêu điểm nằm trên cạnh hoặc nằm trong tam giác tạo bởi 3 đường thẳng $x = 6,y = 0,(d)$.
\end{baitoan}

\begin{baitoan}[\cite{TLCT_THCS_Toan_9_dai_so}, VD10.1, p. 52]
	Giải \& biện luận phương trình $(m - 1)x + my = m + 3$.
\end{baitoan}

\begin{baitoan}[\cite{TLCT_THCS_Toan_9_dai_so}, VD10.2, p. 52]
	Tìm tất cả các nghiệm của phương trình $x - y + 1 = 0$ thỏa mãn đẳng thức $y^2 - y\sqrt{x} - 2x = 0$.
\end{baitoan}

\begin{baitoan}[\cite{TLCT_THCS_Toan_9_dai_so}, VD10.3, pp. 52--53]
	Xét phương trình $ax + by = c$ với $a,b,c\in\mathbb{Z},(a,b)\ne(0,0)$. Đặt $d = \mbox{\rm ƯCLN}(a,b)$. Chứng minh: (a) Phương trình $ax + by = c$ có nghiệm nguyên $(x,y)\in\mathbb{Z}^2$ khi \& chỉ khi $c\divby d$. (b) Nếu biết 1 nghiệm nguyên $(x_0,y_0)\in\mathbb{Z}^2$ thì phương trình $ax + by = c$ giải được hoàn toàn.
\end{baitoan}

\begin{baitoan}[\cite{TLCT_THCS_Toan_9_dai_so}, VD10.4, p. 53]
	Giải phương trình $11x + 7y = 5$ trong $\mathbb{Z}$.
\end{baitoan}

\begin{baitoan}[\cite{TLCT_THCS_Toan_9_dai_so}, VD10.5, p. 54]
	Đếm số bộ $(x,y,z)\in\mathbb{N}^3$ thỏa phương trình: $x + y + z = 2012$. (b) $x + y + z = a\in\mathbb{N}$.
\end{baitoan}

\begin{baitoan}[\cite{TLCT_THCS_Toan_9_dai_so}, 10.1., p. 55]
	Giải phương trình $5x - 7y = 1$.
\end{baitoan}

\begin{baitoan}[\cite{TLCT_THCS_Toan_9_dai_so}, 10.2., p. 55]
	Giải \& biện luận phương trình $mx + (3m - 2)y = 2m + 1$.
\end{baitoan}

\begin{baitoan}[\cite{TLCT_THCS_Toan_9_dai_so}, 10.3., p. 55]
	Tìm tất cả các nghiệm của phương trình $x - y + 1 = 0$ thỏa đẳng thức $(y - 1)^2 + 2(y - 1)\sqrt{x} - 3x = 0$.
\end{baitoan}

\begin{baitoan}[\cite{TLCT_THCS_Toan_9_dai_so}, 10.4., p. 55]
	Giải phương trình $123x - 5y = 0$ trong $\mathbb{Z}$.
\end{baitoan}

\begin{baitoan}[\cite{TLCT_THCS_Toan_9_dai_so}, 10.5., p. 55]
	Cho phương trình $3x + my = m + 2$. Tìm $m\in\mathbb{R}$ để phương trình: (a) Nhận cặp $(3,5)$ làm 1 nghiệm. (b) Nhận cặp $\left(\frac{5}{3},0\right)$ làm nghiệm.
\end{baitoan}

\begin{baitoan}[\cite{TLCT_THCS_Toan_9_dai_so}, 10.6., p. 55]
	Đếm số cặp $(x,y)\in\mathbb{Z}^2$ với $x\ge12,y\ge17$ thỏa phương trình $x + y = 100$.
\end{baitoan}

\begin{baitoan}[\cite{TLCT_THCS_Toan_9_dai_so}, 10.7., p. 55]
	Đếm số bộ $(x,y,z)\in\mathbb{Z}^3$ với $x\ge1,y\ge7,z\ge0$ thỏa phương trình $x + y + z = 2012$.
\end{baitoan}

\begin{baitoan}[\cite{TLCT_THCS_Toan_9_dai_so}, 10.8., p. 55]
	Chứng minh tổng số các nghiệm tự nhiên của các phương trình $x + 2y = n,2x + 3y = n - 1,\ldots,nx + (n + 1)y = 1,(n + 1)x + (n + 2)y = 0$ đúng bằng $n + 1$, $\forall n\in\mathbb{N}^\star$.
\end{baitoan}

\begin{baitoan}[\cite{TLCT_THCS_Toan_9_dai_so}, 10.9., p. 55]
	Chứng minh tổng số nghiệm tự nhiên của các phương trình $x + 4y = 3n - 1,4x + 9y = 5n - 4,9x + 16y = 7n - 9,\ldots,n^2x + (n + 1)^2y = n(n + 1)$ đúng bằng $n$, $\forall n\in\mathbb{N}^\star$.
\end{baitoan}

%------------------------------------------------------------------------------%

\section{Diophantine Equation -- Phương Trình Nghiệm Nguyên}

\begin{dinhly}[Nghiệm nguyên của phương trình bậc nhất 2 ẩn]
	Phương trình $ax + by = c$ với $a,b,c\in\mathbb{Z},(a,b)\ne(0,0)$ có nghiệm nguyên khi \& chỉ khi $c\divby\mbox{\rm ƯCLN}(a,b)$.
\end{dinhly}

\begin{baitoan}[\cite{Tuyen_Toan_9_old}, VD21, p. 48]
	Tìm nghiệm nguyên của phương trình $35x + 20y = 600$.
\end{baitoan}

\begin{baitoan}[\cite{Tuyen_Toan_9_old}, VD22, p. 48]
	Tìm nghiệm nguyên của phương trình $7x + 4y = 23$.
\end{baitoan}

\begin{dinhly}[Phương pháp tìm 1 nghiệm riêng]
	Nếu $(x_0,y_0)\in\mathbb{Z}^2$ là 1 nghiệm nguyên của phương trình $ax + by = c$ với $a,b,c\in\mathbb{Z},ab\ne0$ thì phương trình có vô số nghiệm nguyên \& mọi nghiệm nguyên của nó đều có thể biểu diễn dưới dạng $(x,y) = (x_0 + by,y_0 - at)$ với $t\in\mathbb{Z}$, i.e., $S = \{(x_0 + by,y_0 - at)|t\in\mathbb{Z}\}$.
\end{dinhly}
Cặp số nguyên $(x_0,y_0)$ gọi là 1 \textit{nghiệm riêng}. Theo định lý, để tìm tất cả các nghiệm nguyên của phương trình $ax + by = c$ với $a,b,c\in\mathbb{Z},ab\ne0$, chỉ cần tìm ra 1 nghiệm riêng của phương trình. Trong trường hợp đơn giản ta có thể tính nhẩm nghiệm riêng này bằng cách thử chọn. Có nhiều công thức cùng biểu thị tập hợp các nghiệm nguyên của 1 phương trình bậc nhất 2 ẩn, tùy theo nghiệm riêng đã tìm được.

\begin{baitoan}[\cite{Tuyen_Toan_9_old}, VD23, p. 50]
	Tìm nghiệm nguyên của phương trình $5x - 3y = 2$.
\end{baitoan}

\begin{dinhnghia}[Phương trình đa thức]
	Phương trình có dạng $f(x,y,z,\ldots) = 0$ trong đó $f(x,y,z,\ldots)\in\mathbb{R}[x,y,z,\ldots]$ là đa thức của các biến $x,y,z,\ldots$ gọi là {\rm phương trình đa thức}.
\end{dinhnghia}

\begin{baitoan}[\cite{Tuyen_Toan_9_old}, VD24, p. 4]
	Tìm nghiệm nguyên của phương trình $x + y + xy = 4$.
\end{baitoan}

\begin{dinhnghia}[Phương trình ước số]
	Phương trình có dạng $\prod_{i=1}^n (a_ix_i + b_i) = (a_1x_1 + b_1)(a_2x_2 + b_2)\cdots(a_nx_n + b_n) = a$ với $n\in\mathbb{N},n\ge2$, $a,a_i,b_i\in\mathbb{Z},\forall i = 1,2,\ldots,n$ gọi là {\rm phương trình ước số} với $n$ biến $x_1,x_2,\ldots,x_n$.
\end{dinhnghia}

\begin{baitoan}[\cite{Tuyen_Toan_9_old}, VD25, p. 52]
	Chứng minh phương trình $4x^2 + 4x = 8y^3 - 2z^2 + 4$ không có nghiệm nguyên.
\end{baitoan}

\begin{baitoan}[\cite{Tuyen_Toan_9_old}, VD26, p. 52]
	Tìm nghiệm nguyên dương của phương trình $\dfrac{1}{x} + \dfrac{1}{y} = \dfrac{1}{2}$.
\end{baitoan}
1số phương pháp giải phương trình nghiệm nguyên: Phương pháp phát hiện tính chia hết của 1 ẩn. Phương pháp tách ra các giá trị nguyên. Phương pháp tìm 1 nghiệm riêng. Phương pháp đưa về phương trình ước số. Phương pháp xét số dư từng vế. Phương pháp dùng bất đẳng thức. See \cite{Binh_PTNN}.

\begin{baitoan}[\cite{Tuyen_Toan_9_old}, 133., p. 53]
	Tìm nghiệm nguyên của phương trình $18x - 30y = 59$.
\end{baitoan}

\begin{baitoan}[\cite{Tuyen_Toan_9_old}, 134., p. 53]
	Tìm nghiệm nguyên của phương trình $22x - 5y = 77$.
\end{baitoan}

\begin{baitoan}[\cite{Tuyen_Toan_9_old}, 135., p. 54]
	Cho phương trình $12x + 19y = 94$. (a) Tìm các nghiệm nguyên. (b) Tìm các nghiệm nguyên dương.
\end{baitoan}

\begin{baitoan}[\cite{Tuyen_Toan_9_old}, 136., p. 54]
	Tìm nghiệm nguyên dương của phương trình: (a) $13x + 3y = 50$. (b) $21x + 31y = 280$.
\end{baitoan}

\begin{baitoan}[\cite{Tuyen_Toan_9_old}, 137., p. 54]
	Cho phương trình $ax + by = ab$ với $a,b\in\mathbb{N}^\star,\mbox{\rm ƯCLN}(a,b) = 1$. Phương trình có nghiệm nguyên dương không?
\end{baitoan}

\begin{baitoan}[\cite{Tuyen_Toan_9_old}, 138., p. 54]
	Tìm nghiệm nguyên của phương trình $4x + 10y = m^2 - 1$ với $m\in\mathbb{Z}$.
\end{baitoan}

\begin{baitoan}[\cite{Tuyen_Toan_9_old}, 139., p. 54]
	Tìm nghiệm nguyên của phương trình $4x + 6y - 5z = 10$.
\end{baitoan}

\begin{baitoan}[\cite{Tuyen_Toan_9_old}, 140., p. 54]
	Tìm nghiệm nguyên của phương trình $6x^2y^3 + 3x^2 - 10y^3 = -2$.
\end{baitoan}

\begin{baitoan}[\cite{Tuyen_Toan_9_old}, 141., p. 54]
	Tìm nghiệm nguyên của phương trình $7(x - 1) + 3y = 2xy$.
\end{baitoan}

\begin{baitoan}[\cite{Tuyen_Toan_9_old}, 142., p. 54]
	Tìm nghiệm nguyên của phương trình $x^2 + 2xy + 2y^2 - 10yz + 25z^2 = 567$.
\end{baitoan}

\begin{baitoan}[\cite{Tuyen_Toan_9_old}, 143., p. 54]
	Tìm nghiệm nguyên dương của phương trình $\dfrac{4}{x} + \dfrac{2}{y} = 1$.
\end{baitoan}

\begin{baitoan}[\cite{Tuyen_Toan_9_old}, 144., p. 54]
	Tìm nghiệm nguyên của phương trình $\dfrac{xy}{z} + \dfrac{yz}{x} + \dfrac{zx}{y} = 3$.
\end{baitoan}

\begin{baitoan}[\cite{Tuyen_Toan_9_old}, 145., p. 54]
	Tìm $x\in\mathbb{Z}$ để $\sqrt{199 - x^2 - 2x} + 2$ là 1 số chính phương chẵn.
\end{baitoan}

\begin{baitoan}[\cite{Tuyen_Toan_9_old}, 146., p. 54]
	1 người bán hàng cần phải trả lại cho khách $25000$ đồng mà chỉ còn 2 loại tiền lẻ là $2000,5000$. Có các cách nào để trả lại cho khách hàng đúng số tiền trên?
\end{baitoan}

\begin{baitoan}[\cite{Tuyen_Toan_9_old}, 147., p. 54]
	2 đội thi đấu cờ với nhau. Mỗi đấu thủ của đội này phải đấu 1 ván với mỗi đấu thủ của đội kia. Biết tổng số ván cờ đã đấu bằng bình phương số đấu thủ của đội thứ nhất cộng với $2$ lần số đấu thủ của đội thứ 2. Tính số người mỗi đội.
\end{baitoan}

%------------------------------------------------------------------------------%

\section{System of 1st-Order Equations of 2 Unknowns -- Hệ Phương Trình Bậc Nhất 2 Ẩn}
\fbox{1} Hệ phương trình bậc nhất 2 ẩn: $a,b,a',b'\in\mathbb{R}$,
\begin{equation}
	\label{system of linear eqns 2 unknowns}
	\left\{\begin{split}
		ax + by &= c,\ (d),(a,b)\ne(0,0),\\
		a'x + b'y &= c',\ (d'),(a',b')\ne(0,0),
	\end{split}\right.
\end{equation}
có 1 nghiệm $\Leftrightarrow(d)$ cắt $(d')\Leftrightarrow\dfrac{a}{a'}\ne\dfrac{b}{b'}$, vô nghiệm $\Leftrightarrow(d)\parallel(d')\Leftrightarrow\dfrac{a}{a'} = \dfrac{b}{b'}\ne\dfrac{c}{c'}$, vô số nghiệm $\Leftrightarrow(d)\equiv(d')\Leftrightarrow\dfrac{a}{a'} = \dfrac{b}{b'} = \dfrac{c}{c'}$. \fbox{2} \textit{Phương pháp thế}: Biểu diễn 1 ẩn theo ẩn kia. Biến hệ phương trình thành hệ mới có 1 phương trình 1 ẩn. Giải phương trình 1 ẩn rồi suy ra nghiệm của hệ. \fbox{3} \textit{Phương pháp cộng đại số}: Nhân 2 vế của 2 phương trình với 1 số thích hợp để các hệ số của 1 ẩn nào đó trong 2 phương trình bằng nhau hoặc đối nhau. Dùng quy tắc cộng được hệ mới có 1 phương trình 1 ẩn. Giải phương trình 1 ẩn rồi suy ra nghiệm của hệ. \fbox{4} \textit{Giải hệ phương trình bằng phương pháp định thức{\tt/}Cramer}: Đặt $D = ab' - a'b,D_x = b'c - bc',D_y = c'a - ca'$. Nếu $D\ne0$, hệ \eqref{system of linear eqns 2 unknowns} có 1 nghiệm duy nhất $(x,y) = \left(\dfrac{D_x}{D},\dfrac{D_y}{D}\right) = \left(\dfrac{b'c - bc'}{ab' - a'b},\dfrac{c'a - ca'}{ab' - a'b}\right)$. Nếu $D = 0,(D_x,D_y)\ne(0,0)$, hệ \eqref{system of linear eqns 2 unknowns} vô nghiệm. Nếu $D = D_x = D_y = 0$, hệ \eqref{system of linear eqns 2 unknowns} có vô số nghiệm. Biểu thức $pq' - p'q$ gọi là 1 \textit{định thức cấp 2}. Phương pháp định thức rất có lợi trong việc giải \& biện luận hệ phương trình bậc nhất nhiều ẩn.\\

\noindent\cite[\S2, pp. 12--18]{SGK_Toan_9_Canh_Dieu_tap_1}: HD1. LT1. HD2. LT2. HD3. LT3. LT4. 1. 2. 3. 4. 5. 6. \cite[\S3, pp. 19--25]{SGK_Toan_9_Canh_Dieu_tap_1}: HD1. LT1. LT2. LT3. HD2. LT4. HD3. LT5. LT6. 1. 2. 3. 4. 5. 6. 7.

\begin{baitoan}[\cite{Binh_boi_duong_Toan_9_tap_2}, H1, p. 14]
	Đoán số nghiệm của hệ phương trình:
	\begin{equation*}
		\left\{\begin{split}
			y &= 2x - 3,\\
			y &= 2 - 3x.
		\end{split}\right.
	\end{equation*}
\end{baitoan}

\begin{baitoan}[\cite{Binh_boi_duong_Toan_9_tap_2}, H2, p. 14]
	Đoán số nghiệm của hệ phương trình:
	\begin{equation*}
		\left\{\begin{split}
			x + y &= 5,\\
			4x + 4y &= 5.
		\end{split}\right.\hspace{5mm}
		\left\{\begin{split}
			x - y &= 5,\\
			-4x + 4y &= -20.
		\end{split}\right.
	\end{equation*}
\end{baitoan}

\begin{baitoan}[\cite{Binh_boi_duong_Toan_9_tap_2}, H3, p. 14]
	Biện luận theo 2 tham số $a,b\in\mathbb{R}$ số nghiệm của hệ phương trình:
	\begin{equation*}
		\left\{\begin{split}
			ax - 3y &= b,\\
			4x - 3y &= 5.
		\end{split}\right.
	\end{equation*}
\end{baitoan}

\begin{baitoan}[\cite{Binh_boi_duong_Toan_9_tap_2}, H4, p. 14]
	Biện luận theo 2 tham số $m,n\in\mathbb{R}$ số nghiệm của hệ phương trình:
	\begin{equation*}
		\left\{\begin{split}
			y &= mx - 4,\\
			y &= -3x + n.
		\end{split}\right.\hspace{5mm}
		\left\{\begin{split}
			mx - y &= 6,\\
			x - \dfrac{1}{3}y &= n.
		\end{split}\right.
	\end{equation*}
\end{baitoan}

\begin{baitoan}[\cite{Binh_boi_duong_Toan_9_tap_2}, H5, p. 14]
	{\rm Đ{\tt/}S?} Nếu sai, sửa cho đúng. (a) 2 hệ phương trình bậc nhất 2 ẩn vô nghiệm thì tương đương với nhau. (b) 2 hệ phương trình bậc nhất 2 ẩn cùng có vô số nghiệm thì tương đương với nhau.
\end{baitoan}

\begin{baitoan}[\cite{Binh_boi_duong_Toan_9_tap_2}, VD1, p. 15]
	Cho 2 hệ phương trình:
	\begin{equation*}
		\left\{\begin{split}
			2x + 2y &= m,\\
			x + y &= 6.
		\end{split}\right.\hspace{5mm}
		\left\{\begin{split}
			x - y &= 2,\\
			mx - 4y &= 12.
		\end{split}\right.
	\end{equation*}
	(a) Chứng minh với $m = 4$ thì 2 hệ tương đương với nhau. (b) Chứng minh với $m = 2$ thì 2 hệ không tương đương với nhau. (c) Tìm tất cả $m\in\mathbb{R}$ để 2 hệ tương đương với nhau.
\end{baitoan}

\begin{baitoan}[\cite{Binh_boi_duong_Toan_9_tap_2}, VD2, p. 15]
	Cho 2 hệ phương trình:
	\begin{equation*}
		\left\{\begin{split}
			2x - y &= 4,\\
			-x + 3y &= 3.
		\end{split}\right.\hspace{5mm}
		\left\{\begin{split}
			mx - y &= 4,\\
			2x + ny &= 16.
		\end{split}\right.
	\end{equation*}
	(a) Tìm nghiệm của hệ 1 bằng cách vẽ đồ thị của 2 đường thẳng trong hệ. (b) Tìm $m,n\in\mathbb{R}$ để 2 hệ tương đương với nhau.
\end{baitoan}

\begin{baitoan}[\cite{Binh_boi_duong_Toan_9_tap_2}, VD3, p. 16]
	\begin{equation*}
		\left\{\begin{split}
			2x &= 4,\\
			-3x + 4y &= -2.
		\end{split}\right.
	\end{equation*}
	(a) Đoán số nghiệm của hệ. (b) Tìm tập nghiệm của hệ bằng phương pháp đồ thị. (c) Vẽ thêm đường thẳng $x + 2y = 4$ trên cùng hệ trục tọa độ. Nhận xét về nghiệm của hệ phương trình:
	\begin{equation*}
		\left\{\begin{split}
			x + 2y &= 4,\\
			-3x + 4y &= -2.
		\end{split}\right.
	\end{equation*}
	Giải hệ này bằng phương pháp thế để kiểm tra.
\end{baitoan}

\begin{baitoan}[\cite{Binh_boi_duong_Toan_9_tap_2}, VD4, p. 17]
	Giải hệ phương trình:
	\begin{equation*}
		\left\{\begin{split}
			x - 2y &= 1,\\
			(m^2 + 2)x - 6y &= 3m,
		\end{split}\right.
	\end{equation*}
	trong các trường hợp: (a) $m = -1$. (b) $m = 0$. (c) $m = 1$. (d) $m\in\mathbb{R}$.
\end{baitoan}

\begin{baitoan}[\cite{Binh_boi_duong_Toan_9_tap_2}, VD5, p. 17]
	Giải hệ phương trình:
	\begin{equation*}
		\left\{\begin{split}
			\sqrt{6}x + \sqrt{2}y &= 2,\\
			\dfrac{x}{\sqrt{2}} - \dfrac{y}{\sqrt{3}} &= -\dfrac{1}{\sqrt{6}},
		\end{split}\right.
	\end{equation*}
	(a) Bằng phương pháp thế. (b) Bằng phương pháp cộng đại số.
\end{baitoan}

\begin{baitoan}[\cite{Binh_boi_duong_Toan_9_tap_2}, VD6, p. 18]
	Cho 2 hệ phương trình:
	\begin{equation*}
		\left\{\begin{split}
			\frac{x}{4} + \frac{y}{3} &= \frac{1}{2},\\
			0.25x + 0.5y &= 1.
		\end{split}\right.\hspace{5mm}
		\left\{\begin{split}
			\sqrt{2}ax + \sqrt{3}by &= 5,\\
			-\sqrt{3}ax + \sqrt{2}by &= 5\sqrt{6}.
		\end{split}\right.
	\end{equation*}
	(a) Giải hệ 1 bằng phương pháp cộng đại số. (b) Biết 2 hệ tương đương với nhau. Tìm $a,b$.
\end{baitoan}

\begin{baitoan}[\cite{Binh_boi_duong_Toan_9_tap_2}, VD7, p. 19]
	Giải hệ phương trình:
	\begin{equation*}
		\left\{\begin{split}
			\frac{2}{2x + 1} - \frac{5}{2y - 1} &= 8,\\
			\frac{3}{2x + 1} - \frac{2}{2y - 1} &= 1.
		\end{split}\right.
	\end{equation*}
\end{baitoan}

\begin{baitoan}[\cite{Binh_boi_duong_Toan_9_tap_2}, VD8, p. 19]
	Giải hệ phương trình:
	\begin{equation*}
		\left\{\begin{split}
			|x + 2| + |y - 1| &= 4,\\
			|x + 2| - y &= 1.
		\end{split}\right.
	\end{equation*}
\end{baitoan}

\begin{baitoan}[\cite{Binh_boi_duong_Toan_9_tap_2}, 2.1., p. 20]
	Không cần vẽ hình, cho biết số nghiệm của hệ phương trình:
	\begin{equation*}
		\left\{\begin{split}
			y &= -5x + 8,\\
			y &= 3x - 2.
		\end{split}\right.\hspace{5mm}
		\left\{\begin{split}
			1.2x + 5.4y &= 15,\\
			1.2x + 5.4y &= -7.
		\end{split}\right.\hspace{5mm}
		\left\{\begin{split}
		5x - y &= 10,\\
		x - \frac{1}{5}y &= 2.
		\end{split}\right.
	\end{equation*}
\end{baitoan}

\begin{baitoan}[\cite{Binh_boi_duong_Toan_9_tap_2}, 2.2., p. 20]
	\begin{equation*}
		\left\{\begin{split}
			2x + y &= 7,\\
			x - 2y &= -4.
		\end{split}\right.
	\end{equation*}
	(a) Tìm tập nghiệm của hệ 1 bằng phương pháp đồ thị. (b) Nghiệm của hệ 1 có phải là nghiệm của phương trình $2.5x - 3y = -4$ hay không? (c) Liệu hệ 1 có tương đương với hệ
	\begin{equation*}
		\left\{\begin{split}
			x - 2y + 4 &= 0,\\
			x - 2y - 4 &= 0.
		\end{split}\right.
	\end{equation*}
\end{baitoan}

\begin{baitoan}[\cite{Binh_boi_duong_Toan_9_tap_2}, 2.3., p. 20]
	Cho 2 hệ phương trình:
	\begin{equation*}
		\left\{\begin{split}
			x - 2y &= 2,\\
			-x + y &= -3.
		\end{split}\right.\hspace{5mm}
		\left\{\begin{split}
			mx - ny &= 8,\\
			2mx + 3ny &= 26.
		\end{split}\right.
	\end{equation*}
	(a) Tìm nghiệm của hệ 1 bằng phương pháp đồ thị. (b) Tìm $m,n\in\mathbb{R}$ để 2 hệ tương đương với nhau.
\end{baitoan}

\begin{baitoan}[\cite{Binh_boi_duong_Toan_9_tap_2}, 2.4., p. 20]
	Giải hệ phương trình bằng phương pháp thế:
	\begin{equation*}
		\left\{\begin{split}
			2x - y &= 10,\\
			4x + 5y &= -8.
		\end{split}\right.\hspace{5mm}
		\left\{\begin{split}
			\frac{x}{3} + \frac{y}{4} &= 4,\\
			x + y &= 14.
		\end{split}\right.
	\end{equation*}
\end{baitoan}

\begin{baitoan}[\cite{Binh_boi_duong_Toan_9_tap_2}, 2.5., p. 20]
	Giải hệ phương trình bằng phương pháp cộng đại số:
	\begin{equation*}
		\left\{\begin{split}
			2x - y &= 8,\\
			3x + 1.5y &= 3.
		\end{split}\right.\hspace{5mm}
		\left\{\begin{split}
			-2x + 3y &= 11,\\
			4x - 6y &= 14.
		\end{split}\right.\hspace{5mm}
		\left\{\begin{split}
			2x - \frac{2y}{\sqrt{2} + 1} &= \sqrt{12},\\
			x - (\sqrt{2} - 1)y &= \sqrt{3}.
		\end{split}\right.
	\end{equation*}
\end{baitoan}

\begin{baitoan}[\cite{Binh_boi_duong_Toan_9_tap_2}, 2.6., p. 20]
	Giải hệ phương trình:
	\begin{equation*}
		\left\{\begin{split}
			\frac{5x + 3}{9} + \frac{4x - y}{3} &= 4,\\
			\frac{2x - 5y}{6} - \frac{2x - 4y}{3} &= 2.
		\end{split}\right.
	\end{equation*}
\end{baitoan}

\begin{baitoan}[\cite{Binh_boi_duong_Toan_9_tap_2}, 2.7., p. 20]
	Tìm $a,b\in\mathbb{R}$ để đồ thị hàm số $y = ax + b$ đi qua 2 điểm $A(3,-4),B(-1,4)$.
\end{baitoan}

\begin{baitoan}[\cite{Binh_boi_duong_Toan_9_tap_2}, 2.8., p. 21]
	Giải hệ phương trình:
	\begin{equation*}
		\left\{\begin{split}
			\frac{10}{x + 3y} + \frac{7}{x - 3y} &= 3,\\
			\frac{15}{x + 3y} - \frac{14}{x - 3y} &= 1.
		\end{split}\right.\hspace{5mm}
		\left\{\begin{split}
			2\sqrt{x - 2} - 5\sqrt{y + 1} &= -4,\\
			\sqrt{x - 2} + 3\sqrt{y + 1} &= 9.
		\end{split}\right.\hspace{5mm}
		\left\{\begin{split}
			(x + 2)^2 - 2(y - 1)^3 &= 25,\\
			3(x + 2)^2 + 5(y - 1)^3 &= -13.
		\end{split}\right.
	\end{equation*}
\end{baitoan}

\begin{baitoan}[\cite{Binh_boi_duong_Toan_9_tap_2}, 2.9., p. 21]
	Giải hệ phương trình:
	\begin{equation*}
		\left\{\begin{split}
			1.5|x - 1| - 2|y + 2| &= -3,\\
			\frac{|x - 1|}{4} + \frac{|y + 2|}{6} &= 1.
		\end{split}\right.\hspace{5mm}
		\left\{\begin{split}
			x - 2y &= 3,\\
			|x - y| + |x + 4y - 5| &= 8.
		\end{split}\right.
	\end{equation*}
\end{baitoan}

\begin{baitoan}[\cite{Binh_boi_duong_Toan_9_tap_2}, 2.10., p. 21]
	Biết 1 đa thức $P(x)\in\mathbb{R}[x]$ bằng đa thức $0$ khi \& chỉ khi tất cả các hệ số của nó bằng $0$. Tìm các giá trị của $a,b\in\mathbb{R}$ để đa thức $P(x) = (\sqrt{a - 1} + 3\sqrt{b - 2} - 9)x + 2\sqrt{a - 1} - 5\sqrt{b - 2} + 4$ bằng đa thức $0$.
\end{baitoan}

\begin{baitoan}[\cite{Binh_boi_duong_Toan_9_tap_2}, 2.11., p. 21]
	\begin{equation*}
		\left\{\begin{split}
			x - my &= 3,\\
			mx - 9y &= 2m - 3.
		\end{split}\right.
	\end{equation*}
	(a) Giải hệ phương trình với $m = 1$. (b) Tìm $m\in\mathbb{R}$ để hệ có nghiệm duy nhất, đồng thời thỏa mãn điều kiện $x > y$.
\end{baitoan}

\begin{dinhly}[B\'ezout]
	Số dư trong phép chia đa thức $f(x)$ cho nhị thức $x - a$ bằng giá trị của đa thức $f(x)$ tại $x = a$, i.e., $f(a)$.
\end{dinhly}

\begin{hequa}
	Đa thức $f(x)\divby x - a$ khi \& chỉ khi $f(a) = 0$.
\end{hequa}
i.e., $f(x)\divby x - a\Leftrightarrow f(a) = 0$, $\forall f\in\mathbb{R}[x]$.

\begin{baitoan}[\cite{Binh_boi_duong_Toan_9_tap_2}, VD1, p. 22]
	Tìm $m,n\in\mathbb{R}$ sao cho đa thức $f(x) = mx^3 + (2m - 1)x^2 - (m + 2n - 3)x + 4n\divby x^2 - 1$.
\end{baitoan}

\begin{baitoan}[\cite{Binh_boi_duong_Toan_9_tap_2}, VD2, p. 22]
	Tìm $m,n\in\mathbb{R}$ sao cho đa thức $f(x) = mx^3 + (m - 3)x^2 - (2m + n - 4)x - (m + 5n - 2)\divby(x - 1)(x + 2)$.
\end{baitoan}

\begin{baitoan}[\cite{Binh_boi_duong_Toan_9_tap_2}, VD, p. 23]
	Giải hệ phương trình:
	\begin{equation*}
		\left\{\begin{split}
			x - 2y + 3z &= 6,\\
			2x + y - z &= 1,\\
			2x - 3y + 4z &= 8.
		\end{split}\right.\hspace{5mm}\left\{\begin{split}
			\frac{2}{x} + \frac{5}{y} - \frac{6}{z} &= 5,\\
			\frac{3}{x} - \frac{2}{y} + \frac{4}{z} &= -8,\\
			\frac{1}{x} - \frac{1}{y} + \frac{2}{z} &= -3.
		\end{split}\right.\hspace{5mm}\left\{\begin{split}
			x + y + z &= 5,\\
			y + z + t &= 9,\\
			z + t + x &= 7,\\
			t + x + y &= 24.
		\end{split}\right.\hspace{5mm}\left\{\begin{split}
			4(x + y) &= 3xy,\\
			12(y + z) &= 5yz,\\
			3(z + x) &= 2zx.
		\end{split}\right.
	\end{equation*}
\end{baitoan}

\begin{baitoan}[\cite{Binh_boi_duong_Toan_9_tap_2}, VD, p. 24]
	Giải hệ phương trình:
	\begin{equation*}
		\left\{\begin{split}
			x^2 - xy + y^2 &= 7,\\
			x^3 + y^3 &= 35.
		\end{split}\right.\hspace{5mm}\left\{\begin{split}
			x + y + z &= 4,\\
			x + 2y + 3z &= 5,\\
			x^2 + y^2 + z^2 &= 14.
		\end{split}\right.\hspace{5mm}\left\{\begin{split}
			x^2 + y^2 &= 2.5xy,\\
			x - y &= 0.25xy.
		\end{split}\right.\hspace{5mm}\left\{\begin{split}
			x^3 + y^3 &= 7,\\
			xy(x + y) &= -2.
		\end{split}\right.
	\end{equation*}
\end{baitoan}

\begin{baitoan}[\cite{Tuyen_Toan_9_old}, VD27, p. 56]
	\begin{equation*}
		\left\{\begin{split}
			ax + by &= 3,\\
			bx + ay &= 3.
		\end{split}\right.
	\end{equation*}
	với $a,b\in\mathbb{N}^\star,a\ne b$. (a) Chứng minh hệ có nghiệm duy nhất. (b) Tìm $(a,b)\in\mathbb{R}^2$ để hệ có nghiệm nguyên dương.
\end{baitoan}

\begin{baitoan}[\cite{Tuyen_Toan_9_old}, 148., p. 56]
	Tìm $m\in\mathbb{R}$ để hệ phương trình sau: (a) Vô nghiệm. (b) Vô số nghiệm.
	\begin{equation*}
		\left\{\begin{split}
			x - my &= m,\\
			mx - 9y &= m + 6.
		\end{split}\right.
	\end{equation*}	
\end{baitoan}

\begin{baitoan}[\cite{Tuyen_Toan_9_old}, 149., p. 57]
	Tìm $m\in\mathbb{R}$ để 2 hệ phương trình tương đương:
	\begin{equation*}
		\left\{\begin{split}
			3x + 5y &= 7,\\
			2x - y &= 6,
		\end{split}\right.\ \&
		\left\{\begin{split}
			3x + 5y &= 7,\\
			x - \frac{1}{2}y &= m.
		\end{split}\right.\hspace{5mm} \left\{\begin{split}
			4x - 3y &= 5,\\
			2x + 5y &= 9,
		\end{split}\right.\ \&
		\left\{\begin{split}
			4x - 3y &= 5,\\
			3x + y &= m.
		\end{split}\right.
	\end{equation*}
\end{baitoan}

\begin{baitoan}[\cite{Tuyen_Toan_9_old}, 150., p. 57, \cite{Dong_23_1001_toan_I}, 171., p. 72]
	\begin{equation*}
		\left\{\begin{split}
			ax + y + z &= 1,\\
			x + ay + z &= a,\\
			x + y + az &= a^2.
		\end{split}\right.
	\end{equation*}
	(a) Chứng minh với $a = -2$ thì hệ này vô nghiệm. (b) Giải \& biện luận hệ phương trình theo tham số $a\in\mathbb{R}$.
\end{baitoan}

\begin{baitoan}[\cite{Tuyen_Toan_9_old}, 151., p. 57]
	Tìm {\rm GTLN, GTNN} của biểu thức $A = x - 2y + 3z$ biết $x,y,z\ge0$ \& thỏa hệ phương trình
	\begin{equation*}
		\left\{\begin{split}
			2x + 4y + 3z &= 8,\\
			3x + y - 3z &= 2.
		\end{split}\right.
	\end{equation*}
\end{baitoan}

\begin{baitoan}[\cite{Tuyen_Toan_9_old}, 152., p. 57, \cite{TLCT_THCS_Toan_9_dai_so}, 11.6, p. 64]
	\begin{equation*}
		\left\{\begin{split}
			ax + by &= c,\\
			bx + cy &= a,\\
			cx + ay &= b,
		\end{split}\right.
	\end{equation*}
	với $a,b,c\in\mathbb{R}^\star$. Biết hệ phương trình này có nghiệm, chứng minh $a^3 + b^3 + c^3 = 3abc$.
\end{baitoan}

\begin{baitoan}[\cite{Tuyen_Toan_9_old}, 153., p. 57]
	Biết mua $3$ bút, $7$ quyển vở, \& $1$ tập giấy hết $14500$ đồng. Nếu mua $4$ bút, $10$ quyển vở, \& $1$ tập giấy thì hết $19500$. Tính số tiền đẻ mua $1$ bút, $1$ quyển vở, \& $1$ tập giấy.
\end{baitoan}
Giải hệ phương trình:

\begin{baitoan}[\cite{Tuyen_Toan_9_old}, VD28, p. 58]
	\begin{equation*}
		\left\{\begin{split}
			\frac{4x + 5y}{xy} &= 2,\\
			20x - 30y + xy &= 0.
		\end{split}\right.
	\end{equation*}
\end{baitoan}

\begin{baitoan}[\cite{Tuyen_Toan_9_old}, VD29, p. 59]
	\begin{equation*}
		\left\{\begin{split}
			2x - y + z &= 12,\\
			3x + 4y - 5z &= -17,\\
			8x - 6y - 3z &= 42.
		\end{split}\right.
	\end{equation*}
\end{baitoan}

\begin{baitoan}[\cite{Tuyen_Toan_9_old}, VD30, p. 60]
	Giải phương trình $\sqrt{1 + \sqrt{x}} + \sqrt{7 - \sqrt{x}} = 4$.
\end{baitoan}
Giải hệ phương trình:

\begin{baitoan}[\cite{Tuyen_Toan_9_old}, 154., p. 61]
	\begin{equation*}
		\left\{\begin{split}
			4\lfloor x\rfloor + y &= 10,\\
			x + 3\lfloor y\rfloor &= 9.
		\end{split}\right.
	\end{equation*}
\end{baitoan}

\begin{baitoan}[\cite{Tuyen_Toan_9_old}, 155., p. 61]
	\begin{equation*}
		\left\{\begin{split}
			x + (m - 1)y &= 2,\\
			(m + 1)x - y &= m + 1.
		\end{split}\right.
	\end{equation*}
	(a) Giải hệ phương trình khi $m = \frac{1}{2}$. (b) Tìm $m\in\mathbb{R}$ để hệ có nghiệm duy nhất $(x,y)$ thỏa mãn điều kiện $x > y$.
\end{baitoan}

\begin{baitoan}[\cite{Tuyen_Toan_9_old}, 156., p. 61]
	Giải \& biện luận hệ phương trình:
	\begin{equation*}
		\left\{\begin{split}
			x - my &= 2,\\
			mx - 4y &= m - 2.
		\end{split}\right.
	\end{equation*}
\end{baitoan}

\begin{baitoan}[\cite{Tuyen_Toan_9_old}, 157., p. 61]
	Giải \& biện luận hệ phương trình:
	\begin{equation*}
		\left\{\begin{split}
			7x - 4y &= 2,\\
			5x - 3y &= 1,\\
			mx + 3y &= m^2 + 6.
		\end{split}\right.
	\end{equation*}
\end{baitoan}

\begin{baitoan}[\cite{Tuyen_Toan_9_old}, 158., p. 61]
	Tìm {\rm GTNN} của biểu thức $A = (6x - 5y - 16)^2 + x^2 + y^2 + 2xy + 2x + 2y + 2$.
\end{baitoan}
Giải hệ phương trình:

\begin{baitoan}[\cite{Tuyen_Toan_9_old}, 159., p. 62]
	\begin{equation*}
		\left\{\begin{split}
			\frac{5(x - 1)}{x + 2y} + \frac{3(y + 1)}{x - 2y} &= 8,\\
			\frac{20(x - 1)}{x + 2y} - \frac{7(y + 1)}{x - 2y} &= -6.
		\end{split}\right.
	\end{equation*}
\end{baitoan}

\begin{baitoan}[\cite{Tuyen_Toan_9_old}, 160., p. 62]
	\begin{equation*}
		\left\{\begin{split}
			5x + 3y &= 31,\\
			\sqrt{\frac{x + 2}{y - 3}} + \sqrt{\frac{y - 3}{x + 2}} &= 2.
		\end{split}\right.
	\end{equation*}
\end{baitoan}

\begin{baitoan}[\cite{Tuyen_Toan_9_old}, 161., p. 62]
	\begin{equation*}
		\left\{\begin{split}
			(2m + 1)x + y &= 2m - 2,\\
			m^2x - y &= m^2 - 3m.
		\end{split}\right.
	\end{equation*}
	Tìm $m\in\mathbb{Z},m\ne-1$ để hệ phương trình có nghiệm nguyên.
\end{baitoan}

\begin{baitoan}[\cite{Tuyen_Toan_9_old}, 162., p. 62]
	Cho đường thẳng $(d):2(m + 2)x - (3m - 1)y + 5m - 11 = 0$. Chứng minh khi $m$ thay đổi thì $(d)$ luôn đi qua 1 điểm cố định. Tìm tọa độ điểm đó.
\end{baitoan}

\begin{baitoan}[\cite{Tuyen_Toan_9_old}, 163., p. 62]
	Cho đường thẳng $(d):(m^2 + 2m + 3)x + (6m^2 + 3m - 2)y + 2m^2 - 5m + 1 = 0$. Khi $m$ thay đổi, $(d)$ có đi qua 1 điểm cố định nào không?
\end{baitoan}
Giải hệ phương trình:

\begin{baitoan}[\cite{Tuyen_Toan_9_old}, 164., p. 62]
	\begin{equation*}
		\left\{\begin{split}
			x - |y - 5| &= 8,\\
			|x + 1| + 3|y - 5| &= 21.
		\end{split}\right.
	\end{equation*}
\end{baitoan}

\begin{baitoan}[\cite{Tuyen_Toan_9_old}, 165., p. 62]
	\begin{equation*}
		\left\{\begin{split}
			x - 4y &= 5,\\
			2|x - 2y| + 3|x + y - 1| &= 7.
		\end{split}\right.
	\end{equation*}
\end{baitoan}

\begin{baitoan}[\cite{Tuyen_Toan_9_old}, 166., p. 62]
	\begin{equation*}
		\left\{\begin{split}
			6xy &= 5(x + y),\\
			3yz &= 2(y + z),\\
			7zx &= 10(z + x).
		\end{split}\right.\hspace{5mm}\left\{\begin{split}
		\frac{xy}{x + y} &= \frac{6}{5},\\
		\frac{yz}{y + z} &= \frac{4}{3},\\
		\frac{zx}{z + x} &= \frac{12}{7}.
		\end{split}\right.
	\end{equation*}
\end{baitoan}

\begin{baitoan}[\cite{Tuyen_Toan_9_old}, 167., p. 63]
	\begin{equation*}
		\left\{\begin{split}
			xy - x - y &= 5,\\
			yz - y - z &= 11,\\
			zx - z - x &= 7.
		\end{split}\right.\hspace{5mm}\left\{\begin{split}
			x(y - z) &= -4,\\
			y(z - x) &= 9,\\
			z(x + y) &= 1.
		\end{split}\right.
	\end{equation*}
\end{baitoan}

\begin{baitoan}[\cite{Tuyen_Toan_9_old}, 168., p. 63]
	\begin{equation*}
		\left\{\begin{split}
			x^2 + 2y + 1 &= 0,\\
			y^2 - 2x + 1 &= 0.
		\end{split}\right.\hspace{5mm}\left\{\begin{split}
			3x^2 + xz - yz + y^2 &= 2,\\
			y^2 + xy - yz + z^2 &= 0,\\
			x^2 - xy - xz - z^2 &= 2.
		\end{split}\right.
	\end{equation*}
\end{baitoan}

\begin{baitoan}[\cite{Tuyen_Toan_9_old}, 169., p. 63]
	Giải phương trình: (a) $\sqrt{2x + 3} + \sqrt{10 - 2x} = 5$. (b) $\sqrt{x - 1} - \sqrt[3]{2 - x} = 5$.
\end{baitoan}

\begin{baitoan}[\cite{Tuyen_Toan_9_old}, 170., p. 63]
	Tìm hàm số $f(x)$ biết hàm số này xác định $\forall x\in\mathbb{R}$ \& $f(x) - 2f(-x) = x + 1$.
\end{baitoan}

\begin{baitoan}[\cite{Dong_23_1001_toan_I}, 141., p. 60]
	\begin{equation*}
		\left\{\begin{split}
			2x + my &= 1,\\
			mx + 2y &= 1.
		\end{split}\right.
	\end{equation*}
	(a) Giải \& biện luận theo tham số $m\in\mathbb{R}$. (b) Tìm $m\in\mathbb{Z}$ để cho hệ có nghiệm duy nhất $(x,y)\in\mathbb{Z}^2$.
\end{baitoan}

\begin{baitoan}[\cite{Dong_23_1001_toan_I}, 142., p. 60]
	\begin{equation*}
		\left\{\begin{split}
			mx + 4y &= 10 - m,\\
			x + my &= 4.
		\end{split}\right.
	\end{equation*}
	(a) Giải \& biện luận theo tham số $m\in\mathbb{R}$. (b) Tìm $m\in\mathbb{Z}$ để cho hệ có nghiệm $(x,y)\in\mathbb{N}^\star\times\mathbb{N}^\star$.
\end{baitoan}

\begin{baitoan}[\cite{Dong_23_1001_toan_I}, 143., p. 61]
	\begin{equation*}
		\left\{\begin{split}
			(m - 1)x - my &= 3m - 1,\\
			2x - y &= m + 5.
		\end{split}\right.
	\end{equation*}
	Tìm $m\in\mathbb{R}$ để hệ có nghiệm duy nhất $(x,y)\in\mathbb{R}^2$ mà $A = x^2 + y^2$ đạt {\rm GTNN}.
\end{baitoan}

\begin{baitoan}[\cite{Dong_23_1001_toan_I}, 144., p. 61]
	\begin{equation*}
		\left\{\begin{split}
			(m + 1)x + my &= 2m - 1,\\
			mx - y &= m^2 - 2.
		\end{split}\right.
	\end{equation*}
	Tìm $m\in\mathbb{R}$ để hệ có nghiệm $(x,y)\in\mathbb{R}^2$ mà $P = xy$ đạt {\rm GTLN}.
\end{baitoan}

\begin{baitoan}[\cite{Dong_23_1001_toan_I}, 145., p. 62]
	\begin{equation*}
		\left\{\begin{split}
			mx + y &= 2m,\\
			x + my &= m + 1.
		\end{split}\right.
	\end{equation*}
	(a) Giải hệ khi $m = -1$. (b) Tìm $m\in\mathbb{R}$ để hệ có vô số nghiệm, trong đó có nghiệm $(x,y) = (1,1)$.
\end{baitoan}

\begin{baitoan}[\cite{Dong_23_1001_toan_I}, 146., p. 62]
	Giải \& biện luận hệ phương trình theo tham số $m\in\mathbb{R}$:
	\begin{equation*}
		\left\{\begin{split}
			mx + 2y &= m + 1,\\
			2x + my &= 3.
		\end{split}\right.
	\end{equation*}
\end{baitoan}

\begin{baitoan}[\cite{Dong_23_1001_toan_I}, 147., p. 62]
	\begin{equation*}
		\left\{\begin{split}
			x + my &= 1,\\
			mx - 3my &= 2m + 3.
		\end{split}\right.
	\end{equation*}
	(a) Giải hệ khi $m = -3$. (b) Giải \& biện luận hệ phương trình theo tham số $m\in\mathbb{R}$.
\end{baitoan}

\begin{baitoan}[\cite{Dong_23_1001_toan_I}, 148., p. 63]
	\begin{equation*}
		\left\{\begin{split}
			x + my &= 2,\\
			mx - 2y &= 1.
		\end{split}\right.
	\end{equation*}
	(a) Giải hệ khi $m = 2$. (b) Tìm $m\in\mathbb{Z}$ để hệ có nghiệm duy nhất $(x,y)\in\mathbb{R}^2$ mà $x > 0,y < 0$. (b) Tìm $m\in\mathbb{Z}$ để hệ có nghiệm duy nhất $(x,y)\in\mathbb{Z}^2$.
\end{baitoan}

\begin{baitoan}[\cite{Dong_23_1001_toan_I}, 149., p. 64]
	\begin{equation*}
		\left\{\begin{split}
			2x + y &= m,\\
			3x - 2y &= 35.
		\end{split}\right.
	\end{equation*}
	Tìm $m\in\mathbb{Z}$ để hệ có nghiệm duy nhất $(x,y)\in\mathbb{R}^2$ mà $x > 0,y < 0$.
\end{baitoan}

\begin{baitoan}[\cite{Dong_23_1001_toan_I}, 150., p. 64]
	\begin{equation*}
		\left\{\begin{split}
			mx - y &= 2,\\
			3x + my &= 5.
		\end{split}\right.
	\end{equation*}
	(a) Giải \& biện luận hệ. (b) Tìm điều của $m\in\mathbb{R}$ để hệ có nghiệm duy nhất $(x,y)\in\mathbb{R}^2$ thỏa mãn hệ thức $x + y = 1 - \dfrac{m^2}{m^2 + 3}$.
\end{baitoan}

\begin{baitoan}[\cite{Dong_23_1001_toan_I}, 151., p. 64]
	\begin{equation*}
		\left\{\begin{split}
			mx + 2my &= m + 1,\\
			x + (m + 1)y &= 2.
		\end{split}\right.
	\end{equation*}
	(a) Chứng minh nếu hệ có nghiệm duy nhất $(x,y)\in\mathbb{R}^2$ thì điểm $M(x,y)$ luôn thuộc 1 đường thẳng cố định khi $m$ thay đổi. (b) Tìm $m\in\mathbb{R}$ để M thuộc góc phần tư thứ nhất. (c) Tìm $m\in\mathbb{R}$ để M thuộc đường tròn có tâm là gốc tọa độ \& bán kính bằng $\sqrt{5}$.
\end{baitoan}

\begin{baitoan}[\cite{Dong_23_1001_toan_I}, 152., p. 65]
	Tìm $m\in\mathbb{R}$ để hệ phương trình
	\begin{equation*}
		\left\{\begin{split}
			mx + 4y &= m + 2,\\
			x + my &= m,
		\end{split}\right.
	\end{equation*}
	có nghiệm duy nhất $(x,y)\in\mathbb{Z}^2$.
\end{baitoan}

\begin{baitoan}[\cite{Dong_23_1001_toan_I}, 153., p. 65]
	\begin{equation*}
		\left\{\begin{split}
			2x + my &= 1,\\
			mx + 2y &= 1.
		\end{split}\right.
	\end{equation*}
	(a) Giải \& biện luận theo tham số $m\in\mathbb{R}$. (b) Tìm $m\in\mathbb{Z}$ để hệ có nghiệm duy nhất $(x,y)\in\mathbb{Z}^2$. (c) Chứng minh khi hệ có nghiệm duy nhất $(x,y)\in\mathbb{R}^2$, điểm $M(x,y)$ luôn chạy trên 1 đường thẳng cố định. (d) Tìm $m\in\mathbb{R}$ để điểm M thuộc đường tròn có tâm là gốc tọa độ \& bán kính bằng $\dfrac{\sqrt{2}}{2}$.
\end{baitoan}

\begin{baitoan}[\cite{Dong_23_1001_toan_I}, 154., p. 66]
	Giải \& biện luận hệ phương trình:
	\begin{equation*}
		\left\{\begin{split}
			2m^2x + 3(m - 1)y &= 3,\\
			m(x + y) - 2y &= 2.
		\end{split}\right.\hspace{5mm}\left\{\begin{split}
			x - 2y &= m + 1,\\
			x + y &= 2 - m.
		\end{split}\right.\hspace{5mm}\left\{\begin{split}
			x - my &= 1,\\
			x - y &= m.
		\end{split}\right.		
	\end{equation*}
\end{baitoan}

\begin{baitoan}[\cite{Dong_23_1001_toan_I}, 155., p. 66]
	\begin{equation*}
		\left\{\begin{split}
			-2mx + y &= 5,\\
			mx + 3y &= 1.
		\end{split}\right.
	\end{equation*}
	(a) Giải hệ phương trình lúc $m = 1$. (b) Giải \& biện luận theo tham số $m\in\mathbb{R}$.
\end{baitoan}

\begin{baitoan}[\cite{Dong_23_1001_toan_I}, 156., p. 67]
	\begin{equation*}
		\left\{\begin{split}
			mx - y &= 1,\\
			-x + y &= -m.
		\end{split}\right.
	\end{equation*}
	(a) Chứng minh khi $m = 1$, hệ phương trình có vô số nghiệm. (b) Giải hệ khi $m\ne1$.
\end{baitoan}

\begin{baitoan}[\cite{Dong_23_1001_toan_I}, 157., p. 67]
	\begin{equation*}
		\left\{\begin{split}
			\frac{x - y}{7} + \frac{2x + y}{17} &= 7,\\
			\frac{4x + y}{5} + \frac{y - 7}{19} &= 15.
		\end{split}\right.
	\end{equation*}
\end{baitoan}

\begin{baitoan}[\cite{Dong_23_1001_toan_I}, 158., p. 67]
	\begin{equation*}
		\left\{\begin{split}
			x + y + z &= 1,\\
			x + 2y + 4z &= 8,\\
			x + 3y + 9z &= 27.
		\end{split}\right.
	\end{equation*}
\end{baitoan}

\begin{baitoan}[\cite{Dong_23_1001_toan_I}, 159., p. 68]
	\begin{equation*}
		\left\{\begin{split}
			x + 2y + 3z &= 11,\\
			2x + 3y + z &= -2,\\
			3x + y + 2z &= 3.
		\end{split}\right.
	\end{equation*}
\end{baitoan}

\begin{baitoan}[\cite{Dong_23_1001_toan_I}, 160., p. 68]
	\begin{equation*}
		\left\{\begin{split}
			\frac{3}{2x + y} + z &= 2,\\
			2y - 3z &= 4,\\
			\frac{2}{2x + y} - y &= \frac{3}{2}.
		\end{split}\right.
	\end{equation*}
\end{baitoan}

\begin{baitoan}[\cite{Dong_23_1001_toan_I}, 161., p. 68]
	Tìm nghiệm nguyên dương của hệ phương trình:
	\begin{equation*}
		\left\{\begin{split}
			\frac{x + y + z}{2} &= 50,\\
			5x + 3y + \frac{z}{3} &= 100.
		\end{split}\right.
	\end{equation*}
\end{baitoan}

\begin{baitoan}[\cite{Dong_23_1001_toan_I}, 162., p. 69]
	Tìm nghiệm nguyên, nguyên dương của phương trình $5x + 7y = 112$.
\end{baitoan}

\begin{baitoan}[\cite{Dong_23_1001_toan_I}, 163., p. 69]
	Tìm nghiệm nguyên dương nhỏ nhất của phương trình: (a) $16x - 25y = 1$. (b) $41x - 37y = 187$.
\end{baitoan}

\begin{baitoan}[\cite{Dong_23_1001_toan_I}, 164., p. 70]
	Tìm nghiệm nguyên dương nhỏ nhất của phương trình:
	\begin{equation*}
		\left\{\begin{split}
			x &= 5y + 3,\\
			x &= 11z + 7.
		\end{split}\right.\hspace{5mm}\left\{\begin{split}
			x + 2y + 3z &= 20,\\
			3x + 5y + 4z &= 37.
		\end{split}\right.
	\end{equation*}
\end{baitoan}

\begin{baitoan}[\cite{Dong_23_1001_toan_I}, 165., p. 70]
	Cần đặt 1 ống nước dài {\rm21 m} bằng 2 loại ống: ống dài {\rm2 m} \& ống dài {\rm3 m}. Mỗi loại cần mấy ống?
\end{baitoan}

\begin{baitoan}[\cite{Dong_23_1001_toan_I}, 166., p. 71]
	1 trường Phổ thông Trung học dùng $100000$ đồng để mua 1 số thiệp hoa làm tặng phẩm cho các học sinh giỏi. Trong số thiệp này, loại $2000$ đồng{\tt/}cái ít hơn $10$ lần loại $1000$ đồng{\tt/}cái, số thiệp hoa còn lại là loại $5000$ đồng{\tt/}cái . Tính số cái mỗi loại thiệp nhà trường mua.
\end{baitoan}

\begin{baitoan}[\cite{Dong_23_1001_toan_I}, 167., p. 71]
	\begin{equation*}
		\left\{\begin{split}
			x + y + z &= 6,\\
			2x - 3y - 5z &= -19,\\
			4x + 9y + 25z &= 97.
		\end{split}\right.
	\end{equation*}
\end{baitoan}

\begin{baitoan}[\cite{Dong_23_1001_toan_I}, 168., p. 72]
	\begin{equation*}
		\left\{\begin{split}
			x + y + z &= 6,\\
			x + 2y + 2z &= 3,\\
			x + 3y + 3z &= 4.
		\end{split}\right.
	\end{equation*}
\end{baitoan}

\begin{baitoan}[\cite{Dong_23_1001_toan_I}, 169., p. 72]
	\begin{equation*}
		\left\{\begin{split}
			x + 2y + 3z &= 0,\\
			x - y + 5z &= 4,\\
			x + 8y - z &= 6.
		\end{split}\right.
	\end{equation*}
\end{baitoan}

\begin{baitoan}[\cite{Dong_23_1001_toan_I}, 170., p. 72]
	\begin{equation*}
		\left\{\begin{split}
			x + y + z + t &= 14\\
			x + y - z - t &= -4,\\
			x - y - z + t &= 0,\\
			x - y + z - t &= -2.
		\end{split}\right.
	\end{equation*}
\end{baitoan}

\begin{baitoan}[\cite{Dong_23_1001_toan_I}, 172., p. 73]
	\begin{equation*}
		\left\{\begin{split}
			x + ay + a^2z &= a^2,\\
			x + by + b^2z &= b^2,\\
			x + cy + c^2z &= c^2.
		\end{split}\right.
	\end{equation*}
\end{baitoan}

\begin{baitoan}[\cite{Dong_23_1001_toan_I}, 173., p. 6]
	\begin{equation*}
		\left\{\begin{split}
			x_1 + x_2 + x_3 + \cdots + x_{2000} &= 1,\\
			x_1 + x_3 + x_4 + \cdots + x_{2000} &= 2,\\
			x_1 + x_2 + x_4 + \cdots + x_{2000} &= 3,\\
			&\ldots,\\
			x_1 + x_2 + \cdots + x_{1999} &= 2000.
		\end{split}\right.
	\end{equation*}
\end{baitoan}

\begin{baitoan}[\cite{Binh_Toan_9_tap_2}, VD68, p. 9]
	Cho hệ phương trình với tham số $a$:
	\begin{equation*}
		\left\{\begin{split}
			(a + 1)x - y &= a + 1,\\
			x + (a - 1)y &= 2.
		\end{split}\right.
	\end{equation*}
	(a) Giải hệ phương trình với $a = 2$. (b) Giải \& biện luận hệ phương trình. (c) Tìm các giá trị nguyên của $a$ để hệ phương trình có nghiệm nguyên. (d) Tìm các giá trị nguyên của $a$ để nghiệm của hệ phương trình thỏa mãn điều kiện $x + y$ nhỏ nhất.
\end{baitoan}

\begin{baitoan}[\cite{Binh_Toan_9_tap_2}, VD69, p. 10]
	Tìm $a,b,c\in\mathbb{Z}$ thỏa mãn cả 2 phương trình $2a + 3b = 6,3a + 4c = 1$.
\end{baitoan}

\begin{baitoan}[\cite{Binh_Toan_9_tap_2}, VD70, p. 10]
	Cho 2 đường thẳng: $d:2x - 3y = 4,d':3x + 5y = 2$. Tìm trên trục $Ox$ điểm có hoành độ là số nguyên dương nhỏ nhất, sao cho nếu qua điểm đó ta dựng đường vuông góc với $Ox$ thì đường vuông góc ấy cắt 2 đường thẳng $d,d'$ tại 2 điểm có tọa độ là các số nguyên.
\end{baitoan}

\begin{baitoan}[\cite{Binh_Toan_9_tap_2}, VD71, p. 11]
	Giải hệ phương trình với 3 ẩn $x,y,z$ \& các tham số $a,b,c$ khác nhau đôi một:
	\begin{equation*}
		\left\{\begin{split}
			a^2x + ay + z &= 5,\\
			b^2x + by + z &= 5,\\
			c^2x + cy + z &= 5.
		\end{split}\right.
	\end{equation*}
\end{baitoan}
Giải hệ phương trình:

\begin{baitoan}[\cite{Binh_Toan_9_tap_2}, 207., p. 12]
	\begin{equation*}
		\left\{\begin{split}
			(x + 3)(y - 5) &= xy,\\
			(x - 2)(y + 5) &= xy,
		\end{split}\right.\hspace{1cm} \left\{\begin{split}
			\frac{1}{x}	+ \frac{1}{y} &= \frac{3}{4},\\
			\frac{1}{6x} + \frac{1}{5y} &= \frac{2}{15}.
		\end{split}\right. 
	\end{equation*}
\end{baitoan}

\begin{baitoan}[\cite{Binh_Toan_9_tap_2}, 208., p. 12]
	\begin{equation*}
		\left\{\begin{split}
			\frac{x}{y} - \frac{x}{y + 12} &= 1,\\
			\frac{x}{y - 12} - \frac{x}{y} &= 2,
		\end{split}\right.\hspace{1cm} \left\{\begin{split}
			4(x + y) &= 5(x - y),\\
			\frac{40}{x + y} + \frac{40}{x - y} &= 9.
		\end{split}\right. 
	\end{equation*}
\end{baitoan}

\begin{baitoan}[\cite{Binh_Toan_9_tap_2}, 209., p. 12]
	\begin{equation*}
		\left\{\begin{split}
			|x - 2| + 2|y - 1| &= 9,\\
			x + |y - 1| &= -1,
		\end{split}\right.\hspace{1cm} \left\{\begin{split}
			x + y + |x| &= 25,\\
			x - y + |y| &= 30.
		\end{split}\right. 
	\end{equation*}
\end{baitoan}

\begin{baitoan}[\cite{Binh_Toan_9_tap_2}, 210., p. 12]
	Tìm các giá trị của $a\in\mathbb{R}$ để 2 hệ phương trình tương đương:
	\begin{equation*}
		\left\{\begin{split}
			2x + 3y &= 8,\\
			3x - y &= 1,
		\end{split}\right.\hspace{1cm} \left\{\begin{split}
			ax - 3y &= -2,\\
			x + y &= 3.
		\end{split}\right. 
	\end{equation*}
\end{baitoan}

\begin{baitoan}[\cite{Binh_Toan_9_tap_2}, 211., p. 12]
	Tìm các giá trị của $m\in\mathbb{R}$ để nghiệm của hệ phương trình sau là 2 số dương:
	\begin{equation*}
		\left\{\begin{split}
			x - y &= 2,\\
			mx + y &= 3.
		\end{split}\right.
	\end{equation*}
\end{baitoan}

\begin{baitoan}[\cite{Binh_Toan_9_tap_2}, 212., p. 12]
	Chứng minh tam giác tạo bởi 3 đường thẳng $y = 3x - 2,y = -\dfrac{1}{3}x + \frac{4}{3},y = -2x + 8$ là tam giác vuông cân.
\end{baitoan}

\begin{baitoan}[\cite{Binh_Toan_9_tap_2}, 213., p. 13]
	Tìm các giá trị của $m\in\mathbb{R}$ để hệ phương trình sau vô nghiệm, vô số nghiệm:
	\begin{equation*}
		\left\{\begin{split}
			2(m + 1)x + (m + 2)y &= m - 3,\\
			(m + 1)x + my &= 3m + 7.
		\end{split}\right.
	\end{equation*}
\end{baitoan}

\begin{baitoan}[\cite{Binh_Toan_9_tap_2}, 214., p. 13]
	Cho hệ phương trình với tham số $m$:
	\begin{equation*}
		\left\{\begin{split}
			mx + 2y &= 1,\\
			3x + (m + 1)y &= -1.
		\end{split}\right.
	\end{equation*}
	(a) Giải hệ phương trình với $m = 3$. (b) Giải \& biện luận hệ phương trình theo $m$. (c) Tìm các giá trị nguyên của $m$ để nghiệm của hệ phương trình là các số nguyên.
\end{baitoan}

\begin{baitoan}[\cite{Binh_Toan_9_tap_2}, 215., p. 13]
	Cho hệ phương trình với tham số $m$:
	\begin{equation*}
		\left\{\begin{split}
			(m - 1)x + y &= 3m - 4,\\
			x + (m - 1)y &= m.
		\end{split}\right.
	\end{equation*}
	(a) Giải \& biện luận hệ phương trình theo $m$. (b) Tìm các giá trị nguyên của $m$ để nghiệm của hệ phương trình là các số nguyên. (c) Tìm các giá trị của $m$ để hệ phương trình có nghiệm dương duy nhất.
\end{baitoan}

\begin{baitoan}[\cite{Binh_Toan_9_tap_2}, 216., p. 13]
	Cho hệ phương trình với tham số $m$:
	\begin{equation*}
		\left\{\begin{split}
			x + my &= m + 1,\\
			mx + y &= 3m - 1.
		\end{split}\right.
	\end{equation*}
	(a) Giải \& biện luận hệ phương trình theo $m$. (b) Trong trường hợp hệ có nghiệm duy nhất, tìm các giá trị của $m$ để tích $xy$ nhỏ nhất.
\end{baitoan}

\begin{baitoan}[\cite{Binh_Toan_9_tap_2}, 217., p. 13]
	Các số không âm $x,y,z$ thỏa mãn hệ phương trình:
	\begin{equation*}
		\left\{\begin{split}
			4x - 4y + 2z &= 1,\\
			8x + 4y + z &= 8.
		\end{split}\right.
	\end{equation*}
	(a) Biểu thị $x,y$ theo $z$. (b) Tìm {\rm GTNN, GTLN} của biểu thức $A = x + y - z$.
\end{baitoan}

\begin{baitoan}[\cite{Binh_Toan_9_tap_2}, 218., p. 13]
	Tìm $a,b,c\in\mathbb{Z}$ thỏa mãn hệ phương trình:
	\begin{equation*}
		\left\{\begin{split}
			2a + 3b &= 5,\\
			3a - 4c &= 6.
		\end{split}\right.
	\end{equation*}
\end{baitoan}

\begin{baitoan}[\cite{Binh_Toan_9_tap_2}, 219., p. 14]
	Tìm trên trục tung các điểm có tung độ là số nguyên, sao cho nếu qua điểm đó ta dựng đường vuông góc với trục tung thì đường vuông góc ấy cắt 2 đường thẳng: $d:x + 2y = 6,d':2x - 3y = 4$ tại các điểm có tọa độ là các số nguyên.
\end{baitoan}

\begin{baitoan}[\cite{Binh_Toan_9_tap_2}, 220., p. 14]
	Tìm trên trục hoành các điểm có hoành độ là số nguyên sao cho nếu qua điểm đó ta dựng đường thẳng vuông góc với trục hoành thì đường vuông góc ấy cắt cả 3 đường thẳng sau tại các điểm có tọa độ là các số nguyên: $d_1:x - 2y = 3,d_2:x - 3y = 2,d_3:x - 5y = -7$.
\end{baitoan}
Giải hệ phương trình ẩn $x,y,z$:

\begin{baitoan}[\cite{Binh_Toan_9_tap_2}, 221., p. 14]
	\begin{equation*}
		\left\{\begin{split}
			x + y + z &= 11,\\
			2x - y + z &= 5,\\
			3x + 2y + z &= 14,
		\end{split}\right.\hspace{1cm}\left\{\begin{split}
			x + y + z + t &= 4,\\
			x + y - z - t &= 8,\\
			x - y + z - t &= 12,\\
			x - y - z + y &= 16.
		\end{split}\right.
	\end{equation*}
\end{baitoan}

\begin{baitoan}[\cite{Binh_Toan_9_tap_2}, 222., p. 14]
	\begin{equation*}
		\left\{\begin{split}
			x + y + z &= 12,\\
			ax + 5y + 4z &= 46,\\
			5x + ay + 3z &= 38,
		\end{split}\right.\hspace{1cm}\left\{\begin{split}
			ax + y + z &= a^2,\\
			x + ay + z &= 3a,\\
			x + y + az &= 2.
		\end{split}\right.
	\end{equation*}
\end{baitoan}

\begin{baitoan}[\cite{Binh_Toan_9_tap_2}, 223., p. 14]
	$a,b,c\in\mathbb{R}$ là tham số, $a + b + c\ne0$.
	\begin{equation*}
		\left\{\begin{split}
			(a + b)(x + y) - cz &= a - b,\\
			(b + c)(y + z) - ax &= b - c,\\
			(c + a)(z + x) - by &= c - a.
		\end{split}\right.
	\end{equation*}
\end{baitoan}

\begin{baitoan}[\cite{Binh_Toan_9_tap_2}, 224., p. 14]
	Giải hệ phương trình với 3 tham số $a,b,c\in\mathbb{R}$ đôi một khác nhau, $a + b + c\ne0$:
	\begin{equation*}
		\left\{\begin{split}
			ax + by + cz &= 0,\\
			bx + cy + az &= 0,\\
			cx + ay + bz &= 0,
		\end{split}\right.\hspace{1cm}\left\{\begin{split}
			ax + by + cz &= a + b + c,\\
			bx + cy + az &= a + b + c,\\
			cx + ay + bz &= a + b + c.
		\end{split}\right.
	\end{equation*}
\end{baitoan}

\begin{baitoan}[\cite{Binh_Toan_9_tap_2}, 225., p. 14]
	\begin{equation*}
		\left\{\begin{split}
			x^2 + xy + xz &= 2,\\
			y^2 + yz + xy &= 3,\\
			z^2 + xz + yz &= 4.
		\end{split}\right.
	\end{equation*}
\end{baitoan}

\begin{baitoan}[\cite{TLCT_THCS_Toan_9_dai_so}, VD11.1, p. 58]
	\begin{equation*}
		\left\{\begin{split}
			\frac{4}{x + y - 1} - \frac{5}{2x - y + 3} &= -\frac{5}{2},\\
			\frac{3}{x + y - 1} + \frac{1}{2x - y + 3} &= -\frac{7}{5}.
		\end{split}\right.
	\end{equation*}
\end{baitoan}

\begin{baitoan}[\cite{TLCT_THCS_Toan_9_dai_so}, VD11.2, p. 59]
	\begin{equation*}
		\left\{\begin{split}
			\sqrt{\frac{x + y}{2}} + \sqrt{\frac{x - y}{3}} &= 14,\\
			\sqrt{\frac{x + y}{8}} - \sqrt{\frac{x - y}{12}} &= 3.
		\end{split}\right.
	\end{equation*}
\end{baitoan}

\begin{baitoan}[\cite{TLCT_THCS_Toan_9_dai_so}, VD11.3, p. 59]
	\begin{equation*}
		\left\{\begin{split}
			mx + y &= 2m - 1,\\
			(2m + 1)x + 7y &= m + 3.
		\end{split}\right.
	\end{equation*}
	(a) Giải \& biện luận hệ phương trình theo tham số $m\in\mathbb{R}$. (b) Khi hệ có nghiệm $(x_0,y_0)\in\mathbb{R}^2$, xác định hệ thức liên hệ giữa $x_0,y_0$ không chứa $m$.
\end{baitoan}

\begin{baitoan}[\cite{TLCT_THCS_Toan_9_dai_so}, VD11.4, p. 60]
	\begin{equation*}
		\left\{\begin{split}
			mx + y &= 2,\\
			-3mx + my &= m - 3.
		\end{split}\right.
	\end{equation*}
	(a) Giải \& biện luận hệ phương trình theo tham số $m\in\mathbb{R}$. (b) Tìm $m\in\mathbb{R}$ để hệ có nghiệm $(x,y)\in\mathbb{Z}^2$.
\end{baitoan}

\begin{baitoan}[\cite{TLCT_THCS_Toan_9_dai_so}, VD11.5, p. 61]
	\begin{equation*}
		\left\{\begin{split}
			x + my &= 2,\\
			mx - 3my &= 3m + 3.
		\end{split}\right.
	\end{equation*}
	Tìm $m\in\mathbb{R}$ để hệ phương trình có đúng 1 nghiệm $(x,y)$ thỏa mãn $y = 8x^2$.
\end{baitoan}

\begin{baitoan}[\cite{TLCT_THCS_Toan_9_dai_so}, VD11.6, p. 62]
	\begin{equation*}
		\left\{\begin{split}
			x + 2y &= 1,\\
			2x - my &= 4.
		\end{split}\right.
	\end{equation*}
	(a) Tìm $m\in\mathbb{R}$ để hệ phương trình có nghiệm. (b) Tìm {\rm GTNN} của biểu thức $A = (x + 2y - 1)^2 + (2x - my - 4)^2$ với $m\ne-4$ \& với $m = -4$.
\end{baitoan}

\begin{baitoan}[\cite{TLCT_THCS_Toan_9_dai_so}, VD11.7, p. 62]
	Có $50$ con gà \& chó. Tính số gà \& chó để có $140$ chân.
\end{baitoan}

\begin{baitoan}[\cite{TLCT_THCS_Toan_9_dai_so}, VD11.8, p. 63]
	Tất cả có $120$ bó cỏ để làm thức ăn cho $30$ con gồm trâu \& bò trong 1 tuần. Biết trong 1 tuần mỗi con trâu ăn hết $5$ bó cỏ, còn mỗi con bò ăn hết $3$ bó cỏ. Tính số trâu \& số bò.
\end{baitoan}

\begin{baitoan}[\cite{TLCT_THCS_Toan_9_dai_so}, 11.1., p. 63]
	Giải \& biện luận hệ phương trình:
	\begin{equation*}
		\left\{\begin{split}
			mx + y &= 2m,\\
			x + my &= m + 1.
		\end{split}\right.
	\end{equation*}
\end{baitoan}

\begin{baitoan}[\cite{TLCT_THCS_Toan_9_dai_so}, 11.2., p. 63]
	Giải \& biện luận hệ phương trình:
	\begin{equation*}
		\left\{\begin{split}
			ax + by &= a^2 + b^2,\\
			bx + ay &= a^2 - b^2.
		\end{split}\right.
	\end{equation*}
\end{baitoan}

\begin{baitoan}[\cite{TLCT_THCS_Toan_9_dai_so}, 11.3., p. 63]
	\begin{equation*}
		\left\{\begin{split}
			mx + (2 - m)y &= -1,\\
			(m - 1)x - my &= 2.
		\end{split}\right.
	\end{equation*}
	(a) Tìm $m\in\mathbb{R}$ để hệ phương trình có nghiệm. (b) Giả sử $(x,y)\in\mathbb{R}^2$ là 1 nghiệm của hệ. Tìm 1 hệ thức của $x,y$ không chứa $m$.
\end{baitoan}

\begin{baitoan}[\cite{TLCT_THCS_Toan_9_dai_so}, 11.4., p. 63]
	\begin{equation*}
		\left\{\begin{split}
			3x + 2y &= -1,\\
			12x - my &= 2.
		\end{split}\right.
	\end{equation*}
	(a) Tìm $m\in\mathbb{R}$ để hệ phương trình có nghiệm. (b) Tìm {\rm GTNN} của biểu thức $A = (3x + 2y + 1)^2 + (12x - my - 2)^2$.
\end{baitoan}

\begin{baitoan}[\cite{TLCT_THCS_Toan_9_dai_so}, 11.5., p. 63]
	Giải \& biện luận hệ phương trình:
	\begin{equation*}
		\left\{\begin{split}
			ax + by &= a + b,\\
			bx - ay &= -2ab.
		\end{split}\right.
	\end{equation*}
\end{baitoan}

\begin{baitoan}[\cite{TLCT_THCS_Toan_9_dai_so}, 11.7., p. 64]
	\begin{equation*}
		\left\{\begin{split}
			x + y + z &= 3,\\
			2x + 3y + 4z &= 18,\\
			x - y + z &= 2.
		\end{split}\right.
	\end{equation*}
\end{baitoan}

\begin{baitoan}[\cite{TLCT_THCS_Toan_9_dai_so}, 11.8., p. 64]
	Giải \& biện luận hệ phương trình:
	\begin{equation*}
		\left\{\begin{split}
			x + y + z &= 3,\\
			ax + by + 4z &= a + b,\\
			ax - by + 6z &= a - b.
		\end{split}\right.
	\end{equation*}
\end{baitoan}

\begin{baitoan}[\cite{TLCT_THCS_Toan_9_dai_so}, 11.9., p. 64]
	$18$ chiếc xe tải có $106$ bánh gồm 3 loại: Loại $4$ bánh chở $5$ tấn, loại $6$ bánh chở $6$ tấn, \& loại $8$ bánh chở $6$ tấn. Tính số xe mỗi loại biết chúng chở được $101$ tấn.
\end{baitoan}

\begin{baitoan}[\cite{TLCT_THCS_Toan_9_dai_so}, 11.10., p. 64]
	Có $100$ trâu gồm trâu đứng, trâu nằm, trâu già, \& chúng ăn hết cỏ $100$ bó cỏ. Tìm số trâu mỗi loại, biết mỗi con trâu đứng ăn $5$ bó cỏ, mỗi con trâu nằm ăn $3$ bó cỏ \& $3$ con trâu già mới ăn hết $1$ bó cỏ.
\end{baitoan}

%------------------------------------------------------------------------------%

\section{Giải Bài Toán Bằng Cách Lập Hệ Phương Trình}
\fbox{1} {\sf Giải bài toán bằng cách lập hệ phương trình:} \textit{Bước 1}: Lập hệ phương trình: Chọn 2 đại lượng chưa biết làm ẩn, đặt đơn vị \& điều kiện thích hợp của ẩn. Biểu diễn các đại lượng chưa biết khác trong bài toán theo ẩn. Lập hệ 2 phương trình biểu thị sự tương quan giữa các đại lượng trong bài toán. \textit{Bước 2}: Giải hệ phương trình. \textit{Bước 3}: Chọn kết quả phù hợp \& kết luận. \fbox{2} Các dạng toán: Toán chuyển động đều{\tt/}không đều, toán năng suất lao động, toán về quan hệ giữa các số, $\ldots$

\begin{baitoan}[\cite{Binh_boi_duong_Toan_9_tap_2}, H1--H2, p. 26]
	So sánh các bước giải toán bằng cách lập hệ phương trình với các bước giải bài toán bằng cách lập phương trình.
\end{baitoan}

\begin{baitoan}[\cite{Binh_boi_duong_Toan_9_tap_2}, VD1, p. 26]
	1 ôtô xuất phát từ A dự định đến B lúc {\rm11:00}. Cùng thời gian xuất phát từ A: Nếu vận tốc tăng {\rm10 km{\tt/}h} thì xe đến B lúc {\rm10:00}. Nếu vận tốc giảm {\rm10 km{\tt/}h} thì xe đến B lúc {\rm12:30}. Tính vận tốc dự định của xe, quãng đường AB \& giờ xuất phát từ A.
\end{baitoan}

\begin{baitoan}[\cite{Binh_boi_duong_Toan_9_tap_2}, VD2, p. 27]
	2 người dự định làm chung 1 công việc \& hoàn thành trong {\rm6 h}. Khi thực hiện người thứ nhất đã làm 1 mình trong {\rm2 h} thì người thứ 2 mới cùng làm chung do đó phải {\rm4 h 48 ph} nữa công việc mới hoàn thành. Nếu mỗi người làm 1 mình thì bao lâu xong công việc?
\end{baitoan}

\begin{baitoan}[\cite{Binh_boi_duong_Toan_9_tap_2}, VD3, p. 27]
	Tìm 1 số có 2 chữ số biết trung bình cộng của nó với số viết theo thứ tự ngược lại bằng $77$ \& số đó hơn số viết theo thứ tự ngược lại $18$ đơn vị.
\end{baitoan}

\begin{baitoan}[\cite{Binh_boi_duong_Toan_9_tap_2}, VD4, p. 28]
	Nếu giảm chiều dài 1 thửa ruộng hình chữ nhật {\rm5 m} \& tăng chiều rộng của thửa ruộng ấy thêm {\rm5 m} thì thửa ruộng thành hình vuông. Nếu tăng chiều dài của thửa ruộng đó thêm {\rm5 m} \& tăng chiều rộng thêm {\rm 8m} thì diện tích thửa ruộng tăng thêm $\rm640\ m^2$. Tính các kích thước của thửa ruộng đó.
\end{baitoan}

\begin{baitoan}[\cite{Binh_boi_duong_Toan_9_tap_2}, VD5, p. 28]
	2 bến sông A, B cách nhau {\rm50 km}. 1 canô chạy xuôi từ A đến B, nghỉ tại bến B trong {\rm1 h} rồi quay lại A hết tất cả {\rm5 h 10 ph}. 1 canô thứ 2 cũng khởi hành từ A với vận tốc riêng bằng vận tốc riêng của canô thứ nhất. Cùng lúc canô thứ 2 khởi hành từ A, 1 bè nứa trôi từ A theo dòng nước. Canô thứ 2 đi được nửa quãng đường AB rồi quay lại ngay thì gặp bè nứa tại 1 điểm cách A là $8\dfrac{1}{3}$ {\rm km}. Biết cả đi \& về 2 canô đi với vận tốc riêng không đổi. Tính vận tốc riêng của 2 canô \& vận tốc dòng nước.
\end{baitoan}

\begin{baitoan}[\cite{Binh_boi_duong_Toan_9_tap_2}, VD6, p. 29]
	$100$ con vịt, gà cùng thỏ, chó. Tất cả có $270$ chân. Số chó hơn thỏ là $7$. Số vịt hơn gà $15$. Tính số con mỗi loại.
\end{baitoan}

\begin{baitoan}[\cite{Binh_boi_duong_Toan_9_tap_2}, 3.1., p. 30]
	1 ôtô tải \& 1 ôtô khách cùng khởi hành 1 lúc từ 2 đầu quãng đường AB dài {\rm285 km}. Ôtô khách khởi hành từ A, đi được {\rm1 h} phải dừng lại nghỉ trên đường {\rm30 ph} sau đó đi tiếp {\rm1 h 30 ph} nữa thì gặp ôtô tải đi từ B đến. Biết vận tốc ôtô khách hơn vận tốc ôtô tải là {\rm15 km{\tt/}h}. Tính vận tốc mỗi ôtô.
\end{baitoan}

\begin{baitoan}[\cite{Binh_boi_duong_Toan_9_tap_2}, 3.2., p. 30]
	Trên 1 dòng sông, canô thứ nhất xuôi dòng {\rm84 km} \& ngược dòng {\rm50 km} hết {\rm5 h 30 ph}, canô thứ 2 xuôi dòng {\rm56 km} \& ngược dòng {\rm60 km} hết {\rm5 h}. Biết vận tốc riêng của 2 canô bằng nhau. Tính vận tốc riêng của 2 canô \& vận tốc dòng nước.
\end{baitoan}

\begin{baitoan}[\cite{Binh_boi_duong_Toan_9_tap_2}, 3.3., p. 30]
	Theo kế hoạch 2 tổ sản xuất $800$ sản phẩm trong thời gian nhất định. Do cải tiến kỹ thuật, tổ 1 vượt mức $20\%$, tổ 2 vượt mức $30\%$ nên cả 2 tổ sản xuất trong thời gian quy định được là $1010$ sản phẩm. Tính số sản phẩm được giao theo kế hoạch của mỗi tổ.
\end{baitoan}

\begin{baitoan}[\cite{Binh_boi_duong_Toan_9_tap_2}, 3.4., p. 30]
	2 vòi nước cùng chảy vào 1 bể nước sau {\rm4 h} được $90\%$ bể. Nếu vòi thứ nhất chảy trong {\rm2 h} rồi bị khóa, vòi thứ 2 chảy tiếp {\rm3 h} thì được $55\%$ bể. Hỏi nếu mỗi vòi chảy 1 mình sau bao lâu thì đầy bể?
\end{baitoan}

\begin{baitoan}[\cite{Binh_boi_duong_Toan_9_tap_2}, 3.5., p. 30]
	Tìm 1 số có 2 chữ số biết chữ số hàng đơn vị hơn chữ số hàng chục $2$ đơn vị. Nếu ta thêm chữ số $0$ vào giữa 2 chữ số thì được số có 3 chữ số hơn $2$ lần số có 2 chữ số ban đầu $472$ đơn vị.
\end{baitoan}

\begin{baitoan}[\cite{Binh_boi_duong_Toan_9_tap_2}, 3.6., p. 30]
	1 thư viện nhà trường có 2 tủ sách. Nếu chuyển $100$ quyển từ tủ thứ nhất sang tủ thứ 2 thì số sách 2 tủ bằng nhau. Nếu chuyển $300$ quyển từ tử thứ 2 sang tủ thứ nhất đồng thời tủ thứ 2 cho mượn $100$ cuốn thì số sách còn lại của tủ thứ 2 bằng nửa số sách có trong tủ thứ nhất. Lúc đầu mỗi tủ có bao nhiêu cuốn sách?
\end{baitoan}

\begin{baitoan}[\cite{Binh_boi_duong_Toan_9_tap_2}, 3.7., p. 30]
	1 đội xe vận tải chở $256$ tấn hàng trên $29$ xe gồm 3 loại: Xe 4 bánh, mỗi xe chở $5$ tấn hàng, xe $4$ bánh mỗi xe chở $8$ tấn hàng, xe $6$ bánh mỗi xe chở $11$ tấn hàng. Tổng số bánh xe là $144$ bánh. Tính số xe mỗi loại.
\end{baitoan}

\begin{baitoan}[\cite{Binh_boi_duong_Toan_9_tap_2}, 3.8., p. 30]
	Tính diện tích hình thang có chiều cao {\rm12 m}. Biết nếu giảm đáy lớn {\rm4 m}, tăng đáy nhỏ {\rm5 m} \& tăng chiều cao {\rm3 m} thì diện tích tăng thêm $\rm60\ m^2$. Nếu chiều cao hình thang không là {\rm12 m} mà bằng hiệu của 2 đáy thì diện tích hình thang bằng $\rm87.5\ m^2$.
\end{baitoan}

\begin{baitoan}[\cite{Binh_boi_duong_Toan_9_tap_2}, BT1, p. 31, Ai Cập 3000 B.C.]
	Chia $100$ đấu lúa mì cho $5$ người sao cho người thứ 2 nhận được hơn người thứ nhất 1 số lúa bằng số lúa mà người thứ 3 nhận được hơn người thứ 2, người thứ 4 hơn người thứ 3 \& người thứ 5 hơn người thứ 4. Thêm vào đó 2 người đầu tiên nhận được 1 số lúa ít hơn $7$ lần số lúa của 3 người còn lại. Tính số lúa mỗi người nhận.
\end{baitoan}

\begin{baitoan}[\cite{Binh_boi_duong_Toan_9_tap_2}, BT2, p. 31]
	Ngựa \& La đi cạnh tranh nhau \& cùng chở vật nặng trên lưng. Ngựa than thở về hành lý quá nặng của mình. La đáp: ``Cậu than thở nỗi gì? Nếu tôi lấy của cậu 1 bao thì hành lý của tôi nặng gấp đôi của cậu. Còn nếu cậu lấy ở trên lưng tôi 1 bao thì hành lý của cậu mới bằng của tôi.'' Tính số bao Ngựa \& La mang.
\end{baitoan}

\begin{baitoan}[\cite{Binh_boi_duong_Toan_9_tap_2}, p. 32]
	Trăm trâu trăm cỏ. Trâu đứng ăn $5$. Trâu nằm ăn $3$. Lọm khọm trâu già. 3 con 1 bó. Tính số trâu đứng, trâu nằm, \& trâu già.
\end{baitoan}

\begin{baitoan}[\cite{Binh_boi_duong_Toan_9_tap_2}, p. 32]
	Ngày xuân dạo chơi chợ phiên. Chỉ có 1 tiền ($= 60$ đồng) mua đủ trái cây. Cam $3$ đồng 1 trái đây. Quýt đường $5$ trái trả ngay 1 đồng. Đoan Hùng bưởi đó vô song. $5$ đồng 1 quả cùng không đắt nào. 1 giờ liệu đã liệu mua sao. Cả cam, quýt, bưởi vừa vào $100$.
\end{baitoan}

\begin{baitoan}[\cite{Tuyen_Toan_9_old}, VD31, p. 63]
	2 bến sông A, B cách nhau {\rm40 km}. 1 canô xuôi từ A đến B rồi quay ngay về A với vận tốc riêng không đổi hết tất cả {\rm2 h 15 ph}. Khi canô khởi hành từ A thì cùng lúc đó, 1 khúc gỗ cũng trôi tự do từ A theo dòng nước \& gặp canô trên đường trở về tại 1 điểm cách A là {\rm 8km}. Tính vận tốc riêng của canô \& vận tốc của dòng nước.
\end{baitoan}

\begin{baitoan}[\cite{Tuyen_Toan_9_old}, 171., pp. 64--65]
	1 lớp học chỉ có 2 loại học sinh là giỏi \& khá. Nếu có $1$ học sinh giỏi chuyển đi thì $\frac{1}{6}$ số học sinh còn lại là học sinh giỏi. Nếu có $1$ học sinh khá chuyển đi thì $\frac{1}{5}$ số học sinh còn lại là học sinh giỏi. Tính số học sinh của lớp.
\end{baitoan}

\begin{baitoan}[\cite{Tuyen_Toan_9_old}, 172., p. 65]
	4 người góp vốn kinh doanh dược tổng số tiền $6$ tỷ đồng. Số tiền người thứ nhất, thứ 2, thứ 3 góp lần lượt bằng $\frac{1}{3},\frac{1}{4},\frac{1}{5}$ tổng số tiền 3 người còn lại. Tính số tiền vốn người thứ 4 góp.
\end{baitoan}

\begin{baitoan}[\cite{Tuyen_Toan_9_old}, 173., p. 65]
	2 địa điểm A, B cách nhau {\rm360 km}. Cùng 1 lúc, 1 xe tải khởi hành từ A chạy về B \& 1 xe con chạy từ B về A. Sau khi gặp nhau xe tải chạy tiếp trong {\rm5 h} nữa thì đến B \& xe con chạy {\rm3 h 12 ph} nữa thì tới A. Tính vận tốc mỗi xe.
\end{baitoan}

\begin{baitoan}[\cite{Tuyen_Toan_9_old}, 174., p. 65]
	Để làm xong 1 công việc, nếu A, B cùng làm thì mất {\rm6 h}, nếu B, C cùng làm thì mất {\rm4.5 h}, nếu A, C cùng làm thì chỉ mất {\rm3 h 36 ph}. Tính thời gian để làm xong công việc đó nếu cả 3 cùng làm.
\end{baitoan}

\begin{baitoan}[\cite{Tuyen_Toan_9_old}, 175., p. 65]
	Để vận chuyển 1 số gạch đến công trình xây dựng, có thể dùng 1 xe loại lớn chở $10$ chuyến hoặc dùng $1$ xe loại nhỏ chở $15$ chuyến. Người ta dùng cả 2 loại xe đó. Biết tổng cộng có tất cả $11$ chuyến xe vừa lớn vừa nhỏ. Tính số chuyến mỗi loại xe đã chở.
\end{baitoan}

\begin{baitoan}[\cite{Tuyen_Toan_9_old}, 176., p. 65]
	Khi thêm {\rm 1 l} acid vào dung dịch acid thì dung dịch mới có nồng độ acid $40\%$. Lại thêm {\rm 1 l} nước vào dung dịch mới ta được dung dịch acid có nồng độ $33\frac{1}{3}\%$. Tính nồng dộ acid trong dung dịch đầu tiên.
\end{baitoan}

\begin{baitoan}[\cite{Binh_Toan_9_tap_2}, VD72., p. 15]
	Điểm trung bình của $100$ học sinh trong 2 lớp 8A, 8B là $7.2$. Tính điểm trung bình của các học sinh mỗi lớp, biết số học sinh lớp 8A gấp rưỡi số học sinh lớp 8B \& điểm trung bình của lớp 8B gấp rưỡi điểm trung bình của lớp 8A.
\end{baitoan}

\begin{baitoan}[\cite{Binh_Toan_9_tap_2}, VD73., p. 15]
	Giả sử có 1 cánh đồng cỏ dày như nhau, mọc cao đều như nhau trên toàn bộ cánh đồng trong suốt thời gian bò ăn cỏ trên cánh đồng ấy. Biết $9$ con bò ăn hết cỏ trên cánh đồng trong $2$ tuần, $6$ con bò ăn hết cỏ trên cánh đồng trong $4$ tuần. Hỏi bao nhiêu con bò ăn hết cỏ trên cánh đồng trong $6$ tuần? (mỗi con bò ăn số cỏ như nhau).
\end{baitoan}

\begin{baitoan}[\cite{Binh_Toan_9_tap_2}, 226.., p. 16]
	Có $45$ người gồm bác sĩ \& luật sư, tuổi trung bình của họ là $40$. Tính số bác sĩ, số luật sư, biết tuổi trung bình của các bác sĩ là $35$, tuổi trung bình của các luật sư là $50$.
\end{baitoan}

\begin{baitoan}[\cite{Binh_Toan_9_tap_2}, 227.., p. 16]
	Trong 1 hội trường có 1 số ghế băng, mỗi ghế băng quy định ngồi 1 số người như nhau. Nếu bớt $2$ ghế băng \& mỗi ghế băng ngồi thêm $1$ người thì thêm được $8$ chỗ. Nếu thêm $3$ ghế băng \& mỗi ghế băng ngồi rút đi $1$ người thì giảm $8$ chỗ. Tính số ghế băng trong hội trường.
\end{baitoan}

\begin{baitoan}[\cite{Binh_Toan_9_tap_2}, 228.., p. 17]
	Có 2 loại quặng sắt: quặng loại I chứa $70\%$ sắt, quặng loại II chứa $40\%$ sắt. Trộn 1 lượng quặng loại I với 1 lượng quặng loại II thì được hỗn hợp quặng chứa $60\%$ sắt. Nếu lấy tăng hơn lúc đầu $5$ tấn quặng loại I \& lấy giảm hơn lúc đầu $5$ tấn quặng loại II thì được hỗn hợp quặng chứa $65\%$ sắt. Tính khối lượng mỗi loại quặng đem trộn lúc đầu.
\end{baitoan}

\begin{baitoan}[\cite{Binh_Toan_9_tap_2}, 229.., p. 17]
	Cho thêm {\rm1 kg} nước vào dung dịch A thì được dung dịch B có nồng độ acid là $20\%$ (nồng độ acid là tỷ số của khối lượng acid so với khối lượng dung dịch). Sau đó lại cho thêm {\rm1 kg} acid vào dung dịch B thì được dung dịch C có nồng độ acid là $33\frac{1}{3}\%$. Tính nồng độ acid trong dung dịch A.
\end{baitoan}

\begin{baitoan}[\cite{Binh_Toan_9_tap_2}, 230.., p. 17]
	2 vòi nước cùng chảy vào 1 bể trong $1$ giờ thì được $\frac{3}{10}$ bể. Nếu vòi I chảy trong $3$ giờ, vòi II chảy trong $2$ giờ thì mới được $\frac{4}{5}$ bể. Hỏi mỗi vòi chảy 1 mình thì trong bao lâu bể sẽ đầy?
\end{baitoan}

\begin{baitoan}[\cite{Binh_Toan_9_tap_2}, 231.., p. 17]
	Lúc {\rm7:00}, An khởi hành từ A để đến gặp Bích tại B lúc {\rm9:30}. Nhưng đến {\rm9:00}, An được biết Bích bắt đầu đi từ B để đến C (không nằm trên quãng đường AB) với vận tốc bằng $3.25$ lần vận tốc của An. Ngay lúc đó, An tăng thêm vận tốc {\rm1km{\tt/}h} \& khi tới B, An đã đi theo đường tắt đến C cùng 1 lúc. Nếu Bích cũng đi theo đường tắt như An thì Bích đến B trước An là $2$ giờ. Tính vận tốc lúc đầu của An.
\end{baitoan}

\begin{baitoan}[\cite{Binh_Toan_9_tap_2}, 232.., p. 17, Newton's problem]
	1 cánh đồng cỏ dày như nhau, mọc cao đều như nhau trên toàn bộ cánh đồng \& trong suốt thời gian bò ăn cỏ trên cánh đồng ấy. Biết $75$ con bò ăn hết cỏ trên {\rm60 a} đồng cỏ trong $12$ ngày, $81$ con bò ăn hết cỏ trên {\rm72 a} đồng cỏ đó trong $15$ ngày. Hỏi bao nhiêu con bò ăn hết cỏ trên {\rm96 a} đồng cỏ trong $18$ ngày? $\rm1\ a = 100\ m^2$.
\end{baitoan}

\begin{baitoan}[\cite{Binh_Toan_9_tap_2}, 233.., p. 17]
	Tìm 1 số có 2 chữ số biết nếu lấy bình phương của số đó trừ đi bình phương của số gồm chính 2 chữ số của số phải tìm viết theo thứ tự ngược lại thì được 1 số chính phương.
\end{baitoan}

\begin{baitoan}[\cite{Binh_Toan_9_tap_2}, 234.., p. 17]
	3 tổ nhân công $A,B,C$ có tuổi trung bình lần lượt là $37,23,41$. Tuổi trung bình của 2 tổ $A,B$ là $29$, tuổi trung bình của 2 tổ $B,C$ là $33$. Tính tuổi trung bình của cả 3 tổ.
\end{baitoan}

%------------------------------------------------------------------------------%

\section{Miscellaneous}
\cite[BTCCI, pp. 26--27]{SGK_Toan_9_Canh_Dieu_tap_1}: 1. 2. 3. 4. 5. 6. 7. 8. 9. 10. 11.

\begin{baitoan}[\cite{Tuyen_Toan_9_old}, VD32, pp. 65--66]
	\begin{equation*}
		\left\{\begin{split}
			mx - y &= 2m,\\
			x - my &= 1 + m.
		\end{split}\right.
	\end{equation*}
	(a)Tìm $m\in\mathbb{R}$ để hệ có nghiệm duy nhất. (b) Tìm $m\in\mathbb{R}$ để hệ có nghiệm nguyên. (c) Chứng minh $A(x,y)$, với $(x,y)$ là nghiệm của hệ, luôn nằm trên 1 đường thẳng cố định. (d) Tìm $m\in\mathbb{R}$ để biểu thức $A = xy$, với $(x,y)$ là nghiệm của hệ  có {\rm GTLN}. Tìm {\rm GTLN} đó.
\end{baitoan}

\begin{baitoan}[\cite{Tuyen_Toan_9_old}, 177., pp. 67--68]
	Cho các đường thẳng $(d_1):2x - y = 2m - 1,(d_2):4x - 3y = 4m + 1$. (a) Chứng minh khi $m$ thay đổi, $(d_1),(d_2)$ luôn cắt nhau. Tìm tọa độ giao điểm $M$ theo $m$. (b) Chứng minh khi $m$ thay đổi thì $M$ luôn di động trên 1 đường thẳng cố định. (c) Tìm {\rm GTNN} của biểu thức $A = x^2 + y^2$ với $x,y$ thỏa mãn phương trình của $(d_1),(d_2)$.
\end{baitoan}

\begin{baitoan}[\cite{Tuyen_Toan_9_old}, 178., p. 68]
	Cho đường thẳng $(d_1):2x - y = 3$. Đường thẳng $(d_2):y = ax + b$ đối xứng với đường thẳng $(d_1)$ qua trục hoành. Tính $a + b$.
\end{baitoan}

\begin{baitoan}[\cite{Tuyen_Toan_9_old}, 179., p. 68]
	Trong mặt phẳng tọa độ $Oxy$, cho điểm $M(3,2)$. Đếm số đường thẳng đi qua M, cắt $Ox$ tại điểm có hoành độ là 1 số nguyên dương, đồng thời cắt $Oy$ tại 1 điểm có tung độ là 1 số nguyên dương.
\end{baitoan}

\begin{baitoan}[\cite{Tuyen_Toan_9_old}, 180., p. 68]
	Tìm 1 số biết nếu thêm $12$ hay bớt đi $12$ thì đều được 1 số chính phương.
\end{baitoan}

\begin{baitoan}[\cite{Tuyen_Toan_9_old}, 181., p. 68]
	2 công nhân được giao làm cùng 1 loại sản phẩm với số lượng thời gian như nhau. Người thứ nhất mỗi giờ làm tăng $1$ sản phẩm nên đã hoàn thành công việc trước thời hạn {\rm2 h}. Người thứ 2 mỗi giờ làm tăng $2$ sản phẩm nên chẳng những đã hoàn thành công việc trước thời ghạn {\rm3 h} mà còn vượt mức $7$ sản phẩm nữa. Tính số lượng sản phẩm mà mỗi người được giao làm.
\end{baitoan}

\begin{baitoan}[\cite{Tuyen_Toan_9_old}, 182., p. 68]
	1 người mua $30$ con chim gồm 3 loại: chim sẻ, chim ngói, \& bồ câu hết tất cả $30$ đồng. Biết $3$ con chim sẻ giá $1$ đồng, $2$ con chim ngói giá $1$ đồng, \& mỗi con bồ câu giá $2$ đồng. Tính số con mỗi loại.
\end{baitoan}

\begin{baitoan}[\cite{Tuyen_Toan_9_old}, 183., p. 68]
	1 con ngựa giá $204$ đồng. Có 3 người mua ngựa nhưng mỗi người đều không đủ tiền mua. Người thứ nhất nói với 2 người kia: Mỗi người cho tôi vay 1 nửa số tiền của mình thì tôi đủ tiền mua ngựa. Người thứ 2 nói với 2 người kia: Mỗi người cho tôi vay $\frac{1}{3}$ số tiền của mình, tôi sẽ mua được ngựa. Người thứ 3 nói: Chỉ cần 2 anh cho tôi vay $\frac{1}{4}$ số tiền của mình thì con ngựa sẽ là của tôi. Tính số tiền mỗi người.
\end{baitoan}

%------------------------------------------------------------------------------%

\printbibliography[heading=bibintoc]
	
\end{document}