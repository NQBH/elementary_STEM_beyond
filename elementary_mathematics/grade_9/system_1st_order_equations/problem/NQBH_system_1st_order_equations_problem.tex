\documentclass{article}
\usepackage[backend=biber,natbib=true,style=alphabetic,maxbibnames=50]{biblatex}
\addbibresource{/home/nqbh/reference/bib.bib}
\usepackage[utf8]{vietnam}
\usepackage{tocloft}
\renewcommand{\cftsecleader}{\cftdotfill{\cftdotsep}}
\usepackage[colorlinks=true,linkcolor=blue,urlcolor=red,citecolor=magenta]{hyperref}
\usepackage{amsmath,amssymb,amsthm,float,graphicx,mathtools,tikz}
\usetikzlibrary{angles,calc,intersections,matrix,patterns,quotes,shadings}
\allowdisplaybreaks
\newtheorem{assumption}{Assumption}
\newtheorem{baitoan}{}
\newtheorem{cauhoi}{Câu hỏi}
\newtheorem{conjecture}{Conjecture}
\newtheorem{corollary}{Corollary}
\newtheorem{dangtoan}{Dạng toán}
\newtheorem{definition}{Definition}
\newtheorem{dinhly}{Định lý}
\newtheorem{dinhnghia}{Định nghĩa}
\newtheorem{example}{Example}
\newtheorem{ghichu}{Ghi chú}
\newtheorem{hequa}{Hệ quả}
\newtheorem{hypothesis}{Hypothesis}
\newtheorem{lemma}{Lemma}
\newtheorem{luuy}{Lưu ý}
\newtheorem{nhanxet}{Nhận xét}
\newtheorem{notation}{Notation}
\newtheorem{note}{Note}
\newtheorem{principle}{Principle}
\newtheorem{problem}{Problem}
\newtheorem{proposition}{Proposition}
\newtheorem{question}{Question}
\newtheorem{remark}{Remark}
\newtheorem{theorem}{Theorem}
\newtheorem{vidu}{Ví dụ}
\usepackage[left=1cm,right=1cm,top=5mm,bottom=5mm,footskip=4mm]{geometry}
\def\labelitemii{$\circ$}
\DeclareRobustCommand{\divby}{%
	\mathrel{\vbox{\baselineskip.65ex\lineskiplimit0pt\hbox{.}\hbox{.}\hbox{.}}}%
}

\title{Problem: System of 1st-Order Equations -- Bài Tập: Hệ Phương Trình Bậc Nhất}
\author{Nguyễn Quản Bá Hồng\footnote{Independent Researcher, Ben Tre City, Vietnam\\e-mail: \texttt{nguyenquanbahong@gmail.com}; website: \url{https://nqbh.github.io}.}}
\date{\today}

\begin{document}
\maketitle
\tableofcontents

%------------------------------------------------------------------------------%

\section{1st-Order Equations of 2 Unknowns -- Phương Trình Bậc Nhất 2 Ẩn}

\begin{baitoan}[\cite{Binh_Toan_9_tap_2}, VD66, p. 5]
	Cho đường thẳng: $d:(m - 2)x + (m - 1)y = 1$ với tham số $m$. (a) Chứng minh đường thẳng $d$ luôn đi qua 1 điểm cố định với mọi giá trị của $m$. (b) Tìm giá trị của $m$ để khoảng cách từ gốc tọa độ O đến $d$ lớn nhất.
\end{baitoan}

\begin{baitoan}[\cite{Binh_Toan_9_tap_2}, VD67, p. 6]
	Tìm các điểm thuộc đường thẳng $3x - 5y = 8$ có tọa độ là các số nguyên \& nằm trên dải song song tạo bởi 2 đường thẳng $y = 10,y = 20$.
\end{baitoan}

\begin{baitoan}[\cite{Binh_Toan_9_tap_2}, 198., p. 8]
	Xét các đường thẳng $d$ có phương trình: $(2m + 3)x + (m + 5)y + 4m - 1 = 0$ với tham số $m$. (a) Vẽ đường thẳng $d$ ứng với $m = -1$. (b) Tìm điểm cố định mà mọi đường thẳng $d$ đều đi qua.
\end{baitoan}

\begin{baitoan}[\cite{Binh_Toan_9_tap_2}, 199., p. 8]
	Tìm các giá trị của $b,c$ để các đường thẳng $4x + by + c = 0,cx - 3y + 9 = 0$ trùng nhau.
\end{baitoan}

\begin{baitoan}[\cite{Binh_Toan_9_tap_2}, 200., p. 8]
	Vẽ đồ thị biểu diễn tập nghiệm của phương trình $x^2 - 2xy + y^2 = 1$.
\end{baitoan}

\begin{baitoan}[\cite{Binh_Toan_9_tap_2}, 201., p. 8]
	Đường thẳng $ax + by = 6$ với $a > 0,b > 0$, tạo với 2 trục tọa độ 1 tam giác có diện tích bằng $9$. Tính $ab$.
\end{baitoan}

\begin{baitoan}[\cite{Binh_Toan_9_tap_2}, 202., p. 8]
	Cho đường thẳng $d:(m + 2)x - my = -1$ với tham số $m$. (a) Tìm điểm cố định mà $d$ luôn đi qua. (b) Tìm giá trị của $m$ để khoảng cách từ gốc tọa độ O đến $d$ lớn nhất.
\end{baitoan}

\begin{baitoan}[\cite{Binh_Toan_9_tap_2}, 203., p. 8]
	Trong hệ trục tọa độ $Oxy$, $A(1,1),B(9,1)$. Viết phương trình của đường thẳng $d\bot AB$ \& chia $\Delta OAB$ thành 2 phần có diện tích bằng nhau.
\end{baitoan}

\begin{baitoan}[\cite{Binh_Toan_9_tap_2}, 204., p. 8]
	Tìm các điểm nằm trên đường thẳng $8x + 9y = -79$, có hoành độ \& tung độ là các số nguyên \& nằm bên trong góc vuông phần tư {\rm III}.
\end{baitoan}

\begin{baitoan}[\cite{Binh_Toan_9_tap_2}, 205., p. 8]
	Cho 2 điểm $A(3,17),B(33,193)$. (a) Viết phương trình của đường thẳng AB. (b) Có bao nhiêu điểm thuộc đoạn thẳng AB \& có hoành độ \& tung độ là các số nguyên?
\end{baitoan}

\begin{baitoan}[\cite{Binh_Toan_9_tap_2}, 206., p. 8]
	(a) Vẽ đồ thị hàm số $d:y = \dfrac{3}{2}x + \dfrac{7}{4}$. (b) Có bao nhiêu điểm nằm trên cạnh hoặc nằm trong tam giác tạo bởi 3 đường thẳng $x = 6,y = 0,d$.
\end{baitoan}

%------------------------------------------------------------------------------%

\section{System of 1st-Order Equations of 2 Unknowns -- Hệ Phương Trình Bậc Nhất 2 Ẩn}

%------------------------------------------------------------------------------%

\section{Miscellaneous}

%------------------------------------------------------------------------------%

\printbibliography[heading=bibintoc]
	
\end{document}