\documentclass{article}
\usepackage[backend=biber,natbib=true,style=alphabetic,maxbibnames=50]{biblatex}
\addbibresource{/home/nqbh/reference/bib.bib}
\usepackage[utf8]{vietnam}
\usepackage{tocloft}
\renewcommand{\cftsecleader}{\cftdotfill{\cftdotsep}}
\usepackage[colorlinks=true,linkcolor=blue,urlcolor=red,citecolor=magenta]{hyperref}
\usepackage{amsmath,amssymb,amsthm,float,graphicx,mathtools,tikz}
\usetikzlibrary{angles,calc,intersections,matrix,patterns,quotes,shadings}
\allowdisplaybreaks
\newtheorem{assumption}{Assumption}
\newtheorem{baitoan}{}
\newtheorem{cauhoi}{Câu hỏi}
\newtheorem{conjecture}{Conjecture}
\newtheorem{corollary}{Corollary}
\newtheorem{dangtoan}{Dạng toán}
\newtheorem{definition}{Definition}
\newtheorem{dinhly}{Định lý}
\newtheorem{dinhnghia}{Định nghĩa}
\newtheorem{example}{Example}
\newtheorem{ghichu}{Ghi chú}
\newtheorem{hequa}{Hệ quả}
\newtheorem{hypothesis}{Hypothesis}
\newtheorem{lemma}{Lemma}
\newtheorem{luuy}{Lưu ý}
\newtheorem{nhanxet}{Nhận xét}
\newtheorem{notation}{Notation}
\newtheorem{note}{Note}
\newtheorem{principle}{Principle}
\newtheorem{problem}{Problem}
\newtheorem{proposition}{Proposition}
\newtheorem{question}{Question}
\newtheorem{remark}{Remark}
\newtheorem{theorem}{Theorem}
\newtheorem{vidu}{Ví dụ}
\usepackage[left=1cm,right=1cm,top=5mm,bottom=5mm,footskip=4mm]{geometry}
\def\labelitemii{$\circ$}
\DeclareRobustCommand{\divby}{%
	\mathrel{\vbox{\baselineskip.65ex\lineskiplimit0pt\hbox{.}\hbox{.}\hbox{.}}}%
}

\title{Problem: System of 1st-Order Equations -- Bài Tập: Hệ Phương Trình Bậc Nhất}
\author{Nguyễn Quản Bá Hồng\footnote{Independent Researcher, Ben Tre City, Vietnam\\e-mail: \texttt{nguyenquanbahong@gmail.com}; website: \url{https://nqbh.github.io}.}}
\date{\today}

\begin{document}
\maketitle
\tableofcontents

%------------------------------------------------------------------------------%

\section{Phương Trình Quy Về Phương Trình Bậc Nhất 1 Ẩn}
\cite[\S1, pp. 5--11]{SGK_Toan_9_Canh_Dieu_tap_1}: HD1. LT1. LT2. HD2. LT3. HD3. LT4. LT5. 1. 2. 3. 4. 5. 6.

%------------------------------------------------------------------------------%

\section{1st-Order Equations of 2 Unknowns -- Phương Trình Bậc Nhất 2 Ẩn}
\fbox{1} Phương trình bậc nhất 2 ẩn: $ax + by = c$ (1), $a,b,c\in\mathbb{R},(a,b)\ne(0,0)$. \fbox{2} $(x_0,y_0)\in\mathbb{R}^2$ là nghiệm của (1) $\Leftrightarrow(x_0,y_0)\in S\Leftrightarrow ax_0 + by_0 = c$. \fbox{3} Tập nghiệm $S$ biểu diễn bởi đường thẳng $(d):ax + by = c$, i.e., $S = \{(x,y)\in\mathbb{R}^2|ax + by = c\} = (d)$. \fbox{4} Nếu $ab\ne0$ thì $(d):ax + by = c\Leftrightarrow y = -\dfrac{a}{b}x + \dfrac{c}{b}$ (hàm số bậc nhất) là đường thẳng cắt cả 2 trục tọa độ $Ox,Oy$ lần lượt tại 2 điểm $\left(\dfrac{c}{a},0\right),\left(0,\dfrac{c}{b}\right)$. \fbox{5} Nếu $a\ne0,b = 0$ thì $(d):ax + 0y = c\Leftrightarrow x = \dfrac{c}{a}$ là đường thẳng song song hoặc trùng với trục tung $Oy$. \fbox{6} Nếu $a = 0,b\ne0$ thì $(d):0x + by = c\Leftrightarrow y = \dfrac{c}{b}$ là đường thẳng song song hoặc trùng với trục hoành $Ox$.

\begin{baitoan}[\cite{Binh_boi_duong_Toan_9_tap_2}, VD1, p. 8]
	Cho phương trình $2x - y = 6$. (a) Tìm công thức nghiệm tổng quát của phương trình. (b) Vẽ đường thẳng biểu diễn tập nghiệm.
\end{baitoan}

\begin{baitoan}[\cite{Binh_boi_duong_Toan_9_tap_2}, VD2, p. 9]
	(a) Cho phương trình $ax + 2y = 4$. Xác định $a\in\mathbb{R}$ để đường thẳng $(d)$ biểu diễn tập nghiệm của phương trình đi qua điểm $A(1,1.5)$. (b) Vẽ 2 đường thẳng $(d),(t):-2x + y = -3$ trên cùng 1 hệ trục tọa độ. Xác định tọa độ giao điểm của 2 đường thẳng.
\end{baitoan}

\begin{baitoan}[\cite{Binh_boi_duong_Toan_9_tap_2}, VD3, p. 9]
	(a) Tìm nghiệm nguyên của phương trình $2x - 3y = 6$. (b) Tìm nghiệm nguyên dương của phương trình $2x + 5y = 9$.
\end{baitoan}

\begin{baitoan}[\cite{Binh_boi_duong_Toan_9_tap_2}, 1.1., p. 10]
	Trong các cặp số $(0,3),(2,1),(1.5,-2),(4,-6),(-2,0)$, cặp số nào là nghiệm của phương trình: (a) $2x - 5y = -1$. (b) $3x + 4y = -6$.
\end{baitoan}

\begin{baitoan}[\cite{Binh_boi_duong_Toan_9_tap_2}, 1.2., p. 10]
	Tìm $a,b\in\mathbb{R}$ để: (a) Điểm $A(0,-3)$ thuộc đường thẳng $2x + by = -6$. (b) Điểm $B(-2,1)$ thuộc đường thẳng $ax + 4y = 8$. (c) Điểm $C(2.5,0)$ thuộc đường thẳng $ax - 5y = 7.5$. (d) Điểm $D(2,-4)$ thuộc đường thẳng $5x + by = -4$.
\end{baitoan}

\begin{baitoan}[\cite{Binh_boi_duong_Toan_9_tap_2}, 1.3., p. 10]
	Tìm nghiệm tổng quát của phương trình: (a) $3x - 5y = 15$. (b) $5x + 0y = -4$. (c) $0x + 9y = 27$. Vẽ 3 đường thẳng biểu diễn 3 tập nghiệm \& nhận xét.
\end{baitoan}

\begin{baitoan}[\cite{Binh_boi_duong_Toan_9_tap_2}, 1.4., p. 10]
	Tìm nghiệm tổng quát của phương trình: (a) $2x + y = -6$. (b) $x - 2y = 2$. Vẽ 2 đường thẳng biểu diễn tập nghiệm của 2 phương trình trên cùng 1 hệ trục tọa độ. Xác định tọa độ giao điểm của 2 đường thẳng.
\end{baitoan}

\begin{baitoan}[\cite{Binh_boi_duong_Toan_9_tap_2}, 1.5., p. 11]
	Tìm nghiệm nguyên của phương trình: (a) $x + 2y = 0$. (b) $3x - y = 0$. (c) $2x - 5y = 10$. (d) $7x + 4y = 13$.
\end{baitoan}

\begin{baitoan}[\cite{Binh_boi_duong_Toan_9_tap_2}, 1.6., p. 11]
	Tìm nghiệm nguyên dương của phương trình: (a) $5x + 3y - 13 = 0$. (b) $28x + 31y = 273$.
\end{baitoan}

\begin{baitoan}[\cite{Binh_boi_duong_Toan_9_tap_2}, 1.7., p. 11]
	Cho đường thẳng $(d):(a - 2)x + (2a + 3)y - a - 5 = 0$. (a) Xác định $a$ để $(d)$ song song với trục hoành. (b) Xác định $a$ để $(d)$ song song với trục tung. (c) Xác định $a$ để $(d)$ đi qua điểm $A(-3,5)$. (d) Xác định điểm cố định mà $(d)$ luôn đi qua $\forall a\in\mathbb{R}$.
\end{baitoan}

\begin{baitoan}[\cite{Binh_boi_duong_Toan_9_tap_2}, 1.8., p. 11]
	Cho 2 đường thẳng $(d_1):ax - 4y = 8,(d_2):5x - 10y = b - 5$. (a) Tính $2016a + 100b$ khi $(d_1),(d_2)$ cắt nhau tại $A(2,-3)$. (b) Tính $10a - 3b$ khi 2 đường thẳng $(d_1),(d_2)$ có vô số điểm chung.
\end{baitoan}

\begin{baitoan}[\cite{Binh_boi_duong_Toan_9_tap_2}, 1.9., p. 11]
	Trên 1 đoạn đường phố thẳng dài {\rm100 m} 1 đội công nhân lắp đường ống dẫn nước. Có 2 loại ống, 1 loại dài {\rm3 m}, 1 loại dài {\rm5 m}. Hỏi có bao nhiêu các lắp các ống nước trên đoạn đường đó (các mối nối không đáng kể)?
\end{baitoan}

\begin{baitoan}[\cite{Binh_boi_duong_Toan_9_tap_2}, p. 12]
	1 người bán hàng phải trả lại cho khách $45000$ đồng. Người đó chỉ có 1 tờ $10000$ đồng đã trả lại cho khách hàng, còn lại chỉ có 2 loại tiền lẻ là $2000$ đồng \& $5000$ đồng. Cho biết người bán hàng sẽ có các cách nào trả lại tiếp cho khách hàng đúng số tiền trên?
\end{baitoan}

\begin{baitoan}[\cite{Binh_boi_duong_Toan_9_tap_2}, p. 12]
	Chứng minh khoảng cách $h$ từ gốc tọa độ $O$ đến đường thẳng $ax + by = c$ được cho bởi công thức $h = OH = \dfrac{|c|}{\sqrt{a^2 + b^2}}$ với $H$ là hình chiếu của $O$ lên đường thẳng.
\end{baitoan}

\begin{baitoan}[\cite{Binh_boi_duong_Toan_9_tap_2}, p. 12, Điều kiện để phương trình bậc nhất 2 ẩn có nghiệm nguyên]
	Chứng minh phương trình $ax + by = c$, $a,b,c\in\mathbb{Z},(a,b)\ne(0,0)$, có nghiệm nguyên khi \& chỉ khi $c\divby\mbox{\rm ƯCLN}(a,b)$.
\end{baitoan}

\begin{dinhnghia}[Điểm nguyên]
	Trong mặt phẳng tọa độ, điểm $M(x,y)$ được gọi là {\rm điểm nguyên} nếu $x,y\in\mathbb{Z}$.
\end{dinhnghia}

\begin{baitoan}[\cite{Binh_boi_duong_Toan_9_tap_2}, p. 12]
	Tìm tất cả các điểm nguyên trên đường thẳng $4x - 5y + 6 = 0$, nằm giữa 2 đường thẳng: (a) $x = -10,x = 20$. (b) $x = a,x = b$ với $a,b\in\mathbb{R}$.
\end{baitoan}

\begin{baitoan}[\cite{Binh_Toan_9_tap_2}, VD66, p. 5]
	Cho đường thẳng: $d:(m - 2)x + (m - 1)y = 1$ với tham số $m$. (a) Chứng minh đường thẳng $d$ luôn đi qua 1 điểm cố định với mọi giá trị của $m$. (b) Tìm giá trị của $m$ để khoảng cách từ gốc tọa độ O đến $d$ lớn nhất.
\end{baitoan}

\begin{baitoan}[\cite{Binh_Toan_9_tap_2}, VD67, p. 6]
	Tìm các điểm thuộc đường thẳng $3x - 5y = 8$ có tọa độ là các số nguyên \& nằm trên dải song song tạo bởi 2 đường thẳng $y = 10,y = 20$.
\end{baitoan}

\begin{baitoan}[\cite{Binh_Toan_9_tap_2}, 198., p. 8]
	Xét các đường thẳng $d$ có phương trình: $(2m + 3)x + (m + 5)y + 4m - 1 = 0$ với tham số $m$. (a) Vẽ đường thẳng $d$ ứng với $m = -1$. (b) Tìm điểm cố định mà mọi đường thẳng $d$ đều đi qua.
\end{baitoan}

\begin{baitoan}[\cite{Binh_Toan_9_tap_2}, 199., p. 8]
	Tìm các giá trị của $b,c$ để các đường thẳng $4x + by + c = 0,cx - 3y + 9 = 0$ trùng nhau.
\end{baitoan}

\begin{baitoan}[\cite{Binh_Toan_9_tap_2}, 200., p. 8]
	Vẽ đồ thị biểu diễn tập nghiệm của phương trình $x^2 - 2xy + y^2 = 1$.
\end{baitoan}

\begin{baitoan}[\cite{Binh_Toan_9_tap_2}, 201., p. 8]
	Đường thẳng $ax + by = 6$ với $a > 0,b > 0$, tạo với 2 trục tọa độ 1 tam giác có diện tích bằng $9$. Tính $ab$.
\end{baitoan}

\begin{baitoan}[\cite{Binh_Toan_9_tap_2}, 202., p. 8]
	Cho đường thẳng $d:(m + 2)x - my = -1$ với tham số $m$. (a) Tìm điểm cố định mà $d$ luôn đi qua. (b) Tìm giá trị của $m$ để khoảng cách từ gốc tọa độ O đến $d$ lớn nhất.
\end{baitoan}

\begin{baitoan}[\cite{Binh_Toan_9_tap_2}, 203., p. 8]
	Trong hệ trục tọa độ $Oxy$, $A(1,1),B(9,1)$. Viết phương trình của đường thẳng $d\bot AB$ \& chia $\Delta OAB$ thành 2 phần có diện tích bằng nhau.
\end{baitoan}

\begin{baitoan}[\cite{Binh_Toan_9_tap_2}, 204., p. 8]
	Tìm các điểm nằm trên đường thẳng $8x + 9y = -79$, có hoành độ \& tung độ là các số nguyên \& nằm bên trong góc vuông phần tư {\rm III}.
\end{baitoan}

\begin{baitoan}[\cite{Binh_Toan_9_tap_2}, 205., p. 8]
	Cho 2 điểm $A(3,17),B(33,193)$. (a) Viết phương trình của đường thẳng AB. (b) Có bao nhiêu điểm thuộc đoạn thẳng AB \& có hoành độ \& tung độ là các số nguyên?
\end{baitoan}

\begin{baitoan}[\cite{Binh_Toan_9_tap_2}, 206., p. 8]
	(a) Vẽ đồ thị hàm số $d:y = \dfrac{3}{2}x + \dfrac{7}{4}$. (b) Có bao nhiêu điểm nằm trên cạnh hoặc nằm trong tam giác tạo bởi 3 đường thẳng $x = 6,y = 0,d$.
\end{baitoan}

%------------------------------------------------------------------------------%

\section{System of 1st-Order Equations of 2 Unknowns -- Hệ Phương Trình Bậc Nhất 2 Ẩn}
\fbox{1} Hệ phương trình bậc nhất 2 ẩn:
\begin{equation}
	\label{system of linear eqns 2 unknowns}
	\left\{\begin{split}
		ax + by &= c,\ (d),(a,b)\ne(0,0),\\
		a'x + b'y &= c',\ (d'),(a',b')\ne(0,0),
	\end{split}\right.
\end{equation}
có 1 nghiệm $\Leftrightarrow(d)$ cắt $(d')\Leftrightarrow\dfrac{a}{a'}\ne\dfrac{b}{b'}$, vô nghiệm $\Leftrightarrow(d)\parallel(d')\Leftrightarrow\dfrac{a}{a'} = \dfrac{b}{b'}\ne\dfrac{c}{c'}$, vô số nghiệm $\Leftrightarrow(d)\equiv(d')\Leftrightarrow\dfrac{a}{a'} = \dfrac{b}{b'} = \dfrac{c}{c'}$. \fbox{2} \textit{Phương pháp thế}: Biểu diễn 1 ẩn theo ẩn kia. Biến hệ phương trình thành hệ mới có 1 phương trình 1 ẩn. Giải phương trình 1 ẩn rồi suy ra nghiệm của hệ. \fbox{3} \textit{Phương pháp cộng đại số}: Nhân 2 vế của 2 phương trình với 1 số thích hợp để các hệ số của 1 ẩn nào đó trong 2 phương trình bằng nhau hoặc đối nhau. Dùng quy tắc cộng được hệ mới có 1 phương trình 1 ẩn. Giải phương trình 1 ẩn rồi suy ra nghiệm của hệ.

\noindent\cite[\S2, pp. 12--18]{SGK_Toan_9_Canh_Dieu_tap_1}: HD1. LT1. HD2. LT2. HD3. LT3. LT4. 1. 2. 3. 4. 5. 6. \cite[\S3, pp. 19--25]{SGK_Toan_9_Canh_Dieu_tap_1}: HD1. LT1. LT2. LT3. HD2. LT4. HD3. LT5. LT6. 1. 2. 3. 4. 5. 6. 7.

\begin{baitoan}[\cite{Binh_Toan_9_tap_2}, VD68, p. 9]
	Cho hệ phương trình với tham số $a$:
	\begin{equation*}
		\left\{\begin{split}
			(a + 1)x - y &= a + 1,\\
			x + (a - 1)y &= 2.
		\end{split}\right.
	\end{equation*}
	(a) Giải hệ phương trình với $a = 2$. (b) Giải \& biện luận hệ phương trình. (c) Tìm các giá trị nguyên của $a$ để hệ phương trình có nghiệm nguyên. (d) Tìm các giá trị nguyên của $a$ để nghiệm của hệ phương trình thỏa mãn điều kiện $x + y$ nhỏ nhất.
\end{baitoan}

\begin{baitoan}[\cite{Binh_Toan_9_tap_2}, VD69, p. 10]
	Tìm $a,b,c\in\mathbb{Z}$ thỏa mãn cả 2 phương trình $2a + 3b = 6,3a + 4c = 1$.
\end{baitoan}

\begin{baitoan}[\cite{Binh_Toan_9_tap_2}, VD70, p. 10]
	Cho 2 đường thẳng: $d:2x - 3y = 4,d':3x + 5y = 2$. Tìm trên trục $Ox$ điểm có hoành độ là số nguyên dương nhỏ nhất, sao cho nếu qua điểm đó ta dựng đường vuông góc với $Ox$ thì đường vuông góc ấy cắt 2 đường thẳng $d,d'$ tại 2 điểm có tọa độ là các số nguyên.
\end{baitoan}

\begin{baitoan}[\cite{Binh_Toan_9_tap_2}, VD71, p. 11]
	Giải hệ phương trình với 3 ẩn $x,y,z$ \& các tham số $a,b,c$ khác nhau đôi một:
	\begin{equation*}
		\left\{\begin{split}
			a^2x + ay + z &= 5,\\
			b^2x + by + z &= 5,\\
			c^2x + cy + z &= 5.
		\end{split}\right.
	\end{equation*}
\end{baitoan}
Giải hệ phương trình:

\begin{baitoan}[\cite{Binh_Toan_9_tap_2}, 207., p. 12]
	\begin{equation*}
		\left\{\begin{split}
			(x + 3)(y - 5) &= xy,\\
			(x - 2)(y + 5) &= xy,
		\end{split}\right.\hspace{1cm} \left\{\begin{split}
			\frac{1}{x}	+ \frac{1}{y} &= \frac{3}{4},\\
			\frac{1}{6x} + \frac{1}{5y} &= \frac{2}{15}.
		\end{split}\right. 
	\end{equation*}
\end{baitoan}

\begin{baitoan}[\cite{Binh_Toan_9_tap_2}, 208., p. 12]
	\begin{equation*}
		\left\{\begin{split}
			\frac{x}{y} - \frac{x}{y + 12} &= 1,\\
			\frac{x}{y - 12} - \frac{x}{y} &= 2,
		\end{split}\right.\hspace{1cm} \left\{\begin{split}
			4(x + y) &= 5(x - y),\\
			\frac{40}{x + y} + \frac{40}{x - y} &= 9.
		\end{split}\right. 
	\end{equation*}
\end{baitoan}

\begin{baitoan}[\cite{Binh_Toan_9_tap_2}, 209., p. 12]
	\begin{equation*}
		\left\{\begin{split}
			|x - 2| + 2|y - 1| &= 9,\\
			x + |y - 1| &= -1,
		\end{split}\right.\hspace{1cm} \left\{\begin{split}
			x + y + |x| &= 25,\\
			x - y + |y| &= 30.
		\end{split}\right. 
	\end{equation*}
\end{baitoan}

\begin{baitoan}[\cite{Binh_Toan_9_tap_2}, 210., p. 12]
	Tìm các giá trị của $a\in\mathbb{R}$ để 2 hệ phương trình tương đương:
	\begin{equation*}
		\left\{\begin{split}
			2x + 3y &= 8,\\
			3x - y &= 1,
		\end{split}\right.\hspace{1cm} \left\{\begin{split}
			ax - 3y &= -2,\\
			x + y &= 3.
		\end{split}\right. 
	\end{equation*}
\end{baitoan}

\begin{baitoan}[\cite{Binh_Toan_9_tap_2}, 211., p. 12]
	Tìm các giá trị của $m\in\mathbb{R}$ để nghiệm của hệ phương trình sau là 2 số dương:
	\begin{equation*}
		\left\{\begin{split}
			x - y &= 2,\\
			mx + y &= 3.
		\end{split}\right.
	\end{equation*}
\end{baitoan}

\begin{baitoan}[\cite{Binh_Toan_9_tap_2}, 212., p. 12]
	Chứng minh tam giác tạo bởi 3 đường thẳng $y = 3x - 2,y = -\dfrac{1}{3}x + \frac{4}{3},y = -2x + 8$ là tam giác vuông cân.
\end{baitoan}

\begin{baitoan}[\cite{Binh_Toan_9_tap_2}, 213., p. 13]
	Tìm các giá trị của $m\in\mathbb{R}$ để hệ phương trình sau vô nghiệm, vô số nghiệm:
	\begin{equation*}
		\left\{\begin{split}
			2(m + 1)x + (m + 2)y &= m - 3,\\
			(m + 1)x + my &= 3m + 7.
		\end{split}\right.
	\end{equation*}
\end{baitoan}

\begin{baitoan}[\cite{Binh_Toan_9_tap_2}, 214., p. 13]
	Cho hệ phương trình với tham số $m$:
	\begin{equation*}
		\left\{\begin{split}
			mx + 2y &= 1,\\
			3x + (m + 1)y &= -1.
		\end{split}\right.
	\end{equation*}
	(a) Giải hệ phương trình với $m = 3$. (b) Giải \& biện luận hệ phương trình theo $m$. (c) Tìm các giá trị nguyên của $m$ để nghiệm của hệ phương trình là các số nguyên.
\end{baitoan}

\begin{baitoan}[\cite{Binh_Toan_9_tap_2}, 215., p. 13]
	Cho hệ phương trình với tham số $m$:
	\begin{equation*}
		\left\{\begin{split}
			(m - 1)x + y &= 3m - 4,\\
			x + (m - 1)y &= m.
		\end{split}\right.
	\end{equation*}
	(a) Giải \& biện luận hệ phương trình theo $m$. (b) Tìm các giá trị nguyên của $m$ để nghiệm của hệ phương trình là các số nguyên. (c) Tìm các giá trị của $m$ để hệ phương trình có nghiệm dương duy nhất.
\end{baitoan}

\begin{baitoan}[\cite{Binh_Toan_9_tap_2}, 216., p. 13]
	Cho hệ phương trình với tham số $m$:
	\begin{equation*}
		\left\{\begin{split}
			x + my &= m + 1,\\
			mx + y &= 3m - 1.
		\end{split}\right.
	\end{equation*}
	(a) Giải \& biện luận hệ phương trình theo $m$. (b) Trong trường hợp hệ có nghiệm duy nhất, tìm các giá trị của $m$ để tích $xy$ nhỏ nhất.
\end{baitoan}

\begin{baitoan}[\cite{Binh_Toan_9_tap_2}, 217., p. 13]
	Các số không âm $x,y,z$ thỏa mãn hệ phương trình:
	\begin{equation*}
		\left\{\begin{split}
			4x - 4y + 2z &= 1,\\
			8x + 4y + z &= 8.
		\end{split}\right.
	\end{equation*}
	(a) Biểu thị $x,y$ theo $z$. (b) Tìm {\rm GTNN, GTLN} của biểu thức $A = x + y - z$.
\end{baitoan}

\begin{baitoan}[\cite{Binh_Toan_9_tap_2}, 218., p. 13]
	Tìm $a,b,c\in\mathbb{Z}$ thỏa mãn hệ phương trình:
	\begin{equation*}
		\left\{\begin{split}
			2a + 3b &= 5,\\
			3a - 4c &= 6.
		\end{split}\right.
	\end{equation*}
\end{baitoan}

\begin{baitoan}[\cite{Binh_Toan_9_tap_2}, 219., p. 14]
	Tìm trên trục tung các điểm có tung độ là số nguyên, sao cho nếu qua điểm đó ta dựng đường vuông góc với trục tung thì đường vuông góc ấy cắt 2 đường thẳng: $d:x + 2y = 6,d':2x - 3y = 4$ tại các điểm có tọa độ là các số nguyên.
\end{baitoan}

\begin{baitoan}[\cite{Binh_Toan_9_tap_2}, 220., p. 14]
	Tìm trên trục hoành các điểm có hoành độ là số nguyên sao cho nếu qua điểm đó ta dựng đường thẳng vuông góc với trục hoành thì đường vuông góc ấy cắt cả 3 đường thẳng sau tại các điểm có tọa độ là các số nguyên: $d_1:x - 2y = 3,d_2:x - 3y = 2,d_3:x - 5y = -7$.
\end{baitoan}
Giải hệ phương trình ẩn $x,y,z$:

\begin{baitoan}[\cite{Binh_Toan_9_tap_2}, 221., p. 14]
	\begin{equation*}
		\left\{\begin{split}
			x + y + z &= 11,\\
			2x - y + z &= 5,\\
			3x + 2y + z &= 14,
		\end{split}\right.\hspace{1cm}\left\{\begin{split}
			x + y + z + t &= 4,\\
			x + y - z - t &= 8,\\
			x - y + z - t &= 12,\\
			x - y - z + y &= 16.
		\end{split}\right.
	\end{equation*}
\end{baitoan}

\begin{baitoan}[\cite{Binh_Toan_9_tap_2}, 222., p. 14]
	\begin{equation*}
		\left\{\begin{split}
			x + y + z &= 12,\\
			ax + 5y + 4z &= 46,\\
			5x + ay + 3z &= 38,
		\end{split}\right.\hspace{1cm}\left\{\begin{split}
			ax + y + z &= a^2,\\
			x + ay + z &= 3a,\\
			x + y + az &= 2.
		\end{split}\right.
	\end{equation*}
\end{baitoan}

\begin{baitoan}[\cite{Binh_Toan_9_tap_2}, 223., p. 14]
	$a,b,c\in\mathbb{R}$ là tham số, $a + b + c\ne0$.
	\begin{equation*}
		\left\{\begin{split}
			(a + b)(x + y) - cz &= a - b,\\
			(b + c)(y + z) - ax &= b - c,\\
			(c + a)(z + x) - by &= c - a.
		\end{split}\right.
	\end{equation*}
\end{baitoan}

\begin{baitoan}[\cite{Binh_Toan_9_tap_2}, 224., p. 14]
	Giải hệ phương trình với 3 tham số $a,b,c\in\mathbb{R}$ đôi một khác nhau, $a + b + c\ne0$:
	\begin{equation*}
		\left\{\begin{split}
			ax + by + cz &= 0,\\
			bx + cy + az &= 0,\\
			cx + ay + bz &= 0,
		\end{split}\right.\hspace{1cm}\left\{\begin{split}
			ax + by + cz &= a + b + c,\\
			bx + cy + az &= a + b + c,\\
			cx + ay + bz &= a + b + c.
		\end{split}\right.
	\end{equation*}
\end{baitoan}

\begin{baitoan}[\cite{Binh_Toan_9_tap_2}, 225., p. 14]
	\begin{equation*}
		\left\{\begin{split}
			x^2 + xy + xz &= 2,\\
			y^2 + yz + xy &= 3,\\
			z^2 + xz + yz &= 4.
		\end{split}\right.
	\end{equation*}
\end{baitoan}

%------------------------------------------------------------------------------%

\section{Giải Bài Toán Bằng Cách Lập Hệ Phương Trình}

\begin{baitoan}[\cite{Binh_Toan_9_tap_2}, VD72., p. 15]
	Điểm trung bình của $100$ học sinh trong 2 lớp 8A, 8B là $7.2$. Tính điểm trung bình của các học sinh mỗi lớp, biết số học sinh lớp 8A gấp rưỡi số học sinh lớp 8B \& điểm trung bình của lớp 8B gấp rưỡi điểm trung bình của lớp 8A.
\end{baitoan}

\begin{baitoan}[\cite{Binh_Toan_9_tap_2}, VD73., p. 15]
	Giả sử có 1 cánh đồng cỏ dày như nhau, mọc cao đều như nhau trên toàn bộ cánh đồng trong suốt thời gian bò ăn cỏ trên cánh đồng ấy. Biết $9$ con bò ăn hết cỏ trên cánh đồng trong $2$ tuần, $6$ con bò ăn hết cỏ trên cánh đồng trong $4$ tuần. Hỏi bao nhiêu con bò ăn hết cỏ trên cánh đồng trong $6$ tuần? (mỗi con bò ăn số cỏ như nhau).
\end{baitoan}

\begin{baitoan}[\cite{Binh_Toan_9_tap_2}, 226.., p. 16]
	Có $45$ người gồm bác sĩ \& luật sư, tuổi trung bình của họ là $40$. Tính số bác sĩ, số luật sư, biết tuổi trung bình của các bác sĩ là $35$, tuổi trung bình của các luật sư là $50$.
\end{baitoan}

\begin{baitoan}[\cite{Binh_Toan_9_tap_2}, 227.., p. 16]
	Trong 1 hội trường có 1 số ghế băng, mỗi ghế băng quy định ngồi 1 số người như nhau. Nếu bớt $2$ ghế băng \& mỗi ghế băng ngồi thêm $1$ người thì thêm được $8$ chỗ. Nếu thêm $3$ ghế băng \& mỗi ghế băng ngồi rút đi $1$ người thì giảm $8$ chỗ. Tính số ghế băng trong hội trường.
\end{baitoan}

\begin{baitoan}[\cite{Binh_Toan_9_tap_2}, 228.., p. 17]
	Có 2 loại quặng sắt: quặng loại I chứa $70\%$ sắt, quặng loại II chứa $40\%$ sắt. Trộn 1 lượng quặng loại I với 1 lượng quặng loại II thì được hỗn hợp quặng chứa $60\%$ sắt. Nếu lấy tăng hơn lúc đầu $5$ tấn quặng loại I \& lấy giảm hơn lúc đầu $5$ tấn quặng loại II thì được hỗn hợp quặng chứa $65\%$ sắt. Tính khối lượng mỗi loại quặng đem trộn lúc đầu.
\end{baitoan}

\begin{baitoan}[\cite{Binh_Toan_9_tap_2}, 229.., p. 17]
	Cho thêm {\rm1 kg} nước vào dung dịch A thì được dung dịch B có nồng độ acid là $20\%$ (nồng độ acid là tỷ số của khối lượng acid so với khối lượng dung dịch). Sau đó lại cho thêm {\rm1 kg} acid vào dung dịch B thì được dung dịch C có nồng độ acid là $33\frac{1}{3}\%$. Tính nồng độ acid trong dung dịch A.
\end{baitoan}

\begin{baitoan}[\cite{Binh_Toan_9_tap_2}, 230.., p. 17]
	2 vòi nước cùng chảy vào 1 bể trong $1$ giờ thì được $\frac{3}{10}$ bể. Nếu vòi I chảy trong $3$ giờ, vòi II chảy trong $2$ giờ thì mới được $\frac{4}{5}$ bể. Hỏi mỗi vòi chảy 1 mình thì trong bao lâu bể sẽ đầy?
\end{baitoan}

\begin{baitoan}[\cite{Binh_Toan_9_tap_2}, 231.., p. 17]
	Lúc {\rm7:00}, An khởi hành từ A để đến gặp Bích tại B lúc {\rm9:30}. Nhưng đến {\rm9:00}, An được biết Bích bắt đầu đi từ B để đến C (không nằm trên quãng đường AB) với vận tốc bằng $3.25$ lần vận tốc của An. Ngay lúc đó, An tăng thêm vận tốc {\rm1km{\tt/}h} \& khi tới B, An đã đi theo đường tắt đến C cùng 1 lúc. Nếu Bích cũng đi theo đường tắt như An thì Bích đến B trước An là $2$ giờ. Tính vận tốc lúc đầu của An.
\end{baitoan}

\begin{baitoan}[\cite{Binh_Toan_9_tap_2}, 232.., p. 17, Newton's problem]
	1 cánh đồng cỏ dày như nhau, mọc cao đều như nhau trên toàn bộ cánh đồng \& trong suốt thời gian bò ăn cỏ trên cánh đồng ấy. Biết $75$ con bò ăn hết cỏ trên {\rm60 a} đồng cỏ trong $12$ ngày, $81$ con bò ăn hết cỏ trên {\rm72 a} đồng cỏ đó trong $15$ ngày. Hỏi bao nhiêu con bò ăn hết cỏ trên {\rm96 a} đồng cỏ trong $18$ ngày? $\rm1\ a = 100\ m^2$.
\end{baitoan}

\begin{baitoan}[\cite{Binh_Toan_9_tap_2}, 233.., p. 17]
	Tìm 1 số có 2 chữ số biết nếu lấy bình phương của số đó trừ đi bình phương của số gồm chính 2 chữ số của số phải tìm viết theo thứ tự ngược lại thì được 1 số chính phương.
\end{baitoan}

\begin{baitoan}[\cite{Binh_Toan_9_tap_2}, 234.., p. 17]
	3 tổ nhân công $A,B,C$ có tuổi trung bình lần lượt là $37,23,41$. Tuổi trung bình của 2 tổ $A,B$ là $29$, tuổi trung bình của 2 tổ $B,C$ là $33$. Tính tuổi trung bình của cả 3 tổ.
\end{baitoan}

%------------------------------------------------------------------------------%

\section{Miscellaneous}
\cite[BTCCI, pp. 26--27]{SGK_Toan_9_Canh_Dieu_tap_1}: 1. 2. 3. 4. 5. 6. 7. 8. 9. 10. 11.

%------------------------------------------------------------------------------%

\printbibliography[heading=bibintoc]
	
\end{document}