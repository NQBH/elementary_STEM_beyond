\documentclass{article}
\usepackage[backend=biber,natbib=true,style=alphabetic,maxbibnames=50]{biblatex}
\addbibresource{/home/nqbh/reference/bib.bib}
\usepackage[utf8]{vietnam}
\usepackage{tocloft}
\renewcommand{\cftsecleader}{\cftdotfill{\cftdotsep}}
\usepackage[colorlinks=true,linkcolor=blue,urlcolor=red,citecolor=magenta]{hyperref}
\usepackage{amsmath,amssymb,amsthm,float,graphicx,mathtools,tikz}
\usetikzlibrary{angles,calc,intersections,matrix,patterns,quotes,shadings}
\allowdisplaybreaks
\newtheorem{assumption}{Assumption}
\newtheorem{baitoan}{}
\newtheorem{cauhoi}{Câu hỏi}
\newtheorem{conjecture}{Conjecture}
\newtheorem{corollary}{Corollary}
\newtheorem{dangtoan}{Dạng toán}
\newtheorem{definition}{Definition}
\newtheorem{dinhly}{Định lý}
\newtheorem{dinhnghia}{Định nghĩa}
\newtheorem{example}{Example}
\newtheorem{ghichu}{Ghi chú}
\newtheorem{hequa}{Hệ quả}
\newtheorem{hypothesis}{Hypothesis}
\newtheorem{lemma}{Lemma}
\newtheorem{luuy}{Lưu ý}
\newtheorem{nhanxet}{Nhận xét}
\newtheorem{notation}{Notation}
\newtheorem{note}{Note}
\newtheorem{principle}{Principle}
\newtheorem{problem}{Problem}
\newtheorem{proposition}{Proposition}
\newtheorem{question}{Question}
\newtheorem{remark}{Remark}
\newtheorem{theorem}{Theorem}
\newtheorem{vidu}{Ví dụ}
\usepackage[left=1cm,right=1cm,top=5mm,bottom=5mm,footskip=4mm]{geometry}
\def\labelitemii{$\circ$}
\DeclareRobustCommand{\divby}{%
	\mathrel{\vbox{\baselineskip.65ex\lineskiplimit0pt\hbox{.}\hbox{.}\hbox{.}}}%
}

\title{Problem: System of 1st-Order Equations -- Bài Tập: Hệ Phương Trình Bậc Nhất}
\author{Nguyễn Quản Bá Hồng\footnote{Independent Researcher, Ben Tre City, Vietnam\\e-mail: \texttt{nguyenquanbahong@gmail.com}; website: \url{https://nqbh.github.io}.}}
\date{\today}

\begin{document}
\maketitle
\tableofcontents

%------------------------------------------------------------------------------%

\section{1st-Order Equations of 2 Unknowns -- Phương Trình Bậc Nhất 2 Ẩn}

\begin{baitoan}[\cite{Binh_Toan_9_tap_2}, VD66, p. 5]
	Cho đường thẳng: $d:(m - 2)x + (m - 1)y = 1$ với tham số $m$. (a) Chứng minh đường thẳng $d$ luôn đi qua 1 điểm cố định với mọi giá trị của $m$. (b) Tìm giá trị của $m$ để khoảng cách từ gốc tọa độ O đến $d$ lớn nhất.
\end{baitoan}

\begin{baitoan}[\cite{Binh_Toan_9_tap_2}, VD67, p. 6]
	Tìm các điểm thuộc đường thẳng $3x - 5y = 8$ có tọa độ là các số nguyên \& nằm trên dải song song tạo bởi 2 đường thẳng $y = 10,y = 20$.
\end{baitoan}

\begin{baitoan}[\cite{Binh_Toan_9_tap_2}, 198., p. 8]
	Xét các đường thẳng $d$ có phương trình: $(2m + 3)x + (m + 5)y + 4m - 1 = 0$ với tham số $m$. (a) Vẽ đường thẳng $d$ ứng với $m = -1$. (b) Tìm điểm cố định mà mọi đường thẳng $d$ đều đi qua.
\end{baitoan}

\begin{baitoan}[\cite{Binh_Toan_9_tap_2}, 199., p. 8]
	Tìm các giá trị của $b,c$ để các đường thẳng $4x + by + c = 0,cx - 3y + 9 = 0$ trùng nhau.
\end{baitoan}

\begin{baitoan}[\cite{Binh_Toan_9_tap_2}, 200., p. 8]
	Vẽ đồ thị biểu diễn tập nghiệm của phương trình $x^2 - 2xy + y^2 = 1$.
\end{baitoan}

\begin{baitoan}[\cite{Binh_Toan_9_tap_2}, 201., p. 8]
	Đường thẳng $ax + by = 6$ với $a > 0,b > 0$, tạo với 2 trục tọa độ 1 tam giác có diện tích bằng $9$. Tính $ab$.
\end{baitoan}

\begin{baitoan}[\cite{Binh_Toan_9_tap_2}, 202., p. 8]
	Cho đường thẳng $d:(m + 2)x - my = -1$ với tham số $m$. (a) Tìm điểm cố định mà $d$ luôn đi qua. (b) Tìm giá trị của $m$ để khoảng cách từ gốc tọa độ O đến $d$ lớn nhất.
\end{baitoan}

\begin{baitoan}[\cite{Binh_Toan_9_tap_2}, 203., p. 8]
	Trong hệ trục tọa độ $Oxy$, $A(1,1),B(9,1)$. Viết phương trình của đường thẳng $d\bot AB$ \& chia $\Delta OAB$ thành 2 phần có diện tích bằng nhau.
\end{baitoan}

\begin{baitoan}[\cite{Binh_Toan_9_tap_2}, 204., p. 8]
	Tìm các điểm nằm trên đường thẳng $8x + 9y = -79$, có hoành độ \& tung độ là các số nguyên \& nằm bên trong góc vuông phần tư {\rm III}.
\end{baitoan}

\begin{baitoan}[\cite{Binh_Toan_9_tap_2}, 205., p. 8]
	Cho 2 điểm $A(3,17),B(33,193)$. (a) Viết phương trình của đường thẳng AB. (b) Có bao nhiêu điểm thuộc đoạn thẳng AB \& có hoành độ \& tung độ là các số nguyên?
\end{baitoan}

\begin{baitoan}[\cite{Binh_Toan_9_tap_2}, 206., p. 8]
	(a) Vẽ đồ thị hàm số $d:y = \dfrac{3}{2}x + \dfrac{7}{4}$. (b) Có bao nhiêu điểm nằm trên cạnh hoặc nằm trong tam giác tạo bởi 3 đường thẳng $x = 6,y = 0,d$.
\end{baitoan}

%------------------------------------------------------------------------------%

\section{System of 1st-Order Equations of 2 Unknowns -- Hệ Phương Trình Bậc Nhất 2 Ẩn}

\begin{baitoan}[\cite{Binh_Toan_9_tap_2}, VD68, p. 9]
	Cho hệ phương trình với tham số $a$:
	\begin{equation*}
		\left\{\begin{split}
			(a + 1)x - y &= a + 1,\\
			x + (a - 1)y &= 2.
		\end{split}\right.
	\end{equation*}
	(a) Giải hệ phương trình với $a = 2$. (b) Giải \& biện luận hệ phương trình. (c) Tìm các giá trị nguyên của $a$ để hệ phương trình có nghiệm nguyên. (d) Tìm các giá trị nguyên của $a$ để nghiệm của hệ phương trình thỏa mãn điều kiện $x + y$ nhỏ nhất.
\end{baitoan}

\begin{baitoan}[\cite{Binh_Toan_9_tap_2}, VD69, p. 10]
	Tìm $a,b,c\in\mathbb{Z}$ thỏa mãn cả 2 phương trình $2a + 3b = 6,3a + 4c = 1$.
\end{baitoan}

\begin{baitoan}[\cite{Binh_Toan_9_tap_2}, VD70, p. 10]
	Cho 2 đường thẳng: $d:2x - 3y = 4,d':3x + 5y = 2$. Tìm trên trục $Ox$ điểm có hoành độ là số nguyên dương nhỏ nhất, sao cho nếu qua điểm đó ta dựng đường vuông góc với $Ox$ thì đường vuông góc ấy cắt 2 đường thẳng $d,d'$ tại 2 điểm có tọa độ là các số nguyên.
\end{baitoan}

\begin{baitoan}[\cite{Binh_Toan_9_tap_2}, VD71, p. 11]
	Giải hệ phương trình với 3 ẩn $x,y,z$ \& các tham số $a,b,c$ khác nhau đôi một:
	\begin{equation*}
		\left\{\begin{split}
			a^2x + ay + z &= 5,\\
			b^2x + by + z &= 5,\\
			c^2x + cy + z &= 5.
		\end{split}\right.
	\end{equation*}
\end{baitoan}
Giải hệ phương trình:

\begin{baitoan}[\cite{Binh_Toan_9_tap_2}, 207., p. 12]
	\begin{equation*}
		\left\{\begin{split}
			(x + 3)(y - 5) &= xy,\\
			(x - 2)(y + 5) &= xy,
		\end{split}\right.\hspace{1cm} \left\{\begin{split}
			\frac{1}{x}	+ \frac{1}{y} &= \frac{3}{4},\\
			\frac{1}{6x} + \frac{1}{5y} &= \frac{2}{15}.
		\end{split}\right. 
	\end{equation*}
\end{baitoan}

\begin{baitoan}[\cite{Binh_Toan_9_tap_2}, 208., p. 12]
	\begin{equation*}
		\left\{\begin{split}
			\frac{x}{y} - \frac{x}{y + 12} &= 1,\\
			\frac{x}{y - 12} - \frac{x}{y} &= 2,
		\end{split}\right.\hspace{1cm} \left\{\begin{split}
			4(x + y) &= 5(x - y),\\
			\frac{40}{x + y} + \frac{40}{x - y} &= 9.
		\end{split}\right. 
	\end{equation*}
\end{baitoan}

\begin{baitoan}[\cite{Binh_Toan_9_tap_2}, 209., p. 12]
	\begin{equation*}
		\left\{\begin{split}
			|x - 2| + 2|y - 1| &= 9,\\
			x + |y - 1| &= -1,
		\end{split}\right.\hspace{1cm} \left\{\begin{split}
			x + y + |x| &= 25,\\
			x - y + |y| &= 30.
		\end{split}\right. 
	\end{equation*}
\end{baitoan}

\begin{baitoan}[\cite{Binh_Toan_9_tap_2}, 210., p. 12]
	Tìm các giá trị của $a\in\mathbb{R}$ để 2 hệ phương trình tương đương:
	\begin{equation*}
		\left\{\begin{split}
			2x + 3y &= 8,\\
			3x - y &= 1,
		\end{split}\right.\hspace{1cm} \left\{\begin{split}
			ax - 3y &= -2,\\
			x + y &= 3.
		\end{split}\right. 
	\end{equation*}
\end{baitoan}

\begin{baitoan}[\cite{Binh_Toan_9_tap_2}, 211., p. 12]
	Tìm các giá trị của $m\in\mathbb{R}$ để nghiệm của hệ phương trình sau là 2 số dương:
	\begin{equation*}
		\left\{\begin{split}
			x - y &= 2,\\
			mx + y &= 3.
		\end{split}\right.
	\end{equation*}
\end{baitoan}

\begin{baitoan}[\cite{Binh_Toan_9_tap_2}, 212., p. 12]
	Chứng minh tam giác tạo bởi 3 đường thẳng $y = 3x - 2,y = -\dfrac{1}{3}x + \frac{4}{3},y = -2x + 8$ là tam giác vuông cân.
\end{baitoan}

\begin{baitoan}[\cite{Binh_Toan_9_tap_2}, 213., p. 13]
	Tìm các giá trị của $m\in\mathbb{R}$ để hệ phương trình sau vô nghiệm, vô số nghiệm:
	\begin{equation*}
		\left\{\begin{split}
			2(m + 1)x + (m + 2)y &= m - 3,\\
			(m + 1)x + my &= 3m + 7.
		\end{split}\right.
	\end{equation*}
\end{baitoan}

\begin{baitoan}[\cite{Binh_Toan_9_tap_2}, 214., p. 13]
	Cho hệ phương trình với tham số $m$:
	\begin{equation*}
		\left\{\begin{split}
			mx + 2y &= 1,\\
			3x + (m + 1)y &= -1.
		\end{split}\right.
	\end{equation*}
	(a) Giải hệ phương trình với $m = 3$. (b) Giải \& biện luận hệ phương trình theo $m$. (c) Tìm các giá trị nguyên của $m$ để nghiệm của hệ phương trình là các số nguyên.
\end{baitoan}

\begin{baitoan}[\cite{Binh_Toan_9_tap_2}, 215., p. 13]
	Cho hệ phương trình với tham số $m$:
	\begin{equation*}
		\left\{\begin{split}
			(m - 1)x + y &= 3m - 4,\\
			x + (m - 1)y &= m.
		\end{split}\right.
	\end{equation*}
	(a) Giải \& biện luận hệ phương trình theo $m$. (b) Tìm các giá trị nguyên của $m$ để nghiệm của hệ phương trình là các số nguyên. (c) Tìm các giá trị của $m$ để hệ phương trình có nghiệm dương duy nhất.
\end{baitoan}

\begin{baitoan}[\cite{Binh_Toan_9_tap_2}, 216., p. 13]
	Cho hệ phương trình với tham số $m$:
	\begin{equation*}
		\left\{\begin{split}
			x + my &= m + 1,\\
			mx + y &= 3m - 1.
		\end{split}\right.
	\end{equation*}
	(a) Giải \& biện luận hệ phương trình theo $m$. (b) Trong trường hợp hệ có nghiệm duy nhất, tìm các giá trị của $m$ để tích $xy$ nhỏ nhất.
\end{baitoan}

\begin{baitoan}[\cite{Binh_Toan_9_tap_2}, 217., p. 13]
	Các số không âm $x,y,z$ thỏa mãn hệ phương trình:
	\begin{equation*}
		\left\{\begin{split}
			4x - 4y + 2z &= 1,\\
			8x + 4y + z &= 8.
		\end{split}\right.
	\end{equation*}
	(a) Biểu thị $x,y$ theo $z$. (b) Tìm {\rm GTNN, GTLN} của biểu thức $A = x + y - z$.
\end{baitoan}

\begin{baitoan}[\cite{Binh_Toan_9_tap_2}, 218., p. 13]
	Tìm $a,b,c\in\mathbb{Z}$ thỏa mãn hệ phương trình:
	\begin{equation*}
		\left\{\begin{split}
			2a + 3b &= 5,\\
			3a - 4c &= 6.
		\end{split}\right.
	\end{equation*}
\end{baitoan}

\begin{baitoan}[\cite{Binh_Toan_9_tap_2}, 219., p. 14]
	Tìm trên trục tung các điểm có tung độ là số nguyên, sao cho nếu qua điểm đó ta dựng đường vuông góc với trục tung thì đường vuông góc ấy cắt 2 đường thẳng: $d:x + 2y = 6,d':2x - 3y = 4$ tại các điểm có tọa độ là các số nguyên.
\end{baitoan}

\begin{baitoan}[\cite{Binh_Toan_9_tap_2}, 220., p. 14]
	Tìm trên trục hoành các điểm có hoành độ là số nguyên sao cho nếu qua điểm đó ta dựng đường thẳng vuông góc với trục hoành thì đường vuông góc ấy cắt cả 3 đường thẳng sau tại các điểm có tọa độ là các số nguyên: $d_1:x - 2y = 3,d_2:x - 3y = 2,d_3:x - 5y = -7$.
\end{baitoan}
Giải hệ phương trình ẩn $x,y,z$:

\begin{baitoan}[\cite{Binh_Toan_9_tap_2}, 221., p. 14]
	\begin{equation*}
		\left\{\begin{split}
			x + y + z &= 11,\\
			2x - y + z &= 5,\\
			3x + 2y + z &= 14,
		\end{split}\right.\hspace{1cm}\left\{\begin{split}
			x + y + z + t &= 4,\\
			x + y - z - t &= 8,\\
			x - y + z - t &= 12,\\
			x - y - z + y &= 16.
		\end{split}\right.
	\end{equation*}
\end{baitoan}

\begin{baitoan}[\cite{Binh_Toan_9_tap_2}, 222., p. 14]
	\begin{equation*}
		\left\{\begin{split}
			x + y + z &= 12,\\
			ax + 5y + 4z &= 46,\\
			5x + ay + 3z &= 38,
		\end{split}\right.\hspace{1cm}\left\{\begin{split}
			ax + y + z &= a^2,\\
			x + ay + z &= 3a,\\
			x + y + az &= 2.
		\end{split}\right.
	\end{equation*}
\end{baitoan}

\begin{baitoan}[\cite{Binh_Toan_9_tap_2}, 223., p. 14]
	$a,b,c\in\mathbb{R}$ là tham số, $a + b + c\ne0$.
	\begin{equation*}
		\left\{\begin{split}
			(a + b)(x + y) - cz &= a - b,\\
			(b + c)(y + z) - ax &= b - c,\\
			(c + a)(z + x) - by &= c - a.
		\end{split}\right.
	\end{equation*}
\end{baitoan}

\begin{baitoan}[\cite{Binh_Toan_9_tap_2}, 224., p. 14]
	Giải hệ phương trình với 3 tham số $a,b,c\in\mathbb{R}$ đôi một khác nhau, $a + b + c\ne0$:
	\begin{equation*}
		\left\{\begin{split}
			ax + by + cz &= 0,\\
			bx + cy + az &= 0,\\
			cx + ay + bz &= 0,
		\end{split}\right.\hspace{1cm}\left\{\begin{split}
			ax + by + cz &= a + b + c,\\
			bx + cy + az &= a + b + c,\\
			cx + ay + bz &= a + b + c.
		\end{split}\right.
	\end{equation*}
\end{baitoan}

\begin{baitoan}[\cite{Binh_Toan_9_tap_2}, 225., p. 14]
	\begin{equation*}
		\left\{\begin{split}
			x^2 + xy + xz &= 2,\\
			y^2 + yz + xy &= 3,\\
			z^2 + xz + yz &= 4.
		\end{split}\right.
	\end{equation*}
\end{baitoan}

%------------------------------------------------------------------------------%

\section{Giải Bài Toán Bằng Cách Lập Hệ Phương Trình}

%------------------------------------------------------------------------------%

\section{Miscellaneous}

%------------------------------------------------------------------------------%

\printbibliography[heading=bibintoc]
	
\end{document}