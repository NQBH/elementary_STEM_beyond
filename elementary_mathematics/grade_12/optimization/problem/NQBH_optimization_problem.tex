\documentclass{article}
\usepackage[backend=biber,natbib=true,style=alphabetic,maxbibnames=50]{biblatex}
\addbibresource{/home/nqbh/reference/bib.bib}
\usepackage[utf8]{vietnam}
\usepackage{tocloft}
\renewcommand{\cftsecleader}{\cftdotfill{\cftdotsep}}
\usepackage[colorlinks=true,linkcolor=blue,urlcolor=red,citecolor=magenta]{hyperref}
\usepackage{amsmath,amssymb,amsthm,float,graphicx,mathtools,tikz}
\usetikzlibrary{angles,calc,intersections,matrix,patterns,quotes,shadings}
\allowdisplaybreaks
\newtheorem{assumption}{Assumption}
\newtheorem{baitoan}{}
\newtheorem{cauhoi}{Câu hỏi}
\newtheorem{conjecture}{Conjecture}
\newtheorem{corollary}{Corollary}
\newtheorem{dangtoan}{Dạng toán}
\newtheorem{definition}{Definition}
\newtheorem{dinhly}{Định lý}
\newtheorem{dinhnghia}{Định nghĩa}
\newtheorem{example}{Example}
\newtheorem{ghichu}{Ghi chú}
\newtheorem{hequa}{Hệ quả}
\newtheorem{hypothesis}{Hypothesis}
\newtheorem{lemma}{Lemma}
\newtheorem{luuy}{Lưu ý}
\newtheorem{nhanxet}{Nhận xét}
\newtheorem{notation}{Notation}
\newtheorem{note}{Note}
\newtheorem{principle}{Principle}
\newtheorem{problem}{Problem}
\newtheorem{proposition}{Proposition}
\newtheorem{question}{Question}
\newtheorem{remark}{Remark}
\newtheorem{theorem}{Theorem}
\newtheorem{vidu}{Ví dụ}
\usepackage[left=1cm,right=1cm,top=5mm,bottom=5mm,footskip=4mm]{geometry}
\def\labelitemii{$\circ$}
\DeclareRobustCommand{\divby}{%
	\mathrel{\vbox{\baselineskip.65ex\lineskiplimit0pt\hbox{.}\hbox{.}\hbox{.}}}%
}
\def\labelitemii{$\circ$}

\title{Problem: Mathematical Optimization -- Bài Tập: Ứng Dụng Toán Học Để Giải Quyết 1 Số Bài Toán Tối Ưu}
\author{Nguyễn Quản Bá Hồng\footnote{A Scientist {\it\&} Creative Artist Wannabe. E-mail: {\tt nguyenquanbahong@gmail.com}. Bến Tre City, Việt Nam.}}
\date{\today}

\begin{document}
\maketitle
\begin{abstract}
	This text is a part of the series {\it Some Topics in Elementary STEM \& Beyond}:
	
	{\sc url}: \url{https://nqbh.github.io/elementary_STEM}.
	
	Latest version:
	\begin{itemize}
		\item {\it Problem: Mathematical Optimization -- Bài Tập: Ứng Dụng Toán Học Để Giải Quyết 1 Số Bài Toán Tối Ưu}.
		
		PDF: {\sc url}: \url{https://github.com/NQBH/elementary_STEM_beyond/blob/main/elementary_mathematics/grade_12/optimization/problem/NQBH_optimization_problem.pdf}.
		
		\TeX: {\sc url}: \url{https://github.com/NQBH/elementary_STEM_beyond/blob/main/elementary_mathematics/grade_12/optimization/problem/NQBH_optimization_problem.tex}.
		\item {\it Problem \& Solution: Mathematical Optimization -- Bài Tập \& Lời Giải: Ứng Dụng Toán Học Để Giải Quyết 1 Số Bài Toán Tối Ưu}.
		
		PDF: {\sc url}: \url{https://github.com/NQBH/elementary_STEM_beyond/blob/main/elementary_mathematics/grade_12/optimization/solution/NQBH_optimization_solution.pdf}.
		
		\TeX: {\sc url}: \url{https://github.com/NQBH/elementary_STEM_beyond/blob/main/elementary_mathematics/grade_12/optimization/solution/NQBH_optimization_solution.tex}.
	\end{itemize}
\end{abstract}
\tableofcontents

%------------------------------------------------------------------------------%

\section{Application of System of Linear Inequations to Solve Some Linear Programming Problems -- Vận Dụng Hệ Bất Phương Trình Bậc Nhất Để Giải Quyết 1 Số Bài Toán Quy Hoạch Tuyến Tính}

\begin{definition}[Linear programming]
	``\emph{Linear programming (LP)}, also called \emph{linear optimization}, is a method to achieve the best outcome, e.g., maximum profit or lower cost, in a \href{https://en.wikipedia.org/wiki/Mathematical_model}{mathematical model} whose requirements \& objective are represented by \href{https://en.wikipedia.org/wiki/Linear_function#As_a_polynomial_function}{linear relationships}. Linear programming is a special case of mathematical programming $\equiv$ \href{https://en.wikipedia.org/wiki/Mathematical_optimization}{mathematical optimization}.'' -- \href{https://en.wikipedia.org/wiki/Linear_programming}{Wikipedia{\tt/}linear programming}
\end{definition}
More formally, linear programming is a technique for the \href{https://en.wikipedia.org/wiki/Mathematical_optimization}{optimization} of a linear \href{https://en.wikipedia.org/wiki/Objective_function}{linear objective function}, subject to \href{https://en.wikipedia.org/wiki/Linear_equality}{linear equality} \& \href{https://en.wikipedia.org/wiki/Linear_inequality}{linear inequality} \href{https://en.wikipedia.org/wiki/Constraint_(mathematics)}{constraints}. Its \href{https://en.wikipedia.org/wiki/Feasible_region}{feasible region} is a \href{https://en.wikipedia.org/wiki/Convex_polytope}{convex polytope}, which is a set defined as the \href{https://en.wikipedia.org/wiki/Intersection_(mathematics)}{intersection} of finitely many \href{https://en.wikipedia.org/wiki/Half-space_(geometry)}{half spaces}, each of which is defined by a linear inequality. Its objective function is a real-valued \href{https://en.wikipedia.org/wiki/Affine_function}{affine (linear) function} defined on this polytope. A linear programming \href{https://en.wikipedia.org/wiki/Algorithm}{algorithm} finds a point in the \href{https://en.wikipedia.org/wiki/Polytope}{polytope} where this function has the largest (or smallest) value if such a point exists.

Linear programs are problems that can be expressed in \href{https://en.wikipedia.org/wiki/Canonical_form}{standard form} as
\begin{equation}
	\label{linear program}
	\tag{lp}
	\mbox{Find a vector }{\bf x}\mbox{ that maximizes{\tt/}minimizes }{\bf c}^\top{\bf x}\mbox{ subject to } A{\bf x}\le{\bf b}\mbox{ \& }{\bf x}\ge{\bf 0}.
\end{equation}
Here the components of ${\bf x}$ are the variables to be determined, ${\bf b},{\bf c}$ are given vectors, \& $A$ is a given matrix. The function whose value is to be maximized (${\bf x}\mapsto{\bf c}^\top{\bf x}$ in this case) is called the \href{https://en.wikipedia.org/wiki/Objective_function}{objective function}. The constraint $A{\bf x}\le{\bf x}$ \& ${\bf x}\ge{\bf 0}$ specify a \href{https://en.wikipedia.org/wiki/Convex_polytope}{convex polytope} over which the objective function is to be optimized.

Linear programming can be applied to various fields of study, which is widely used in mathematics \&, to a lesser extent, in business, economics, \& to some engineering problems. There is a close connection between linear programs, eigenequations, \href{https://en.wikipedia.org/wiki/John_von_Neumann}{John von Neumann}'s general equilibrium model, \& structural equilibrium models (see \href{https://en.wikipedia.org/wiki/Dual_linear_program}{dual linear program}). Industries using linear programming models include transportation, energy, telecommunications, \& manufacturing. It has proven useful in modeling diverse types of problems in \href{https://en.wikipedia.org/wiki/Automated_planning_and_scheduling}{planning}, \href{https://en.wikipedia.org/wiki/Routing}{routing}, \href{https://en.wikipedia.org/wiki/Scheduling_(production_processes)}{scheduling}, \href{https://en.wikipedia.org/wiki/Assignment_problem}{assignment}, \& design.

\begin{dinhnghia}[Quy hoạch tuyến tính]
	Bài toán \emph{quy hoạch tuyến tính} là bài toán tìm {\rm GTLN{\tt/}GTNN} của \emph{hàm mục tiêu} trong điều kiện hàm mục tiêu là hàm bậc nhất đối với các biến \& mỗi 1 điều kiện ràng buộc là bất phương trình bậc nhất đối với các biến (không kể điều kiện ràng buộc biến thuộc tập số nào, e.g., $\mathbb{N},\mathbb{Q},\mathbb{R},\mathbb{C}$.
\end{dinhnghia}

%------------------------------------------------------------------------------%

\section{Miscellaneous}

%------------------------------------------------------------------------------%

\printbibliography[heading=bibintoc]
	
\end{document}