\documentclass{article}
\usepackage[backend=biber,natbib=true,style=alphabetic,maxbibnames=50]{biblatex}
\addbibresource{/home/nqbh/reference/bib.bib}
\usepackage[utf8]{vietnam}
\usepackage{tocloft}
\renewcommand{\cftsecleader}{\cftdotfill{\cftdotsep}}
\usepackage[colorlinks=true,linkcolor=blue,urlcolor=red,citecolor=magenta]{hyperref}
\usepackage{amsmath,amssymb,amsthm,float,graphicx,mathtools,tikz}
\usetikzlibrary{angles,calc,intersections,matrix,patterns,quotes,shadings}
\allowdisplaybreaks
\newtheorem{assumption}{Assumption}
\newtheorem{baitoan}{}
\newtheorem{cauhoi}{Câu hỏi}
\newtheorem{conjecture}{Conjecture}
\newtheorem{corollary}{Corollary}
\newtheorem{dangtoan}{Dạng toán}
\newtheorem{definition}{Definition}
\newtheorem{dinhly}{Định lý}
\newtheorem{dinhnghia}{Định nghĩa}
\newtheorem{example}{Example}
\newtheorem{ghichu}{Ghi chú}
\newtheorem{hequa}{Hệ quả}
\newtheorem{hypothesis}{Hypothesis}
\newtheorem{lemma}{Lemma}
\newtheorem{luuy}{Lưu ý}
\newtheorem{nhanxet}{Nhận xét}
\newtheorem{notation}{Notation}
\newtheorem{note}{Note}
\newtheorem{principle}{Principle}
\newtheorem{problem}{Problem}
\newtheorem{proposition}{Proposition}
\newtheorem{question}{Question}
\newtheorem{remark}{Remark}
\newtheorem{theorem}{Theorem}
\newtheorem{vidu}{Ví dụ}
\usepackage[left=1cm,right=1cm,top=5mm,bottom=5mm,footskip=4mm]{geometry}
\def\labelitemii{$\circ$}
\DeclareRobustCommand{\divby}{%
	\mathrel{\vbox{\baselineskip.65ex\lineskiplimit0pt\hbox{.}\hbox{.}\hbox{.}}}%
}
\def\labelitemii{$\circ$}

\title{Problem: Applications in Mathematical Finance\\Bài Tập: Ứng Dụng Toán Học Trong 1 Số Vấn Đề Liên Quan Đến Tài Chính}
\author{Nguyễn Quản Bá Hồng\footnote{A Scientist {\it\&} Creative Artist Wannabe. E-mail: {\tt nguyenquanbahong@gmail.com}. Bến Tre City, Việt Nam.}}
\date{\today}

\begin{document}
\maketitle
\begin{abstract}
	This text is a part of the series {\it Some Topics in Elementary STEM \& Beyond}:
	
	{\sc url}: \url{https://nqbh.github.io/elementary_STEM}.
	
	Latest version:
	\begin{itemize}
		\item {\it Problem: Applications in Mathematical Finance -- Bài Tập: Ứng Dụng Toán Học Trong 1 Số Vấn Đề Liên Quan Đến Tài Chính}.
		
		PDF: {\sc url}: \url{https://github.com/NQBH/elementary_STEM_beyond/blob/main/elementary_mathematics/grade_12/finance/problem/NQBH_finance_problem.pdf}.
		
		\TeX: {\sc url}: \url{https://github.com/NQBH/elementary_STEM_beyond/blob/main/elementary_mathematics/grade_12/finance/problem/NQBH_finance_problem.tex}.
		\item {\it Problem \& Solution: Applications in Mathematical Finance -- Bài Tập \& Lời Giải: Ứng Dụng Toán Học Trong 1 Số Vấn Đề Liên Quan Đến Tài Chính}.
		
		PDF: {\sc url}: \url{https://github.com/NQBH/elementary_STEM_beyond/blob/main/elementary_mathematics/grade_12/finance/solution/NQBH_finance_solution.pdf}.
		
		\TeX: {\sc url}: \url{https://github.com/NQBH/elementary_STEM_beyond/blob/main/elementary_mathematics/grade_12/finance/solution/NQBH_finance_solution.tex}.
	\end{itemize}
\end{abstract}
\tableofcontents

%------------------------------------------------------------------------------%

\section{Some Problems on Money, Rate -- 1 Số Vấn Đề Về Tiền Tệ, Lãi Suất}

\subsection{Khái niệm về tiền tệ}
Giá trị của mỗi loại hàng hóa được đo lường bằng giá trị của tiền tệ. Tiền tệ dùng để đo lường giá trị của các loại hàng hóa.

\begin{definition}
	\emph{Money} is any item or verifiable record that is generally accepted as \href{https://en.wikipedia.org/wiki/Payment}{payment} for \href{https://en.wikipedia.org/wiki/Goods_and_services}{goods \& services} \& repayment of \href{https://en.wikipedia.org/wiki/Debt}{debts}, e.g. \href{https://en.wikipedia.org/wiki/Taxes}{taxes}, in a particular country or socio-economic context. The primary functions which distinguish money are: \href{https://en.wikipedia.org/wiki/Medium_of_exchange}{medium of exchange}, a \href{https://en.wikipedia.org/wiki/Unit_of_account}{unit of account}, a \href{https://en.wikipedia.org/wiki/Store_of_value}{store of value}, \& sometimes, a \href{https://en.wikipedia.org/wiki/Standard_of_deferred_payment}{standard of deferred payment}.'' -- \href{https://en.wikipedia.org/wiki/Money}{Wikipedia{\tt/}money}
\end{definition}
``Money was historically an \href{https://en.wikipedia.org/wiki/Emergence#Economics}{emergent market phenomenon} that possessed intrinsic value as a \href{https://en.wikipedia.org/wiki/Commodity_money}{commodity}; nearly all contemporary money systems are based on unbacked \href{https://en.wikipedia.org/wiki/Fiat_money}{fiat money} without \href{https://en.wikipedia.org/wiki/Use_value}{use value}. Its value is consequently derived by social convention, having been declared by a \href{https://en.wikipedia.org/wiki/Government}{government} or regulatory entity to be \href{https://en.wikipedia.org/wiki/Legal_tender}{legal tender}; i.e., it must be accepted as a form of payment within the boundaries of the country, for ``all debts, public \& private,'' in the case of the \href{https://en.wikipedia.org/wiki/United_States_dollar}{United States dollar}.
	
The \href{https://en.wikipedia.org/wiki/Money_supply}{money supply} of a country comprises all \href{https://en.wikipedia.org/wiki/Currency_in_circulation}{currency in circulation} (\href{https://en.wikipedia.org/wiki/Banknote}{banknotes} \& \href{https://en.wikipedia.org/wiki/Coin}{coins} currently issued) \&, depending on the particular definition used, 1 or more types of \href{https://en.wikipedia.org/wiki/Demand_deposit}{bank money} (the balances held in \href{https://en.wikipedia.org/wiki/Transactional_account}{checking accounts}, \href{https://en.wikipedia.org/wiki/Savings_account}{saving accounts}, \& other types of \href{https://en.wikipedia.org/wiki/Bank_accounts}{bank accounts}). Bank money, whose value exists on the books of financial institutions \& can be converted into physical notes or used for cashless payment, forms by far the largest part of \href{https://en.wikipedia.org/wiki/Broad_money}{broad money} in developed countries.'' -- \href{https://en.wikipedia.org/wiki/Money}{Wikipedia{\tt/}money}

\begin{dinhnghia}[Tiền tệ]
	\emph{Tiền tệ} là phương tiện trao đổi hàng hóa \& dịch vụ được chấp nhận thanh toán trong 1 khu vực nhất định hoặc giữa 1 nhóm người cụ thể. Bản thân tiền tệ không thực sự có giá trị mà thay vào đó chúng có được giá trị từ sự chấp nhận chung từ mọi người ở 1 khu vực nhất định hoặc giữa 1 nhóm người cụ thể trong thanh toán để đổi lấy hàng hóa, dịch vụ, \& hoàn trả các khoản nợ.
\end{dinhnghia}
Tiền tệ là {\it vật trung gian môi giới} trong trao đổi hàng hóa, dịch vụ, là {\it phương tiện} giúp cho quá trình trao đổi được thực hiện dễ dàng hơn. Bản chất của tiền tệ được thể hiện rõ hơn qua 2 thuộc tính:
\begin{itemize}
	\item {\it Giá trị sử dụng của tiền tệ} thường được hiểu là khả năng thỏa mãn nhu cầu trao đổi của xã hội, nhu cầu sử dụng làm vật trung gian trong trao đổi. Giá trị sử dụng của loại tiền tệ là do xã hội quy định.
	\item {\it Giá trị của tiền tệ} thường được hiểu là khả năng đổi được nhiều hay ít hàng hóa trong trao đổi.
\end{itemize}
Trong mỗi quốc gia, tiền tệ có 4 chức năng cơ bản: phương tiện trao đổi, phương tiện đo lường \& tính toán giá trị, phương tiện thanh toán, phương tiện tích lũy.

{\bf Personal experiment.} Lúc còn ở châu Âu, tôi thường dùng \href{https://wise.com/}{Wise}, {\sc url}: \url{https://wise.com/}, trước đây gọi là TransferWise, 1 công ty cộng nghệ tài chính tập trung vào chuyển tiền toàn cầu có trụ sở chính tại London \& tài khoản ngân hàng ở Bỉ (Belgium), để chuyển tiền về cho gia đình. Bạn có thể xem các tỷ giá hối đoái được cập nhật liên tục ở trang web của Wise. Có nhiều ứng dụng tương tự, nhưng thời điểm 2020, lúc ứng dụng còn gọi là TransferWise, sau đó 1--2 năm mới đổi tên thành Wise, tôi chuộng dùng Wise. Để tiết kiệm \& tối ưu nguồn tiền, bạn nên tìm hiểu kỹ thêm.

%------------------------------------------------------------------------------%

\subsection{Definition of Exchange Rate \& How to Compute It -- Khái Niệm \& Cách Tính Lãi Suất}

\begin{definition}[Exchange rate]
	``In \href{https://en.wikipedia.org/wiki/Finance}{finance}, an \emph{exchange rate} is the rate at which 1 \href{https://en.wikipedia.org/wiki/Currency}{currency} will be exchanged for another currency. Currencies are most commonly national currencies, but may be sub-national as in the case of Hong Kong or supra-national as in the case of the \href{https://en.wikipedia.org/wiki/Euro}{euro \texteuro}. The exchange rate is also regarded as the value of one country's currency in relation to another currency.'' -- \href{https://en.wikipedia.org/wiki/Exchange_rate}{Wikipedia{\tt/}exchange rate}
\end{definition}

\begin{dinhnghia}[Lãi suất]
	\emph{Lãi suất} là tỷ lệ phần trăm của tiền gửi vốn vào ngân hàng mà ngân hàng có trách nhiệm phải trả cho người gửi tiền trong 1 khoảng thời gian đã xác định, thông thường được tính theo năm.
\end{dinhnghia}

\begin{dinhly}[Công thức lãi kép]
	Nếu 1 khoản tiền gốc $A\in(0,\infty)$ được gửi tiết kiệm (theo thể thức lãi kép) với lãi suất $r\in(0,\infty)$ mỗi kỳ ($r$ được biểu thị dưới dạng số phập phân) thì tổng số tiền $S$ nhận được, cả vốn lẫn lãi, sau $n\in\mathbb{N}$ kỳ gửi cho bởi \emph{công thức lãi kép} $S(n) = A(1 + r)^n$.
\end{dinhly}

\begin{dinhly}[Công thức lãi kép]
	Nếu 1 khoản tiền gốc $A\in(0,\infty)$ được gửi tiết kiệm (theo thể thức lãi kép) với lãi suất hàng năm $r\in(0,\infty)$ ($r$ được biểu thị dưới dạng số phập phân), được tính lãi $k\in\mathbb{N}^\star$ trong 1 năm, thì tổng số tiền $S$ nhận được, cả vốn lẫn lãi, sau $n\in\mathbb{N}$ kỳ gửi cho bởi \emph{công thức lãi kép} $S(n) = A\left(1 + \frac{r}{k}\right)^n$.
\end{dinhly}

\subsection{Definition of Inflation. Inflation Index -- Khái Niệm Về Lạm Phát. Chỉ Số Lạm Phát}
Khi giá của hàng hóa tăng cao, 1 đơn vị tiền tệ sẽ mua được ít hàng hóa hơn hay tiền tệ bị giảm giá trị.

\begin{dinhnghia}[Lạm phát]
	\emph{Lạm phát} là hiện tượng tăng giá liên tục của hàng hóa, dịch vụ dẫn đến giảm sức mua của đồng tiền. Có thể hiểu 1 cách đơn giản, khi lạm phát xảy ra, với cùng 1 số tiền người ta chỉ có thể mua được 1 số lượng hàng hóa, dịch vụ ít hơn so với trước đây. Do đó, lạm phát phản ánh sự suy giảm sức mua trên 1 đơn vị tiền tệ.
\end{dinhnghia}
Để đo lường mức độ lạm phát của 1 thời kỳ, các nhà kinh tế học đưa ra {\it chỉ số lạm phát} (inflation index) của thời kỳ đó. 

\cite[Chuyên đề II, \S3, LT1--3, 1., 2., 3., 4., 5., pp. 20--25]{CDHT_Toan_12_Canh_Dieu}.

%------------------------------------------------------------------------------%

\section{Credit. Debit -- Tín Dụng. Vay Nợ}

%------------------------------------------------------------------------------%

\section{Miscellaneous}

%------------------------------------------------------------------------------%

\printbibliography[heading=bibintoc]
	
\end{document}