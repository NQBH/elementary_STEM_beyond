\documentclass{article}
\usepackage[backend=biber,natbib=true,style=authoryear,maxbibnames=10]{biblatex}
\addbibresource{/home/nqbh/reference/bib.bib}
\usepackage[utf8]{vietnam}
\usepackage{tocloft}
\renewcommand{\cftsecleader}{\cftdotfill{\cftdotsep}}
\usepackage[colorlinks=true,linkcolor=blue,urlcolor=red,citecolor=magenta]{hyperref}
\usepackage{amsmath,amssymb,amsthm,float,graphicx,mathtools}
\allowdisplaybreaks
\newtheorem{assumption}{Assumption}
\newtheorem{baitoan}{Bài toán}
\newtheorem{cauhoi}{Câu hỏi}
\newtheorem{conjecture}{Conjecture}
\newtheorem{corollary}{Corollary}
\newtheorem{dangtoan}{Dạng toán}
\newtheorem{definition}{Definition}
\newtheorem{dinhly}{Định lý}
\newtheorem{dinhnghia}{Định nghĩa}
\newtheorem{example}{Example}
\newtheorem{ghichu}{Ghi chú}
\newtheorem{hequa}{Hệ quả}
\newtheorem{hypothesis}{Hypothesis}
\newtheorem{lemma}{Lemma}
\newtheorem{luuy}{Lưu ý}
\newtheorem{nhanxet}{Nhận xét}
\newtheorem{notation}{Notation}
\newtheorem{note}{Note}
\newtheorem{principle}{Principle}
\newtheorem{problem}{Problem}
\newtheorem{proposition}{Proposition}
\newtheorem{question}{Question}
\newtheorem{remark}{Remark}
\newtheorem{theorem}{Theorem}
\newtheorem{vidu}{Ví dụ}
\usepackage[left=1cm,right=1cm,top=5mm,bottom=5mm,footskip=4mm]{geometry}
\def\labelitemii{$\circ$}
\DeclareRobustCommand{\divby}{%
	\mathrel{\vbox{\baselineskip.65ex\lineskiplimit0pt\hbox{.}\hbox{.}\hbox{.}}}%
}

\title{Problem: Application of Derivative to Survey \& Draw Graph of Functions\\Bài Tập: Ứng Dụng Đạo Hàm Để Khảo Sát \& Vẽ Đồ Thị của Hàm Số}
\author{Nguyễn Quản Bá Hồng\footnote{Independent Researcher, Ben Tre City, Vietnam\\e-mail: \texttt{nguyenquanbahong@gmail.com}; website: \url{https://nqbh.github.io}.}}
\date{\today}

\begin{document}
\maketitle
\tableofcontents

%------------------------------------------------------------------------------%

\section{Tính Đơn Điệu của Hàm Số}

%------------------------------------------------------------------------------%

\begin{baitoan}[\cite{SGK_Toan_12_giai_tich_nang_cao}, Ví dụ 1, p. 5]
	Chứng minh hàm số $f(x) = \sqrt{1 - x^2}$ nghịch biến trên đoạn $[0,1]$.
\end{baitoan}

\begin{baitoan}[\cite{SGK_Toan_12_giai_tich_nang_cao}, Ví dụ 2, p. 6]
	Xét chiều biến thiên của hàm số $y = x + \frac{4}{x}$.
\end{baitoan}

\begin{baitoan}[\cite{SGK_Toan_12_giai_tich_nang_cao}, H1, p. 6]
	Xét chiều biến thiên của hàm số $y = \frac{1}{3}x^3 - \frac{3}{2}x^2 + 2x - 3$.
\end{baitoan}

\begin{baitoan}[\cite{SGK_Toan_12_giai_tich_nang_cao}, Ví dụ 3, p. 6]
	Xét chiều biến thiên của hàm số $y = \frac{4}{3}x^3 - 2x^2 + x - 3$.
\end{baitoan}

\begin{baitoan}[\cite{SGK_Toan_12_giai_tich_nang_cao}, H2, p. 7]
	Xét chiều biến thiên của hàm số $y = 2x^5 + 5x^4 + \frac{10}{3}x^3 - \frac{7}{3}$.
\end{baitoan}

\begin{baitoan}[\cite{SGK_Toan_12_giai_tich_nang_cao}, 1., p. 7]
	Xét chiều biến thiên của hàm số: (a) $y = 2x^3 + 3x^2 + 1$. (b) $y = x^3 - 2x^2 + x + 1$. (c) $y = x + \frac{3}{x}$. (d) $y = x - \frac{2}{x}$. (e) $y = x^4 - 2x^2 - 5$. (f) $y = \sqrt{4 - x^2}$.
\end{baitoan}

\begin{baitoan}[\cite{SGK_Toan_12_giai_tich_nang_cao}, 2., p. 7]
	Chứng minh: (a) Hàm số $y = \frac{x - 2}{x + 2}$ đồng biến trên mỗi khoảng xác định của nó. (b) Hàm số $y = \frac{-x^2 - 2x + 3}{x + 1}$ nghịch biến trên mỗi khoảng xác định của nó.
\end{baitoan}

\begin{baitoan}[\cite{SGK_Toan_12_giai_tich_nang_cao}, 3., p. 8]
	Chứng minh các hàm số sau đây đồng biến trên $\mathbb{R}$: (a) $f(x) = x^3 - 6x^2 + 17x + 4$. (b) $f(x) = x^3 + x - \cos x - 4$.
\end{baitoan}

\begin{baitoan}[\cite{SGK_Toan_12_giai_tich_nang_cao}, 4., p. 8]
	Với giá trị nào của $a$ hàm số $y = ax - x^3$ nghịch biến trên $\mathbb{R}$?
\end{baitoan}

\begin{baitoan}[\cite{SGK_Toan_12_giai_tich_nang_cao}, 5., p. 8]
	Tìm các giá trị của tham số $a$ để hàm số $f(x) = \frac{1}{3}x^3 + ax^2 + 4x + 3$ đồng biến trên $\mathbb{R}$.
\end{baitoan}

\begin{baitoan}[\cite{SGK_Toan_12_giai_tich_nang_cao}, 6., p. 8]
	Xét chiều biến thiên của hàm số: (a) $y = \frac{1}{3}x^3 - 2x^2 + 4x - 5$. (b) $y = -\frac{4}{3}x^3 + 6x^2 - 9x - \frac{2}{3}$. (c) $y = \frac{x^2 - 8x + 9}{x - 5}$. (d) $y = \sqrt{2x - x^2}$. (e) $y = \sqrt{x^2 - 2x + 3}$. (f) $y = \frac{1}{x + 1} - 2x$.
\end{baitoan}

\begin{baitoan}[\cite{SGK_Toan_12_giai_tich_nang_cao}, 7., p. 8]
	Chứng minh hàm số $f(x) = \cos2x - 2x + 3$ nghịch biến trên $\mathbb{R}$.
\end{baitoan}

\begin{baitoan}[\cite{SGK_Toan_12_giai_tich_nang_cao}, 8., p. 8--9]
	Chứng minh bất đẳng thức: (a) $\sin x < x$, $\forall x\in\mathbb{R}$, $x > 0$; $\sin x > x$, $\forall x\in\mathbb{R}$, $x < 0$. (b) $\cos x > 1 - \frac{x^2}{2}$, $\forall x\in\mathbb{R}$, $x\ne0$. (c) $\sin x > x - \frac{x^3}{6}$, $\forall x\in\mathbb{R}$, $x > 0$; $\sin x < x - \frac{x^3}{6}$, $\forall x\in\mathbb{R}$, $x < 0$.
\end{baitoan}

\begin{baitoan}[\cite{SGK_Toan_12_giai_tich_nang_cao}, 9., p. 9]
	Chứng minh: $\sin x + \tan x > 2x$, $\forall x\in\left(0,\frac{\pi}{2}\right)$.
\end{baitoan}

\begin{baitoan}[\cite{SGK_Toan_12_giai_tich_nang_cao}, 10., p. 9]
	Số dân của 1 thị trấn sau $t$ năm kể từ năm 1970 được ước tính bởi công thức $f(t) = \frac{26t + 10}{t + 5}$ ($f(t)$ được tính bằng nghìn người). (a) Tính số dân của thị trấn vào năm 1980 \& năm 1995. (b) Xem $f$ là 1 hàm số xác định trên nửa khoảng $[0,+\infty)$. Tìm $f'$ \& xét chiều biến thiên của hàm số $f$ trên nửa khoảng $[0,+\infty)$. (c) Đạo hàm của hàm số $f$ biểu thị tốc độ tăng dân số của thị trấn (tính bằng nghìn người{\tt/}năm). Tính tốc độ tăng dân số vào năm 1990 \& năm 2008 của thị trấn. Vào năm nào thì tốc độ tăng dân số là $0.125$ nghìn người{\tt/}năm?
\end{baitoan}

%------------------------------------------------------------------------------%

\section{Cực Trị của Hàm Số}

%------------------------------------------------------------------------------%

\printbibliography[heading=bibintoc]
	
\end{document}