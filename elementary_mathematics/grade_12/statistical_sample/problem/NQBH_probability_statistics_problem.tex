\documentclass{article}
\usepackage[backend=biber,natbib=true,style=alphabetic,maxbibnames=50]{biblatex}
\addbibresource{/home/nqbh/reference/bib.bib}
\usepackage[utf8]{vietnam}
\usepackage{tocloft}
\renewcommand{\cftsecleader}{\cftdotfill{\cftdotsep}}
\usepackage[colorlinks=true,linkcolor=blue,urlcolor=red,citecolor=magenta]{hyperref}
\usepackage{amsmath,amssymb,amsthm,float,graphicx,mathtools,tikz}
\usetikzlibrary{angles,calc,intersections,matrix,patterns,quotes,shadings}
\allowdisplaybreaks
\newtheorem{assumption}{Assumption}
\newtheorem{baitoan}{}
\newtheorem{cauhoi}{Câu hỏi}
\newtheorem{conjecture}{Conjecture}
\newtheorem{corollary}{Corollary}
\newtheorem{dangtoan}{Dạng toán}
\newtheorem{definition}{Definition}
\newtheorem{dinhluat}{Định luật}
\newtheorem{dinhly}{Định lý}
\newtheorem{dinhnghia}{Định nghĩa}
\newtheorem{example}{Example}
\newtheorem{ghichu}{Ghi chú}
\newtheorem{hequa}{Hệ quả}
\newtheorem{hypothesis}{Hypothesis}
\newtheorem{lemma}{Lemma}
\newtheorem{luuy}{Lưu ý}
\newtheorem{nhanxet}{Nhận xét}
\newtheorem{notation}{Notation}
\newtheorem{note}{Note}
\newtheorem{principle}{Principle}
\newtheorem{problem}{Problem}
\newtheorem{proposition}{Proposition}
\newtheorem{question}{Question}
\newtheorem{remark}{Remark}
\newtheorem{theorem}{Theorem}
\newtheorem{vidu}{Ví dụ}
\usepackage[left=1cm,right=1cm,top=5mm,bottom=5mm,footskip=4mm]{geometry}
\def\labelitemii{$\circ$}
\DeclareRobustCommand{\divby}{%
	\mathrel{\vbox{\baselineskip.65ex\lineskiplimit0pt\hbox{.}\hbox{.}\hbox{.}}}%
}
\def\labelitemii{$\circ$}

\title{Problem: Statistical Sample -- Bài Tập: Các Số Đặc Trưng Đo Mức Độ Phân Tán Cho Mẫu Số Liệu Ghép Nhóm}
\author{Nguyễn Quản Bá Hồng\footnote{A Scientist {\it\&} Creative Artist Wannabe. E-mail: {\tt nguyenquanbahong@gmail.com}. Bến Tre City, Việt Nam.}}
\date{\today}

\begin{document}
\maketitle
\begin{abstract}
	This text is a part of the series {\it Some Topics in Elementary STEM \& Beyond}:
	
	{\sc url}: \url{https://nqbh.github.io/elementary_STEM}.
	
	Latest version:
	\begin{itemize}
		\item {\it Problem: Statistical Sample -- Bài Tập: Các Số Đặc Trưng Đo Mức Độ Phân Tán Cho Mẫu Số Liệu Ghép Nhóm}.
		
		PDF: {\sc url}: \url{.pdf}.
		
		\TeX: {\sc url}: \url{.tex}.
		\item {\it Problem \& Solution: Statistical Sample -- Bài Tập \& Lời Giải: Các Số Đặc Trưng Đo Mức Độ Phân Tán Cho Mẫu Số Liệu Ghép Nhóm}.
		
		PDF: {\sc url}: \url{.pdf}.
		
		\TeX: {\sc url}: \url{.tex}.
	\end{itemize}
\end{abstract}
\tableofcontents

%------------------------------------------------------------------------------%

\section{Khoảng Biến Thiên. Khoảng Tứ Phân Vị Của Mẫu Số Liệu Ghép Nhóm}
\cite[Chap. III, \S1, pp. 84--88]{SGK_Toan_12_Canh_Dieu_tap_1}: HD1. LT1. HD2. LT2. 1. 2. 3.

%------------------------------------------------------------------------------%

\section{Phương Sai, Độ Lệch Chuẩn Của Mẫu Số Liệu Ghép Nhóm}
\cite[Chap. III, \S2, pp. 84--88]{SGK_Toan_12_Canh_Dieu_tap_1}: HD. LT. 1. 2. 3.

%------------------------------------------------------------------------------%

\section{Miscellaneous}
\cite[BTCCIII, p. 93]{SGK_Toan_12_Canh_Dieu_tap_1}: 1. 2. 3.

%------------------------------------------------------------------------------%

\section{Conditional Probability -- Xác Suất Có Điều Kiện}
\cite[Chap. VI, \S1, pp. 90--96]{SGK_Toan_12_Canh_Dieu_tap_2}: HD1. LT1. LT2. LT3. HD2. LT4. 1. 2. 3. 4. 5. 6. 7. 8.

%------------------------------------------------------------------------------%

\section{Công Thức Xác Suất Toàn Phần. Công Thức Bayes}
\cite[Chap. VI, \S2, pp. 97--102]{SGK_Toan_12_Canh_Dieu_tap_2}: HD1. LT1. LT2. HD2. LT3. LT4. 1. 2. 3. 4. 5.

%------------------------------------------------------------------------------%

\section{Miscellaneous}
\cite[BTCCVI, p. 103]{SGK_Toan_12_Canh_Dieu_tap_2}: 1. 2. 3. 4. 5.

%------------------------------------------------------------------------------%

\printbibliography[heading=bibintoc]
	
\end{document}