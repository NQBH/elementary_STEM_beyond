\documentclass{article}
\usepackage[backend=biber,natbib=true,style=alphabetic,maxbibnames=50]{biblatex}
\addbibresource{/home/nqbh/reference/bib.bib}
\usepackage[utf8]{vietnam}
\usepackage{tocloft}
\renewcommand{\cftsecleader}{\cftdotfill{\cftdotsep}}
\usepackage[colorlinks=true,linkcolor=blue,urlcolor=red,citecolor=magenta]{hyperref}
\usepackage{amsmath,amssymb,amsthm,float,graphicx,mathtools,tikz}
\usetikzlibrary{angles,calc,intersections,matrix,patterns,quotes,shadings}
\allowdisplaybreaks
\newtheorem{assumption}{Assumption}
\newtheorem{baitoan}{}
\newtheorem{cauhoi}{Câu hỏi}
\newtheorem{conjecture}{Conjecture}
\newtheorem{corollary}{Corollary}
\newtheorem{dangtoan}{Dạng toán}
\newtheorem{definition}{Definition}
\newtheorem{dinhluat}{Định luật}
\newtheorem{dinhly}{Định lý}
\newtheorem{dinhnghia}{Định nghĩa}
\newtheorem{example}{Example}
\newtheorem{ghichu}{Ghi chú}
\newtheorem{hequa}{Hệ quả}
\newtheorem{hypothesis}{Hypothesis}
\newtheorem{lemma}{Lemma}
\newtheorem{luuy}{Lưu ý}
\newtheorem{nhanxet}{Nhận xét}
\newtheorem{notation}{Notation}
\newtheorem{note}{Note}
\newtheorem{principle}{Principle}
\newtheorem{problem}{Problem}
\newtheorem{proposition}{Proposition}
\newtheorem{question}{Question}
\newtheorem{remark}{Remark}
\newtheorem{theorem}{Theorem}
\newtheorem{vidu}{Ví dụ}
\usepackage[left=1cm,right=1cm,top=5mm,bottom=5mm,footskip=4mm]{geometry}
\def\labelitemii{$\circ$}
\DeclareRobustCommand{\divby}{%
	\mathrel{\vbox{\baselineskip.65ex\lineskiplimit0pt\hbox{.}\hbox{.}\hbox{.}}}%
}
\def\labelitemii{$\circ$}

\title{Problem: Antiderivative {\it\&} Integral -- Bài Tập: Nguyên Hàm {\it\&} Tích Phân}
\author{Nguyễn Quản Bá Hồng\footnote{A Scientist {\it\&} Creative Artist Wannabe. E-mail: {\tt nguyenquanbahong@gmail.com}. Bến Tre City, Việt Nam.}}
\date{\today}

\begin{document}
\maketitle
\begin{abstract}
	This text is a part of the series {\it Some Topics in Elementary STEM \& Beyond}:
	
	{\sc url}: \url{https://nqbh.github.io/elementary_STEM}.
	
	Latest version:
	\begin{itemize}
		\item {\it Problem: Antiderivative \& Integral -- Bài Tập: Nguyên Hàm \& Tích Phân}.
		
		PDF: {\sc url}: \url{https://github.com/NQBH/elementary_STEM_beyond/blob/main/elementary_mathematics/grade_12/integral/problem/NQBH_integral_problem.pdf}.
		
		\TeX: {\sc url}: \url{https://github.com/NQBH/elementary_STEM_beyond/blob/main/elementary_mathematics/grade_12/integral/problem/NQBH_integral_problem.tex}.
		\item {\it Problem \& Solution: Antiderivative \& Integral -- Bài Tập \& Lời Giải: Nguyên Hàm \& Tích Phân}.
		
		PDF: {\sc url}: \url{https://github.com/NQBH/elementary_STEM_beyond/blob/main/elementary_mathematics/grade_12/integral/solution/NQBH_integral_solution.pdf}.
		
		\TeX: {\sc url}: \url{https://github.com/NQBH/elementary_STEM_beyond/blob/main/elementary_mathematics/grade_12/integral/solution/NQBH_integral_solution.tex}.
	\end{itemize}
\end{abstract}
\tableofcontents

%------------------------------------------------------------------------------%

\section{Antiderivative -- Nguyên Hàm}
\fbox{1} $\left(\int f(x)dx\right)' = f(x)$. \fbox{2} {\sf Tính chất của nguyên hàm}: $\int [f(x) + g(x)]dx = \int f(x)dx + \int g(x)dx$. $\int af(x)dx = a\int f(x)dx$, $\forall a\in\mathbb{R}$. $d\left(\int f(x)dx\right) = f(x)dx$.\\
\\
\noindent\cite[Chap. IV, \S1, pp. 3--8]{SGK_Toan_12_Canh_Dieu_tap_2}: HD1. LT1. HD2. LT2. LT3. HD3. LT4. HD4. LT5. 1. 2. 3. 4. 5. 6.

\begin{baitoan}[\cite{TLCT_giai_tich_12}, VD1, p. 106]
	Tính $\int \cos^23xdx$.
\end{baitoan}

\begin{baitoan}[\cite{TLCT_giai_tich_12}, VD2, p. 106]
	Tìm hàm số $f$ thỏa $f''(x) = 12x^2 + 6x - 4,f(0) = 4,f(1) = 1$.
\end{baitoan}

\begin{baitoan}
	Tìm hàm số $f$ thỏa $f(a) = b$ \&: (a) $f'(x) = c$. (b) $f'(x) = cx + d$. (c) $f'(x) = cx^2 + dx + e$. (d) $f'(x) = \sum_{i=0}^n a_ix^i$.
\end{baitoan}

\begin{baitoan}
	Tìm hàm số $f$ thỏa $f(a) = m,f(b) = n$ \&: (a) $f''(x) = c$. (b) $f''(x) = cx + d$. (c) $f''(x) = cx^2 + dx + e$. (d) $f''(x) = \sum_{i=0}^n a_ix^i$.
\end{baitoan}

\begin{baitoan}[\cite{TLCT_giai_tich_12}, VD3, p. 106]
	Cho $f(x) = \dfrac{x^3 + 2}{x^2 - 1}$. (a) Viết $f(x)$ dưới dạng $f(x) = ax + \dfrac{b}{x + 1} + \dfrac{c}{x - 1}$. (b) Tính $\int f(x)dx$.
\end{baitoan}

\begin{baitoan}[\cite{TLCT_giai_tich_12}, VD4, p. 108]
	Tính $\int x^2(1 - x)^7dx$.
\end{baitoan}

\begin{baitoan}[\cite{TLCT_giai_tich_12}, VD5, p. 108]
	Tính: (a) $\int \dfrac{\cos x - \sin x}{\cos x + \sin x}dx$. (b) $\int \dfrac{7\cos x - 4\sin x}{\cos x + \sin x}dx$.
\end{baitoan}

\begin{baitoan}[\cite{TLCT_giai_tich_12}, VD6, p. 109]
	Tính: (a) $\int xe^{-x}dx$. (b) $\int \sqrt{x}\ln xdx$.
\end{baitoan}

\begin{baitoan}[\cite{TLCT_giai_tich_12}, VD7, p. 110]
	Tính $\int \dfrac{x^2}{(\cos x + x\sin x)^2}dx$.
\end{baitoan}

\begin{baitoan}[\cite{TLCT_giai_tich_12}, VD8, p. 110]
	Tính $\int \sin x\cos xdx$.
\end{baitoan}

\begin{baitoan}[\cite{TLCT_giai_tich_12}, 1., p. 110]
	Tính $\int \dfrac{e^{\tan x}}{\cos^2x}dx$.
\end{baitoan}

\begin{baitoan}[\cite{TLCT_giai_tich_12}, 2., p. 110]
	Tính: (a) $\int \sin2x\cos xdx$. (b) $\int \cot^22xdx$.
\end{baitoan}

\begin{baitoan}[\cite{TLCT_giai_tich_12}, 3., p. 111]
	Tìm hàm số $f(x)$ thỏa: (a) $f'(x) = 4\sqrt{x} - x,f(4) = 0$. (b) $f'(x) = x - \dfrac{1}{x^2} + 2,f(1) = 2$.
\end{baitoan}

\begin{baitoan}[\cite{TLCT_giai_tich_12}, 4., p. 111]
	Tính: (a) $\int 3x^2\sqrt{x^3 + 1}dx$. (b) $\int \dfrac{2x + 4}{x^2 + 4x - 5}dx$.
\end{baitoan}

\begin{baitoan}[\cite{TLCT_giai_tich_12}, 5., p. 111]
	Tính $\int xe^{x^2}dx$.
\end{baitoan}

\begin{baitoan}[\cite{TLCT_giai_tich_12}, 6., p. 111]
	Tính: (a) $\int x^3\ln2xdx$. (b) $\int x^2\cos2xdx$.
\end{baitoan}

\begin{baitoan}[\cite{TLCT_giai_tich_12}, 7., p. 111]
	Tính: (a) $\int \dfrac{x^3}{(6x^4 + 5)^5}dx$. (b) $\int x^2e^xdx$.
\end{baitoan}

%------------------------------------------------------------------------------%

\section{Antivative of Some Elementary Functions -- Nguyên Hàm Của 1 Số Hàm Số Sơ Cấp}
\fbox{1} (a) $\int dx = x + C$. (b) $\int (x + a)^\alpha dx = \dfrac{(x + a)^{\alpha + 1}}{\alpha + 1} + C$, $\forall a,\alpha\in\mathbb{R},\alpha\ne-1$. (c) $\int \dfrac{1}{x + a}dx = \ln|x + a| + C$, $\forall a\in\mathbb{R}$. (d) $\int \sin\alpha dx = -\dfrac{\cos\alpha x}{\alpha} + C$, $\int \cos\alpha xdx = \dfrac{\sin\alpha x}{\alpha} + C$, $\forall\alpha\in\mathbb{R}^\star$. (e) $\int a^xdx = \dfrac{a^x}{\ln a} + C$, $\forall a\in(0,\infty),a\ne-1$. (f) $\int \dfrac{1}{\cos^2x}dx = \tan x + C$, $\int\dfrac{1}{\sin^2x}dx = -\cot x + C$. \fbox{2} {\sf Công thức đổi biến}: \fbox{$\int f(u(x))u'(x)dx = F(u(x)) + C$}, \fbox{$\int f(u)du = F(u(x)) + C$}. \fbox{5} {\sf Công thức nguyên hàm từng phần}: \fbox{$\int u(x)v'(x)dx = u(x)v(x) - \int v(x)u'(x)dx$}, \fbox{$\int udv = uv - \int vdu$}.\\
\\
\cite[Chap. IV, \S2, pp. 9--16]{SGK_Toan_12_Canh_Dieu_tap_2}: HD1. LT1. LT2. HD2. LT3. HD3. LT4. LT5. HD4. LT6. 1. 2. 3. 4. 5. 6. 7. 8.

%------------------------------------------------------------------------------%

\section{Integral -- Tích Phân}
\fbox{1} $\int_a^b f(x)dx = F(x)|_a^b = \left(\int f(x)dx\right)|_a^b$. \fbox{2} (a) {\sf Tính chất của tích phân}: (a) $\int_a^a f(x)dx = 0$. (b) $\int_a^b f(x)dx = -\int_b^a f(x)$. (c) $\int_a^b f(x)dx + \int_b^c f(x)dx = \int_a^c f(x)dx$. (d) $\int_a^b (f(x) + g(x))dx = \int_a^b f(x)dx + \int_a^b g(x)dx$. (f) $\int_a^b kf(x)dx = k\int_a^b f(x)dx$, $\forall k\in\mathbb{R}$. \fbox{3} {\sf Công thức đổi biến}: \fbox{$\int_a^b f(u(x))u'(x)dx = \int_{u(a)}^{u(b)} f(u)du$}. \fbox{4} {\sf Công thức tích phân từng phần}: $\int_a^b udv = uv|_a^b - \int_a^b vdu$, \fbox{$\int_a^b u(x)v'(x)dx = u(b)v(b) - u(a)v(a) - \int_a^b u'(x)v(x)dx$}.\\
\\
\noindent\cite[Chap. IV, \S3, pp. 17--27]{SGK_Toan_12_Canh_Dieu_tap_2}: LT1. LT2. LT3. LT4. LT5. LT6. LT7. LT8. LT9. 1. 2. 3. 4. 5. 6. 7. 8. 9.

\begin{baitoan}[\cite{TLCT_giai_tich_12}, VD1, p. 113]
	Tính: (a) $\int_4^5 \left(x^2 + \dfrac{1}{x}\right)^2dx$. (b) $I = \int_{\frac{\pi}{4}}^{\frac{\pi}{3}} \dfrac{dx}{\sin2x}$. (c) $I = \int_1^e x^2\ln xdx$.
\end{baitoan}

\begin{baitoan}[\cite{TLCT_giai_tich_12}, VD2, p. 114]
	Cho $a\in\left(0,\dfrac{\pi}{2}\right)$. Chứng minh $\int_e^{\tan a} \dfrac{xdx}{1 + x^2} + \int_e^{\cot a} \dfrac{dx}{x(1 + x^2)} = -1$.
\end{baitoan}

\begin{baitoan}[\cite{TLCT_giai_tich_12}, VD3, p. 114]
	Tìm nguyên hàm của hàm số
	\begin{equation*}
		f(x) = \left\{\begin{split}
			-&x,&&\mbox{if } x < -1,\\
			&1,&&\mbox{if } -1\le x\le1,\\
			&x,&&\mbox{if } x > 1.
		\end{split}\right.
	\end{equation*}
\end{baitoan}

\begin{baitoan}[\cite{TLCT_giai_tich_12}, VD4, p. 115]
	Cho hàm số $g(x) = \int_{\sqrt{x}}^{x^2} \sqrt{t}\sin tdt$ xác định với $x > 0$. Tìm $g'(x)$.
\end{baitoan}

\begin{baitoan}[\cite{TLCT_giai_tich_12}, VD5, p. 117]
	Cho dãy $(u_n)$ xác định bởi công thức $u_n = \dfrac{1}{n}\sum_{i=1}^n \sqrt{\dfrac{i}{n}}$. Tính $\lim_{n\to\infty} u_n$.
\end{baitoan}

\begin{baitoan}[\cite{TLCT_giai_tich_12}, VD6, p. 118]
	Cho dãy $(u_n)$ xác định bởi công thức $u_n = \sum_{i=1}^n \dfrac{1}{2n + 2i - 1} = \dfrac{1}{2n + 1} + \dfrac{1}{2n + 3} + \cdots + \dfrac{1}{4n - 1}$. Tính $\lim_{n\to\infty} u_n$.
\end{baitoan}

\begin{baitoan}[\cite{TLCT_giai_tich_12}, VD7, p. 119]
	Tính $I = \int_1^2 xe^{x^2}dx$.
\end{baitoan}

\begin{baitoan}[\cite{TLCT_giai_tich_12}, VD8, p. 120]
	Tính: (a) $I = \int_{-1}^1 \dfrac{dx}{x^2 + 1}$. (b) $I = \int_{\pi}^{2\pi} \dfrac{x\sin x}{1 + \cos^2x}dx$.
\end{baitoan}

\begin{baitoan}[\cite{TLCT_giai_tich_12}, VD9, p. 121]
	Tính $I = \int_0^{\frac{\pi}{4}} \dfrac{(1 + \sin x\cos x)e^x}{1 + \cos2x}dx$.
\end{baitoan}

\begin{baitoan}[\cite{TLCT_giai_tich_12}, VD10, p. 121]
	Tính $u_n = \int_0^\pi \cos^nx\cos nxdx$.
\end{baitoan}

\begin{baitoan}[\cite{TLCT_giai_tich_12}, VD11, p. 122]
	Giả sử f là hàm liên tục. Chứng minh $\int_0^a f(x)(a - x)dx = \int_0^a\left(\int_0^x f(t)dt\right)dx$.
\end{baitoan}

\begin{baitoan}[\cite{TLCT_giai_tich_12}, 8., p. 123]
	Tính: (a) $I = \int_0^1 x^3e^{x^2}dx$. (b) $I = \int_0^{\ln2} e^{7x}dx$.
\end{baitoan}

\begin{baitoan}[\cite{TLCT_giai_tich_12}, 9., p. 123]
	Tính: (a) $I = \int_0^{\frac{\pi}{3}} \tan xdx$. (b) $I = \int_0^3 \dfrac{xdx}{1 + x^2}$.
\end{baitoan}

\begin{baitoan}[\cite{TLCT_giai_tich_12}, 10., p. 123]
	Tính: (a) $I = \int_0^{\frac{\pi}{3}} \tan^2xdx$. (b) $I = \int_1^e (\ln x)^2dx$.
\end{baitoan}

\begin{baitoan}[\cite{TLCT_giai_tich_12}, 11., p. 123]
	Tính: (a) $I = \int_0^1 x^2e^{4x}dx$. (b) $I = \int_4^7 \dfrac{dx}{\sqrt{(x - 4)(7 - x)}}$.
\end{baitoan}

\begin{baitoan}[\cite{TLCT_giai_tich_12}, 12., p. 123]
	Cho hàm số
	\begin{equation*}
		f(x) = \left\{\begin{split}
			-&2(x + 1),&&\mbox{khi } x\le0,\\
			&k(1 - x^2),&&\mbox{khi } x > 0.
		\end{split}\right.
	\end{equation*}
	Tìm $k\in\mathbb{R}$ để $\int_{-1}^1 f(x)dx = 1$.
\end{baitoan}

\begin{baitoan}[\cite{TLCT_giai_tich_12}, 13., p. 123]
	Cho hàm số $g(x) = \int_{2x}^{3x} \dfrac{t^2 - 1}{t^2 + 1}dt$. Tìm $g'(x)$.
\end{baitoan}

\begin{baitoan}[\cite{TLCT_giai_tich_12}, 14., p. 123]
	Tìm hàm số f \& $a\in(0,\infty)$ thỏa $\int_a^x \dfrac{f(t)}{t^2}dt + 6 = 2\sqrt{x}$, $\forall x\in(0,\infty)$.
\end{baitoan}

\begin{baitoan}[\cite{TLCT_giai_tich_12}, 15., p. 123]
	Cho hàm $f(x)$ liên tục \& $a\in(0,\infty)$. Giả sử $\forall x\in[0,a]$, có $f(x) > 0,f(x)f(a - x) = 1$. Tính $I = \int_0^a \dfrac{dx}{1 + f(x)}$ theo a.
\end{baitoan}

\begin{baitoan}[\cite{TLCT_giai_tich_12}, 16., p. 123]
	Tính $I = \int_{-1}^1 \dfrac{dx}{(e^x + 1)(x^2 + 1)}$.
\end{baitoan}

\begin{baitoan}[\cite{TLCT_giai_tich_12}, 17., p. 123]
	Cho dãy $(u_n)$ xác định bởi công thức $u_n = \sum_{i=1}^n \dfrac{i^3}{n^4}$. Tính $\lim_{n\to\infty} u_n$.
\end{baitoan}

\begin{baitoan}[\cite{TLCT_giai_tich_12}, 18., p. 123]
	Cho dãy $(u_n)$ xác định bởi công thức $u_n = \sum_{i=1}^n \dfrac{i^2}{i^3 + n^3}$. Tính $\lim_{n\to\infty} u_n$.
\end{baitoan}

%------------------------------------------------------------------------------%

\section{Geometrical Application of Integral -- Ứng Dụng Hình Học Của Tích Phân}
Cho các hàm $f,g\in C(\mathbb{R})$. \fbox{1} Hình phẳng giới hạn bởi đồ thị hàm số $y = f(x),y = g(x)$ \& 2 đường thẳng $x = a,x = b$ có diện tích $S = \int_a^b |f(x) - g(x)|dx$. \fbox{2} Hình phẳng giới hạn bởi các đường cong với phương trình $x = f(y),x = g(y)$ \& 2 đường thẳng $y = c,y = d$, $c < d$ có diện tích $S = \int_c^d |f(y) - g(y)|dy$. \fbox{3} Đường cong $\mathcal{C}:y = f(x),f\in C^2([a,b])$ từ điểm $A(a,f(a))$ đến điểm $B(b,f(b))$ có độ dài $L = \int_a^b \sqrt{1 + (f'(x))^2}dx$. \fbox{4} Đường cong $\mathcal{C}:x = f(y),f\in C^2([c,d])$ từ điểm $C(g(c),c)$ đến điểm $D(g(d),d)$ có độ dài $L = \int_c^d \sqrt{1 + (g'(y))^2}dy$.\\
\\
\cite[Chap. IV, \S4, pp. 28--41]{SGK_Toan_12_Canh_Dieu_tap_2}: HD1. LT1. HD2. LT2. HD3. LT3. LT4. HD4. LT5. 1. 2. 3. 4. 5. 6. 7. 8. 9. 10.

\begin{baitoan}[\cite{TLCT_giai_tich_12}, VD1, p. 126]
	Tính diện tích hình phẳng giới hạn bởi đồ thị 2 hàm số $y = \sin x,y = \cos x$ \& 2 đường thẳng $x = 0,x = \dfrac{\pi}{2}$.
\end{baitoan}

\begin{baitoan}[\cite{TLCT_giai_tich_12}, VD2, p. 126]
	Tính diện tích hình phẳng $\mathcal{H}$ giới hạn bởi đường thẳng $y = x - 1$ \& parabol $y^2 = 2x + 6$.
\end{baitoan}

\begin{baitoan}[\cite{TLCT_giai_tich_12}, VD3, p. 128]
	Tính độ dài đường cong $\mathcal{C}:y^2 = x^3$ đi từ điểm $A(1,1)$ đến điểm $B(4,8)$.
\end{baitoan}

\begin{baitoan}[\cite{TLCT_giai_tich_12}, VD4, p. 129]
	Tìm độ dài cung parabol $\mathcal{C}:y^2 = x$ từ điểm $A(0,0)$ đến điểm $B\left(\frac{1}{4},\frac{1}{2}\right)$.
\end{baitoan}

%------------------------------------------------------------------------------%

\section{Miscellaneous}
\cite[BTCCIV, pp. 42--44]{SGK_Toan_12_Canh_Dieu_tap_2}: 1. 2. 3. 4. 5. 6. 7. 8. 9. 10. 11. 12. 13.

%------------------------------------------------------------------------------%

\printbibliography[heading=bibintoc]
	
\end{document}