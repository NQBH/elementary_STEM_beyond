\documentclass{article}
\usepackage[backend=biber,natbib=true,style=alphabetic,maxbibnames=50]{biblatex}
\addbibresource{/home/nqbh/reference/bib.bib}
\usepackage[utf8]{vietnam}
\usepackage{tocloft}
\renewcommand{\cftsecleader}{\cftdotfill{\cftdotsep}}
\usepackage[colorlinks=true,linkcolor=blue,urlcolor=red,citecolor=magenta]{hyperref}
\usepackage{amsmath,amssymb,amsthm,float,graphicx,mathtools,tikz}
\usetikzlibrary{angles,calc,intersections,matrix,patterns,quotes,shadings}
\allowdisplaybreaks
\newtheorem{assumption}{Assumption}
\newtheorem{baitoan}{}
\newtheorem{cauhoi}{Câu hỏi}
\newtheorem{conjecture}{Conjecture}
\newtheorem{corollary}{Corollary}
\newtheorem{dangtoan}{Dạng toán}
\newtheorem{definition}{Definition}
\newtheorem{dinhly}{Định lý}
\newtheorem{dinhnghia}{Định nghĩa}
\newtheorem{example}{Example}
\newtheorem{ghichu}{Ghi chú}
\newtheorem{hequa}{Hệ quả}
\newtheorem{hypothesis}{Hypothesis}
\newtheorem{lemma}{Lemma}
\newtheorem{luuy}{Lưu ý}
\newtheorem{nhanxet}{Nhận xét}
\newtheorem{notation}{Notation}
\newtheorem{note}{Note}
\newtheorem{principle}{Principle}
\newtheorem{problem}{Problem}
\newtheorem{proposition}{Proposition}
\newtheorem{question}{Question}
\newtheorem{remark}{Remark}
\newtheorem{theorem}{Theorem}
\newtheorem{vidu}{Ví dụ}
\usepackage[left=1cm,right=1cm,top=5mm,bottom=5mm,footskip=4mm]{geometry}
\def\labelitemii{$\circ$}
\DeclareRobustCommand{\divby}{%
	\mathrel{\vbox{\baselineskip.65ex\lineskiplimit0pt\hbox{.}\hbox{.}\hbox{.}}}%
}

\title{Problem: Antiderivative, Integral -- Bài Tập: Nguyên Hàm, Tích Phân}
\author{Nguyễn Quản Bá Hồng\footnote{Independent Researcher, Ben Tre City, Vietnam\\e-mail: \texttt{nguyenquanbahong@gmail.com}; website: \url{https://nqbh.github.io}.}}
\date{\today}

\begin{document}
\maketitle
\tableofcontents

%------------------------------------------------------------------------------%

\section{Antiderivative -- Nguyên Hàm}
\cite[Chap. IV, \S1, pp. 3--8]{SGK_Toan_12_Canh_Dieu_tap_2}: HD1. LT1. HD2. LT2. LT3. HD3. LT4. HD4. LT5. 1. 2. 3. 4. 5. 6.

%------------------------------------------------------------------------------%

\section{Antivative of Some Elementary Functions -- Nguyên Hàm Của 1 Số Hàm Số Sơ Cấp}
\cite[Chap. IV, \S2, pp. 9--16]{SGK_Toan_12_Canh_Dieu_tap_2}: HD1. LT1. LT2. HD2. LT3. HD3. LT4. LT5. HD4. LT6. 1. 2. 3. 4. 5. 6. 7. 8.

%------------------------------------------------------------------------------%

\section{Integral -- Tích Phân}
\cite[Chap. IV, \S3, pp. 17--27]{SGK_Toan_12_Canh_Dieu_tap_2}: HD1. LT1. HD2. LT2. HD3. LT3. HD4. LT4. LT5. LT6. LT7. LT8. LT9. 1. 2. 3. 4. 5. 6. 7. 8. 9.

%------------------------------------------------------------------------------%

\section{Geometrical Application of Integral -- Ứng Dụng Hình Học Của Tích Phân}
\cite[Chap. IV, \S4, pp. 28--41]{SGK_Toan_12_Canh_Dieu_tap_2}: HD1. LT1. HD2. LT2. HD3. LT3. LT4. HD4. LT5. 1. 2. 3. 4. 5. 6. 7. 8. 9. 10.

%------------------------------------------------------------------------------%

\section{Miscellaneous}
\cite[BTCCIV, pp. 42--44]{SGK_Toan_12_Canh_Dieu_tap_2}: 1. 2. 3. 4. 5. 6. 7. 8. 9. 10. 11. 12. 13.

%------------------------------------------------------------------------------%

\printbibliography[heading=bibintoc]
	
\end{document}