\documentclass{article}
\usepackage[backend=biber,natbib=true,style=alphabetic,maxbibnames=50]{biblatex}
\addbibresource{/home/nqbh/reference/bib.bib}
\usepackage[utf8]{vietnam}
\usepackage{tocloft}
\renewcommand{\cftsecleader}{\cftdotfill{\cftdotsep}}
\usepackage[colorlinks=true,linkcolor=blue,urlcolor=red,citecolor=magenta]{hyperref}
\usepackage{amsmath,amssymb,amsthm,float,graphicx,mathtools,tikz}
\usetikzlibrary{angles,calc,intersections,matrix,patterns,quotes,shadings}
\allowdisplaybreaks
\newtheorem{assumption}{Assumption}
\newtheorem{baitoan}{}
\newtheorem{cauhoi}{Câu hỏi}
\newtheorem{conjecture}{Conjecture}
\newtheorem{corollary}{Corollary}
\newtheorem{dangtoan}{Dạng toán}
\newtheorem{definition}{Definition}
\newtheorem{dinhluat}{Định luật}
\newtheorem{dinhly}{Định lý}
\newtheorem{dinhnghia}{Định nghĩa}
\newtheorem{example}{Example}
\newtheorem{ghichu}{Ghi chú}
\newtheorem{hequa}{Hệ quả}
\newtheorem{hypothesis}{Hypothesis}
\newtheorem{lemma}{Lemma}
\newtheorem{luuy}{Lưu ý}
\newtheorem{nhanxet}{Nhận xét}
\newtheorem{notation}{Notation}
\newtheorem{note}{Note}
\newtheorem{principle}{Principle}
\newtheorem{problem}{Problem}
\newtheorem{proposition}{Proposition}
\newtheorem{question}{Question}
\newtheorem{remark}{Remark}
\newtheorem{theorem}{Theorem}
\newtheorem{vidu}{Ví dụ}
\usepackage[left=1cm,right=1cm,top=5mm,bottom=5mm,footskip=4mm]{geometry}
\def\labelitemii{$\circ$}
\DeclareRobustCommand{\divby}{%
	\mathrel{\vbox{\baselineskip.65ex\lineskiplimit0pt\hbox{.}\hbox{.}\hbox{.}}}%
}
\def\labelitemii{$\circ$}

\title{Problem: Equations of Plane, Line, {\it\&} Sphere in 3D Space\\Bài Tập: Phương Trình Mặt Phẳng, Đường Thẳng, Mặt Cầu Trong Không Gian}
\author{Nguyễn Quản Bá Hồng\footnote{A Scientist {\it\&} Creative Artist Wannabe. E-mail: {\tt nguyenquanbahong@gmail.com}. Bến Tre City, Việt Nam.}}
\date{\today}

\begin{document}
\maketitle
\begin{abstract}
	This text is a part of the series {\it Some Topics in Elementary STEM \& Beyond}:
	
	{\sc url}: \url{https://nqbh.github.io/elementary_STEM}.
	
	Latest version:
	\begin{itemize}
		\item {\it Problem: Equations of Plane, Line, {\it\&} Sphere in 3D Space -- Bài Tập: Phương Trình Mặt Phẳng, Đường Thẳng, Mặt Cầu Trong Không Gian}.
		
		PDF: {\sc url}: \url{https://github.com/NQBH/elementary_STEM_beyond/blob/main/elementary_mathematics/grade_12/3D_plane_line_sphere_equation/problem/NQBH_3D_plane_line_sphere_equation_problem.pdf}.
		
		\TeX: {\sc url}: \url{https://github.com/NQBH/elementary_STEM_beyond/blob/main/elementary_mathematics/grade_12/3D_plane_line_sphere_equation/problem/NQBH_3D_plane_line_sphere_equation_problem.tex}.
		\item {\it Problem \& Solution: Equations of Plane, Line, {\it\&} Sphere in 3D Space -- Bài Tập \& Lời Giải: Phương Trình Mặt Phẳng, Đường Thẳng, Mặt Cầu Trong Không Gian}.
		
		PDF: {\sc url}: \url{https://github.com/NQBH/elementary_STEM_beyond/blob/main/elementary_mathematics/grade_12/3D_plane_line_sphere_equation/solution/NQBH_3D_plane_line_sphere_equation_solution.pdf}.
		
		\TeX: {\sc url}: \url{https://github.com/NQBH/elementary_STEM_beyond/blob/main/elementary_mathematics/grade_12/3D_plane_line_sphere_equation/solution/NQBH_3D_plane_line_sphere_equation_solution.tex}.
	\end{itemize}
\end{abstract}
\tableofcontents

%------------------------------------------------------------------------------%

\section{Plane Equation -- Phương Trình Mặt Phẳng}
\cite[Chap. V, \S1, pp. 50--64]{SGK_Toan_12_Canh_Dieu_tap_2}: HD1. LT1. HD2. LT2. HD3. LT3. HD4. LT4. HD5. LT5. HD6. LT6. HD7. LT7. LT8. HD8. LT9. HD9. LT10. HD10. LT11. LT12. 1. 2. 3. 4. 5. 6. 7. 8. 9. 10. 11. 12.

\begin{baitoan}[\cite{SGK_Toan_12_hinh_hoc_co_ban}, 1., p. 80]
	Viết phương trình của mặt phẳng: (a) Đi qua điểm $M(1,-2,4)$ \& nhận $\vec{n} = (2,3,5)$ làm vector pháp tuyến; (b) Đi qua điểm $A(0,-1,2)$ \& song song với giá của mỗi vector $\vec{u} = (3,2,1)$ \& $\vec{v} = (-3,0,1)$; (c) Đi qua 3 điểm $A(-3,0,0)$, $B(0,-2,0)$, \& $C(0,0,-1)$.
\end{baitoan}

\begin{baitoan}[\cite{SGK_Toan_12_hinh_hoc_co_ban}, 2., p. 80]
	Viết phương trình mặt phẳng trung trực của đoạn thẳng $AB$ với $A(2,3,7)$, $B(4,1,3)$.
\end{baitoan}

\begin{baitoan}[\cite{SGK_Toan_12_hinh_hoc_co_ban}, 3., p. 80]
	(a) Lập phương trình của các mặt phẳng tọa độ $(Oxy),(Oyz),(Oxz)$. (b) Lập phương trình của các mặt phẳng đi qua điểm $M(2,6,-3)$ \& lần lượt song song với các mặt phẳng tọa độ.
\end{baitoan}

\begin{baitoan}[\cite{SGK_Toan_12_hinh_hoc_co_ban}, 4., p. 80]
	Lập phương trình của mặt phẳng: (a) Chứa trục $Ox$ \& điểm $P(4,-1,2)$; (b) Chứa trục $Oy$ \& điểm $Q(1,4,-3)$; (c) Chứa trục $Oz$ \& điểm $R(3,-4,7)$.
\end{baitoan}

\begin{baitoan}[\cite{SGK_Toan_12_hinh_hoc_co_ban}, 5., p. 80]
	Cho tứ diện có các đỉnh là $A(5,1,3),B(1,6,2),C(5,0,4),D(4,0,6)$. (a) Viết phương trình của các mặt phẳng $(ACD),(BCD)$. (b) Viết phương trình mặt phẳng $(\alpha)$ đi qua cạnh $AB$ \& song song với cạnh $CD$.
\end{baitoan}

\begin{baitoan}[\cite{SGK_Toan_12_hinh_hoc_co_ban}, 6., p. 80]
	Viết phương trình mặt phẳng $(\alpha)$ đi qua điểm $M(2,-1,2)$ \& song song với mặt phẳng $(\beta)$: $2x - y + 3z + 4 = 0$.
\end{baitoan}

\begin{baitoan}[\cite{SGK_Toan_12_hinh_hoc_co_ban}, 7., p. 80]
	Lập phương trình mặt phẳng $(\alpha)$ đi qua 2 điểm $A(1,0,1),B(5,2,3)$ \& vuông góc với mặt phẳng $(\beta)$: $2x - y + z - 7 = 0$.
\end{baitoan}

\begin{baitoan}[\cite{SGK_Toan_12_hinh_hoc_co_ban}, 8., p. 81]
	Xác định các giá trị của $m,n$ để mỗi cặp mặt phẳng sau đây là 1 cặp mặt phẳng song song với nhau: (a) $2x + my + 3z - 5 = 0$ \& $nx - 8y - 6z + 2 = 0$. (b) $3x - 5y + mz - 3 = 0$ \& $2x + ny - 3z + 1 = 0$.
\end{baitoan}

\begin{baitoan}[\cite{SGK_Toan_12_hinh_hoc_co_ban}, 9., p. 81]
	Tính khoảng cách từ điểm $A(2,4,-3)$ lần lượt đến các mặt phẳng sau: (a) $2x - y + 2z - 9 = 0$; (b) $12x - 5z + 5 = 0$; (c) $x = 0$.
\end{baitoan}

\begin{baitoan}[\cite{SGK_Toan_12_hinh_hoc_co_ban}, 10., p. 81]
	Giải bài toán sau đây bằng phương pháp tọa độ: Cho hình lập phương $ABCD.A'B'C'D'$ cạnh bằng $1$. (a) Chứng minh 2 mặt phẳng $(AB'D'),(BC'D)$ song song với nhau. (b) Tính khoảng cách giữa 2 mặt phẳng nói trên.
\end{baitoan}

%------------------------------------------------------------------------------%

\section{Line Equation -- Phương Trình Đường Thẳng}
\cite[Chap. V, \S2, pp. 65--80]{SGK_Toan_12_Canh_Dieu_tap_2}: HD1. LT1. HD2. LT2. HD3. LT3. HD4. LT4. HD5. LT5. HD6. LT6. HD7. LT7. HD8. LT8. HD9. LT9. 1. 2. 3. 4. 5. 6. 7. 8. 9. 10. 11.

\begin{baitoan}[\cite{SGK_Toan_12_hinh_hoc_co_ban}, 1., p. 89]
	Viết phương trình tham số của đường thẳng $d$ trong mỗi trường hợp sau: (a) $d$ đi qua điểm $M(5,4,1)$ \& có vector chỉ phương $\vec{a} = (2,-3,1)$; (b) $d$ đi qua điểm $A(2,-1,3)$ \& vuông góc với mặt phẳng $(\alpha)$ có phương trình $x + y - z + 5 = 0$; (c) $d$ đi qua điểm $B(2,0,-3)$ \& song song với đường thẳng $\Delta$: $x = 1 + 2t$, $y = -3 + 3t$, $z = 4t$; (d) $d$ đi qua 2 điểm $P(1,2,3)$ \& $Q(5,4,4)$.
\end{baitoan}

\begin{baitoan}[\cite{SGK_Toan_12_hinh_hoc_co_ban}, 2., p. 89]
	Viết phương trình tham số của đường thẳng là hình chiếu vuông góc của đường thẳng $d$: $x = 2 + t$, $y = -3 + 2t$, $z = 1 + 3t$ lần lượt trên các mặt phẳng sau: (a) $(Oxy)$; (b) $(Oyz)$.
\end{baitoan}

\begin{baitoan}[\cite{SGK_Toan_12_hinh_hoc_co_ban}, 3., p. 90]
	Xét vị trí tương đối của các cặp đường thẳng $d,d'$ cho bởi các phương trình sau:
	\begin{equation*}
		d:\left\{\begin{split}
			x &= -3 + 2t,\\
			y &= -2 + 3t,\\
			z &= 6 + 4t,
		\end{split}\right.,\ d':\left\{\begin{split}
			x &= 5 + t',\\
			y &= -1 - 4t',\\
			z &= 20 + t',
		\end{split}\right.
	\end{equation*}
\end{baitoan}

\begin{baitoan}[\cite{SGK_Toan_12_hinh_hoc_co_ban}, 4., p. 90]
	
\end{baitoan}

\begin{baitoan}[\cite{SGK_Toan_12_hinh_hoc_co_ban}, 5., p. 90]
	
\end{baitoan}

\begin{baitoan}[\cite{SGK_Toan_12_hinh_hoc_co_ban}, 6., p. 90]
	
\end{baitoan}

\begin{baitoan}[\cite{SGK_Toan_12_hinh_hoc_co_ban}, 7., p. 91]
	
\end{baitoan}

\begin{baitoan}[\cite{SGK_Toan_12_hinh_hoc_co_ban}, 8., p. 91]
	
\end{baitoan}

\begin{baitoan}[\cite{SGK_Toan_12_hinh_hoc_co_ban}, 9., p. 91]
	
\end{baitoan}

\begin{baitoan}[\cite{SGK_Toan_12_hinh_hoc_co_ban}, 10., p. 91]
	
\end{baitoan}

%------------------------------------------------------------------------------%

\section{Sphere Equation -- Phương Trình Mặt Cầu}
\cite[Chap. V, \S3, pp. 81--8]{SGK_Toan_12_Canh_Dieu_tap_2}: HD1. LT1. LT2. LT3. LT4. LT5. 1. 2. 3. 4. 5. 6. 7.

%------------------------------------------------------------------------------%

\section{Miscellaneous}
\cite[BTCCV, pp. 81--8]{SGK_Toan_12_Canh_Dieu_tap_2}: HD1. LT1. LT2. LT3. LT4. LT5. 1. 2. 3. 4. 5. 6. 7. 8. 9. 10. 11. 12. 13. 14.

%------------------------------------------------------------------------------%

\printbibliography[heading=bibintoc]
	
\end{document}