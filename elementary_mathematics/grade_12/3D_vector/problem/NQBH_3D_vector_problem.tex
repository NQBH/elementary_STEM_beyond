\documentclass{article}
\usepackage[backend=biber,natbib=true,style=alphabetic,maxbibnames=50]{biblatex}
\addbibresource{/home/nqbh/reference/bib.bib}
\usepackage[utf8]{vietnam}
\usepackage{tocloft}
\renewcommand{\cftsecleader}{\cftdotfill{\cftdotsep}}
\usepackage[colorlinks=true,linkcolor=blue,urlcolor=red,citecolor=magenta]{hyperref}
\usepackage{amsmath,amssymb,amsthm,float,graphicx,mathtools,tikz}
\usetikzlibrary{angles,calc,intersections,matrix,patterns,quotes,shadings}
\allowdisplaybreaks
\newtheorem{assumption}{Assumption}
\newtheorem{baitoan}{}
\newtheorem{cauhoi}{Câu hỏi}
\newtheorem{conjecture}{Conjecture}
\newtheorem{corollary}{Corollary}
\newtheorem{dangtoan}{Dạng toán}
\newtheorem{definition}{Definition}
\newtheorem{dinhly}{Định lý}
\newtheorem{dinhnghia}{Định nghĩa}
\newtheorem{example}{Example}
\newtheorem{ghichu}{Ghi chú}
\newtheorem{hequa}{Hệ quả}
\newtheorem{hypothesis}{Hypothesis}
\newtheorem{lemma}{Lemma}
\newtheorem{luuy}{Lưu ý}
\newtheorem{nhanxet}{Nhận xét}
\newtheorem{notation}{Notation}
\newtheorem{note}{Note}
\newtheorem{principle}{Principle}
\newtheorem{problem}{Problem}
\newtheorem{proposition}{Proposition}
\newtheorem{question}{Question}
\newtheorem{remark}{Remark}
\newtheorem{theorem}{Theorem}
\newtheorem{vidu}{Ví dụ}
\usepackage[left=1cm,right=1cm,top=5mm,bottom=5mm,footskip=4mm]{geometry}
\def\labelitemii{$\circ$}
\DeclareRobustCommand{\divby}{%
	\mathrel{\vbox{\baselineskip.65ex\lineskiplimit0pt\hbox{.}\hbox{.}\hbox{.}}}%
}

\title{Problem: Coordinates of Vectors in 3D Space\\Bài Tập: Tọa Độ Của Vector Trong Không Gian}
\author{Nguyễn Quản Bá Hồng\footnote{A Scientist {\it\&} Creative Artist Wannabe. E-mail: {\tt nguyenquanbahong@gmail.com}. Bến Tre City, Việt Nam.}}
\date{\today}

\begin{document}
\maketitle
\begin{abstract}
	This text is a part of the series {\it Some Topics in Elementary STEM \& Beyond}:
	
	{\sc url}: \url{https://nqbh.github.io/elementary_STEM}.
	
	Latest version:
	\begin{itemize}
		\item {\it Problem: Coordinates of Vectors in 3D Space -- Bài Tập: Tọa Độ Của Vector Trong Không Gian}.
		
		PDF: {\sc url}: \url{https://github.com/NQBH/elementary_STEM_beyond/blob/main/elementary_mathematics/grade_12/3D_vector/problem/NQBH_3D_vector_problem.pdf}.
		
		\TeX: {\sc url}: \url{https://github.com/NQBH/elementary_STEM_beyond/blob/main/elementary_mathematics/grade_12/3D_vector/problem/NQBH_3D_vector_problem.tex}.
		\item {\it Problem \& Solution: Coordinates of Vectors in 3D Space -- Bài Tập \& Lời Giải: Tọa Độ Của Vector Trong Không Gian}.
		
		PDF: {\sc url}: \url{https://github.com/NQBH/elementary_STEM_beyond/blob/main/elementary_mathematics/grade_12/3D_vector/solution/NQBH_3D_vector_solution.pdf}.
		
		\TeX: {\sc url}: \url{https://github.com/NQBH/elementary_STEM_beyond/blob/main/elementary_mathematics/grade_12/3D_vector/solution/NQBH_3D_vector_solution.tex}.
	\end{itemize}
\end{abstract}
\tableofcontents

%------------------------------------------------------------------------------%

\section{Vector \& Vector Calculus in 3D Space -- Vector \& Các Phép Toán Vector Trong Không Gian}
\cite[Chap. II, \S1, pp. 56--64]{SGK_Toan_12_Canh_Dieu_tap_1}: HD1. LT1. HD2. LT2. HD3. LT3. HD4. LT4. HD5. LT5. HD6. LT6. HD7. LT7. 1. 2. 3. 4. 5.

%------------------------------------------------------------------------------%

\section{Coordinate of Vector -- Tọa Độ Của Vector}
\cite[Chap. II, \S2, pp. 56--73]{SGK_Toan_12_Canh_Dieu_tap_1}: HD1. LT1. HD2. LT2. HD3. LT3. HD4. LT4. HD5. LT5. HD6. LT6. 1. 2. 3. 4. 5. 6. 7. 8. 9.

\begin{baitoan}[\cite{SGK_Toan_12_hinh_hoc_co_ban}, 1., p. 68]
	Cho 3 vector $\vec{a} = (2,-5,3),\vec{b} = (0,2,-1),\vec{c} = (1,7,2)$. (a) Tính tọa độ của vector $\vec{d} = 4\vec{a} - \frac{1}{3}\vec{b} + 3\vec{c}$. (b) Tính tọa độ của vector $\vec{e} = \vec{a} - 4\vec{b} - 2\vec{c}$.
\end{baitoan}

\begin{baitoan}[\cite{SGK_Toan_12_hinh_hoc_co_ban}, 2., p. 68]
	Cho 3 điểm $A = (1,-1,1),B = (0,1,2),C = (1,0,1)$. Tìm tọa độ trọng tâm $G$ của $\Delta ABC$.
\end{baitoan}

\begin{baitoan}[\cite{SGK_Toan_12_hinh_hoc_co_ban}, 3., p. 68]
	Cho hình hộp $ABCD.A'B'C'D'$ biết $A = (1,0,1),B = (2,1,2),D = (1,-1,1),C' = (4,5,-5)$. Tính tọa độ các đỉnh còn lại của hình hộp.
\end{baitoan}

\begin{baitoan}[\cite{SGK_Toan_12_hinh_hoc_co_ban}, 4., p. 68]
	Tính: (a) $\vec{a}\cdot\vec{b}$ với $\vec{a} = (3,0,-6),\vec{b} = (2,-4,0)$. (b) $\vec{c}\cdot\vec{d}$ với $\vec{c} = (1,-5,2),\vec{d} = (4,3,-5)$.
\end{baitoan}

\begin{baitoan}[\cite{SGK_Toan_12_hinh_hoc_co_ban}, 5., p. 68]
	Tìm tâm \& bán kính của các mặt cầu có phương trình sau: (a) $x^2 + y^2 + z^2 - 8x - 2y + 1 = 0$; (b) $3x^2 + 3y^2 + 3z^2 - 6x + 8y + 15z - 3 = 0$.
\end{baitoan}

\begin{baitoan}[\cite{SGK_Toan_12_hinh_hoc_co_ban}, 6., p. 68]
	Lập phương trình mặt cầu trong 2 mặt cầu trong 2 trường hợp sau: (a) Có đường kính $AB$ với $A = (4,-3,7),B = (2,1,3)$. (b) Đi qua điểm $A = (5,-2,1)$ \& có tâm $C = (3,-3,1)$.
\end{baitoan}

%------------------------------------------------------------------------------%

\section{Vector Calculus in Coordinate System -- Biểu Thức Tọa Độ Của Các Phép Toán Vector}
\cite[Chap. II, \S3, pp. 74--81]{SGK_Toan_12_Canh_Dieu_tap_1}: HD1. LT1. HD2. LT2. HD3. LT3. HD4. LT4. 1. 2. 3. 4. 5. 6. 7. 8.

%------------------------------------------------------------------------------%

\section{Miscellaneous}
\cite[BTCCII, pp. 82--83]{SGK_Toan_12_Canh_Dieu_tap_1}: 1. 2. 3. 4. 5. 6. 7. 8. 9. 10. 11. 12. 13.14. 15.

%------------------------------------------------------------------------------%

\printbibliography[heading=bibintoc]
	
\end{document}