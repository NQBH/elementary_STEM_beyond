\documentclass{article}
\usepackage[backend=biber,natbib=true,style=alphabetic,maxbibnames=50]{biblatex}
\addbibresource{/home/nqbh/reference/bib.bib}
\usepackage[utf8]{vietnam}
\usepackage{tocloft}
\renewcommand{\cftsecleader}{\cftdotfill{\cftdotsep}}
\usepackage[colorlinks=true,linkcolor=blue,urlcolor=red,citecolor=magenta]{hyperref}
\usepackage{amsmath,amssymb,amsthm,float,graphicx,mathtools,tikz}
\usetikzlibrary{angles,calc,intersections,matrix,patterns,quotes,shadings}
\allowdisplaybreaks
\newtheorem{assumption}{Assumption}
\newtheorem{baitoan}{}
\newtheorem{cauhoi}{Câu hỏi}
\newtheorem{conjecture}{Conjecture}
\newtheorem{corollary}{Corollary}
\newtheorem{dangtoan}{Dạng toán}
\newtheorem{definition}{Definition}
\newtheorem{dinhly}{Định lý}
\newtheorem{dinhnghia}{Định nghĩa}
\newtheorem{example}{Example}
\newtheorem{ghichu}{Ghi chú}
\newtheorem{hequa}{Hệ quả}
\newtheorem{hypothesis}{Hypothesis}
\newtheorem{lemma}{Lemma}
\newtheorem{luuy}{Lưu ý}
\newtheorem{nhanxet}{Nhận xét}
\newtheorem{notation}{Notation}
\newtheorem{note}{Note}
\newtheorem{principle}{Principle}
\newtheorem{problem}{Problem}
\newtheorem{proposition}{Proposition}
\newtheorem{question}{Question}
\newtheorem{remark}{Remark}
\newtheorem{theorem}{Theorem}
\newtheorem{vidu}{Ví dụ}
\usepackage[left=1cm,right=1cm,top=5mm,bottom=5mm,footskip=4mm]{geometry}
\def\labelitemii{$\circ$}
\DeclareRobustCommand{\divby}{%
	\mathrel{\vbox{\baselineskip.65ex\lineskiplimit0pt\hbox{.}\hbox{.}\hbox{.}}}%
}

\title{Problem: Coordinates of Vectors in 3D Space\\Bài Tập: Tọa Độ Của Vector Trong Không Gian}
\author{Nguyễn Quản Bá Hồng\footnote{A Scientist {\it\&} Creative Artist Wannabe. E-mail: {\tt nguyenquanbahong@gmail.com}. Bến Tre City, Việt Nam.}}
\date{\today}

\begin{document}
\maketitle
\begin{abstract}
	This text is a part of the series {\it Some Topics in Elementary STEM \& Beyond}:
	
	{\sc url}: \url{https://nqbh.github.io/elementary_STEM}.
	
	Latest version:
	\begin{itemize}
		\item {\it Problem: Coordinates of Vectors in 3D Space -- Bài Tập: Tọa Độ Của Vector Trong Không Gian}.
		
		PDF: {\sc url}: \url{https://github.com/NQBH/elementary_STEM_beyond/blob/main/elementary_mathematics/grade_12/3D_vector/problem/NQBH_3D_vector_problem.pdf}.
		
		\TeX: {\sc url}: \url{https://github.com/NQBH/elementary_STEM_beyond/blob/main/elementary_mathematics/grade_12/3D_vector/problem/NQBH_3D_vector_problem.tex}.
		\item {\it Problem \& Solution: Coordinates of Vectors in 3D Space -- Bài Tập \& Lời Giải: Tọa Độ Của Vector Trong Không Gian}.
		
		PDF: {\sc url}: \url{https://github.com/NQBH/elementary_STEM_beyond/blob/main/elementary_mathematics/grade_12/3D_vector/solution/NQBH_3D_vector_solution.pdf}.
		
		\TeX: {\sc url}: \url{https://github.com/NQBH/elementary_STEM_beyond/blob/main/elementary_mathematics/grade_12/3D_vector/solution/NQBH_3D_vector_solution.tex}.
	\end{itemize}
\end{abstract}
\tableofcontents

%------------------------------------------------------------------------------%

\section{Vector \& Vector Calculus in 3D Space -- Vector \& Các Phép Toán Vector Trong Không Gian}
\cite[Chap. II, \S1, pp. 56--64]{SGK_Toan_12_Canh_Dieu_tap_1}: HD1. LT1. HD2. LT2. HD3. LT3. HD4. LT4. HD5. LT5. HD6. LT6. HD7. LT7. 1. 2. 3. 4. 5.

\begin{baitoan}[\cite{SGK_Toan_11_hinh_hoc_co_ban}, 1, p. 85]
	Cho tứ diện $ABCD$. Chỉ ra các vector có điểm đầu là $A$ \& điểm cuối là các đỉnh còn lại của hình tứ diện. Các vector đó có cùng nằm trong 1 mặt phẳng không?
\end{baitoan}

\begin{baitoan}[\cite{SGK_Toan_11_hinh_hoc_co_ban}, 2, p. 85]
	Cho hình hộp $ABCD.A'B'C'D'$. Kể tên các vector có điểm đầu \& điểm cuối là các đỉnh của hình hộp \& bằng $\overrightarrow{AB}$.
\end{baitoan}

\begin{baitoan}[\cite{SGK_Toan_11_hinh_hoc_co_ban}, Ví dụ 1, p. 86]
	Cho tứ diện $ABCD$. Chứng minh: $\overrightarrow{AC} + \overrightarrow{BD} = \overrightarrow{AD} + \overrightarrow{BC}$.
\end{baitoan}

\begin{baitoan}[\cite{SGK_Toan_11_hinh_hoc_co_ban}, 3, p. 86]
	Cho hình hộp $ABCD.EFGH$. Thực hiện các phép toán: (a) $\overrightarrow{AB} + \overrightarrow{CD} + \overrightarrow{EF} + \overrightarrow{GH}$; (b) $\overrightarrow{BE} - \overrightarrow{CH}$.
\end{baitoan}

\begin{baitoan}[\cite{SGK_Toan_11_hinh_hoc_co_ban}, Ví dụ 2, p. 87]
	Cho tứ diện $ABCD$. Gọi $M,N$ lần lượt là trung điểm của $AD,BC$, \& $G$ là trọng tâm của $\Delta BCD$. Chứng minh: (a) $\overrightarrow{MN} = \frac{1}{2}(\overrightarrow{AB} + \overrightarrow{DC})$; (b) $\overrightarrow{AB} + \overrightarrow{AC} + \overrightarrow{AD} = 3\overrightarrow{AG}$.
\end{baitoan}

\begin{baitoan}[\cite{SGK_Toan_11_hinh_hoc_co_ban}, 4, p. 87]
	Trong không gian cho 2 vector $\vec{a},\vec{b}$ đều khác vector không. Xác định các vector $\vec{m} = 2\vec{a}$, $\vec{n} = -3\vec{b}$, \& $\vec{p} = \vec{m} + \vec{n}$.
\end{baitoan}

\begin{baitoan}[\cite{SGK_Toan_11_hinh_hoc_co_ban}, Ví dụ 3, p. 88]
	Cho tứ diện $ABCD$. Gọi $M,N$ lần lượt là trung điểm của $AB,CD$. Chứng minh 3 vector $\overrightarrow{BC},\overrightarrow{AD},\overrightarrow{MN}$ đồng phẳng.
\end{baitoan}

\begin{baitoan}[\cite{SGK_Toan_11_hinh_hoc_co_ban}, 5, p. 89]
	Cho hình hộp $ABCD.EFGH$. Gọi $I,K$ lần lượt là trung điểm của $AB,BC$. Chứng minh các đường thẳng $IK,ED$ song song với mặt phẳng $(AFC)$. Từ đó suy ra 3 vector $\overrightarrow{AF},\overrightarrow{IK},\overrightarrow{ED}$ đồng phẳng.
\end{baitoan}

\begin{baitoan}[\cite{SGK_Toan_11_hinh_hoc_co_ban}, 6, p. 89]
	Cho 2 vector $\vec{a},\vec{b}$ đều khác vector $\vec{0}$. Xác định vector $\vec{c} = 2\vec{a} - \vec{b}$ \& giải thích tại sao 3 vector $\vec{a},\vec{b},\vec{c}$ đồng phẳng.
\end{baitoan}

\begin{baitoan}[\cite{SGK_Toan_11_hinh_hoc_co_ban}, 7, p. 89]
	Cho 3 vector $\vec{a},\vec{b},\vec{c}$ trong không gian. Chứng minh nếu $m\vec{a} + n\vec{b} + p\vec{c} = \vec{0}$ \& 1 số trong 3 số $m,n,p\in\mathbb{R}$ khác $0$ thì 3 vector $\vec{a},\vec{b},\vec{c}$ đồng phẳng.
\end{baitoan}

\begin{baitoan}[\cite{SGK_Toan_11_hinh_hoc_co_ban}, Ví dụ 4, p. 89]
	Cho tứ diện $ABCD$. Gọi $M,N$ lần lượt là trung điểm của $AB$ \& $CD$. Trên các cạnh $AD,BC$ lần lượt lấy các điểm $P,Q$ sao cho $\overrightarrow{AP} = \frac{2}{3}\overrightarrow{AD}$ \& $\overrightarrow{BQ} = \frac{2}{3}\overrightarrow{BC}$. Chứng minh 4 điểm $M,N,P,Q$ cùng thuộc 1 mặt phẳng.
\end{baitoan}

\begin{baitoan}[\cite{SGK_Toan_11_hinh_hoc_co_ban}, Ví dụ 5, p. 91]
	Cho hình hộp $ABCD.EFGH$ có $\overrightarrow{AB} = \vec{a}$, $\overrightarrow{AD} = \vec{b}$, $\overrightarrow{AE} = \vec{c}$. Gọi $I$ là trung điểm của $BG$. Biểu thị vector $\overrightarrow{AI}$ qua 3 vector $\vec{a},\vec{b},\vec{c}$.
\end{baitoan}

\begin{baitoan}[\cite{SGK_Toan_11_hinh_hoc_co_ban}, 1., p. 91]
	Cho hình lăng trụ tứ giác $ABCD.A'B'C'D'$. Mặt phẳng $(P0$ cắt các cạnh bên $AA',BB',CC',DD'$ lần lượt tại $I,K,L,M$. Xét các vector có các điểm đầu là các điểm $I,K,L,M$ \& có các điểm cuối là các đỉnh của hình lăng trụ. Chỉ ra các vector: (a) Cùng phương với $\overrightarrow{IA}$; (b) Cùng hướng với $\overrightarrow{IA}$; (c) Ngược hướng với $\overrightarrow{IA}$.
\end{baitoan}

\begin{baitoan}[\cite{SGK_Toan_11_hinh_hoc_co_ban}, 2., p. 91]
	Cho hình hộp $ABCD.A'B'C'D'$. Chứng minh: (a) $\overrightarrow{AB} + \overrightarrow{B'C'} + \overrightarrow{DD'} = \overrightarrow{AC'}$; (b) $\overrightarrow{BD} - \overrightarrow{D'D} - \overrightarrow{B'D'} = \overrightarrow{BB'}$; (c) $\overrightarrow{AC} + \overrightarrow{BA'} + \overrightarrow{DB} + \overrightarrow{C'D} = \vec{0}$.
\end{baitoan}

\begin{baitoan}[\cite{SGK_Toan_11_hinh_hoc_co_ban}, 3., p. 91]
	Cho hình bình hành $ABCD$. Gọi $S$ là 1 điểm nằm ngoài mặt phẳng chứa hình bình hành. Chứng minh: $\overrightarrow{SA} + \overrightarrow{SC} = \overrightarrow{SB} + \overrightarrow{SD}$
\end{baitoan}

\begin{baitoan}[\cite{SGK_Toan_11_hinh_hoc_co_ban}, 4., p. 92]
	Cho tứ diện $ABCD$. Gọi $M,N$ lần lượt là trung điểm của $AB,CD$. Chứng minh: (a) $\overrightarrow{MN} = \frac{1}{2}(\overrightarrow{AD} + \overrightarrow{BC})$; (b) $\overrightarrow{MN} = \frac{1}{2}(\overrightarrow{AC} + \overrightarrow{BD})$.
\end{baitoan}

\begin{baitoan}[\cite{SGK_Toan_11_hinh_hoc_co_ban}, 5., p. 92]
	Cho tứ diện $ABCD$. Xác định 2 điểm $E,F$ sao cho: (a) $\overrightarrow{AE} = \overrightarrow{AB} + \overrightarrow{AC} + \overrightarrow{AD}$; (b) $\overrightarrow{AF} = \overrightarrow{AB} + \overrightarrow{AC} - \overrightarrow{AD}$.
\end{baitoan}

\begin{baitoan}[\cite{SGK_Toan_11_hinh_hoc_co_ban}, 6., p. 92]
	Cho tứ diện $ABCD$. Gọi $G$ là trọng tâm của $\Delta ABC$. Chứng minh: $\overrightarrow{DA} + \overrightarrow{DB} + \overrightarrow{DC} = 3\overrightarrow{DG}$.
\end{baitoan}

\begin{baitoan}[\cite{SGK_Toan_11_hinh_hoc_co_ban}, 7., p. 92]
	Gọi $M,N$ lần lượt là trung điểm của $AC,BD$ của tứ diện $ABCD$. Gọi $I$ là trung điểm của $MN$ \& $P$ là 1 điểm bất kỳ trong không gian. Chứng minh: (a) $\overrightarrow{IA} + \overrightarrow{IB} + \overrightarrow{IC} + \overrightarrow{ID} = \vec{0}$; (b) $\overrightarrow{PI} = \frac{1}{4}(\overrightarrow{PA} + \overrightarrow{PB} + \overrightarrow{PC} + \overrightarrow{PD})$.
\end{baitoan}

\begin{baitoan}[\cite{SGK_Toan_11_hinh_hoc_co_ban}, 8., p. 92]
	Cho hình lăng trụ tam giác $ABC.A'B'C'$ có $\overrightarrow{AA'} = \vec{a}$, $\overrightarrow{AB} = \vec{b}$, $\overrightarrow{AC} = \vec{c}$. Phân tích (hay biểu thị) các vector $\overrightarrow{B'C},\overrightarrow{BC'}$ qua các vector $\vec{a},\vec{b},\vec{c}$.
\end{baitoan}

\begin{baitoan}[\cite{SGK_Toan_11_hinh_hoc_co_ban}, 9., p. 92]
	Cho $\Delta ABC$. Lấy điểm $S$ nằm ngoài mặt phẳng $(ABC)$. Trên đoạn $SA$ lấy điểm $M$ sao cho $\overrightarrow{MS} = -2\overrightarrow{MA}$ \& trên đoạn $BC$ lấy điểm $N$ sao cho $\overrightarrow{NB} = -\frac{1}{2}\overrightarrow{NC}$. Chứng minh 3 vector $\overrightarrow{AB},\overrightarrow{MN},\overrightarrow{SC}$ đồng phẳng.
\end{baitoan}

\begin{baitoan}[\cite{SGK_Toan_11_hinh_hoc_co_ban}, 10., p. 92]
	Cho hình hộp $ABCD.EFGH$. Gọi $K$ là giao điểm của $AH$ \& $DE$, $I$ là giao điểm của $BH$ \& $DF$. Chứng minh 3 vector $\overrightarrow{AC},\overrightarrow{KI},\overrightarrow{FG}$ đồng phẳng.
\end{baitoan}

%------------------------------------------------------------------------------%

\section{Coordinate of Vector -- Tọa Độ Của Vector}
\cite[Chap. II, \S2, pp. 56--73]{SGK_Toan_12_Canh_Dieu_tap_1}: HD1. LT1. HD2. LT2. HD3. LT3. HD4. LT4. HD5. LT5. HD6. LT6. 1. 2. 3. 4. 5. 6. 7. 8. 9.

\begin{baitoan}[\cite{SGK_Toan_12_hinh_hoc_co_ban}, 1., p. 68]
	Cho 3 vector $\vec{a} = (2,-5,3),\vec{b} = (0,2,-1),\vec{c} = (1,7,2)$. (a) Tính tọa độ của vector $\vec{d} = 4\vec{a} - \frac{1}{3}\vec{b} + 3\vec{c}$. (b) Tính tọa độ của vector $\vec{e} = \vec{a} - 4\vec{b} - 2\vec{c}$.
\end{baitoan}

\begin{baitoan}[\cite{SGK_Toan_12_hinh_hoc_co_ban}, 2., p. 68]
	Cho 3 điểm $A = (1,-1,1),B = (0,1,2),C = (1,0,1)$. Tìm tọa độ trọng tâm $G$ của $\Delta ABC$.
\end{baitoan}

\begin{baitoan}[\cite{SGK_Toan_12_hinh_hoc_co_ban}, 3., p. 68]
	Cho hình hộp $ABCD.A'B'C'D'$ biết $A = (1,0,1),B = (2,1,2),D = (1,-1,1),C' = (4,5,-5)$. Tính tọa độ các đỉnh còn lại của hình hộp.
\end{baitoan}

\begin{baitoan}[\cite{SGK_Toan_12_hinh_hoc_co_ban}, 4., p. 68]
	Tính: (a) $\vec{a}\cdot\vec{b}$ với $\vec{a} = (3,0,-6),\vec{b} = (2,-4,0)$. (b) $\vec{c}\cdot\vec{d}$ với $\vec{c} = (1,-5,2),\vec{d} = (4,3,-5)$.
\end{baitoan}

\begin{baitoan}[\cite{SGK_Toan_12_hinh_hoc_co_ban}, 5., p. 68]
	Tìm tâm \& bán kính của các mặt cầu có phương trình sau: (a) $x^2 + y^2 + z^2 - 8x - 2y + 1 = 0$; (b) $3x^2 + 3y^2 + 3z^2 - 6x + 8y + 15z - 3 = 0$.
\end{baitoan}

\begin{baitoan}[\cite{SGK_Toan_12_hinh_hoc_co_ban}, 6., p. 68]
	Lập phương trình mặt cầu trong 2 mặt cầu trong 2 trường hợp sau: (a) Có đường kính $AB$ với $A = (4,-3,7),B = (2,1,3)$. (b) Đi qua điểm $A = (5,-2,1)$ \& có tâm $C = (3,-3,1)$.
\end{baitoan}

%------------------------------------------------------------------------------%

\section{Vector Calculus in Coordinate System -- Biểu Thức Tọa Độ Của Các Phép Toán Vector}
\cite[Chap. II, \S3, pp. 74--81]{SGK_Toan_12_Canh_Dieu_tap_1}: HD1. LT1. HD2. LT2. HD3. LT3. HD4. LT4. 1. 2. 3. 4. 5. 6. 7. 8.

%------------------------------------------------------------------------------%

\section{Miscellaneous}
\cite[BTCCII, pp. 82--83]{SGK_Toan_12_Canh_Dieu_tap_1}: 1. 2. 3. 4. 5. 6. 7. 8. 9. 10. 11. 12. 13.14. 15.

%------------------------------------------------------------------------------%

\printbibliography[heading=bibintoc]
	
\end{document}