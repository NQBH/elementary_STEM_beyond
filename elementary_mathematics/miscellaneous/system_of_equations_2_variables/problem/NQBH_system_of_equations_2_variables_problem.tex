\documentclass{article}
\usepackage[backend=biber,natbib=true,style=alphabetic,maxbibnames=50]{biblatex}
\addbibresource{/home/nqbh/reference/bib.bib}
\usepackage[utf8]{vietnam}
\usepackage{tocloft}
\renewcommand{\cftsecleader}{\cftdotfill{\cftdotsep}}
\usepackage[colorlinks=true,linkcolor=blue,urlcolor=red,citecolor=magenta]{hyperref}
\usepackage{amsmath,amssymb,amsthm,float,graphicx,mathtools}
\allowdisplaybreaks
\newtheorem{assumption}{Assumption}
\newtheorem{baitoan}{}
\newtheorem{cauhoi}{Câu hỏi}
\newtheorem{conjecture}{Conjecture}
\newtheorem{corollary}{Corollary}
\newtheorem{dangtoan}{Dạng toán}
\newtheorem{definition}{Definition}
\newtheorem{dinhly}{Định lý}
\newtheorem{dinhnghia}{Định nghĩa}
\newtheorem{example}{Example}
\newtheorem{ghichu}{Ghi chú}
\newtheorem{hequa}{Hệ quả}
\newtheorem{hypothesis}{Hypothesis}
\newtheorem{lemma}{Lemma}
\newtheorem{luuy}{Lưu ý}
\newtheorem{nhanxet}{Nhận xét}
\newtheorem{notation}{Notation}
\newtheorem{note}{Note}
\newtheorem{principle}{Principle}
\newtheorem{problem}{Problem}
\newtheorem{proposition}{Proposition}
\newtheorem{question}{Question}
\newtheorem{remark}{Remark}
\newtheorem{theorem}{Theorem}
\newtheorem{vidu}{Ví dụ}
\usepackage[left=1cm,right=1cm,top=5mm,bottom=5mm,footskip=4mm]{geometry}
\def\labelitemii{$\circ$}
\DeclareRobustCommand{\divby}{%
	\mathrel{\vbox{\baselineskip.65ex\lineskiplimit0pt\hbox{.}\hbox{.}\hbox{.}}}%
}

\title{Problem: System of Equations of 2 Variables via Basic Calculus Operators\\Bài Tập: Hệ Phương Trình 2 Biến với Các Phép Tính Cơ Bản}
\author{Nguyễn Quản Bá Hồng\footnote{Independent Researcher, Ben Tre City, Vietnam\\e-mail: \texttt{nguyenquanbahong@gmail.com}; website: \url{https://nqbh.github.io}.}}
\date{\today}

\begin{document}
\maketitle
\begin{abstract}
	In this context, we aim to solve systems of equations of 2 variables of the following form:
	\begin{equation*}
		\left\{\begin{split}
			F(x,y) &= a,\\
			G(x,y) & = b,
		\end{split}\right.
	\end{equation*}
	where $F,G$ are basic calculus operators, e.g., $f(x,y) = x\pm y$, $f(x,y) = \alpha x + \beta y$, $f(x,y) = xy$, $f(x,y) = \frac{x}{y}$, $f(x,y) = x^y$, $f(x,y) = x^\alpha\pm y^\alpha$ for some $\alpha\in\mathbb{R}$.
	
	Last updated version: \href{https://github.com/NQBH/elementary_STEM_beyond/blob/main/elementary_mathematics/miscellaneous/system_of_equations_2_variables/problem/NQBH_system_of_equations_2_variables_problem.pdf}{GitHub{\tt/}NQBH{\tt/}elementary STEM \& beyond{\tt/}elementary mathematics{\tt/}miscellaneous{\tt/}system of equations of 2 variables{\tt/}problem: system of equations of 2 variables [pdf]}.\footnote{\textsc{url}: \url{https://github.com/NQBH/elementary_STEM_beyond/blob/main/elementary_mathematics/miscellaneous/system_of_equations_2_variables/problem/NQBH_system_of_equations_2_variables_problem.pdf}.} [\href{https://github.com/NQBH/elementary_STEM_beyond/blob/main/elementary_mathematics/miscellaneous/system_of_equations_2_variables/problem/NQBH_system_of_equations_2_variables_problem.tex}{\TeX}]\footnote{\textsc{url}: \url{https://github.com/NQBH/elementary_STEM_beyond/blob/main/elementary_mathematics/miscellaneous/system_of_equations_2_variables/problem/NQBH_system_of_equations_2_variables_problem.tex}.}.
\end{abstract}
\tableofcontents

%------------------------------------------------------------------------------%

\section{System of Equations of 2 Variables: Basic Operators -- Hệ Phương Trình 2 Ẩn: Các Toán Tử Cơ Bản $\pm,\cdot,:$}
Ta lần lượt xét các hệ phương trình 2 biến, e.g., $x,y$, có 2 phương trình trong hệ được tạo thành từ 2 trong các toán tử: toán tử cộng $+$: $f(x,y) = x + y$, toán tử trừ $-$: $f(x,y) = x - y$, toán tử nhân $\cdot$: $f(x,y) = xy$, toán tử chia $:$, i.e., $f(x,y) = \frac{x}{y}$, toán tử tổng bình phương $f(x,y) = x^2 + y^2$, toán tử hiệu bình phương $f(x,y) = x^2 - y^2$, toán tử tổng lập phương $f(x,y) = x^3 + y^3$, toán tử hiệu lập phương $f(x,y) = x^4 + y^4$, toán tử tổng lũy thừa bậc $n$ tổng quát $f_n(x,y) = x^n + y^n$, $\forall n\in\mathbb{N}$, toán tử hiệu lũy thừa bậc $n$ tổng quát $f_n(x,y) = x^n - y^n$, $\forall n\in\mathbb{N}$.

\begin{baitoan}[Hệ phương trình tổng--hiệu 2 biến]
	Giải hệ phương trình:
	\begin{equation}
		\left\{\begin{split}
			x + y &= a,\\
			x - y &= b,
		\end{split}\right.\mbox{ với }	a,b\in\mathbb{R}.
	\end{equation}
\end{baitoan}

\begin{baitoan}[Hệ phương trình tổng--tích 2 biến]
	Giải hệ phương trình:
	\begin{equation}
		\left\{\begin{split}
			x + y &= a,\\
			xy &= b,
		\end{split}\right.\mbox{ với }	a,b\in\mathbb{R}.	
	\end{equation}
\end{baitoan}

\begin{baitoan}[Hệ phương trình hiệu--tích 2 biến]
	Giải hệ phương trình:
	\begin{equation}
		\left\{\begin{split}
			x - y &= a,\\
			xy &= b,
		\end{split}\right.\mbox{ với }	a,b\in\mathbb{R}.	
	\end{equation}
\end{baitoan}

\begin{baitoan}[Hệ phương trình tổng--thương 2 biến]
	Giải hệ phương trình:
	\begin{equation}
		\left\{\begin{split}
			x + y &= a,\\
			\frac{x}{y} &= b,
		\end{split}\right.\mbox{ với }	a,b\in\mathbb{R}.
	\end{equation}
\end{baitoan}

\begin{baitoan}[Hệ phương trình hiệu--thương 2 biến]
	Giải hệ phương trình:
	\begin{equation}
		\left\{\begin{split}
			x - y &= a,\\
			\frac{x}{y} &= b,
		\end{split}\right.\mbox{ với }	a,b\in\mathbb{R}.
	\end{equation}
\end{baitoan}

\begin{baitoan}[Hệ phương trình tích--thương 2 biến]
	Giải hệ phương trình:
	\begin{equation}
		\left\{\begin{split}
			xy &= a,\\
			\frac{x}{y} &= b,
		\end{split}\right.\mbox{ với }	a,b\in\mathbb{R}.
	\end{equation}
\end{baitoan}

\begin{baitoan}[Hệ phương trình tổng--tổng bình phương 2 biến]
	Giải hệ phương trình:
	\begin{equation}
		\left\{\begin{split}
			x + y &= a,\\
			x^2 + y^2 &= b,
		\end{split}\right.\mbox{ với }	a,b\in\mathbb{R}.
	\end{equation}
\end{baitoan}

\begin{baitoan}[Hệ phương trình hiệu--tổng bình phương 2 biến]
	Giải hệ phương trình:
	\begin{equation}
		\left\{\begin{split}
			x - y &= a,\\
			x^2 + y^2 &= b,
		\end{split}\right.\mbox{ với }	a,b\in\mathbb{R}.
	\end{equation}
\end{baitoan}

\begin{baitoan}[Hệ phương trình tích--tổng bình phương 2 biến]
	Giải hệ phương trình:
	\begin{equation}
		\left\{\begin{split}
			xy &= a,\\
			x^2 + y^2 &= b,
		\end{split}\right.\mbox{ với }	a,b\in\mathbb{R}.
	\end{equation}
\end{baitoan}

\begin{baitoan}[Hệ phương trình thương--tổng bình phương 2 biến]
	Giải hệ phương trình:
	\begin{equation}
		\left\{\begin{split}
			\frac{x}{y} &= a,\\
			x^2 + y^2 &= b,
		\end{split}\right.\mbox{ với }	a,b\in\mathbb{R}.
	\end{equation}
\end{baitoan}

\begin{baitoan}[Hệ phương trình tổng--hiệu bình phương 2 biến]
	Giải hệ phương trình:
	\begin{equation}
		\left\{\begin{split}
			x + y &= a,\\
			x^2 - y^2 &= b,
		\end{split}\right.\mbox{ với }	a,b\in\mathbb{R}.
	\end{equation}
\end{baitoan}

\begin{baitoan}[Hệ phương trình hiệu--hiệu bình phương 2 biến]
	Giải hệ phương trình:
	\begin{equation}
		\left\{\begin{split}
			x - y &= a,\\
			x^2 - y^2 &= b,
		\end{split}\right.\mbox{ với }	a,b\in\mathbb{R}.
	\end{equation}
\end{baitoan}

\begin{baitoan}[Hệ phương trình tích--hiệu bình phương 2 biến]
	Giải hệ phương trình:
	\begin{equation}
		\left\{\begin{split}
			xy &= a,\\
			x^2 - y^2 &= b,
		\end{split}\right.\mbox{ với }	a,b\in\mathbb{R}.
	\end{equation}
\end{baitoan}

\begin{baitoan}[Hệ phương trình thương--hiệu bình phương 2 biến]
	Giải hệ phương trình:
	\begin{equation}
		\left\{\begin{split}
			\frac{x}{y} &= a,\\
			x^2 - y^2 &= b,
		\end{split}\right.\mbox{ với }	a,b\in\mathbb{R}.
	\end{equation}
\end{baitoan}

\section{System of Equations of 2 Variables: Advanced Operators -- Hệ Phương Trình 2 Ẩn: Các Toán Tử Nâng Cao $x^y$, $\log_xy$}
Ta lần lượt xét các hệ phương trình 2 biến, e.g., $x,y$, có 2 phương trình trong hệ được tạo thành từ 2 trong các toán tử: toán tử cộng $+$: $f(x,y) = x + y$, toán tử trừ $-$: $f(x,y) = x - y$, toán tử nhân $\cdot$: $f(x,y) = xy$, toán tử chia $:$: $f(x,y) = \frac{x}{y}$, toán tử tổng bình phương $f(x,y) = x^2 + y^2$, toán tử hiệu bình phương $f(x,y) = x^2 - y^2$, toán tử tổng lập phương $f(x,y) = x^3 + y^3$, toán tử hiệu lập phương $f(x,y) = x^4 + y^4$, toán tử tổng lũy thừa bậc $n$ tổng quát $f_n(x,y) = x^n + y^n$, $\forall n\in\mathbb{N}$, toán tử hiệu lũy thừa bậc $n$ tổng quát $f_n(x,y) = x^n - y^n$, $\forall n\in\mathbb{N}$, toán tử lũy thừa $f(x,y) = x^y$, toán tử logarith: $f(x,y) = \log_xy$.

\begin{baitoan}[Hệ phương trình tổng--lũy thừa 2 biến]
	Giải hệ phương trình:
	\begin{equation}
		\left\{\begin{split}
			x + y &= a,\\
			x^y &= b,
		\end{split}\right.\mbox{ với }	a,b\in\mathbb{R}.
	\end{equation}
\end{baitoan}

\begin{baitoan}[Hệ phương trình hiệu--lũy thừa 2 biến]
	Giải hệ phương trình:
	\begin{equation}
		\left\{\begin{split}
			x - y &= a,\\
			x^y &= b,
		\end{split}\right.\mbox{ với }	a,b\in\mathbb{R}.
	\end{equation}
\end{baitoan}

\begin{baitoan}[Hệ phương trình tích--lũy thừa 2 biến]
	Giải hệ phương trình:
	\begin{equation}
		\left\{\begin{split}
			xy &= a,\\
			x^y &= b,
		\end{split}\right.\mbox{ với }	a,b\in\mathbb{R}.
	\end{equation}
\end{baitoan}

\begin{baitoan}[Hệ phương trình thương--lũy thừa 2 biến]
	Giải hệ phương trình:
	\begin{equation}
		\left\{\begin{split}
			\frac{x}{y} &= a,\\
			x^y &= b,
		\end{split}\right.\mbox{ với }	a,b\in\mathbb{R}.
	\end{equation}
\end{baitoan}


%------------------------------------------------------------------------------%

\section{Miscellaneous}

%------------------------------------------------------------------------------%

\printbibliography[heading=bibintoc]
	
\end{document}