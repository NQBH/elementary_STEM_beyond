\documentclass{article}
\usepackage[backend=biber,natbib=true,style=alphabetic,maxbibnames=50]{biblatex}
\addbibresource{/home/nqbh/reference/bib.bib}
\usepackage[utf8]{vietnam}
\usepackage{tocloft}
\renewcommand{\cftsecleader}{\cftdotfill{\cftdotsep}}
\usepackage[colorlinks=true,linkcolor=blue,urlcolor=red,citecolor=magenta]{hyperref}
\usepackage{amsmath,amssymb,amsthm,float,graphicx,mathtools,tikz}
\usetikzlibrary{angles,calc,intersections,matrix,patterns,quotes,shadings}
\allowdisplaybreaks
\newtheorem{assumption}{Assumption}
\newtheorem{baitoan}{}
\newtheorem{cauhoi}{Câu hỏi}
\newtheorem{conjecture}{Conjecture}
\newtheorem{corollary}{Corollary}
\newtheorem{dangtoan}{Dạng toán}
\newtheorem{definition}{Definition}
\newtheorem{dinhly}{Định lý}
\newtheorem{dinhnghia}{Định nghĩa}
\newtheorem{example}{Example}
\newtheorem{ghichu}{Ghi chú}
\newtheorem{hequa}{Hệ quả}
\newtheorem{hypothesis}{Hypothesis}
\newtheorem{lemma}{Lemma}
\newtheorem{luuy}{Lưu ý}
\newtheorem{nhanxet}{Nhận xét}
\newtheorem{notation}{Notation}
\newtheorem{note}{Note}
\newtheorem{principle}{Principle}
\newtheorem{problem}{Problem}
\newtheorem{proposition}{Proposition}
\newtheorem{question}{Question}
\newtheorem{remark}{Remark}
\newtheorem{theorem}{Theorem}
\newtheorem{vidu}{Ví dụ}
\usepackage[left=1cm,right=1cm,top=5mm,bottom=5mm,footskip=4mm]{geometry}
\def\labelitemii{$\circ$}
\DeclareRobustCommand{\divby}{%
	\mathrel{\vbox{\baselineskip.65ex\lineskiplimit0pt\hbox{.}\hbox{.}\hbox{.}}}%
}

\title{Problem: Algebraic {\it\&} Rational Fractions\\Bài Tập: Phân Thức Đại Số {\it\&} Phân Thức Đại Số Hữu Tỷ}
\author{Nguyễn Quản Bá Hồng\footnote{Independent Researcher, Ben Tre City, Vietnam\\e-mail: \texttt{nguyenquanbahong@gmail.com}; website: \url{https://nqbh.github.io}.}}
\date{\today}

\begin{document}
\maketitle
\tableofcontents

%------------------------------------------------------------------------------%

\section{Tính Chất Cơ Bản của Phân Thức. Rút Gọn Phân Thức}

\begin{baitoan}[\cite{Tuyen_Toan_8}, VD20, p. 28]
	(a) Cho $x,y\in\mathbb{R}$ thỏa $\dfrac{xy}{x^2 + y^2} = \dfrac{5}{8}$. Rút gọn phân thức $A = \dfrac{x^2 - 2xy + y^2}{x^2 + 2xy + y^2}$. (b) Cho $a,b,c,d,x,y,\alpha\in\mathbb{R}$ thỏa $\dfrac{xy}{x^2 + y^2} = \alpha$. Rút gọn phân thức $B = \dfrac{ax^2 + bxy + ay^2}{cx^2 + dxy + cy^2}$.
\end{baitoan}

\begin{baitoan}[\cite{Tuyen_Toan_8}, 141., p. 29]
	So sánh: (a) $\dfrac{201 - 200}{201 + 200}$ \& $\dfrac{201^2 - 200^2}{201^2 + 200^2}$. (b) $\dfrac{1999\cdot 4001 + 2000}{2000\cdot 4001 - 2001}$ \& $\dfrac{1501\cdot 1503 - 1500\cdot 1498}{6002}$.
\end{baitoan}

\begin{baitoan}[Mở rộng \cite{Tuyen_Toan_8}, 141a., p. 29]
	Biện luận theo các tham số $a,b\in\mathbb{R}$ để so sánh $A = \dfrac{a - b}{a + b}$ \& $B = \dfrac{a^2 - b^2}{a^2 + b^2}$.
\end{baitoan}

\begin{baitoan}[\cite{Tuyen_Toan_8}, 142., p. 29]
	Chứng minh: $\forall n\in\mathbb{N},n > 1$: (a) $A = \dfrac{n^3 - 1}{n^5 + n + 1}$ không tối giản. (b) $B = \dfrac{6n + 1}{8n + 1}$ tối giản. (c) $C = \dfrac{10n^2 + 9n + 4}{20n^2 + 20n + 9}$ tối giản. (d) Có thể mở rộng từ $\mathbb{N}$ lên $\mathbb{Z}$ được không?
\end{baitoan}

\begin{baitoan}[\cite{Tuyen_Toan_8}, 143., p. 29]
	Viết mỗi đa thức sau dưới dạng 1 phân thức đại số với tử \& mẫu là những đa thức có 2 hạng tử: (a) $A = \sum_{i=0}^{19} x^i = x^{19} + x^{18} + x^{17} + \cdots + x + 1$. (b) $B = (x + 1)(x^2 + 1)(x^4 + 1)\cdots(x^{32} + 1)$.	
\end{baitoan}
Rút gọn phân thức:

\begin{baitoan}[\cite{Tuyen_Toan_8}, 144., p. 29]
	(a) $A = \dfrac{n!}{(n - 1)!(n + 1)}$. (b) $\dfrac{(n + 1)! - n!}{(n + 1)! + n!}$.	
\end{baitoan}

\begin{baitoan}[\cite{Tuyen_Toan_8}, 145., p. 29]
	(a) $A = \dfrac{(x^2 - y)(y + 1) + x^2y^2 - 1}{(x^2 + y)(y + 1) + x^2y^2 + 1}$. (b) $B = \dfrac{x^2(y - z) + y^2(z - x) + z^2(x - y)}{x^2y - x^2z + y^2z - y^3}$.	
\end{baitoan}

\begin{baitoan}[\cite{Tuyen_Toan_8}, 146., p. 29]
	(a) $\dfrac{x^4 - 4x^2 + 3}{x^4 + 6x^2 - 7}$. (b) $\dfrac{x^4 + x^3 - x - 1}{x^4 + x^3 + 2x^2 + x + 1}$. (c) $\dfrac{x^3 + 3x^2 - 4}{x^3 - 3x + 2}$.	
\end{baitoan}

\begin{baitoan}[\cite{Tuyen_Toan_8}, 147., p. 29]
	(a) $\dfrac{x^3 + x^2 - 4x - 4}{x^3 + 8x^2 + 17x + 10}$. (b) $\dfrac{x^4 + 6x^3 + 9x^2 - 1}{x^4 + 6x^3 + 7x^2 - 6x + 1}$.	
\end{baitoan}

\begin{baitoan}[\cite{Tuyen_Toan_8}, 148., p. 29]
	Cho $\dfrac{x}{a} = \dfrac{y}{b} = \dfrac{z}{c}$. Rút gọn phân thức $A = \dfrac{x^2 + y^2 + z^2}{(ax + by + cz)^2}$.
\end{baitoan}

\begin{baitoan}[\cite{Tuyen_Toan_8}, 149., p. 30]
	Cho $x,y,z\in\mathbb{R}^\star,x + y + z = 0$. Rút gọn phân thức: (a) $A = \dfrac{x^2 + y^2 + z^2}{(x - y)^2 + (y - z)^2 + (z - x)^2}$. (b) $B = \dfrac{(x^2 + y^2 - z^2)(y^2 + z^2 - x^2)(z^2 + x^2 - y^2)}{16xyz}$.	
\end{baitoan}

\begin{baitoan}[\cite{Tuyen_Toan_8}, 150., p. 30]
	Cho $x^3 + y^3 + z^3 = 3xyz$. Rút gọn phân thức $A = \dfrac{xyz}{(x + y)(y + z)(z + x)}$.
\end{baitoan}

\begin{baitoan}[\cite{Binh_Toan_8_tap_1}, VD28, p. 18]
	Cho phân thức $A = \dfrac{(a^2 + b^2 + c^2)(a + b + c)^2 + (ab + bc + ca)^2}{(a + b + c)^2 - (ab + bc + ca)^2}$. (a) Tìm {\rm ĐKXĐ}. (b) Rút gọn A.
\end{baitoan}

\begin{baitoan}[\cite{Binh_Toan_8_tap_1}, VD29, p. 19]
	Rút gọn phân thức $A = \dfrac{(b - c)^3 9 (c - a)^3 + (a - b)^3}{a^2(b - c) + b^2(c - a) + c^2(a - b)}$.
\end{baitoan}

\begin{baitoan}[\cite{Binh_Toan_8_tap_1}, VD30, p. 19]
	Chứng minh phân số $\dfrac{n^3 + 2n}{n^4 + 3n^2 + 1}$ tối giản, $\forall n\in\mathbb{Z}$.
\end{baitoan}

\begin{baitoan}[\cite{Binh_Toan_8_tap_1}, VD28, p. 19]
	(a) Chứng minh $\sum_{i=0}^{31} x^i = 1 + x + x^2 + \cdots + x^{31} = (1 + x)(1 + x^2)(1 + x^4)(1 + x^8)(1 + x^{16})$.
\end{baitoan}

\begin{baitoan}[\cite{Binh_Toan_8_tap_1}, 106., p. 20]
	Tìm $x\in\mathbb{R}$ thỏa: (a) $\dfrac{x^4 + x^3 + x + 1}{x^4 - x^3 + 2x^2 - x + 1}$. (b) $\dfrac{x^4 - 5x^2 + 4}{x^4 - 10x^2 + 9}$.
\end{baitoan}
Rút gọn phân thức:

\begin{baitoan}[\cite{Binh_Toan_8_tap_1}, 107., p. 20]
	(a) $A = \dfrac{1235\cdot2469 - 1234}{1234\cdot2469 + 1235}$. (b) $B = \dfrac{4002}{1000\cdot1002 - 999\cdot1001}$.
\end{baitoan}

\begin{baitoan}[\cite{Binh_Toan_8_tap_1}, 108., p. 20]
	(a) $\dfrac{3x^3 - 7x^2 + 5x - 1}{2x^3 - x^2 - 4x + 3}$. (b) $\dfrac{(x - y)^3 - 3xy(x + y) + y^3}{x - 6y}$. (c) $\dfrac{x^2 + y^2 + z^2 - 2xy - 2yz + 2zx}{x^2 - 2xy + y^2 - z^2}$.
\end{baitoan}

\begin{baitoan}[\cite{Binh_Toan_8_tap_1}, 109., p. 20]
	$\forall n\in\mathbb{N}$: (a) $\dfrac{(n + 1)!}{n!(n + 2)}$. (b) $\dfrac{n!}{(n + 1)! - n!}$. (c) $\dfrac{(n + 1)! - (n + 2)!}{(n + 1)! + (n + 2)!}$.
\end{baitoan}

\begin{baitoan}[\cite{Binh_Toan_8_tap_1}, 110., p. 20]
	(a) $\dfrac{a^2(b - c) + b^2(c - a) + c^2(a - b)}{ab^2 - ac^2 - b^3 + bc^2}$. (b) $\dfrac{2x^3 - 7x^2 - 12x + 45}{3x^3 - 19x^2 + 33x - 9}$. (c) $\dfrac{x^3 - y^3 + z^3 + 3xyz}{(x + y)^2 + (y + z)^2 + (z - x)^2}$. (d) $\dfrac{x^3 + y^3 + z^3 - 3xyz}{(x - y)^2 + (y - z)^2 + (z - x)^2}$.
\end{baitoan}

\begin{baitoan}[\cite{Binh_Toan_8_tap_1}, 111., p. 20]
	Chứng minh phân số tối giản $\forall n\in\mathbb{N}$: (a) $\dfrac{3n + 1}{5n + 2}$. (b) $\dfrac{12n + 1}{30n + 2}$. (c) $\dfrac{n^3 + 2n}{n^4 + 3n^2 + 1}$. (d) $\dfrac{2n + 1}{2n^2 - 1}$.
\end{baitoan}

\begin{baitoan}[\cite{Binh_Toan_8_tap_1}, 112., p. 20]
	Chứng minh phân số $\dfrac{n^7 + n^2 + 1}{n^8 + n + 1}$ không tối giản $\forall n\in\mathbb{N}$.
\end{baitoan}

\begin{baitoan}[\cite{Binh_Toan_8_tap_1}, 113., p. 20]
	Viết gọn biểu thức $(x^2 - x + 1)(x^4 - x^2 + 1)(x^8 - x^4 + 1)(x^{16} - x^8 + 1)(x^{32} - x^{16} + 1)$ dưới dạng 1 phân thức.
\end{baitoan}

\begin{baitoan}[\cite{Binh_Toan_8_tap_1}, 114., p. 20]
	Cho $x,y,z\in\mathbb{R}^\star,\dfrac{(ax + by + cz)^2}{x^2 + y^2 + z^2} = a^2 + b^2 + c^2$. Chứng minh $\dfrac{a}{x} = \dfrac{b}{y} = \dfrac{c}{z}$.
\end{baitoan}

\begin{baitoan}[\cite{Binh_Toan_8_tap_1}, 115., p. 20]
	Cho biết $ax + by + cz = 0$. Rút gọn $A = \dfrac{bc(y - z)^2 + ca(z - x)^2 + ab(x - y)^2}{ax^2 + by^2 + cz^2}$.
\end{baitoan}

\begin{baitoan}[\cite{Binh_Toan_8_tap_1}, 116., p. 20]
	Rút gọn $\dfrac{x^2 + y^2 + z^2}{(y - z)^2 + (z - x)^2 + (x - y)^2}$ biết $x + y + z = 0$.
\end{baitoan}

\begin{baitoan}[\cite{Binh_Toan_8_tap_1}, 117., p. 21]
	Tính giá trị biểu thức $A = \dfrac{x - y}{x + y}$ biết $x^2 - 2y^2 = xy,y\ne0,x + y\ne0$.
\end{baitoan}

\begin{baitoan}[\cite{Binh_Toan_8_tap_1}, 118., p. 21]
	Tính giá trị biểu thức $A = \dfrac{3x - 2y}{3x + 2y}$ biết $9x^2 + 4y^2 = 20xy,2y < 3x < 0$.
\end{baitoan}

\begin{baitoan}[\cite{Binh_Toan_8_tap_1}, 119., p. 21]
	Cho $x,y\in\mathbb{R}^\star,3x - y = 3z,2x + y = 7z$. Tính giá trị biểu thức $A = \dfrac{x^2 - 2xy}{x^2 + y^2}$.
\end{baitoan}

\begin{baitoan}[\cite{Binh_Toan_8_tap_1}, 120., p. 21]
	Tìm $x\in\mathbb{Z}$ để phân thức có giá trị nguyên: (a) $\dfrac{3}{2x - 1}$. (b) $\dfrac{5}{x^2 + 1}$. (c) $\dfrac{7}{x^2 - x + 1}$. (d) $\dfrac{x^2 - 59}{x + 8}$. (e) $\dfrac{x + 2}{x^2 + 4}$. (f) Mở rộng.
\end{baitoan}

\begin{baitoan}[\cite{Binh_Toan_8_tap_1}, 121., p. 21]
	Tìm $x\in\mathbb{Q}$ để phân thức $\dfrac{10}{x^2 + 1}\in\mathbb{Z}$.
\end{baitoan}

\begin{baitoan}[\cite{Binh_Toan_8_tap_1}, 122., p. 21]
	Chứng minh nếu 3 chữ số $a,b,c\ne0$ thỏa $\overline{ab}:\overline{bc} = a:c$ thì $\overline{abbb}:\overline{bbbc} = a:c$.
\end{baitoan}

\begin{baitoan}[\cite{Binh_Toan_8_tap_1}, 123., p. 21]
	Điểm trung bình môn Toán của các học sinh nam \& nữ 2 lớp 8A, 8B được thống kê ở bảng:
	\begin{table}[H]
		\centering
		\begin{tabular}{|c|c|c|c|}
			\hline
			& Lớp 8A & Lớp 8B & Cả 2 lớp 8A, 8B \\
			\hline
			Nam & $7.1$ & $8.1$ & $7.9$ \\
			\hline
			Nữ & $7.6$ & $9.0$ &  \\
			\hline
			Cả lớp & $7.4$ & $8.4$ &  \\
			\hline
		\end{tabular}
	\end{table}
	\noindent Tính điểm trung bình môn Toán của các học sinh của cả 2 lớp 8A, 8B.
\end{baitoan}

\begin{baitoan}[\cite{TLCT_THCS_Toan_8_dai_so}, VD5.1, p. 39]
	Dùng định nghĩa 2 phân thức bằng nhau, chứng minh 2 phân thức sau bằng nhau: $\dfrac{a^2 - 2ab - 3b^2}{a^2 - 4ab + 3b^2}$ \& $\dfrac{a + b}{a - b}$ với $a\ne b$ \& $a\ne 3b$.
\end{baitoan}

\begin{baitoan}[\cite{TLCT_THCS_Toan_8_dai_so}, VD5.2, p. 39]
	Dùng định nghĩa 2 phân thức bằng nhau, xét sự bằng nhau của 2 phân thức $\dfrac{(3x + 2)(x + 5)}{4(3x + 2)}$ \& $\dfrac{x + 5}{4}$ trong các trường hợp biến $x$ thuộc các tập hợp: (a) $x\in\mathbb{N}$. (b) $x\in\mathbb{Z}$. (c) $x\in\mathbb{Q}$.
\end{baitoan}

\begin{baitoan}[\cite{TLCT_THCS_Toan_8_dai_so}, VD5.3, p. 39]
	So sánh $A = \dfrac{2013^2 - 2012^2}{2013^2 + 2012^2}$ với $B = \dfrac{2013 - 2012}{2013 + 2012}$.
\end{baitoan}

\begin{baitoan}[\cite{TLCT_THCS_Toan_8_dai_so}, VD5.4, p. 40]
	Chứng minh: $\sum_{i=0}^{63} a^i = \prod_{i=0}^{5} (1 + a^{2^i})$, i.e., $1 + a + a^2 + \cdots + a^{63} = (1 + a)(1 + a^2)(1 + a^4)\cdots(1 + a^{32})$.
\end{baitoan}

\begin{baitoan}[\cite{TLCT_THCS_Toan_8_dai_so}, VD5.5, p. 40]
	Rút gọn phân thức $A = \dfrac{x^3 - 7x + 6}{x^3 + 5x^2 - 2x - 24}$.
\end{baitoan}

\begin{baitoan}[\cite{TLCT_THCS_Toan_8_dai_so}, VD5.6, p. 40]
	Rút gọn phân thức $A = \dfrac{a^{30} + a^{20} + a^{10} + 1}{a^{2042} + a^{2032} + a^{2022} + a^{2012} + a^{30} + a^{20} + a^{10} + 1}$.
\end{baitoan}

\begin{baitoan}[\cite{TLCT_THCS_Toan_8_dai_so}, 5.1, p.. 41]
	Dùng định nghĩa 2 phân thức bằng nhau, tìm đa thức $A$ trong các trường hợp: (a) $\dfrac{A}{3x - 2} = \dfrac{15x^2 + 10x}{9x^2 - 4}$. (b) $\dfrac{3x^2 - 5x - 2}{A} = \dfrac{x - 2}{2x - 3}$. (c) $\dfrac{x^2 - 4}{x^2 + x - 6} = \dfrac{x^2 + 4x + 4}{A}$. (d) $\dfrac{2x + 1}{x^3 + x^2 - x + 2} = \dfrac{A}{x^3 + 1}$.
\end{baitoan}

\begin{baitoan}[\cite{TLCT_THCS_Toan_8_dai_so}, 5.2, p.. 41]
	Biến đổi mỗi phân thức sau thành 1 phân thức bằng nó \& có tử thức là đa thức $B$ cho sau đây: (a) $\dfrac{2x - 5}{3x^2 + 4}$ \& $B = 2x^2 - 3x - 5$. (b) $\dfrac{(x + 1)(x^2 + x - 6)}{(x^2 - 9)(x^2 + 3x + 2)}$ \& $B = x - 2$.
\end{baitoan}

\begin{baitoan}[\cite{TLCT_THCS_Toan_8_dai_so}, 5.3, p.. 41]
	Rút gọn biểu thức: (a) $\dfrac{2^{18}\cdot 54^3 + 15\cdot 4^{10}\cdot 9^4}{2\cdot 12^9 + 6^{10}\cdot 2^{10}}$. (b) $\dfrac{4^{15}\cdot 27^6\cdot 42 - 3\cdot 72^{10}}{4^4\cdot 25\cdot 36^{10}  - 4^5\cdot 6^{19}\cdot 35}$. (c) $\dfrac{880\cdot(15^2\cdot 3^{18} + 27^7)}{4^2\cdot 15^4\cdot 3^{16} - 2^4\cdot 9^{11}}$.
\end{baitoan}

\begin{baitoan}[\cite{TLCT_THCS_Toan_8_dai_so}, 5.4, p.. 41]
	Rút gọn: (a) $M = \dfrac{4024\cdot 2014 - 2}{2011 + 2012\cdot 2013}$. (b) $N = \dfrac{2012\cdot 2013 + 2014}{2010 - 2012\cdot 2015}$. (c) $P = \dfrac{66666\cdot 87564 - 33333}{22222\cdot 87560 + 77777}$.
\end{baitoan}

\begin{baitoan}[\cite{TLCT_THCS_Toan_8_dai_so}, 5.5, p.. 41]
	Rút gọn phân thức: (a) $Q = \dfrac{x^2 + 2x - 8}{x^2 + x - 12}$. (b) $R = \dfrac{3x^2 + 5xy - 2y^2}{3x^2 - 7xy + 2y^2}$. (c) $S = \dfrac{x^6 - 14x^4 + 49x^2 - 36}{x^4 + 4x^3 - x^2 - 16x - 12}$. (d) $T = \dfrac{x^6 - y^6}{x^6 + 2x^4y^2 + 2x^2y^4 + y^6}$.	
\end{baitoan}

\begin{baitoan}[\cite{TLCT_THCS_Toan_8_dai_so}, 5.6, p.p. 41--42]
	Rút gọn: (a) $A = \dfrac{a^4 - 5a^2 + 4}{a^4 - a^2 + 4a - 4}$. (b) $B = \dfrac{a^3 - 3a + 2}{2a^3 - 7a^2 + 8a - 3}$. (c) $C = \dfrac{a^2 - 2ab + b^2 - c^2}{a^2 + b^2 + c^2 - 2ab - 2bc + 2ca}$. (d) $D = \dfrac{a^3 - 7a + 6}{a^2(a + 3)^3 - 4a(a + 3)^3 + 4(a + 3)^3}$. (e) $E = \dfrac{a^3 + b^3 + c^3 - 3abc}{(a - b)^2 + (b - c)^2 + (c - a)^2}$.	
\end{baitoan}

\begin{baitoan}[\cite{TLCT_THCS_Toan_8_dai_so}, 5.7, p.. 42]
	Rút gọn phân thức: (a) $A = \dfrac{xy^2 - xz^2 - y^3 + yz^2}{x^2(z - y) + y^2(x - z) + z^2(y - x)}$. (b) $B = \dfrac{x^4(y^2 - z^2) + y^4(z^2 - x^2) + z^4(x^2 - y^2)}{x^2(y - z) + y^2(z - x) + z^2(x - y)}$.	
\end{baitoan}

\begin{baitoan}[\cite{TLCT_THCS_Toan_8_dai_so}, 5.8, p.. 42]
	Rút gọn phân thức: (a) $A = \dfrac{(x + y + z)^2 - 3xy - 3yz - 3zx}{9xyz - 3x^3 - 3y^3 - 3z^3}$. (b) $B = \dfrac{x^3 - y^3 + z^3 + 3xyz}{(x + y)^2 + (y + z)^2 + (z - x)^2}$. (c) $C = \dfrac{(x - y)^3 + (y - z)^3 + (z - x)^3}{(x^2 - y^2)^3 + (y^2 - z^2)^3 + (z^2 - x^2)^3}$.	
\end{baitoan}

\begin{baitoan}[\cite{TLCT_THCS_Toan_8_dai_so}, 5.9, p.. 42]
	Rút gọn phân thức với $n\in\mathbb{N}^\star$: (a) $\dfrac{(n + 2)!}{n!(n + 2)(n + 3)}$. (b) $\dfrac{n!}{n! + (n - 1)!}$. (c) $\dfrac{(n + 3)! - (n + 2)!}{(n + 2)! + (n + 3)!}$.	
\end{baitoan}

\begin{baitoan}[\cite{TLCT_THCS_Toan_8_dai_so}, 5.10, p.. 42]
	Chứng minh các phân số sau là tối giản $\forall n\in\mathbb{N}$: (a) $\dfrac{3n + 2}{4n + 3}$. (b) $\dfrac{12n + 1}{2(10n + 1)}$. (c) $\dfrac{2n + 3}{2n^2 + 4n + 1}$.	
\end{baitoan}

\begin{baitoan}[\cite{TLCT_THCS_Toan_8_dai_so}, 5.11, p.. 42]
	Chứng minh phân số $\dfrac{n^7 + 2n^2 + n + 2}{n^8 + n^2 + 2n + 2}$ không tối giản, $\forall n\in\mathbb{N}^\star$.
\end{baitoan}

\begin{baitoan}[\cite{TLCT_THCS_Toan_8_dai_so}, 5.12, p.. 42]
	Viết gọn biểu thức sau dưới dạng 1 phân thức: $P = (x^4 - x^2 + 1)(x^8 - x^4 + 1)(x^{16} - x^8 + 1)(x^{32} + x^{16} + 1)$.
\end{baitoan}

\begin{baitoan}[\cite{TLCT_THCS_Toan_8_dai_so}, 5.13, p.. 42]
	Rút gọn phân thức: (a) $\dfrac{|x - 2| + |x - 1| + x}{2x^2 - 7x + 3}$ với $x < 1$. (b) $\dfrac{|x - 4||x - 5|}{x^3 - 9x^2 + 20x}$ với $4 < x < 5$.	
\end{baitoan}

\begin{baitoan}[\cite{TLCT_THCS_Toan_8_dai_so}, 5.14, p.. 43]
	Rút gọn phân thức: (a) $T = \dfrac{(x + 2)(x + 3)(x + 4)(x + 5) + 1}{x^2 + 7x + 11}$. (b) $U = \dfrac{x^3 - 53x + 88}{(x - 1)(x - 3)(x - 5)(x - 7) + 16}$.	
\end{baitoan}

\begin{baitoan}[\cite{TLCT_THCS_Toan_8_dai_so}, 5.15, p.. 43]
	Cho $\dfrac{a}{x} = \dfrac{b}{y} = \dfrac{c}{z}$ \& $x,y,z\ne 0$. Chứng minh: $\dfrac{x^2 + y^2 + z^2}{(ax + by + cz)^2} = \dfrac{1}{a^2 + b^2 + c^2}$.
\end{baitoan}

\begin{baitoan}[\cite{TLCT_THCS_Toan_8_dai_so}, 5.16, p.. 43]
	Cho $ax + by + cz = 0$. Rút gọn phân thức: $V = \dfrac{ax^2 + by^2 + cz^2}{bc(y - z)^2 + ca(z - x)^2 + ab(x - y)^2}$.
\end{baitoan}

\begin{baitoan}[\cite{TLCT_THCS_Toan_8_dai_so}, 5.17, p.. 43]
	Cho $x + y + z = 0$. Chứng minh: $\dfrac{9(x^2 + y^2 + z^2)}{(x - y)^2 + (y - z)^2 + (z - x)^2} = 3$.
\end{baitoan}

\begin{baitoan}[\cite{TLCT_THCS_Toan_8_dai_so}, 5.18, p.. 43]
	Chứng minh: $\dfrac{x^2 + y^2 - z^2 - 2zt + 2xy - t^2}{x + y - z - t} = \dfrac{x^2 - y^2 + z^2 - 2zt + 2xz - t^2}{x - y + z - t}$.
\end{baitoan}

\begin{baitoan}[\cite{TLCT_THCS_Toan_8_dai_so}, 5.19., p. 43]
	Rút gọn: $X = \dfrac{(2^4 + 4)(6^4 + 4)(10^4 + 4)(14^4 + 4)}{(4^4 + 4)(8^4 + 4)(12^4 + 4)(16^4 + 4)}$.
\end{baitoan}

%------------------------------------------------------------------------------%

\section{Operations $\pm$ on Algebraic Fractions -- Phép $\pm$ Các Phân Thức Đại Số}

\begin{baitoan}[\cite{Tuyen_Toan_8}, VD21, p. 30]
	Tính: $A = \dfrac{x^2}{(x - y)(x - z)} + \dfrac{y^2}{(y - z)(y - x)} + \dfrac{z^2}{(z - x)(z - y)}$.
\end{baitoan}

\begin{baitoan}[\cite{Tuyen_Toan_8}, VD22, p. 31]
	Tính hợp lý: (a) $A(x,n) = \sum_{i=0}^n \dfrac{1}{(x + i)(x + i + 1)} = \dfrac{1}{x(x + 1)} + \dfrac{1}{(x + 1)(x + 2)} + \dfrac{1}{(x + 1)(x + 2)} + \cdots + \dfrac{1}{(x + n)(x + n + 1)}$, $\forall n\in\mathbb{N}$. (b) $A(x,99)$.
\end{baitoan}

\begin{baitoan}[\cite{Tuyen_Toan_8_old}, VD18, p. 41]
	Tính: $A = \dfrac{x^2 - yz}{(x + y)(x + z)} + \dfrac{y^2 - zx}{(y + z)(y + x)} + \dfrac{z^2 - xy}{(z + x)(z + y)}$.
\end{baitoan}

\begin{baitoan}[\cite{Tuyen_Toan_8}, 151., pp. 31--32]
	Tính: (a) $\dfrac{x^2}{(x - y)^2(x + y)} - \dfrac{2xy^2}{x^4 - 2x^2y^2 + y^4} + \dfrac{y^2}{(x^2 - y^2)(x + y)}$. (b) $\dfrac{1}{x - 1} - \dfrac{1}{x + 1} - \dfrac{2}{x^2 + 1} - \dfrac{4}{x^4 + 1} - \dfrac{8}{x^8 + 1} - \dfrac{16}{x^{16} + 1}$. (c) Mở rộng.
\end{baitoan}

\begin{baitoan}[\cite{Tuyen_Toan_8}, 152., p. 32]
	Tính: (a) $A = \dfrac{2}{x - y} + \dfrac{2}{y - z} + \dfrac{2}{z - x} + \dfrac{(x - y)^2 + (y - z)^2 + (z - x)^2}{(x - y)(y - z)(z - x)}$. (b) $B = \dfrac{yz}{(x + y)(y + z)} + \dfrac{zx}{(y + z)(y + x)} + \dfrac{xy}{(z + x)(z + y)} + \dfrac{2xyz}{(x + y)(y + z)(z + x)}$.
\end{baitoan}

\begin{baitoan}[\cite{Tuyen_Toan_8}, 153., p. 32]
	(a) Tính $A = \dfrac{a}{x^2 + ax} + \dfrac{a}{x^2 + 3ax + 2a^2} + \dfrac{a}{x^2 + 5ax + 6a^2} + \cdots + \dfrac{a}{x^2 + 19ax + 90a^2} + \dfrac{1}{x + 10a}$. (b) Mở rộng.
\end{baitoan}

\begin{baitoan}[\cite{Tuyen_Toan_8_old}, 162., p. 42]
	Tính: (a) $\dfrac{1}{x(x - y)(x - z)} + \dfrac{1}{y(y - x)(y - z)} + \dfrac{1}{z(z - x)(z - y)}$. (b) $\dfrac{1}{(y - z)(x^2 + xz - y^2 - yz)} + \dfrac{1}{(z - x)(y^2 + xy - z^2 - zx)} + \dfrac{1}{(x - y)(z^2 + yz - x^2 - xy)}$.
\end{baitoan}

\begin{baitoan}[\cite{Tuyen_Toan_8}, 154., p. 32]
	Cho $A = 1 + \dfrac{1}{x} + \dfrac{x + 1}{xy} + \dfrac{(x + 1)(y + 1)}{xyz} + \dfrac{(x + 1)(y + 1)(z + 1)}{xyzt}$. Chứng minh có thể viết A dưới dạng 1 phân thức có tử \& mẫu đều là tích của 4 nhân tử.
\end{baitoan}

\begin{baitoan}[\cite{Tuyen_Toan_8_old}, 167., p. 43]
	Cho $xy = a$, $yz = b$, $zx = c$ với $a,b,c\in\mathbb{R}^\star$. Tính $x^2 + y^2 + z^2$.
\end{baitoan}

\begin{baitoan}[\cite{Tuyen_Toan_8}, 155., p. 32]
	Cho $\dfrac{x}{y + z} + \dfrac{y}{z + x} + \dfrac{z}{x + y} = 1$. Tính $S = \dfrac{x^2}{y + z} + \dfrac{y^2}{z + x} + \dfrac{z^2}{x + y}$.
\end{baitoan}

\begin{baitoan}[\cite{Tuyen_Toan_8}, 156., p. 32]
	Cho $x,y,z\in\mathbb{R}^\star,x + y + z = 0$. Tính: (a) $A = \dfrac{x^2}{x^2 - y^2 - z^2} + \dfrac{y^2}{y^2 - z^2 - x^2} + \dfrac{z^2}{z^2 - x^2 - y^2}$. (b) $B = \dfrac{1}{x^2 + y^2 - z^2} + \dfrac{1}{y^2 + z^2 - x^2} + \dfrac{1}{z^2 + x^2 - y^2}$.
\end{baitoan}

\begin{baitoan}[\cite{Tuyen_Toan_8}, 157., p. 32]
	Cho $x,y,z\in\mathbb{R}$ thỏa $\dfrac{x}{y} - \dfrac{y}{z} - \dfrac{z}{x} = \dfrac{y}{x} - \dfrac{z}{y} - \dfrac{x}{z}$. Chứng minh trong 3 số $x,y,z$ tồn tại 2 số bằng nhau hoặc đối nhau.
\end{baitoan}

\begin{baitoan}[\cite{Tuyen_Toan_8}, 159., p. 32]
	Cho $\dfrac{A}{x + 1} + \dfrac{B}{x - 2} = \dfrac{32x - 19}{x^2 - x - 2}$. Tính $AB$.
\end{baitoan}

\begin{baitoan}[\cite{Tuyen_Toan_8}, 159., p. 32]
	1 tổ dự định sản xuất $x$ sản phẩm trong $12$ giờ. Nhưng thực tế trong $9$ giờ tổ đã sản xuất vượt mức dự định là $3$ sản phẩm. Viết biểu thức biểu diễn số sản phẩm đó tổ đó sản xuất vượt dự định trong mỗi giờ.
\end{baitoan}

%------------------------------------------------------------------------------%

\section{Operations $\cdot,:$ on Algebraic Fractions -- Phép $\cdot,:$ Các Phân Thức Đại Số}

\begin{baitoan}[\cite{Tuyen_Toan_8}, VD23, p. 33]
	(a) Chứng minh $A = \left(1 - \dfrac{3}{2\cdot4}\right)\left(1 - \dfrac{3}{3\cdot5}\right)\left(1 - \dfrac{3}{4\cdot6}\right)\cdots\left(1 - \dfrac{3}{n(n + 2)}\right) > \dfrac{1}{4}$, $\forall n\in\mathbb{N},n\ge2$. (b) Mở rộng.
\end{baitoan}

\begin{baitoan}[\cite{Tuyen_Toan_8}, VD24, p. 33]
	Cho $A = \dfrac{x - y}{x + y},B = \dfrac{y - z}{y + z},C = \dfrac{z - x}{z + x}$. Chứng minh $(1 + A)(1 + B)(1 + C) = (1 - A)(1 - B)(1 - C)$.
\end{baitoan}

\begin{baitoan}[\cite{Tuyen_Toan_8}, 160., p. 34]
	Tính: (a) $\dfrac{x^2 + x - 6}{x^2 + 4x + 3}\cdot\dfrac{x^2 - 4x - 5}{x^2 - 10x + 25}$. (b) $\dfrac{x(y^2 - z) + y(x - xy)}{(x - y)^2 + (y - z)^2 + (z - x)^2}:\dfrac{xy^2 - xz(2y - z)}{2(x^3 + y^3 + z^3 - 3xyz)}$.
\end{baitoan}

\begin{baitoan}[\cite{Tuyen_Toan_8}, 161., p. 34]
	Tính: (a) $A = \prod_{i=2}^n 1 - \dfrac{1}{i^2} = \left(1 - \dfrac{1}{2^2}\right)\left(1 - \dfrac{1}{3^2}\right)\cdots\left(1 - \dfrac{1}{n^2}\right)$, $\forall n\in\mathbb{N},n\ge2$. (b) $B = \dfrac{1^4 + 4}{3^4 + 4}\cdot\dfrac{5^4 + 4}{7^4 + 4}\cdot\dfrac{9^4 + 4}{11^4 + 4}\cdots\dfrac{17^4 + 4}{17^4 + 4}$. (c) Mở rộng.
\end{baitoan}

\begin{baitoan}[\cite{Tuyen_Toan_8}, 162., p. 34]
	Chứng minh $A = \left(1 + \dfrac{4}{5}\right)\left(1 + \dfrac{4}{12}\right)\left(1 + \dfrac{4}{21}\right)\cdots\left(1 + \dfrac{4}{n(n + 4)}\right) < 6$, $\forall n\in\mathbb{N}^\star$.
\end{baitoan}

\begin{baitoan}[\cite{Tuyen_Toan_8}, 163., p. 35]
	Cho $A = \dfrac{x - y}{1 + xy},B = \dfrac{y - z}{1 + yz},C = \dfrac{z - x}{1 + zx}$. Chứng minh $A + B + C = ABC$.
\end{baitoan}

\begin{baitoan}[\cite{Tuyen_Toan_8}, 164., p. 35]
	Cho $a,b\in\mathbb{R},ab = 1,a + b\ne0$. Tính $A = \dfrac{1}{(a + b)^3}\left(\dfrac{1}{a^3} + \dfrac{1}{b^3}\right) + \dfrac{3}{(a + b)^4}\left(\dfrac{1}{a^2} + \dfrac{1}{b^2}\right) + \dfrac{6}{(a + b)^5}\left(\dfrac{1}{a} + \dfrac{1}{b}\right)$.
\end{baitoan}

\begin{baitoan}[\cite{Tuyen_Toan_8}, 165., p. 35]
	Cho $A = \dfrac{4yz - x^2}{yz + 2x^2},B = \dfrac{4zx - y^2}{zx + 2x^2},C = \dfrac{4xy - z^2}{xy + 2z^2}$. Chứng minh nếu $x + y + z = 0$ \& $x,y,z$ khác nhau đôi một thì $ABC$ là 1 hằng số.
\end{baitoan}

\begin{baitoan}[\cite{Binh_Toan_8_tap_1}, VD32, p. 21]
	Cho $a,b,c\in\mathbb{R}^\star,a + b + c = 0$. Rút gọn biểu thức $A = \dfrac{ab}{a^2 + b^2 - c^2} + \dfrac{bc}{b^2 + c^2 - a^2} + \dfrac{ca}{c^2 + a^2 - b^2}$.
\end{baitoan}

\begin{baitoan}[\cite{Binh_Toan_8_tap_1}, VD33, p. 22]
	Rút gọn biểu thức $A = \dfrac{1}{1 - x} + \dfrac{1}{1 + x} + \dfrac{2}{1 + x^2} + \dfrac{4}{1 + x^4} + \dfrac{8}{1 + x^8}$.
\end{baitoan}

\begin{baitoan}[\cite{Binh_Toan_8_tap_1}, VD34, p. 22]
	Rút gọn biểu thức $A = \sum_{i=1}^n \dfrac{2i + 1}{[i(i + 1)]^2} = \dfrac{3}{(1\cdot2)^2} + \dfrac{5}{(2\cdot3)^2} + \cdots + \dfrac{2n + 1}{[n(n + 1)]^2}$.
\end{baitoan}

\begin{baitoan}[\cite{Binh_Toan_8_tap_1}, VD35, p. 22]
	Xác định $a,b,c\in\mathbb{R}$ thỏa $\dfrac{1}{(x^2 + 1)(x - 1)} = \dfrac{ax + b}{x^2 + 1} + \dfrac{c}{x - 1}$.
\end{baitoan}

\begin{baitoan}[\cite{Binh_Toan_8_tap_1}, VD36, p. 22]
	Cho $A = \dfrac{1}{(x + y)^3}\left(\dfrac{1}{x^4} - \dfrac{1}{y^4}\right),B = \dfrac{2}{(x + y)^4}\left(\dfrac{1}{x^3} - \dfrac{1}{y^3}\right),C = \dfrac{2}{(x + y)^45}\left(\dfrac{1}{x^2} - \dfrac{1}{y^2}\right)$. Tính $A + B + C$.
\end{baitoan}

\begin{baitoan}[\cite{Binh_Toan_8_tap_1}, 124., p. 23]
	Tính: (a) $\dfrac{x + 3}{x + 1} - \dfrac{2x - 1}{x - 1} - \dfrac{x - 3}{x^2 - 1}$. (b) $\dfrac{1}{x(x + y)} + \dfrac{1}{y(x + y)} + \dfrac{1}{x(x - y)} + \dfrac{1}{y(y - x)}$.
\end{baitoan}

\begin{baitoan}[\cite{Binh_Toan_8_tap_1}, 125., p. 23]
	Tính: (a) $A = \dfrac{1}{(a - b)(a - c)} + \dfrac{1}{(b - c)(b - a)} + \dfrac{1}{(c - a)(c - b)}$. (b) $B = \dfrac{a}{(a - b)(a - c)} + \dfrac{b}{(b - a)(b - c)} + \dfrac{c}{(c - a)(c - b)}$. (c) $C = \dfrac{1}{a(a - b)(a - c)} + \dfrac{1}{b(b - a)(b - c)} + \dfrac{1}{c(c - a)(c - b)}$. (d) $D = \dfrac{bc}{(a - b)(a - c)} + \dfrac{ca}{(b - c)(b - a)} + \dfrac{ab}{(c - a)(c - b)}$. (e) $E = \dfrac{a^2}{(a - b)(a - c)} + \dfrac{b^2}{(b - a)(b - c)} + \dfrac{c^2}{(c - a)(c - b)}$.
\end{baitoan}

\begin{baitoan}[\cite{Binh_Toan_8_tap_1}, 126., p. 24]
	Cho $a,b,c\in\mathbb{Z}$ đôi một khác nhau. Chứng minh biểu thức có giá trị nguyên: (a) $A = \dfrac{a^3}{(a - b)(a - c)} + \dfrac{b^3}{(b - a)(b - c)} + \dfrac{c^3}{(c - a)(c - b)}$. (b) $B = \dfrac{a^4}{(a - b)(a - c)} + \dfrac{b^4}{(b - a)(b - c)} + \dfrac{c^4}{(c - a)(c - b)}$.
\end{baitoan}

\begin{baitoan}[\cite{Binh_Toan_8_tap_1}, 127., p. 24]
	Cho $3y - x = 6$. Tính giá trị biểu thức $A = \dfrac{x}{y - 2} + \dfrac{2x - 3y}{x - 6}$.
\end{baitoan}

\begin{baitoan}[\cite{Binh_Toan_8_tap_1}, 128., p. 24]
	Tìm $x,y,z\in\mathbb{R}$ thỏa: (a) $\dfrac{x^2}{2} + \dfrac{y^2}{3} + \dfrac{z^2}{4} = \dfrac{x^2 + y^2 + z^2}{5}$. (b) $x^2 + y^2 + \dfrac{1}{x^2} + \dfrac{1}{y^2} = 4$.
\end{baitoan}

\begin{baitoan}[\cite{Binh_Toan_8_tap_1}, 129., p. 24]
	Cho $a,b,c\in\mathbb{R}^\star,\dfrac{1}{a} + \dfrac{1}{b} + \dfrac{1}{c} = \dfrac{1}{a^2} + \dfrac{1}{b^2} + \dfrac{1}{c^2} = 2$. Chứng minh $a + b + c = abc$.
\end{baitoan}

\begin{baitoan}[\cite{Binh_Toan_8_tap_1}, 130., p. 24]
	Cho $a,b,c\in\mathbb{R}^\star,\dfrac{1}{a} + \dfrac{1}{b} + \dfrac{1}{c} = 0,a + b + c = 1$. Tính giá trị biểu thức $a^2 + b^2 + c^2$.
\end{baitoan}

\begin{baitoan}[\cite{Binh_Toan_8_tap_1}, 131., p. 24]
	Cho $a,b,c,x,y,z\in\mathbb{R}^\star,\dfrac{x}{a} + \dfrac{y}{b} + \dfrac{z}{c} = 0,\dfrac{a}{x} + \dfrac{b}{y} + \dfrac{c}{z} = 2$. Tính giá trị biểu thức $\dfrac{a^2}{x^2} + \dfrac{b^2}{y^2} + \dfrac{c^2}{z^2}$.
\end{baitoan}

\begin{baitoan}[\cite{Binh_Toan_8_tap_1}, 132., p. 24]
	Cho $a,b,c\in\mathbb{R}^\star,(a + b + c)^2 = a^2 + b^2 + c^2$. Chứng minh $\dfrac{1}{a^3} + \dfrac{1}{b^3} + \dfrac{1}{c^3} = \dfrac{3}{abc}$.
\end{baitoan}

\begin{baitoan}[\cite{Binh_Toan_8_tap_1}, 133., p. 24]
	Cho $a,b,c\in\mathbb{R}^\star,\dfrac{a}{b} + \dfrac{b}{c} + \dfrac{c}{a} = \dfrac{b}{a} + \dfrac{c}{b} + \dfrac{a}{c}$. Chứng minh trong 3 số $a,b,c$ tồn tại 2 số bằng nhau.
\end{baitoan}

\begin{baitoan}[\cite{Binh_Toan_8_tap_1}, 134., p. 24]
	Tìm $x\in\mathbb{Z}$ để phân thức có giá trị nguyên: (a) $A = \dfrac{2x^3 - 6x^2 + x - 8}{x - 3}$. (b) $B = \dfrac{x^4 - 2x^3 - 3x^2 + 8x - 1}{x^2 - 2x + 1}$. (c) $C = \dfrac{x^4 + 3x^3 + 2x^2 + 6x - 2}{x^2 + 2}$.
\end{baitoan}

\begin{baitoan}[\cite{Binh_Toan_8_tap_1}, 135., p. 25]
	Rút gọn biểu thức $A = \dfrac{x + 3a}{2 - x} + \dfrac{x - 3a}{2 + x} - \dfrac{2a}{4 - x^2} + a$ với $x = \dfrac{a}{3a + 2}$.
\end{baitoan}

\begin{baitoan}[\cite{Binh_Toan_8_tap_1}, 136., p. 25]
	Rút gọn biểu thức $A = \dfrac{2}{a - b} + \dfrac{2}{b - c} + \dfrac{2}{c - a} + \dfrac{(a - b)^2 + (b - c)^2 + (c - a)^2}{(a - b)(b - c)(c - a)}$.
\end{baitoan}

\begin{baitoan}[\cite{Binh_Toan_8_tap_1}, 137., p. 25]
	Cho $a,b,c\in\mathbb{R}^\star,\dfrac{a + b - c}{ab} - \dfrac{b + c - a}{bc} - \dfrac{c + a - b}{ca} = 0$. Chứng minh trong 3 phân thức ở vế trái, có ít nhất 1 phân thức bằng $0$.
\end{baitoan}

\begin{baitoan}[\cite{Binh_Toan_8_tap_1}, 138., p. 25]
	Cho $a,b,c\in\mathbb{R}^\star,x,y,z\in\mathbb{R},\dfrac{ay - bx}{c} = \dfrac{bz - cy}{a} = \dfrac{cx - az}{b}$. Chứng minh mỗi phân thức này bằng $0$.
\end{baitoan}

\begin{baitoan}[\cite{Binh_Toan_8_tap_1}, 139., p. 25]
	Xác định $a,b,c\in\mathbb{R}$ để: (a) $\dfrac{1}{x(x^2 + 1)} = \dfrac{a}{x} + \dfrac{bx + c}{x^2 + 1}$. (b) $\dfrac{1}{x^2 - 4} = \dfrac{a}{x - 2} + \dfrac{b}{x + 2}$. (c) $\dfrac{1}{(x + 1)^2(x + 2)} = \dfrac{a}{x + 1} + \dfrac{b}{(x + 1)^2} + \dfrac{c}{x + 2}$.
\end{baitoan}

\begin{baitoan}[\cite{Binh_Toan_8_tap_1}, 140., p. 25]
	Rút gọn biểu thức $A = (ab + bc + ca)\left(\dfrac{1}{a} + \dfrac{1}{b} + \dfrac{1}{c}\right) - abc\left(\dfrac{1}{a^2} + \dfrac{1}{b^2} + \dfrac{1}{c^2}\right)$.
\end{baitoan}

\begin{baitoan}[\cite{Binh_Toan_8_tap_1}, 141., p. 25]
	Cho $a,b,c\in\mathbb{R}^\star$ khác nhau đôi một, $\dfrac{1}{a} + \dfrac{1}{b} + \dfrac{1}{c} = 0$. Rút gọn biểu thức: (a) $A = \dfrac{1}{a^2 + 2bc} + \dfrac{1}{b^2 + 2ca} + \dfrac{1}{c^2 + 2ab}$. (b) $B = \dfrac{bc}{a^2 + 2bc} + \dfrac{ca}{2b^2 + ca} + \dfrac{ab}{2c^2 + ab}$. (c) $C = \dfrac{a^2}{a^2 + 2bc} + \dfrac{b^2}{b^2 + 2ca} + \dfrac{c^2}{c^2 + 2ab}$.
\end{baitoan}

\begin{baitoan}[\cite{Binh_Toan_8_tap_1}, 142., p. 25]
	Cho $a,b,c\in\mathbb{R}^\star$ khác nhau đôi một, $\dfrac{a + b}{c} = \dfrac{b + c}{a} = \dfrac{c + a}{b}$. Tính giá trị biểu thức $A = \left(1 + \dfrac{a}{b}\right)\left(1 + \dfrac{b}{c}\right)\left(1 + \dfrac{c}{a}\right)$.
\end{baitoan}

\begin{baitoan}[\cite{Binh_Toan_8_tap_1}, 143., p. 25]
	Cho $x,y\in\mathbb{R},(x + y)^3 + x + y = x^3y^3 + xy$. Tính giá trị biểu thức $A = \dfrac{1}{x} + \dfrac{1}{y}$.
\end{baitoan}

\begin{baitoan}[\cite{Binh_Toan_8_tap_1}, 144., p. 25]
	Cho $a,b,c\in\mathbb{R},a^3 + b^3 + c^3 = 3abc,a + b + c\ne0$. Tính giá trị biểu thức $A = \dfrac{a^2 + b^2 + c^2}{(a + b + c)^2}$.
\end{baitoan}

\begin{baitoan}[\cite{Binh_Toan_8_tap_1}, 145., p. 26]
	Rút gọn biểu thức $A = \dfrac{1}{a^2 - 5a + 6} + \dfrac{1}{a^2 - 7a + 12} + \dfrac{1}{a^2 - 9a + 20} + \dfrac{1}{a^2 - 11a + 30}$.
\end{baitoan}

\begin{baitoan}[\cite{Binh_Toan_8_tap_1}, 146., p. 26]
	Cho $a,b,c\in\mathbb{R},abc = 1,a + b + c = \dfrac{1}{a} + \dfrac{1}{b} + \dfrac{1}{c}$. Chứng minh trong 3 số $a,b,c$ tồn tại 1 số bằng $1$.
\end{baitoan}

\begin{baitoan}[\cite{Binh_Toan_8_tap_1}, 147., p. 26]
	Chứng minh nếu $x + y + z = a,\dfrac{1}{x} + \dfrac{1}{y} + \dfrac{1}{z} = \dfrac{1}{a}$ thì tồn tại 1 trong 3 số $x,y,z$ bằng $a$.
\end{baitoan}

\begin{baitoan}[\cite{Binh_Toan_8_tap_1}, 148., p. 26]
	2 biểu thức $x + y + z,\dfrac{1}{x} + \dfrac{1}{y} + \dfrac{1}{z}$ có thể cùng có giá trị bằng $0$ được không? 
\end{baitoan}

\begin{baitoan}[\cite{Binh_Toan_8_tap_1}, 149., p. 26]
	Tính giá trị biểu thức $A = \dfrac{1}{x + 2} + \dfrac{1}{y + 2} + \dfrac{1}{z + 2}$ biết $2a = by + cz,2b = cz + ax,2c = ax + by,a + b + c\ne0$.
\end{baitoan}

\begin{baitoan}[\cite{Binh_Toan_8_tap_1}, 150., p. 26]
	(a) Cho $a,b,c\in\mathbb{R},abc = 2$. Rút gọn biểu thức $A = \dfrac{a}{ab + a + 2} + \dfrac{b}{bc + b + 1} + \dfrac{2c}{ca + 2c + 2}$. (b) Cho $a,b,c\in\mathbb{R},abc = 1$. Rút gọn biểu thức $A = \dfrac{a}{ab + a + 1} + \dfrac{b}{bc + b + 1} + \dfrac{c}{ca + c + 1}$. (c) Mở rộng.
\end{baitoan}

\begin{baitoan}[\cite{Binh_Toan_8_tap_1}, 151., p. 26]
	Cho $a,b,c\in\mathbb{R},ac\ne0,a\ne b,b\ne c,\dfrac{a}{c} = \dfrac{a - b}{b - c}$. Chứng minh $\dfrac{1}{a} + \dfrac{1}{a - b} = \dfrac{1}{b - c} - \dfrac{1}{c}$.
\end{baitoan}

\begin{baitoan}[\cite{Binh_Toan_8_tap_1}, 152., p. 26]
	Cho $a,b,c\in\mathbb{R}^\star,a + b + c = 0$. Rút gọn biểu thức: (a) $A = \dfrac{a^2}{bc} + \dfrac{b^2}{ca} + \dfrac{c^2}{ab}$. (b) $B = \dfrac{a^2}{a^2 - b^2 - c^2} + \dfrac{b^2}{b^2 - c^2 - a^2} + \dfrac{c^2}{c^2 - a^2 - b^2}$. (c) $C = \dfrac{ab^2}{a^2 + b^2 - c^2} + \dfrac{bc^2}{b^2 + c^2 - a^2} + \dfrac{ca^2}{c^2 + a^2 - b^2}$. (d) $D = \dfrac{a^4}{a^4 - (b^2 - c^2)^2} + \dfrac{b^4}{b^4 - (c^2 - a^2)^2} + \dfrac{c^4}{c^4 - (a^2 - b^2)^2}$. 
\end{baitoan}

\begin{baitoan}[\cite{Binh_Toan_8_tap_1}, 153., p. 26]
	Cho $a,b,c\in\mathbb{R}^\star,a + b + c = 0$. Tính giá trị biểu thức $A = \left(\dfrac{a - b}{c} + \dfrac{b - c}{a} + \dfrac{c - a}{b}\right)\left(\dfrac{c}{a - b} + \dfrac{a}{b - c} + \dfrac{b}{c - a}\right)$.
\end{baitoan}

\begin{baitoan}[\cite{Binh_Toan_8_tap_1}, 154., p. 27]
	Chứng minh nếu $(a^2 - bc)(b - abc) = (b^2 - ca)(a - abc)$ \& $abc(a - b)\ne0$ thì $\dfrac{1}{a} + \dfrac{1}{b} + \dfrac{1}{c} = a + b + c$.
\end{baitoan}

\begin{baitoan}[\cite{Binh_Toan_8_tap_1}, 155., p. 27]
	Cho $a,b,c\in\mathbb{R},x,y,z\in\mathbb{R}^\star,a + b + c = x + y + z = 0,\dfrac{a}{x} + \dfrac{b}{y} + \dfrac{c}{z} = 0$. Chứng minh $ax^2 + by^2 + cz^2 = 0$.
\end{baitoan}

\begin{baitoan}[\cite{Binh_Toan_8_tap_1}, 156., p. 27]
	Cho $\dfrac{xy + 1}{y} = \dfrac{yz + 1}{z} = \dfrac{zx + 1}{x}$. Chứng minh $x = y = z$ hoặc $x^2y^2z^2 = 1$.
\end{baitoan}

\begin{baitoan}[\cite{Binh_Toan_8_tap_1}, 157., p. 27]
	Cho $a,b,c\in\mathbb{R},\dfrac{a}{b + c} + \dfrac{b}{c + a} + \dfrac{c}{a + b} = 1$. Chứng minh $\dfrac{a^2}{b + c} + \dfrac{b^2}{c + a} + \dfrac{c^2}{a + b} = 0$
\end{baitoan}

\begin{baitoan}[\cite{Binh_Toan_8_tap_1}, 158., p. 27]
	Cho $a,b,c\in\mathbb{R},\dfrac{a}{b - c} + \dfrac{b}{c - a} + \dfrac{c}{a - b} = 1$. Chứng minh $\dfrac{a}{(b - c)^2} + \dfrac{b}{(c - a)^2} + \dfrac{c}{(a - b)^2} = 0$.
\end{baitoan}

\begin{baitoan}[\cite{Binh_Toan_8_tap_1}, 159., p. 27]
	Cho $x\in\mathbb{R}^\star,x + \dfrac{1}{x} = a\in\mathbb{R}$. Tính biểu thức theo $a$: (a) $x^2 + \dfrac{1}{x^2}$. (b) $x^3 + \dfrac{1}{x^3}$. (c) $x^4 + \dfrac{1}{x^4}$. (d) $x^5 + \dfrac{1}{x^5}$.
\end{baitoan}

\begin{baitoan}[\cite{Binh_Toan_8_tap_1}, 160., p. 27]
	Cho $x\in\mathbb{R}^\star,\left(x^2 - \dfrac{1}{x^2}\right):\left(x^2 + \dfrac{1}{x^2}\right) = a\in\mathbb{R}$. Tính biểu thức $A = \left(x^4 - \dfrac{1}{x^4}\right):\left(x^4 + \dfrac{1}{x^4}\right)$ theo $a$.
\end{baitoan}

\begin{baitoan}[\cite{Binh_Toan_8_tap_1}, 161., p. 27]
	Cho $x\in\mathbb{R},x^2 - 4x + 1 = 0$. Tính giá trị biểu thức $A = \dfrac{x^4 + 1}{x^2}$. (b) $B = \dfrac{x^4 + x^2 + 1}{x^2}$.
\end{baitoan}

\begin{baitoan}[\cite{Binh_Toan_8_tap_1}, 162., p. 27]
	Cho $a,x\in\mathbb{R},\dfrac{x}{x^2 - x + 1} = a$. Tính $A = \dfrac{x^2}{x^4 + x^2 + 1}$ theo $a$.
\end{baitoan}

\begin{baitoan}[\cite{Binh_Toan_8_tap_1}, 163., p. 27]
	Cho $a,b,c,x\in\mathbb{R},x = \dfrac{b^2 + c^2 - a^2}{2bc},y = \dfrac{a^2 - (b - c)^2}{(b + c)^2 - a^2}$. Tính giá trị biểu thức $A =  x + y + xy$.
\end{baitoan}

\begin{baitoan}[\cite{Binh_Toan_8_tap_1}, 164., p. 27]
	(a) Mức sản xuất của 1 xí nghiệp năm 2001 tăng $a\%$ so với năm 2000, năm 2002 tăng $b\%$ so với năm 2001. Tính mức sản xuất của xí nghiệp đó năm 2002 tăng so với năm 2000. (b) 1 số $a$ tăng $m\%$, sau đó lại giảm đi $n\%$, $a,m,n\in\mathbb{R},a,m,n > 0$, thì được số $b$. Tìm liên hệ giữa $m,n$ để $a < b$.
\end{baitoan}

%------------------------------------------------------------------------------%

\section{Rational Expression Transformation -- Biến Đổi Biểu Thức Hữu Tỷ}

\begin{baitoan}[\cite{Tuyen_Toan_8}, VD25, p. 35]
	Cho $A = \dfrac{2}{x} - \left(\dfrac{x^2}{x^2 - xy} + \dfrac{x^2 - y^2}{xy} - \dfrac{y^2}{y^2 - xy}\right):\dfrac{x^2 - xy + y^2}{x - y}$. (a) Tìm {\rm ĐKXĐ}. (b) Rút gọn A. (c) Tính giá trị của A với $|2x - 1| = 1,|y + 1| = \frac{1}{2}$.
\end{baitoan}

\begin{baitoan}[\cite{Tuyen_Toan_8}, 166., p. 36]
	Cho 3 phân thức $A = \dfrac{x^2 + x - 2}{x^2 - 4},B = \dfrac{x^2 - y^2}{x^3 - y^3},C = \dfrac{x - y}{x^2 + y^2 + 4x - 2y + 5}$. Tìm các giá trị của $x,y$ để: (a) Giá trị mỗi phân thức này được xác định. (b) Giá trị mỗi phân thức này bằng $0$.
\end{baitoan}

\begin{baitoan}[\cite{Tuyen_Toan_8}, 167., pp. 36--37]
	(a) Tìm {\rm GTLN} của phân thức $A = \dfrac{5}{x^2 - 6x + 10}$. (b) Tìm {\rm GTNN} của phân thức $B = \dfrac{-8}{x^2 - 2x + 5}$. (c) Mở rộng.
\end{baitoan}

\begin{baitoan}[\cite{Tuyen_Toan_8}, 168., p. 37]
	Cho biểu thức $A = \dfrac{1}{x + y + z}\cdot\dfrac{1}{xy + yz + zx}\left(\dfrac{1}{x} + \dfrac{1}{y} + \dfrac{1}{z}\right)\left(\dfrac{1}{xy} + \dfrac{1}{yz} + \dfrac{1}{zx}\right)$. Chứng minh $A > 0$, $\forall x,y,z\in\mathbb{R}^\star$.
\end{baitoan}

\begin{baitoan}[\cite{Tuyen_Toan_8}, 169., p. 37]
	Cho biểu thức $A = \dfrac{x + \frac{1}{y}}{y + \frac{1}{x}}$. (a) Rút gọn A. (b) Tìm $x,y\in\mathbb{Z},x + y\le50$ để $A = 8$.
\end{baitoan}

\begin{baitoan}[\cite{Tuyen_Toan_8}, 170., p. 37]
	Cho $x,y,z\in\mathbb{R}^\star,\dfrac{x - y - z}{x} = \dfrac{y - z - x}{y} = \dfrac{z - x - y}{z}$. Tính\\$A = \left(1 + \dfrac{y}{x}\right)\left(1 + \dfrac{z}{y}\right)\left(1 + \dfrac{x}{z}\right)$.
\end{baitoan}

\begin{baitoan}[\cite{Tuyen_Toan_8}, 171., p. 37]
	Cho $x,y,z\in\mathbb{R},x,y,z\ne-1$. Chứng minh giá trị của biểu thức $A = \dfrac{xy + 2x + 1}{xy + x + y + 1}  + \dfrac{yz + 2y + 1}{yz + y + z + 1} + \dfrac{zx + 2z + 1}{zx + z + x + 1}$ không phụ thuộc vào, i.e., độc lập với 3 biến $x,y,z$.
\end{baitoan}

\begin{baitoan}[\cite{Tuyen_Toan_8}, 172., p. 37]
	Cho $x,y,z\in\mathbb{R}^\star,x + y + z\ne0$ thỏa $x = by + cz,y = cz + ax,z = ax + by$. Chứng minh đẳng thức $\dfrac{1}{1 + a} + \dfrac{1}{1 + b} + \dfrac{1}{1 + c} = 2$.
\end{baitoan}

\begin{baitoan}[\cite{Tuyen_Toan_8}, 173., p. 37]
	Cho $\dfrac{x^n - x^{-n}}{x^n + x^{-n}} = a\in\mathbb{R}$ với $n\in\mathbb{N}^\star$. Tính $\dfrac{x^{2n} - x^{-2n}}{x^{2n} + x^{-2n}}$ theo $a$.
\end{baitoan}

%------------------------------------------------------------------------------%

\section{Algebraic Fraction \& Fraction -- Phân Thức \& Phân Số}

\begin{baitoan}[\cite{Binh_Toan_8_tap_1}, VD37, p. 28]
	Chứng minh phân số viết được dưới dạng hiệu của 2 phân số có tử bằng $1$: (a) $\dfrac{n - 1}{n!}$, $\forall n\in\mathbb{N}^\star$. (b) $\dfrac{2n}{n^4 + n^2 + 1}$, $\forall n\in\mathbb{N}$.
\end{baitoan}

\begin{baitoan}[\cite{Binh_Toan_8_tap_1}, VD38, p. 28]
	Chứng minh $\sum_{i=1}^n \dfrac{1}{(2i + 1)^2} = \dfrac{1}{3^2} + \dfrac{1}{5^2} + \cdots + \dfrac{1}{(2n + 1)^2} < \dfrac{1}{4}$, $\forall n\in\mathbb{N}^\star$.
\end{baitoan}

\begin{baitoan}[\cite{Binh_Toan_8_tap_1}, VD39, p. 28]
	Chứng minh $A = \prod_{i=2}^9 \dfrac{i^3 + 1}{i^3 - 1} = \dfrac{2^3 + 1}{2^3 - 1}\cdot\dfrac{3^3 + 1}{3^3 - 1}\cdots\dfrac{9^3 + 1}{9^3 - 1} < \dfrac{3}{2}$.
\end{baitoan}

\begin{baitoan}[\cite{Binh_Toan_8_tap_1}, VD40, p. 29]
	Chứng minh $A = \sum_{i=0}^n \dfrac{2i + 1}{(2i + 1)^4 + 4} = \dfrac{1}{1^4 + 4} + \dfrac{3}{3^4 + 4} + \cdots + \dfrac{2n + 1}{(2n + 1)^4 + 4} < \dfrac{1}{4}$.
\end{baitoan}

\begin{baitoan}[\cite{Binh_Toan_8_tap_1}, VD41, p. 29]
	Chứng minh $\sum_{i=2}^n \dfrac{1}{i^3} = \dfrac{1}{2^3} + \dfrac{1}{3^3} + \cdots + \dfrac{1}{n^3} < \dfrac{1}{4}$.
\end{baitoan}

\begin{baitoan}[\cite{Binh_Toan_8_tap_1}, VD42, p. 30]
	Chứng minh $A = \sum_{i=1}^{2^n - 1} \dfrac{1}{i} = 1 + \dfrac{1}{2} + \dfrac{1}{3} + \cdots + \dfrac{1}{2^n - 1} < n$, $\forall n\in\mathbb{N}$, $n\ge2$.
\end{baitoan}

\begin{baitoan}[\cite{Binh_Toan_8_tap_1}, 165., p. 30]
	Rút gọn biểu thức: (a) $A = \prod_{i=2}^n 1 - \dfrac{1}{i^2} = \left(1 - \dfrac{1}{2^2}\right)\left(1 - \dfrac{1}{3^2}\right)\cdots\left(1 - \dfrac{1}{n^2}\right)$. (b) $B = \prod_{i=0}^n \dfrac{(2i + 1)^2}{(2i + 2)^2 - 1} = \dfrac{1^2}{2^2 - 1}\cdot\dfrac{3^2}{4^2 - 1}\cdot\dfrac{5^2}{6^2 - 1}\cdots\dfrac{(2n + 1)^2}{(2n + 2)^2 - 1}$.
\end{baitoan}

\begin{baitoan}[\cite{Binh_Toan_8_tap_1}, 166., p. 30]
	Cho $A = \dfrac{2}{1}\cdot\dfrac{4}{3}\cdot\dfrac{6}{5}\cdots\dfrac{200}{199}$. Chứng minh $14 < A < 20$.
\end{baitoan}

\begin{baitoan}[\cite{Binh_Toan_8_tap_1}, 167., p. 30]
	Chứng minh $\dfrac{1}{3}\cdot\dfrac{4}{6}\cdot\dfrac{7}{9}\cdot\dfrac{10}{12}\cdots\dfrac{208}{210} < \dfrac{1}{25}$.
\end{baitoan}

\begin{baitoan}[\cite{Binh_Toan_8_tap_1}, 168., p. 30]
	Rút gọn biểu thức: (a) $A = \sum_{i=1}^{n-1} \dfrac{1}{i(i + 1)} = \dfrac{1}{1\cdot2} + \dfrac{1}{2\cdot3} + \cdots + \dfrac{1}{(n - 1)n}$. (b) $\sum_{i=0}^n \dfrac{1}{(3i + 2)(3i + 5)} = \dfrac{1}{2\cdot5} + \dfrac{1}{5\cdot8} + \dfrac{1}{8\cdot11} + \cdots + \dfrac{1}{(3n + 2)(3n + 5)}$. (c) $\sum_{i=2}^n \dfrac{1}{(i - 1)i(i + 1)} = \dfrac{1}{1\cdot2\cdot3} + \dfrac{1}{2\cdot3\cdot4} + \cdots + \dfrac{1}{(n - 1)n(n + 1)}$.
\end{baitoan}

\begin{baitoan}[\cite{Binh_Toan_8_tap_1}, 169., p. 30]
	Chứng minh $\forall n\in\mathbb{N}^\star$: (a) $\sum_{i=1}^n \dfrac{1}{(2i)^2} = \dfrac{1}{2^2} + \dfrac{1}{4^2} + \cdots + \dfrac{1}{(2n)^2} < \dfrac{1}{2}$. (b) $\sum_{i=1}^n \dfrac{2i + 1}{i^2(i + 1)^2} = \dfrac{3}{4} + \dfrac{5}{36} + \dfrac{7}{144} + \cdots + \dfrac{2n + 1}{n^2(n + 1)^2} < 1$.
\end{baitoan}

\begin{baitoan}[\cite{Binh_Toan_8_tap_1}, 170., p. 30]
	Chứng minh $A = \sum_{i=2}^n \dfrac{1}{i^2} = \dfrac{1}{2^2} + \dfrac{1}{3^2} + \cdots + \dfrac{1}{n^2} < \dfrac{2}{3}$.
\end{baitoan}

\begin{baitoan}[\cite{Binh_Toan_8_tap_1}, 171., p. 31]
	Chứng minh $A = \sum_{i=3}^n \dfrac{1}{i^3} = \dfrac{1}{3^3} + \dfrac{1}{4^3} + \cdots + \dfrac{1}{n^3} < \dfrac{1}{12}$, $\forall n\in\mathbb{N}$, $n\ge3$.
\end{baitoan}

\begin{baitoan}[\cite{Binh_Toan_8_tap_1}, 172., p. 31]
	Chứng minh $A = \prod_{i=1}^n 1 + \dfrac{1}{i(i + 2)} = \left(1 + \dfrac{1}{1\cdot3}\right)\left(1 + \dfrac{1}{2\cdot4}\right)\cdots\dfrac{1}{n(n + 2)} < 2$, $\forall n\in\mathbb{N}^\star$.
\end{baitoan}

\begin{baitoan}[\cite{Binh_Toan_8_tap_1}, 173., p. 31]
	Chứng minh $A = \prod_{i=1}^n 1 - \dfrac{2}{i(i + 1)} = \left(1 - \dfrac{2}{6}\right)\left(1 - \dfrac{2}{12}\right)\cdots\left(1 - \dfrac{2}{n(n + 1)}\right) > \dfrac{1}{3}$, $\forall n\in\mathbb{N},n\ge2$.
\end{baitoan}

\begin{baitoan}[\cite{Binh_Toan_8_tap_1}, 174., p. 31]
	(a) Rút gọn biểu thức $A = \dfrac{3^2 - 1}{5^2 - 1}\cdot\dfrac{7^2 - 1}{9^2 - 1}\cdot\dfrac{11^2 - 1}{13^2 - 1}\cdots\dfrac{43^2 - 1}{45^2 - 1}$. (b) Chứng minh $B = \prod_{i=2}^n \dfrac{i^3 - 1}{i^3 + 1} = \dfrac{2^3 - 1}{2^3 + 1}\cdot\dfrac{3^3 - 1}{3^3 + 1}\cdots\dfrac{n^3 - 1}{n^3 + 1} > \dfrac{2}{3}$. (c) Chứng minh $C = \prod_{i=2}^{20} \dfrac{2^i + 1}{2^i} = \dfrac{2^2 + 1}{2^2}\cdot\dfrac{2^3 + 1}{2^3}\cdots\dfrac{2^{20} + 1}{2^{20}} < 2$.
\end{baitoan}

\begin{baitoan}[\cite{Binh_Toan_8_tap_1}, 175., p. 31]
	Rút gọn biểu thức $A = \dfrac{(1^4 + 4)(5^4 + 4)(9^4 + 4)\cdots(21^4 + 4)}{(3^4 + 4)(7^4 + 4)(11^4 + 4)\cdots(23^4 + 4)}$.
\end{baitoan}

\begin{baitoan}[\cite{Binh_Toan_8_tap_1}, 176., p. 31]
	Chứng minh: (a) $A = \sum_{i=1}^n \dfrac{i}{4i^4 + 1} = \dfrac{1}{4\cdot1^4 + 1} + \dfrac{2}{4\cdot2^4 + 1} + \cdots + \dfrac{n}{4n^4 + 1} < 1$, $\forall n\in\mathbb{N}^\star$. (b) $B = \sum_{i=1}^{50} \dfrac{i}{1 + i^2 + i^4} = \dfrac{1}{1 + 1^2 + 1^4} + \dfrac{2}{1 + 2^2 + 2^4} + \cdots + \dfrac{50}{1 + 50^2 + 50^4} < \dfrac{1}{2}$.
\end{baitoan}

\begin{baitoan}[\cite{Binh_Toan_8_tap_1}, 177., p. 31]
	Chứng minh: (a) $A = \sum_{i=2}^n \dfrac{i - 1}{i!} = \dfrac{1}{2!} + \dfrac{2}{3!} + \dfrac{3}{4!} + \cdots + \dfrac{n - 1}{n!} < 1$, $\forall n\in\mathbb{N}$, $n\ge2$. (b) $B = \sum_{i=1}^n \dfrac{i^2 + i - 1}{(i + 1)!} = \dfrac{1}{2!} + \dfrac{5}{3!} +  \dfrac{11}{4!} + \cdots + \dfrac{n^2 + n - 1}{(n + 1)!} < 2$, $\forall n\in\mathbb{N}^\star$.
\end{baitoan}

\begin{baitoan}[\cite{Binh_Toan_8_tap_1}, 178., p. 31]
	Chứng minh $A = \sum_{i=1}^{100} \dfrac{i}{2^i} = \dfrac{1}{2} + \dfrac{2}{2^2} + \dfrac{3}{2^3} + \cdots + \dfrac{100}{2^{100}} < 2$.
\end{baitoan}

\begin{baitoan}[\cite{Binh_Toan_8_tap_1}, 179., p. 31]
	Chứng minh $A = \sum_{i=1}^{100} \dfrac{i}{3^i} = \dfrac{1}{3} + \dfrac{2}{3^2} + \dfrac{3}{3^3} + \cdots + \dfrac{100}{3^{100}} < \dfrac{3}{4}$.
\end{baitoan}

\begin{baitoan}[\cite{Binh_Toan_8_tap_1}, 180., p. 31]
	Chứng minh $1 < \dfrac{1}{n + 1} + \dfrac{1}{n + 2} + \cdots + \dfrac{1}{3n + 1} < 2$, $\forall n\in\mathbb{N}^\star$.
\end{baitoan}

\begin{baitoan}[\cite{Binh_Toan_8_tap_1}, 181., p. 31]
	Chứng minh $\dfrac{3}{5} < \sum_{i=2004}^{4006} \dfrac{1}{i} = \dfrac{1}{2004} + \dfrac{1}{2005} + \cdots + \dfrac{1}{4006} < \dfrac{3}{4}$.
\end{baitoan}

\begin{baitoan}[\cite{Binh_Toan_8_tap_1}, 182., p. 32]
	(a) Chứng minh $\sum_{i=1}^{2^n - 1} \dfrac{1}{i} = 1 + \dfrac{1}{2} + \cdots + \dfrac{1}{2^n - 1} > \dfrac{n}{2}$, $\forall n\in\mathbb{N}^\star$. (b) Chứng minh $\forall a\in\mathbb{R},a > 0$, luôn tìm được $n\in\mathbb{N}^\star$ để $\sum_{i=1}^n \dfrac{1}{i} = 1 + \dfrac{1}{2} + \cdots + \dfrac{1}{n} > a$.
\end{baitoan}

\begin{baitoan}[\cite{Binh_Toan_8_tap_1}, 183., p. 32]
	Rút gọn biểu thức $\left(\dfrac{n - 1}{1} + \dfrac{n - 2}{2} + \cdots + \dfrac{2}{n - 2} + \dfrac{1}{n - 1}\right):\left(\dfrac{1}{2} + \dfrac{1}{3} + \cdots + \dfrac{1}{n}\right)$, $\forall n\in\mathbb{N}^\star$.
\end{baitoan}

\begin{baitoan}[\cite{Binh_Toan_8_tap_1}, 184., p. 32]
	Rút gọn biểu thức $\dfrac{\dfrac{1}{1(2n - 1)} + \dfrac{1}{3(2n - 3)} + \dfrac{1}{5(2n - 5)} + \cdots + \dfrac{1}{(2n - 3)\cdot3} + \dfrac{1}{(2n - 1)\cdot1}}{1 + \dfrac{1}{3} + \dfrac{1}{5} + \cdots + \dfrac{1}{2n - 1}}$.
\end{baitoan}

\begin{baitoan}[\cite{Binh_Toan_8_tap_1}, 185., p. 32]
	Tìm $a,b\in\mathbb{N}$ để: (a) $a - b = \dfrac{a}{b}$. (b) $a - b = \dfrac{a}{2b}$.
\end{baitoan}

\begin{baitoan}[\cite{Binh_Toan_8_tap_1}, 186., p. 32]
	Cho $a,b\in\mathbb{N}^\star,a > b$. Tìm $c\in\mathbb{N}^\star,b\ne c$ sao cho $\dfrac{a^3 + b^3}{a^3 + c^3} = \dfrac{a + b}{a + c}$.
\end{baitoan}

\begin{baitoan}[\cite{Binh_Toan_8_tap_1}, 187., p. 32]
	Cho dãy số $a_1,a_2,a_3,\ldots$ sao cho $a_{n+1} = \dfrac{a_n - 1}{a_n + 1}$. (a) Chứng minh $a_1 = a_5$. (b) Xác định 5 số đầu của dãy biết $a_{101} = 3$.
\end{baitoan}

\begin{baitoan}[\cite{Binh_Toan_8_tap_1}, 188., p. 32]
	Tìm phân số $\dfrac{m}{n}\ne0$ \& $k\in\mathbb{N}$ biết $\dfrac{m}{n} = \dfrac{m + k}{nk}$.
\end{baitoan}

\begin{baitoan}[\cite{Binh_Toan_8_tap_1}, 189., p. 32]
	Cho $a,b\in\mathbb{N},a < b$. Tìm tổng các phân số tối giản có mẫu bằng $7$, mỗi phân số lớn hơn $a$ nhưng nhỏ hơn $b$.
\end{baitoan}

\begin{baitoan}[\cite{Binh_Toan_8_tap_1}, 190., p. 32]
	Chứng minh tổng không là số nguyên: (a) $A = \sum_{i=2}^n \dfrac{1}{i} = \dfrac{1}{2} + \dfrac{1}{3} + \cdots + \dfrac{1}{n}$, $\forall n\in\mathbb{N},n\ge2$. (b) $B = \sum_{i=1}^n \dfrac{1}{2i + 1} = \dfrac{1}{3} + \dfrac{1}{5} + \dfrac{1}{7} + \cdots + \dfrac{1}{2n + 1}$, $\forall n\in\mathbb{N}^\star$.
\end{baitoan}

%------------------------------------------------------------------------------%

\section{Miscellaneous}

\begin{baitoan}[\cite{Tuyen_Toan_8}, VD26, p. 38]
	Cho $A = \left(\dfrac{x^2 + 3x}{x^3 + 3x^2 + 9x + 27} + \dfrac{3}{x^2 + 9}\right):\left(\dfrac{1}{x - 3} - \dfrac{6x}{x^3 - 3x^2 + 9x - 27}\right)$. (a) Rút gọn A. (b) Với $x > 0$ thì A không nhận các giá trị nào? (c) Tìm $x\in\mathbb{Z}$ để $A\in\mathbb{Z}$.
\end{baitoan}

\begin{baitoan}[\cite{Tuyen_Toan_8}, 174., p. 38]
	Cho biểu thức $A = \dfrac{|x + 1| + 2x}{3x^2 - 2x - 1}$. (a) Rút gọn A rồi tính giá trị của A với $x = -2,x = \frac{3}{4}$.
\end{baitoan}

\begin{baitoan}[\cite{Tuyen_Toan_8}, 175., p. 38]
	Tìm $a,b,c\in\mathbb{R}$ để $\dfrac{x^2 + x + 4}{(x + 2)^3} = \dfrac{a}{x + 2} + \dfrac{b}{(x + 2)^2} + \dfrac{c}{(x + 2)^2}$.
\end{baitoan}

\begin{baitoan}[\cite{Tuyen_Toan_8_old}, 186., p. 51]
	Cho $x,y,z\in\mathbb{R}^\star,x\ne y$. Tính: (a) $A = \dfrac{|x|}{x} + \dfrac{|y|}{y} + \dfrac{|z|}{z} + \dfrac{|xyz|}{xyz}$. (b) $B = \dfrac{xy}{|xy|} + \dfrac{x - y}{|x - y|}\left(\dfrac{x}{|x|} - \dfrac{y}{|y|}\right)$.
\end{baitoan}

\begin{baitoan}[\cite{Tuyen_Toan_8}, 176., p. 39]
	Cho $x,y,z\in\mathbb{R}^\star$ thỏa $x + y + z = xyz,\dfrac{1}{x} + \dfrac{1}{y} + \dfrac{1}{z} = \sqrt{3}$. Tính $A = \dfrac{1}{x^2} + \dfrac{1}{y^2} + \dfrac{1}{z^2}$.
\end{baitoan}

\begin{baitoan}[\cite{Tuyen_Toan_8}, 177., p. 39]
	Cho $\dfrac{x}{y - z} + \dfrac{y}{z - x} + \dfrac{z}{x - y} = 0$ với $x\ne y,y\ne z,z\ne x$. Tính $A = \dfrac{x}{(y - z)^2} + \dfrac{y}{(z - x)^2} + \dfrac{z}{(x - y)^2}$.
\end{baitoan}

\begin{baitoan}[\cite{Tuyen_Toan_8}, 178., p. 39]
	Cho biểu thức $A = \dfrac{a^2 + b^2 - c^2}{2ab} + \dfrac{b^2 + c^2 - a^2}{2bc} + \dfrac{c^2 + a^2 - b^2}{2ca}$. Chứng minh: (a) Nếu $a,b,c$ là độ dài 3 cạnh 1 tam giác thì $A > 1$. (b) Nếu $A = 1$ thì 2 trong 3 phân thức dã cho của biểu thức A bằng $1$ \& phân thức còn lại bằng $-1$.
\end{baitoan}

\begin{baitoan}[\cite{Tuyen_Toan_8}, 179., p. 39]
	Cho biểu thức $A = 1 + \dfrac{x + 3}{x^2 + 5x + 6}:\left(\dfrac{8x^2}{4x^3 - 8x^2} - \dfrac{3x}{3x^2 - 12} - \dfrac{1}{x + 2}\right)$. (a) Rút gọn A. (b) Tìm $x\in\mathbb{R}$ để $A = 0,A = 1$. (c) Tìm $x\in\mathbb{R}$ để $A < 0,A > 0$.
\end{baitoan}

\begin{baitoan}[\cite{Tuyen_Toan_8}, 180., p. 39]
	Cho biểu thức $A = \left(\dfrac{2x - x^2}{2x^2 + 8} - \dfrac{2x^2}{x^3 - 2x^2 + 4x - 8}\right)\left(\dfrac{2}{x^2} + \dfrac{1 - x}{x}\right)$. (a) Rút gọn A. (b) Tìm $x\in\mathbb{Z}$ để $A\in\mathbb{Z}$.
\end{baitoan}

%------------------------------------------------------------------------------%

\printbibliography[heading=bibintoc]
	
\end{document}