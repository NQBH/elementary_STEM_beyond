\documentclass{article}
\usepackage[backend=biber,natbib=true,style=alphabetic,maxbibnames=50]{biblatex}
\addbibresource{/home/nqbh/reference/bib.bib}
\usepackage[utf8]{vietnam}
\usepackage{tocloft}
\renewcommand{\cftsecleader}{\cftdotfill{\cftdotsep}}
\usepackage[colorlinks=true,linkcolor=blue,urlcolor=red,citecolor=magenta]{hyperref}
\usepackage{amsmath,amssymb,amsthm,float,graphicx,mathtools,tikz}
\usetikzlibrary{angles,calc,intersections,matrix,patterns,quotes,shadings}
\allowdisplaybreaks
\newtheorem{assumption}{Assumption}
\newtheorem{baitoan}{Bài toán}
\newtheorem{cauhoi}{Câu hỏi}
\newtheorem{conjecture}{Conjecture}
\newtheorem{corollary}{Corollary}
\newtheorem{dangtoan}{Dạng toán}
\newtheorem{definition}{Definition}
\newtheorem{dinhly}{Định lý}
\newtheorem{dinhnghia}{Định nghĩa}
\newtheorem{example}{Example}
\newtheorem{ghichu}{Ghi chú}
\newtheorem{hequa}{Hệ quả}
\newtheorem{hypothesis}{Hypothesis}
\newtheorem{lemma}{Lemma}
\newtheorem{luuy}{Lưu ý}
\newtheorem{nhanxet}{Nhận xét}
\newtheorem{notation}{Notation}
\newtheorem{note}{Note}
\newtheorem{principle}{Principle}
\newtheorem{problem}{Problem}
\newtheorem{proposition}{Proposition}
\newtheorem{question}{Question}
\newtheorem{remark}{Remark}
\newtheorem{theorem}{Theorem}
\newtheorem{vidu}{Ví dụ}
\usepackage[left=1cm,right=1cm,top=5mm,bottom=5mm,footskip=4mm]{geometry}
\def\labelitemii{$\circ$}
\DeclareRobustCommand{\divby}{%
	\mathrel{\vbox{\baselineskip.65ex\lineskiplimit0pt\hbox{.}\hbox{.}\hbox{.}}}%
}

\title{Problem: Multivariate Polynomial -- Bài Tập: Đa Thức Nhiều Biến}
\author{Nguyễn Quản Bá Hồng\footnote{Independent Researcher, Ben Tre City, Vietnam\\e-mail: \texttt{nguyenquanbahong@gmail.com}; website: \url{https://nqbh.github.io}.}}
\date{\today}

\begin{document}
\maketitle
\tableofcontents

%------------------------------------------------------------------------------%

\section{Multivariate Monomial Polynomial -- Đơn Thức \& Đa Thức Nhiều Biến}

\begin{baitoan}[\cite{Tuyen_Toan_8_new}, Ví dụ 1, p. 4]
	Cho 3 biểu thức $A = \dfrac{4xy}{x^2 - 2xy + y^2}$, $B = x^2 - 2xy + y^2$, $C = -4xy$. (a) Cho biết biểu thức nào là đơn thức nhiều biến, là đa thức nhiều biến? (b) Với $x = -\dfrac{1}{2}$, $y = \dfrac{1}{2}$, chứng minh 2 biểu thức $B,C$ có cùng 1 giá trị.
\end{baitoan}

\begin{baitoan}[\cite{Tuyen_Toan_8_new}, 1., p. 5]
	Cho đơn thức $A = -2mx^3y^4$, $m$ là hằng. Cho biết: (a) Hệ số \& phần biến của đơn thức A. (b) Bậc của đơn thức A đối với từng biến \& đối với tập hợp các biến.
\end{baitoan}

\begin{baitoan}[\cite{Tuyen_Toan_8_new}, 2., p. 5]
	Cho $x^2 = 3$, $y^2 = \frac{1}{3}$. Tính giá trị của đa thức $A = x^4 - x^2y^2 + y^4$.
\end{baitoan}

\begin{baitoan}[\cite{Tuyen_Toan_8_new}, 3., p. 5]
	Tìm các đơn thức đồng dạng trong 5 đơn thức sau ($a\ne0$ là hằng): $P = \frac{4}{5}x^4y^3xy$, $Q = \frac{2}{3}a^3x^3y^2x^2y$, $R = 6a^2x^2y^4ax^3$, $M = -10$, $N = \frac{7}{6}$.
\end{baitoan}

\begin{baitoan}[\cite{Tuyen_Toan_8_new}, 4., p. 5]
	Cho 3 đơn thức nhiều biến: $A = ab^2x^4y^3$, $B = ax^4y^3$, $C = b^2x^4y^3$. Các đơn thức nào đồng dạng với nhau nếu: (a) $a,b$ là hằng $\ne0$ còn $x,y$ là biến. (b) $a\ne0$ là hằng còn $b,x,y$ là biến. (c) $b\ne0$ là hằng còn $a,x,y$ là biến.
\end{baitoan}

\begin{baitoan}[\cite{Tuyen_Toan_8_new}, 5., p. 5]
	Cho biểu thức $A = \dfrac{-4ax^2y^5}{(b + 1)^3}$. Trong 3 trường hợp sau đây, trường hợp nào A là đơn thức? (a) $a,b$ là hằng. (b) $a$ là hằng. (c) $b$ là hằng. Trong trường hợp đó, cho biết hệ số \& bậc của đơn thức đối với mỗi biến \& đối với tập hợp của biến.
\end{baitoan}

%------------------------------------------------------------------------------%

\section{Operators $\pm$ Multivariate Polyonimals -- Phép $\pm$ Đa Thức Nhiều Biến}

\begin{baitoan}[\cite{Tuyen_Toan_8_new}, Ví dụ 2, p. 6]
	Cho 2 đơn thức $A = 3m^2x^2y^3z$, $B = 12x^2y^3z$ ($m\ne0$ là hằng). (a) Tính hiệu $A - B$. (b) Xác định $m$ để giá trị của 2 đơn thức $A,B$ luôn bằng nhau với mọi $x,y,z\in\mathbb{R}$.
\end{baitoan}

\begin{baitoan}[\cite{Tuyen_Toan_8_new}, Ví dụ 3, p. 6]
	Cho 3 đa thức $A = 8a - 9b$, $B = 5b - c$, $C = 3c - 2a$ trong đó $a,b,c\in\mathbb{N}$. Không thực hiện phép tính, cho biết tính $ABC$ có giá trị là số chẵn hay lẻ?
\end{baitoan}

\begin{baitoan}[\cite{Tuyen_Toan_8_new}, 6., p. 7]
	Cho 2 đa thức $A = 3x^4 - 2x^3y + 5xy^3 - y^4$, $B = -8x^4 + 2x^3y - 9x^2y^2 - xy^3 + 4y^4$. Tính tổng $A + B$ \& hiệu $A - B$ bằng 2 cách: Cộng trừ theo hàng ngang. Cộng trừ theo cột dọc.
\end{baitoan}

\begin{baitoan}[\cite{Tuyen_Toan_8_new}, 7., p. 7]
	Chứng minh $o\forall n\in\mathbb{N}^\star$: (a) $8\cdot2^n + 2^{n+1}$ có tận cùng bằng chữ số $0$. (b) $3^{n+3} - 2\cdot3^n + 2^{n+5} - 7\cdot2^n\divby25$. (c) $4^{n+3} + 4^{n+2} - 4^{n+1} - 4^n\divby300$. 
\end{baitoan}

\begin{baitoan}[\cite{Tuyen_Toan_8_new}, 8., p. 7]
	Viết tích $31\cdot5^2$ thành tổng của 3 lũy thừa cơ số $5$ với số mũ là 3 số tự nhiên liên tiếp.
\end{baitoan}

\begin{baitoan}[\cite{Tuyen_Toan_8_new}, 9., p. 7]
	Viết 2 số tự nhiên sau dưới dạng 1 đa thức có 2 biến $x,y$: (a) $\overline{xyz}$. (b) $\overline{yxy5}$.
\end{baitoan}

\begin{baitoan}[\cite{Tuyen_Toan_8_new}, 10., p. 7]
	Cho da thức $P = ax^4y^3 + 10xy^2 + 4y^3 - 2x^4y^3 - 3xy^2 + bx^3y^4$. biết $a,b$ là hằng \& đa thức $P$ có bậc $3$, tìm $a,b$.
\end{baitoan}

\begin{baitoan}[\cite{Tuyen_Toan_8_new}, 11., p. 7]
	Tính tổng $S = \overline{ab} + \overline{abc} + \overline{ba} -\overline{bac}$.
\end{baitoan}

\begin{baitoan}[\cite{Tuyen_Toan_8_new}, 12., p. 7]
	Chứng minh tổng của 4 số lẻ liên tiếp thì chia hết cho $8$.
\end{baitoan}

\begin{baitoan}[\cite{Tuyen_Toan_8_new}, 13., p. 7]
	Cho 3 đa thức $A = 16x^4 - 8x^3y + 7x^2y^2 - 9y^4$, $B = -15x^4 + 3x^3y - 5x^2y^2 - 6y^4$, $C = 5x^3y + 3x^2y^2 + 17y^4 + 1$. Chứng minh ít nhất 1 trong 3 đa thức này có giá trị dương $\forall x,y\in\mathbb{R}$.
\end{baitoan}

\begin{baitoan}[\cite{Tuyen_Toan_8_new}, 14., p. 7]
	Cho đa thức $A = 2x^2 + |7x - 1| - (5 - x + 2x^2)$. (a) Thu gọn A. (b) Tìm $x$ để $A = 2$.
\end{baitoan}

\begin{baitoan}[\cite{Tuyen_Toan_8_new}, 15., p. 7]
	Tính giá trị của 2 đa thức sau biết $x - y = 0$. (a) $A = 7x - 7y + 4ax - 4ay - 5$. (b) $B = x(x^2 + y^2) - y(x^2 + y^2) + 3$.
\end{baitoan}

\begin{baitoan}[\cite{Tuyen_Toan_8_new}, 16., p. 7]
	Cho 2 đa thức $A = xyz - xy^2 - xz^2$, $B = y^3 + z^3$. Chứng minh nếu $x - y - z = 0$ thì $A,B$ là 2 đa thức đối nhau.
\end{baitoan}

\begin{baitoan}[\cite{Tuyen_Toan_8_new}, 17., p. 7]
	Tính giá trị của đa thức $A = 4x^4 + 7x^2y^2 + 3y^4 + 5y^2$ với $x^2 + y^2 = 5$.
\end{baitoan}

%------------------------------------------------------------------------------%

\section{Operators $\cdot,:$ Multivariate Polynomial -- Phép $\cdot,:$ Đa Thức Nhiều Biến}

%------------------------------------------------------------------------------%

\section{Miscellaneous}

%------------------------------------------------------------------------------%

\printbibliography[heading=bibintoc]
	
\end{document}