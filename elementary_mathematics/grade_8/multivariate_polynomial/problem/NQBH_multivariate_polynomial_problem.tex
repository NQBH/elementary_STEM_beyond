\documentclass{article}
\usepackage[backend=biber,natbib=true,style=alphabetic,maxbibnames=50]{biblatex}
\addbibresource{/home/nqbh/reference/bib.bib}
\usepackage[utf8]{vietnam}
\usepackage{tocloft}
\renewcommand{\cftsecleader}{\cftdotfill{\cftdotsep}}
\usepackage[colorlinks=true,linkcolor=blue,urlcolor=red,citecolor=magenta]{hyperref}
\usepackage{amsmath,amssymb,amsthm,float,graphicx,mathtools,tikz}
\usetikzlibrary{angles,calc,intersections,matrix,patterns,quotes,shadings}
\allowdisplaybreaks
\newtheorem{assumption}{Assumption}
\newtheorem{baitoan}{}
\newtheorem{cauhoi}{Câu hỏi}
\newtheorem{conjecture}{Conjecture}
\newtheorem{corollary}{Corollary}
\newtheorem{dangtoan}{Dạng toán}
\newtheorem{definition}{Definition}
\newtheorem{dinhly}{Định lý}
\newtheorem{dinhnghia}{Định nghĩa}
\newtheorem{example}{Example}
\newtheorem{ghichu}{Ghi chú}
\newtheorem{hequa}{Hệ quả}
\newtheorem{hypothesis}{Hypothesis}
\newtheorem{lemma}{Lemma}
\newtheorem{luuy}{Lưu ý}
\newtheorem{nhanxet}{Nhận xét}
\newtheorem{notation}{Notation}
\newtheorem{note}{Note}
\newtheorem{principle}{Principle}
\newtheorem{problem}{Problem}
\newtheorem{proposition}{Proposition}
\newtheorem{question}{Question}
\newtheorem{remark}{Remark}
\newtheorem{theorem}{Theorem}
\newtheorem{vidu}{Ví dụ}
\usepackage[left=1cm,right=1cm,top=5mm,bottom=5mm,footskip=4mm]{geometry}
\def\labelitemii{$\circ$}
\DeclareRobustCommand{\divby}{%
	\mathrel{\vbox{\baselineskip.65ex\lineskiplimit0pt\hbox{.}\hbox{.}\hbox{.}}}%
}

\title{Problem: Multivariate Polynomial $\mathbb{R}[{\bf x}]$ -- Bài Tập: Đa Thức Nhiều Biến $\mathbb{R}[{\bf x}]$}
\author{Nguyễn Quản Bá Hồng\footnote{A Scientist {\it\&} Creative Artist Wannabe. E-mail: {\tt nguyenquanbahong@gmail.com}. Bến Tre City, Việt Nam.}}
\date{\today}

\begin{document}
\maketitle
\begin{abstract}
	This text is a part of the series {\it Some Topics in Elementary STEM \& Beyond}:
	
	{\sc url}: \url{https://nqbh.github.io/elementary_STEM}.
	
	Latest version:
	\begin{itemize}
		\item {\it Problem: Multivariate Polynomial $\mathbb{R}[{\bf x}]$ -- Bài Tập: Đa Thức Nhiều Biến $\mathbb{R}[{\bf x}]$}.
		
		PDF: {\sc url}: \url{https://github.com/NQBH/elementary_STEM_beyond/blob/main/elementary_mathematics/grade_8/multivariate_polynomial/problem/NQBH_multivariate_polynomial_problem.pdf}.
		
		\TeX: {\sc url}: \url{https://github.com/NQBH/elementary_STEM_beyond/blob/main/elementary_mathematics/grade_8/multivariate_polynomial/problem/NQBH_multivariate_polynomial_problem.tex}.
		\item {\it Problem \& Solution: Multivariate Polynomial $\mathbb{R}[{\bf x}]$ -- Bài Tập \& Lời Giải: Đa Thức Nhiều Biến $\mathbb{R}[{\bf x}]$}.
		
		PDF: {\sc url}: \url{https://github.com/NQBH/elementary_STEM_beyond/blob/main/elementary_mathematics/grade_8/multivariate_polynomial/problem/NQBH_multivariate_polynomial_solution.pdf}.
		
		\TeX: {\sc url}: \url{https://github.com/NQBH/elementary_STEM_beyond/blob/main/elementary_mathematics/grade_8/multivariate_polynomial/problem/NQBH_multivariate_polynomial_solution.tex}.
	\end{itemize}
\end{abstract}
\tableofcontents

%------------------------------------------------------------------------------%

\section{Multivariate Monomial Polynomial -- Đơn Thức \& Đa Thức Nhiều Biến}

\begin{baitoan}[\cite{Tuyen_Toan_8}, VD1, p. 4]
	Cho 3 biểu thức $A = \dfrac{4xy}{x^2 - 2xy + y^2}$, $B = x^2 - 2xy + y^2$, $C = -4xy$. (a) Cho biết biểu thức nào là đơn thức nhiều biến, là đa thức nhiều biến? (b) Với $x = -\dfrac{1}{2}$, $y = \dfrac{1}{2}$, chứng minh 2 biểu thức $B,C$ có cùng 1 giá trị.
\end{baitoan}

\begin{baitoan}[\cite{Tuyen_Toan_8}, 1., p. 5]
	Cho đơn thức $A = -2mx^3y^4$, $m$ là hằng. Cho biết: (a) Hệ số \& phần biến của đơn thức A. (b) Bậc của đơn thức A đối với từng biến \& đối với tập hợp các biến.
\end{baitoan}

\begin{baitoan}[\cite{Tuyen_Toan_8}, 2., p. 5]
	Cho $x^2 = 3$, $y^2 = \frac{1}{3}$. Tính giá trị của đa thức $A = x^4 - x^2y^2 + y^4$.
\end{baitoan}

\begin{baitoan}[\cite{Tuyen_Toan_8}, 3., p. 5]
	Tìm các đơn thức đồng dạng trong 5 đơn thức sau ($a\ne0$ là hằng): $P = \frac{4}{5}x^4y^3xy$, $Q = \frac{2}{3}a^3x^3y^2x^2y$, $R = 6a^2x^2y^4ax^3$, $M = -10$, $N = \frac{7}{6}$.
\end{baitoan}

\begin{baitoan}[\cite{Tuyen_Toan_8}, 4., p. 5]
	Cho 3 đơn thức nhiều biến: $A = ab^2x^4y^3$, $B = ax^4y^3$, $C = b^2x^4y^3$. Các đơn thức nào đồng dạng với nhau nếu: (a) $a,b$ là hằng $\ne0$ còn $x,y$ là biến. (b) $a\ne0$ là hằng còn $b,x,y$ là biến. (c) $b\ne0$ là hằng còn $a,x,y$ là biến.
\end{baitoan}

\begin{baitoan}[\cite{Tuyen_Toan_8}, 5., p. 5]
	Cho biểu thức $A = \dfrac{-4ax^2y^5}{(b + 1)^3}$. Trong 3 trường hợp sau đây, trường hợp nào A là đơn thức? (a) $a,b$ là hằng. (b) $a$ là hằng. (c) $b$ là hằng. Trong trường hợp đó, cho biết hệ số \& bậc của đơn thức đối với mỗi biến \& đối với tập hợp của biến.
\end{baitoan}

%------------------------------------------------------------------------------%

\section{Operators $\pm$ Multivariate Polyonimals -- Phép $\pm$ Đa Thức Nhiều Biến}

\begin{baitoan}[\cite{Tuyen_Toan_8}, VD2, p. 6]
	Cho 2 đơn thức $A = 3m^2x^2y^3z$, $B = 12x^2y^3z$ ($m\ne0$ là hằng). (a) Tính hiệu $A - B$. (b) Xác định $m$ để giá trị của 2 đơn thức $A,B$ luôn bằng nhau với mọi $x,y,z\in\mathbb{R}$.
\end{baitoan}

\begin{baitoan}[\cite{Tuyen_Toan_8}, VD3, p. 6]
	Cho 3 đa thức $A = 8a - 9b$, $B = 5b - c$, $C = 3c - 2a$ trong đó $a,b,c\in\mathbb{N}$. Không thực hiện phép tính, cho biết tính $ABC$ có giá trị là số chẵn hay lẻ?
\end{baitoan}

\begin{baitoan}[\cite{Tuyen_Toan_8}, 6., p. 7]
	Cho 2 đa thức $A = 3x^4 - 2x^3y + 5xy^3 - y^4$, $B = -8x^4 + 2x^3y - 9x^2y^2 - xy^3 + 4y^4$. Tính tổng $A + B$ \& hiệu $A - B$ bằng 2 cách: Cộng trừ theo hàng ngang. Cộng trừ theo cột dọc.
\end{baitoan}

\begin{baitoan}[\cite{Tuyen_Toan_8}, 7., p. 7]
	Chứng minh $o\forall n\in\mathbb{N}^\star$: (a) $8\cdot2^n + 2^{n+1}$ có tận cùng bằng chữ số $0$. (b) $3^{n+3} - 2\cdot3^n + 2^{n+5} - 7\cdot2^n\divby25$. (c) $4^{n+3} + 4^{n+2} - 4^{n+1} - 4^n\divby300$. 
\end{baitoan}

\begin{baitoan}[\cite{Tuyen_Toan_8}, 8., p. 7]
	Viết tích $31\cdot5^2$ thành tổng của 3 lũy thừa cơ số $5$ với số mũ là 3 số tự nhiên liên tiếp.
\end{baitoan}

\begin{baitoan}[\cite{Tuyen_Toan_8}, 9., p. 7]
	Viết 2 số tự nhiên sau dưới dạng 1 đa thức có 2 biến $x,y$: (a) $\overline{xyz}$. (b) $\overline{yxy5}$.
\end{baitoan}

\begin{baitoan}[\cite{Tuyen_Toan_8}, 10., p. 7]
	Cho da thức $P = ax^4y^3 + 10xy^2 + 4y^3 - 2x^4y^3 - 3xy^2 + bx^3y^4$. biết $a,b$ là hằng \& đa thức $P$ có bậc $3$, tìm $a,b$.
\end{baitoan}

\begin{baitoan}[\cite{Tuyen_Toan_8}, 11., p. 7]
	Tính tổng $S = \overline{ab} + \overline{abc} + \overline{ba} -\overline{bac}$.
\end{baitoan}

\begin{baitoan}[\cite{Tuyen_Toan_8}, 12., p. 7]
	Chứng minh tổng của 4 số lẻ liên tiếp thì chia hết cho $8$.
\end{baitoan}

\begin{baitoan}[\cite{Tuyen_Toan_8}, 13., p. 7]
	Cho 3 đa thức $A = 16x^4 - 8x^3y + 7x^2y^2 - 9y^4$, $B = -15x^4 + 3x^3y - 5x^2y^2 - 6y^4$, $C = 5x^3y + 3x^2y^2 + 17y^4 + 1$. Chứng minh ít nhất 1 trong 3 đa thức này có giá trị dương $\forall x,y\in\mathbb{R}$.
\end{baitoan}

\begin{baitoan}[\cite{Tuyen_Toan_8}, 14., p. 7]
	Cho đa thức $A = 2x^2 + |7x - 1| - (5 - x + 2x^2)$. (a) Thu gọn A. (b) Tìm $x$ để $A = 2$.
\end{baitoan}

\begin{baitoan}[\cite{Tuyen_Toan_8}, 15., p. 7]
	Tính giá trị của 2 đa thức sau biết $x - y = 0$. (a) $A = 7x - 7y + 4ax - 4ay - 5$. (b) $B = x(x^2 + y^2) - y(x^2 + y^2) + 3$.
\end{baitoan}

\begin{baitoan}[\cite{Tuyen_Toan_8}, 16., p. 7]
	Cho 2 đa thức $A = xyz - xy^2 - xz^2$, $B = y^3 + z^3$. Chứng minh nếu $x - y - z = 0$ thì $A,B$ là 2 đa thức đối nhau.
\end{baitoan}

\begin{baitoan}[\cite{Tuyen_Toan_8}, 17., p. 7]
	Tính giá trị của đa thức $A = 4x^4 + 7x^2y^2 + 3y^4 + 5y^2$ với $x^2 + y^2 = 5$.
\end{baitoan}

\begin{baitoan}[\cite{Binh_Toan_8_tap_1}, VD1, p. 3]
	(a) Xác định dấu của $c$ biết $2a^3bc$ trái dấu với $-3a^5b^3c^2$. (b) Mở rộng.
\end{baitoan}

\begin{baitoan}[\cite{Binh_Toan_8_tap_1}, VD2, p. 3]
	Tính hợp lý giá trị biểu thức $A = 2\dfrac{1}{135}\cdot\dfrac{1}{651} - \dfrac{1}{105}\cdot3\dfrac{650}{651} - \dfrac{4}{315\cdot651} + \dfrac{4}{105}$ bằng cách thay số bởi chữ.
\end{baitoan}

\begin{baitoan}[\cite{Binh_Toan_8_tap_1}, VD3, p. 3]
	Tìm các số có 3 chữ số sao cho hiệu của số ấy \& số gồm 3 chữ số ấy viết theo thứ tự ngược lại là 1 số chính phương.
\end{baitoan}

\begin{baitoan}[\cite{Binh_Toan_8_tap_1}, VD4, p. 3]
	Cho đa thức $P(x) = x^3 + ax^2 + bx + c$ với 3 hằng số $a,b,c\in\mathbb{R}$. Biết $P(2) = P(3) = 0$. Tính $P(5) - P(0)$.
\end{baitoan}

\begin{baitoan}[\cite{Binh_Toan_8_tap_1}, VD5, p. 4]
	Chứng minh đẳng thức $(x - a)(y - a)(z - a) = xyz + a^2(x + y + z) - a(xy + yz + zx) - a^3$.
\end{baitoan}

\begin{baitoan}[\cite{Binh_Toan_8_tap_1}, 1., p. 4]
	Rút gọn biểu thức $(2x - 3y) - (x - 2y)$ với $x = a^2 + 2ab + b^2,y = a^2 - 2ab + b^2$.
\end{baitoan}

\begin{baitoan}[\cite{Binh_Toan_8_tap_1}, 2., p. 4]
	Xác định đa thức A biết $A + 6x^2 - 4xy = 7x^2 - 8xy + y^2$.
\end{baitoan}

\begin{baitoan}[\cite{Binh_Toan_8_tap_1}, 3., p. 4]
	Chứng minh: (a) Tổng của 1 số tự nhiên có 2 chữ số với số gồm 2 chữ số ấy viết theo thứ tự ngược lại là 1 số chia hết cho $11$. (b) Hiệu của 1 số tự nhiên có 2 chữ số với số gồm 2 chữ số ấy viết theo thứ tự ngược lại là 1 số chia hết cho $9$.
\end{baitoan}

\begin{baitoan}[\cite{Binh_Toan_8_tap_1}, 4., p. 4]
	Tìm các số tự nhiên có 2 chữ số sao cho tổng của số ấy \& số viết theo thứ tự ngược lại là 1 số chính phương.
\end{baitoan}

\begin{baitoan}[\cite{Binh_Toan_8_tap_1}, 5., p. 4]
	Tìm số tự nhiên $\overline{abc},a > b > c > 0$ sao cho $\overline{abc} + \overline{bca} + \overline{cab} = 666$.
\end{baitoan}

\begin{baitoan}[\cite{Binh_Toan_8_tap_1}, 6., p. 4]
	Có số tự nhiên $\overline{abc}$ nào mà tổng $\overline{abc} + \overline{bca} + \overline{cab}$ là 1 số chính phương không?
\end{baitoan}

\begin{baitoan}[\cite{Binh_Toan_8_tap_1}, 7., p. 4]
	(a) Tìm số tự nhiên chia hết cho $7$ có 3 chữ số biết tổng các chữố của số đó bằng $14$. (b) Tìm số tự nhiên chia hết cho $7$ có 3 chữ số khác nhau \& tổng các chữ số chia hết cho $7$.
\end{baitoan}

\begin{baitoan}[\cite{Binh_Toan_8_tap_1}, 8., p. 4]
	(a) Tìm số tự nhiên $\overline{abc}$ có 3 chữ số khác nhau sao cho $3a + 5b = 8c$. (b) Tìm số tự nhiên $\overline{abc}$ có 3 chữ số khác nhau \& khác $0$ sao cho $\overline{abc}$ bằng trung bình cộng của $\overline{bca},\overline{cab}$.
\end{baitoan}

\begin{baitoan}[\cite{Binh_Toan_8_tap_1}, 9., p. 4]
	Tìm các số tự nhiên có 2 chữ số sao cho số đó bằng: (a) $6$ lần tích các chữ số của số đó. (b) $2$ lần tích các chữ số của số đó.
\end{baitoan}

\begin{baitoan}[\cite{Binh_Toan_8_tap_1}, 10., p. 4]
	Tìm số tự nhiên $\overline{abcd}$ sao cho số đó chia hết cho tích của $\overline{ab},\overline{cd}$.
\end{baitoan}

\begin{baitoan}[\cite{Binh_Toan_8_tap_1}, 11., p. 4]
	2 đơn thức $-3x^4y,5x^2y^3$ có thể cùng giá trị dương không?
\end{baitoan}

\begin{baitoan}[\cite{Binh_Toan_8_tap_1}, 12., p. 4]
	Chứng minh 3 đơn thức $-\frac{1}{4}x^3y^4,-\frac{4}{5}x^4y^3,\frac{1}{2}xy$ không thể cùng có giá trị âm.
\end{baitoan}

\begin{baitoan}[\cite{Binh_Toan_8_tap_1}, 13., p. 5]
	2 đơn thức $-2a^5b^2,3a^2b^6$ cùng dấu. Tìm dấu của $a$.
\end{baitoan}

\begin{baitoan}[\cite{Binh_Toan_8_tap_1}, 14., p. 5]
	4 đơn thức $ad,-bc,-ac,-bd$ có thể cùng có giá trị âm không?
\end{baitoan}

\begin{baitoan}[\cite{Binh_Toan_8_tap_1}, 15., p. 5]
	Rút gọn biểu thức $(a + b + c)(a^2 + b^2 + c^2 - ab - bc - ca$.
\end{baitoan}

\begin{baitoan}[\cite{Binh_Toan_8_tap_1}, 16., p. 5]
	Cho $a + b + c = 2p$. (a) Chứng minh đẳng thức $2bc + b^2 + c^2 - a^2 = 4p(p - a)$. (b) Viết \& chứng minh các đẳng thức tương tự.
\end{baitoan}

\begin{baitoan}[\cite{Binh_Toan_8_tap_1}, 17., p. 5]
	Cho $a + b + c = 0$. Chứng minh $A = B = C$ với $A = a(a + b)(a + c),B = b(b + c)(b + a),C = c(c + a)(c + b)$.
\end{baitoan}

\begin{baitoan}[\cite{Binh_Toan_8_tap_1}, 18., p. 5]
	Xét 2 ví dụ: $53\cdot57 = 3021,72\cdot78 = 5616$. Xây dựng quy tắc nhân nhẩm 2 số có 2 chữ số, trong đó các chữ số hàng chục bằng nhau, còn các chữ số hàng dơn vị có tổng bằng $10$.
\end{baitoan}

\begin{baitoan}[\cite{Binh_Toan_8_tap_1}, 19., p. 5]
	(a) Tìm đa thức $P(x)$ có các hệ số là số tự nhiên nhỏ hơn $10$ sao cho $P(10) = 520$. (b) Tìm đa thức $P(x)$ có các hệ số là số tự nhiên nhỏ hơn $4$ sao cho $P(4) = 156$.
\end{baitoan}

%------------------------------------------------------------------------------%

\section{Operators $\cdot,:$ Multivariate Polynomial -- Phép $\cdot,:$ Đa Thức Nhiều Biến}

\begin{baitoan}[\cite{Tuyen_Toan_8}, VD4, p. 8]
	Cho 3 đơn thức $A = -3xy^3$, $B = 8xy^2$, $C = \frac{5}{3}x^2y$. Chứng minh 3 đơn thức này không thể cùng có giá trị dương.
\end{baitoan}

\begin{baitoan}[\cite{Tuyen_Toan_8}, VD5, p. 9]
	Chứng minh đẳng thức $(x + y)(x + y + 2) - 2(x + 1)(y + 1) + 2 = x^2y^2$.
\end{baitoan}

\begin{baitoan}[\cite{Tuyen_Toan_8}, VD6, p. 9]
	Tìm giá trị của biểu thức $A = (5x^5 + 5x^4):5x^2 - (2x^4 - 8x^2 - 6x + 12):(2x - 4)$ tại $x = -2$.
\end{baitoan}

\begin{baitoan}[\cite{Tuyen_Toan_8}, 18., p. 9]
	Cho biểu thức $E = x(x - y) + y(x + y) - (x + y)(x - y) - 2y^2$. Với mọi giá trị của $x,y$ thì giá trị của biểu thức $E$ là 1 số âm hay là 1 số dương?
\end{baitoan}

\begin{baitoan}[\cite{Tuyen_Toan_8}, 19., p. 9]
	Cho $xy = 1$. Chứng minh đẳng thức $x(y + 1) + y(x + 1) = (x + 1)(y + 1)$.
\end{baitoan}

\begin{baitoan}[\cite{Tuyen_Toan_8}, 20., p. 9]
	Chứng minh đẳng thức $(x - y)(x^3 + x^2y + xy^2 + y^3) = x^4 - y^4$.
\end{baitoan}

\begin{baitoan}[\cite{Tuyen_Toan_8}, 21., p. 9]
	Tìm $n\in\mathbb{N}$ để mỗi phép chia sau đều là phép chia hết: (a) $7x^{n+2}y^n:4x^3y^4$. (b) $-\frac{2}{3}x^{2n}y^7:\frac{4}{9}x^{n+3}y^n$.
\end{baitoan}

\begin{baitoan}[\cite{Tuyen_Toan_8}, 22., p. 10]
	Tìm $x,y$ biết: $[(x - 2y)(x - 7y) - (x - 2y)(x + 2y)]:(x - 2y) = 18$.
\end{baitoan}

\begin{baitoan}[\cite{Tuyen_Toan_8}, 23., p. 10]
	Tìm giá trị của biểu thức $A = (3x^4 - x^2 - 2x):(3x^2 + 3x + 2) + (x^4 - x^2):(x^2 - x)$ tại $x = -5$.
\end{baitoan}

\begin{baitoan}[\cite{Tuyen_Toan_8}, 24., p. 10]
	Không làm phép chia đa thức, tìm số dư trong phép chia đa thức $f(x)$ cho đa thức $g(x)$ trong 3 trường hợp sau: (a) $f(x) = x^{101} + x^{102} + x^{103} + 51$, $g(x) = x + 1$. (b) $f(x) = 2x^3 - 3x^2 + 4x - 17$, $g(x) = x - 2$. (c) $f(x) = x^4 + 5x^3 + 6x + 30$, $g(x) = x + 5$.
\end{baitoan}

\begin{baitoan}[\cite{Tuyen_Toan_8}, 25., p. 10]
	Tìm các giá trị của $m,n$ để đa thức $A = 2x^4 + 3x^3 - 3x^2 + mx + n$ chia hết cho đa thức $B = x^2 + 1$.
\end{baitoan}

\begin{baitoan}[\cite{Tuyen_Toan_8}, 26., p. 10]
	Chứng minh đa thức $f(x) = (x^2 + 4x - 20)^{51} + (x^3 - 2x - 22)^{50} - 2$ chia hết cho đa thức $x - 3$.
\end{baitoan}

\begin{baitoan}[\cite{Tuyen_Toan_8}, 27., p. 10]
	Cho đa thức $A = -3x^3 + 20x^2 + 20x + 10$. Chia đa thức A cho đa thức B được thương là $3x + 1$ \& dư $x + 6$. Tìm đa thức B.
\end{baitoan}

\begin{baitoan}[\cite{Tuyen_Toan_8}, 28., p. 10]
	Cho đa thức $4x^3 + ax + b$ chia hết cho 2 đa thức $x - 2$ \& $x + 1$. Tính $2a - 3b$.
\end{baitoan}

\begin{baitoan}[\cite{Tuyen_Toan_8}, 29., p. 10]
	Tìm giá trị nguyên của $x$ để giá trị của đa thức $A = 10x^4 - 13x^3 - 9x^2 + x + 19$ chia hết cho giá trị của đa thức $B = 2x - 3$.
\end{baitoan}

%------------------------------------------------------------------------------%

\section{Algebraic Identity -- Hằng Đẳng Thức Đáng Nhớ}

\begin{baitoan}[\cite{Tuyen_Toan_8}, VD7, p. 11]
	Cho $x + y = 9$, $xy = 14$. Tính giá trị của 3 biểu thức: $x - y,x^2 + y^2,x^3 + y^3$.
\end{baitoan}

\begin{baitoan}[\cite{Tuyen_Toan_8}, VD8, p. 12]
	Tìm {\rm GTNN} của biểu thức $A = (x + 3y - 5)^2 - 6xy + 26$.
\end{baitoan}

\begin{baitoan}[\cite{Tuyen_Toan_8}, 30., p. 12]
	Chứng minh đẳng thức: (a) $(2 + 1)(2^2 + 1)(2^4 + 1)(2^8 + 1)(2^{16} + 1) = 2^{32} - 1$. (b) $100^2 + 103^2 + 105^2 + 94^2 = 101^2 + 98^2 + 96^2 + 107^2$.
\end{baitoan}

\begin{baitoan}[Mở rộng \cite{Tuyen_Toan_8}, 30., p. 12]
	Tính: (a) $\prod_{i=1}^n (2^{2^i} + 1) = (2 + 1)(2^{2^1} + 1)(2^{2^2} + 1)(2^{2^3} + 1)\cdots(2^{2^n} + 1)$, $\prod_{i=m}^n (2^{2^i} + 1) = (2^{2^m} + 1)(2^{2^{m+1}} + 1)\cdots(2^{2^n} + 1)$. (b) $\prod_{i=1}^n (a^{2^i} + 1)$, $\prod_{i=m}^n (a^{2^i} + 1)$. (c) $\prod_{i=m}^n (a^{2^i} + b^{2^i})$.
\end{baitoan}

\begin{baitoan}[\cite{Tuyen_Toan_8}, 31., p. 12]
	Tính hợp lý, $\forall a,b\in\mathbb{R}$, $\forall m,n\in\mathbb{N}$, $m\le n$: (a) $\dfrac{258^2 - 242^2}{254^2 - 246^2}$. (b) $263^2 + 74\cdot263 + 37^2$. (c) $136^2 - 92\cdot136 + 46^2$. (d) $(50^2 + 48^2 + 46^2 + \cdots + 2^2) - (49^2 + 47^2 + 45^2 + \cdots + 1^2)$.
\end{baitoan}

\begin{baitoan}[\cite{Tuyen_Toan_8}, 32., p. 12]
	Cho $a,b\in\mathbb{R}$ thỏa $2(a^2 + b^2) = (a - b)^2$. Chứng minh $a,b$ là 2 số đối nhau.
\end{baitoan}

\begin{baitoan}[\cite{Tuyen_Toan_8}, 33., p. 12]
	Cho $a,b,x,y\in\mathbb{R}^\star$ thỏa $(a^2 + b^2)(x^2 + y^2) = (ax + by)^2$. Tìm hệ thức liên hệ giữa 4 số $a,b,x,y$.
\end{baitoan}

\begin{baitoan}[\cite{Tuyen_Toan_8}, 34., p. 12]
	Cho $a^2 + b^2 + c^2 = ab + bc + ca$. Chứng minh $a = b = c$.
\end{baitoan}

\begin{baitoan}[\cite{Tuyen_Toan_8}, 35., p. 12]
	Chứng minh không có $x,y\in\mathbb{R}$ nào thỏa mãn đẳng thức: (a) $3x^2 + y^2 + 10x - 2xy + 26 = 0$. (b) $4x^2 + 3y^2 - 4x + 30y + 78 = 0$.
\end{baitoan}

\begin{baitoan}[\cite{Tuyen_Toan_8}, 36., p. 12]
	Cho $a\in\mathbb{N}$. Chứng minh đẳng thức $(10a + 5)^2 = 100a(a + 1) + 25$. Áp dụng để tính nhẩm $35^2,85^2,105^2$.
\end{baitoan}

\begin{baitoan}[\cite{Tuyen_Toan_8}, 37., p. 13]
	Chứng minh: (a) Biểu thức $A = x^2 + x + 1$ luôn luôn dương $\forall x\in\mathbb{R}$. (b) Biểu thức $B = x^2 - xy + y^2$ luôn luôn dương $\forall x\in\mathbb{R}$ không đồng thời bằng $0$. (c) Biểu thức $C = 4x - 10 - x^2$ luôn luôn âm $\forall x\in\mathbb{R}$. (d) Tìm các biểu thức bậc 2 luôn dương dương, luôn luôn âm tương tự.
\end{baitoan}

\begin{baitoan}[\cite{Tuyen_Toan_8}, 38., p. 13]
	Tìm {\rm GTNN} của biểu thức: (a) $A = 25x^2 + 3y^2 - 10x + 11$. (b) $B = (x - 3)^2 + (x - 11)^2$. (c) $C = (x + 1)(x - 2)(x - 3)(x - 6)$.
\end{baitoan}

\begin{baitoan}[\cite{Tuyen_Toan_8}, 39., p. 13]
	Tìm {\rm GTLN} của biểu thức: (a) $2x - x^2$. (b) $B = 19 - 6x - 9x^2$.
\end{baitoan}

\begin{baitoan}[\cite{Tuyen_Toan_8}, 40., p. 13]
	Chứng minh: (a) 2 số chẵn hơn kém nhau $4$ đơn vị thì hiệu các bình phương của chúng chia hết cho $16$. (b) 2 số lẻ hơn kém nhau $6$ đơn vị thì hiệu bình phương của chúng chia hết cho $24$.
\end{baitoan}

\begin{baitoan}[\cite{Tuyen_Toan_8}, 41., p. 13]
	Cho $x > y > 0$, $x - y = 7,xy = 60$. Không tính $x,y$, tính: (a) $x^2 - y^2$. (b) $x^4 + y^4$.
\end{baitoan}

\begin{baitoan}[\cite{Tuyen_Toan_8}, 42., p. 13]
	Cho $a + b + c = 2p$. Chứng minh: (a) $a^2 - b^2 - c^2 + 2bc = 4(p - b)(p - c)$. (b) $p^2 + (p - a)^2 + (p - b)^2 + (p - c)^2 = a^2 + b^2 + c^2$.
\end{baitoan}

\begin{baitoan}[\cite{Tuyen_Toan_8}, 43., p. 13]
	Cho $a = m^2 + n^2,b^2 = m^2 - n^2,c = 2mn$. Chứng minh $a^2 = b^2 + c^2$.
\end{baitoan}

\begin{baitoan}[\cite{Tuyen_Toan_8}, 44., p. 13]
	Tính giá trị biểu thức: (a) $A = x^3 + 9x^2 + 27x + 27$ với $x = -103$. (b) $B = x^3 - 15x^2 + 75x$ với $x = 25$. (c) $C = (x + 1)(x - 1)(x^2 + x + 1)(x^2 - x + 1)$ với $x = -3$.
\end{baitoan}

\begin{baitoan}[\cite{Tuyen_Toan_8}, 45., p. 13]
	Cho $x - y = 2$. Tính giá trị biểu thức $A = 2(x^3 - y^3) - 3(x + y)^2$.
\end{baitoan}

\begin{baitoan}[\cite{Tuyen_Toan_8}, 46., p. 13]
	Cho $x + y + z = 0$. Chứng minh $x^3 + y^3 + z^3 = 3xyz$.
\end{baitoan}

\begin{baitoan}[\cite{Tuyen_Toan_8}, 47., p. 13]
	Rút gọn biểu thức $A = (x - y - 1)^3 - (x - y + 1)^3 + 6(x - y)^2$.
\end{baitoan}

\begin{baitoan}[\cite{Tuyen_Toan_8}, 48., p. 13]
	Cho $(x + 2y)(x^2 - 2xy + 4y^2) = 0,(x - 2y)(x^2 + 2xy + 4y^2) = 16$. Tìm $x,y$.
\end{baitoan}

\begin{baitoan}[\cite{Tuyen_Toan_8}, 49., p. 13]
	Chứng minh: $742^3 - 692^3\divby200$. (b) $685^3 + 315^3\divby25000$.
\end{baitoan}

\begin{baitoan}[\cite{Tuyen_Toan_8}, 50., p. 13]
	Cho $a + b + c + d = 0$. Chứng minh: $a^3 + b^3 + c^3 + d^3 = 3(b + c)(ad - bc)$.
\end{baitoan}

\begin{baitoan}[\cite{Tuyen_Toan_8}, 51., p. 13]
	Cho $a + b + c = 0$. Chứng minh: (a) $(ab + bc + ca)^2 = a^2b^2 + b^2c^2 + c^2a^2$. (b) $a^4 + b^4 + c^4 = 2(ab + bc + ca)^2$.
\end{baitoan}

\begin{baitoan}[\cite{Tuyen_Toan_8}, 52., p. 14]
	Xác định 2 hệ số $a,b$ để đa thức $A = x^4 - 2x^3 + 3x^2 + ax + b$ là bình phương của 1 đa thức.
\end{baitoan}

\begin{baitoan}[\cite{Tuyen_Toan_8}, 53., p. 14]
	Cho $a + b + c = 0,a^2 + b^2 + c^2 = 1$. Chứng minh $a^4 + b^4 + c^4 = \frac{1}{2}$.
\end{baitoan}

\begin{baitoan}[\cite{Tuyen_Toan_8}, 54., p. 14]
	Cho $a,b,c\in\mathbb{R}$ không đồng thời bằng $0$. Chứng minh có ít nhất 1 trong 3 biểu thức sau có giá trị dương: $x = (a - b + c)^2 + 8ab,y = (a - b + c)^2 + 8bc,z = (a - b + c)^2 - 8ca$.
\end{baitoan}

\begin{baitoan}[\cite{Tuyen_Toan_8}, 55., p. 14]
	Tính tổng các hệ số của tất cả các hạng tử trong khai triển của nhị thức: (a) $(5x - 3)^2$. (b) $(3x - 4y)^{20}$.
\end{baitoan}

\begin{baitoan}[\cite{Tuyen_Toan_8}, 56., p. 14]
	Đa thức $(x + 2)^5$ được khai triển theo lũy thừa giảm của $x$. Biết hạng tử thứ 2 \& hạng tử thứ 3 có giá trị bằng nhau khi cho $x = a,y = b$, trong đó $a,b$ là 2 số thực dương, $a - b = 1$. Tìm $a,b$.
\end{baitoan}

\begin{baitoan}[\cite{Tuyen_Toan_8}, 57., p. 14]
	Tính: (a) $(x + 2)^5$. (b) $(x - 1)^6$. (c) $(x - 1)^5$.
\end{baitoan}

\begin{baitoan}[\cite{Tuyen_Toan_8}, 58., p. 14]
	Tìm số dư của phép chia $38^{10}$ cho $13$ \& $38^9$ cho $13$.
\end{baitoan}

\begin{baitoan}[\cite{Tuyen_Toan_8}, 59., p. 14]
	Chứng minh 2 chữ số tận cùng của $7^{43}$ là $43$.
\end{baitoan}

\begin{baitoan}[\cite{Binh_Toan_8_tap_1}, VD6, p. 5]
	Chứng minh số $3599$ viết được dưới dạng tích của 2 số tự nhiên khác $1$.
\end{baitoan}

\begin{baitoan}[\cite{Binh_Toan_8_tap_1}, VD7, p. 5]
	Tìm $x\in\mathbb{N}$ biết $x + 15,x - 74$ là 2 số chính phương.
\end{baitoan}

\begin{baitoan}[\cite{Binh_Toan_8_tap_1}, VD8, p. 6]
	Chứng minh biểu thức $A = x^2 + 2(x + 1)^2 + 3(x + 2)^2 + 4(x + 3)^2$ viết được dưới dạng tổng các bình phương của 2 biểu thức.
\end{baitoan}

\begin{baitoan}[\cite{Binh_Toan_8_tap_1}, VD9, p. 6]
	Tìm {\rm GTNN} của tổng $a + b + c + d$ biết $\dfrac{a}{2} = \dfrac{b}{4} = \dfrac{c}{6} = \dfrac{d}{b + 8}$.
\end{baitoan}

\begin{baitoan}[\cite{Binh_Toan_8_tap_1}, VD10, p. 6]
	Cho $x + y + z = 0,xy + yz + zx = 0$. Chứng minh $x = y = z$.
\end{baitoan}

\begin{baitoan}[\cite{Binh_Toan_8_tap_1}, VD11, p. 6, tổng các số chính phương đan dấu]
	(a) Tính $A = -1^2 + 2^2 - 3^2 + 4^2 - \cdots - 99^2 + 100^2$. (b) Tính $A = -1^2 + 2^2 - 3^2 + 4^2 - \cdots + (-1)^nn^2$, $\forall n\in\mathbb{N}^\star$.
\end{baitoan}

\begin{baitoan}[\cite{Binh_Toan_8_tap_1}, VD12, p. 7]
	Cho $x + y = a + b,x^2 + y^2 = a^2 + b^2$. Chứng minh $x^3 + y^3 = a^3 + b^3$.
\end{baitoan}

\begin{baitoan}[\cite{Binh_Toan_8_tap_1}, VD13, p. 7]
	(a) Cho $a - b = 1,a^3 - b^3 = 1$. Tính $a,b$. (b) Cho $a + b = m,a - b = n$. Tính $ab,a^3 - b^3$ theo $m,n$.
\end{baitoan}

\begin{baitoan}[\cite{Binh_Toan_8_tap_1}, 20., p. 7]
	Tính: (a) $\dfrac{63^2 - 47^2}{215^2 - 105^2}$. (b) $\dfrac{437^2 - 363^2}{537^2 - 463^2}$.
\end{baitoan}

\begin{baitoan}[\cite{Binh_Toan_8_tap_1}, 21., p. 7]
	So sánh $A = 26^2 - 24^2,B = 27^2 - 25^2$.
\end{baitoan}

\begin{baitoan}[\cite{Binh_Toan_8_tap_1}, 22., p. 7]
	Tìm $x\in\mathbb{R}$ biết $4(x + 1)^2 + (2x - 1)^2 - 8(x - 1)(x + 1) = 11$.
\end{baitoan}

\begin{baitoan}[\cite{Binh_Toan_8_tap_1}, 23., p. 8]
	Rút gọn biểu thức: (a) $2x(2x - 1)^2 - 3x(x + 3)(x - 3) - 4x(x + 1)^2$. (b) $(a - b + c)^2 - (b - c)^2 + 2ab - 2ac$. (c) $(3x + 1)^2 - 2(3x + 1)(3x + 5) + (3x + 5)^2$. (d) $(3 + 1)(3^2 + 1)(3^4 + 1)(3^8 + 1)(3^{16} + 1)(3^{32} + 1)$. (e) $(a + b - c)^2 + (a - b + c)^2 - 2(b - c)^2$. (f) $(a + b + c)^2 + (a - b - c)^2 + (b - c - a)^2 + (c - a - b)^2$. (g) $(a + b + c + d)^2 + (a + b - c - d)^2 + (a + c - b - d)^2 + (a + d - b - c)^2$.
\end{baitoan}

\begin{baitoan}[\cite{Binh_Toan_8_tap_1}, 24., p. 8]
	Tính $A = 1^2 - 2^2 - 3^2 + 4^2 + 5^2 - 6^2 - 7^2 + 8^2 + \cdots + 97^2 - 98^2 - 99^2 + 100^2$.
\end{baitoan}

\begin{baitoan}[\cite{Binh_Toan_8_tap_1}, 25., p. 8]
	Cho $a^2 + b^2 + c^2 = m$. Tính giá trị biểu thức $A = (2a + 2b - c)^2 + (2b + 2c - a)^2 + (2c + 2a - b)^2$ theo $m$.
\end{baitoan}

\begin{baitoan}[\cite{Binh_Toan_8_tap_1}, 26., p. 8]
	Viết số dưới dạng tích của 2 số tự nhiên khác $1$: (a0 $899$. (b) $9991$.
\end{baitoan}

\begin{baitoan}[\cite{Binh_Toan_8_tap_1}, 27., p. 8]
	Chứng minh hiệu $7778^2 - 2223^2$ là 1 số gồm toàn các chữ số như nhau.
\end{baitoan}

\begin{baitoan}[\cite{Binh_Toan_8_tap_1}, 28., p. 8]
	Xác định đa thức $P(x)$ thỏa $P(x - 1) = x^2 - 4x + 7$.
\end{baitoan}

\begin{baitoan}[\cite{Binh_Toan_8_tap_1}, 29., p. 8]
	Chứng minh hằng đẳng thức: (a) $(a + b + c)^2 + a^2 + b^2 + c^2 = (a + b)^2 + (b + c)^2 + (c + a)^2$. (b) $x^4 + y^4 + (x + y)^4 = 2(x^2 + xy + y^2)^2$.
\end{baitoan}

\begin{baitoan}[\cite{Binh_Toan_8_tap_1}, 30., p. 8]
	Cho $a^2 - b^2 = 4c^2$. Chứng minh hằng đẳng thức $(5a - 3b + 8c)(5a - 3b - 8c) = (3a - 5b)^2$.
\end{baitoan}

\begin{baitoan}[\cite{Binh_Toan_8_tap_1}, 31., p. 8]
	Chứng minh nếu $(a^2 + b^2)(x^2 + y^2) = (ax + by)^2$ với $x,y\in\mathbb{R}^\star$ thì $\dfrac{a}{x} = \dfrac{b}{y}$.
\end{baitoan}

\begin{baitoan}[\cite{Binh_Toan_8_tap_1}, 32., p. 8]
	Chứng minh nếu $(a^2 + b^2 + c^2)(x^2 + y^2 + z^2) = (ax + by + cz)^2$ với $x,y,z\in\mathbb{R}^\star$ thì $\dfrac{a}{x} = \dfrac{b}{y} = \dfrac{c}{z}$.
\end{baitoan}

\begin{baitoan}[\cite{Binh_Toan_8_tap_1}, 33., p. 8]
	Cho $(a + b)^2 = 2(a^2 + b^2$. Chứng minh $a = b$.
\end{baitoan}

\begin{baitoan}[\cite{Binh_Toan_8_tap_1}, 34., p. 8]
	Chứng minh $a = b = c$ nếu: (a) $a^2 + b^2 + c^2 = ab + bc + ca$. (b) $(a + b + c)^2 = 3(a^2 + b^2 + c^2)$. (c) $(a + b + c)^2 = 3(ab + bc + ca)$.
\end{baitoan}

\begin{baitoan}[\cite{Binh_Toan_8_tap_1}, 35., p. 8]
	Viết biểu thức dưới dạng tổng của 3 bình phương: (a) $(a + b + c)^2 + a^2 + b^2 + c^2$. (b) $2(a - b)(c - b) + 2(b - a)(c - a) + 2(b - c)(a - c)$.
\end{baitoan}

\begin{baitoan}[\cite{Binh_Toan_8_tap_1}, 36., p. 8]
	Tính giá trị biểu thức $a^4 + b^4 + c^4$ biết $a + b + c = 0$ \&: (a) $a^2 + b^2 + c^2 = 2$. (b) $a^2 + b^2 + c^2 = 1$. (c) $a^2 + b^2 + c^2 = m\in\mathbb{R}$.
\end{baitoan}

\begin{baitoan}[\cite{Binh_Toan_8_tap_1}, 37., p. 8]
	Cho $a + b + c = 0$. Chứng minh $a^4 + b^4 + c^4$ bằng mỗi biểu thức: (a) $2(a^2b^2 + b^2c^2 + c^2a^2)$. (b) $2(ab + bc + ca)^2$. (c) $\frac{1}{2}(a^2 + b^2 + c^2)^2$.
\end{baitoan}

\begin{baitoan}[\cite{Binh_Toan_8_tap_1}, 38., p. 8]
	Chứng minh biểu thức luôn luôn có giá trị dương với mọi giá trị của biến: (a) $9x^2 - 6x + 2$. (b) $x^2 + x + 1$. (c) $2x^2 + 2x + 1$.
\end{baitoan}

\begin{baitoan}[\cite{Binh_Toan_8_tap_1}, 39., p. 9]
	Tìm {\rm GTNN} của biểu thức: (a) $A = x^2 - 3x + 5$. (b) $B = (2x - 1)^2 + (x + 2)^2$.
\end{baitoan}

\begin{baitoan}[\cite{Binh_Toan_8_tap_1}, 40., p. 9]
	Tìm {\rm GTLN} của biểu thức: (a) $A = 4 - x^2 + 2x$. (b) $B = 4x - x^2$.
\end{baitoan}

\begin{baitoan}[\cite{Binh_Toan_8_tap_1}, 41., p. 9]
	Tìm {\rm GTNN} của biểu thức $A = x^2 + y^2 - x - y - xy$.
\end{baitoan}

\begin{baitoan}[\cite{Binh_Toan_8_tap_1}, 42., p. 9]
	Chứng minh: (a) Nếu $p,p^2 + 8$ là 2 số nguyên tố thì $p^2 + 2$ cũng là số nguyên tố. (b) Nếu $p,8p^2 + 1$ là 2 số nguyên tố thì $2p + 1$ cũng là số nguyên tố.
\end{baitoan}

\begin{baitoan}[\cite{Binh_Toan_8_tap_1}, 43., p. 9]
	Chứng minh $999991,1000027$ là 2 hợp số.
\end{baitoan}

\begin{baitoan}[\cite{Binh_Toan_8_tap_1}, 44., p. 9]
	Tính: (a) $(x - 2)^3 - x(x + 1)(x - 1) + 6x(x - 3)$. (b) $(x - 2)(x^2 - 2x + 4)(x + 2)(x^2 + 2x + 4)$.
\end{baitoan}

\begin{baitoan}[\cite{Binh_Toan_8_tap_1}, 45., p. 9]
	Tìm $x\in\mathbb{R}$ biết: (a) $(x - 3)(x^2 + 3x + 9) + x(x + 2)(2 - x) = 1$. (b) $(x + 1)^3 - (x - 1)^3 - 6(x - 1)^2 = -10$.
\end{baitoan}

\begin{baitoan}[\cite{Binh_Toan_8_tap_1}, 46., p. 9]
	Rút gọn biểu thức: (a) $(a + b + c)^3 - (b + c - a)^3 - (c + a - b)^3 - (a + b - c)^3$. (b) $(a + b)^3 + (b + c)^3 + (c + a)^3 - 3(a + b)(b + c)(c + a)$.
\end{baitoan}

\begin{baitoan}[\cite{Binh_Toan_8_tap_1}, 47., p. 9]
	Chứng minh hằng đẳng thức: (a) $(a + b + c)^3 - a^3 - b^3 - c^3 = 3(a + b)(b + c)(c + a)$. (b) $a^3 + b^3 + c^3 - 3abc = (a + b + c)(a^2 + b^2 + c^2 - ab - bc - ca)$.
\end{baitoan}

\begin{baitoan}[\cite{Binh_Toan_8_tap_1}, 48., p. 9]
	Cho $a + b + c = 0$. Chứng minh $a^3 + b^3 + c^3 = 3abc$.
\end{baitoan}

\begin{baitoan}[\cite{Binh_Toan_8_tap_1}, 49., p. 9]
	Cho $x + y = a,xy = b$ với $a,b\in\mathbb{R}$ thỏa điều kiện $a^2\ge4b$. Tính giá trị biểu thức theo $a,b$: (a) $x^2 + y^2$. (b) $x^3 + y^3$. (c) $x^4 + y^4$. (d) $x^5 + y^5$.
\end{baitoan}

\begin{baitoan}[\cite{Binh_Toan_8_tap_1}, 50., p. 9]
	(a) Cho $x + y = 1$. Tính giá trị biểu thức $x^3 + y^3 + 3xy$. (b) Cho $x - y = 1$. Tính giá trị biểu thức $x^3 - y^3 - 3xy$. 
\end{baitoan}

\begin{baitoan}[\cite{Binh_Toan_8_tap_1}, 51., p. 9]
	Cho $a + b = 1$. Tính giá trị biểu thức $A = a^3 + b^3 + 3ab(a^2 + b^2) + 6a^2b^2(a + b)$.
\end{baitoan}

\begin{baitoan}[\cite{Binh_Toan_8_tap_1}, 52., p. 9]
	(a) Cho $x + y = 2,x^2 + y^2 = 10$. Tính giá trị biểu thức $x^3 + y^3$. (b) Cho $x + y = a,x^2 + y^2 = b$ với $a,b\in\mathbb{R}$ thỏa điều kiện $2b\ge a^2$. Tính giá trị biểu thức $x^3 + y^3$ theo $a,b$.
\end{baitoan}

\begin{baitoan}[\cite{Binh_Toan_8_tap_1}, 53., p. 9]
	Cho $a + b + c = 4,a^2 + b^2 + c^2 = 30$. Tính giá trị biểu thức $a^3 + b^3 + c^3 - 3abc$.
\end{baitoan}

\begin{baitoan}[\cite{Binh_Toan_8_tap_1}, 54., p. 9]
	Cho $(a + b + c)^3 - (b + c - a)^3 - (c + a - b)^3 - (a + b - c)^3 = 1$. Tính $abc$.
\end{baitoan}

\begin{baitoan}[\cite{Binh_Toan_8_tap_1}, 55., p. 9]
	Cho $x + y = 1,ab(x^2 + y^2) + xy(a^2 + b^2) = ab$. Biết $xy\ne0$, chứng minh $a = b$.
\end{baitoan}

\begin{baitoan}[\cite{Binh_Toan_8_tap_1}, 56., p. 9]
	Chứng minh nếu $a + b = c$ thì $a^4 + b^4 + c^4 = 2a^2b^2 + 2b^2c^2 + 2c^2a^2$.
\end{baitoan}

\begin{baitoan}[\cite{Binh_Toan_8_tap_1}, 57., pp. 9--10]
	Chứng minh: (a) Nếu $n$ là tổng của 2 số chính phương thì $2n$ cũng là tổng của 2 số chính phương. (b) Nếu $2n$ là tổng của 2 số chính phương thì $n$ cũng là tổng của 2 số chính phương. (c) Nếu $n$ là tổng của 2 số chính phương thì $n^2$ cũng là tổng của 2 số chính phương. (d) Nếu $m,n$ đều là tổng của 2 số chính phương thì tích $mn$ cũng là tổng của 2 số chính phương.
\end{baitoan}

\begin{baitoan}[\cite{Binh_Toan_8_tap_1}, 58., p. 10]
	Mỗi số sau là bình phương của số tự nhiên nào? (a) $A = \underbrace{9\ldots9}_n\underbrace{0\ldots0}_n25$. (b) $B = \underbrace{9\ldots9}_n8\underbrace{0\ldots0}_n1$. (c) $C = \underbrace{4\ldots4}_n\underbrace{8\ldots8}_{n-1}9$. (d) $D = \underbrace{1\ldots1}_n\underbrace{2\ldots2}_{n+1}5$. (e) $E = \underbrace{1\ldots1}_{2n} - \underbrace{2\ldots2}_n$. (f) $F = \underbrace{1\ldots1}_{2n} + \underbrace{4\ldots4}_n + 1$.
\end{baitoan}

\begin{baitoan}[\cite{Binh_Toan_8_tap_1}, 59., p. 10]
	(a) Cho $a = \underbrace{1\ldots1}_n,b = 1\underbrace{0\ldots0}_{n-1}5$. Chứng minh $ab + 1$ là số chính phương. (b) Cho 1 dãy số có số hạng đầu là $16$, các số hạng sau là số tạo thành bằng cách viết chèn số $15$ vào chính giữa số hạng liền trước: $16,1156,111556,\ldots$ Chứng minh mọi số hạng của dãy đều là số chính phương.
\end{baitoan}

\begin{baitoan}[\cite{Binh_Toan_8_tap_1}, 60., p. 10]
	Chứng minh $ab + 1$ là số chính phương với $a = \underbrace{1\ldots1}_n2,b = \underbrace{1\ldots1}_n4$.
\end{baitoan}

\begin{baitoan}[\cite{Binh_Toan_8_tap_1}, 61., p. 10]
	Chứng minh $\forall a\in\mathbb{N}$, tồn tại $b\in\mathbb{N}$b sao cho $ab + 4$ là số chính phương.
\end{baitoan}

\begin{baitoan}[\cite{Binh_Toan_8_tap_1}, 62., p. 10]
	Cho $a = \underbrace{1\ldots1}_{2n},b = \underbrace{1\ldots1}_{n+1},c = \underbrace{6\ldots6}_n$. Chứng minh $a + b + c + 8$ là số chính phương.
\end{baitoan}

\begin{baitoan}[\cite{Binh_Toan_8_tap_1}, 63., p. 10]
	Chứng minh $a$ là số chính phương nếu $a,b\in\mathbb{N}$ thỏa mãn: (a) $a^2 + b^2 = 2a(2a + b)$. (b) $a^2 + b^2 + 2ab = 6a + 2b - 1$.
\end{baitoan}

\begin{baitoan}[\cite{Binh_Toan_8_tap_1}, 64., p. 10]
	Cho $a,b,c\in\mathbb{N}$ thỏa mãn $a^2 + b^2 + c^2 = 2(ab + bc + ca)$. Chứng minh: (a) $ab + bc + ca$ là số chính phương. (b) $ab,bc,ca$ là 3 số chính phương.
\end{baitoan}

\begin{baitoan}[\cite{Binh_Toan_8_tap_1}, 65., p. 10]
	Tìm số chính phương lớn nhất nhỏ hơn $1\underbrace{0\ldots0}_{20}1\underbrace{0\ldots0}_{21}84$.
\end{baitoan}

\begin{baitoan}[\cite{Binh_Toan_8_tap_1}, 66., p. 10]
	(a) Chứng minh biểu thức $10^{150} + 5\cdot10^{50} + 1$ không là lập phương của 1 số tự nhiên. (b) Chứng minh tích 3 số nguyên dương liên tiếp không là lập phương của 1 số tự nhiên.
\end{baitoan}

\begin{baitoan}[\cite{Binh_Toan_8_tap_1}, 67., p. 10]
	Chứng minh số $A = \frac{1}{3}\left(\underbrace{1\ldots1}_{3n} - \underbrace{3\ldots3}_n\underbrace{0\ldots0}_n\right)$ là lập phương của 1 số tự nhiên.
\end{baitoan}

\begin{baitoan}[\cite{Binh_Toan_8_tap_1}, 68., p. 10]
	Chia $27$ quả cân có khối lượng $10,20,30,\ldots,270$ {\rm g} thành 3 nhóm có khối lượng bằng nhau.
\end{baitoan}

\begin{baitoan}[\cite{Binh_Toan_8_tap_1}, 69., p. 10]
	(a) Chia $18$ quả cân có khối lượng $1^2,2^2,\ldots,18^2$ {\rm g} thành 3 nhóm có khối lượng bằng nhau. (b) Chia $27$ quả cân có khối lượng $1^2,2^2,\ldots,27^2$ {\rm g} thành 3 nhóm có khối lượng bằng nhau.
\end{baitoan}

%------------------------------------------------------------------------------%

\section{Phân Tích Đa Thức Thành Nhân Tử. Các Phương Pháp Thông Thường}

\begin{baitoan}[\cite{Tuyen_Toan_8}, VD9, p. 15]
	Cho $x,y\in\mathbb{R}$, $x\ne y$, thỏa $9x(x - y) - 10(y - x)^2 = 0$. Chứng minh $x = 10y$.
\end{baitoan}

\begin{baitoan}[\cite{Tuyen_Toan_8}, VD10, p. 15]
	Cho $A = 4a^2b^2 - (a^2 + b^2 + c^2)^2$ trong đó $a,b,c\in\mathbb{R}$ là độ dài 3 cạnh 1 tam giác. Chứng minh $A > 0$.
\end{baitoan}

\begin{baitoan}[\cite{Tuyen_Toan_8}, 60., p. 16]
	Phân tích đa thức thành nhân tử: (a) $5x(x - 2y) + 2(2y - x)^2$. (b) $7x(y - 4)^2 - (4 - y)^3$. (c) $(4x - 8)(x^2 + 6) - (4x - 8)(x + 7) + 9(8 - 4x)$.
\end{baitoan}

\begin{baitoan}[\cite{Tuyen_Toan_8}, 61., p. 16]
	Chứng minh: (a) $43^2 + 43\cdot17\divby60$. (b) $27^5 - 3^{11}\divby80$.
\end{baitoan}

\begin{baitoan}[\cite{Tuyen_Toan_8}, 62., p. 16]
	Tìm 1 số biết 3 lần bình phương của nó đúng bằng 2 lần lập phương của số ấy.
\end{baitoan}

\begin{baitoan}[\cite{Tuyen_Toan_8}, 63., p. 16]
	Có $x,y,z\in\mathbb{Z}$ nào thỏa mãn đồng thời:x
	\begin{equation*}
		\left\{\begin{split}
			x^3 + xyz &= 957,\\
			y^3 + xyz &= 795,\\
			z^3 + xyz &= 579.
		\end{split}\right.
	\end{equation*}
\end{baitoan}

\begin{baitoan}[\cite{Tuyen_Toan_8}, 64., p. 16]
	Chứng minh số $\underbrace{1\ldots1}_n\underbrace{2\ldots2}_n$ là tích 2 số nguyên liên tiếp.
\end{baitoan}
Phân tích đa thức thành nhân tử:

\begin{baitoan}[\cite{Tuyen_Toan_8}, 65., p. 16]
    (a) $100x^2 - (x^2 + 25)^2$. (b) $(x - y + 5)^2 - 2(x - y + 5) + 1$.
\end{baitoan}

\begin{baitoan}[\cite{Tuyen_Toan_8}, 66., p. 16]
	$(x^2 + 4y^2 - 5)^2 - 16(x^2y^2 + 2xy + 1)$.
\end{baitoan}

\begin{baitoan}[\cite{Tuyen_Toan_8}, 67., p. 16]
	Chứng minh: (a) $21^{10} - 1\divby200$. (b) $39^{20} + 39^{13}\divby40$. (c) $2^{60} + 5^{30}\divby41$. (d) $2025^{2027} + 2027^{2025}\divby2026$.
\end{baitoan}

\begin{baitoan}[\cite{Tuyen_Toan_8}, 68., p. 16]
	Cho $n\in\mathbb{N}$ lẻ. Chứng minh $24^n + 1\divby25$ nhưng $24^n + 1\not{\divby}\ 23$.
\end{baitoan}

\begin{baitoan}[\cite{Tuyen_Toan_8}, 69., p. 16]
	Cho $a\in\mathbb{N}$ lẻ, $a > 1$. Chứng minh $(a - 1)^{\frac{1}{2}(a - 1)} - 1\divby a - 2$.
\end{baitoan}
Phân tích đa thức thành nhân tử:

\begin{baitoan}[\cite{Tuyen_Toan_8}, 70., p. 16]
	(a) $x^2 - xz - 9y^2 + 3yz$. (b) $x^3 - x^2 - 5x + 125$. (c) $x^3 + 2x^2 - 6x - 27$. (d) $12x^3 + 4x^2 - 27x - 9$.
\end{baitoan}

\begin{baitoan}[\cite{Tuyen_Toan_8}, 71., p. 16]
	(a) $x^4 - 25x^2 + 20x - 4$. (b) $x^2(x^2 - 6) - x^2 + 9$. (c) $ab(x^2 + y^2) - xy(a^2 + b^2)$.
\end{baitoan}

\begin{baitoan}[\cite{Tuyen_Toan_8}, 72., p. 16]
	Tìm các cặp số $x,y\in\mathbb{R}$ sao cho $x - y = xy - 1$.
\end{baitoan}

\begin{baitoan}[\cite{Tuyen_Toan_8}, 73., p. 16]
	Cho $x,y\in\mathbb{R}$, $x\ne y$ sao cho $x^2 - y = y^2 - x$. Tính giá trị biểu thức $A = x^2 + 2xy + y^2 - 3x - 3y$.
\end{baitoan}

\begin{baitoan}[\cite{Tuyen_Toan_8}, 74., p. 16]
	Cho $\dfrac{a - b}{b - c} = \dfrac{c - d}{d - a}$. Chứng minh $a = c$ hoặc $a + c = b + d$.
\end{baitoan}
Phân tích đa thức thành nhân tử:

\begin{baitoan}[\cite{Tuyen_Toan_8}, 75., p. 17]
	(a) $4x^4 + 4x^3 - x^2 - x$. (b) $x^6 - x^4 - 9x^3 + 9x^2$. (c) $x^4 - 4x^3 + 8x^2 - 16x + 16$.
\end{baitoan}

\begin{baitoan}[\cite{Tuyen_Toan_8}, 76., p. 17]
	(a) $(xy + 4)^2 - 4(x + y)^2$. (b) $(ab - xy)^2 - (bx - ay)^2$. (c) $(x^2 + 8x - 34)^2 - (3x^2 - 8x - 2)^2$.
\end{baitoan}

\begin{baitoan}[\cite{Tuyen_Toan_8}, 77., p. 17]
	(a) $(a + b + c)^2 + (a - b + c)^2 - 4b^2$. (b) $a(b^2 - c^2) - b(c^2 - a^2) + c(a^2 - b^2)$. (c) $a^5 + b^5 - (a + b)^5$.
\end{baitoan}

\begin{baitoan}[\cite{Tuyen_Toan_8}, 78., p. 17]
	Chứng minh: (a) $999^4 + 999$ tận cùng 3 chữ số $0$. (b) $49^5 - 49\divby100$.
\end{baitoan}

\begin{baitoan}[\cite{Tuyen_Toan_8}, 79., p. 17]
	Chứng minh: (a) Lập phương của 1 số nguyên trừ đi số nguyên đó thì chia hết cho $6$. (b) Nếu tổng của 3 số nguyên chia hết cho $6$ thì tổng các lập phương của chúng chia hết cho $6$.
\end{baitoan}

\begin{baitoan}[\cite{Tuyen_Toan_8}, 80., p. 17]
	Cho $a\ne\pm b,a(a + b)(a + c) = b(b + c)(b + a)$. Chứng minh $a + b + c$.
\end{baitoan}

\begin{baitoan}[\cite{Tuyen_Toan_8}, 81., p. 17]
	Cho $x^2y - y^2x + x^2z - z^2x + y^2z + z^2y = 2xyz$. Chứng minh trong 3 số $x,y,z$ ít nhất cũng có 2 số bằng nhau hoặc đối nhau.
\end{baitoan}

\begin{baitoan}[\cite{Tuyen_Toan_8}, 82., p. 17]
	1 tập hợp gồm $n\in\mathbb{N}$ số nguyên dương khác nhau có tổng là $360$, $n > 2$. Chia tập hợp này thành 2 tập hợp con của $A,B$ sao cho chúng không có phần tử chung, tập hợp A gồm có 2 phần tử, tập hợp B gồm các phần tử còn lại. Hỏi có tồn tại hay không cách chia như trên để tích các phần tử của A bằng tổng các phần tử của B.
\end{baitoan}

\begin{baitoan}[\cite{Binh_Toan_8_tap_1}, VD14, p. 11]
	Phân tích đa thức thành nhân tử: $x^4 + x^3 + 2x^2 + x + 1$.
\end{baitoan}

\begin{baitoan}[\cite{Binh_Toan_8_tap_1}, VD15, p. 11]
	Cho $a + b + c = 0$. Rút gọn biểu thức $A = a^3 + b^3 + c(a^2 + b^2) - abc$.
\end{baitoan}

\begin{baitoan}[\cite{Binh_Toan_8_tap_1}, VD16, p. 11]
	Phân tích đa thức thành nhân tử: $A = (x - y)^3 + (y - z)^3 + (z - x)^3$.
\end{baitoan}

\begin{baitoan}[\cite{Binh_Toan_8_tap_1}, VD17, p. 11]
	(a) Phân tích đa thức thành nhân tử: $a^3 + b^3 + c^3 - 3abc$. (b) Cho $a^3 + b^3 + c^3 = 3abc$. Chứng minh $a + b + c = 0$ hoặc $a = b = c$. (c) Cho $a^3 + b^3 + 3ab = 1$. Tính giá trị biểu thức $a + b$.
\end{baitoan}

\begin{baitoan}[\cite{Binh_Toan_8_tap_1}, VD18, p. 11]
	Phân tích đa thức thành nhân tử: (a) $(a + b + c)^3 - a^3 - b^3 - c^3$. (b) $8(x + y + z)^3 - (x + y)^3 - (y + z)^3 - (z + x)^3$.
\end{baitoan}

\begin{baitoan}[\cite{Binh_Toan_8_tap_1}, VD19, p. 12]
	Phân tích đa thức thành nhân tử: $A = x^2(y - z) + y^2(z - x) + z^2(x - y)$.
\end{baitoan}

\begin{baitoan}[\cite{Binh_Toan_8_tap_1}, VD20, p. 12]
	Xét hằng đẳng thức $(x + 1)^3 = x^3 + 3x^2 + 3x + 1$. Lần lượt cho $x$ bằng $1,2,\ldots,n$ rồi cộng từng vế $n$ đẳng thức thu được để tính giá trị biểu thức $S(n) = \sum_{i=1}^n i^2 = 1^2 + 2^2 + \cdots + n^2$.
\end{baitoan}

\begin{baitoan}[\cite{Binh_Toan_8_tap_1}, VD21, pp. 12--13]
	(a) Cho $A(n) = \sum_{i=1}^n i(i + 1) = 1\cdot2 + 2\cdot3 + \ldots + n(n + 1)$, $\forall n\in\mathbb{N}^\star$. Chứng minh $A(n) = \frac{1}{3}n(n + 1)(n + 2)$. (b) Áp dụng để tính $S_2(n) = \sum_{i=1}^n i^2 = 1^2 + 2^2 + \cdots + n^2$.
\end{baitoan}
Phân tích đa thức thành nhân tử:

\begin{baitoan}[\cite{Binh_Toan_8_tap_1}, 70., p. 13]
	(a) $(ab - 1)^2 + (a + b)^2$. (b) $x^3 + 2x^2 + 2x + 1$. (c) $x^3 - 4x^2 + 12x - 27$. (d) $x^4 - 2x^3 + 2x - 1$. (e) $x^4 + 2x^3 + 2x^2 + 2x + 1$.
\end{baitoan}

\begin{baitoan}[\cite{Binh_Toan_8_tap_1}, 71., p. 13]
	(a) $x^2 - 2x - 4y^2 - 4y$. (b) $x^4 + 2x^3 - 4x - 4$. (c) $x^2(1 - x^2) - 4 - 4x^2$. (d) $(1 + 2x)(1 - 2x) - x(x + 2)(x - 2)$. (e) $x^2 + y^2 - x^2y^2 + xy - x - y$.
\end{baitoan}

\begin{baitoan}[\cite{Binh_Toan_8_tap_1}, 72., p. 13]
	Chứng minh $199^3 - 199\divby200$.
\end{baitoan}

\begin{baitoan}[\cite{Binh_Toan_8_tap_1}, 73., p. 13]
	Tìm $a,b\in\mathbb{R}$ biết $a^4 + b^4 = a^5 + b^5 = a^6 + b^6$.
\end{baitoan}

\begin{baitoan}[\cite{Binh_Toan_8_tap_1}, 74., p. 13]
	Tính giá trị biểu thức $A = x^6 - 2x^4 + x^3 + x^2 - x$ biết $x^3 - x = 6$.
\end{baitoan}

\begin{baitoan}[\cite{Binh_Toan_8_tap_1}, 75., p. 13]
	Tính $x + y$ biết $x^3 + y^3 + 4xy = x^2 + y^2 + 4$.
\end{baitoan}
Phân tích đa thức thành nhân tử:

\begin{baitoan}[\cite{Binh_Toan_8_tap_1}, 76., p. 13]
	(a) $a(b^2 + c^2 + bc) + b(c^2 + a^2 + ca) + c(a^2 + b^2 + ab)$. (b) $(a + b + c)(ab + bc + ca) - abc$. (c) $a(a + 2b)^3 - b(2a + b)^3$.
\end{baitoan}

\begin{baitoan}[\cite{Binh_Toan_8_tap_1}, 77., pp. 13--14]
	(a) $ab(a + b) - bc(b + c) + ac(a - c)$. (b) $a(b^2 + c^2) + b(c^2 + a^2) + c(a^2 + b^2) + 2abc$. (c) $(a + b)(a^2 - b^2) + (b + c)(b^2 - c^2) + (c + a)(c^2 - a^2)$. (d) $a^3(b - c) + b^3(c - a) + c^3(a - b)$. (e) $a^3(c - b^2) + b^3(a - c^2) + c^3(b - a^2) + abc(abc - 1)$.
\end{baitoan}

\begin{baitoan}[\cite{Binh_Toan_8_tap_1}, 78., p. 14]
	(a) $a(b + c)^2(b - c) + b(c + a)^2(c - a) + c(a + b)^2(a - b)$. (b) $a(b - c)^3 + b(c - a)^3 + c(a - b)^3$. (c) $a^2b^2(a - b) + b^2c^2(b - c) + c^2a^2(c - a)$. (d) $a(b^2 + c^2) + b(c^2 + a^2) + c(a^2 + b^2) - 2abc - a^3 - b^3 - c^3$. (e) $a^4(b - c) + b^4(c - a) + c^4(a - b)$.
\end{baitoan}

\begin{baitoan}[\cite{Binh_Toan_8_tap_1}, 79., p. 14]
	(a) $(a + b + c)^3 - (a + b - c)^3 - (b + c - a)^3 - (c + a - b)^3$. (b) $abc - (ab + bc + ca) + a + b + c - 1$.
\end{baitoan}

\begin{baitoan}[\cite{Binh_Toan_8_tap_1}, 80., p. 14]
	Chứng minh nếu $a^2(b - c) + b^2(c - a) + c^2(a - b) = 0$ thì tồn tại 2 trong 3 số $a,b,c$ bằng nhau.
\end{baitoan}

\begin{baitoan}[\cite{Binh_Toan_8_tap_1}, 81., p. 14]
	Chứng minh: (a) Nếu $a^2 + b^2 = 2ab$ thì $a = b$. (b) Nếu $a^3 + b^3 + c^3 = 3abc$ \& $a,b,c > 0$ thì $a = b = c$. (c) Nếu $a^4 + b^4 + c^4 + d^4 = 4abcd$ \& $a,b,c,d > 0$ thì $a = b = c = d$.
\end{baitoan}

\begin{baitoan}[\cite{Binh_Toan_8_tap_1}, 82., p. 14]
	Chứng minh: (a) Nếu $ab + bc + ca = 1$ thì $1 + a^2 = (a + b)(a + c)$. (b) Nếu $a + b + c = 1$ thì $a + bc = (a + b)(a + c)$. (c) Nếu $m = a + b + c$ thì $(am + bc)(bm + ca)(cm + ab) = (a + b)^2(b + c)^2(c + a)^2$.
\end{baitoan}

\begin{baitoan}[\cite{Binh_Toan_8_tap_1}, 83., p. 14]
	Cho $a^2 + b^2 = 1,c^2 + d^2 = 1,ac + bd = 0$. Chứng minh $ab + cd = 0$.
\end{baitoan}

\begin{baitoan}[\cite{Binh_Toan_8_tap_1}, 84., p. 14]
	Xét hằng đẳng thức $(x + 1)^2 = x^2 + 2x + 1$. Lần lượt cho $x$ bằng $1,2,\ldots,n$ rồi cộng từng vế $n$ đẳng thức thu được để tính giá trị biểu thức $S_1(n) = \sum_{i=1}^n i = 1 + 2 + \cdots + n$.
\end{baitoan}

\begin{baitoan}[\cite{Binh_Toan_8_tap_1}, 85., p. 14]
	(a) Tính $S_3(n) = \sum_{i=1}^n i^3 = 1^3 + 2^3 + \cdots + n^3$. (b) Chứng minh $S_3(n) = S_1^2(n)$, $\forall n\in\mathbb{N}^\star$.
\end{baitoan}

\begin{baitoan}[\cite{Binh_Toan_8_tap_1}, 86., p. 14]
	Tính: (a) $A(n) = \sum_{i=1}^n i(i + 2) = 1\cdot3 + 2\cdot5 + \cdots + n(n + 2)$. (b) $B(n) = \sum_{i=1}^n i(i + 1)(i + 2) = 1\cdot2\cdot3 + 2\cdot3\cdot4 + \cdots + n(n + 1)(n + 2)$. (c) $C(n) = \sum_{i=1}^n i(i + 1)(2i + 1) = 1\cdot2\cdot3 + 2\cdot3\cdot5 + \cdots + n(n + 1)(2n + 1)$.
\end{baitoan}

%------------------------------------------------------------------------------%

\section{Phân Tích Đa Thức Thành Nhân Tử Bằng 1 Số Phương Pháp Khác}
Phân tích đa thức thành nhân tử:

\begin{baitoan}[\cite{Tuyen_Toan_8}, VD11, p. 17]
	$A = 4x^2 - 8x + 3$.
\end{baitoan}

\begin{baitoan}[\cite{Tuyen_Toan_8}, VD12, p. 18]
	$A = 4x^4 + y^4$.
\end{baitoan}

\begin{baitoan}[\cite{Tuyen_Toan_8}, VD13, p. 18]
	$A = (x^2 - 3x - 1)^2 - 12(x^2 - 3x - 1) + 27$.
\end{baitoan}

\begin{baitoan}[\cite{Tuyen_Toan_8}, VD14, p. 19]
	Phân tích đa thức thành tích của 2 tam thức bậc 2 với hệ số nguyên: $A = x^4 - 3x^3 + 6x^2 - 5x + 3$.
\end{baitoan}
Phân tích đa thức thành nhân tử:

\begin{baitoan}[\cite{Tuyen_Toan_8}, 83., p. 19]
	(a) $3x^2 - 11x + 6$. (b) $8x^2 + 10x - 3$. (c) $8x^2 - 2x - 1$.
\end{baitoan}

\begin{baitoan}[\cite{Tuyen_Toan_8}, 84., p. 19]
	(a) $6x^2 + 7xy + 2y^2$. (b) $9x^2 - 9xy - 4y^2$. (c) $x^2 - y^2 + 10x - 6y + 16$.
\end{baitoan}

\begin{baitoan}[\cite{Tuyen_Toan_8}, 85., p. 19]
	(a) $x^3 + x + 2$. (b) $x^3 - 2x - 1$. (c) $x^3 + 3x^2 - 4$.
\end{baitoan}

\begin{baitoan}[\cite{Tuyen_Toan_8}, 86., p. 19]
	(a) $x^3y^3 + x^2y^2 + 4$. (b) $x^3 + 3x^2y - 9xy^2 + 5y^3$.
\end{baitoan}

\begin{baitoan}[\cite{Tuyen_Toan_8}, 87., p. 20]
	(a) $x^4 + x^3 + 6x^2 + 5x + 5$. (b) $x^4 - 2x^3 - 12x^2 + 12x + 36$. (c) $x^8y^8 + x^4y^4 + 1$.
\end{baitoan}

\begin{baitoan}[\cite{Tuyen_Toan_8}, 88., p. 20]
	(a) $x^5 - x^4 + x^3 - x^2 - 2x + 2$. (b) $x^5 + x^4 - x^3 + x^2 - x + 2$.
\end{baitoan}

\begin{baitoan}[\cite{Tuyen_Toan_8}, 89., p. 20]
	(a) $x^4 + y^4 + (x + y)^4$. (b) $2(x^2 + x + 1)^2 - (2x + 1)^2 - (x^2 + 2x)^2$.
\end{baitoan}

\begin{baitoan}[\cite{Tuyen_Toan_8}, 90., p. 20]
	(a) $xy(x + y) + yz(y + z) + zx(z + x) + 3xyz$. (b) $xy(x + y) - yz(y + z) - zx(z - x)$. (c) $x(y^2 - z^2) + y(z^2 - x^2) + z(x^2 - y^2)$.
\end{baitoan}

\begin{baitoan}[\cite{Tuyen_Toan_8}, 91., p. 20]
	Cho $a\in\mathbb{Z}$. Chứng minh $a^5 - a\divby30$.
\end{baitoan}

\begin{baitoan}[\cite{Tuyen_Toan_8}, 92., p. 20]
	Cho $x > y > z$. Chứng minh biểu thức $A = x^4(y - z) + y^4(z - x) + z^4(x - y)$ luôn luôn dương.
\end{baitoan}

\begin{baitoan}[\cite{Tuyen_Toan_8}, 93., p. 20]
	Cho $x,y,z$ là 3 số thực dương thỏa $(x + y)(y + z)(z + x) = 8xyz$. Chứng minh $x = y = z$.
\end{baitoan}
Phân tích đa thức thành nhân tử:

\begin{baitoan}[\cite{Tuyen_Toan_8}, 94., p. 20]
	(a) $x^4 + 5x^3 + 10x - 4$. (b) $x^3 + y^3 + z^3 - 3xyz$
\end{baitoan}

\begin{baitoan}[\cite{Tuyen_Toan_8}, 95., p. 20]
	(a) $x^7 + x^2 + 1$. (b) $x^8 + x + 1$.
\end{baitoan}

\begin{baitoan}[\cite{Tuyen_Toan_8}, 96., p. 20]
	(a) $x^5 + x^4 + 1$. (b) $x^{10} + x^5 + 1$.
\end{baitoan}

\begin{baitoan}[\cite{Tuyen_Toan_8}, 97., p. 20]
	Cho $x\in\mathbb{Z}$. Chứng minh $x^{200} + x^{100} + 1\divby x^4 + x^2 + 1$.
\end{baitoan}
Phân tích đa thức thành nhân tử:

\begin{baitoan}[\cite{Tuyen_Toan_8}, 98., p. 20]
	(a) $A = x^2 - 2xy + y^2 + 3x - 3y - 4$. (b) $B = (12x^2 - 12xy + 3y^2) - 10(2x - y) + 8$.
\end{baitoan}

\begin{baitoan}[\cite{Tuyen_Toan_8}, 99., p. 20]
	(a) $A = (a - b)^3 + (b - c)^3 + (c - a)^3$. (b) $B = (a + b - 2c)^3 + (b + c - 2a)^3 + (c + a - 2b)^3$.
\end{baitoan}

\begin{baitoan}[\cite{Tuyen_Toan_8}, 100., p. 20]
	(a) Chứng minh: $(x + y + z)^3 - x^3 - y^3 - z^3 = 3(x + y)(y + z)(z + x)$. (b) Phân tích đa thức thành nhân tử: $A = (a + b + c)^3 + (a - b - c)^3 + (b - c - a)^3 + (c - a - b)^3$.
\end{baitoan}
Phân tích đa thức thành nhân tử:

\begin{baitoan}[\cite{Tuyen_Toan_8}, 101., p. 20]
	(a) $A = (x^2 - 2x)(x^2 - 2x - 1) - 6$. (b) $B = (x^2 + 4x - 3)^2 - 5x(x^2 + 4x - 3) + 6x^2$. (c) $C = (x^2 + x + 4) + 8x(x^2 + x + 4) + 15x^2$.
\end{baitoan}

\begin{baitoan}[\cite{Tuyen_Toan_8}, 102., p. 20]
	$2(x^2 - 6x + 1)^2 + 5(x^2 - 6x + 1)(x^2 + 1) + 2(x^2 + 1)^2$.
\end{baitoan}

\begin{baitoan}[\cite{Tuyen_Toan_8}, 103., p. 21]
	Cho $A = 4(x - 2)(x - 1)(x + 4)(x + 8) + 25x^2$. Chứng minh A không có giá trị âm.
\end{baitoan}

\begin{baitoan}[\cite{Tuyen_Toan_8}, 104., p. 21]
	Cho đa thức $A = 3x^4 + 11x^3 - 7x^2 - 2x - 1$. Phân tích A thành tích của 1 nhị thức bậc nhất với 1 đa thức bậc 3 có hệ số nguyên sao cho hệ số cao nhất của đa thức bậc 3 là $1$. 
\end{baitoan}

\begin{baitoan}[\cite{Tuyen_Toan_8}, 105., p. 21]
	Cho đa thức $A = x^4 - 6x^3 + 11x^2 - 6x + 1$. Phân tích A thành tích của 2 tam thức bậc 2 với hệ số nguyên.
\end{baitoan}

\begin{baitoan}[\cite{Tuyen_Toan_8}, 106., p. 21]
	Cho đa thức $A = x^4 - x^3 + 2x^2 - 11x - 5$. Phân tích A thành tích của 2 tam thức bậc 2 với hệ số nguyên \& các hệ số cao nhất đều mang dấu dương.
\end{baitoan}

%------------------------------------------------------------------------------%

\section{Monomial \& Polynomial Divisions -- Chia Đơn Thức \& Đa Thức}

\begin{baitoan}[\cite{Binh_Toan_8_tap_1}, VD22, p. 15]
	Tìm $n\in\mathbb{N}$ để đa thức $A = 3x^{n-1}y^6 - 5x^{n+1}y^4$ chia hết cho đơn thức $B = 2x^3y^n$.	
\end{baitoan}

\begin{baitoan}[\cite{Binh_Toan_8_tap_1}, VD23, p. 15]
	Tìm dư khi chia $x^7 + x^5 + x^3 + 1$ cho $x^2 - 1$.
\end{baitoan}

\begin{baitoan}[\cite{Binh_Toan_8_tap_1}, VD24, p. 16]
	Chứng minh $x^{8n} + x^{4n} + 1$ chia hết cho $x^{2n} + x^n + 1$, $\forall n\in\mathbb{N}$.
\end{baitoan}

\begin{baitoan}[\cite{Binh_Toan_8_tap_1}, VD25, p. 16]
	Chứng minh $x^{3m+1} + x^{3n+2} + 1\divby x^2 + x + 1$, $\forall m,n\in\mathbb{N}$.
\end{baitoan}

\begin{baitoan}[\cite{Binh_Toan_8_tap_1}, VD26, p. 15]
	Chứng minh $x^{6m+4} + x^{6n+2} + 1\divby x^2 - x + 1$, $\forall m,n\in\mathbb{N}$.
\end{baitoan}

\begin{baitoan}[\cite{Binh_Toan_8_tap_1}, VD27, p. 16]
	Cho $A(x) = (x^2 + x - 1)^{10} + (x^2 - x + 1)^{10} - 2$. Chứng minh $A(x)\divby x^2 - x$.
\end{baitoan}

\begin{baitoan}[\cite{Binh_Toan_8_tap_1}, 87., p. 17]
	Chứng minh giá trị biểu thức $A = \frac{2}{3}x^2y^3:\left(-\frac{1}{3}xy\right) + 2x(y - 1)(y + 1)$ không phụ thuộc vào giá trị của biến $y$, $\forall x,y\in\mathbb{R}^\star$.
\end{baitoan}

\begin{baitoan}[\cite{Binh_Toan_8_tap_1}, 88., p. 17]
	Tìm $n\in\mathbb{N}$ để đơn thức $A = 4x^{n+1}y^2$ chia hết cho đơn thức $B = 3x^3y^{n-1}$.
\end{baitoan}

\begin{baitoan}[\cite{Binh_Toan_8_tap_1}, 89., p. 17]
	Tính rồi tìm {\rm GTNN} của biểu thức $A = (9xy^2 - 6x^2y):-3xy + (6x^2y + 2x^4):2x^2$.
\end{baitoan}

\begin{baitoan}[\cite{Binh_Toan_8_tap_1}, 90., p. 17]
	Tìm $n\in\mathbb{N}$ để $A = 7x^{n-1}y^5 - 5x^3y^4$ chia hết cho $B = 5x^2y^n$.
\end{baitoan}

\begin{baitoan}[\cite{Binh_Toan_8_tap_1}, 91., p. 17]
	Rút gọn biểu thức $\left[(x^3 + y^3) - 2(x^2 - y^2) + 3(x + y)^2\right]:(x + y)$.
\end{baitoan}

\begin{baitoan}[\cite{Binh_Toan_8_tap_1}, 92., p. 17]
	chứng minh tồn tại $m,n\in\mathbb{R}$ để $x^4 + 1\divby x^2 + mx + n$.
\end{baitoan}

\begin{baitoan}[\cite{Binh_Toan_8_tap_1}, 93., p. 17]
	Tìm dư khi chia đa thức: (a) $x^{41}:(x^2 + 1)$. (b) $x^{43}:(x^2 + 1)$.
\end{baitoan}

\begin{baitoan}[\cite{Binh_Toan_8_tap_1}, 94., p. 17]
	Tìm dư khi chia $x^{99} + x^{55} + x^{11} + x + 7$ cho: (a) $x + 1$. (b) $x^2 + 1$.
\end{baitoan}

\begin{baitoan}[\cite{Binh_Toan_8_tap_1}, 95., p. 17]
	Khi chia $x^8$ cho $x + \frac{1}{2}$, được thương là $B(x)$ \& dư là số $r_1$. Khi chia $B(x)$ cho $x + \frac{1}{2}$, được thương là $C(x)$ \& dư là số $r_2$. Tính $r_2$.
\end{baitoan}

\begin{baitoan}[\cite{Binh_Toan_8_tap_1}, 96., p. 17]
	Chứng minh đa thức $x^{10} + x^2 + 1$ chia hết cho: (a) $x^2 + x + 1$. (b) $x^2 - x + 1$.
\end{baitoan}

\begin{baitoan}[\cite{Binh_Toan_8_tap_1}, 97., p. 17]
	Chứng minh: (a) $x^{50} + x^{10} + 1\divby x^{20} + x^{10} + 1$. (b) $x^2 - x^9 - x^{1945}\divby x^2 - x + 1$. (c) $x^{10} - 10x + 9\divby(x - 1)^2$. (d) $8x^9 - 9x^8 + 1\divby(x - 1)^2$.
\end{baitoan}

\begin{baitoan}[\cite{Binh_Toan_8_tap_1}, 98., p. 17]
	Chứng minh $A(x) = x^{99} + x^{88} + x^{77} + \cdots + x^{11} + 1$ chia hết cho $B(x) = x^9 + x^8 + x^7 + \cdots + x + 1$.
\end{baitoan}

\begin{baitoan}[\cite{Binh_Toan_8_tap_1}, 99., p. 18]
	Chứng minh đa thức $(x + y)^6 + (x - y)^6\divby x^2 + y^2$.
\end{baitoan}

\begin{baitoan}[\cite{Binh_Toan_8_tap_1}, 100., p. 18]
	Chứng minh $\forall n\in\mathbb{N}$: (a) $(x + 1)^{2n} - x^{2n} - 2x - 1\divby x(x + 1)(2x + 1)$. (b) $x^{4n+2} + 2x^{2n+1} + 1\divby(x + 1)^2$. (c) $(x + 1)^{4n+2} + (x - 1)^{4n+2}\divby x^2 + 1$.
\end{baitoan}

\begin{baitoan}[\cite{Binh_Toan_8_tap_1}, 101., p. 18]
	Chứng minh $(x^n - 1)(x^{n+1} - 1)\divby(x + 1)(x - 1)^2$, $\forall n\in\mathbb{N}$.
\end{baitoan}

\begin{baitoan}[\cite{Binh_Toan_8_tap_1}, 102., p. 18]
	Chứng minh $x^{6m+4} + x^{6n+2} + 1\divby x^4 + x^2 + 1$.
\end{baitoan}

\begin{baitoan}[\cite{Binh_Toan_8_tap_1}, 103., p. 18]
	Tìm $n\in\mathbb{N}$ sao cho $x^{2n} + x^n + 1\divby x^2 + x + 1$.
\end{baitoan}

\begin{baitoan}[\cite{Binh_Toan_8_tap_1}, 104., p. 18]
	Xác định hệ số $k\in\mathbb{R}$ để đa thức $A = x^3 + y^3 + z^3 + kxyz\divby x + y + z$.
\end{baitoan}

\begin{baitoan}[\cite{Binh_Toan_8_tap_1}, 105., p. 18]
	Cho đa thức $F(x)$ có các hệ số nguyên. Biết $F(0),F(1)$ là 2 số lẻ. Chứng minh đa thức $F(x)$ không có nghiệm nguyên.
\end{baitoan}

%------------------------------------------------------------------------------%

\section{Số Chính Phương}

\begin{baitoan}[\cite{Tuyen_Toan_8}, VD15, p. 22]
	Chứng minh $A = \underbrace{1\ldots1}_{2n} - 8\cdot\underbrace{1\ldots1}_n + 1$ là 1 số chính phương. 
\end{baitoan}

\begin{baitoan}[\cite{Tuyen_Toan_8}, VD16, p. 22]
	Chứng minh: (a) Tổng của 3 số chính phương liên tiếp không là 1 số chính phương. (b) Tổng $S = \sum_{i=1}^{30} i^2 = 1^2 + 2^2 + \ldots + 30^2$ không là 1 số chính phương.
\end{baitoan}

\begin{baitoan}[\cite{Tuyen_Toan_8}, 107., p. 23]
	Có 2 số chính phương nào mà: (a) Có tổng bằng $4567$? (b) Có hiệu bằng $7654$?
\end{baitoan}

\begin{baitoan}[\cite{Tuyen_Toan_8}, 108., p. 23]
	Chứng minh tổng của $20$ số chính phương liên tiếp không thể là số chính phương.
\end{baitoan}

\begin{baitoan}[\cite{Tuyen_Toan_8}, 109., p. 23]
	Cho 5 số chính phương bất kỳ có chữ số hàng đơn vị đều bằng $6$ còn chữ số hàng chục thì khác nhau. Chứng minh tổng các chữ số hàng chục của 5 số chính phương đó cũng là 1 số chính phương.
\end{baitoan}

\begin{baitoan}[\cite{Tuyen_Toan_8}, 110., p. 23]
	Cho $a,b,c\ne0$ là các chữ số. (a) Tính tổng $S$ của tất cả các số có 3 chữ số tạo thành bởi cả 3 chữ số $a,b,c$. (b) Chứng minh $S$ không phải là số chính phương.
\end{baitoan}

\begin{baitoan}[\cite{Tuyen_Toan_8}, 111., p. 23]
	Tìm 1 số chính phương có 4 chữ số biết 2 chữ số đầu giống nhau, 2 chữ số cuối giống nhau.
\end{baitoan}

\begin{baitoan}[\cite{Tuyen_Toan_8}, 112., p. 23]
	Chứng minh nếu $n + 1,2n + 1$ đều là số chính phương thì $n\divby24$.
\end{baitoan}

\begin{baitoan}[\cite{Tuyen_Toan_8}, 113., p. 23]
	Tìm $n\in\mathbb{N}$ biết trong 3 mệnh đề sau có 2 mệnh đề đúng \& 1 mệnh đề sai: (a) $n$ có chữ số tận cùng là $2$. (b) $n + 20$ là 1 số chính phương. (c) $n - 69$ là 1 số chính phương.
\end{baitoan}

\begin{baitoan}[\cite{Tuyen_Toan_8}, 114., p. 23]
	Cho $a$ là tổng của 2 số chính phương. Chứng minh: (a) $2a$ cũng là tổng của 2 số chính phương. (b) $a^2$ cũng là tổng của 2 số chính phương.
\end{baitoan}

\begin{baitoan}[\cite{Tuyen_Toan_8}, 115., p. 23]
	Cho $a,b,c,d$ là 4 số chính phương. Chứng minh $(a + b)(c + d)$ là tổng của 2 số chính phương.
\end{baitoan}

\begin{baitoan}[\cite{Tuyen_Toan_8}, 116., p. 23]
	Cho $x,y,z\in\mathbb{Z}$ sao cho $x = y + z$. Chứng minh $2(xy + xz - yz)$ là tổng của 3 số chính phương.
\end{baitoan}

\begin{baitoan}[\cite{Tuyen_Toan_8}, 117., p. 23]
	Cho $a,b,c,d\in\mathbb{Z}$ thỏa $a - b = c + d$. Chứng minh $a^2 + b^2 + c^2 + d^2$ luôn là tổng của 3 số chính phương.
\end{baitoan}

\begin{baitoan}[\cite{Tuyen_Toan_8}, 118., p. 23]
	Cho 2 số chính phương liên tiếp. Chứng minh tổng của 2 số đó cộng với tích của chúng là 1 số chính phương lẻ.
\end{baitoan}

\begin{baitoan}[\cite{Tuyen_Toan_8}, 119., p. 24]
	Cho $a_n = \sum_{i=1}^n i  = 1 + 2 + \cdots + n$. (a) Tính $a_{n+1}$. (b) Chứng minh $a_n + a_{n+1}$ là 1 số chính phương.
\end{baitoan}

\begin{baitoan}[\cite{Tuyen_Toan_8}, 120., p. 24]
	Cho $a$ là tích của 4 số nguyên liên tiếp. Chứng minh $a + 1$ là 1 số chính phương.
\end{baitoan}

\begin{baitoan}[\cite{Tuyen_Toan_8}, 121., p. 24]
	(a) Cho $a = \underbrace{1\ldots1}_n5,b = \underbrace{1\ldots1}_n9$. Chứng minh $ab + 4$ là 1 số chính phương. (b) Cho $a = \underbrace{1\ldots1}_n,b = 1\underbrace{0\ldots0}_{n-2}11$, $n\in\mathbb{N}$, $n\ge2$. Chứng minh $ab + 4$ là 1 số chính phương.
\end{baitoan}

\begin{baitoan}[\cite{Tuyen_Toan_8}, 122., p. 24]
	Cho $A = \underbrace{1\ldots1}_n\underbrace{5\ldots5}_n + 1$. Chứng minh $A$ là 1 số chính phương.
\end{baitoan}

\begin{baitoan}[\cite{Tuyen_Toan_8}, 123., p. 24]
	Chứng minh: (a) $A = \underbrace{1\ldots1}_{2n} + \underbrace{4\ldots4}_n + 1$, $n\in\mathbb{N}$ là số chính phương. (b) $B = \underbrace{1\ldots1}_{2n} + \underbrace{1\ldots1}_{n + 1} + \underbrace{6\ldots6}_n + 8$, $n\in\mathbb{N}$ là số chính phương.
\end{baitoan}

\begin{baitoan}[\cite{Tuyen_Toan_8}, 124., p. 24]
	Cho $a,b,c\in\mathbb{Z}$ thỏa $ab + bc + ca = 1$. Chứng minh $(a^2 + 1)(b^2 + 1)(c^2 + 1)$ là 1 số chính phương.
\end{baitoan}

\begin{baitoan}[\cite{Tuyen_Toan_8}, 125., p. 24]
	Tìm tất cả $n\in\mathbb{N}$ sao cho $n^2 + 1234$ là 1 số chính phương.
\end{baitoan}

\begin{baitoan}[\cite{Tuyen_Toan_8}, 126., p. 24]
	Tìm tất cả $n\in\mathbb{N}$ sao cho $2^n + 2^4 + 2^7$ là 1 số chính phương.
\end{baitoan}

\begin{baitoan}[\cite{Tuyen_Toan_8}, 127., p. 24]
	Tìm tất cả $n\in\mathbb{N}$ sao cho $n^2 + 2x + 200$ là 1 số chính phương.
\end{baitoan}

\begin{baitoan}[\cite{Tuyen_Toan_8}, 128., p. 24]
	Cho $A = p^4$ với $p$ là 1 số nguyên tố. (a) Số $A$ có các ước dương nào? (b) Tìm các giá trị của $p$ để tổng các ước dương của $A$ là 1 số chính phương.
\end{baitoan}

\begin{baitoan}[\cite{Tuyen_Toan_8}, 129., p. 24]
	Cho $a,b\in\mathbb{N}^\star$ thỏa $ab + 1$ là 1 số chính phương. Chứng minh tồn tại $c\in\mathbb{N}^\star$ sao cho $ac + 1,bc + 1$ đều là số chính phương.
\end{baitoan}

%------------------------------------------------------------------------------%

\section{Miscellaneous}

\begin{baitoan}[\cite{Tuyen_Toan_8}, VD17, p. 25]
	Phân tích đa thức thành nhân tử: $A = (x - 3)(x - 1)(x + 1)(x + 3) + 15$.
\end{baitoan}

\begin{baitoan}[\cite{Tuyen_Toan_8}, VD19, p. 26]
	Chứng minh $A = n^3 + (n + 1)^3 + (n + 2)^3\divby9$, $\forall n\in\mathbb{N}^\star$.
\end{baitoan}

\begin{baitoan}[\cite{Tuyen_Toan_8}, 130., p. 27]
	Cho $a,b,x,y\in\mathbb{R}$ sao cho $ab = 1,ax + by = 2$. Chứng minh $xy\le1$.
\end{baitoan}

\begin{baitoan}[\cite{Tuyen_Toan_8}, 131., p. 27]
	Chứng minh $A = x(x - y)(x + y)(x + 2y) + y^4$ là bình phương của 1 đa thức.
\end{baitoan}

\begin{baitoan}[\cite{Tuyen_Toan_8}, 132., p. 27]
	Cho $x,y\in\mathbb{R}$ thỏa $x + y + z = 0$. Chứng minh: $(x^2 + y^2 + z^2)^2 = 2(x^4 + y^4 + z^4)$.
\end{baitoan}

\begin{baitoan}[\cite{Tuyen_Toan_8}, 133., p. 27]
	Cho $x,y\in\mathbb{R}^\star$ thỏa $(x + y)^5 = x^5 + y^5$. Chứng minh $x,y$ là 2 số đối nhau.
\end{baitoan}

\begin{baitoan}[\cite{Tuyen_Toan_8}, 134., p. 27]
	Cho $n\in\mathbb{N}$, $n > 1$. Chứng minh: (a) $20^n - 1$ là hợp số. (b) $1000^n + 1$ là hợp số.
\end{baitoan}

\begin{baitoan}[\cite{Tuyen_Toan_8}, 135., p. 27]
	Phân tích đa thức thành nhân tử: (a) $A = x^9 - x^7 + x^6 - x^5 - x^4 + x^3 - x^2 + 1$. (b) $B = x^7 + x^5 + 1$.
\end{baitoan}

\begin{baitoan}[\cite{Tuyen_Toan_8}, 136., p. 27]
	Cho $x^2 - y^2 = 1$. Tính giá trị biểu thức $A = 2(x^6 - y^6) - 3(x^4 + y^4)$.
\end{baitoan}

\begin{baitoan}[\cite{Tuyen_Toan_8}, 137., p. 27]
	Tìm số dư của phép chia $S:5$ với $S = \sum_{i=1}^8 i^n = 1^n + 2^n + \cdots + 8^n$ với $n\in\mathbb{N}$, $n$ lẻ.
\end{baitoan}

\begin{baitoan}[\cite{Tuyen_Toan_8}, 138., p. 27]
	Cho $a$ là 1 số chính phương, $m\in\mathbb{N}$ tùy ý. Chứng minh bao giờ cũng có $n\in\mathbb{N}$ sao cho $a + mn$ là 1 số chính phương.
\end{baitoan}

\begin{baitoan}[\cite{Tuyen_Toan_8}, 139., p. 27]
	Cho $x,y\in\mathbb{Z}$. Chứng minh: (a) $A = (x + 1)(x + 3)(x + 4)(x + 6) + 9$ là 1 số chính phương. (b) $B = (x - y)(x - 2y)(x - 3y)(x - 4y) + y^4$ là 1 số chính phương.
\end{baitoan}

\begin{baitoan}[\cite{Tuyen_Toan_8}, 140., p. 27]
	Chứng minh $\forall n\in\mathbb{N}^\star$: (a) $1 + 3 + 5 + \cdots + (2n - 1) = n^2$. (b) $1^3 + 2^3 + \cdots + n^3 = (1 + 2 + \cdots + n)^2 = \frac{1}{4}n^2(n + 1)^2$.
\end{baitoan}

%------------------------------------------------------------------------------%

\printbibliography[heading=bibintoc]
	
\end{document}