\documentclass{article}
\usepackage[backend=biber,natbib=true,style=alphabetic,maxbibnames=50]{biblatex}
\addbibresource{/home/nqbh/reference/bib.bib}
\usepackage[utf8]{vietnam}
\usepackage{tocloft}
\renewcommand{\cftsecleader}{\cftdotfill{\cftdotsep}}
\usepackage[colorlinks=true,linkcolor=blue,urlcolor=red,citecolor=magenta]{hyperref}
\usepackage{amsmath,amssymb,amsthm,float,graphicx,mathtools,tikz}
\usetikzlibrary{angles,calc,intersections,matrix,patterns,quotes,shadings}
\allowdisplaybreaks
\newtheorem{assumption}{Assumption}
\newtheorem{baitoan}{}
\newtheorem{cauhoi}{Câu hỏi}
\newtheorem{conjecture}{Conjecture}
\newtheorem{corollary}{Corollary}
\newtheorem{dangtoan}{Dạng toán}
\newtheorem{definition}{Definition}
\newtheorem{dinhly}{Định lý}
\newtheorem{dinhnghia}{Định nghĩa}
\newtheorem{example}{Example}
\newtheorem{ghichu}{Ghi chú}
\newtheorem{hequa}{Hệ quả}
\newtheorem{hypothesis}{Hypothesis}
\newtheorem{lemma}{Lemma}
\newtheorem{luuy}{Lưu ý}
\newtheorem{nhanxet}{Nhận xét}
\newtheorem{notation}{Notation}
\newtheorem{note}{Note}
\newtheorem{principle}{Principle}
\newtheorem{problem}{Problem}
\newtheorem{proposition}{Proposition}
\newtheorem{question}{Question}
\newtheorem{remark}{Remark}
\newtheorem{theorem}{Theorem}
\newtheorem{vidu}{Ví dụ}
\usepackage[left=1cm,right=1cm,top=5mm,bottom=5mm,footskip=4mm]{geometry}
\def\labelitemii{$\circ$}
\DeclareRobustCommand{\divby}{%
	\mathrel{\vbox{\baselineskip.65ex\lineskiplimit0pt\hbox{.}\hbox{.}\hbox{.}}}%
}

\title{Problem: Congruent Triangles -- Bài Tập: Tam Giác Bằng Nhau}
\author{Nguyễn Quản Bá Hồng\footnote{A Scientist {\it\&} Creative Artist Wannabe. E-mail: {\tt nguyenquanbahong@gmail.com}. Bến Tre City, Việt Nam.}}
\date{\today}

\begin{document}
\maketitle
\begin{abstract}
	This text is a part of the series {\it Some Topics in Elementary STEM \& Beyond}:
	
	{\sc url}: \url{https://nqbh.github.io/elementary_STEM}.
	
	Latest version:
	\begin{itemize}
		\item {\it Problem: Congruent Triangles -- Bài Tập: Tam Giác Bằng Nhau}.
		
		PDF: {\sc url}: \url{https://github.com/NQBH/elementary_STEM_beyond/blob/main/elementary_mathematics/grade_7/congruent_triangle/problem/NQBH_congruent_triangle_problem.pdf}.
		
		\TeX: {\sc url}: \url{https://github.com/NQBH/elementary_STEM_beyond/blob/main/elementary_mathematics/grade_7/congruent_triangle/problem/NQBH_congruent_triangle_problem.tex}.
		\item {\it Problem \& Solution: Congruent Triangles -- Bài Tập \& Lời Giải: Tam Giác Bằng Nhau}.
		
		PDF: {\sc url}: \url{https://github.com/NQBH/elementary_STEM_beyond/blob/main/elementary_mathematics/grade_7/congruent_triangle/problem/NQBH_congruent_triangle_solution.pdf}.
		
		\TeX: {\sc url}: \url{https://github.com/NQBH/elementary_STEM_beyond/blob/main/elementary_mathematics/grade_7/congruent_triangle/problem/NQBH_congruent_triangle_solution.tex}.
	\end{itemize}
\end{abstract}
\tableofcontents

%------------------------------------------------------------------------------%

\section{Congruent Triangles -- Bài Tập: Tam Giác Bằng Nhau}

\begin{baitoan}[\cite{Hung_Mai_Toan_7_hinh_hoc}, 3.1., p. 26]
	Cho 2 điểm $A,B$ chạy trên $Ox,Oy$ sao cho $OA + OB = m$. Chứng minh đường trung trực của đoạn thẳng AB luôn đi qua 1 điểm cố định.
\end{baitoan}

\begin{baitoan}[\cite{Hung_Mai_Toan_7_hinh_hoc}, 3.2., p. 27]
	Cho $\Delta ABC$ nhọn có điểm M là trung điểm AC. Lấy điểm K thuộc đoạn BM sao cho $AK = BC$. AK giao BC tại L. Chứng minh $LK = BL$.
\end{baitoan}

\begin{baitoan}[\cite{Hung_Mai_Toan_7_hinh_hoc}, 3.3., p. 27]
	Cho $\Delta ABC$ có $AB = AC$, $\widehat{A} = 40^\circ$. Điểm K thuộc cạnh AC sao cho $\widehat{KBC} = 30^\circ$. Điểm L nằm trong $\Delta ABC$ sao cho $\widehat{ABL} = 30^\circ$, AL là phân giác $\widehat{BAC}$. Chứng minh $AK = AL$.
\end{baitoan}

\begin{baitoan}[\cite{Hung_Mai_Toan_7_hinh_hoc}, 3.4., p. 27]
	Cho $\Delta ABC$ có $\widehat{A} = 60^\circ$, 2 điểm $E,F$ thuộc tia $BA,CA$ sao cho $BE = CF = BC$. I là tâm đường tròn nội tiếp $\Delta ABC$. Chứng minh $E,F,I$ thẳng hàng.
\end{baitoan}

\begin{baitoan}[\cite{Hung_Mai_Toan_7_hinh_hoc}, 3.5., p. 28]
	Cho $\Delta ABC$ có đường cao AH. Biết $\widehat{ABC} = 75^\circ$, $AH = \frac{1}{2}BC$. Chứng minh $\Delta ABC$ cân.
\end{baitoan}

\begin{baitoan}[\cite{Hung_Mai_Toan_7_hinh_hoc}, 3.6., p. 28]
	Cho $\Delta ABC$ có trực tâm H, M là trung điểm BC. Đường thẳng qua H vuông góc HM cắt $AB,AC$ lần lượt ở $P,Q$. Chứng minh $HP = HQ$.
\end{baitoan}

\begin{baitoan}[\cite{Hung_Mai_Toan_7_hinh_hoc}, 3.7., p. 29]
	Cho $\Delta ABC$ với điểm N nằm trong $\Delta ABC$ sao cho $\widehat{ABN} = \widehat{ACN}$. M là trung điểm BC. $NH,NK$ là đường vuông góc hạ từ N xuống $AB,AC$. Chứng minh $\Delta MHK$ cân.
\end{baitoan}

\begin{baitoan}[\cite{Hung_Mai_Toan_7_hinh_hoc}, 3.8., p. 29]
	Cho $\Delta ABC$ cân tại A, đường phân giác BE. $F\in BC$ sao cho $\widehat{BEF} = 90^\circ$. Chứng minh $BF = 2CE$.
\end{baitoan}

\begin{baitoan}[\cite{Hung_Mai_Toan_7_hinh_hoc}, 3.9., p. 30]
	Cho $\Delta ABC$ cân tại A, điểm M nằm trong $\Delta ABC$ sao cho $\widehat{AMB} = \widehat{AMC}$. Chứng minh AM là phân giác $\widehat{A}$.
\end{baitoan}

\begin{baitoan}[\cite{Hung_Mai_Toan_7_hinh_hoc}, 3.10., p. 30]
	Cho $\Delta ABC$ là trung điểm BC. Dựng 2 tam giác vuông cân $AEB,AFC$ bên ngoài $\Delta ABC$. Chứng minh $\Delta MEF$ vuông cân.
\end{baitoan}

\begin{baitoan}[\cite{Hung_Mai_Toan_7_hinh_hoc}, 3.11., p. 31]
	Cho $\Delta ABC$ vuông tại A, M là trung điểm AB, H là hình chiếu vuông góc hạ từ M xuống BC. Điểm K thuộc đoạn AM sao cho $AK = BH$. Chứng minh $\Delta CHK$ cân.
\end{baitoan}

\begin{baitoan}[\cite{Hung_Mai_Toan_7_hinh_hoc}, 3.12., p. 31]
	Cho $\Delta ABC$ vuông cân tại A. Vẽ $\Delta BCK$ cân tại C sao cho $C,K$ nằm khác phía đối với AB, $\widehat{BCK} = 30^\circ$. Tính $\widehat{BAK}$.
\end{baitoan}

\begin{baitoan}[\cite{Hung_Mai_Toan_7_hinh_hoc}, 3.13., p. 32]
	Cho $\Delta ABC$. Lấy $M\in AC,N\in AB$ sao cho $\widehat{MBC} = 2\alpha = 2\widehat{ABM},\widehat{BCN} = 2\beta = 2\widehat{ACN}$. P là giao điểm của $BM,CN$. Biết $PM = PN$. Chứng minh $\Delta ABC$ vuông hoặc cân.
\end{baitoan}

\begin{baitoan}[\cite{Hung_Mai_Toan_7_hinh_hoc}, 3.14., p. 32]
	Cho $\Delta ABC$, M là trung điểm BC. Dựng 2 tam giác vuông cân $ABE,ACF$ bên ngoài $\Delta ABC$. Chứng minh $AM\bot EF$.
\end{baitoan}

\begin{baitoan}[\cite{Hung_Mai_Toan_7_hinh_hoc}, 3.15., p. 33]
	Cho $\Delta ABC$ có đường cao AH, $M,N$ là chân đường vuông góc hạ từ H xuống $AB,AC$. Biết $MB = NC$. Chứng minh $\Delta ABC$ cân.
\end{baitoan}

\begin{baitoan}[\cite{Hung_Mai_Toan_7_hinh_hoc}, 3.16., p. 33]
	Cho $\widehat{xOy}$. $A,B$ chạy trên $Ox,Oy$ sao cho $OA - OB = m$. Chứng minh trung trực AB đi qua 1 điểm cố định.
\end{baitoan}

\begin{baitoan}[\cite{Hung_Mai_Toan_7_hinh_hoc}, 3.17., p. 33]
	Cho $\Delta ABC$ vuông tại A, đường cao AH. E thuộc tia AH, K thuộc tia đối của tia HA sao cho $AE = HK$. Kẻ đường thẳng qua E song song BC cắt AC tại F. Chứng minh $\widehat{BKF} = 90^\circ$.
\end{baitoan}

\begin{baitoan}[\cite{Hung_Mai_Toan_7_hinh_hoc}, 3.18., p. 33]
	Cho $\Delta ABC$ có đường phân giác $AA'$. Lấy 2 điểm $M,N$ nằm trong $\Delta ABC$ sao cho $AA'$ là trung trực của MN. Lấy $C',B'$ là 2 điểm đối xứng với M qua $AB,AC$. Chứng minh AN là trung trực của $B'C'$.
\end{baitoan}

\begin{baitoan}[\cite{Hung_Mai_Toan_7_hinh_hoc}, 3.19., p. 33]
	Cho $\Delta ABC$ vuông tại A, đường cao AH.  $I,J$ là tâm đường tròn nội tiếp $\Delta ABH,\Delta ACH$. IJ cắt $AB,AC$ lần lượt ở $E,F$. Chứng minh A là tâm đường tròn ngoại tiếp $\Delta EFH$.
\end{baitoan}

\begin{baitoan}[\cite{Hung_Mai_Toan_7_hinh_hoc}, 3.20., p. 34]
	Cho $\Delta ABC$, dựng $\Delta ABZ,\Delta ACY$ đều bên ngoài $\Delta ABC$. Vẽ $\Delta BCX$ cân tại X bên ngoài $\Delta ABC$ sao cho $\widehat{BXC} = 120^\circ$. Chứng minh $AX\bot YZ$.
\end{baitoan}

\begin{baitoan}[\cite{Hung_Mai_Toan_7_hinh_hoc}, 3.21., p. 34]
	Cho $\Delta ABC$, I là tâm đường tròn nội tiếp. $BE,CF$ là 2 đường phân giác trong. Biết $IE = IF$. Chứng minh $\widehat{BAC} = 60^\circ$ hoặc $\Delta ABC$ cân.
\end{baitoan}

\begin{baitoan}[\cite{Hung_Mai_Toan_7_hinh_hoc}, 3.22., p. 34]
	Cho $\Delta ABC$, I là tâm đường tròn nội tiếp. $AD,BE,CF$ là 3 đường phân giác. Biết $ID = IE = IF$. Chứng minh $\Delta ABC$ đều.
\end{baitoan}

\begin{baitoan}[\cite{Hung_Mai_Toan_7_hinh_hoc}, 3.23., p. 34]
	Cho $\Delta ABC$, $\widehat{A} = 60^\circ$. Đường phân giác $BE,CF$. Chứng minh $BF + CE = BC$.
\end{baitoan}

\begin{baitoan}[\cite{Hung_Mai_Toan_7_hinh_hoc}, 3.24., p. 34]
	Cho $\Delta ABC$, đường phân giác AD. Lấy $E,F$ thuộc cạnh $AB,AC$ sao cho $\Delta BDE$ cân tại B, $\Delta CDF$ cân tại C. Chứng minh $EF\parallel BC$.
\end{baitoan}

\begin{baitoan}[\cite{Hung_Mai_Toan_7_hinh_hoc}, 3.25., p. 34]
	Cho $\Delta ABC$, $\widehat{ABC} = 70^\circ,\widehat{ACB} = 50^\circ$. Lấy điểm D nằm khác phía A đối với BC sao cho $\widehat{CBD} = 40^\circ,\widehat{BCD} = 20^\circ$. Chứng minh $AD\bot BC$.
\end{baitoan}

\begin{baitoan}[\cite{Hung_Mai_Toan_7_hinh_hoc}, 3.26., p. 34]
	Cho $\Delta ABC$. Kẻ đường cao $BE,CF$. $X,Y,Z$ lần lượt là trung điểm $EF,BF,CE$. K là giao điểm của đường thẳng qua Y vuông góc BX, đường thẳng qua Z vuông góc CX. Chứng minh K thuộc trung trực BC.
\end{baitoan}

\begin{baitoan}[\cite{Hung_Mai_Toan_7_hinh_hoc}, 3.27., p. 34]
	Cho $\Delta ABC$, 3 đường cao $AD,BE,CF$ cắt nhau tại H. $X,Y,Z,T$ là chân đường vuông góc hạ từ D xuống $AB,BE,CF,AC$. Chứng minh $X,Y,Z,T$ thẳng hàng.
\end{baitoan}

%------------------------------------------------------------------------------%

\section{Pythagore Theorem -- Định Lý Pythagore}

\begin{baitoan}[\cite{Hung_Mai_Toan_7_hinh_hoc}, 4.1., p. 40]
	Cho $\Delta ABC$ vuông tại A, phân giác BD, kẻ $DE\bot BC$, $E\in BC$. F là giao điểm của $AB,DE$. Chứng minh: (a) BD là trung trực AE. (b) $\Delta ACF$ cân. (c) $AD < CD$. (d) $AE\parallel CF$.
\end{baitoan}

\begin{baitoan}[\cite{Hung_Mai_Toan_7_hinh_hoc}, 4.2., p. 40]
	Cho $\Delta ABC$ vuông tại A, phân giác BD. Trên tia BC lấy điểm E sao cho $AB = BE$. (a) Chứng minh $BE\bot DE$. (b) Chứng minh BD là đường trung trực của AE. (c) Kẻ $AH\bot BC$. So sánh $CE,EH$.
\end{baitoan}

\begin{baitoan}[\cite{Hung_Mai_Toan_7_hinh_hoc}, 4.3., p. 41]
	Cho $\Delta ABC$ vuông tại A, $AB = 8$ {\rm cm}, $AC = 6$ {\rm cm}. (a) Tính BC. (b) Trên cạnh AC lấy điểm E sao cho $AE = 2$ {\rm cm}, trên tia đối của tia AB lấy điểm D sao cho $AB = AD$. Chứng minh $\Delta BEC = \Delta DEC$. (c) Chứng minh DE đi qua trung điểm BC.
\end{baitoan}

\begin{baitoan}[\cite{Hung_Mai_Toan_7_hinh_hoc}, 4.4., p. 41]
	Cho $\Delta ABC$ vuông tại A, $\widehat{B} = 60^\circ$. Vẽ $AH\bot BC$, $H\in BC$. (a) So sánh $AB,AC$; $BH,CH$. (b) Lấy điểm D thuộc tia đối của tia HA sao cho $AH = DH$. Chứng minh $\Delta ACH = \Delta DCH$. (c) Tính $\widehat{BDC}$.
\end{baitoan}

\begin{baitoan}[\cite{Hung_Mai_Toan_7_hinh_hoc}, 4.5., p. 41]
	Cho $\Delta ABC$ vuông tại A, đường cao AH, trên đó lấy điểm D. Tren tia đối của tia HA lấy E sao cho $AD = EH$. Đường vuông góc với AH tại D cắt AC tại F. Chứng minh $BE\bot EF$.
\end{baitoan}

\begin{baitoan}[\cite{Hung_Mai_Toan_7_hinh_hoc}, 4.6., p. 41]
	Từ 1 điểm O tùy ý trong $\Delta ABC$, kẻ $OA_1,OB_1,OC_1$ lần lượt vuông góc với 3 cạnh $BC,CA,AB$. Chứng minh: $AB_1^2 + BC_1^2 + CA_1^2 = AC_1^2 + BA_1^2 + CB_1^2$.
\end{baitoan}

\begin{baitoan}[\cite{Hung_Mai_Toan_7_hinh_hoc}, 4.7., p. 41]
	Cho $\Delta ABC$ cân tại A, $\widehat{A} = 30^\circ,BC = 2$ {\rm cm}. Trên cạnh AC lấy điểm D sao cho $\widehat{CBD} = 60^\circ$. Chứng minh $AD = \sqrt{2}$.
\end{baitoan}

%------------------------------------------------------------------------------%

\section{Quan Hệ Giữa Các Yếu Tố Trong Tam Giác. Bất Đẳng Thức Tam Giác}

\begin{baitoan}[\cite{Hung_Mai_Toan_7_hinh_hoc}, 5.1., p. 42]
	Cho $\Delta ABC$. Chứng minh: (a) Đối diện với cạnh lớn nhất là góc $> 60^\circ$. (b) Đối diện với cạnh nhỏ nhất là góc $< 60^\circ$.
\end{baitoan}

\begin{baitoan}[\cite{Hung_Mai_Toan_7_hinh_hoc}, 5.2., p. 43]
	Cho $\Delta ABC$ vuông tại A. Chứng minh: (a) $\widehat{C} > 30^\circ\Leftrightarrow AB > \frac{1}{2}BC$. (b) $\widehat{C} < 30^\circ\Leftrightarrow AB < \frac{1}{2}BC$. (c) $\widehat{C} = 30^\circ\Leftrightarrow AB = \frac{1}{2}BC$.
\end{baitoan}

\begin{baitoan}[\cite{Hung_Mai_Toan_7_hinh_hoc}, 5.3., p. 44]
	Cho $\Delta ABC$, $\widehat{A}\le60^\circ$. Chứng minh: $BC^2\le AB^2 + AC^2 - AB\cdot AC$.
\end{baitoan}

\begin{baitoan}[\cite{Hung_Mai_Toan_7_hinh_hoc}, 5.4., p. 44]
	Cho $\Delta ABC$, $\widehat{A}\ge60^\circ$. Chứng minh: $BC^2\ge AB^2 + AC^2 - AB\cdot AC$.
\end{baitoan}

\begin{baitoan}[\cite{Hung_Mai_Toan_7_hinh_hoc}, 5.5., p. 46]
	Cho $\Delta ABC$, $\widehat{A}\le60^\circ$. Chứng minh $2BC\ge AB + AC$.
\end{baitoan}

\begin{baitoan}[\cite{Hung_Mai_Toan_7_hinh_hoc}, 5.6., p. 46]
	Cho $\Delta ABC$, $\widehat{A}\ge120^\circ$. Chứng minh: $BC^2\ge AB^2 + AC^2 + AB\cdot AC$.
\end{baitoan}

\begin{baitoan}[\cite{Hung_Mai_Toan_7_hinh_hoc}, 5.7., p. 47]
	Cho $\Delta ABC$ đều, M nằm trong $\Delta ABC$ sao cho $\widehat{BMC}\ge120^\circ$. Chứng minh $2MA\ge MB + MC$.
\end{baitoan}

\begin{baitoan}[\cite{Hung_Mai_Toan_7_hinh_hoc}, 5.8., p. 47]
	Cho $\Delta ABC$ đều, M nằm trong $\Delta ABC$ sao cho $\widehat{BMC}\le120^\circ$. Chứng minh $MA^2\le MB^2 + MC^2 - MB\cdot MC$.
\end{baitoan}

\begin{baitoan}[\cite{Hung_Mai_Toan_7_hinh_hoc}, 5.9., p. 48]
	Cho $\Delta ABC,\Delta A'B'C'$ có $AB = A'B',AC = A'C'$. Chứng minh $\widehat{A}\ge\widehat{A'}\Leftrightarrow BC\ge B'C'$.
\end{baitoan}

\begin{baitoan}[\cite{Hung_Mai_Toan_7_hinh_hoc}, 5.10., p. 49]
	Cho $\Delta ABC$, trung tuyến AM, $P\in AM$. Chứng minh $AB > AC\Leftrightarrow PB > PC$.
\end{baitoan}

\begin{baitoan}[\cite{Hung_Mai_Toan_7_hinh_hoc}, 5.11., p. 49]
	Cho $\Delta ABC$, M là trung điểm BC. D nằm giữa $B,M$. Lấy F sao cho M là trung điểm DE. Chứng minh $AB < AC\Leftrightarrow AD < AE$.
\end{baitoan}

\begin{baitoan}[\cite{Hung_Mai_Toan_7_hinh_hoc}, 5.12., p. 50]
	Cho $\Delta ABC$, trung tuyến AM. Chứng minh $AB > AC\Leftrightarrow\widehat{BAM} > \widehat{CAM}$.
\end{baitoan}

\begin{baitoan}[\cite{Hung_Mai_Toan_7_hinh_hoc}, 5.13., p. 51]
	Cho $\Delta ABC$, trung tuyến AM, P thuộc đoạn AM. Chứng minh $\widehat{BPM} > \widehat{CPM}\Leftrightarrow \widehat{BAM} > \widehat{CAM}$.
\end{baitoan}

\begin{baitoan}[\cite{Hung_Mai_Toan_7_hinh_hoc}, 5.14., p. 51]
	Cho $\Delta ABC$, trung tuyến AM, $D\in BM$. Điểm E sao cho M là trung điểm DE. Chứng minh $\widehat{BAM} > \widehat{CAM}\Leftrightarrow\widehat{DAM} > \widehat{EAM}$.
\end{baitoan}

\begin{baitoan}[\cite{Hung_Mai_Toan_7_hinh_hoc}, 5.15., p. 51]
	Cho $\Delta ABC,\Delta A'B'C'$ vuông tại $A,A'$. Biết $AB = A'B'$. Chứng minh $BC > B'C'\Leftrightarrow AC > A'C'$.
\end{baitoan}

%\begin{baitoan}[\cite{Hung_Mai_Toan_7_hinh_hoc}, 5.16., p. 52]
%	
%\end{baitoan}
%
%\begin{baitoan}[\cite{Hung_Mai_Toan_7_hinh_hoc}, 5.1., p. 5]
%	
%\end{baitoan}
%
%\begin{baitoan}[\cite{Hung_Mai_Toan_7_hinh_hoc}, 5.1., p. 5]
%	
%\end{baitoan}

%------------------------------------------------------------------------------%

\section{Miscellaneous}

%------------------------------------------------------------------------------%

\printbibliography[heading=bibintoc]
	
\end{document}