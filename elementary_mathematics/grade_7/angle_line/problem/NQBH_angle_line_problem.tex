\documentclass{article}
\usepackage[backend=biber,natbib=true,style=alphabetic,maxbibnames=50]{biblatex}
\addbibresource{/home/nqbh/reference/bib.bib}
\usepackage[utf8]{vietnam}
\usepackage{tocloft}
\renewcommand{\cftsecleader}{\cftdotfill{\cftdotsep}}
\usepackage[colorlinks=true,linkcolor=blue,urlcolor=red,citecolor=magenta]{hyperref}
\usepackage{amsmath,amssymb,amsthm,float,graphicx,mathtools,tikz}
\usetikzlibrary{angles,calc,intersections,matrix,patterns,quotes,shadings}
\allowdisplaybreaks
\newtheorem{assumption}{Assumption}
\newtheorem{baitoan}{}
\newtheorem{cauhoi}{Câu hỏi}
\newtheorem{conjecture}{Conjecture}
\newtheorem{corollary}{Corollary}
\newtheorem{dangtoan}{Dạng toán}
\newtheorem{definition}{Definition}
\newtheorem{dinhly}{Định lý}
\newtheorem{dinhnghia}{Định nghĩa}
\newtheorem{example}{Example}
\newtheorem{ghichu}{Ghi chú}
\newtheorem{hequa}{Hệ quả}
\newtheorem{hypothesis}{Hypothesis}
\newtheorem{lemma}{Lemma}
\newtheorem{luuy}{Lưu ý}
\newtheorem{nhanxet}{Nhận xét}
\newtheorem{notation}{Notation}
\newtheorem{note}{Note}
\newtheorem{principle}{Principle}
\newtheorem{problem}{Problem}
\newtheorem{proposition}{Proposition}
\newtheorem{question}{Question}
\newtheorem{remark}{Remark}
\newtheorem{theorem}{Theorem}
\newtheorem{vidu}{Ví dụ}
\usepackage[left=1cm,right=1cm,top=5mm,bottom=5mm,footskip=4mm]{geometry}
\def\labelitemii{$\circ$}
\DeclareRobustCommand{\divby}{%
	\mathrel{\vbox{\baselineskip.65ex\lineskiplimit0pt\hbox{.}\hbox{.}\hbox{.}}}%
}

\title{Problem: Angle {\it\&} Line -- Bài Tập: Góc {\it\&} Đường Thẳng}
\author{Nguyễn Quản Bá Hồng\footnote{Independent Researcher, Ben Tre City, Vietnam\\e-mail: \texttt{nguyenquanbahong@gmail.com}; website: \url{https://nqbh.github.io}.}}
\date{\today}

\begin{document}
\maketitle
\tableofcontents

%------------------------------------------------------------------------------%

\section{2 Góc Đối Đỉnh}

\begin{baitoan}[\cite{Hung_Mai_Toan_7_hinh_hoc}, 1.1., p. 5]
	Chứng minh: (a) Phân giác của 2 góc đối đỉnh là 2 tia đối nhau. (b) Phân giác ngoài của 2 góc đối đỉnh là 2 tia đối nhau.
\end{baitoan}

\begin{baitoan}[\cite{Hung_Mai_Toan_7_hinh_hoc}, 1.2., p. 6]
	Cho $\widehat{xOy}$ với Ot là phân giác $\widehat{xOy},\widehat{x'Oy'}$ với $Ot'$ là phân giác trong $\widehat{x'Oy'}$. Biết $Ox'$ là tia đối của $Ox,Ot'$ là tia đối của Ot. Chứng minh $Oy'$ là tia đối của Oy.
\end{baitoan}

\begin{baitoan}[\cite{Hung_Mai_Toan_7_hinh_hoc}, 1.3., p. 7]
	Cho 2 đường thẳng $xx',yy'$ cắt nhau tại O. Tia Om nằm giữa 2 tia $Ox',Oy'$. Ot là phân giác $\widehat{xOy}$. Chứng minh $\frac{1}{2}|\widehat{mOx'} - \widehat{mOy'}| + \widehat{mOt} = 180^\circ$.
\end{baitoan}

\begin{baitoan}[\cite{Hung_Mai_Toan_7_hinh_hoc}, 1.4., p. 8]
	Cho 2 đường thẳng $xx',yy'$ cắt nhau tại O. Tia Om nằm giữa 2 tia $Ox',Oy$. Ot là phân giác trong $\widehat{xOy}$. Chứng minh $\widehat{x'Om} + \widehat{y'Om} + 2\widehat{mOt} = 360^\circ$.
\end{baitoan}

\begin{baitoan}[\cite{Hung_Mai_Toan_7_hinh_hoc}, 1.5., p. 8]
	Cho $\widehat{xOy}$, tia Ot nằm giữa 2 tia $Ox,Oy$ sao cho $\widehat{yOt} = 2\widehat{xOt}$. $Ox'$ là tia đối của tia $Ox,Oy'$ là tia đối của tia Oy. Tia Om nằm giữa 2 tia $Ox',Oy$. Chứng minh: $\frac{1}{3}(2\widehat{mOx'} + \widehat{mOy'}) + \widehat{mOt} = 180^\circ$.
\end{baitoan}

\begin{baitoan}[\cite{Hung_Mai_Toan_7_hinh_hoc}, 1.6., p. 9]
	Cho $xx',yy',tt'$ cắt nhau tại O sao cho tia Ot nằm giữa 2 tia $Ox,Op$ với Op là phân giác trong $\widehat{xOy}$. Tia Oq nằm giữa 2 tia $Ot,Op$ sao cho $\widehat{tOp} = 3\widehat{qOp}$. Tia Om nằm giữa $Ox',Oy$. Chứng minh: $\frac{1}{3}(\widehat{mOx'} + \widehat{mOy'} + \widehat{mOt'}) + \widehat{mOq} = 180^\circ$.
\end{baitoan}

\begin{baitoan}[\cite{Hung_Mai_Toan_7_hinh_hoc}, 1.7., p. 10]
	Cho 4 đường thẳng $d_1,d_2,d_3,d_4$ đồng quy tại O. (a) Có bao nhiêu cặp góc đối đỉnh? (b) Chứng minh trong các góc tạo thành có 1 góc $\le45^\circ$.
\end{baitoan}

%------------------------------------------------------------------------------%

\section{2 Đường Thẳng Song Song \& 2 Đường Thẳng Vuông Góc}

%------------------------------------------------------------------------------%

\section{Miscellaneous}

%------------------------------------------------------------------------------%

\printbibliography[heading=bibintoc]
	
\end{document}