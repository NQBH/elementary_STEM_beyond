\documentclass{article}
\usepackage[backend=biber,natbib=true,style=alphabetic,maxbibnames=50]{biblatex}
\addbibresource{/home/nqbh/reference/bib.bib}
\usepackage[utf8]{vietnam}
\usepackage{tocloft}
\renewcommand{\cftsecleader}{\cftdotfill{\cftdotsep}}
\usepackage[colorlinks=true,linkcolor=blue,urlcolor=red,citecolor=magenta]{hyperref}
\usepackage{amsmath,amssymb,amsthm,float,graphicx,mathtools,tikz}
\usetikzlibrary{angles,calc,intersections,matrix,patterns,quotes,shadings}
\allowdisplaybreaks
\newtheorem{assumption}{Assumption}
\newtheorem{baitoan}{}
\newtheorem{cauhoi}{Câu hỏi}
\newtheorem{conjecture}{Conjecture}
\newtheorem{corollary}{Corollary}
\newtheorem{dangtoan}{Dạng toán}
\newtheorem{definition}{Definition}
\newtheorem{dinhly}{Định lý}
\newtheorem{dinhnghia}{Định nghĩa}
\newtheorem{example}{Example}
\newtheorem{ghichu}{Ghi chú}
\newtheorem{hequa}{Hệ quả}
\newtheorem{hypothesis}{Hypothesis}
\newtheorem{lemma}{Lemma}
\newtheorem{luuy}{Lưu ý}
\newtheorem{nhanxet}{Nhận xét}
\newtheorem{notation}{Notation}
\newtheorem{note}{Note}
\newtheorem{principle}{Principle}
\newtheorem{problem}{Problem}
\newtheorem{proposition}{Proposition}
\newtheorem{question}{Question}
\newtheorem{remark}{Remark}
\newtheorem{theorem}{Theorem}
\newtheorem{vidu}{Ví dụ}
\usepackage[left=1cm,right=1cm,top=5mm,bottom=5mm,footskip=4mm]{geometry}
\def\labelitemii{$\circ$}
\DeclareRobustCommand{\divby}{%
	\mathrel{\vbox{\baselineskip.65ex\lineskiplimit0pt\hbox{.}\hbox{.}\hbox{.}}}%
}

\title{Problem: Angle {\it\&} Line -- Bài Tập: Góc {\it\&} Đường Thẳng}
\author{Nguyễn Quản Bá Hồng\footnote{A Scientist {\it\&} Creative Artist Wannabe. E-mail: {\tt nguyenquanbahong@gmail.com}. Bến Tre City, Việt Nam.}}
\date{\today}

\begin{document}
\maketitle
\begin{abstract}
	This text is a part of the series {\it Some Topics in Elementary STEM \& Beyond}:
	
	{\sc url}: \url{https://nqbh.github.io/elementary_STEM}.
	
	Latest version:
	\begin{itemize}
		\item {\it Problem: Angle \& Line -- Bài Tập: Góc \& Đường Thẳng}.
		
		PDF: {\sc url}: \url{https://github.com/NQBH/elementary_STEM_beyond/blob/main/elementary_mathematics/grade_7/angle_line/problem/NQBH_angle_line_problem.pdf}.
		
		\TeX: {\sc url}: \url{https://github.com/NQBH/elementary_STEM_beyond/blob/main/elementary_mathematics/grade_7/angle_line/problem/NQBH_angle_line_problem.tex}.
		\item {\it Problem \& Solution: Angle \& Line -- Bài Tập \& Lời Giải: Góc \& Đường Thẳng}.
		
		PDF: {\sc url}: \url{https://github.com/NQBH/elementary_STEM_beyond/blob/main/elementary_mathematics/grade_7/angle_line/problem/NQBH_angle_line_solution.pdf}.
		
		\TeX: {\sc url}: \url{https://github.com/NQBH/elementary_STEM_beyond/blob/main/elementary_mathematics/grade_7/angle_line/problem/NQBH_angle_line_solution.tex}.
	\end{itemize}
\end{abstract}
\tableofcontents

%------------------------------------------------------------------------------%

\section{2 Góc Đối Đỉnh}

\begin{baitoan}[\cite{Hung_Mai_Toan_7_hinh_hoc}, 1.1., p. 5]
	Chứng minh: (a) Phân giác của 2 góc đối đỉnh là 2 tia đối nhau. (b) Phân giác ngoài của 2 góc đối đỉnh là 2 tia đối nhau.
\end{baitoan}

\begin{baitoan}[\cite{Hung_Mai_Toan_7_hinh_hoc}, 1.2., p. 6]
	Cho $\widehat{xOy}$ với Ot là phân giác $\widehat{xOy},\widehat{x'Oy'}$ với $Ot'$ là phân giác trong $\widehat{x'Oy'}$. Biết $Ox'$ là tia đối của $Ox,Ot'$ là tia đối của Ot. Chứng minh $Oy'$ là tia đối của Oy.
\end{baitoan}

\begin{baitoan}[\cite{Hung_Mai_Toan_7_hinh_hoc}, 1.3., p. 7]
	Cho 2 đường thẳng $xx',yy'$ cắt nhau tại O. Tia Om nằm giữa 2 tia $Ox',Oy'$. Ot là phân giác $\widehat{xOy}$. Chứng minh $\frac{1}{2}|\widehat{mOx'} - \widehat{mOy'}| + \widehat{mOt} = 180^\circ$.
\end{baitoan}

\begin{baitoan}[\cite{Hung_Mai_Toan_7_hinh_hoc}, 1.4., p. 8]
	Cho 2 đường thẳng $xx',yy'$ cắt nhau tại O. Tia Om nằm giữa 2 tia $Ox',Oy$. Ot là phân giác trong $\widehat{xOy}$. Chứng minh $\widehat{x'Om} + \widehat{y'Om} + 2\widehat{mOt} = 360^\circ$.
\end{baitoan}

\begin{baitoan}[\cite{Hung_Mai_Toan_7_hinh_hoc}, 1.5., p. 8]
	Cho $\widehat{xOy}$, tia Ot nằm giữa 2 tia $Ox,Oy$ sao cho $\widehat{yOt} = 2\widehat{xOt}$. $Ox'$ là tia đối của tia $Ox,Oy'$ là tia đối của tia Oy. Tia Om nằm giữa 2 tia $Ox',Oy$. Chứng minh: $\frac{1}{3}(2\widehat{mOx'} + \widehat{mOy'}) + \widehat{mOt} = 180^\circ$.
\end{baitoan}

\begin{baitoan}[\cite{Hung_Mai_Toan_7_hinh_hoc}, 1.6., p. 9]
	Cho $xx',yy',tt'$ cắt nhau tại O sao cho tia Ot nằm giữa 2 tia $Ox,Op$ với Op là phân giác trong $\widehat{xOy}$. Tia Oq nằm giữa 2 tia $Ot,Op$ sao cho $\widehat{tOp} = 3\widehat{qOp}$. Tia Om nằm giữa $Ox',Oy$. Chứng minh: $\frac{1}{3}(\widehat{mOx'} + \widehat{mOy'} + \widehat{mOt'}) + \widehat{mOq} = 180^\circ$.
\end{baitoan}

\begin{baitoan}[\cite{Hung_Mai_Toan_7_hinh_hoc}, 1.7., p. 10]
	Cho 4 đường thẳng $d_1,d_2,d_3,d_4$ đồng quy tại O. (a) Có bao nhiêu cặp góc đối đỉnh? (b) Chứng minh trong các góc tạo thành có 1 góc $\le45^\circ$.
\end{baitoan}

%------------------------------------------------------------------------------%

\section{2 Đường Thẳng Song Song \& 2 Đường Thẳng Vuông Góc}

\begin{baitoan}[\cite{Hung_Mai_Toan_7_hinh_hoc}, 2.1., p. 12]
	Cho $\widehat{xOy}$ \& tia Oz nằm giữa $Ox,Oy$. $Om,On$ lần lượt là phân giác trong Cho $\widehat{xOz},\widehat{yOz}$. Giả sử $\widehat{mOn} = 90^\circ$. Chứng minh $Ox,Oy$ là 2 tia đối nhau.
\end{baitoan}

\begin{baitoan}[\cite{Hung_Mai_Toan_7_hinh_hoc}, 2.2., p. 12]
	Cho $\widehat{xOy}$ nhọn. Dựng $Om\bot Ox$ sao cho $Om,Oy$ khác phía đối với Ox. Dựng $On\bot Oy$ sao cho $On,Ox$ khác phía đối với Oy. (a) Chứng minh $\widehat{xOn} = \widehat{yOm}$. (b) Chứng minh $\widehat{xOy} + \widehat{mOn} = 180^\circ$.
\end{baitoan}

\begin{baitoan}[\cite{Hung_Mai_Toan_7_hinh_hoc}, 2.3., p. 13]
	Cho $\widehat{xOy}$ nhọn. Dựng $Om\bot Ox$ sao cho $Om,Oy$ cùng phía đối với Ox.Dựng $On\bot Oy$ sao cho $On,Ox$ cùng phía đối với Oy. Chứng minh $\widehat{xOy} + \widehat{mOn} = 180^\circ$.
\end{baitoan}

\begin{baitoan}[\cite{Hung_Mai_Toan_7_hinh_hoc}, 2.4., p. 13]
	Cho $\widehat{xOy}$ bẹt, tia Oz bất kỳ sao cho $\widehat{xOz}$ nhọn. Tia Om nằm giữa 2 tia $Ox,Oz$ sao cho $\widehat{xOm} = 2\widehat{zOm}$. Tia $On\bot Om$. Chứng minh $\widehat{zOn} - \frac{1}{2}\widehat{yOn} = 45^\circ$.
\end{baitoan}

\begin{baitoan}[\cite{Hung_Mai_Toan_7_hinh_hoc}, 2.5., p. 14]
	Cho $\widehat{xOy}$ bẹt, tia Oz bất kỳ sao cho $\widehat{xOz}$ nhọn. Tia Om nằm giữa 2 tia $Ox,Oz$ sao cho $\widehat{xOm} = \frac{3}{4}\widehat{xOz}$. Tia $On\bot Om$. Chứng minh $\widehat{zOn} - \frac{1}{3}\widehat{yOn} = 60^\circ$.
\end{baitoan}

\begin{baitoan}[\cite{Hung_Mai_Toan_7_hinh_hoc}, 2.6., p. 14]
	Cho $\widehat{xOy}$ bẹt, tia Oz bất kỳ sao cho $\widehat{xOz}$ nhọn. Tia Om nằm giữa 2 tia $Ox,Oz$ sao cho $\widehat{zOm} = \frac{1}{7}\widehat{xOz}$. Tia $On\bot Om$. Chứng minh $\widehat{zOn} - \frac{1}{6}\widehat{yOn} = 75^\circ$.
\end{baitoan}

\begin{baitoan}[\cite{Hung_Mai_Toan_7_hinh_hoc}, 2.7., p. 15]
	Cho đoạn thẳng AB, 2 tia $Ax,By$ dựng cùng phía với AB sao cho $\widehat{BAx} = 2\alpha,\widehat{ABy} = 3\alpha$. Tìm $\alpha$ để $Ax\parallel By$.
\end{baitoan}

\begin{baitoan}[\cite{Hung_Mai_Toan_7_hinh_hoc}, 2.8., p. 15]
	Cho 2 đường thẳng $a\parallel b$, $d$ là đường thẳng cắt $a,b$. Chứng minh: (a) Phân giác của 2 góc đồng vị thì song song. (b) Phân giác của 2 góc so le trong thì song song. (c) Phân giác của 2 góc trong cùng phía thì vuông góc.
\end{baitoan}

\begin{baitoan}[\cite{Hung_Mai_Toan_7_hinh_hoc}, 2.10., p. 17]
	Cho 2 tia $Ax\parallel By$ với $Ax,By$ cùng phía đối với AB. Điểm C bất kỳ trên mặt phẳng, biết $\widehat{CAx} = \alpha,\widehat{CBy} = \beta$. Tính $\widehat{ACB}$.
\end{baitoan}

\begin{baitoan}[\cite{Hung_Mai_Toan_7_hinh_hoc}, 2.14., p. 23]
	Cho $\Delta ABC$, phân giác trong AD, M bất kỳ thuộc đường thẳng BC. Qua M vẽ đường thẳng song song AD cắt $AB,AC$ lần lượt ở $P,Q$. Chứng minh $\Delta APQ$ cân.
\end{baitoan}

\begin{baitoan}[\cite{Hung_Mai_Toan_7_hinh_hoc}, 2.15., p. 24]
	Cho 5 đường thẳng nằm trong mặt phẳng thỏa mãn không có 2 đường thẳng nào song song. Chứng minh tồn tại cặp đường thẳng tạo với nhau 1 góc $< 36^\circ$.
\end{baitoan}
\cite[2.9., p. 17, 2.11, p. 20, 2.12., p. 21, 2.13, p. 22]{Hung_Mai_Toan_7_hinh_hoc}.

%------------------------------------------------------------------------------%

\section{Miscellaneous}

%------------------------------------------------------------------------------%

\printbibliography[heading=bibintoc]
	
\end{document}