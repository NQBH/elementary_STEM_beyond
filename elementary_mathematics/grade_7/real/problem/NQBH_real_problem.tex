\documentclass{article}
\usepackage[backend=biber,natbib=true,style=alphabetic,maxbibnames=50]{biblatex}
\addbibresource{/home/nqbh/reference/bib.bib}
\usepackage[utf8]{vietnam}
\usepackage{tocloft}
\renewcommand{\cftsecleader}{\cftdotfill{\cftdotsep}}
\usepackage[colorlinks=true,linkcolor=blue,urlcolor=red,citecolor=magenta]{hyperref}
\usepackage{amsmath,amssymb,amsthm,float,graphicx,mathtools,tikz}
\usetikzlibrary{angles,calc,intersections,matrix,patterns,quotes,shadings}
\allowdisplaybreaks
\newtheorem{assumption}{Assumption}
\newtheorem{baitoan}{}
\newtheorem{cauhoi}{Câu hỏi}
\newtheorem{conjecture}{Conjecture}
\newtheorem{corollary}{Corollary}
\newtheorem{dangtoan}{Dạng toán}
\newtheorem{definition}{Definition}
\newtheorem{dinhly}{Định lý}
\newtheorem{dinhnghia}{Định nghĩa}
\newtheorem{example}{Example}
\newtheorem{ghichu}{Ghi chú}
\newtheorem{hequa}{Hệ quả}
\newtheorem{hypothesis}{Hypothesis}
\newtheorem{lemma}{Lemma}
\newtheorem{luuy}{Lưu ý}
\newtheorem{nhanxet}{Nhận xét}
\newtheorem{notation}{Notation}
\newtheorem{note}{Note}
\newtheorem{principle}{Principle}
\newtheorem{problem}{Problem}
\newtheorem{proposition}{Proposition}
\newtheorem{question}{Question}
\newtheorem{remark}{Remark}
\newtheorem{theorem}{Theorem}
\newtheorem{vidu}{Ví dụ}
\usepackage[left=1cm,right=1cm,top=5mm,bottom=5mm,footskip=4mm]{geometry}
\def\labelitemii{$\circ$}
\DeclareRobustCommand{\divby}{%
	\mathrel{\vbox{\baselineskip.65ex\lineskiplimit0pt\hbox{.}\hbox{.}\hbox{.}}}%
}

\title{Problem: Real $\mathbb{R}$ -- Bài Tập: Số Thực $\mathbb{R}$}
\author{Nguyễn Quản Bá Hồng\footnote{Independent Researcher, Ben Tre City, Vietnam\\e-mail: \texttt{nguyenquanbahong@gmail.com}; website: \url{https://nqbh.github.io}.}}
\date{\today}

\begin{document}
\maketitle
\tableofcontents

%------------------------------------------------------------------------------%

\section{Irrational. Real -- Số Vô Tỷ. Số Thực}

\begin{baitoan}[\cite{Binh_boi_duong_Toan_7_tap_1}, H1, p. 26]
	Hoàn thành: (a) $-\dfrac{1}{30}$ là số thập phân $\ldots$ (b) $0.222\ldots$ là số thập phân vô hạn $\ldots$ (c) $\dfrac{72}{75} = \ldots$ (d) $\sqrt{2}$ là số $\ldots$ (e) $\dfrac{4}{9} = \ldots$ (f) $0.(142857) = \ldots$
\end{baitoan}

\begin{baitoan}[\cite{Binh_boi_duong_Toan_7_tap_1}, H2, p. 26]
	{\rm Đ{\tt/}S?} (a) $\sqrt{a} > 0$ với $a\ge0$. (b) $-\sqrt{a}\le0$ với $a\ge0$. (c) Các điểm biểu diễn số hữu tỷ không lấp đầy trục số. (d) Các điểm biểu diễn số vô tỷ lấp đầy trục số. (e) Nếu $a$ là số thực thì $a$ là số vô tỷ.
\end{baitoan}

\begin{baitoan}[\cite{Binh_boi_duong_Toan_7_tap_1}, VD1, p. 27]
	Viết 4 phân số $-\dfrac{9}{40},\dfrac{5}{11},-\dfrac{28}{175},\dfrac{11}{24}$ dưới dạng 1 số thập phân hữu hạn hoặc 1 số thập phân vô hạn tuần hoàn \& giải thích vì sao viết được như vậy.
\end{baitoan}

\begin{baitoan}[\cite{Binh_boi_duong_Toan_7_tap_1}, VD2, p. 27]
	Tính diện tích của các hình chữ nhật có số đo 2 cạnh lần lượt là $a,b$ {\rm cm}, biết $a$ là tử \& $b$ là mẫu của {\rm phân số tối giản} viết từ 3 số thập phân $0.26,0.454545\ldots,0.13777\ldots$
\end{baitoan}

\begin{baitoan}[\cite{Binh_boi_duong_Toan_7_tap_1}, VD3, p. 27]
	Làm tròn mỗi số $12.064,9.272727\ldots,3.14159$ đến hàng đơn vị \& đến chữ số thập phân thứ $1,2,3$.
\end{baitoan}

\begin{baitoan}[\cite{Binh_boi_duong_Toan_7_tap_1}, VD4, p. 28]
	Tại SEA Games 27, vận động viên Nguyễn Thị Ánh Viên đã về Nhất nội dung {\rm200 m} bơi ngửa với thời gian $2$ phút $14$ giây $80$, giành Huy chương Vàng \& trở thành vận động viên đầu tiên phá kỷ lục của SEA Games 27. Về thứ Nhì là Yosaputra Venesia (Indonesia) với thời gian $2$ phút $20$ giây $35$ \& về thứ 3 là Lim Shen Meagan (Singapore) với thời gian $2$ phút $21$ giây $19$. Hỏi thời gian gần đúng đến hàng đơn vị giây của mỗi vận động viên là bao nhiêu?
\end{baitoan}

\begin{baitoan}[\cite{Binh_boi_duong_Toan_7_tap_1}, VD5, p. 28]
	Cho 6 phân số $\dfrac{1}{7},\dfrac{2}{7},\dfrac{3}{7},\dfrac{4}{7},\dfrac{5}{7},\dfrac{6}{7}$. (a) Các phân số trên đều đổi được ra số thập phân vô hạn tuần hoàn. Tìm chu kỳ \& nhận xét các chữ số trong chu kỳ của các số thập phân vô hạn tuần hoàn trên. (b) Làm tròn các số thập phân trên đến chữ số thập phân thứ 2, thứ 4, \& thứ 6. (c) Tìm chữ số thứ $100$ sau dấu phẩy của số thập phân viết từ phân số $\dfrac{5}{7}$.
\end{baitoan}

\begin{baitoan}[\cite{Binh_boi_duong_Toan_7_tap_1}, VD6, p. 29]
	1 vệ tinh bay trên quỹ đạo tròn vòng quanh Trái Đất. Biết quỹ đạo của vệ tinh có độ dài là {\rm66000 km}. Hỏi độ dài quỹ đạo của vệ tinh sẽ tăng bao nhiêu {\rm km} nếu bán kính của quỹ đạo tăng lên {\rm7km}? 
\end{baitoan}

\begin{baitoan}[\cite{Binh_boi_duong_Toan_7_tap_1}, VD7, p. 29]
	1 đơn vị đo chiều dài của Anh là inch, được ký hiệu là {\rm in} \& $\rm1\ in = 2.54\ cm$. (a) Hỏi {\rm1 cm} gần bằng bao nhiêu {\rm in} (làm tròn đến chữ số thập phân thứ 2)? (b) Tivi {\rm21 in} là màn hình tivi có đường chéo bằng {\rm21 in}. Tivi {\rm21 in, 23 in, 27 in, 29 in} có đường chéo màn hình bằng bao nhiêu {\rm cm} (làm tròn đến chữ số thập phân thứ nhất)? 
\end{baitoan}

\begin{baitoan}[\cite{Binh_boi_duong_Toan_7_tap_1}, 4.1., p. 29]
	Viết 6 phân số $-\dfrac{11}{35},\dfrac{9}{80},-\dfrac{48}{150},\dfrac{44}{121},\dfrac{55}{75},\dfrac{73}{81}$ dưới dạng 1 số thập phân hữu hạn hoặc 1 số thập phân vô hạn tuần hoàn. Giải thích vì sao chúng viết được như vậy.
\end{baitoan}

\begin{baitoan}[\cite{Binh_boi_duong_Toan_7_tap_1}, 4.2., p. 29]
	Lấy số $\pi$ gần bằng $\dfrac{22}{7}$. Tính diện tích hình tròn biết bán kính là $0.(45)$ {\rm cm}, $\dfrac{21}{22}$ {\rm cm}.
\end{baitoan}

\begin{baitoan}[\cite{Binh_boi_duong_Toan_7_tap_1}, 4.3., p. 29]
	1 vệ tinh bay trên quỹ đạo hình tròn vòng quanh Trái Đất. Biết quỹ đạo của vệ tinh có độ dài là {\rm66000 km}. Hỏi độ dài quỹ đạo của vệ tinh giảm bao nhiêu {\rm km} nếu bán kính của quỹ đạo giảm {\rm70 km}?
\end{baitoan}

\begin{baitoan}[\cite{Binh_boi_duong_Toan_7_tap_1}, 4.4., p. 29]
	Viết 4 số $0.(0001),-0.3(18),-2.37(1),3.24(81)$ dưới dạng 1 phân số.
\end{baitoan}

\begin{baitoan}[\cite{Binh_boi_duong_Toan_7_tap_1}, 4.5., p. 30]
	So sánh: (a) $A = \dfrac{2021}{\sqrt{2022}}$ \& $B = \dfrac{2022}{\sqrt{2021}}$. (b) $A = \dfrac{\sqrt{121}}{\sqrt{12321}}$ \& $B = \dfrac{\sqrt{12321}}{\sqrt{1234321}}$.
\end{baitoan}

\begin{baitoan}[\cite{Binh_boi_duong_Toan_7_tap_1}, 4.6., p. 30]
	Viết mỗi phân số sau dưới dạng số thập phân ta được 1 số thập phân hữu hạn hay số thập phân vô hạn tuần hoàn ($n\in\mathbb{N}^\star$): (a) $A = \dfrac{11n^2 + 121n}{55n}$. (b) $B = \dfrac{79! + 79}{5609}$.
\end{baitoan}

\begin{baitoan}[\cite{Binh_boi_duong_Toan_7_tap_1}, 4.7., p. 30]
	Tìm 2 số thập phân $\overline{0.abc}$ \& $\overline{0.(abc)}$ biết: (a) $\dfrac{1}{\overline{0.abc}} = n$. (b) $\dfrac{1}{\overline{0.(abc)}} = n$, trong đó $n\in\mathbb{N}$ \&: (a) $a,b,c$ là 3 chữ số khác nhau. (b) $a,b,c$ là 3 chữ số không nhất thiết khác nhau.
\end{baitoan}

\begin{baitoan}[\cite{Binh_boi_duong_Toan_7_tap_1}, 4.8., p. 30]
	(a0 Các bánh xe của 1 chiếc xe tải chạy tới vận tốc {\rm60 km{\tt/}h}, thực hiện $4$ vòng quay trong 1 giây. Hỏi đường kính của bánh xe là bao nhiêu? (b) 1 hình tròn nằm ``khít'' trong 1 hình vuông. Biết cạnh hình vuông bằng {\rm3.72 cm} \& đường kính hình tròn bằng {\rm2.48 cm}. Tính diện tích phần hình vuông còn lại không bị hình tròn che.
\end{baitoan}

\begin{baitoan}[\cite{Binh_boi_duong_Toan_7_tap_1}, 4.9., p. 30]
	Tìm số thập phân thứ $2021$ của phân số $\frac{15}{19}$ khi viết dưới dạng số thập phân.
\end{baitoan}

\begin{baitoan}[\cite{Binh_boi_duong_Toan_7_tap_1}, 1, p. 30]
	So sánh các số thập phân: (a) $0.(26)$ \& $0.261$. (b) $\overline{0.(a_1a_2)},\overline{0.a_1(a_2a_1)},\overline{0.(a_1a_2a_1a_2)}$. (c) $0.15$ \& $0.14(9)$.
\end{baitoan}

\begin{baitoan}[\cite{Binh_boi_duong_Toan_7_tap_1}, 2, p. 30]
	Biết $a + b = 9$. Tính $\overline{0.a(b)} + \overline{0.b(a)}$.
\end{baitoan}

\begin{baitoan}[\cite{Tuyen_Toan_7}, VD20, p. 19]
	Tính độ dài mỗi cạnh của 1 sân hình vuông có diện tích lần lượt là $\rm16m^2$, $\rm6.25m^2$, $\rm6m^2$. Trong mỗi trường hợp, cho biết độ dài mỗi cạnh được biểu diễn bằng số hữu tỷ hay vô tỷ?
\end{baitoan}

\begin{baitoan}[\cite{Tuyen_Toan_7}, VD21, p. 19]
	Cho $A = \dfrac{5}{\sqrt{x} - 3}$. Tìm số chính phương $x$ để biểu thức $A$ có giá trị nguyên.\hfill{\sf Ans:} $4,16,64$.
\end{baitoan}

\begin{baitoan}[\cite{Tuyen_Toan_7}, 69., p. 19]
	Tính: (a) $\sqrt{(-5)^2} + \sqrt{5^2} - \sqrt{(-3)^2} - \sqrt{3^2}$. (b) $\left[\sqrt{4^2} + \sqrt{(-4)^2}\right]\cdot\sqrt{\dfrac{1}{4^3}} - \sqrt{\dfrac{1}{3^4}}$.\hfill{\sf Ans:} (a) $4$. (b) $\frac{8}{9}$.
\end{baitoan}

\begin{baitoan}[\cite{Tuyen_Toan_7}, 70., pp. 19--20]
	Tìm $x\in\mathbb{R}$ biết: (a) $4x^2 - 1 = 0$. (b) $2x^2 + 0.82 = 1$.\hfill{\sf Ans:} (a) $\pm\frac{1}{2}$. (b) $\pm0.3$.
\end{baitoan}

\begin{baitoan}[\cite{Tuyen_Toan_7}, 71., p. 20]
	Tìm $x\ge 0$ biết: (a) $7 - \sqrt{x} = 0$. (b) $3\sqrt{x} + 1 = 40$. (c) $\dfrac{5}{12}\sqrt{x} - \dfrac{1}{6} = \dfrac{1}{3}$. (d) $\sqrt{x + 1} + 2 = 0$.
	\\\mbox{}\hfill{\sf Ans:} (a) $49$. (b) $169$. (c) $1.44$. (d) $\overline{\exists}$.
\end{baitoan}

\begin{baitoan}[\cite{Tuyen_Toan_7}, 72., p. 20]
	Cho $A = \sqrt{\sqrt{x} - 1}{2}$. Tìm số chính phương $x < 50$ để $A$ có giá trị nguyên.\hfill{\sf Ans:} $1,9,25,49$.
\end{baitoan}

\begin{baitoan}[\cite{Tuyen_Toan_7}, 73., p. 20]
	Cho $A = \sqrt{9}{\sqrt{x} - 5}$. Tìm số chính phương $x$ để $A$ có giá trị nguyên.\hfill{\sf Ans:} $4,16,36,64,196$.
\end{baitoan}

\begin{baitoan}[\cite{Tuyen_Toan_7}, 74., p. 20]
	Bên trong 1 hình vuông cạnh $5$ có $76$ điểm. Chứng minh: Tồn tại $4$ điểm trong các điểm đó thuộc 1 hình tròn có bán kính là $\frac{3}{4}$.
\end{baitoan}

\begin{baitoan}[\cite{Binh_Toan_7_tap_1}, \S7, VD11]
	Chứng minh: (a) $\sqrt{15}$ là số vô tỷ. (b) Nếu $a\in\mathbb{N}$ không phải là số chính phương thì $\sqrt{a}$ là số vô tỷ.	
\end{baitoan}

\begin{baitoan}[\cite{Binh_Toan_7_tap_1}, \S7, 69.]
	Tìm $x$ biết: (a) $x^2 = 81$. (b) $(x - 1)^2 = \dfrac{9}{16}$. (c) $x - 2\sqrt{x} = 0$. (d) $x = \sqrt{x}$.	
\end{baitoan}

\begin{baitoan}[\cite{Binh_Toan_7_tap_1}, \S7, 70.]
	Cho $A = \dfrac{\sqrt{x} + 1}{\sqrt{x} - 1}$. Chứng minh với $x = \dfrac{16}{9}$ \& $x = \dfrac{25}{9}$ thì $A$ có giá trị là số nguyên.
\end{baitoan}

\begin{baitoan}[\cite{Binh_Toan_7_tap_1}, \S7, 71.]
	Cho $A = \dfrac{\sqrt{x} + 1}{\sqrt{x} - 3}$. Tìm số nguyên $x$ để $A$ có giá trị là 1 số nguyên.
\end{baitoan}

\begin{baitoan}[\cite{Binh_Toan_7_tap_1}, \S7, 72.]
	Chứng minh: (a) $\sqrt{2}$ là số vô tỷ. (b) $5 - \sqrt{2}$ là số vô tỷ.	
\end{baitoan}

\begin{baitoan}[\cite{Binh_Toan_7_tap_1}, \S7, 73.]
	(a) Có 2 số vô tỷ nào mà tích là 1 số hữu tỷ hay không? (b) Có 2 số vô tỷ dương nào mà tổng là 1 số hữu tỷ hay không?	
\end{baitoan}

\begin{baitoan}[\cite{Binh_Toan_7_tap_1}, \S7, 74.]
	Ký hiệu $\lfloor x\rfloor$ là số nguyên lớn nhất không vượt quá $x$. Tính giá trị của tổng: $\sum_{i=1}^{35} \lfloor\sqrt{i}\rfloor = \lfloor\sqrt{1}\rfloor + \lfloor\sqrt{2}\rfloor + \cdots + \lfloor\sqrt{35}\rfloor$.
\end{baitoan}

\begin{baitoan}[\cite{Binh_Toan_7_tap_1}, \S7, 75.]
	Cho $a,b\in\mathbb{R}$ sao cho các tập hợp $\{a^2 + a,b\}$ \& $\{b^2 + b,b\}$ bằng nhau. Chứng minh $a = b$.
\end{baitoan}

\begin{baitoan}[\cite{Tuyen_Toan_7}, VD22, p. 20]
	Không dùng bảng số hoặc máy tính, so sánh $\sqrt{50 + 2}$ \& $\sqrt{50} + \sqrt{2}$.
\end{baitoan}
\noindent\textit{Hint.} Tìm các số chính phương gần với các số đã cho dưới dấu căn để loại bỏ căn thức.

\begin{proof}[Giải]
	$\sqrt{50} + \sqrt{2} > \sqrt{49} + \sqrt{1} = 7 + 1 = 8 = \sqrt{64} > \sqrt{52} = \sqrt{50 + 2}$.
\end{proof}

\begin{baitoan}[Mở rộng \cite{Tuyen_Toan_7}, VD22, p. 20]
	So sánh $\sqrt{a + b}$ \& $\sqrt{a} + \sqrt{b}$, với $a,b\ge 0$.
\end{baitoan}

\begin{proof}[Giải]
	$(\sqrt{a} + \sqrt{b})^2 - (\sqrt{a + b})^2 = a + 2\sqrt{ab} + b - (a + b) = 2\sqrt{ab}\ge 0\Rightarrow(\sqrt{a} + \sqrt{b})^2\ge(\sqrt{a + b})^2\Rightarrow|\sqrt{a} + \sqrt{b}|\ge|\sqrt{a + b}|$ mà $\sqrt{a} + \sqrt{b}\ge 0$ \& $\sqrt{a + b}\ge 0$ suy ra\footnote{Vì chúng không âm nên có thể bỏ dấu giá trị tuyệt đối ở bất đẳng thức cuối mà không cần đổi dấu.} $\sqrt{a} + \sqrt{b}\ge\sqrt{a + b}$. Dấu ``$=$'' xảy ra khi \& chỉ khi $ab = 0$, i.e., ít nhất 1 trong 2 số $a,b$ phải bằng $0$. Vậy, $\sqrt{a} + \sqrt{b}\ge\sqrt{a + b}$, $\forall a,b\in\mathbb{R}$, ``$=$''$\Leftrightarrow ab = 0$.
\end{proof}

\begin{baitoan}[\cite{Tuyen_Toan_7}, 75., p. 21]
	Tìm $x$ biết: $6\sqrt{x} + \sqrt{12.25} = 8$.\hfill{\sf Ans:} $0.5625$.
\end{baitoan}

\begin{baitoan}[\cite{Tuyen_Toan_7}, 76., p. 21]
	So sánh:
	
	(a) $4\frac{8}{33}$ \& $3\sqrt{2}$;
	(b) $5\sqrt{(-10)^2}$ \& $10\sqrt{(-5)^2}$.
	\hfill{\sf Ans:} (a) $4\frac{8}{33} < 3\sqrt{2}$. (b) `$=$'.
\end{baitoan}

\begin{baitoan}[\cite{Tuyen_Toan_7}, 77., p. 21]
	Không dùng bảng số hoặc máy tính, so sánh:
	
	(a) $\sqrt{26} + \sqrt{17}$ \& $9$.
	(b) $\sqrt{8} - \sqrt{5}$ \& $1$;
	(c) $\sqrt{63 - 27}$ \& $\sqrt{63} - \sqrt{27}$.
	\hfill{\sf Ans:} (a) $\sqrt{26} + \sqrt{17} > 9$. (b) $\sqrt{8} - \sqrt{5} > 1$. (c) $\sqrt{63 - 27} > \sqrt{63} - \sqrt{27}$.
\end{baitoan}

\begin{baitoan}[Mở rộng \cite{Tuyen_Toan_7}, 77., p. 21]
	So sánh $\sqrt{a - b}$ \& $\sqrt{a} - \sqrt{b}$ với $a,b\in\mathbb{R}$, $a\ge b\ge 0$.
\end{baitoan}

\begin{proof}[Giải]
	Trong bất đẳng thức $\sqrt{a} + \sqrt{b}\ge\sqrt{a + b}$, $\forall a,b\in\mathbb{R}$ đã chứng minh, thay $a$ bởi $a - b$ thu được $\sqrt{a - b} + \sqrt{b}\ge\sqrt{(a - b) + b}$, i.e., $\sqrt{a - b}\ge\sqrt{a} - \sqrt{b}$. Đẳng thức xảy ra khi \& chỉ khi $(a - b)b = 0$, i.e., $a = b$ hoặc $b = 0$. Vậy, $\sqrt{a - b}\ge\sqrt{a} - \sqrt{b}$, $\forall a,b\in\mathbb{R}$, $a\ge b\ge 0$, `$=$'$\Leftrightarrow(b = 0)\lor(a = b)$.
\end{proof}

\begin{baitoan}[\cite{Tuyen_Toan_7}, 78., p. 21]
	So sánh $A$ \& $B$ biết: $A = \sqrt{225} - \frac{1}{\sqrt{5}} - 1$, $B = \sqrt{196} - \frac{1}{\sqrt{6}}$.\hfill{\sf Ans:} $A < B$.
\end{baitoan}

\begin{baitoan}[\cite{Tuyen_Toan_7}, 79., p. 21]
	Cho $P = \frac{1}{2} + \sqrt{x}$, $Q = 7 - 2\sqrt{x - 1}$. Tìm:
	
	(a) Giá trị nhỏ nhất của $P$;
	(b) Giá trị lớn nhất của $Q$.
	\hfill{\sf Ans:} (a) $\min P = \frac{1}{2}\Leftrightarrow x = 0$. (b) $\max Q = 7\Leftrightarrow x = 1$.
\end{baitoan}

\begin{baitoan}[\cite{Tuyen_Toan_7}, 80., p. 21]
	Xét xem các số $x$ \& $y$ có thể là số vô tỷ không nếu biết:
	
	(a) $x + y$ \& $x - y$ đều là số hữu tỷ;
	(b) $x + y$ \& $\frac{x}{y}$ đều là số hữu tỷ.
	\hfill{\sf Ans:} (a) $x,y\in\mathbb{Q}$. (b) Có thể, e.g., $\{x,y\} = \{\pm\sqrt{2}\}$.
\end{baitoan}

%------------------------------------------------------------------------------%

\subsection{Giá Trị Tuyệt Đối của  Số Thực \& Biểu Thức}

\begin{baitoan}[\cite{Tuyen_Toan_7}, VD24, p. 22]
	Cho $A = |\sqrt{2} - \sqrt{3}| - |-\sqrt{7}| + |\sqrt{7} - \sqrt{3}|$, $B = |\sqrt{5} - \sqrt{7}| - |\sqrt{7} - \sqrt{6}|$. So sánh $A$ \& $B$.\hfill{\sf Ans:} $A = -\sqrt{2} < 0 < \sqrt{6} - \sqrt{5} = B$.
\end{baitoan}	

\begin{baitoan}[\cite{Tuyen_Toan_7}, VD25, p. 22]
	Tìm $x,y\in\mathbb{R}$ biết: $|x + y| + |y - \sqrt{11}| = 0$.\hfill{\sf Ans:} $x = -\sqrt{11}$, $y = \sqrt{11}$.
\end{baitoan}
\noindent\textit{Hint}: Dựa vào tính chất giá trị tuyệt đối của mọi số thực đều không âm ($\ge 0$). Tổng của 2 số không âm mà bằng $0$ thì mỗi số hạng phải bằng $0$.

\begin{baitoan}[\cite{Tuyen_Toan_7}, VD26, p. 22]
	Tìm $x$ biết: $|x + \sqrt{2}| = \sqrt{3}$.\hfill{\sf Ans:} $-\sqrt{3} - \sqrt{2}$.
\end{baitoan}

\begin{baitoan}[\cite{Tuyen_Toan_7}, 81., p. 22]
	Tính:
	
	(a) $|-2.15| - |-3.75| + \left|\frac{4}{3} + \frac{4}{15}\right|$;
	(b) $|-\sqrt{42} - \sqrt{53}| - |\sqrt{53} - \sqrt{61}|$\\$+ |\sqrt{61} - \sqrt{42}| - |-\sqrt{53}|$;
	(c) $|-150| - |100|:|-4| + |37|\cdot|-3|$.
	\hfill{\sf Ans:} (a) $0$. (b) $\sqrt{53}$. (c) $236$.
\end{baitoan}

\begin{baitoan}[\cite{Tuyen_Toan_7}, 82., p. 22]
	Tìm $x$ biết:
	
	(a) $\left|5x - \frac{3}{4}\right| + \frac{7}{4} = 3$;
	(b) $9 - |x - \sqrt{10}| = 10$.
	\hfill{\sf Ans:} (a) $\frac{2}{5}$, $-\frac{1}{10}$. (b) $\overline{\exists}$.
\end{baitoan}

\begin{baitoan}[\cite{Tuyen_Toan_7}, 83., p. 22]
	Tìm $x,y$ biết $|x - \sqrt{3}| + |y + \sqrt{5}| = 0$.\hfill{\sf Ans:} $x = \sqrt{3}$, $y = -\sqrt{5}$.
\end{baitoan}

\begin{baitoan}[Mở rộng \cite{Tuyen_Toan_7}, 83., p. 22]
	Tìm $x,y$ biết $|x - a| + |y - b| = 0$ với $a,b\in\mathbb{R}$ cho trước.\hfill{\sf Ans:} $x = a$, $y = b$.
\end{baitoan}

\begin{baitoan}[\cite{Tuyen_Toan_7}, 84., p. 23]
	Tìm $x,y$ biết $\left|\frac{1}{2} - \frac{1}{3} + x\right| = -\frac{1}{4} - |y|$.\hfill{\sf Ans:} $\overline{\exists}$.
\end{baitoan}

\begin{baitoan}[\cite{Tuyen_Toan_7}, 85., p. 23]
	Cho $x,y$ là 2 số thực cùng dấu \& $|x| > |y|$. So sánh $x$ \& $y$.\\\mbox{}\hfill{\sf Ans:} Nếu $x,y > 0$: $x > y$. Nếu $x,y < 0$: $x < y$.
\end{baitoan}

\begin{baitoan}[\cite{Tuyen_Toan_7}, 86., p. 23]
	Tìm $x$ thỏa mãn các bất đẳng thức sau:
	
	(a) $\left|x - \frac{5}{3}\right| < \frac{1}{3}$;
	(b) $\left|x + \frac{11}{2}\right| > |-5.5|$.
	\\\mbox{}\hfill{\sf Ans:} (a) $\frac{4}{3} < x < 2$. (b) $x > 0\lor x < -11$.
\end{baitoan}

\begin{baitoan}[\cite{Tuyen_Toan_7}, 87., p. 23]
	Tìm giá trị nhỏ nhất của biểu thức:
	
	(a) $M = \left|x + \frac{15}{19}\right|$;
	(b) $N = \left|x - \frac{4}{7}\right| - \frac{1}{2}$.
	\\\mbox{}\hfill{\sf Ans:} (a) $\min M = 0\Leftrightarrow x = -\frac{15}{19}$. (b) $\min N = -\frac{1}{2}\Leftrightarrow x = \frac{4}{7}$.
\end{baitoan}

\begin{baitoan}[\cite{Tuyen_Toan_7}, 88., p. 23]
	Tìm giá trị lớn nhất của biểu thức:
	
	(a) $P = -\left|\frac{5}{3} - x\right|$;
	(b) $Q = 9 - \left|x - \frac{1}{10}\right|$.
	\\\mbox{}\hfill{\sf Ans:} (a) $\max P = 0\Leftrightarrow x = \frac{5}{3}$. (b) $\max Q = 9\Leftrightarrow x = \frac{1}{10}$.
\end{baitoan}

\begin{baitoan}[\cite{Tuyen_Toan_7}, 89., p. 23]
	Tìm giá trị nhỏ nhất của biểu thức: $A = |x - 1| + |9 - x|$.\\\mbox{}\hfill{\sf Ans:} $\min A = 8\Leftrightarrow 1\le x\le 9$.
\end{baitoan}

\begin{baitoan}[\cite{Tuyen_Toan_7}, 90., p. 23]
	Tìm giá trị lớn nhất của biểu thức: $B = |x - 4| + |x - 7|$.\\\mbox{}\hfill{\sf Ans:} $\min B = 3\Leftrightarrow 4\le x\le 7$.
\end{baitoan}

\subsubsection{Tính giá trị của 1 biểu thức}

\begin{baitoan}[\cite{Binh_Toan_7_tap_1}, VD15, p. 19]
	Tính giá trị của biểu thức $A = 3x^2 - 2x + 1$ với $|x| = \frac{1}{2}$.
\end{baitoan}

\subsubsection{Rút gọn biểu thức chứa dấu giá trị tuyệt đối}

\begin{baitoan}[\cite{Binh_Toan_7_tap_1}, VD16, p. 20]
	Rút gọn biểu thức $|a| + a$.
\end{baitoan}

\subsubsection{Tìm giá trị của biến trong đẳng thức chứa dấu giá trị tuyệt đối}

\begin{baitoan}[\cite{Binh_Toan_7_tap_1}, VD17, p. 20]
	Tìm $x$ thỏa $2|3x - 1| + 1 = 5$.
\end{baitoan}

\begin{baitoan}[\cite{Binh_Toan_7_tap_1}, VD18, p. 20]
	Tìm $x$ thỏa $|x - 5| - x = 3$.
\end{baitoan}

\begin{baitoan}[\cite{Binh_Toan_7_tap_1}, VD19, p. 20]
	Tìm $x$ thỏa $|x - 2| = 2x - 3$.
\end{baitoan}

\begin{baitoan}[\cite{Binh_Toan_7_tap_1}, VD20, p. 20]
	Với giá trị nào của $a,b$ thì đẳng thức $|a(b - 2)| = a(2 - b)$ đúng?
\end{baitoan}

\begin{baitoan}[\cite{Binh_Toan_7_tap_1}, VD21, p. 21]
	Tìm các số $a,b\in\mathbb{R}$ thỏa $a + b = |a| - |b|$.
\end{baitoan}

\subsubsection{Tìm giá trị nhỏ nhất, giá trị lớn nhất của biểu thức chứa dấu giá trị tuyệt đối}

\begin{baitoan}[\cite{Binh_Toan_7_tap_1}, VD22, p. 21]
	Tìm giá trị nhỏ nhất của biểu thức $A = 2|3x - 1| - 4$.
\end{baitoan}

\begin{baitoan}[\cite{Binh_Toan_7_tap_1}, VD23, p. 21]
	Tìm giá trị lớn nhất của biểu thức $B = 10 - 4|x - 2|$.
\end{baitoan}

\begin{baitoan}[\cite{Binh_Toan_7_tap_1}, VD24, p. 21]
	Tìm giá trị nhỏ nhất của biểu thức $C = \frac{6}{|x| - 3}$ với $x\in\mathbb{Z}$.
\end{baitoan}

\begin{baitoan}[\cite{Binh_Toan_7_tap_1}, VD25, p. 21]
	Tìm giá trị lớn nhất của biểu thức $A = x - |x|$.
\end{baitoan}

\begin{baitoan}[\cite{Binh_Toan_7_tap_1}, 70., p. 22]
	Tìm tất cả các số $a$ thỏa mãn 1 trong các điều kiện sau:
	
	(a) $a = |a|$;
	(b) $a < |a|$;
	(c) $a > |a|$;
	(d) $|a| = -a$;
	(e) $a\le|a|$.
	
\end{baitoan}

\begin{baitoan}[\cite{Binh_Toan_7_tap_1}, 71., p. 22]
	Bổ sung các điều kiện để các khẳng định sau là đúng:
	
	(a) $|a| = |b|\Rightarrow a = b$;
	(b) $a > b\Rightarrow|a| > |b|$.
	
\end{baitoan}

\begin{baitoan}[\cite{Binh_Toan_7_tap_1}, 72., p. 22]
	Cho $|x| = |y|$, $x < 0$, $y > 0$. Trong các khẳng định sau, khẳng định nào sai?
	
	(a) $x^2y > 0$;
	(b) $x + y = 0$;
	(c) $xy < 0$;
	(d) $\frac{1}{x} - \frac{1}{y} = 0$;
	(e) $\frac{x}{y} + 1 = 0$.
	
\end{baitoan}

\begin{baitoan}[\cite{Binh_Toan_7_tap_1}, 73., p. 22]
	Tìm giá trị của các biểu thức:
	
	(a) $A = 6x^3 - 3x^2 + 2|x| + 4$ với $x = -\frac{2}{3}$;
	(b) $B = 2|x| - 3|y|$ với $x = \frac{1}{2}$, $y = -3$;
	(c) $C = 2|x - 2| - 3|1 - x|$ với $x = 4$;
	(d) $D = \frac{5x^2 - 7x + 1}{3x - 1}$ với $|x| = \frac{1}{2}$.
	
\end{baitoan}

\begin{baitoan}[\cite{Binh_Toan_7_tap_1}, 74., p. 22]
	Rút gọn các biểu thức:
	
	(a) $|a| - a$;
	(b) $|a|a$;
	(c) $|a|:a$.
	
\end{baitoan}

\begin{baitoan}[\cite{Binh_Toan_7_tap_1}, 75., p. 22]
	Tìm $x$ trong các đẳng thức:
	
	(a) $|2x - 3| = 5$;
	(b) $|2x - 1| = |2x + 3|$;
	(c) $|x - 1| + 3x = 1$;
	(d) $|5x - 3| - x = 7$.
	
\end{baitoan}

\begin{baitoan}[\cite{Binh_Toan_7_tap_1}, 76., p. 23]
	Tìm các số $a$ \& $b$ thỏa mãn 1 trong các điều kiện sau:
	
	(a) $a + b = |a| + |b|$:
	(b) $a + b = |b| - |a|$.
	
\end{baitoan}

\begin{baitoan}[\cite{Binh_Toan_7_tap_1}, 77., p. 23]
	Có bao nhiêu cặp số nguyên $(x;y)$ thỏa mãn 1 trong các điều kiện sau:
	
	(a) $|x| + |y| = 20$;
	(b) $|x| + |y| < 20$.
	
\end{baitoan}

\begin{baitoan}[\cite{Binh_Toan_7_tap_1}, 78., p. 23]
	Điền vào chỗ chấm các dấu $\ge,\le,=$ để các khẳng định sau đúng với mọi $a,b$. Phát biểu mỗi khẳng định đó thành 1 tính chất \& chỉ rõ khi nào xảy ra dấu đẳng thức?
	
	(a) $|a + b|\ldots|a| + |b|$;
	(b) $|a - b|\ldots|a| - |b|$ với $|a|\ge|b|$;
	(c) $|ab|\ldots|a||b|$;
	(d) $\left|\frac{a}{b}\right|\ldots\frac{|a|}{|b|}$.
	
\end{baitoan}

\begin{baitoan}[\cite{Binh_Toan_7_tap_1}, 79., p. 23]
	Tìm giá trị nhỏ nhất của các biểu thức:
	
	(a) $A = 2|3x - 2| - 1$;
	(b) $B = 5|1 - 4x| - 1$;
	(c) $C = x^2 + 3|y - 2| - 1$;
	(d) $D = x + |x|$.
	
\end{baitoan}

\begin{baitoan}[\cite{Binh_Toan_7_tap_1}, 80., p. 23]
	Tìm giá trị lớn nhất của các biểu thức:
	
	(a) $A = 5 - |2x - 1|$;
	(b) $B = \frac{1}{|x - 2| + 3}$.
	
\end{baitoan}

\begin{baitoan}[\cite{Binh_Toan_7_tap_1}, 81., p. 23]
	Tìm giá trị lớn nhất của biểu thức: $C = \frac{x + 2}{|x|}$ với $x\in\mathbb{Z}$.
\end{baitoan}

%------------------------------------------------------------------------------%

\subsection{Làm Tròn Số \& Ước Lượng Kết Quả}

\begin{baitoan}[\cite{Tuyen_Toan_7}, VD27, p. 24]
	Làm tròn:
	
	(a) Số $348.62$ đến hàng chục;
	(b) Số $-67.(506)$ đến hàng phần mười \& hàng phần trăm.
	\hfill{\sf Ans:} (a) $350$. (b) $-67.5$, $-67.51$.
\end{baitoan}

\begin{baitoan}[\cite{Tuyen_Toan_7}, VD28, p. 24]
	Làm tròn:
	
	(a) Số $924578$ với độ chính xác $500$;
	(b) Số $56.9827$ với độ chính xác $0.5$ \& độ chính xác $0.005$.
	\hfill{\sf Ans:} (a) $925000$. (b) $57$, $56.98$.
\end{baitoan}

\begin{baitoan}[\cite{Tuyen_Toan_7}, VD29, p. 24]
	1 khu đất hình chữ nhật có kích thước {\rm7.56 m} \& {\rm5.173 m}. Tính diện tích khu đất đó bằng 2 cách. Cách 1: Làm tròn số trước rồi mới thực hiện các phép tính sau (làm tròn đến hàng đơn vị). Cách 2: Thực hiện các phép tính trước rồi làm tròn kết quả sau (làm tròn đến hàng đơn vị).\hfill{\sf Ans:} $\approx\rm 8\cdot 5 = 40m^2$, $\rm39.10788\approx 39m^2$.
\end{baitoan}

\begin{baitoan}[\cite{Tuyen_Toan_7}, 91., p. 24]
	Đầu năm $2021$ dân số nước ta nếu làm tròn đến hàng triệu thì được $98000000$ người. Hỏi dân số lúc đó:
	
	(a) Nhiều nhất là tới bao nhiêu người?
	(b) Ít nhất là có bao nhiêu người? 
	\hfill{\sf Ans:} (a) $98499999$, $97500000$.
\end{baitoan}

\begin{baitoan}[\cite{Tuyen_Toan_7}, 92., p. 24]
	1 trận đấu bóng đá có $198 792$ khán giả. Để dễ nhớ người ta nói trên trận đấu này có khoảng $200000$ khán giả. Hỏi số liệu đó đã được làm tròn đến hàng nào?\hfill{\sf Ans:} Trăm, chục, hoặc nghìn.
\end{baitoan}

\begin{baitoan}[\cite{Tuyen_Toan_7}, 93., p. 24]
	Cho số $\pi = 3.141592\ldots$. Làm tròn số đó với độ chính xác lần lượt là $0.5$, $0.005$, $0.00005$.\hfill{\sf Ans:} $3$, $3.14$, $3.1416$.
\end{baitoan}

\begin{baitoan}[\cite{Tuyen_Toan_7}, 94., p. 24]
	Thực hiện phép chia $19:24$ rồi làm tròn kết quả với độ chính xác $0.05$.\\\mbox{}\hfill{\sf Ans:} $0.791(6)\approx0.8$.
\end{baitoan}

\begin{baitoan}[\cite{Tuyen_Toan_7}, 95., p. 24]
	Dùng máy  tính để tính $\sqrt{148} + \sqrt{65}$ rồi làm tròn kết quả với độ chính xác $0.5$.\\\mbox{}\hfill{\sf Ans:} $20.22778281\approx20$.
\end{baitoan}

\begin{baitoan}[\cite{Tuyen_Toan_7}, 96., p. 25]
	Áp dụng quy tắc làm tròn số để ước lượng giá trị của biểu thức sau: $A = \frac{53.7\cdot 12.8}{24.56}$.\\\mbox{}\hfill{\sf Ans:} $A\approx 28$.
\end{baitoan}

\begin{baitoan}[\cite{Tuyen_Toan_7}, 97., p. 25]
	Trong học kỳ vừa qua điểm kiểm tra môn Toán của Bình như sau: Điểm kiểm tra thường xuyên (hệ số 1): $8$, $9$, $8$, $9$. Điểm kiểm tra giữa kỳ (hệ số 2): $9$. Điểm kiểm tra cuối kỳ (hệ số 3): $8$. Tính điểm trung bình môn Toán của Bình (làm tròn kết quả đến hàng phần mười).\hfill{\sf Ans:} $8.(4)\approx8.4$.
\end{baitoan}

\begin{baitoan}[\cite{Tuyen_Toan_7}, 98., p. 25]
	Để tính số năm tăng gấp đôi tổng sản phẩm quốc nội (GDP) của 1 quốc gia ta có thể dùng công thức $n = \frac{72}{g}$, trong đó: $g\%$ là {\rm tốc độ tăng trưởng GDP} trong giai đoạn đang xét; $n$ là số năm để tăng gấp đôi GDP. Hỏi: (a) Nếu tốc độ tăng trưởng GDP trong giai đoạn hiện nay của Việt Nam khoảng $7.1\%$ thì sau bao nhiêu năm nữa GDP của nước ta tăng gấp đôi (làm tròn đến hàng đơn vị)? (b) Nếu muốn sau $7$ năm, GDP tăng gấp đôi thì tốc độ tăng trưởng hằng năm là bao nhiêu $\%$ (làm tròn đến hàng phần mười)?\hfill{\sf Ans:} (a) $10$. (b) $10.3\%$.
\end{baitoan}

%------------------------------------------------------------------------------%

\section{Tỷ Lệ Thức}

\begin{baitoan}[\cite{Binh_boi_duong_Toan_7_tap_1}, H1, p. 32]
	{\rm Đ{\tt/}S?} Nếu sai, sửa cho đúng. (a) $\dfrac{0.3}{1.5} = \dfrac{1}{5}$ là 1 đẳng thức giữa 2 phân số. (b) $0.3:1.5 = 1:5$ là 1 tỷ lệ thức. (c) $\dfrac{5}{0.3} = \dfrac{1}{1.5}$ là 1 tỷ lệ thức. (d) $\dfrac{0.3}{1.5} = \frac{1}{5} = \dfrac{1\frac{1}{2}}{7\frac{1}{2}}$ là 1 dãy tỷ số bằng nhau.
\end{baitoan}

\begin{baitoan}[\cite{Binh_boi_duong_Toan_7_tap_1}, H2, p. 32]
	Điền số: (a) $\dfrac{\ldots}{3} = \dfrac{4}{12} = \dfrac{5}{\ldots}$. (b) $\dfrac{0.1}{\ldots} = \dfrac{\ldots}{14} = \dfrac{0.3}{6}$.
\end{baitoan}

\begin{baitoan}[\cite{Binh_boi_duong_Toan_7_tap_1}, H3, p. 32]
	Số nào không thể thêm vào tập hợp $A = \{3,6,9\}$ để tạo ra 1 tỷ lệ thức? {\sf A.} $2$. {\sf B.} $4.5$. {\sf C.} $12$. {\sf D.} $18$.
\end{baitoan}

\begin{baitoan}[\cite{Binh_boi_duong_Toan_7_tap_1}, VD1, p. 32]
	Thay tỷ số giữa 2 số $2.25$ \& $2\dfrac{5}{8}$ bằng tỷ số giữa các số nguyên.
\end{baitoan}

\begin{baitoan}[\cite{Binh_boi_duong_Toan_7_tap_1}, VD2, p. 32]
	{\rm Đ{\tt/}S?} Nếu sai, sửa cho đúng. Để kiểm tra 4 số $-0.2,0.1,0.2,-0.1$ có tạo thành 1 tỷ lệ thức không, A thực hiện 3 bước: Bước 1. Xếp 4 số đã cho theo thứ tự tăng dần: $-0.2 < -0.1 < 0.1 < 0.2$. Bước 2. So sánh tích của số nhỏ nhất \& lớn nhất với tích 2 số ở giữa. $-0.2\cdot0.2\ne-0.1\cdot0.1$ (vì $-0.04\ne-0.01$). Bước 3. Vậy 4 số trên không lập thành 1 tỷ lệ thức.
\end{baitoan}

\begin{baitoan}[\cite{Binh_boi_duong_Toan_7_tap_1}, VD3, p. 33]
	Tìm $x\in\mathbb{R}$ thỏa: (a) $7.5:x = 2.25:4\frac{1}{6}$. (b) $49x:10.5 = 3\frac{3}{4}:3\frac{1}{8}$. (c) $0.06:x = x:24$. (d) $(x - 1)^3:25 = -5:8$.
\end{baitoan}

\begin{baitoan}[\cite{Binh_boi_duong_Toan_7_tap_1}, VD4, p. 33]
	Tìm $x,y\in\mathbb{R}$ thỏa: (a) $x:y = 20:9,x - y = -22$. (b) $3x = 4y,x + 2y = 35$. (c) $x:2 = 2y:3,xy = 27$.
\end{baitoan}

\begin{baitoan}[\cite{Binh_boi_duong_Toan_7_tap_1}, VD5, p. 33]
	Tìm số hạng thứ 4 để lập thành 1 tỷ lệ thức với 3 số: (a) $4,8,16$. (b) $-3,-6,9$. (c) $2^2,2^4,2^6$.
\end{baitoan}

\begin{baitoan}[\cite{Binh_boi_duong_Toan_7_tap_1}, VD6, p. 34]
	Tính giá trị của $k$ biết $k = \dfrac{\overline{ab}}{\overline{abc}} = \dfrac{\overline{bc}}{\overline{abc}} = \dfrac{\overline{ca}}{\overline{abc}}$.
\end{baitoan}

\begin{baitoan}[\cite{Binh_boi_duong_Toan_7_tap_1}, VD7, p. 34]
	Tìm số hữu tỷ để khi thêm số hữu tỷ đó vào cả tử \& mẫu của phân số $\frac{13}{29}$ sẽ được 1 phân số mới bằng $\frac{1}{3}$.
\end{baitoan}

\begin{baitoan}[\cite{Binh_boi_duong_Toan_7_tap_1}, VD8, p. 34]
	Tìm số $\overline{ab}$ biết $\dfrac{\overline{ab}}{31} = \dfrac{a + b}{4},\overline{ab} - \overline{ba} = 36$.
\end{baitoan}

\begin{baitoan}[\cite{Binh_boi_duong_Toan_7_tap_1}, 5.1., p. 34]
	Thay tỷ số bằng tỷ số giữa 2 số nguyên: (a) $3.5:5.04$. (b) $1\frac{19}{21}:4\frac{2}{7}$. (c) $1\frac{21}{25}:0.23$.
\end{baitoan}

\begin{baitoan}[\cite{Binh_boi_duong_Toan_7_tap_1}, 5.2., p. 34]
	Lập được tỷ lệ thức từ từng nhóm 4 số: (a) $-1,-3,-9,27$. (b) $0.4,0.04,0.004,0.0004$. (c) $-1,-\frac{1}{2},-\frac{1}{3},-\frac{1}{6}$. (d) $3^{-3},3^{-5},3^{-7},3^{-11}$.
\end{baitoan}

\begin{baitoan}[\cite{Binh_boi_duong_Toan_7_tap_1}, 5.3., p. 35]
	Tìm $x$ trong tỷ lệ thức: (a) $6:x = 6.5:(-29.25)$. (b) $14\frac{2}{3}:\left(-80\frac{2}{3}\right) = 0.5x:35\frac{3}{4}$. (c) $4:x = x:0.16$. (d) $(1 - x)^3:(-0.5625) = 0.525:0.7$.
\end{baitoan}

\begin{baitoan}[\cite{Binh_boi_duong_Toan_7_tap_1}, 5.4., p. 35]
	Tìm $a,b,c\in\mathbb{Q}$: (a) $5a - 3b - 3c = -536,\dfrac{a}{4} = \dfrac{b}{6},\dfrac{b}{5} = \dfrac{c}{8}$. (b) $3a - 5b + 7c = 86,\dfrac{a + 3}{5} = \dfrac{b - 2}{3} = \dfrac{c - 1}{7}$. (c) $5a = 8b = 3c,a - 2b + c = 34$. (d) $3a = 7b,a^2 - b^2 = 160$.
\end{baitoan}

\begin{baitoan}[\cite{Binh_boi_duong_Toan_7_tap_1}, 5.5., p. 35]
	Tìm $a,b,c\in\mathbb{Q}$: (a) $15a = 10b = 6c,abc = -1920$. (b) $a^2 + 3b^2 - 2c^2 = -16,\dfrac{a}{2} = \dfrac{b}{3} = \dfrac{c}{4}$. (c) $a^3 + b^2 + c^3 = 792,\dfrac{a}{2} = \dfrac{b}{3} = \dfrac{c}{4}$.
\end{baitoan}

\begin{baitoan}[\cite{Binh_boi_duong_Toan_7_tap_1}, 5.6., p. 35]
	Cho tỷ lệ thức $\dfrac{\overline{ab}}{\overline{bc}} = \dfrac{a}{c},c\ne0$. Chứng minh $\dfrac{a\underbrace{b\ldots b}_{n-1}}{\underbrace{b\ldots b}_{n-1}c} = \dfrac{a}{c}$, $\forall n\in\mathbb{N}^\star$.
\end{baitoan}

\begin{baitoan}[\cite{Binh_boi_duong_Toan_7_tap_1}, 5.7., p. 35]
	Cho $abcd\ne0,b^2 = ca,c^2 = bd$. Chứng minh tỷ lệ thức $\dfrac{a^3 + b^3 + c^3}{b^3 + c^3 + d^3} = \frac{a}{d}$.
\end{baitoan}

\begin{baitoan}[\cite{Binh_boi_duong_Toan_7_tap_1}, 5.8., p. 35]
	Tìm $x,y\in\mathbb{Q}$ thỏa $\dfrac{2x + 1}{5} = \dfrac{3y - 2}{7} = \dfrac{2x + 3y - 1}{6x}$.
\end{baitoan}

\begin{baitoan}[\cite{Binh_boi_duong_Toan_7_tap_1}, 5.9., p. 35]
	Chứng minh 4 số $a,b,c,d\in\mathbb{R}$ lập thành 1 tỷ lệ thức nếu $(a + b + c + d)(a - b - c - d) = (a - b + c - d)(a + b - c - d)$.
\end{baitoan}

\begin{baitoan}[\cite{Binh_boi_duong_Toan_7_tap_1}, 5.10., p. 35]
	Tìm $x,y\in\mathbb{Q}$ thỏa $(x - 20):(x - 10) = (x + 40):(x + 70)$.
\end{baitoan}

\begin{baitoan}[\cite{Binh_boi_duong_Toan_7_tap_1}, 5.11., p. 35]
	Chứng minh nếu $\dfrac{a}{x} = \dfrac{b}{y} = \dfrac{c}{z} = m$ thì $\dfrac{ak^2 + bk + c}{xk^2 + yk + z} = m$, $\forall k\in\mathbb{N}$.
\end{baitoan}

\begin{baitoan}[\cite{Binh_boi_duong_Toan_7_tap_1}, p. 35]
	Cho tỷ lệ thức $\dfrac{a}{b} = \dfrac{c}{d}$. Chứng minh tỷ lệ thức $\dfrac{a}{a - b} = \dfrac{c}{c - d}$. Tìm các tỷ lệ thức tương tự.
\end{baitoan}

\begin{baitoan}[\cite{Tuyen_Toan_7}, VD30, p. 26]
	Tìm $x\in\mathbb{R}$ biết: $\frac{x + 6}{32} = \frac{8}{x + 6}$.\hfill{\sf Ans:} $-22$, $10$.
\end{baitoan}

\begin{baitoan}[\cite{Tuyen_Toan_7}, VD31, p. 26]
	Tìm $x,y,z\in\mathbb{R}$ biết $\frac{x}{y} = \frac{10}{9}$, $\frac{y}{z} = \frac{3}{4}$ \& $x - y + z = 78$.\\\mbox{}\hfill{\sf Ans:} $(x,y,z) = (60,54,72)$.
\end{baitoan}

\begin{baitoan}[\cite{Tuyen_Toan_7}, VD32, p. 26]
	Cho 3 số $a,b,c$ sao cho $a + b + c\ne 0$. Biết $\frac{b + c}{a} = \frac{c + a}{b} = \frac{a + b}{c} = k$. Tính giá trị của $k$.\hfill{\sf Ans:} $2$.
\end{baitoan}

\begin{baitoan}[\cite{Tuyen_Toan_7}, 99., p. 27]
	Lập các tỷ lệ thức có thể được từ 4 số sau: $3$, $-2$, $-9$, $6$.\\\mbox{}\hfill{\sf Ans:} $\frac{3}{-2} = \frac{-9}{6}$, $\frac{6}{-2} = \frac{-9}{3}$, $\frac{3}{-9} = \frac{-2}{6}$, $\frac{6}{-9} = \frac{-2}{3}$.
\end{baitoan}

\begin{baitoan}[\cite{Tuyen_Toan_7}, 100., p. 27]
	Tìm $x$ trong mỗi tỷ lệ thức sau:
	
	(a) $\frac{x - 3}{x + 5} = \frac{5}{7}$;
	(b) $\frac{x + 4}{20} = \frac{5}{x + 4}$;
	(c) $\frac{x + 1}{x^2} = \frac{1}{x}$.
	\\\mbox{}\hfill{\sf Ans:} (a) $23$. (b) $6$, $-14$. (c) $\overline{\exists}$.
\end{baitoan}

\begin{baitoan}[\cite{Tuyen_Toan_7}, 101., p. 27]
	Chứng minh nếu $\frac{a}{b} = \frac{c}{d}$ thì $\frac{a^2 + b^2}{c^2 + d^2} = \frac{ab}{cd}$.
\end{baitoan}

\begin{baitoan}[\cite{Tuyen_Toan_7}, 102., p. 27]
	Cho $P = \frac{x + 2y - 3z}{x - 2y + 3z}$. Tính giá trị của $P$ biết các số $x,y,z$ tỷ lệ với các số $5,4,3$.\\\mbox{}\hfill{\sf Ans:} $\frac{2}{3}$.
\end{baitoan}

\begin{baitoan}[\cite{Tuyen_Toan_7}, 103., p. 27]
	Cho các số $A,B,C$ tỷ lệ với các số $a,b,c$. Chứng minh giá trị của biểu thức $Q = \frac{Ax + By + C}{ax + by + c}$ không phụ thuộc vào giá trị của $x$ \& $y$.
\end{baitoan}

\begin{baitoan}[\cite{Tuyen_Toan_7}, 104., p. 27]
	Tìm các số $x,y,z$ biết:
	
	(a) $\frac{x}{4} = \frac{y}{3} = \frac{z}{9}$ \& $x - 3y + 4z = 62$;
	(b) $\frac{x}{y} = \frac{9}{7}$, $\frac{y}{z} = \frac{7}{3}$, \& $x - y + z = -15$;
	(c) $\frac{x}{y} = \frac{7}{20}$, $\frac{y}{z} = \frac{5}{8}$, \& $2x + 5y - 2z = 100$.
	\\\mbox{}\hfill{\sf Ans:} (a) $(x,y,z) = (8,6,18)$. (b) $(x,y,z) = (-27,-21,-9)$. (c) $(x,y,z) = (14,40,64)$.
\end{baitoan}

\begin{baitoan}[\cite{Tuyen_Toan_7}, 105., p. 27]
	Tìm các số $x,y,z$ biết:
	
	(a) $5x = 8y = 20z$ \& $x - y - z = 3$;
	(b) $\frac{6}{11}x = \frac{9}{2}y = \frac{18}{5}z$ \& $-x + y + z = -120$.
	\hfill{\sf Ans:} $(x,y,z) = (165,20,25)$.
\end{baitoan}

\begin{baitoan}[\cite{Tuyen_Toan_7}, 106., p. 27]
	1 hộp đựng $70$ quả bóng. Tỷ số giữa số bóng đỏ \& số bóng trắng là $2:3$. Tỷ số giữa số bóng trắng \& số bóng xanh là $3:5$. Tính số bóng đỏ \& số bóng xanh.\hfill{\sf Ans:} $14$, $35$.
\end{baitoan}

\begin{baitoan}[\cite{Tuyen_Toan_7}, 107., p. 27]
	3 kho có tất cả $710$ tấn thóc. Sau khi chuyển $\frac{1}{5}$ số thóc ở kho I, $\frac{1}{6}$ số thóc ở kho II \& $\frac{1}{11}$ số thóc ở kho III thì số thóc còn lại ở 3 kho bằng nhau. Hỏi lúc đầu mỗi kho có bao nhiêu tấn thóc?\ \hfill{\sf Ans:} $250$, $240$, $220$.
\end{baitoan}

\begin{baitoan}[\cite{Tuyen_Toan_7}, 108., p. 28]
	Chia số $x$ thành 3 phần theo thứ tự tỷ lệ với $2,3,4$ rồi lại chia $x$ theo thứ tự tỷ lệ với $3,5,7$ thì có 1 phần giảm đi $1$. Tìm $x$.\hfill{\sf Ans:} $45$.
\end{baitoan}

\begin{baitoan}[\cite{Tuyen_Toan_7}, 109., p. 28]
	1 khu vườn hình chữ nhật có diện tích $\rm300m^2$, 2 cạnh tỷ lệ với $4$ \& $3$. Tính chiều dài, chiều rộng của khu vườn.\hfill{\sf Ans:} {\rm20 m, 15 m}.
\end{baitoan}

\begin{baitoan}[\cite{Tuyen_Toan_7}, 110., p. 28]
	Tìm $x,y,z$ biết: $\frac{x}{12} = \frac{y}{9} = \frac{z}{5}$ \& $xyz = 20$.\hfill{\sf Ans:} $(x,y,z) = \left(4,3,\frac{5}{3}\right)$.
\end{baitoan}

\begin{baitoan}[\cite{Tuyen_Toan_7}, 111., p. 28]
	Tìm $x,y,z$ biết: $\frac{x}{5} = \frac{y}{7} = \frac{z}{3}$ \& $x^2 + y^2 + z^2 = 385$.\\\mbox{}\hfill{\sf Ans:} $(x,y,z) = \{(15,21,9),(-15,-21,-9)\}$.
\end{baitoan}

\begin{baitoan}[\cite{Tuyen_Toan_7}, 112., p. 28]
	Tìm 2 phân số tối giản biết hiệu của chúng là $\frac{3}{196}$, các tử số tỷ lệ với $3$ \& $5$; các mẫu số tỷ lệ với $4$ \& $7$.\hfill{\sf Ans:} $(x,y) = \left(\frac{9}{28},\frac{15}{49}\right)$.
\end{baitoan}

\begin{baitoan}[\cite{Tuyen_Toan_7}, 113., p. 28]
	Tìm $x,y,z$ biết: $\frac{12x - 15y}{7} = \frac{20z - 12x}{9} = \frac{15y - 20z}{11}$ \& $x + y + z = 48$.\\\mbox{}\hfill{\sf Ans:} $(x,y,z) = (20,16,12)$.
\end{baitoan}

\begin{baitoan}[\cite{Tuyen_Toan_7}, 114., p. 28]
	Cho dãy tỷ số bằng nhau: $\frac{2a + b + c + d}{a} = \frac{a + 2b + c + d}{b} = \frac{a + b + 2c + d}{c} = \frac{a + b + c + 2d}{d}$. Tính giá trị của biểu thức $M = \frac{a + b}{c + d} + \frac{b + c}{d + a} + \frac{c + d}{a + b} + \frac{d + a}{b + c}$.\hfill{\sf Ans:} $-4$.
\end{baitoan}

\begin{baitoan}[\cite{Binh_Toan_7_tap_1}, VD26, p.24]
	Cho 3 số $6,8,24$.
	
	(a) Tìm số $x$, sao cho $x$ cùng với 3 số trên lập thành 1 tỷ lệ thức.
	(b) Có thể lập được tất cả bao nhiêu tỷ lệ thức?
	
\end{baitoan}

\begin{proof}[Giải]
	
	(a) Trong 3 số $6,8,24$, có  cách chọn ra tích của 2 trong 3 số ấy. Với mỗi tích, có 1 cách lập đẳng thức với tích của số còn lại \& số $x$. Có: $6\cdot8 = 24x\Leftrightarrow x = 2$, $6\cdot24 = 8x\Leftrightarrow x = 18$, $8\cdot24 = 6x\Leftrightarrow x = 32$.
	(b) Với tích $6\cdot 8 = 24\cdot 2$, lập được 4 tỷ lệ thức: $\frac{6}{24} = \frac{2}{8}$, $\frac{6}{2} = \frac{24}{8}$, $\frac{8}{24} = \frac{2}{6}$, $\frac{8}{2} = \frac{24}{6}$. Tương tự với các tích $6\cdot24 = 8\cdot 18$ \& $8\cdot24 = 6\cdot32$. Tất cả có $4\cdot3 = 12$ tỷ lệ thức.
	
\end{proof}

\begin{baitoan}[\cite{Binh_Toan_7_tap_1}, \S4, VD7]
	Cho tỷ lệ thức $\frac{a}{b} = \frac{c}{d}$. Chứng minh: $\frac{a}{a - b} = \frac{c}{c - d}$ (giả thiết $a\ne b$, $c\ne d$ \& mỗi số $a,b,c,d\ne 0$).
\end{baitoan}

\begin{baitoan}[\cite{Binh_Toan_7_tap_1}, VD8, \S4]
	Cho tỷ lệ thức $\frac{x}{2} = \frac{y}{5}$. Biết $xy = 90$. Tính $x$ \& $y$.
\end{baitoan}

\begin{baitoan}[\cite{Binh_Toan_7_tap_1}, VD29, p. 25]
	Cho dãy số $10,11,\ldots,n$. Tìm số $n$ nhỏ nhất để trong dãy đó ta chọn được 4 số khác nhau lập thành 1 tỷ lệ thức.
\end{baitoan}

\begin{baitoan}[\cite{Binh_Toan_7_tap_1}, \S4, 53.]
	Tìm $x\in\mathbb{Q}$ trong tỷ lệ thức:
	
	(a) $0.4:x = x:0.9$.
	(b) $13\frac{1}{3}:1\frac{1}{3} = 26:(2x - 1)$.
	(c) $0.2:1\frac{1}{5} = \frac{2}{3}:(6x + 7)$.
	(d) $\frac{37 - x}{x + 13} = \frac{3}{7}$.
	
\end{baitoan}

\begin{baitoan}[\cite{Binh_Toan_7_tap_1}, \S4, 54.]
	Cho tỷ lệ thức $\frac{3x - y}{x + y} = \frac{3}{4}$. Tìm giá trị của tỷ số $\frac{x}{y}$.
\end{baitoan}

\begin{baitoan}[\cite{Binh_Toan_7_tap_1}, \S4, 55.]
	Cho tỷ lệ thức $\frac{a}{b} = \frac{c}{d}$. Chứng minh các tỷ lệ thức sau (giả thiết các tỷ lệ thức đều có nghĩa):
	
	(a) $\frac{2a + 3b}{2a - 3b} = \frac{2c + 3d}{2c - 3d}$.
	(b) $\frac{ab}{cd} = \frac{a^2 - b^2}{c^2 - d^2}$.
	(c) $\left(\frac{a + b}{c + d}\right)^2 = \frac{a^2 + b^2}{c^2 + d^2}$.
	
\end{baitoan}

\begin{baitoan}[\cite{Binh_Toan_7_tap_1}, \S4, 56.]
	Chứng minh: ta có tỷ lệ thức $\frac{a}{b} = \frac{c}{d}$ nếu có 1 trong các đẳng thức sau (giả thiết các tỷ lệ thức đều có nghĩa):
	
	(a) $\frac{a + b}{a - b} = \frac{c + d}{c - d}$.
	(b) $(a + b + c + d)(a - b - c + d) = (a - b + c - d)(a + b - c - d)$.
	
\end{baitoan}

\begin{baitoan}[\cite{Binh_Toan_7_tap_1}, \S4, 57.]
	Cho tỷ lệ thức $\frac{a + b + c}{a + b - c} = \frac{a - b + c}{a - b - c}$ trong đó $b\ne 0$. Chứng minh $c = 0$.
\end{baitoan}

\begin{baitoan}[\cite{Binh_Toan_7_tap_1}, \S4, 58.]
	Cho tỷ lệ thức $\frac{a + b}{b + c} = \frac{c + d}{d + a}$. Chứng minh: $a = c$ hoặc $a + b + c + d = 0$.
\end{baitoan}

\begin{baitoan}[\cite{Binh_Toan_7_tap_1}, \S4, 59.]
	Có thể lập được 1 tỷ lệ thức từ 4 trong các số sau không (mỗi số chỉ chọn 1 lần)? Nếu có thì lập được bao nhiêu tỷ lệ thức?
	
	(a) $3,4,5,6,7$.
	(b) $1,2,4,8,16$.
	(c) $1,3,9,27,81,243$.
	
\end{baitoan}

\begin{baitoan}[\cite{Binh_Toan_7_tap_1}, \S4, 60.]
	Cho 4 số $2,4,8,16$. Tìm $x\in\mathbb{Q}$ cùng với 3 trong 4 số trên lập được thành 1 tỷ lệ thức.
\end{baitoan}

%------------------------------------------------------------------------------%

\subsection{Chứng Minh Tỷ Lệ Thức}

\begin{baitoan}[\cite{Tuyen_Toan_7}, VD33, p. 28]
	Cho tỷ lệ thức $\frac{a}{b} = \frac{c}{d}\ne 1$ với $a,b,c,d\ne 0$. Chứng minh $\frac{a - b}{a} = \frac{c - d}{c}$.
\end{baitoan}

\begin{baitoan}[\cite{Tuyen_Toan_7}, VD34, p. 29]
	Cho $a = b + c$ \& $c = \frac{bd}{b - d}$, $b\ne 0$, $d\ne 0$. Chứng minh $\frac{a}{b} = \frac{c}{d}$.
\end{baitoan}

\begin{baitoan}[\cite{Tuyen_Toan_7}, 115., p. 29]
	Cho $\frac{a}{k} = \frac{x}{a}$, $\frac{b}{k} = \frac{y}{b}$ với $y\ne 0$. Chứng minh $\frac{a^2}{b^2} = \frac{x}{y}$.
\end{baitoan}

\begin{baitoan}[\cite{Tuyen_Toan_7}, 116., p. 29]
	Cho tỷ lệ thức $\frac{x}{y} = \frac{a}{b}$ \& $c = x + y$. Chứng minh $\frac{1}{x} = \frac{a + b}{ac}$ (giả thiết các tỷ số đều có nghĩa).
\end{baitoan}

\begin{baitoan}[\cite{Tuyen_Toan_7}, 117., p. 30]
	Cho tỷ lệ thức $\frac{a}{b} = \frac{c}{d}\ne\pm 1$ \& $c\ne 0$. Chứng minh:
	
	(a) $\left(\frac{a - b}{c - d}\right)^2 = \frac{ab}{cd}$;
	(b) $\left(\frac{a + b}{c + d}\right)^3 = \frac{a^3 - b^3}{c^3 - d^3}$.
	
\end{baitoan}

\begin{baitoan}[\cite{Tuyen_Toan_7}, 118., p. 30]
	Cho tỷ lệ thức $\frac{a}{b} = \frac{c}{d}$, $c\ne\pm\frac{3}{5}d$. Chứng minh $\frac{5a + 3b}{5c + 3d} = \frac{5a - 3b}{5c - 3d}$.
\end{baitoan}

\begin{baitoan}[\cite{Tuyen_Toan_7}, 119., p. 30]
	Cho $b^2 = ac$, $c^2 = bd$ với $b,c,d\ne 0$, $b + c\ne d$, \& $b^3 + c^3\ne d^3$. Chứng minh $\frac{a^3 + b^3 - c^3}{b^3 + c^3 - d^3} = \left(\frac{a + b - c}{b + c - d}\right)^3$.
\end{baitoan}

\begin{baitoan}[\cite{Tuyen_Toan_7}, 120., p. 30]
	Chứng minh nếu $2(x + y) = 5(y + z) = 3(z + x)$ thì $\frac{x - y}{4} = \frac{y - z}{5}$.
\end{baitoan}

\begin{baitoan}[\cite{Tuyen_Toan_7}, 121., p. 30]
	Cho $b^2 = ac$, $a,b,c\ne 0$. Chứng minh $\frac{a^2 + b^2}{b^2 + c^2} = \frac{a}{c}$.
\end{baitoan}

\begin{baitoan}[\cite{Tuyen_Toan_7}, 122., p. 30]
	Cho $\frac{a + b}{a - b} = \frac{c + a}{c - a}$. Chứng minh nếu 3 số $a,b,c$ đều khác $0$ thì từ 3 số này (có 1 số được dùng $2$ lần) có thể lập thành 1 tỷ lệ thức.
\end{baitoan}

\begin{baitoan}[\cite{Tuyen_Toan_7}, 123., p. 30]
	Chứng minh nếu $a^2 = bc$, $a\ne b$, $a\ne c$ thì $\frac{a + b}{a - b} = \frac{c + a}{c - a}$.
\end{baitoan}

\begin{baitoan}[\cite{Tuyen_Toan_7}, 124., p. 30]
	Cho biểu thức $M = \frac{ax + by}{cx + dy}$, $c,d\ne 0$. Chứng minh nếu giá trị của biểu thức $M$ không phụ thuộc vào giá trị của $x$ \& $y$ thì 4 số $a,b,c,d$ lập thành 1 tỷ lệ thức.
\end{baitoan}

\begin{baitoan}[\cite{Tuyen_Toan_7}, 125., p. 30]
	Cho $\frac{a}{x + 2y + z} = \frac{b}{2x + y - z} = \frac{c}{4x - 4y + z}$. Chứng minh $\frac{x}{a + 2b + c} = \frac{y}{2a + b - c} = \frac{z}{4a - 4b + c}$ với $xyz\ne 0$ \& các mẫu số đều khác $0$.
\end{baitoan}

%------------------------------------------------------------------------------%

\section{Tính Chất Dãy Tỷ Số Bằng Nhau}

\begin{baitoan}[\cite{Binh_Toan_7_tap_1}, \S5, VD9]
	Tìm các số $x,y,z$ biết $\frac{x}{3} = \frac{y}{4}$, $\frac{y}{5} = \frac{z}{7}$ \& $2x + 3y - z = 186$.
\end{baitoan}

\begin{baitoan}[\cite{Binh_Toan_7_tap_1}, \S5, VD10]
	Tìm các số $x,y,z$ biết $\frac{y + z + 1}{x} = \frac{x + z + 2}{y} = \frac{x + y - 3}{z} = \frac{1}{x + y + z}$.
\end{baitoan}

\begin{baitoan}[\cite{Binh_Toan_7_tap_1}, \S5, 61.]
	Tìm các số $x,y,z$ biết:
	
	(a) $\frac{x}{10} = \frac{y}{10} = \frac{z}{21}$ \& $5x + y - 2z = 28$.
	(b) $3x = 2y$, $7y = 5z$, $x - y + z = 32$.
	(c) $\frac{x}{3} = \frac{y}{4}$, $\frac{y}{3} = \frac{z}{5}$, $2x - 3y + z = 6$.
	(d) $\frac{2x}{3} = \frac{3y}{4} = \frac{4z}{5}$ \& $x + y + z = 49$.
	(e) $\frac{x - 1}{2} = \frac{y - 2}{3} = \frac{z - 3}{4}$ \& $2x + 3y - z = 50$.
	(f) $\frac{x}{2} = \frac{y}{3} = \frac{z}{5}$ \& $xyz = 810$.
	
\end{baitoan}

\begin{baitoan}[\cite{Binh_Toan_7_tap_1}, \S5, 62.]
	Tìm $x$ biết $\frac{1 + 2y}{18} = \frac{1 + 4y}{24} = \frac{1 + 6y}{6x}$.
\end{baitoan}

\begin{baitoan}[\cite{Binh_Toan_7_tap_1}, \S5, 63.]
	Tìm phân số $\frac{a}{b}$ biết nếu cộng thêm cùng 1 số khác $0$ vào tử \& mẫu thì giá trị của phân số đó không đổi.
\end{baitoan}

\begin{baitoan}[\cite{Binh_Toan_7_tap_1}, \S5, 64.]
	Cho $\frac{a}{b} = \frac{b}{c} = \frac{c}{d}$. Chứng minh $\left(\frac{a + b + c}{b + c + d}\right)^3 = \frac{a}{d}$.
\end{baitoan}

\begin{baitoan}[\cite{Binh_Toan_7_tap_1}, \S5, 65.]
	Cho $\frac{a}{b} = \frac{b}{c} = \frac{c}{a}$. Chứng minh $a = b = c$.
\end{baitoan}

\begin{baitoan}[\cite{Binh_Toan_7_tap_1}, \S5, 66.]
	Vì sao tỷ số của 2 hỗn số dạng $a\frac{1}{b}$ \& $b\frac{1}{a}$ luôn luôn bằng phân số $\frac{a}{b}$?
\end{baitoan}

\begin{baitoan}[\cite{Binh_Toan_7_tap_1}, \S5, 67.]
	Cho 3 tỷ số bằng nhau là $\frac{a}{b + c},\frac{b}{c + a},\frac{c}{a + b}$. Tìm giá trị của mỗi tỷ số đó.
\end{baitoan}

%------------------------------------------------------------------------------%

\section{Đại Lượng Tỷ Lệ Thuận}

\begin{baitoan}[\cite{Binh_boi_duong_Toan_7_tap_1}, H1, p. 37]
	Điền số thích hợp: (a) Chu vi 1 hình vuông tỷ lệ thuận với cạnh hình vuông, hệ số tỷ lệ là $\ldots$. (b) Số hàng mua được tỷ lệ thuận với $\ldots$ nếu giá hàng không đổi. (c) Chu vi 1 hình tròn tỷ lệ thuận với $\ldots$ có hệ số tỷ lệ là $\pi$. (d) Diện tích 1 tam giác có đáy là hằng số $a > 0$ tỷ lệ thuận với đường cao tương ứng theo hệ số tỷ lệ $\ldots$
\end{baitoan}

\begin{baitoan}[\cite{Binh_boi_duong_Toan_7_tap_1}, H2, p. 37]
	{\rm Đ{\tt/}S?} Nếu sai, sửa cho đúng. (a) Quãng đường tỷ lệ thuận với vận tốc nếu thời gian không đổi. (b) Nếu vận tốc không đổi thì quãng đường \& thời gian là 2 đại lượng tỷ lệ thuận. (c) Trên cùng quãng đường, vận tốc \& thời gian là 2 đại lượng tỷ lệ thuận. (d) Trên cùng quãng đường, vận tốc \& thời gian là 2 đại lượng không tỷ lệ thuận.
\end{baitoan}

\begin{baitoan}[\cite{Binh_boi_duong_Toan_7_tap_1}, H3, pp. 37--38]
	{\rm Đ{\tt/}S?} Nếu sai, sửa cho đúng. (a) 1 số hữu tỷ khác $0$ \& số đối của nó là 2 đại lượng tỷ lệ thuận. (b) 2 đại lượng $x,\sqrt{x}$ là 2 đại lượng tỷ lệ thuận. (c) Số vòng quay của kim giờ \& kim phút trong cùng 1 thời gian là 2 đại lượng tỷ lệ thuận. (d) Nếu ta cùng thêm 1 số vào tất cả các giá trị của 2 đại lượng tỷ lệ thuận, ta sẽ được các số mới cũng là các giá trị của 2 đại lượng tỷ lệ thuận.
\end{baitoan}

\begin{baitoan}[\cite{Binh_boi_duong_Toan_7_tap_1}, VD1, p. 38]
	Cho các giá trị tương ứng của $x,y$ trong 2 bảng:
	\begin{table}[H]
		\centering
		\begin{tabular}{|c|c|c|c|c|c|}
			\hline
			$x$ & $0.65$ & $2.75$ & $0.6$ & $1.34$ & 37 \\
			\hline
			$y$ & $5.2$ & 22 & $4.8$ & $10.72$ & 296 \\
			\hline
		\end{tabular}\hspace{1cm}
		\begin{tabular}{|c|c|c|c|c|c|}
			\hline
			$x$ & $-2^5$ & $-2^3$ & $-2$ & $-2^2$ & $-2^4$ \\
			\hline
			$y$ & 128 & 32 & 4 & $-16$ & $-64$ \\
			\hline
		\end{tabular}
	\end{table}
	\noindent(a) Trong mỗi bảng, 2 đại lượng $x,y$ có tỷ lệ thuận với nhau không? (b) Nếu 2 đại lượng $x,y$ tỷ lệ thuận với nhau, chỉ ra hệ số tỷ lệ \& viết công thức biểu thị sự tương quan đó.
\end{baitoan}

\begin{baitoan}[\cite{Binh_boi_duong_Toan_7_tap_1}, VD2, p. 38]
	Biết $x,y$ trong bảng là 2 đại lượng tỷ lệ thuận, điền số thích hợp:
	\begin{table}[H]
		\centering
		\begin{tabular}{|c|c|c|c|c|c|}
			\hline
			$x$ & 1 & 2 & & & 5 \\
			\hline
			$y$ & $-\frac{2}{3}$ & & $-2$ & $-2\frac{2}{3}$ & \\
			\hline
		\end{tabular}
	\end{table}
\end{baitoan}

\begin{baitoan}[\cite{Binh_boi_duong_Toan_7_tap_1}, VD3, p. 39]
	Cho $x,y$ là 2 đại lượng tỷ lệ thuận. Ký hiệu $x_1,x_2$ là 2 giá trị của $x$ mà $x_1 = -1,x_2 = -3$. Gọi $y_1,y_2$ là 2 giá trị tương ứng của đại lượng $y$ mà $y_1 - y_2 = -2$. (a) Tìm $y_1,y_2$. (b) Tìm công thức mà 2 đại lượng $x,y$ liên hệ với nhau.
\end{baitoan}

\begin{baitoan}[\cite{Binh_boi_duong_Toan_7_tap_1}, VD4, p. 39]
	Từ {\rm100 kg} thóc cho ta {\rm65 kg} gạo. Biết chất bột chiếm trong gạo là $80\%$. (a) Tính {\rm kg} chất bột có trong {\rm30 kg} thóc. (b) Từ {\rm1 kg} gạo làm ra {\rm2.2 kg} bún tươi. Tính số {\rm kg} thóc để làm ra {\rm14.3 kg} bún tươi.
\end{baitoan}

\begin{baitoan}[\cite{Binh_boi_duong_Toan_7_tap_1}, VD5, p. 40]
	Chia số $38$ thành 3 số sao cho số thứ nhất \& số thứ 2 tỷ lệ theo $0.8,0.375$, còn số thứ 2 \& số thứ 3 tỷ lệ theo $0.25,1.75$.
\end{baitoan}

\begin{baitoan}[\cite{Binh_boi_duong_Toan_7_tap_1}, VD6, p. 40]
	1 xe đạp \& 1 xe máy khởi hành cùng lúc từ thành phố A đến thành phố B. Vì vận tốc của xe đạp nhỏ hơn vận tốc của xe máy là {\rm18 km{\tt/}h} nên khi xe máy tới B thì xe đạp mới tới C (C nằm giữa A,B), cách B 1 quãng đường bằng $0.6$ lần quãng đường AB. Tính vận tốc mỗi xe.
\end{baitoan}

\begin{baitoan}[\cite{Binh_boi_duong_Toan_7_tap_1}, VD7, p. 40]
	1 nông trường trồng rừng phòng hộ trên 3 khu đất. Biết diện tích khu thứ nhất bằng $40\%$ diện tích tổng cộng của cả 3 khu đất. Còn diện tích khu đất thứ 2 \& thứ 3 tỷ lệ theo $1.5,1.(3)$. Biết diện tích khu đất thứ nhất lớn hơn diện tích khu đất thứ 3 là {\rm12 ha}, tính diện tích tổng cộng của cả 3 khu đất.
\end{baitoan}

\begin{baitoan}[\cite{Binh_boi_duong_Toan_7_tap_1}, VD8, p. 41]
	Anh hơn em $3$ tuổi. Tìm tuổi anh \& tuổi em, biết tuổi anh hiện nay bằng $2$ lần tuổi em khi tuổi anh bằng tuổi em hiện nay.
\end{baitoan}

\begin{baitoan}[\cite{Binh_boi_duong_Toan_7_tap_1}, VD9, p. 41]
	Tìm tổng của 5 số, biết số thứ nhất \& thứ 3 tỷ lệ với $5,9$, số thứ 2 \& thứ 4 tỷ lệ với $0.2,0.(3)$, số thứ 2 \& thứ 3 tỷ lệ với $3,4.5$, số thứ 5 \& thứ 4 tỷ lệ với $1,2.5$, còn tổng của số thứ 3 \& thứ 4 là $3.8$.
\end{baitoan}

\begin{baitoan}[\cite{Binh_boi_duong_Toan_7_tap_1}, 6.1., p. 41]
	1 con ruồi có $6$ chân. Điền số thích hợp:
	\begin{table}[H]
		\centering
		\begin{tabular}{|c|c|c|c|c|c|c|}
			\hline
			Số con ruồi & 1 & 4 &  & 17 & 42 & \\
			\hline
			Số chân ruồi & & & 42 &  &  & 438 \\
			\hline
		\end{tabular}
	\end{table}
\end{baitoan}

\begin{baitoan}[\cite{Binh_boi_duong_Toan_7_tap_1}, 6.2., p. 42]
	1 cửa hàng áo thời trang đã tăng giá các loại áo thêm $7\%$. Điền số thích hợp:
	\begin{table}[H]
		\centering
		\begin{tabular}{|c|c|c|c|c|}
			\hline
			Giá gốc (đồng) & 234000 &  & 4270000 &  \\
			\hline
			Tăng thêm (đồng) &  & 28000 &  & 61600 \\
			\hline
			Giá sau khi tăng (đồng) &  &  &  &  \\
			\hline
		\end{tabular}
	\end{table}
\end{baitoan}

\begin{baitoan}[\cite{Binh_boi_duong_Toan_7_tap_1}, 6.3., p. 42]
	Biết thời gian di chuyển là $20$ phút. Điền số thích hợp:
	\begin{table}[H]
		\centering
		\begin{tabular}{|c|c|c|c|c|c|c|}
			\hline
			Vận tốc (km{\tt/}h) & 60 & 30 & 24 & 15 & 12 & 6 \\
			\hline
			Quãng đường (km) &  &  &  &  &  &  \\
			\hline
		\end{tabular}
	\end{table}
\end{baitoan}

\begin{baitoan}[\cite{Binh_boi_duong_Toan_7_tap_1}, 6.4., p. 42]
	Trong rừng Amazon, 1 con thú ăn kiến có thể ăn được $1000$ con kiến trong $40$ phút. Hỏi con thú đó ăn được bao nhiêu con kiến trong $30$ phút, $1$ giờ $30$ phút, $2$ giờ $10$ phút, $3$ giờ $50$ phút? (Giả sử có đủ số kiến \& con thú ăn với tốc độ không đổi).
\end{baitoan}

\begin{baitoan}[\cite{Binh_boi_duong_Toan_7_tap_1}, 6.5., p. 42]
	Biết từ {\rm12 kg} lúa mì cho ra {\rm11 kg} bột mì, còn từ {\rm10 kg} bột mì sẽ làm ra {\rm13 kg} bánh mì. (a) Từ {\rm1440 kg} lúa mì sẽ làm ra bao nhiêu {\rm kg} bánh mì? (b) Cần bao nhiêu {\rm kg} bột mì để làm ra {\rm260 kg} bánh mì?
\end{baitoan}

\begin{baitoan}[\cite{Binh_boi_duong_Toan_7_tap_1}, 6.6., p. 42]
	1 quả trứng đà điều làm món trứng tráng tương đương với $24$ quả trứng gà. Với $6$ quả trứng gà đủ làm món trứng tráng cho $5$ người ăn. Tính số quả trứng đà điểu làm món trứng tráng cho $100$ người ăn.
\end{baitoan}

\begin{baitoan}[\cite{Binh_boi_duong_Toan_7_tap_1}, 6.7., p. 42]
	Để làm ra $10$ bát chè nhãn lồng hạt sen, nguyên liệu chính cần có $80$ quả nhãn lồng \& {\rm300 g} đường. 1 cửa hàng chè ngày thứ 2 bán được $240$ bát chè, ngày thứ 3 bán được $150$ bát chè \& ngày thứ 4 bán được $180$ bát chè. (a) Tính số đường cần dùng cho 3 ngày thứ 2, thứ 3, \& thứ 4. (b) Nếu cửa hàng đã mua sẵn {\rm21 kg} đường thì sau 3 ngày thứ 2, thứ 3, \& thứ 4 với số đường còn lại sẽ làm được bao nhiêu bát chè \& cần sử dụng bao nhiêu quả nhãn lồng?
\end{baitoan}

\begin{baitoan}[\cite{Binh_boi_duong_Toan_7_tap_1}, 6.8., p. 42]
	Nem rán là 1 món đặc sắc mang đậm hương vị dân tộc. Trong mâm cổ dịp lễ, Tết cổ truyền của người Việt Nam không thể thiếu được món nem. Để chuẩn bị món nem rán cho $6$ mâm cỗ, bên cạnh các loại rau \& gia vị, thì nguyên liệu chính là {\rm2 kg} thịt nạc vai \& $3$ quả trứng gà. (a) Hỏi cần bao nhiêu {\rm kg} thịt nạc vai \& bao nhiêu quả trứng gà để chuẩn bị món nem rán cho $102$ mâm cỗ? (b) Nếu mua $12$ hộp trứng gà ($10$ quả{\tt/}hộp) thì phải mua bao nhiêu {\rm kg} thịt nạc vai \& sẽ làm được bao nhiêu mâm cỗ khi sử dụng hết số trứng gà đó để làm món nem rán?
\end{baitoan}

\begin{baitoan}[\cite{Binh_boi_duong_Toan_7_tap_1}, 6.9., p. 43]
	1 chiếc cân lò xo có 1 đầu gắn vào 1 thanh ngang cố định, còn đầu kia có móc vật cần cân. Thực hiện cân 1 số đồ vật \& cho kết quả ở bảng:
	\begin{table}[H]
		\centering
		\begin{tabular}{|c|c|c|c|c|}
			\hline
			Khối lượng cân (g) & 0 & 250 & 500 & 1000 \\
			\hline
			Chiều dài lò xo (mm) & 33 & 43 & 53 & 73 \\
			\hline
		\end{tabular}
	\end{table}
	\noindent Qua bảng ta thấy khối lượng tăng lên thì chiều dài lò xo cũng tăng lên. Khối lượng đồ vật \& chiều dài lò xo có là 2 đại lượng tỷ thuận không?
\end{baitoan}

\begin{baitoan}[\cite{Binh_boi_duong_Toan_7_tap_1}, 6.10., p. 43]
	Bầu trời đêm dông lóe sáng 1 tia chớp. 1 người nghe thấy tiếng sấm sau đó {\rm21 s}. Tính khoảng cách từ chỗ xuất hiện tia chớp đến chỗ người đó đứng biết vận tốc của âm thanh là {\rm343 m{\tt/}s}.
\end{baitoan}

\begin{baitoan}[\cite{Binh_boi_duong_Toan_7_tap_1}, 6.11., p. 43]
	Giáp Tết cổ truyền, 1 cửa hàng mua {\rm128 kg} gồm các nguyên liệu: gạo, thịt, đỗ theo tỷ lệ $5,2,1$ để làm bánh chưng. Biết với {\rm5 kg} gạo, {\rm2 kg} thịt, {\rm1 kg} đỗ làm được $10$ chiếc bánh chưng loại to.n Tính số chiếc bánh chưng loại to sẽ làm ra được với {\rm128 kg} các nguyên liệu.
\end{baitoan}

\begin{baitoan}[\cite{Binh_boi_duong_Toan_7_tap_1}, 6.12., p. 43]
	Lãi suất ngân hàng của năm 2021 là khoảng $7\%${\tt/}năm. Tính số tiền lãi hằng tháng nếu số tiền gửi là $200$ triệu đồng Việt Nam.
\end{baitoan}

\begin{baitoan}[\cite{Binh_boi_duong_Toan_7_tap_1}, 6.13., p. 43]
	Cho $x,y$ là 2 đại lượng tỷ lệ thuận. Biết tổng 2 giá trị nào đó của $x$ bằng $1$ \& tổng 2 giá trị tương ứng của $y$ bằng $-2$. Viết công thức liên hệ giữa $y,x$.
\end{baitoan}

\begin{baitoan}[\cite{Binh_boi_duong_Toan_7_tap_1}, 6.14., p. 43]
	2 người cùng làm xong 1 công việc trong $3$ giờ. Nếu người A làm sớm hơn $1$ giờ \& người B làm chậm đi nửa giờ thì họ hoàn thành công việc sớm $18$ phút. Ngược lại, nếu người B làm sớm hơn $1$ giờ \& người A làm chậm đi nửa giờ thì người A nhận tiền công ít hơn so với thực tế là $56000$ đồng. Tính số tiền công người A nhận được.
\end{baitoan}

\begin{baitoan}[\cite{Tuyen_Toan_7}, VD35, p. 31]
	Cho $y$ tỷ lệ thuận với $x$ với hệ số tỷ lệ là 1 số âm. Biết tổng các bình phương 2 giá trị của $y$ là $18$, tổng các bình phương 2 giá trị tương ứng của $x$ là $2$. Viết công thức liên hệ giữa $y$ \& $x$.
\end{baitoan}

\begin{proof}[Giải]
	Gọi 2 giá trị của $y$ là $y_1,y_2$, 2 giá trị tương ứng của $x$ là $x_1,x_2$. Vì $y$ tỷ lệ thuận với $x$ nên $\frac{y_1}{x_1} = \frac{y_2}{x_2} = k$. Suy ra $k^2 = \frac{y_1^2}{x_1^2} = \frac{y_2^2}{x_2^2} = \frac{y_1^2 + y_2^2}{x_1^2 + x_2^2} = \frac{18}{2} = 9\Rightarrow k = \pm\sqrt{9} = \pm3$, mà $k < 0$, nên $k = -3$. Công thức liên hệ: $y = -3x$. 
\end{proof}
\noindent\textit{Nhận xét.} Nếu thay đổi giả thiết ``hệ số tỷ lệ là 1 số âm'' thành ``hệ số tỷ lệ là 1 số dương'', công thức liên hệ giữa $x,y$ sẽ là $y = 3x$. ``Vì $y$ tỷ lệ thuận với $x$ nên muốn viết được công thức liên hệ giữa $y$ \& $x$ ta phải xác định được hệ số tỷ lệ $k$. Ta tính $k$ nhờ áp dụng tính chất của 2 đại lượng tỷ lệ thuận \& của dãy tỷ số bằng nhau.'' -- \cite[p. 31]{Tuyen_Toan_7}

\begin{baitoan}[Mở rộng \cite{Tuyen_Toan_7}, VD35, p. 31]
	Cho $y$ tỷ lệ thuận với $x$. Biết tổng các bình phương 2 giá trị của $y$ là $a > 0$, tổng các bình phương 2 giá trị tương ứng của $x$ là $b > 0$. Viết công thức liên hệ giữa $y$ \& $x$ theo $a,b$.
\end{baitoan}

\begin{proof}[Giải]
	Gọi 2 giá trị của $y$ là $y_1,y_2$, 2 giá trị tương ứng của $x$ là $x_1,x_2$. Vì $y$ tỷ lệ thuận với $x$ nên $\frac{y_1}{x_1} = \frac{y_2}{x_2} = k$. Suy ra $k^2 = \frac{y_1^2}{x_1^2} = \frac{y_2^2}{x_2^2} = \frac{y_1^2 + y_2^2}{x_1^2 + x_2^2} = \frac{a}{b}\Rightarrow k = \pm\sqrt{\frac{a}{b}}$ (vì $a,b > 0$, nên $\frac{a}{b} > 0$, nên có thể lấy căn được). Công thức liên hệ: $y = \sqrt{\frac{a}{b}}x$ hoặc $y = -\sqrt{\frac{a}{b}}x$ (có thể viết gộp lại thành $y = \pm\sqrt{\frac{a}{b}}x$). 
\end{proof}

\begin{baitoan}[\cite{Tuyen_Toan_7}, VD36, p. 31]
	1 xe tải chạy từ $A$ đến $B$ mất {\rm6 h}, trong khi đó 1 xe con chạy từ $B$ đến $A$ chỉ mất {\rm3 h}. Nếu 2 xe khởi hành cùng 1 lúc thì sau bao lâu sẽ gặp nhau?
\end{baitoan}

\begin{proof}[1st Giải]
	Gọi quãng đường xe tải \& xe con đã đi cho đến khi gặp nhau lần lượt là $s_1,s_2$; vận tốc của chúng theo thứ tự là $v_1,v_2$. Trong cùng 1 thời gian, quãng đường đi được tỷ lệ thuận với vận tốc nên: $\frac{s_1}{v_1} = \frac{s_2}{v_2} = t$ ($t$: thời gian cần tìm). Coi quãng đường $AB$ là đơn vị quy ước thì: $s_1 + s_2 = 1$, $v_1 = \frac{1}{6}$, $v_2 = \frac{1}{3}$. Do đó $t = \frac{s_1}{\frac{1}{6}} = \frac{s_2}{\frac{1}{3}} = \frac{s_1 + s_2}{\frac{1}{6} + \frac{1}{3}} = \frac{1}{\frac{3}{6}} = 2$. Vậy sau khi khởi hành $2$h thì 2 xe gặp nhau.
\end{proof}

\begin{proof}[2nd Giải]
	Đặt $s\coloneqq AB$. Gọi $v_1,v_2$ là vận tốc xe tải \& xe con. Có $v_1 = \frac{s}{t_1} = \frac{s}{6}$, $v_2 = \frac{s}{t_2} = \frac{s}{3}$. Gọi$t_0$ là thời gian kể từ lúc khởi hành đến lúc 2 xe gặp nhau, có $v_1t_0 + v_2t_0 = s\Rightarrow t_0 = \frac{s}{v_1 + v_2} = \frac{s}{\frac{s}{6} + \frac{s}{3}} = 2$h.
\end{proof}
\noindent\textit{Nhận xét.} ``Trong bài này không cần biết quãng đường $AB$ dài bao nhiêu, không cần biết vận tốc của mỗi xe là bao nhiêu mà vẫn tính được thời gian 2 xe gặp nhau sau khi cùng khởi hành. Bí quyết là trong cùng 1 thời gian thì quãng đường đi \& vận tốc đi là 2 đại lượng tỷ lệ thuận.'' -- \cite[p. 31]{Tuyen_Toan_7}

\begin{baitoan}[\cite{Tuyen_Toan_7}, VD37, p. 32]
	Mức nước sinh hoạt của nhà Thủy được thống kê trong bảng sau:
	
	\begin{table}[H]
		\centering
		\begin{tabular}{|c|c|c|c|c|}
			\hline
			Thời điểm & Cuối tháng $6$ & Cuối tháng $7$ & Cuối tháng $8$ & Cuối tháng $9$ \\
			\hline
			Chỉ số đồng hồ đo nước $\rm m^3$ & $204$ & $220$ & $237$ & $250$ \\
			\hline
		\end{tabular}
	\end{table}
	\noindent Biết tổng số tiền nước nhà Thủy phải trả trong quý III là $184000$ đồng. Tính tiền nước phải trả trong mỗi tháng $7,8,9$.
\end{baitoan}

\begin{proof}[1st Giải]
	Số $\rm m^3$ nước đã dùng trong các tháng $7,8,9$ lần lượt là: $220 - 204 = 16\rm m^3$, $237 - 220 = 17\rm m^3$, $250 - 237 = 13\rm m^3$. Gọi số tiền nước trong các tháng $7,8,9$ lần lượt là $x,y,z$. Phải chia $184000$ đồng thành 3 phần tỷ lệ với $16,17,13$. Có: $\frac{x}{16} = \frac{y}{17} = \frac{z}{13} = \frac{x + y + z}{16 + 17 + 13} = \frac{184000}{46} = 4000$. Suy ra $x = 4000\cdot16 = 64000$, $y = 4000\cdot17 = 68000$, $z = 4000\cdot13 = 52000$. Vậy số tiền nước trong 3 tháng $7,8,9$ lần lượt là $64000$ đồng, $68000$ đồng, $52000$ đồng.
\end{proof}

\begin{baitoan}[\cite{Tuyen_Toan_7}, 126., p. 32]
	1 số $M$ được chia làm $3$ phần sao cho phần thứ nhất \& phần thứ 2 tỷ lệ với $5$ \& $6$, phần thứ 2 \& phần thứ 3 tỷ lệ với $8$ \& $9$. Biết phần thứ 3 hơn phần thứ 2 là $150$. Tìm số $M$.
\end{baitoan}

\begin{baitoan}[\cite{Tuyen_Toan_7}, 127., p. 32]
	1 đội thủy lợi có $10$ người làm trong $8$ ngày đào đắp được $\rm200m^3$ đất. 1 đội khác có $12$ người làm trong $7$ ngày đào đắp được bao nhiêu mét khối đất? (giả thiết năng suất của mỗi người đều như nhau).
\end{baitoan}

\begin{baitoan}[\cite{Tuyen_Toan_7}, 128., p. 32]
	2 bể nước hình hộp chữ nhật có diện tích đáy bằng nhau. Biết hiệu thể tích nước trong 2 bể là $\rm1.8m^3$, hiệu chiều cao nước trong 2 bể là $\rm0.6m$. Tính diện tích đáy của mỗi bể?
\end{baitoan}

\begin{baitoan}[\cite{Tuyen_Toan_7}, 129., p. 32]
	Đoạn đường $AB$ dài $\rm275km$. Cùng 1 lúc, 1 ô tô chạy từ A \& 1 xe máy chạy từ B đi ngược chiều để gặp nhau. Vận tốc của ô tô là {\rm60 km{\tt/}h}, vận tốc của xe máy là {\rm50 km{\tt/}h}. Tính xem đến khi gặp nhau thì mỗi xe đã đi được quãng đường bao nhiêu.
\end{baitoan}

\begin{baitoan}[\cite{Tuyen_Toan_7}, 130., p. 32]
	Vận tốc riêng của 1 canô là {\rm21 km{\tt/}h}, vận tốc dòng nước là {\rm3 km{\tt/}h}. Hỏi với thời gian để canô chạy ngược dòng được {\rm30 km} thì canô chạy xuôi dòng được bao nhiêu {\rm km}?
\end{baitoan}

\begin{baitoan}[\cite{Tuyen_Toan_7}, 131., p. 32]
	1 ô tô chạy từ A đến B với vận tốc {\rm65 km{\tt/}h}, cùng lúc đó 1 xe máy chạy từ B đến A với vận tốc {\rm40 km{\tt/}h}. Biết khoảng cách AB là {\rm540 km} \& M là trung điểm của AB. Hỏi sau khi khởi hành bao lâu thì ô tô cách M 1 khoảng cách bằng $\frac{1}{2}$ khoảng cách từ xe máy đến M?
\end{baitoan}

\begin{baitoan}[\cite{Tuyen_Toan_7}, 132., p. 32]
	Người ta trồng cây ở 1 bên của đoạn đường dài {\rm30 m} với khoảng cách giữa 2 cây liên tiếp là {\rm5 m}. Nếu cả 2 đầu đường đều trồng cây thì số cây được trồng là $\frac{30}{5} + 1 = 7$ (cây). Nếu đoạn đường dài {\rm300 m}, gấp $10$ lần đoạn đường {\rm30 m} thì số cây trồng phải gấp $10$ lần, tức là phải trồng $7\cdot 10 = 70$ (cây). Lập luận đó đúng hay sai?
\end{baitoan}

%------------------------------------------------------------------------------%

\section{Đại Lượng Tỷ Lệ Nghịch}

\begin{baitoan}[\cite{Binh_boi_duong_Toan_7_tap_1}, H1, p. 44]
	Điền thích hợpL: (a) Trên cùng quãng đường, vận tốc \& thời gian là 2 đại lượng $\ldots$ (b) Với 1 số tiền cho trước thì số hàng mua được \& $\ldots$ là 2 đại lượng tỷ lệ nghịch. (c) 1 số hữu tỷ $x\ne0$ \& số nghịch đảo của $x$ là 2 đại lượng tỷ lệ nghịch, có hệ số tỷ lệ là $\ldots$ (d) Trong các tam giác có cùng diện tích, số đo cạnh đáy \& số đo $\ldots$ là 2 đại lượng tỷ lệ nghịch.
\end{baitoan}

\begin{baitoan}[\cite{Binh_boi_duong_Toan_7_tap_1}, H2, p. 44]
	{\rm Đ{\tt/}S?} Nếu sai, sửa cho đúng. (a) Nếu vận tốc không đổi thì quãng đường \& thời gian là 2 đại lượng tỷ lệ nghịch. (b) Nếu thời gian không đổi thì quãng đường \& vận tốc là 2 đại lượng tỷ lệ nghịch. (c) Nếu quãng đường không đổi thì vận tốc \& thời gian là 2 đại lượng tỷ lệ nghịch. (d) Trên cùng quãng đường, vận tốc \& thời gian là 2 đại lượng tỷ lệ thuận.
\end{baitoan}

\begin{baitoan}[\cite{Binh_boi_duong_Toan_7_tap_1}, H3, p. 44]
	{\rm Đ{\tt/}S?} Nếu sai, sửa cho đúng. (a) Chu vi hình vuông \& cạnh là 2 đại lượng tỷ lệ nghịch với hệ số tỷ lệ là $4$. (b) 2 đại lượng $x,\sqrt{x}$ là 2 đại lượng tỷ lệ nghịch. (c) Chu vi 1 đường tròn tỷ lệ nghịch với bán kính đường tròn theo hệ số tỷ lệ $2\pi$. (d) Nếu ta cùng nhân 1 số khác $0$ vào tất cả các giá trị của 2 đại lượng tỷ lệ nghịch, ta sẽ được các số mới cũng là các giá trị của 2 đại lượng tỷ lệ nghịch.
\end{baitoan}

\begin{baitoan}[\cite{Binh_boi_duong_Toan_7_tap_1}, VD1, p. 45]
	Điền số thích hợp vào ô trống trong bảng biết $x,y$ là 2 đại lượng tỷ lệ nghịch:
	\begin{table}[H]
		\centering
		\begin{tabular}{|c|c|c|c|c|c|}
			\hline
			$x$ & 1 & $-0.25$ &  & $-4$ &  \\
			\hline
			$y$ &  & 8 & $2.5$ &  & $-12$ \\
			\hline
		\end{tabular}
	\end{table}
\end{baitoan}

\begin{baitoan}[\cite{Binh_boi_duong_Toan_7_tap_1}, VD2, p. 45]
	Chia số $330$ thành 3 phần tỷ lệ nghịch với $0.4,0.6,1.2$.
\end{baitoan}

\begin{baitoan}[\cite{Binh_boi_duong_Toan_7_tap_1}, VD3, p. 45]
	Cho $x,y$ là 2 đại lượng tỷ lệ nghịch với hệ số tỷ lệ là $a\ne0$. Biết $y,z$ cũng là 2 đại lượng tỷ lệ nghịch với hệ số tỷ lệ là $b\ne0$. Theo A quan hệ giữa 2 đại lượng $x,z$ là tương quan tỷ lệ nghịch. Lời giải của A {\rm Đ{\tt/}S?} Nếu sai, sửa cho đúng: $x,y$ tỷ lệ nghịch với hệ số tỷ lệ $a\ne0\Rightarrow x = \dfrac{y}{a} = y:a$. $y,z$ tỷ lệ nghịch với hệ số tỷ lệ $b\ne0\Rightarrow y = \dfrac{z}{b} = z:b$. Suy ra $x = y:a = \dfrac{z}{b}:a = \dfrac{z}{ab}$ ($ab\ne0$). Vậy $x,z$ tỷ lệ nghịch với nhau với hệ số tỷ lệ là $ab$.
\end{baitoan}

\begin{baitoan}[\cite{Binh_boi_duong_Toan_7_tap_1}, VD4, p. 45]
	Chia số $4500$ thành 3 số sao cho $80\%$ số thứ nhất bằng $53\frac{1}{3}\%$ số thứ 2 \& bằng $40\%$ số thứ 3.
\end{baitoan}

\begin{baitoan}[\cite{Binh_boi_duong_Toan_7_tap_1}, VD5, p. 46]
	1 người vào siêu thị mua hoa quả \& nhẩm tính thấy với số tiền mình mang đi có thể mua được hoặc {\rm3 kg} nho, hoặc {\rm5 kg} mận, hoặc {\rm4 kg} táo. Tính giá tiền mỗi loại quả, biết số tiền mua {\rm3 kg} tóa nhiều hơn số tiền mua {\rm2 kg} mận là {\rm210000 đ}.
\end{baitoan}

\begin{baitoan}[\cite{Binh_boi_duong_Toan_7_tap_1}, VD6, p. 46]
	Cho $x,y$ là 2 đại lượng tỷ lệ nghịch. Khi $x$ nhận 2 giá trị $x_1 = -3,x_2 = 2$ thì 2 giá trị t ương ứng $y_1,y_2$ có hiệu bằng $13$. Viết công thức liên hệ giữa $x,y$.
\end{baitoan}

\begin{baitoan}[\cite{Binh_boi_duong_Toan_7_tap_1}, VD7, p. 46]
	Cho $x,y$ là 2 đại lượng tỷ lệ nghịch với hệ số tỷ lệ dương. Biết đại lượng $x$ có 2 giá trị mà tích bằng $2$ \& hiệu bình phương 2 giá trị đó là $3$, còn hiệu bình phương tương ứng của $y$ là $-12$. Viết công thức liên hệ giữa $x,y$.
\end{baitoan}

\begin{baitoan}[\cite{Binh_boi_duong_Toan_7_tap_1}, VD8, p. 47]
	Cho $c_1,c_2,c_3,\ldots,c_n,c$ là cá đại lượng nào đó. Có 2 tập hợp $A = \{c_1,c_2,\ldots,c_n|n\in\mathbb{N}\},B = \{c\}$. Biết tương quan giữa đại lượng $c$ của tập hợp $B$ với lần lượt các phần tử của tập hợp $A$ là tương quan tỷ lệ nghịch. Tìm tương quan đôi một giữa các phần tử của tập hợp $A$ với nhau.
\end{baitoan}

\begin{baitoan}[\cite{Binh_boi_duong_Toan_7_tap_1}, VD9, p. 47]
	Các giá trị tương ứng của $x,y$ được cho trong bảng:
	\begin{table}[H]
		\centering
		\begin{tabular}{|c|c|c|c|c|c|}
			\hline
			$x$ & $-2^1$ & $2^2$ & $-2^3$ & $-2^4$ & $2^5$ \\
			\hline
			$y$ & 4 & $-16$ & 32 & 64 & $-128$ \\
			\hline
		\end{tabular}
	\end{table}
	\noindent Tìm tương quan giữa 2 đại lượng $x,y$.
\end{baitoan}

\begin{baitoan}[\cite{Binh_boi_duong_Toan_7_tap_1}, 7.1., p. 47]
	Chia số $1208$ thành 3 số tỷ lệ nghịch với $0.(6),0.7,1.5$. Tìm 3 số đó.
\end{baitoan}

\begin{baitoan}[\cite{Binh_boi_duong_Toan_7_tap_1}, 7.2., p. 47]
	1 cửa hàng có 3 tấm vải tổng cộng dài {\rm86.1 m}. Sau khi bán $28\%$ tấm vải thứ nhất, $40\%$ tấm vải thứ 2 \& $64\%$ tấm vải thứ 3 thì chiều dài còn lại của 3 tấm vải bằng nhau. Tính chiều dài mỗi tấm vải khi chưa bán.
\end{baitoan}

\begin{baitoan}[\cite{Binh_boi_duong_Toan_7_tap_1}, 7.3., p. 47]
	Với số tiền trước đây mua được {\rm32.9 kg} bột mì thì nay mua được {\rm40 kg} bột mì. Hỏi bột mì đã hạ giá bao nhiêu $\%$?
\end{baitoan}

\begin{baitoan}[\cite{Binh_boi_duong_Toan_7_tap_1}, 7.4., p. 48]
	Biết $78$ người hoàn thành 1 công việc trong $65$ ngày. (a) Nếu năng suất lao động của mỗi người như nhau thì cần thêm bao nhiêu người nữa để hoàn thành công việc đó trong $39$ ngày? (b) Nếu cải tiến công cụ để năng suất lao động tăng thêm $20\%$ thì cần giảm bao nhiêu người mà vẫn hoàn thành công việc đó trong $65$ ngày?
\end{baitoan}

\begin{baitoan}[\cite{Binh_boi_duong_Toan_7_tap_1}, 7.5., p. 48]
	Cà phê hạ giá $23\frac{1}{7}\%$. Với số tiền trước đây mua được {\rm5.38 kg} cà phê thì nay sẽ mua được bao nhiêu {\rm kg} cà phê hạ giá?
\end{baitoan}

\begin{baitoan}[\cite{Binh_boi_duong_Toan_7_tap_1}, 7.6., p. 48]
	Có 2 đội công nhân làm đường. Đội I có $35$ người làm trong $16$ ngày thì đào được $\rm864\ m^3$ đất. Hỏi đội II với $20$ ngừoi làm trong $14$ ngày sẽ đào được bao nhiêu $\rm m^3$ đất? Giả thiết năng suất lao động của mỗi người như nhau.
\end{baitoan}

\begin{baitoan}[\cite{Binh_boi_duong_Toan_7_tap_1}, 7.7., p. 48]
	1 học sinh đi bộ từ nhà đến trường cần $50$ phút, còn đi xe đạp chỉ cần {\rm0.3 h}. Tính quãng đường từ nhà đến trường biết vận tốc xe đạp lớn hơn vận tốc đi bộ là {\rm8 km{\tt/}h}.
\end{baitoan}

\begin{baitoan}[\cite{Binh_boi_duong_Toan_7_tap_1}, 7.8., p. 48]
	Cho $x,y$ là 2 đại lượng tỷ lệ nghịch. Gọi $x_1,x_2$ là 2 giá trị nào đó của $x$, còn $y_1,y_2$ là 2 giá trị tương ứng của $y$. Biết $x_1 = -3,y_2 = 5,5x_2 - 3y_1 = -60$. (a) Tìm $x_2,y_1$. (b) Viết công thức liên hệ giữa $x,y$.
\end{baitoan}

\begin{baitoan}[\cite{Binh_boi_duong_Toan_7_tap_1}, 7.9., p. 48]
	Gọi $x,y,z$ lần lượt là số vòng quay của kim giờ, kim phút, \& kim giây trong cùng 1 đơn vị thời gian. (a) Điền số thích hợp:
	\begin{table}[H]
		\centering
		\begin{tabular}{|c|c|c|c|c|c|}
			\hline
			$x$ & 1 &  &  &  &  \\
			\hline
			$y$ &  & 1 &  &  &  \\
			\hline
			$z$ &  &  & 1 & $0.5$ & 5 \\
			\hline
		\end{tabular}
	\end{table}
	\noindent(b) Viết công thức biểu diễn $x,y,z$ đôi một với nhau.
\end{baitoan}

\begin{baitoan}[\cite{Binh_boi_duong_Toan_7_tap_1}, 7.10., p. 48]
	Để thanh lý cửa hàng, ông chủ cửa hàng ôtô quyết định giảm giá mỗi chiếc xe xuống $10\%$. Nhưng sau đó, ông nhận thấy mình sẽ lỗ nếu bán với giá này, nên quyết định tăng giá đã giảm lên $5\%$. Tính $\%$ mức giảm giá thực của ông chủ cửa hàng.
\end{baitoan}

\begin{baitoan}[\cite{Binh_boi_duong_Toan_7_tap_1}, 7.11., p. 48]
	1 xe tải chạy từ thành phố A đến hải cảng B gồm 3 chặng đường có độ dài bằng nhau, nhưng chất lượng mặt đường xấu tốt khác nhau nên vận tốc mỗi chặng đường lần lượt bằng {\rm40,24,60 km{\tt/}h}. Tính độ dài quãng đường AB biết tổng thời gian đi từ A đến B là $5$ giờ.
\end{baitoan}

\begin{baitoan}[\cite{Binh_boi_duong_Toan_7_tap_1}, 7.12., p. 48]
	Để truyền 1 chuyển động, có thể dùng dây xích nối 2 bánh xe có răng, hoặc các bánh xe có răng khớp với nhau, hoặc dùng dây curoa. Xét 1 bộ máy truyền chuyển động có 2 bánh xe khớp răng với nhau. (a) Nếu bánh xe thứ nhất có $65$ răng \& quay $36$ vòng{\tt/}phút thì bánh xe thứ 2 có $45$ răng sẽ quay được bao nhiêu vòng{\tt/}phút? (b) Để bánh xe thứ 2 quay được $78$ vòng{\tt/}phút thì cần thiết kế bánh xe thứ 2 có bao nhiêu răng?
\end{baitoan}

\begin{baitoan}[\cite{Binh_boi_duong_Toan_7_tap_1}, 7.13., pp. 48--49]
	Khoảng cách giữa 2 ga tàu $A,B$ bằng {\rm28 km}. Cùng lúc đó có 2 đoàn tàu, 1 khởi hành tứ A, 1 từ B. Nếu chuyển động cùng chiều thì sau 1 thời gian tàu thứ nhất đi từ A sẽ đuổi kịp tàu thứ 2 đi từ B. Nếu chuyển động ngược chiều thì thời gian 2 tàu gặp nhau chỉ bằng thời gian tàu thứ nhất đuổi kịp tàu thứ 2. Hỏi 2 tàu gặp nhau ở vị trí nào giữa 2 ga $A,B$? Giả sử giữa 2 ga tàu có 2 tuyển đường sắt song song với nhau, mỗi tàu chuyển động riêng trên 1 tuyến đường.
\end{baitoan}

\begin{baitoan}[\cite{Binh_boi_duong_Toan_7_tap_1}, 7.14., p. 49]
	Trước đây $25$ năm tuổi cha \& con tỷ lệ nghịch với $5,41$. (a) Hiện nay tuổi cha bằng $2.2$ lần tuổi con. Tính tuổi mỗi người. (b) Khi tuổi cha gấp $3$ lần tuổi con thì con bao nhiêu tuổi?
\end{baitoan}

\begin{baitoan}[\cite{Binh_boi_duong_Toan_7_tap_1}, p. 49]
	2 bạn $A,B$ mang 1 số tiền vừa đủ để mua $20$ quyển vở. Nhân dịp năm học mới cửa hàng bán hạ giá $20\%$. Tính số quyển vở có thể mua được tối đa.
\end{baitoan}

\begin{baitoan}[\cite{Binh_boi_duong_Toan_7_tap_1}, p. 49]
	Nhân dịp Noel, 1 cửa hàng sách giảm giá $10\%$ giá bìa. Tuy vậy, cửa hàng vẫn còn lãi $12.5\%$ so với giá mua. Hỏi ngày thường cửa hàng đó lãi bao nhiêu $\%$ so với giá mua?
\end{baitoan}

\begin{baitoan}[\cite{Tuyen_Toan_7}, VD38, p. 33]
	2 ô tô cùng khởi hành từ A đến B. Vận tốc của ô tô I là {\rm50 km{\tt/}h}, vận tốc của ô tô II là {\rm60 km{\tt/}h}. Ô tô I đi đến B sau ô tô II là $36$ phút. Tính quãng đường AB.
\end{baitoan}

\begin{baitoan}[\cite{Tuyen_Toan_7}, VD39, p. 34]
	1 số $M$ được chia thành 3 phần tỷ lệ nghịch với $5,2,4$. Biết tổng các lập phương của 3 phần đó là $9512$. Tìm số $M$.
\end{baitoan}

\begin{baitoan}[\cite{Tuyen_Toan_7}, 133., p. 34]
	2 bác mua gạo hết cùng 1 số tiền. Bác thứ nhất mua loại {\rm24000 đồng{\tt/}kg}, bác thứ 2 mua loại {\rm28800 đồng{\tt/}kg}. Biết bác thứ nhất mua nhiều hơn bác thứ 2 là {\rm2 kg}. Hỏi mỗi bác mua bao nhiêu {\rm kg} gạo?
\end{baitoan}

\begin{baitoan}[\cite{Tuyen_Toan_7}, 134., p. 34]
	2 cạnh của 1 tam giác dài {\rm25 cm} \& {\rm36 cm}. Tổng độ dài 2 đường cao tương ứng là {\rm48.8 cm}. Tính độ dài của mỗi đường cao nói trên.
\end{baitoan}

\begin{baitoan}[\cite{Tuyen_Toan_7}, 135., p. 34]
	1 xe ô tô chạy từ A đến B gồm 3 chặng đường dài như nhau nhưng chất lượng mặt đường tốt xấu khác nhau. Vận tốc trên mỗi chặng đường lần lượt là {\rm72 km{\tt/}h}, {\rm60 km{\tt/}h}, {\rm40 km{\tt/}h}. Biết tổng thời gian xe chạy từ A đến B là {\rm4 h}. Tính quãng đường AB.
\end{baitoan}

\begin{baitoan}[\cite{Tuyen_Toan_7}, 136., p. 34]	
	1 xe ô tô chạy từ A đến B với vận tốc {\rm50 km{\tt/}h} rồi chạy từ B về A với vận tốc {\rm40 km{\tt/}h}. Cả đi lẫn về mất {\rm4h30m}. Tính thời gian đi \& về.
\end{baitoan}

\begin{baitoan}[\cite{Tuyen_Toan_7}, 137., p. 34]
	1 ô tô dự định chạy từ A đến B trong thời gian nhất định. Nếu xe chạy với vận tốc {\rm54 km{\tt/}h} thì đến nơi sớm hơn {\rm1 h}. Nếu xe chạy với vận tốc {\rm63 km{\tt/}h} thì đến nơi sớm hơn {\rm2 h}. Tính quãng đường AB \& thời gian dự định đi.
\end{baitoan}

\begin{baitoan}[\cite{Tuyen_Toan_7}, 138., p. 34]
	Để làm xong 1 công việc thì $21$ công nhân cần làm trong $15$ ngày. Do cải tiến công cụ lao động nên năng suất lao động của mỗi người tăng thêm $25\%$. Hỏi $18$ công nhân phải làm bao lâu mới xong công việc đó?
\end{baitoan}

\begin{baitoan}[\cite{Tuyen_Toan_7}, 139., p. 34]
	Để làm xong 1 công việc, 1 số công nhân cần làm trong 1 số ngày. 1 bạn học sinh lập luận: Nếu số công nhân tăng thêm $\frac{1}{3}$ thì thời gian sẽ giảm đi $\frac{1}{3}$. Đúng hay sai?
\end{baitoan}

%------------------------------------------------------------------------------%

\subsection{Chia Tỷ Lệ}
``Trong các bài toán về chia 1 số thành các phần tỷ lệ thuận hoặc tỷ lệ nghịch với các số cho trước, cần chú ý:

\fbox{\bf 1} $x,y,z$ tỷ lệ thuận với $a,b,c\Leftrightarrow x:y:z = a:b:c\Leftrightarrow\frac{x}{a} = \frac{y}{b} = \frac{z}{c}$.
\fbox{\bf 2} $x,y,z$ tỷ lệ nghịch với $m,n,p\Leftrightarrow x:y:z = \frac{1}{m}:\frac{1}{n}:\frac{1}{p}$.'' -- \cite{Binh_Toan_7_tap_1}


\begin{baitoan}[\cite{Binh_Toan_7_tap_1}, VD16]
	2 xe ô tô cùng khởi hành 1 lúc từ 2 địa điểm A \& B. Xe thứ nhất đi quãng đường AB hết $\rm4h15ph$, xe thứ 2 đi quãng đường BA hết $\rm3h45ph$. Đến chỗ gặp nhau, xe thứ 2 đi được quãng đường dài hơn quãng đường xe thứ nhất đã đi là {\rm20 km}. Tính quãng đường AB.
\end{baitoan}

\begin{baitoan}[\cite{Binh_Toan_7_tap_1}, VD17]
	Để đi từ A đến B có thể dùng các phương tiện: máy bay, ô tô, xe lửa. Vận tốc của máy bay, ô tô, xe lửa có tỷ lệ với $6$; $2$; $1$. Biết thời gian đi từ A đến B bằng máy bay ít hơn so với đi bằng ô tô là $6$ giờ. Hỏi thời gian xe lửa đi quãng đường AB là bao lâu?
\end{baitoan}

\begin{baitoan}[\cite{Binh_Toan_7_tap_1}, VD18]
	3 kho A, B, C chứa 1 số gạo. Người ta nhập vào kho A thêm $\frac{1}{7}$ số gạo của kho đó, xuất ở kho B đi $\frac{1}{9}$ số gạo của kho đó, xuất ở kho C đi $\frac{2}{7}$ số gạo của kho đó. Khi đó số gạo của 3 kho bằng nhau. Tính số gạo ở mỗi kho lúc đầu, biết kho B chứa nhiều hơn kho A là $20$ tạ gạo.
\end{baitoan}

\begin{baitoan}[\cite{Binh_Toan_7_tap_1}, VD19]
	3 đội công nhân I, II, III phải vận chuyển tổng cộng {\rm1530 kg} hàng từ kho theo thứ tự đến 3 địa điểm cách kho {\rm1500 m, 2000 m, 3000 m}. Phân chia số hàng cho mỗi đội sao cho khối lượng hàng tỷ lệ nghịch với khoảng cách cần chuyển.
\end{baitoan}

\begin{baitoan}[\cite{Binh_Toan_7_tap_1}, VD20]
	3 xí nghiệp cùng xây dựng chung 1 cái cầu hết $38$ triệu đồng. Xí nghiệp I có $40$ xe ở cách cầu {\rm1.5 km}, xí nghiệp II có $20$ xe ở cách cầu {\rm3 km}, xí nghiệp III có $30$ xe ở cách cầu {\rm1 km}. Hỏi mỗi xí nghiệp phải trả cho việc xây dựng cầu bao nhiêu tiền, biết số tiền phải trả tỷ lệ thuận với số xe \& tỷ lệ nghịch với khoảng cách từ xí nghiệp đến cầu?
\end{baitoan}

\begin{baitoan}[\cite{Binh_Toan_7_tap_1}, 106.]
	
	(a) Tính thời gian từ lúc 2 kim đồng hồ gặp nhau lần trước đến lúc chúng gặp nhau lần tiếp theo.
	(b) Trong 1 ngày, 2 kim đồng hồ tạo với nhau góc vuông bao nhiêu lần?
	
\end{baitoan}

\begin{baitoan}[\cite{Binh_Toan_7_tap_1}, 107.]
	1 ống dài được kéo bởi 1 máy kéo trên đường. Tuấn chạy dọc từ đầu ống đến cuối ống theo hướng chuyển động của máy kéo thì đếm được $140$ bước. Sau đó Tuấn quay lại chạy dọc ống theo chiều ngược lại thì đếm được $20$ bước. Biết mỗi bước chạy của Tuấn dài {\rm1 m}. Tính độ dài của ống.
\end{baitoan}

\begin{baitoan}[\cite{Binh_Toan_7_tap_1}, 108.]
	5 lớp 7A, 7B, 7C, 7D, 7E nhận chăm sóc vườn trường có diện tích $\rm300\ m^2$. Lớp 7A nhận $15\%$ diện tích vườn, lớp 7B nhận $\frac{1}{5}$ diện tích còn lại. Diện tích còn lại của vườn sau khi 2 lớp trên nhận được đem chia cho 3 lớp 7C, 7D, 7E tỷ lệ với $\frac{1}{2},\frac{1}{4},\frac{5}{16}$. Tính diện tích vườn giao cho mỗi lớp.
\end{baitoan}

\begin{baitoan}[\cite{Binh_Toan_7_tap_1}, 109.]
	3 công nhân được thưởng $100000$ đồng, số tiền thưởng được phân chia tỷ lệ với mức sản xuất của mỗi người. Biết mức sản xuất của người thứ nhất so với mức sản xuất của người thứ 2 bằng $5:3$, mức sản xuất của người thứ 3 bằng $25$\% tổng số mức sản xuất của 2 người kia. Tính số tiền mỗi người được thưởng.
\end{baitoan}

\begin{baitoan}[\cite{Binh_Toan_7_tap_1}, 110.]
	1 công trường dự định phân chia số đất cho 3 đội I, II, III tỷ lệ với $7,6,5$. Nhưng sau đó vì số người của các đội thay đổi nên đã chia lại tỷ lệ với $6,5,4$. Như vậy có 1 đội làm nhiều hơn so với dự định là $\rm6m^3$ đất. Tính số đất đã phân chia cho mỗi đội.
\end{baitoan}

\begin{baitoan}[\cite{Binh_Toan_7_tap_1}, 111.]
	Trong 1 đợt lao đông, 3 khối 7, 8, 9 chuyển được $\rm912m^3$ đất. Trung bình mỗi học sinh khối 7, 8, 9 theo thứ tự làm được $\rm1.2m^3,1.4m^3,1.6m^3$. Số học sinh khối 7 \& khối 8 tỷ lệ với 1 \& 3, số học sinh khối 8 \& khối 9 tỷ lệ với 4 \& 5. Tính số học sinh của mỗi khối.
\end{baitoan}

\begin{baitoan}[\cite{Binh_Toan_7_tap_1}, 112.]
	3 tổ công nhân có mức sản xuất tỷ lệ với $5,4,3$. Tổ I tăng năng suất $10$\%, tổ II tăng năng suất $20$\%, tổ III tăng năng suất $10$\%. Do đó trong cùng 1 thời gian, tổ I làm được nhiều hơn tổ II là $7$ sản phẩm. Tính số sản phẩm mỗi tổ đã làm được trong thời gian đó.
\end{baitoan}

\begin{baitoan}[\cite{Binh_Toan_7_tap_1}, 113.]
	Tìm 3 số tự nhiên, biết $\operatorname{BCNN}$ của chúng bằng $3150$, tỷ số của số thứ nhất \& số thứ 2 là $5:9$, tỷ số của số thứ nhất \& số thứ 3 là $10:7$.
\end{baitoan}

\begin{baitoan}[\cite{Binh_Toan_7_tap_1}, 114.]
	3 tấm vải theo thứ tự giá $120000$ đồng, $192000$ đồng, \& $144000$ đồng. Tấm thứ nhất \& tấm thứ 2 có cùng chiều dài, tấm thứ 2, \& tấm thứ 3 có cùng chiều rộng. Tổng của 3 chiều dài là {\rm110 m}, tổng của 3 chiều rộng là {\rm2.1 m}. Tính kích thước của mỗi tấm vải, biết giá $\rm1\ m^2$ của 3 tấm vải bằng nhau.
\end{baitoan}

\begin{baitoan}[\cite{Binh_Toan_7_tap_1}, 115.]
	Có 3 gói tiền: gói thứ nhất gồm toàn tờ $500$ đồng, gói thứ 2 gồm toàn tờ $2000$ đồng, gói thứ 3 gồm toàn tờ $5000$ đồng. Biết tổng số tờ giấy bạc của 3 gói là $540$ tờ \& số tiền ở các gói bằng nhau. Tính số tờ giấy bạc mỗi loại.
\end{baitoan}

\begin{baitoan}[\cite{Binh_Toan_7_tap_1}, 116.]
	3 công nhân tiện được tất cả $860$ dụng cụ trong cùng 1 thời gian. Để tiện 1 dụng cụ, người thứ nhất cần {\rm5 ph}, người thứ 2 cần {\rm6 ph}, người thứ 3 cần {\rm9 ph}. Tính số dụng cụ mỗi người tiện được.
\end{baitoan}

\begin{baitoan}[\cite{Binh_Toan_7_tap_1}, 117.]
	3 em bé: Ánh $5$ tuổi, Bích $6$ tuổi, Châu $10$ tuổi được bà chia cho $42$ chiếc kẹo. Số kẹo được chia tỷ lệ nghịch với số tuổi của mỗi em. Hỏi mỗi em được chia bao nhiêu chiếc kẹo?
\end{baitoan}

\begin{baitoan}[\cite{Binh_Toan_7_tap_1}, 118.]
	Tìm 3 phân số, biết tổng của chúng bằng $3\frac{3}{70}$, các tử của chúng tỷ lệ với $3,4,5$, các mẫu của chúng tỷ lệ với $5,1,2$.
\end{baitoan}

\begin{baitoan}[\cite{Binh_Toan_7_tap_1}, 119.]
	Tìm số tự nhiên có 3 chữ số, biết số đó là bội của $72$ \& các chữ số của nó nếu xếp từ nhỏ đến lớn thì tỷ lệ với $1,2,3$.
\end{baitoan}

\begin{baitoan}[\cite{Binh_Toan_7_tap_1}, 120.]
	Độ dài 3 cạnh của 1 tam giác tỷ lệ với $2,3,4$. 3 chiều cao tương ứng với 3 cạnh đó tỷ lệ với 3 số nào?
\end{baitoan}

\begin{baitoan}[\cite{Binh_Toan_7_tap_1}, 121.]
	3 đường cao của $\Delta ABC$ có độ dài bằng $4,12,x$. Biết $x\in\mathbb{N}^\star$. Tìm $x$ (cho biết {\rm bất đẳng thức tam giác}: mỗi cạnh của tam giác nhỏ hơn tổng 2 cạnh kia \& lớn hơn hiệu của chúng).
\end{baitoan}

\begin{baitoan}[\cite{Binh_Toan_7_tap_1}, 122.]
	Cho $\Delta ABC$. Có góc ngoài của tam giác tại $A,B,C$ tỷ lệ với $4,5,6$. Các góc trong tương ứng tỷ lệ với các số nào?
\end{baitoan}

\begin{baitoan}[\cite{Binh_Toan_7_tap_1}, 123.]
	Tìm 2 số khác $0$ biết tổng, hiệu, tích của chúng tỷ lệ với $5,1,12$.
\end{baitoan}

%------------------------------------------------------------------------------%

\section{Miscellaneous}
\textsf{Nội dung.} Định nghĩa số vô tỷ, căn bậc 2 số học; tập hợp $\mathbb{R}$ các số thực; giá trị tuyệt đối của 1 số thực; làm tròn số; tỷ lệ thức \& dãy tỷ số bằng nhau; đại lượng tỷ lệ thuận, tỷ lệ nghịch.

\begin{baitoan}[\cite{Tuyen_Toan_7}, VD40, p. 35]
	So sánh $\sqrt{24} + \sqrt{14}$ \& $\sqrt{84}$.
\end{baitoan}

\begin{baitoan}[\cite{Tuyen_Toan_7}, VD41, p. 35]
	3 công nhân được lĩnh tổng cộng $18 500 000$ đồng tiền thưởng. Số tiền thưởng của mỗi người tỷ lệ nghịch với số ngày nghỉ của họ. Biết số ngày nghỉ lần lượt là $5,4,6$ ngày. Tính số tiền thưởng của mỗi người.
\end{baitoan}

\begin{baitoan}[\cite{Tuyen_Toan_7}, 140., p. 35]
	Trong các số sau, những số nào là số hữu tỷ, những số nào là số vô tỷ? $0.4343\ldots$, $-13.9$, $\pi$, $59.8637$, $3.464101615\ldots$, $\sqrt{10}$, $6 + \sqrt{2}$, $\frac{61}{172}$, số $x > 0$ mà $x^2 = 7$, số $y > 0$ mà $y^2 = 121$.
\end{baitoan}

\begin{baitoan}[\cite{Tuyen_Toan_7}, 141., p. 36]
	So sánh: (a) $5 + \sqrt{99}$ \& $\sqrt{21} + \sqrt{93}$. (b) $\sqrt{54} + \sqrt{230}$ \& $22$.	
\end{baitoan}

\begin{baitoan}[\cite{Tuyen_Toan_7}, 142., p. 36]
	Viết các số sau dưới dạng số thập phân hữu hạn hoặc vô hạn (làm tròn đến hàng phần trăm). Sắp xếp kết quả theo thứ tự tăng dần: $\dfrac{37}{7},\dfrac{43}{8},\sqrt{29}$.
\end{baitoan}

\begin{baitoan}[\cite{Tuyen_Toan_7}, 143., p. 36]
	Tìm $x$ biết: $\left|x + \dfrac{1}{101}\right| + \left|x + \dfrac{2}{101}\right| = \left|x + \dfrac{3}{101}\right| + \cdots + \left|x + \dfrac{100}{101}\right| = 101x$.
\end{baitoan}

\begin{baitoan}[\cite{Tuyen_Toan_7}, 144., p. 36]
	Cho $2026$ số thực $a_1,a_2,a_3,\ldots,a_{2026}$ sao cho bất kỳ $5$ số nào trong chúng cũng có tổng bằng $0$. Tìm $a_{2026}$.
\end{baitoan}

\begin{baitoan}[\cite{Tuyen_Toan_7}, 145., p. 36]
	Tìm 3 phân số tối giản biết tổng của chúng bằng $5\frac{25}{63}$, tử số của chúng tỷ lệ nghịch với $20,4,5$ \& mẫu số của chúng tỷ lệ thuận với $1,3,7$.
\end{baitoan}

\begin{baitoan}[\cite{Tuyen_Toan_7}, 146., p. 36]
	Chu vi 1 tam giác là {\rm60 cm}. Các đường cao có độ dài là {\rm12 cm, 15 cm, 20 cm}. Tính độ dài mỗi cạnh của tam giác đó.
\end{baitoan}

\begin{baitoan}[\cite{Tuyen_Toan_7}, 147., p. 36]
	Nếu ta cộng từng 2 cạnh của 1 tam giác thì 3 tổng tỷ lệ với $5,6,7$. Chứng minh tam giác này có 1 đường cao dài gấp $2$ lần 1 đường cao khác.
\end{baitoan}

\begin{baitoan}[\cite{Tuyen_Toan_7}, 148., p. 36]
	1 xe ô tô khởi hành từ A, dự định chạy với vận tốc {\rm60 km{\tt/}h} thì sẽ tới B lúc 11:00. Sau khi chạy được nửa quãng đường vì đường hẹp \& xấu nên vận tốc ô tô giảm xuống còn {\rm40 km{\tt/}h} do đó đến 11:00 xe vẫn còn cách B là {\rm40 km}. (a) Tính khoảng cách AB. (b) Xe khởi hành lúc mấy giờ?
\end{baitoan}

\begin{baitoan}[\cite{Tuyen_Toan_7}, 149., p. 36]
	1 đơn vị làm đường lúc đầu đặt kế hoạch giao cho 3 đội I, II, III, mỗi đội làm 1 đoạn đường có chiều dài tỷ lệ với $7,8,9$. Về sau do thiết bị máy móc \& nhân lực của các đội thay đổi nên kế hoạch đã được điều chỉnh, mỗi đội làm 1 đoạn đường có chiều dài tỷ lệ với $6,7,8$. Như vậy đội III phải làm nhiều hơn so với kế hoạch ban đầu là {\rm0.5 km} đường. Tính chiều dài đoạn đường mà mỗi đội phải làm theo kế hoạch mới.
\end{baitoan}

%------------------------------------------------------------------------------%

\printbibliography[heading=bibintoc]
	
\end{document}