\documentclass{article}
\usepackage[backend=biber,natbib=true,style=alphabetic,maxbibnames=50]{biblatex}
\addbibresource{/home/nqbh/reference/bib.bib}
\usepackage[utf8]{vietnam}
\usepackage{tocloft}
\renewcommand{\cftsecleader}{\cftdotfill{\cftdotsep}}
\usepackage[colorlinks=true,linkcolor=blue,urlcolor=red,citecolor=magenta]{hyperref}
\usepackage{amsmath,amssymb,amsthm,float,graphicx,mathtools}
\allowdisplaybreaks
\newtheorem{assumption}{Assumption}
\newtheorem{baitoan}{}
\newtheorem{cauhoi}{Câu hỏi}
\newtheorem{conjecture}{Conjecture}
\newtheorem{corollary}{Corollary}
\newtheorem{dangtoan}{Dạng toán}
\newtheorem{definition}{Definition}
\newtheorem{dinhly}{Định lý}
\newtheorem{dinhnghia}{Định nghĩa}
\newtheorem{example}{Example}
\newtheorem{ghichu}{Ghi chú}
\newtheorem{hequa}{Hệ quả}
\newtheorem{hypothesis}{Hypothesis}
\newtheorem{lemma}{Lemma}
\newtheorem{luuy}{Lưu ý}
\newtheorem{nhanxet}{Nhận xét}
\newtheorem{notation}{Notation}
\newtheorem{note}{Note}
\newtheorem{principle}{Principle}
\newtheorem{problem}{Problem}
\newtheorem{proposition}{Proposition}
\newtheorem{question}{Question}
\newtheorem{remark}{Remark}
\newtheorem{theorem}{Theorem}
\newtheorem{vidu}{Ví dụ}
\usepackage[left=1cm,right=1cm,top=5mm,bottom=5mm,footskip=4mm]{geometry}
\def\labelitemii{$\circ$}
\DeclareRobustCommand{\divby}{%
	\mathrel{\vbox{\baselineskip.65ex\lineskiplimit0pt\hbox{.}\hbox{.}\hbox{.}}}%
}

\title{Problem: Set $\mathbb{Q}$ of Rationals -- Bài Tập: Tập Hợp $\mathbb{Q}$ Các Số Hữu Tỷ}
\author{Nguyễn Quản Bá Hồng\footnote{Independent Researcher, Ben Tre City, Vietnam\\e-mail: \texttt{nguyenquanbahong@gmail.com}; website: \url{https://nqbh.github.io}.}}
\date{\today}

\begin{document}
\maketitle
\begin{abstract}
	Last updated version: \href{https://github.com/NQBH/elementary_STEM_beyond/blob/main/elementary_mathematics/grade_7/rational/problem/NQBH_rational_problem.pdf}{GitHub{\tt/}NQBH{\tt/}elementary STEM \& beyond{\tt/}elementary mathematics{\tt/}grade 7{\tt/}rational{\tt/}problem: set $\mathbb{Q}$ of rationals [pdf]}.\footnote{\textsc{url}: \url{https://github.com/NQBH/elementary_STEM_beyond/blob/main/elementary_mathematics/grade_7/rational/problem/NQBH_rational_problem.pdf}.} [\href{https://github.com/NQBH/elementary_STEM_beyond/blob/main/elementary_mathematics/grade_7/rational/problem/NQBH_rational_problem.tex}{\TeX}]\footnote{\textsc{url}: \url{https://github.com/NQBH/elementary_STEM_beyond/blob/main/elementary_mathematics/grade_7/rational/problem/NQBH_rational_problem.tex}.}. 
\end{abstract}
\tableofcontents

%------------------------------------------------------------------------------%

\section{Set $\mathbb{Q}$ of Rationals -- Tập Hợp $\mathbb{Q}$ Các Số Hữu Tỷ}

\begin{baitoan}[\cite{Tuyen_Toan_7}, Ví dụ 1, p. 5]
	Cho $x = \dfrac{12}{b - 15}$ với $b\in\mathbb{Z}$. Xác định $b$ để: (a) $x\in\mathbb{Q}$. (b) $x$ là 1 số hữu tỷ dương. (c) $x$ là 1 số hữu tỷ âm. (d) $0 < x < 1$.\hfill{\sf Ans: (a) $b\ne 15$. (b) $b > 15$. (c) $b < 15$. (d) $b > 27$.}
\end{baitoan}

\begin{baitoan}[\cite{Tuyen_Toan_7}, Ví dụ 2, p. 5]
	So sánh: $\dfrac{-16}{27},\dfrac{-16}{29},\dfrac{-19}{27}$.\hfill{\sf Ans: $\dfrac{-19}{27} < \dfrac{-16}{27} < \dfrac{-16}{29}$.}
\end{baitoan}

\begin{baitoan}[\cite{Tuyen_Toan_7}, 1., p. 5]
	Cho 2 số hữu tỷ $x = \dfrac{-5}{7}$, $y = \dfrac{-2}{3}$. 2 số hữu tỷ này còn được biểu diễn bởi phân số nào trong các phân số sau: $\dfrac{9}{11},\dfrac{4}{-6},\dfrac{15}{-21},\dfrac{-35}{49},\dfrac{-10}{15},\dfrac{-6}{-9}$.\hfill{\sf Ans: $x = \dfrac{15}{-21} = \dfrac{-35}{49}$, $y = \dfrac{4}{-6} = \dfrac{-10}{15}$.}
\end{baitoan}

\begin{baitoan}[\cite{Tuyen_Toan_7}, 2., p. 6]
	Sắp xếp các số hữu tỷ sau theo thứ tự tăng dần: (a) $\dfrac{19}{33},\dfrac{6}{11},\dfrac{13}{22}$. (b) $\dfrac{-18}{12},\dfrac{-10}{7},\dfrac{-8}{5}$.\\\mbox{}\hfill{\sf Ans: (a) $\dfrac{6}{11} < \dfrac{19}{33} < \dfrac{13}{22}$. (b) $\dfrac{-8}{5} < \dfrac{-18}{12} < \dfrac{-10}{7}$.}
\end{baitoan}

\begin{baitoan}[\cite{Tuyen_Toan_7}, 3., p. 6]
	So sánh các số hữu tỷ sau bằng cách nhanh nhất: (a) $-5$ \& $\dfrac{1}{63}$. (b) $\dfrac{-18}{17}$ \& $\dfrac{-999}{1000}$. (c) $\dfrac{-17}{35}$ \& $\dfrac{-43}{85}$. (d) $-0.76$ \& $\dfrac{-19}{28}$.\hfill{\sf Ans: (a) $-5 < \dfrac{1}{63}$. (b) $\dfrac{-18}{17} < \dfrac{-999}{1000}$. (c) $\dfrac{-17}{35} > \dfrac{-43}{85}$. (d) $-0.76 < \dfrac{-19}{28}$.}
\end{baitoan}

\begin{baitoan}[\cite{Tuyen_Toan_7}, 4., p. 6]
	Tìm các số hữu tỷ biểu diễn dưới dạng phân số có mẫu số bằng $10$, lớn hơn $\dfrac{-7}{13}$ nhưng nhỏ hơn $\dfrac{-4}{13}$.
\end{baitoan}

\begin{baitoan}[\cite{Tuyen_Toan_7}, 5., p. 6]
	Dùng $4$ chữ số $1$ \& dấu $-$ (nếu cần thiết) để biểu diễn (không dùng phép tính lũy thừa): (a) Các số nguyên $-1$, $-111$. (b) Số hữu tỷ âm lớn nhất.
\end{baitoan}

\begin{baitoan}[\cite{Tuyen_Toan_7}, 6., p. 6]
	Cho các số nguyên dương $a < b < c < d < m < n$. Chứng minh: $\dfrac{a + c + m}{ a + b + c + d + m + n} < \dfrac{1}{2}$.
\end{baitoan}

\begin{baitoan}[\cite{Tuyen_Toan_7}, 7., p. 6]
	Với cùng 1 khối lượng thành phẩm, vàng 4 số 9 \& vàng 3 số 9, loại nào có hàm lượng vàng nhiều hơn?
\end{baitoan}

%------------------------------------------------------------------------------%

\section{Basic Calculus on $\mathbb{Q}$ -- Phép $\pm,\cdot,:$ Số Hữu Tỷ}

\begin{baitoan}[\cite{Tuyen_Toan_7}, Ví dụ 3, p. 7]
	Tính bằng cách hợp lý (nếu có thể): (a) $-\dfrac{5}{18} + \dfrac{32}{45} - \dfrac{9}{10}$. (b) $\left(-\dfrac{1}{4} + \dfrac{7}{33} - \dfrac{5}{3}\right) - \left(-\dfrac{15}{12} + \dfrac{6}{11} - \dfrac{48}{49}\right)$.\\\mbox{}\hfill{\sf Ans: (a) $-\dfrac{7}{15}$. (b) $-\dfrac{1}{49}$.}
\end{baitoan}

\begin{baitoan}[\cite{Tuyen_Toan_7}, Ví dụ 4, p. 7]
	So sánh các tích sau bằng cách hợp lý nhất: $P_1 = \left(-\dfrac{43}{51}\right)\cdot\left(\dfrac{-19}{80}\right)$, $P_2 = \left(-\dfrac{7}{13}\right)\cdot\left(-\dfrac{4}{65}\right)\cdot\left(-\dfrac{8}{31}\right)$, $P_3 = \dfrac{-5}{10}\cdot\dfrac{-4}{10}\cdot\dfrac{-3}{10}\cdots\dfrac{3}{10}\cdot\dfrac{4}{10}\cdot\dfrac{5}{10}$.\hfill{\sf Ans: $P_2 < P_3 < P_1$.}
\end{baitoan}

\begin{baitoan}[\cite{Tuyen_Toan_7}, Ví dụ 5, p. 7]
	Tìm giá trị của $x\in\mathbb{Q}$ để biểu thức sau có giá trị dương  $P = (x + 5)(x + 9)$.\\\mbox{}\hfill{\sf Ans: $x > -5\lor x < -9$.}
\end{baitoan}

\begin{baitoan}[\cite{Tuyen_Toan_7}, 8., p. 7]
	Tìm $x$ biết: $\dfrac{11}{13} - \left(\dfrac{5}{42} - x\right) = -\left(\dfrac{15}{28} - \dfrac{11}{13}\right)$.\hfill{\sf Ans: $-\dfrac{5}{12}$.}
\end{baitoan}

\begin{baitoan}[\cite{Tuyen_Toan_7}, 9., p. 7]
	Cho $S = (a + b + c) - (a - b + c) + (a - b - c) + c$ với $a = 0.1$, $b = 0.01$, $c = 0.001$. Tính $S$.\hfill{\sf Ans: $S = 0.11$.}
\end{baitoan}

\begin{baitoan}[\cite{Tuyen_Toan_7}, 10., p. 7]
	Tính hợp lý: (a) $\dfrac{11}{125} - \dfrac{17}{18} - \dfrac{5}{7} + \dfrac{4}{9} + \dfrac{17}{14}$. (b) $1 - \dfrac{1}{2} + 2 - \dfrac{2}{3} + 3 - \dfrac{3}{4} + 4 - \dfrac{1}{4} - 3 - \dfrac{1}{3} - 2 - \dfrac{1}{2} - 1$.\\\mbox{}\hfill{\sf Ans: (a) $\dfrac{11}{125}$. (b) $1$.}
\end{baitoan}

\begin{baitoan}[\cite{Tuyen_Toan_7}, 11., p. 7]
	Cho các số hữu tỷ $x = \dfrac{a}{9}$ \& $y = \dfrac{b}{9}$ trong đó $a$ là các số nguyên âm liên tiếp từ $-5$ đến $-1$; $b$ là các số nguyên dương liên tiếp từ $1$ đến $8$. Tính tổng $x + y$.\hfill{\sf Ans: $\dfrac{7}{3}$.}
\end{baitoan}

\begin{baitoan}[\cite{Tuyen_Toan_7}, 12., p. 8]
	Cho $A = \dfrac{1}{2} + \dfrac{1}{4} + \dfrac{1}{8} + \dfrac{1}{16} + \dfrac{1}{32}$; $B = \dfrac{3}{2} + \dfrac{5}{4} + \dfrac{9}{8} + \dfrac{17}{16} + \dfrac{33}{32} - 6$. Tính $A$ \& $B$.\hfill{\sf Ans: $A = \dfrac{31}{32}$, $B = -\dfrac{1}{32}$.}
\end{baitoan}

\begin{baitoan}[\cite{Tuyen_Toan_7}, 13., p. 8]
	Cho $31$ số hữu tỷ sao cho bất kỳ 3 số nào trong chúng cũng có tổng là 1 số âm. Chứng minh tổng của $31$ số đó là 1 số âm.
\end{baitoan}

\begin{baitoan}[\cite{Tuyen_Toan_7}, 14., p. 8]
	Tìm $x$ biết: (a) $\left(\dfrac{1}{7}x - \dfrac{2}{7}\right)\left(-\dfrac{1}{5}x + \dfrac{3}{5}\right)\left(\dfrac{1}{3}x + \dfrac{4}{3}\right) = 0$. (b) $\dfrac{1}{6}x + \dfrac{1}{10}x - \dfrac{4}{15}x + 1 = 0$.
	\\\mbox{}\hfill{\sf Ans: (a) $x\in\{2,3,-4\}$. (b) $\overline{\exists}$.}
\end{baitoan}

\begin{baitoan}[\cite{Tuyen_Toan_7}, 15., p. 8]
	Tính sau bằng cách hợp lý nhất: (a) $\left(-\dfrac{40}{51}\cdot 0.32\cdot\dfrac{17}{20}\right):\dfrac{64}{75}$. (b) $-\dfrac{10}{11}\cdot\dfrac{8}{9} + \dfrac{7}{18}\cdot\dfrac{10}{11}$. (c) $\dfrac{3}{14}:\dfrac{1}{28} - \dfrac{13}{21}:\dfrac{1}{28} + \dfrac{29}{42}:\dfrac{1}{28} - 8$. (d) $-1\dfrac{5}{7}\cdot 15 + \dfrac{2}{7}(-15) + (-105)\cdot\left(\dfrac{2}{3} - \dfrac{4}{5} + \dfrac{1}{7}\right)$.\hfill{\sf Ans: (a) $-\dfrac{1}{4}$. (b) $-\dfrac{5}{11}$. (c) $0$. (d) $-31$.}
\end{baitoan}

\begin{baitoan}[\cite{Tuyen_Toan_7}, 16., p. 8]
	Tính giá trị các biểu thức sau: (a) $A = 7x - 2x - \dfrac{2}{3}y + \dfrac{7}{9}y$ với $x = -\dfrac{1}{10}$, $y = 4.8$. (b) $B = x + \dfrac{0.2 - 0.375 + \dfrac{5}{11}}{-0.3 + \dfrac{9}{16} - \dfrac{15}{22}}$ với $x = -\dfrac{1}{3}$.\hfill{\sf Ans: (a) $\dfrac{1}{30}$. (b) $-1$.}
\end{baitoan}

\begin{baitoan}[\cite{Tuyen_Toan_7}, 17., p. 8]
	Tìm giá trị của $x$ để các biểu thức sau có giá trị dương: (a) $A = x^2 + 4x$. (b) $B = (x - 3)(x + 7)$. (c) $C = \left(\dfrac{1}{2} - x\right)\left(\dfrac{1}{3} - x\right)$.\hfill{\sf Ans: (a) $x > 0\lor x < -4$. (b) $x > 3\lor x < -7$. (c) $x < \dfrac{1}{3}\lor x > \dfrac{1}{2}$.}
\end{baitoan}

\begin{baitoan}[\cite{Tuyen_Toan_7}, 18., p. 8]
	Tìm các giá trị của $x$ để các biểu thức sau có giá trị âm: (a) $D = x^2 - \dfrac{2}{5}x$. (b) $E = \dfrac{x - 2}{x - 6}$.\\\mbox{}\hfill{\sf Ans: (a) $0 < x < \dfrac{2}{5}$. (b) $2 < x < 6$.}
\end{baitoan}

\begin{baitoan}[\cite{Tuyen_Toan_7}, 19., p. 8]
	Tìm $x,y\in\mathbb{Q}$, $y\ne 0$ thỏa $x - y = xy = x:y$.\hfill{\sf Ans: $x = -\dfrac{1}{2}$, $y = -1$.}
\end{baitoan}

\begin{baitoan}[\cite{Tuyen_Toan_7}, 20., p. 8]
	Cho $100$ số hữu tỷ trong đó tích của bất kỳ 3 số nào cũng là 1 số âm. Chứng minh: (a) Tích của $100$ số đó là 1 số dương. (b) Tất cả $100$ số đó đều là số âm.	
\end{baitoan}

\begin{baitoan}[\cite{Binh_Toan_7_tap_1}, Ví dụ 1, p. 3]
	Tính $A = \dfrac{1}{2} - \dfrac{1}{3} - \dfrac{1}{6} - \dfrac{1}{2} - \dfrac{1}{3} - \dfrac{1}{6} - \dfrac{1}{2} - \dfrac{1}{3} - \dfrac{1}{6} - \cdots$ ($A$ có $300$ số hạng).\hfill{\sf Ans: $-99$.}
\end{baitoan}

\begin{baitoan}[\cite{Binh_Toan_7_tap_1}, Ví dụ 2, p. 4]
	Cho phân số $\dfrac{a}{b}\ne 1$.	(a) Tìm phân số $x$ sao cho nhân $x$ với $\dfrac{a}{b}$ cũng bằng cộng $x$ với $\dfrac{a}{b}$. (b) Tìm giá trị của $x$ trong câu (a) nếu $\dfrac{a}{b} = \dfrac{7}{5}$, nếu $\dfrac{a}{b} = \dfrac{8}{11}$. \hfill{\sf Ans: (a) $x = \dfrac{a}{a - b}$. (b) $x = -\dfrac{8}{3}$.}
\end{baitoan}

\begin{baitoan}[\cite{Binh_Toan_7_tap_1}, Ví dụ 3, p. 4]
	Tìm $x\in\mathbb{Q}$, $x < 0$ để $\dfrac{4}{x - 1}\in\mathbb{Z}$.\hfill{\sf Ans: $-1$, $-3$, $-\dfrac{1}{3}$.}
\end{baitoan}

\begin{baitoan}[\cite{Binh_Toan_7_tap_1}, Ví dụ 4, p. 5]
	Tân đạp xe từ trường về nhà với thời gian dự kiến. Nhưng Tân đã dùng $\dfrac{2}{3}$ thời gian dự kiến để đi $\dfrac{3}{4}$ quãng đường với vận tốc $v_1$, rồi đi quãng đường còn lại với vận tốc $v_2$ \& đã về nhà đúng thời điểm dự kiến. Tính tỷ số $v_1:v_2$.\hfill{\sf Ans: $\dfrac{3}{2}$.}
\end{baitoan}

\begin{baitoan}[\cite{Binh_Toan_7_tap_1}, Mở rộng Ví dụ 4, p. 5]
	Tân đạp xe từ trường về nhà với thời gian dự kiến. Nhưng Tân đã dùng $a$ thời gian dự kiến để đi $b$ quãng đường với vận tốc $v_1$, $a,b > 0$, $a + b < 1$, rồi đi quãng đường còn lại với vận tốc $v_2$ \& đã về nhà đúng thời điểm dự kiến. Tính tỷ số $v_1:v_2$ theo $a,b$.
\end{baitoan}

\begin{baitoan}[\cite{Binh_Toan_7_tap_1}, 1., p. 5]
	So sánh các số hữu tỷ: (a) $-\dfrac{18}{91}$ \& $-\dfrac{23}{114}$. (b) $-\dfrac{22}{35}$ \& $-\dfrac{103}{177}$.\hfill{\sf Ans: (a) $\dfrac{-18}{81} > \dfrac{-23}{114}$. (b) $\dfrac{-22}{35} < \dfrac{-103}{177}$.}
\end{baitoan}

\begin{baitoan}[\cite{Binh_Toan_7_tap_1}, 2., p. 5]
	Tìm $2$ phân số có tử bằng $9$, biết giá trị của mỗi phân số đó lớn hơn $-\dfrac{11}{13}$ \& nhỏ hơn $-\dfrac{11}{15}$.\hfill{\sf Ans: $\dfrac{9}{-11}$, $\dfrac{9}{-12}$.}
\end{baitoan}

\begin{baitoan}[\cite{Binh_Toan_7_tap_1}, 3., p. 5]
	Cho các số hữu tỷ $\dfrac{a}{b}$ \& $\dfrac{c}{d}$ với mẫu dương, trong đó $\dfrac{a}{b} < \dfrac{c}{d}$. Chứng minh: (a) $ab < bc$. (b) $\dfrac{a}{b} < \dfrac{a + c}{b + d} < \dfrac{c}{d}$.\hfill{\sf Hint: (b) Xét 2 hiệu $\dfrac{a + c}{b + d} - \dfrac{a}{b}$, $\dfrac{c}{d} - \dfrac{a + c}{b + d}$.}
\end{baitoan}

\begin{baitoan}[\cite{Binh_Toan_7_tap_1}, 4., p. 5]
	Tính: (a) $\dfrac{-2}{3} + \dfrac{3}{4} - \dfrac{-1}{6} + \dfrac{-2}{5}$. (b) $\dfrac{-2}{3} + \dfrac{-1}{5} + \dfrac{3}{4} - \dfrac{5}{6} - \dfrac{-7}{10}$. (c) $\dfrac{1}{2} - \dfrac{-2}{5} + \dfrac{1}{3} + \dfrac{5}{7} - \dfrac{-1}{6} + \dfrac{-4}{35} + \dfrac{1}{41}$. (d) $\dfrac{1}{100\cdot 99} - \dfrac{1}{99\cdot 98} - \dfrac{1}{98\cdot 97} - \cdots - \dfrac{1}{3\cdot 2} - \dfrac{1}{2\cdot 1}$.\hfill{\sf Ans: (a) $-\dfrac{3}{20}$. (b) $-\dfrac{1}{4}$. (c) $2\dfrac{1}{41}$. (d) $-\dfrac{9799}{9900}$.}
\end{baitoan}

\begin{baitoan}[\cite{Binh_Toan_7_tap_1}, 5., pp. 5--6]
	Ký hiệu $\lfloor x\rfloor$ là số nguyên lớn nhất không vượt quá $x$, được gọi là \emph{phần nguyên} của $x$, e.g., $\lfloor 1.5\rfloor = 1$, $\lfloor 5\rfloor = 5$, $\lfloor -2.5\rfloor = -3$. (a) Tính $\lfloor-\dfrac{1}{7}\rfloor,\lfloor 3.7\rfloor,\lfloor-4\rfloor,\lfloor-\dfrac{43}{10}\rfloor$. (b) Cho $x = 3.7$. So sánh: $A = \lfloor x\rfloor + \lfloor x + \dfrac{1}{5}\rfloor + \lfloor x + \dfrac{2}{5}\rfloor + \lfloor x + \dfrac{3}{5}\rfloor$ $+ \lfloor x + \dfrac{4}{5}\rfloor$ \& $B = \lfloor 5x\rfloor$. (c) Tính $ \lfloor\dfrac{100}{3}\rfloor + \lfloor\dfrac{100}{3^2}\rfloor + \lfloor\dfrac{100}{3^3}\rfloor + \lfloor\dfrac{100}{3^4}\rfloor$. (d) Tính $ \lfloor\dfrac{50}{2}\rfloor + \lfloor\dfrac{50}{2^2}\rfloor + \lfloor\dfrac{50}{2^3}\rfloor + \lfloor\dfrac{50}{2^4}\rfloor + \lfloor\dfrac{50}{2^5}\rfloor$.  (e) Cho $x\in\mathbb{Q}$. So sánh $\lfloor x\rfloor$ với $x$, so sánh $\lfloor x\rfloor$ với $y$ trong đó $y\in\mathbb{Z}$, $y < x$.\hfill{\sf Ans: (a) $-1,3,-4,-5$. (b) `$=$'. (c) $48$. (d) $47$. (e) $y\le\lfloor x\rfloor\le x$.}
\end{baitoan}

\begin{baitoan}[\cite{Binh_Toan_7_tap_1}, 6., p. 6]
	Cho các số hữu tỷ $x$ bằng $1.4089, 0.1398, -0.4771, -1.2592$. (a) Viết các số đó dưới dạng tổng của 1 số nguyên $a$ \& 1 số thập phân $b$ không âm nhỏ hơn $1$. (b) Tính tổng các số hữu tỷ trên bằng 2 cách: tính theo cách thông thường, tính tổng các số được viết dưới dạng ở (a). (c) So sánh $a$ \& $\lfloor x\rfloor$ trong trường hợp ở câu (a). Lưu ý: Trong cách viết này, $a$ là \emph{phần nguyên} của $x$, còn $b$ là \emph{phần lẻ} của $x$. Ký hiệu phần lẻ của $x$ là $\{x\}$ thì $x = \lfloor x\rfloor + \{x\}$.\hfill{\sf Ans: (a) $1 + 0.1089$, $0 + 0.1398$, $-1 + 0.5229$, $-2 + 0.7408$. (b) $-0.1876$, $-0.1876$. (c) `$=$'.}
\end{baitoan}

\begin{baitoan}[\cite{Binh_Toan_7_tap_1}, 7., p. 6]
	Tìm $n\in\mathbb{Z}$ để phân số sau có giá trị là 1 số nguyên \& tính giá trị đó: (a) $A = \dfrac{3n + 9}{n - 4}$. (b) $B = \dfrac{6n + 5}{2n - 1}$.\hfill{\sf Ans: (a) $n\in\{-17,-3,1,3,5,7,11,25\}$. (b) $n\in\{0,1\}$.}
\end{baitoan}

\begin{baitoan}[\cite{Binh_Toan_7_tap_1}, 8., p. 6]
	Tìm $x,y\in\mathbb{Z}$, biết: $\dfrac{5}{x} + \dfrac{y}{4} = \dfrac{1}{8}$.\hfill{\sf Ans: $(x,y)\in\{(40,0),(-40,1),(8,-2),(-8,3)\}$.}
\end{baitoan}

\begin{baitoan}[\cite{Binh_Toan_7_tap_1}, 9., p. 6]
	Viết tất cả các số nguyên có giá trị tuyệt đối nhỏ hơn $20$ theo thứ tự tùy ý. Lấy mỗi số trừ đi số thứ tự của nó ta được 1 hiệu. Tổng của tất cả các hiệu đó bằng bao nhiêu?\hfill{\sf Ans: $-780$.}
\end{baitoan}

\begin{baitoan}[\cite{Binh_Toan_7_tap_1}, 10., p. 6]
	Tính: (a) $\dfrac{\left(\dfrac{3}{10} - \dfrac{4}{15} - \dfrac{7}{20}\right)\cdot\dfrac{5}{19}}{\left(\dfrac{1}{14} + \dfrac{1}{7} - \dfrac{-3}{35}\right)\cdot\dfrac{-4}{3}}$. (b) $\dfrac{(1 + 2 + \cdots + 100)\left(\dfrac{1}{3} - \dfrac{1}{5} - \dfrac{1}{7} - \dfrac{1}{9}\right)\cdot(6.3\cdot 12 - 21\cdot 3.6)}{\dfrac{1}{2} + \dfrac{1}{3} + \cdots + \dfrac{1}{100}}$. (c) $\dfrac{\dfrac{1}{9} - \dfrac{1}{7} - \dfrac{1}{11}}{\dfrac{4}{9} - \dfrac{4}{7} - \dfrac{4}{11}} + \dfrac{\dfrac{3}{5} - \dfrac{3}{25} - \dfrac{3}{125} - \dfrac{3}{625}}{\dfrac{4}{5} - \dfrac{4}{25} - \dfrac{4}{125} - \dfrac{4}{625}}$.\hfill{\sf Ans: (a) $\dfrac{5}{24}$. (b) $0$. (c) $1$.}
\end{baitoan}

\begin{baitoan}[\cite{Binh_Toan_7_tap_1}, 11., p. 7]
	Tìm $x\in\mathbb{Q}$, biết: (a) $\dfrac{2}{3}x - 4 = -12$. (b) $\dfrac{3}{4} + \dfrac{1}{4}:x = -3$. (c) $|3x - 5| = 4$. (d) $\dfrac{x + 1}{10} + \dfrac{x + 1}{11} + \dfrac{x + 1}{12} = \dfrac{x + 1}{13} + \dfrac{x + 1}{14}$. (e) $\dfrac{x + 4}{2000} + \dfrac{x + 3}{2001} = \dfrac{x + 2}{2002} + \dfrac{x + 1}{2003}$.\hfill{\sf Ans: (a) $-12$. (b) $-\dfrac{1}{15}$. (c) $3$, $\dfrac{1}{3}$. (d) $-1$. (e) $-2004$.}
\end{baitoan}

\begin{baitoan}[\cite{Binh_Toan_7_tap_1}, 12., p. 7]
	Cho phân số $\dfrac{a}{b}$ với $a,b\in\mathbb{N}^\star$. Tìm phân số $x$ sao cho $\dfrac{a}{b} - x = \dfrac{a}{b}\cdot x$.\hfill{\sf Ans: $\dfrac{a}{a + b}$.}
\end{baitoan}

\begin{baitoan}[\cite{Binh_Toan_7_tap_1}, 13., p. 7]
	Trung bình cộng của 2 số lớn hơn số thứ nhất $75$\% thì nhỏ hơn số thứ 2 bao nhiêu \%?\hfill{\sf Ans: $30$\%.}
\end{baitoan}

\begin{baitoan}[\cite{Binh_Toan_7_tap_1}, 14., p. 7]
	Chứng minh: (a) $\sum_{i=1}^{99} \dfrac{i}{(i+1)!} = \dfrac{1}{2!} + \dfrac{2}{3!} + \dfrac{3}{4!} + \cdots + \dfrac{99}{100!} < 1$. (b) $\sum_{i=1}^{99} \dfrac{i(i + 1) - 1}{(i+1)!} = \dfrac{1\cdot 2 - 1}{2!} + \dfrac{2\cdot 3 - 1}{3!} + \dfrac{3\cdot 4 - 1}{4!} + \cdots + \dfrac{99\cdot 100 - 1}{100!} < 2$.	
\end{baitoan}

\begin{baitoan}[\cite{Binh_Toan_7_tap_1}, 15., p. 7]
	(a) Người ta viết $7$ số hữu tỷ trên 1 vòng tròn. Tìm các số đó, biết tích của $2$ số bất kỳ cạnh nhau bằng $16$. (b) Cũng hỏi như trên đối với $n$ số.\hfill{\sf Ans: (a) $7$ số $4$ hoặc $7$ số $-4$. (b) $n$ lẻ: $n$ số $4$ hoặc $n$ số $-4$. $n$ chẵn: $a_1 = a_3 = \cdots = a_{n-1} = m\in\mathbb{Q}$, $m\ne 0$ tùy ý, $a_2 = a_4 = \cdots = a_n = \dfrac{16}{m}$.}
\end{baitoan}

\begin{baitoan}[\cite{Binh_Toan_7_tap_1}, 16., p. 7]
	Có tồn tại hay không $2$ số dương $a,b$ khác nhau sao cho $\dfrac{1}{a} - \dfrac{1}{b} = \dfrac{1}{a - b}$?\hfill{\sf Ans: $\overline{\exists}$.}
\end{baitoan}

\begin{baitoan}[Mở rộng \cite{Binh_Toan_7_tap_1}, 16., p. 7]
	Có tồn tại hay không $2$ số $a,b$ khác nhau sao cho $\dfrac{1}{a} - \dfrac{1}{b} = \dfrac{1}{a - b}$?\hfill{\sf Ans: $\overline{\exists}$.}
\end{baitoan}

\begin{baitoan}[\cite{Binh_Toan_7_tap_1}, 17., p. 7]
	(a) Chứng minh: $\dfrac{1}{1\cdot 2} + \dfrac{1}{3\cdot 4} + \dfrac{1}{5\cdot 6} + \cdots + \dfrac{1}{49\cdot 50} = \dfrac{1}{26} + \dfrac{1}{27} + \dfrac{1}{28} + \cdots + \dfrac{1}{50}$. (b) Cho $B = \dfrac{1}{1\cdot 2} + \dfrac{1}{3\cdot 4} + \dfrac{1}{5\cdot 6} + \cdots + \dfrac{1}{99\cdot 100}$. Chứng minh $\dfrac{7}{12} < B < \dfrac{5}{6}$.		
\end{baitoan}

\begin{baitoan}[\cite{Binh_Toan_7_tap_1}, 18., p. 7]
	Tìm $a,b\in\mathbb{Q}$ sao cho: (a) $a - b = 2(a + b) = a:b$. (b) $a + b = ab = a:b$.
	\\\mbox{}\hfill{\sf Ans: (a) $a = -2.25$, $b = 0.75$. (b) $a = \dfrac{1}{2}$, $b = -1$.}
\end{baitoan}

\begin{baitoan}[\cite{Binh_Toan_7_tap_1}, 19., p. 7]
	Tìm $x\in\mathbb{Q}$, sao cho tổng của số đó với số nghịch đảo của nó là 1 số nguyên.\hfill{\sf Ans: $\pm 1$.}
\end{baitoan}

\begin{baitoan}[\cite{Binh_Toan_7_tap_1}, 20., p. 8]
	Viết tất cả các số hữu tỷ dương  thành dãy gồm các nhóm phân số có tổng của tử \& mẫu lần lượt bằng $2,3,4,5,\ldots$, các phân số trong cùng 1 nhóm được đặt trong dấu ngoặc: $\left(\dfrac{1}{1}\right),\left(\dfrac{2}{1},\dfrac{1}{2}\right),\left(\dfrac{3}{1},\dfrac{2}{2},\dfrac{1}{3}\right),\left(\dfrac{4}{1},\dfrac{3}{2},\dfrac{2}{3},\dfrac{1}{4}\right),\ldots$. Tìm phân số thứ $200$ của dãy.\hfill{\sf Ans: $\dfrac{11}{10}$.}
\end{baitoan}

%------------------------------------------------------------------------------%

\section{Exponentiation on $\mathbb{Q}$ -- Phép Tính Lũy Thừa với Số Mũ Tự Nhiên của 1 Số Hữu Tỷ}

%------------------------------------------------------------------------------%

\section{Miscellaneous}

%------------------------------------------------------------------------------%

\printbibliography[heading=bibintoc]

\end{document}