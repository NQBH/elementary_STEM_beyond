\documentclass{article}
\usepackage[backend=biber,natbib=true,style=alphabetic,maxbibnames=50]{biblatex}
\addbibresource{/home/nqbh/reference/bib.bib}
\usepackage[utf8]{vietnam}
\usepackage{tocloft}
\renewcommand{\cftsecleader}{\cftdotfill{\cftdotsep}}
\usepackage[colorlinks=true,linkcolor=blue,urlcolor=red,citecolor=magenta]{hyperref}
\usepackage{amsmath,amssymb,amsthm,float,graphicx,mathtools}
\allowdisplaybreaks
\newtheorem{assumption}{Assumption}
\newtheorem{baitoan}{}
\newtheorem{cauhoi}{Câu hỏi}
\newtheorem{conjecture}{Conjecture}
\newtheorem{corollary}{Corollary}
\newtheorem{dangtoan}{Dạng toán}
\newtheorem{definition}{Definition}
\newtheorem{dinhly}{Định lý}
\newtheorem{dinhnghia}{Định nghĩa}
\newtheorem{example}{Example}
\newtheorem{ghichu}{Ghi chú}
\newtheorem{hequa}{Hệ quả}
\newtheorem{hypothesis}{Hypothesis}
\newtheorem{lemma}{Lemma}
\newtheorem{luuy}{Lưu ý}
\newtheorem{nhanxet}{Nhận xét}
\newtheorem{notation}{Notation}
\newtheorem{note}{Note}
\newtheorem{principle}{Principle}
\newtheorem{problem}{Problem}
\newtheorem{proposition}{Proposition}
\newtheorem{question}{Question}
\newtheorem{remark}{Remark}
\newtheorem{theorem}{Theorem}
\newtheorem{vidu}{Ví dụ}
\usepackage[left=1cm,right=1cm,top=5mm,bottom=5mm,footskip=4mm]{geometry}
\def\labelitemii{$\circ$}
\DeclareRobustCommand{\divby}{%
	\mathrel{\vbox{\baselineskip.65ex\lineskiplimit0pt\hbox{.}\hbox{.}\hbox{.}}}%
}

\title{Problem: Set $\mathbb{Q}$ of Rationals -- Bài Tập: Tập Hợp $\mathbb{Q}$ Các Số Hữu Tỷ}
\author{Nguyễn Quản Bá Hồng\footnote{Independent Researcher, Ben Tre City, Vietnam\\e-mail: \texttt{nguyenquanbahong@gmail.com}; website: \url{https://nqbh.github.io}.}}
\date{\today}

\begin{document}
\maketitle
\begin{abstract}
	Last updated version: \href{https://github.com/NQBH/elementary_STEM_beyond/blob/main/elementary_mathematics/grade_7/rational/problem/NQBH_rational_problem.pdf}{GitHub{\tt/}NQBH{\tt/}elementary STEM \& beyond{\tt/}elementary mathematics{\tt/}grade 7{\tt/}rational{\tt/}problem: set $\mathbb{Q}$ of rationals [pdf]}.\footnote{\textsc{url}: \url{https://github.com/NQBH/elementary_STEM_beyond/blob/main/elementary_mathematics/grade_7/rational/problem/NQBH_rational_problem.pdf}.} [\href{https://github.com/NQBH/elementary_STEM_beyond/blob/main/elementary_mathematics/grade_7/rational/problem/NQBH_rational_problem.tex}{\TeX}]\footnote{\textsc{url}: \url{https://github.com/NQBH/elementary_STEM_beyond/blob/main/elementary_mathematics/grade_7/rational/problem/NQBH_rational_problem.tex}.}. 
\end{abstract}
\tableofcontents

%------------------------------------------------------------------------------%

\section{Set $\mathbb{Q}$ of Rationals -- Tập Hợp $\mathbb{Q}$ Các Số Hữu Tỷ}

\begin{baitoan}[\cite{Binh_boi_duong_Toan_7_tap_1}, VD1, p. 7]
	Cho $a_1 = \dfrac{1}{2},a_2 = \dfrac{-1}{2}$, $a_{n+2} = a_{n+1} + a_n$, $\forall n\in\mathbb{N}^\star$, i.e., mỗi số hạng bằng tổng của 2 số hạng liền trước nó. Tính $a_3,a_4,\ldots,a_{10}$. Tìm công thức tổng quát của $a_n$, $\forall n\in\mathbb{N}^\star$.
\end{baitoan}

\begin{baitoan}[\cite{Binh_boi_duong_Toan_7_tap_1}, VD2, p. 7]
	Tính: (a) $A = \dfrac{-90}{189} + \dfrac{45}{84} - \dfrac{75}{126}$. (b) $B = 11\dfrac{7}{21} + 8\dfrac{16}{24} - 10\dfrac{21}{35}$. (c) $C = \dfrac{37}{43}\cdot\dfrac{17}{29} - \dfrac{21}{41}\cdot\dfrac{1}{2} + \dfrac{9}{58}:1\dfrac{6}{37} - \dfrac{6}{29}:1\dfrac{20}{21}$. (d) $D = 1:\left[\left(1:\dfrac{2}{3} - 1:2\dfrac{3}{5}\right):2\dfrac{3}{13}\right]$.
\end{baitoan}

\begin{baitoan}[\cite{Binh_boi_duong_Toan_7_tap_1}, VD3, p. 7]
	Tìm $x\in\mathbb{Q}$ thỏa: (a) $\left(\dfrac{2}{5} - x\right):1\dfrac{1}{3} + \dfrac{1}{2} = -4$. (b) $\left(-3 + \dfrac{3}{x} - \dfrac{1}{3}\right):\left(1 + \dfrac{2}{5} + \dfrac{2}{3}\right) = -\dfrac{5}{4}$. (c) $\dfrac{-3x}{4}\cdot\left(\dfrac{1}{x} + \dfrac{2}{7}\right) = 0$.
\end{baitoan}

\begin{baitoan}[\cite{Binh_boi_duong_Toan_7_tap_1}, VD4, p. 8]
	Tính: (a) $A = \dfrac{1 - \dfrac{1}{3} - \dfrac{1}{1 - \dfrac{1}{3}}}{1 + \dfrac{1}{3} + \dfrac{1}{1 + \dfrac{1}{3}}}$. (b) $B = \dfrac{\dfrac{5}{7} - \dfrac{5}{17} + \dfrac{5}{37}}{\dfrac{3}{7} - \dfrac{3}{17} + \dfrac{3}{37}}:\dfrac{\dfrac{7}{5} - \dfrac{7}{4} + \dfrac{7}{3} - \dfrac{7}{2}}{\dfrac{1}{2} - \dfrac{1}{3} + \dfrac{1}{4} - \dfrac{1}{5}}$. (c) $C = \dfrac{15}{11\cdot14} + \dfrac{15}{14\cdot17} + \dfrac{15}{17\cdot20} + \cdots + \dfrac{15}{68\cdot71}$.
\end{baitoan}

\begin{baitoan}[\cite{Binh_boi_duong_Toan_7_tap_1}, VD5, p. 8]
	Cho $x = a + b - c$, $y = a - b + c$, $z = -a + b + c$. Tính $A = x + y + z$ biết $a = 0.1,b = -0.01,c = -0.001$.
\end{baitoan}

\begin{baitoan}[\cite{Binh_boi_duong_Toan_7_tap_1}, VD6, p. 9]
	Biết $3x + 2 = 2021\dfrac{1}{2}$. Có tính được $6x + 5$ không?
\end{baitoan}

\begin{baitoan}[\cite{Binh_boi_duong_Toan_7_tap_1}, VD7, p. 9]
	Viết $60$ số hữu tỷ $-1$ hoặc $+1$ thành 1 vòng tròn theo chiều kim đồng hồ sao cho tích của 3 số bất kỳ cạnh nhau bằng $-1$. Tìm tổng của $60$ số đó.
\end{baitoan}

\begin{baitoan}[\cite{Binh_boi_duong_Toan_7_tap_1}, 1.1., p. 9]
	Cho $a_1 = \dfrac{1}{2},a_2 = \dfrac{-1}{2}$, $a_{n+2} = a_{n+1} - a_n$, $\forall n\in\mathbb{N}^\star$, i.e., mỗi số hạng bằng hiệu của 2 số hạng liền trước nó. Tính $a_3,a_4,\ldots,a_{10}$. Tìm công thức tổng quát của $a_n$, $\forall n\in\mathbb{N}^\star$.
\end{baitoan}

\begin{baitoan}[\cite{Binh_boi_duong_Toan_7_tap_1}, 1.2., p. 9]
	Tính: (a) $A = \left(1 + \dfrac{2}{3} - \dfrac{3}{4}\right) - \left(1 + \dfrac{5}{4}\right) + \left(\dfrac{2}{5} - 2\right)$. (b) $B = -\dfrac{1}{10} - \dfrac{1}{100} - \dfrac{1}{1000} - \dfrac{1}{10000} - \dfrac{1}{100000} - \dfrac{1}{1000000}$. (c) $C = \left(5 - \dfrac{3}{2} - \dfrac{1}{8}\right):\left(2 - \dfrac{5}{2} - \dfrac{3}{4}\right)$. (d) $D = 3 - \dfrac{\dfrac{1}{2} + 1}{1 - \dfrac{1}{2}}$.
\end{baitoan}

\begin{baitoan}[\cite{Binh_boi_duong_Toan_7_tap_1}, 1.3., p. 10]
	Tính: (a) $A = \left(\dfrac{\dfrac{3}{5}}{\dfrac{2}{3} - \dfrac{4}{5}} - \dfrac{\dfrac{1}{2}}{\dfrac{1}{8} - 1}\right)\cdot\dfrac{\dfrac{1}{2}}{\dfrac{1}{16} - 2}$. (b) $B = 1 - \dfrac{1}{1 - \dfrac{2}{1 - \dfrac{3}{1 - \dfrac{1}{4}}}}$.
\end{baitoan}

\begin{baitoan}[\cite{Binh_boi_duong_Toan_7_tap_1}, 1.4., p. 10]
	Tìm $x$ thỏa: (a) $-\dfrac{2}{3}\cdot x + 4 = -12$. (b) $-\dfrac{3}{4} + \dfrac{1}{4}:x = -3$.
\end{baitoan}

\begin{baitoan}[\cite{Binh_boi_duong_Toan_7_tap_1}, 1.5., p. 10]
	Tính: (a) $A = -\dfrac{3}{11\cdot14} - \dfrac{3}{14\cdot17} - \dfrac{3}{17\cdot20} - \cdots - \dfrac{3}{98\cdot101}$. (b) $B = -1 - \dfrac{1}{3} - \dfrac{1}{6} - \dfrac{1}{10} - \dfrac{1}{15} - \cdots - \dfrac{1}{1225}$. (c) $C = \dfrac{2\cdot2021}{1 + \dfrac{1}{1 + 2} + \dfrac{1}{1 + 2 + 3} + \cdots + \dfrac{1}{1 + 2 + \cdots + 2021}}$.
\end{baitoan}

\begin{baitoan}[\cite{Binh_boi_duong_Toan_7_tap_1}, 1.6., p. 10]
	Tìm $x$ thỏa: (a) $\dfrac{x - 100}{24} + \dfrac{x - 98}{26} + \dfrac{x - 96}{28} = 3$.
\end{baitoan}

\begin{baitoan}[\cite{Binh_boi_duong_Toan_7_tap_1}, 1.7., p. 10]
	Tìm $x$ thỏa: (a) $\dfrac{x + 1}{2} + \dfrac{x + 1}{3} + \dfrac{x + 1}{4} = \dfrac{x + 1}{5} + \dfrac{x + 1}{6}$. (b) $\dfrac{x + 1}{2021} + \dfrac{x + 2}{2020} + \dfrac{x + 3}{2019} = \dfrac{x + 10}{2012} + \dfrac{x + 11}{2011} + \dfrac{x + 12}{2010}$.
\end{baitoan}

\begin{baitoan}[\cite{Binh_boi_duong_Toan_7_tap_1}, 1.8., p. 10]
	Tích 2 số hữu tỷ bằng hiệu của chúng. Tìm hiệu 2 số nghịch đảo của 2 số hữu tỷ đã cho.
\end{baitoan}

\begin{baitoan}[\cite{Binh_boi_duong_Toan_7_tap_1}, 1.9., p. 10]
	Cho dãy số viết theo quy luật: $-1,-2,-\dfrac{1}{2},-3,-1,-\dfrac{1}{3},-4,-1\dfrac{1}{2},-\dfrac{2}{3},-\dfrac{1}{4},-5,\ldots$ (a) Tìm quy luật của dãy số. (b) Số hạng thứ $124$ của dãy số là số nào? (c) Tìm công thức tổng quát của số hạng của dãy số.
\end{baitoan}

\begin{baitoan}[\cite{Tuyen_Toan_7}, Ví dụ 1, p. 5]
	Cho $x = \dfrac{12}{b - 15}$ với $b\in\mathbb{Z}$. Xác định $b$ để: (a) $x\in\mathbb{Q}$. (b) $x$ là 1 số hữu tỷ dương. (c) $x$ là 1 số hữu tỷ âm. (d) $0 < x < 1$.\hfill{\sf Ans: (a) $b\ne 15$. (b) $b > 15$. (c) $b < 15$. (d) $b > 27$.}
\end{baitoan}

\begin{baitoan}[\cite{Tuyen_Toan_7}, Ví dụ 2, p. 5]
	So sánh: $\dfrac{-16}{27},\dfrac{-16}{29},\dfrac{-19}{27}$.\hfill{\sf Ans: $\dfrac{-19}{27} < \dfrac{-16}{27} < \dfrac{-16}{29}$.}
\end{baitoan}

\begin{baitoan}[\cite{Tuyen_Toan_7}, 1., p. 5]
	Cho 2 số hữu tỷ $x = \dfrac{-5}{7}$, $y = \dfrac{-2}{3}$. 2 số hữu tỷ này còn được biểu diễn bởi phân số nào trong các phân số sau: $\dfrac{9}{11},\dfrac{4}{-6},\dfrac{15}{-21},\dfrac{-35}{49},\dfrac{-10}{15},\dfrac{-6}{-9}$.\hfill{\sf Ans: $x = \dfrac{15}{-21} = \dfrac{-35}{49}$, $y = \dfrac{4}{-6} = \dfrac{-10}{15}$.}
\end{baitoan}

\begin{baitoan}[\cite{Tuyen_Toan_7}, 2., p. 6]
	Sắp xếp các số hữu tỷ sau theo thứ tự tăng dần: (a) $\dfrac{19}{33},\dfrac{6}{11},\dfrac{13}{22}$. (b) $\dfrac{-18}{12},\dfrac{-10}{7},\dfrac{-8}{5}$.\\\mbox{}\hfill{\sf Ans: (a) $\dfrac{6}{11} < \dfrac{19}{33} < \dfrac{13}{22}$. (b) $\dfrac{-8}{5} < \dfrac{-18}{12} < \dfrac{-10}{7}$.}
\end{baitoan}

\begin{baitoan}[\cite{Tuyen_Toan_7}, 3., p. 6]
	So sánh các số hữu tỷ sau bằng cách nhanh nhất: (a) $-5$ \& $\dfrac{1}{63}$. (b) $\dfrac{-18}{17}$ \& $\dfrac{-999}{1000}$. (c) $\dfrac{-17}{35}$ \& $\dfrac{-43}{85}$. (d) $-0.76$ \& $\dfrac{-19}{28}$.\hfill{\sf Ans: (a) $-5 < \dfrac{1}{63}$. (b) $\dfrac{-18}{17} < \dfrac{-999}{1000}$. (c) $\dfrac{-17}{35} > \dfrac{-43}{85}$. (d) $-0.76 < \dfrac{-19}{28}$.}
\end{baitoan}

\begin{baitoan}[\cite{Tuyen_Toan_7}, 4., p. 6]
	Tìm các số hữu tỷ biểu diễn dưới dạng phân số có mẫu số bằng $10$, lớn hơn $\dfrac{-7}{13}$ nhưng nhỏ hơn $\dfrac{-4}{13}$.
\end{baitoan}

\begin{baitoan}[\cite{Tuyen_Toan_7}, 5., p. 6]
	Dùng $4$ chữ số $1$ \& dấu $-$ (nếu cần thiết) để biểu diễn (không dùng phép tính lũy thừa): (a) Các số nguyên $-1$, $-111$. (b) Số hữu tỷ âm lớn nhất.
\end{baitoan}

\begin{baitoan}[\cite{Tuyen_Toan_7}, 6., p. 6]
	Cho các số nguyên dương $a < b < c < d < m < n$. Chứng minh: $\dfrac{a + c + m}{ a + b + c + d + m + n} < \dfrac{1}{2}$.
\end{baitoan}

\begin{baitoan}[\cite{Tuyen_Toan_7}, 7., p. 6]
	Với cùng 1 khối lượng thành phẩm, vàng 4 số 9 \& vàng 3 số 9, loại nào có hàm lượng vàng nhiều hơn?
\end{baitoan}

%------------------------------------------------------------------------------%

\section{Basic Calculus on $\mathbb{Q}$ -- Phép $\pm,\cdot,:$ Số Hữu Tỷ}

\begin{baitoan}[\cite{Tuyen_Toan_7}, Ví dụ 3, p. 7]
	Tính bằng cách hợp lý (nếu có thể): (a) $-\dfrac{5}{18} + \dfrac{32}{45} - \dfrac{9}{10}$. (b) $\left(-\dfrac{1}{4} + \dfrac{7}{33} - \dfrac{5}{3}\right) - \left(-\dfrac{15}{12} + \dfrac{6}{11} - \dfrac{48}{49}\right)$.\\\mbox{}\hfill{\sf Ans: (a) $-\dfrac{7}{15}$. (b) $-\dfrac{1}{49}$.}
\end{baitoan}

\begin{baitoan}[\cite{Tuyen_Toan_7}, Ví dụ 4, p. 7]
	So sánh các tích sau bằng cách hợp lý nhất: $P_1 = \left(-\dfrac{43}{51}\right)\cdot\left(\dfrac{-19}{80}\right)$, $P_2 = \left(-\dfrac{7}{13}\right)\cdot\left(-\dfrac{4}{65}\right)\cdot\left(-\dfrac{8}{31}\right)$, $P_3 = \dfrac{-5}{10}\cdot\dfrac{-4}{10}\cdot\dfrac{-3}{10}\cdots\dfrac{3}{10}\cdot\dfrac{4}{10}\cdot\dfrac{5}{10}$.\hfill{\sf Ans: $P_2 < P_3 < P_1$.}
\end{baitoan}

\begin{baitoan}[\cite{Tuyen_Toan_7}, Ví dụ 5, p. 7]
	Tìm giá trị của $x\in\mathbb{Q}$ để biểu thức sau có giá trị dương  $P = (x + 5)(x + 9)$.\\\mbox{}\hfill{\sf Ans: $x > -5\lor x < -9$.}
\end{baitoan}

\begin{baitoan}[\cite{Tuyen_Toan_7}, 8., p. 7]
	Tìm $x$ biết: $\dfrac{11}{13} - \left(\dfrac{5}{42} - x\right) = -\left(\dfrac{15}{28} - \dfrac{11}{13}\right)$.\hfill{\sf Ans: $-\dfrac{5}{12}$.}
\end{baitoan}

\begin{baitoan}[\cite{Tuyen_Toan_7}, 9., p. 7]
	Cho $S = (a + b + c) - (a - b + c) + (a - b - c) + c$ với $a = 0.1$, $b = 0.01$, $c = 0.001$. Tính $S$.\hfill{\sf Ans: $S = 0.11$.}
\end{baitoan}

\begin{baitoan}[\cite{Tuyen_Toan_7}, 10., p. 7]
	Tính hợp lý: (a) $\dfrac{11}{125} - \dfrac{17}{18} - \dfrac{5}{7} + \dfrac{4}{9} + \dfrac{17}{14}$. (b) $1 - \dfrac{1}{2} + 2 - \dfrac{2}{3} + 3 - \dfrac{3}{4} + 4 - \dfrac{1}{4} - 3 - \dfrac{1}{3} - 2 - \dfrac{1}{2} - 1$.\\\mbox{}\hfill{\sf Ans: (a) $\dfrac{11}{125}$. (b) $1$.}
\end{baitoan}

\begin{baitoan}[\cite{Tuyen_Toan_7}, 11., p. 7]
	Cho các số hữu tỷ $x = \dfrac{a}{9}$ \& $y = \dfrac{b}{9}$ trong đó $a$ là các số nguyên âm liên tiếp từ $-5$ đến $-1$; $b$ là các số nguyên dương liên tiếp từ $1$ đến $8$. Tính tổng $x + y$.\hfill{\sf Ans: $\dfrac{7}{3}$.}
\end{baitoan}

\begin{baitoan}[\cite{Tuyen_Toan_7}, 12., p. 8]
	Cho $A = \dfrac{1}{2} + \dfrac{1}{4} + \dfrac{1}{8} + \dfrac{1}{16} + \dfrac{1}{32}$; $B = \dfrac{3}{2} + \dfrac{5}{4} + \dfrac{9}{8} + \dfrac{17}{16} + \dfrac{33}{32} - 6$. Tính $A$ \& $B$.\hfill{\sf Ans: $A = \dfrac{31}{32}$, $B = -\dfrac{1}{32}$.}
\end{baitoan}

\begin{baitoan}[\cite{Tuyen_Toan_7}, 13., p. 8]
	Cho $31$ số hữu tỷ sao cho bất kỳ 3 số nào trong chúng cũng có tổng là 1 số âm. Chứng minh tổng của $31$ số đó là 1 số âm.
\end{baitoan}

\begin{baitoan}[\cite{Tuyen_Toan_7}, 14., p. 8]
	Tìm $x$ biết: (a) $\left(\dfrac{1}{7}x - \dfrac{2}{7}\right)\left(-\dfrac{1}{5}x + \dfrac{3}{5}\right)\left(\dfrac{1}{3}x + \dfrac{4}{3}\right) = 0$. (b) $\dfrac{1}{6}x + \dfrac{1}{10}x - \dfrac{4}{15}x + 1 = 0$.
	\\\mbox{}\hfill{\sf Ans: (a) $x\in\{2,3,-4\}$. (b) $\overline{\exists}$.}
\end{baitoan}

\begin{baitoan}[\cite{Tuyen_Toan_7}, 15., p. 8]
	Tính sau bằng cách hợp lý nhất: (a) $\left(-\dfrac{40}{51}\cdot 0.32\cdot\dfrac{17}{20}\right):\dfrac{64}{75}$. (b) $-\dfrac{10}{11}\cdot\dfrac{8}{9} + \dfrac{7}{18}\cdot\dfrac{10}{11}$. (c) $\dfrac{3}{14}:\dfrac{1}{28} - \dfrac{13}{21}:\dfrac{1}{28} + \dfrac{29}{42}:\dfrac{1}{28} - 8$. (d) $-1\dfrac{5}{7}\cdot 15 + \dfrac{2}{7}(-15) + (-105)\cdot\left(\dfrac{2}{3} - \dfrac{4}{5} + \dfrac{1}{7}\right)$.\hfill{\sf Ans: (a) $-\dfrac{1}{4}$. (b) $-\dfrac{5}{11}$. (c) $0$. (d) $-31$.}
\end{baitoan}

\begin{baitoan}[\cite{Tuyen_Toan_7}, 16., p. 8]
	Tính giá trị các biểu thức sau: (a) $A = 7x - 2x - \dfrac{2}{3}y + \dfrac{7}{9}y$ với $x = -\dfrac{1}{10}$, $y = 4.8$. (b) $B = x + \dfrac{0.2 - 0.375 + \dfrac{5}{11}}{-0.3 + \dfrac{9}{16} - \dfrac{15}{22}}$ với $x = -\dfrac{1}{3}$.\hfill{\sf Ans: (a) $\dfrac{1}{30}$. (b) $-1$.}
\end{baitoan}

\begin{baitoan}[\cite{Tuyen_Toan_7}, 17., p. 8]
	Tìm giá trị của $x$ để các biểu thức sau có giá trị dương: (a) $A = x^2 + 4x$. (b) $B = (x - 3)(x + 7)$. (c) $C = \left(\dfrac{1}{2} - x\right)\left(\dfrac{1}{3} - x\right)$.\hfill{\sf Ans: (a) $x > 0\lor x < -4$. (b) $x > 3\lor x < -7$. (c) $x < \dfrac{1}{3}\lor x > \dfrac{1}{2}$.}
\end{baitoan}

\begin{baitoan}[\cite{Tuyen_Toan_7}, 18., p. 8]
	Tìm các giá trị của $x$ để các biểu thức sau có giá trị âm: (a) $D = x^2 - \dfrac{2}{5}x$. (b) $E = \dfrac{x - 2}{x - 6}$.\\\mbox{}\hfill{\sf Ans: (a) $0 < x < \dfrac{2}{5}$. (b) $2 < x < 6$.}
\end{baitoan}

\begin{baitoan}[\cite{Tuyen_Toan_7}, 19., p. 8]
	Tìm $x,y\in\mathbb{Q}$, $y\ne 0$ thỏa $x - y = xy = x:y$.\hfill{\sf Ans: $x = -\dfrac{1}{2}$, $y = -1$.}
\end{baitoan}

\begin{baitoan}[\cite{Tuyen_Toan_7}, 20., p. 8]
	Cho $100$ số hữu tỷ trong đó tích của bất kỳ 3 số nào cũng là 1 số âm. Chứng minh: (a) Tích của $100$ số đó là 1 số dương. (b) Tất cả $100$ số đó đều là số âm.	
\end{baitoan}

\begin{baitoan}[\cite{Binh_Toan_7_tap_1}, Ví dụ 1, p. 3]
	Tính $A = \dfrac{1}{2} - \dfrac{1}{3} - \dfrac{1}{6} - \dfrac{1}{2} - \dfrac{1}{3} - \dfrac{1}{6} - \dfrac{1}{2} - \dfrac{1}{3} - \dfrac{1}{6} - \cdots$ ($A$ có $300$ số hạng).\hfill{\sf Ans: $-99$.}
\end{baitoan}

\begin{baitoan}[\cite{Binh_Toan_7_tap_1}, Ví dụ 2, p. 4]
	Cho phân số $\dfrac{a}{b}\ne 1$.	(a) Tìm phân số $x$ sao cho nhân $x$ với $\dfrac{a}{b}$ cũng bằng cộng $x$ với $\dfrac{a}{b}$. (b) Tìm giá trị của $x$ trong câu (a) nếu $\dfrac{a}{b} = \dfrac{7}{5}$, nếu $\dfrac{a}{b} = \dfrac{8}{11}$. \hfill{\sf Ans: (a) $x = \dfrac{a}{a - b}$. (b) $x = -\dfrac{8}{3}$.}
\end{baitoan}

\begin{baitoan}[\cite{Binh_Toan_7_tap_1}, Ví dụ 3, p. 4]
	Tìm $x\in\mathbb{Q}$, $x < 0$ để $\dfrac{4}{x - 1}\in\mathbb{Z}$.\hfill{\sf Ans: $-1$, $-3$, $-\dfrac{1}{3}$.}
\end{baitoan}

\begin{baitoan}[\cite{Binh_Toan_7_tap_1}, Ví dụ 4, p. 5]
	Tân đạp xe từ trường về nhà với thời gian dự kiến. Nhưng Tân đã dùng $\dfrac{2}{3}$ thời gian dự kiến để đi $\dfrac{3}{4}$ quãng đường với vận tốc $v_1$, rồi đi quãng đường còn lại với vận tốc $v_2$ \& đã về nhà đúng thời điểm dự kiến. Tính $\dfrac{v_1}{v_2}$.\hfill{\sf Ans: $\dfrac{3}{2}$.}
\end{baitoan}

\begin{baitoan}[\cite{Binh_Toan_7_tap_1}, Mở rộng Ví dụ 4, p. 5]
	Tân đạp xe từ trường về nhà với thời gian dự kiến. Nhưng Tân đã dùng $a$ thời gian dự kiến để đi $b$ quãng đường với vận tốc $v_1$, $a,b > 0$, $a + b < 1$, rồi đi quãng đường còn lại với vận tốc $v_2$ \& đã về nhà đúng thời điểm dự kiến. Tính tỷ số $v_1:v_2$ theo $a,b$.
\end{baitoan}

\begin{baitoan}[\cite{Binh_Toan_7_tap_1}, 1., p. 5]
	So sánh các số hữu tỷ: (a) $-\dfrac{18}{91}$ \& $-\dfrac{23}{114}$. (b) $-\dfrac{22}{35}$ \& $-\dfrac{103}{177}$.\hfill{\sf Ans: (a) $\dfrac{-18}{81} > \dfrac{-23}{114}$. (b) $\dfrac{-22}{35} < \dfrac{-103}{177}$.}
\end{baitoan}

\begin{baitoan}[\cite{Binh_Toan_7_tap_1}, 2., p. 5]
	Tìm $2$ phân số có tử bằng $9$, biết giá trị của mỗi phân số đó lớn hơn $-\dfrac{11}{13}$ \& nhỏ hơn $-\dfrac{11}{15}$.
\end{baitoan}

\begin{baitoan}[\cite{Binh_Toan_7_tap_1}, 3., p. 5]
	Cho các số hữu tỷ $\dfrac{a}{b}$ \& $\dfrac{c}{d}$ với mẫu dương, trong đó $\dfrac{a}{b} < \dfrac{c}{d}$. Chứng minh: (a) $ab < bc$. (b) $\dfrac{a}{b} < \dfrac{a + c}{b + d} < \dfrac{c}{d}$.\hfill{\sf Hint: (b) Xét 2 hiệu $\dfrac{a + c}{b + d} - \dfrac{a}{b}$, $\dfrac{c}{d} - \dfrac{a + c}{b + d}$.}
\end{baitoan}

\begin{baitoan}[\cite{Binh_Toan_7_tap_1}, 4., p. 5]
	Tính: (a) $\dfrac{-2}{3} + \dfrac{3}{4} - \dfrac{-1}{6} + \dfrac{-2}{5}$. (b) $\dfrac{-2}{3} + \dfrac{-1}{5} + \dfrac{3}{4} - \dfrac{5}{6} - \dfrac{-7}{10}$. (c) $\dfrac{1}{2} - \dfrac{-2}{5} + \dfrac{1}{3} + \dfrac{5}{7} - \dfrac{-1}{6} + \dfrac{-4}{35} + \dfrac{1}{41}$. (d) $\dfrac{1}{100\cdot 99} - \dfrac{1}{99\cdot 98} - \dfrac{1}{98\cdot 97} - \cdots - \dfrac{1}{3\cdot 2} - \dfrac{1}{2\cdot 1}$.\hfill{\sf Ans: (a) $-\dfrac{3}{20}$. (b) $-\dfrac{1}{4}$. (c) $2\dfrac{1}{41}$. (d) $-\dfrac{9799}{9900}$.}
\end{baitoan}

\begin{baitoan}[\cite{Binh_Toan_7_tap_1}, 5., pp. 5--6]
	Ký hiệu $\lfloor x\rfloor$ là số nguyên lớn nhất không vượt quá $x$, được gọi là \emph{phần nguyên} của $x$, e.g., $\lfloor 1.5\rfloor = 1$, $\lfloor 5\rfloor = 5$, $\lfloor -2.5\rfloor = -3$. (a) Tính $\left\lfloor-\dfrac{1}{7}\right\rfloor,\lfloor 3.7\rfloor,\lfloor-4\rfloor,\left\lfloor-\dfrac{43}{10}\right\rfloor$. (b) Cho $x = 3.7$. So sánh: $A = \lfloor x\rfloor + \left\lfloor x + \dfrac{1}{5}\right\rfloor + \left\lfloor x + \dfrac{2}{5}\right\rfloor + \left\lfloor x + \dfrac{3}{5}\right\rfloor + \left\lfloor x + \dfrac{4}{5}\right\rfloor$ \& $B = \lfloor 5x\rfloor$. (c) Tính $ \left\lfloor\dfrac{100}{3}\right\rfloor + \left\lfloor\dfrac{100}{3^2}\right\rfloor + \left\lfloor\dfrac{100}{3^3}\right\rfloor + \left\lfloor\dfrac{100}{3^4}\right\rfloor$. (d) Tính $\left\lfloor\dfrac{50}{2}\right\rfloor + \left\lfloor\dfrac{50}{2^2}\right\rfloor + \left\lfloor\dfrac{50}{2^3}\right\rfloor + \left\lfloor\dfrac{50}{2^4}\right\rfloor + \left\lfloor\dfrac{50}{2^5}\right\rfloor$.  (e) Cho $x\in\mathbb{Q}$. So sánh $\lfloor x\rfloor$ với $x$, so sánh $\lfloor x\rfloor$ với $y$ trong đó $y\in\mathbb{Z}$, $y < x$.\\\mbox{}\hfill{\sf Ans: (a) $-1,3,-4,-5$. (b) `$=$'. (c) $48$. (d) $47$. (e) $y\le\lfloor x\rfloor\le x$.}
\end{baitoan}

\begin{baitoan}[\cite{Binh_Toan_7_tap_1}, 6., p. 6]
	Cho các số hữu tỷ $x$ bằng $1.4089, 0.1398, -0.4771, -1.2592$. (a) Viết các số đó dưới dạng tổng của 1 số nguyên $a$ \& 1 số thập phân $b$ không âm nhỏ hơn $1$. (b) Tính tổng các số hữu tỷ trên bằng 2 cách: tính theo cách thông thường, tính tổng các số được viết dưới dạng ở (a). (c) So sánh $a$ \& $\lfloor x\rfloor$ trong trường hợp ở câu (a). Lưu ý: Trong cách viết này, $a$ là \emph{phần nguyên} của $x$, còn $b$ là \emph{phần lẻ} của $x$. Ký hiệu phần lẻ của $x$ là $\{x\}$ thì $x = \lfloor x\rfloor + \{x\}$.\\\mbox{}\hfill{\sf Ans: (a) $1 + 0.1089$, $0 + 0.1398$, $-1 + 0.5229$, $-2 + 0.7408$. (b) $-0.1876$, $-0.1876$. (c) `$=$'.}
\end{baitoan}

\begin{baitoan}[\cite{Binh_Toan_7_tap_1}, 7., p. 6]
	Tìm $n\in\mathbb{Z}$ để phân số sau có giá trị là 1 số nguyên \& tính giá trị đó: (a) $A = \dfrac{3n + 9}{n - 4}$. (b) $B = \dfrac{6n + 5}{2n - 1}$.\\\mbox{}\hfill{\sf Ans: (a) $n\in\{-17,-3,1,3,5,7,11,25\}$. (b) $n\in\{0,1\}$.}
\end{baitoan}

\begin{baitoan}[\cite{Binh_Toan_7_tap_1}, 8., p. 6]
	Tìm $x,y\in\mathbb{Z}$, biết: $\dfrac{5}{x} + \dfrac{y}{4} = \dfrac{1}{8}$.\hfill{\sf Ans: $(x,y)\in\{(40,0),(-40,1),(8,-2),(-8,3)\}$.}
\end{baitoan}

\begin{baitoan}[\cite{Binh_Toan_7_tap_1}, 9., p. 6]
	Viết tất cả các số nguyên có giá trị tuyệt đối nhỏ hơn $20$ theo thứ tự tùy ý. Lấy mỗi số trừ đi số thứ tự của nó ta được 1 hiệu. Tổng của tất cả các hiệu đó bằng bao nhiêu?\hfill{\sf Ans: $-780$.}
\end{baitoan}

\begin{baitoan}[\cite{Binh_Toan_7_tap_1}, 10., p. 6]
	Tính: (a) $\dfrac{\left(\dfrac{3}{10} - \dfrac{4}{15} - \dfrac{7}{20}\right)\cdot\dfrac{5}{19}}{\left(\dfrac{1}{14} + \dfrac{1}{7} - \dfrac{-3}{35}\right)\cdot\dfrac{-4}{3}}$. (b) $\dfrac{(1 + 2 + \cdots + 100)\left(\dfrac{1}{3} - \dfrac{1}{5} - \dfrac{1}{7} - \dfrac{1}{9}\right)\cdot(6.3\cdot 12 - 21\cdot 3.6)}{\dfrac{1}{2} + \dfrac{1}{3} + \cdots + \dfrac{1}{100}}$.\\(c) $\dfrac{\dfrac{1}{9} - \dfrac{1}{7} - \dfrac{1}{11}}{\dfrac{4}{9} - \dfrac{4}{7} - \dfrac{4}{11}} + \dfrac{\dfrac{3}{5} - \dfrac{3}{25} - \dfrac{3}{125} - \dfrac{3}{625}}{\dfrac{4}{5} - \dfrac{4}{25} - \dfrac{4}{125} - \dfrac{4}{625}}$.\hfill{\sf Ans: (a) $\dfrac{5}{24}$. (b) $0$. (c) $1$.}
\end{baitoan}

\begin{baitoan}[\cite{Binh_Toan_7_tap_1}, 11., p. 7]
	Tìm $x\in\mathbb{Q}$, biết: (a) $\dfrac{2}{3}x - 4 = -12$. (b) $\dfrac{3}{4} + \dfrac{1}{4}:x = -3$. (c) $|3x - 5| = 4$. (d) $\dfrac{x + 1}{10} + \dfrac{x + 1}{11} + \dfrac{x + 1}{12} = \dfrac{x + 1}{13} + \dfrac{x + 1}{14}$. (e) $\dfrac{x + 4}{2000} + \dfrac{x + 3}{2001} = \dfrac{x + 2}{2002} + \dfrac{x + 1}{2003}$.\hfill{\sf Ans: (a) $-12$. (b) $-\dfrac{1}{15}$. (c) $3$, $\dfrac{1}{3}$. (d) $-1$. (e) $-2004$.}
\end{baitoan}

\begin{baitoan}[\cite{Binh_Toan_7_tap_1}, 12., p. 7]
	Cho phân số $\dfrac{a}{b}$ với $a,b\in\mathbb{N}^\star$. Tìm phân số $x$ sao cho $\dfrac{a}{b} - x = \dfrac{a}{b}\cdot x$.\hfill{\sf Ans: $\dfrac{a}{a + b}$.}
\end{baitoan}

\begin{baitoan}[\cite{Binh_Toan_7_tap_1}, 13., p. 7]
	Trung bình cộng của 2 số lớn hơn số thứ nhất $75$\% thì nhỏ hơn số thứ 2 bao nhiêu \%?\hfill{\sf Ans: $30$\%.}
\end{baitoan}

\begin{baitoan}[\cite{Binh_Toan_7_tap_1}, 14., p. 7]
	Chứng minh: (a) $\sum_{i=1}^{99} \dfrac{i}{(i+1)!} = \dfrac{1}{2!} + \dfrac{2}{3!} + \dfrac{3}{4!} + \cdots + \dfrac{99}{100!} < 1$. (b) $\sum_{i=1}^{99} \dfrac{i(i + 1) - 1}{(i+1)!} = \dfrac{1\cdot 2 - 1}{2!} + \dfrac{2\cdot 3 - 1}{3!} + \dfrac{3\cdot 4 - 1}{4!} + \cdots + \dfrac{99\cdot 100 - 1}{100!} < 2$.	
\end{baitoan}

\begin{baitoan}[\cite{Binh_Toan_7_tap_1}, 15., p. 7]
	(a) Người ta viết $7$ số hữu tỷ trên 1 vòng tròn. Tìm các số đó, biết tích của $2$ số bất kỳ cạnh nhau bằng $16$. (b) Cũng hỏi như trên đối với $n$ số.\hfill{\sf Ans: (a) $7$ số $4$ hoặc $7$ số $-4$. (b) $n$ lẻ: $n$ số $4$ hoặc $n$ số $-4$. $n$ chẵn: $a_1 = a_3 = \cdots = a_{n-1} = m\in\mathbb{Q}$, $m\ne 0$ tùy ý, $a_2 = a_4 = \cdots = a_n = \dfrac{16}{m}$.}
\end{baitoan}

\begin{baitoan}[\cite{Binh_Toan_7_tap_1}, 16., p. 7]
	Có tồn tại hay không $2$ số dương $a,b$ khác nhau sao cho $\dfrac{1}{a} - \dfrac{1}{b} = \dfrac{1}{a - b}$?\hfill{\sf Ans: $\overline{\exists}$.}
\end{baitoan}

\begin{baitoan}[Mở rộng \cite{Binh_Toan_7_tap_1}, 16., p. 7]
	Có tồn tại hay không $2$ số $a,b$ khác nhau sao cho $\dfrac{1}{a} - \dfrac{1}{b} = \dfrac{1}{a - b}$?\hfill{\sf Ans: $\overline{\exists}$.}
\end{baitoan}

\begin{baitoan}[\cite{Binh_Toan_7_tap_1}, 17., p. 7]
	(a) Chứng minh: $\dfrac{1}{1\cdot 2} + \dfrac{1}{3\cdot 4} + \dfrac{1}{5\cdot 6} + \cdots + \dfrac{1}{49\cdot 50} = \dfrac{1}{26} + \dfrac{1}{27} + \dfrac{1}{28} + \cdots + \dfrac{1}{50}$. (b) Cho $B = \dfrac{1}{1\cdot 2} + \dfrac{1}{3\cdot 4} + \dfrac{1}{5\cdot 6} + \cdots + \dfrac{1}{99\cdot 100}$. Chứng minh $\dfrac{7}{12} < B < \dfrac{5}{6}$.		
\end{baitoan}

\begin{baitoan}[\cite{Binh_Toan_7_tap_1}, 18., p. 7]
	Tìm $a,b\in\mathbb{Q}$ sao cho: (a) $a - b = 2(a + b) = a:b$. (b) $a + b = ab = a:b$.
	\\\mbox{}\hfill{\sf Ans: (a) $a = -2.25$, $b = 0.75$. (b) $a = \dfrac{1}{2}$, $b = -1$.}
\end{baitoan}

\begin{baitoan}[\cite{Binh_Toan_7_tap_1}, 19., p. 7]
	Tìm $x\in\mathbb{Q}$, sao cho tổng của số đó với số nghịch đảo của nó là 1 số nguyên.\hfill{\sf Ans: $\pm 1$.}
\end{baitoan}

\begin{baitoan}[\cite{Binh_Toan_7_tap_1}, 20., p. 8]
	Viết tất cả các số hữu tỷ dương  thành dãy gồm các nhóm phân số có tổng của tử \& mẫu lần lượt bằng $2,3,4,5,\ldots$, các phân số trong cùng 1 nhóm được đặt trong dấu ngoặc: $\left(\dfrac{1}{1}\right),\left(\dfrac{2}{1},\dfrac{1}{2}\right),\left(\dfrac{3}{1},\dfrac{2}{2},\dfrac{1}{3}\right),\left(\dfrac{4}{1},\dfrac{3}{2},\dfrac{2}{3},\dfrac{1}{4}\right),\ldots$. Tìm phân số thứ $200$ của dãy.\hfill{\sf Ans: $\dfrac{11}{10}$.}
\end{baitoan}

%------------------------------------------------------------------------------%

\section{Exponentiation on $\mathbb{Q}$ -- Phép Tính Lũy Thừa với Số Mũ Tự Nhiên của 1 Số Hữu Tỷ}

\begin{baitoan}[\cite{Tuyen_Toan_7}, Ví dụ 6, p. 9]
	Chứng minh: Không tồn tại 3 số hữu tỷ $x,y,z$ sao cho $xy = \dfrac{13}{15}$, $yz = \dfrac{11}{3}$, $zx = -\dfrac{3}{13}$.
\end{baitoan}

\begin{baitoan}[\cite{Tuyen_Toan_7}, Ví dụ 7, p. 9]
	Tìm $x$ biết $(3^x)^2:3^3 = \dfrac{1}{243}$.\hfill{\sf Ans: $-1$.}
\end{baitoan}

\begin{baitoan}[\cite{Tuyen_Toan_7}, Ví dụ 8, p. 9]
	Tìm $x$ biết: $(3x^2 - 51)^{2n} = (-24)^{2n}$, $n\in\mathbb{N}^\star$.\hfill{\sf Ans: $\pm3$, $\pm5$.}
\end{baitoan}

\begin{baitoan}[\cite{Tuyen_Toan_7}, 21., p. 10]
	Viết dưới dạng 1 lũy thừa với số mũ tự nhiên lớn hơn $1$: (a) $64$, $81$, $-216$. (b) $-\dfrac{1}{27}$, $\dfrac{8}{729}$, $\dfrac{16}{625}$.
\end{baitoan}

\begin{baitoan}[\cite{Tuyen_Toan_7}, 22., p. 10]
	Dùng lũy thừa với số mũ nguyên âm để viết gọn: (a) Đường kính của nguyên tử cỡ {\rm0.000 000 001 m}. (b) Đường kính của hạt nhân nguyên tử cỡ {\rm0.000 000 000 000 001 m}. (c) Khối lượng hạt nhân nguyên tử cỡ $0.\underbrace{000\ldots 00}_{23}1$ {\rm g}.
\end{baitoan}

\begin{baitoan}[\cite{Tuyen_Toan_7}, 23., p. 10]
	Viết các biểu thức sau dưới dạng lũy thừa của 1 số nguyên: (a) $12^3:(3^{-4}\cdot 64)$. (b) $\left(\dfrac{3}{7}\right)^5\cdot\left(\dfrac{7}{3}\right)^{-1}\cdot\left(\dfrac{5}{3}\right)^6:\left(\dfrac{343}{625}\right)^{-2}$. (c) $5^4\cdot 125\cdot(2.5)^{-5}\cdot 0.04$.\hfill{\sf Ans: (a) $3^7$. (b) $5^{-2}$. (c) $2^5$.}
\end{baitoan}

\begin{baitoan}[\cite{Tuyen_Toan_7}, 24., p. 10]
	Cho $A = (ax + by)^2$, $B = (a^2 + b^2)(x^2 + y^2)$. So sánh giá trị của 2 biểu thức $A$ \& $B$ biết: $a = 2$, $b = -1$, $x = \dfrac{8}{11}$, $y = \dfrac{-5}{11}$.\hfill{\sf Ans: $A = \dfrac{441}{121} < \dfrac{445}{121} = B$.}
\end{baitoan}

\begin{baitoan}[\cite{Tuyen_Toan_7}, 25., p. 10]
	So sánh $\left(\dfrac{1}{8}\right)^6$ với $\left(\dfrac{1}{32}\right)^4$.\hfill{\sf Ans: $\left(\dfrac{1}{8}\right)^6 > \left(\dfrac{1}{32}\right)^4$.}
\end{baitoan}

\begin{baitoan}[\cite{Tuyen_Toan_7}, 26., p. 10]
	So sánh $4^{30}$ với $1000\cdot 32^{10}$.\hfill{\sf Ans: $4^{30} > 1000\cdot 32^{10}$.}
\end{baitoan}

\begin{baitoan}[\cite{Tuyen_Toan_7}, 27., p. 10]
	Tìm $x$ biết: (a) $5^x\cdot(5^3)^2 = 625$. (b) $\left(\dfrac{12}{25}\right)^x = \left(\dfrac{3}{5}\right)^2 - \left(-\dfrac{3}{5}\right)^4$. (c) $\left(-\dfrac{3}{4}\right)^{3x - 1} = \dfrac{256}{81}$.\\\mbox{}\hfill{\sf Ans: (a) $-2$. (b) $2$. (c) $-1$.}
\end{baitoan}

\begin{baitoan}[\cite{Tuyen_Toan_7}, 28., p. 10]
	Tìm $x\in\mathbb{N}$ biết: (a) $8 < 2^x\le 2^9:2^5$. (b) $27 < 81^3:3^x < 243$. (c) $\left(\dfrac{2}{5}\right)^x > \left(\dfrac{5}{2}\right)^{-3}\cdot\left(-\dfrac{2}{5}\right)^2$.\\\mbox{}\hfill{\sf Ans: (a) $4$. (b) $8$. (c) $x\in\{0,1,2,3,4\}$.}
\end{baitoan}

\begin{baitoan}[\cite{Tuyen_Toan_7}, 29., p. 10]
	Tìm $x$ biết: (a) $(5x + 1)^2 = \dfrac{36}{49}$. (b) $\left(x - \dfrac{2}{9}\right)^3 = \left(\dfrac{2}{3}\right)^6$. (c) $(8x - 1)^{2n + 1} = 5^{2n + 1}$, với $n\in\mathbb{N}$ nào đó.\\\mbox{}\hfill{\sf Ans: (a) $-\dfrac{1}{35}$, $-\dfrac{13}{35}$. (b) $\dfrac{2}{3}$. (c) $\dfrac{3}{4}$.}
\end{baitoan}

\begin{baitoan}[\cite{Tuyen_Toan_7}, 30., p. 10]
	Tìm $x,y$ biết: (a) $x^2 + \left(y - \dfrac{1}{10}\right)^4 = 0$. (b) $\left(\dfrac{1}{2}x - 5\right)^{20} + \left(y^2 - \dfrac{1}{4}\right)^{10}\le 0$.\\\mbox{}\hfill{\sf Ans: (a) $x = 0$, $y = \dfrac{1}{10}$. (b) $x = 10$, $y = \pm\dfrac{1}{2}$.}
\end{baitoan}

\begin{baitoan}[\cite{Tuyen_Toan_7}, 31., p. 10]
	Tìm $x\in\mathbb{Z}$ biết: $(x - 7)^{x + 1} - (x - 7)^{x + 11} = 0$.\hfill{\sf Ans: $6,7,8$.}
\end{baitoan}

\begin{baitoan}[\cite{Tuyen_Toan_7}, 32., p. 10]
	Tìm $x,y$ biết: $x(x - y) = \dfrac{3}{10}$, $y(x - y) = -\dfrac{3}{50}$.\hfill{\sf Ans: $(x,y)\in\left\{\left(\dfrac{1}{2},-\dfrac{1}{10}\right),\left(-\dfrac{1}{2},\dfrac{1}{10}\right)\right\}$.}
\end{baitoan}

\begin{baitoan}[\cite{Tuyen_Toan_7}, 33., p. 11]
	Tìm: (a) Giá trị nhỏ nhất {\rm GTNN} của biểu thức $A = \left(2x + \dfrac{1}{3}\right)^2 - 1$. (b) Giá trị lớn nhất {\rm GTLN} của biểu thức $B = -\left(\dfrac{4}{9}x - \dfrac{2}{15}\right)^6 + 3$.\hfill{\sf Ans: (a) $\min A = -1$, $x = -\dfrac{1}{6}$. (b) $\max B = 3$, $x = \dfrac{3}{10}$.}
\end{baitoan}

\begin{baitoan}[\cite{Binh_Toan_7_tap_1}, Ví dụ 5, p. 8]
	(a) Chứng minh: $2^{10}\approx 10^3$ \& $9^{10}\approx 80^5$. (b) Dùng nhận xét ở (a) để chứng minh $9^{10}\approx 3.2\cdot 10^9$.	
\end{baitoan}

\begin{baitoan}[\cite{Binh_Toan_7_tap_1}, Ví dụ 6, p. 8]
	Tính: $A = \sum_{i=1}^{10} \dfrac{i}{2^i} = \dfrac{1}{2} + \dfrac{2}{2^2} + \dfrac{3}{2^3} + \cdots + \dfrac{10}{2^{10}}$.\hfill{\sf Ans: $\dfrac{509}{256}$.}
\end{baitoan}

\begin{baitoan}[\cite{Binh_Toan_7_tap_1}, Ví dụ 7, p. 9]
	(a) Có thể khẳng định $x^2$ luôn luôn lớn hơn $x$ hay không? (b) Khi nào thì $x^2 < x$?\\\mbox{}\hfill{\sf Ans: (a) Không. (b) $0 < x < 1$.}
\end{baitoan}

\begin{baitoan}[\cite{Binh_Toan_7_tap_1}, Ví dụ 8, p. 9]
	Tìm $a,b,c\in\mathbb{Q}$, biết: $ab = 2$, $bc = 3$, $ca = 54$.\hfill{\sf Ans: $(a,b,c)\in\left\{\left(6,\dfrac{1}{3},9\right),\left(-6,-\dfrac{1}{3},-9\right)\right\}$.}
\end{baitoan}

\begin{baitoan}[\cite{Binh_Toan_7_tap_1}, Ví dụ 9, p. 9]
	Rút gọn: $A = \sum_{i=0}^{50} 5^i = 1 + 5 + 5^2 + \cdots + 5^{49} + 5^{50}$.\hfill{\sf Ans: $\dfrac{5^{51} - 1}{4}$.}
\end{baitoan}

\begin{baitoan}[\cite{Binh_Toan_7_tap_1}, Ví dụ 10, p. 9]
	Cho $B = \sum_{i=1}^{99} \left(\dfrac{1}{2}\right)^i = \dfrac{1}{2} + \left(\dfrac{1}{2}\right)^2 + \cdots + \left(\dfrac{1}{2}\right)^{98} + \left(\dfrac{1}{2}\right)^{99}$. Chứng minh $B < 1$.
\end{baitoan}

\begin{baitoan}[\cite{Binh_Toan_7_tap_1}, 21., p. 10]
	Chứng minh: (a) $7^6 + 7^5 - 7^4\divby 55$. (b) $16^5 + 2^{15}\divby 33$. (c) $81^7 - 27^9 - 9^{13}\divby 405$.\\\mbox{}\hfill{\sf Ans: (a) $7^4\cdot 55$. (b) $2^{15}\cdot 33$. (c) $3^{26}\cdot 5$.}
\end{baitoan}

\begin{baitoan}[\cite{Binh_Toan_7_tap_1}, 22., p. 10]
	Điền vào chỗ chấm ($\cdots$) các từ ``bằng nhau'' hoặc ``đối nhau'' cho đúng: (a) Nếu 2 số đối nhau thì bình phương của chúng $\ldots$. (b) Nếu 2 số đối nhau thì lập phương của chúng $\ldots$. (c) Lũy thừa chẵn cùng bậc của 2 số đối nhau thì $\ldots$. (d)Lũy thừa lẻ cùng bậc của 2 số đối nhau thì $\ldots$.
\end{baitoan}

\begin{baitoan}[\cite{Binh_Toan_7_tap_1}, 23., p. 10 \& mở rộng]
	Các đẳng thức sau có đúng với mọi $a,b\in\mathbb{Q}$ hay không? (a) $-a^3 = (-a)^3$. (b) $-a^5 = (-a)^5$. (c) $-a^2 = (-a)^2$. (d)$-a^4 = (-a)^4$. (e)$-a^{2n+1} = (-a)^{2n+1}$, $\forall n\in\mathbb{N}$. (f) $a^{2n} = (-a)^{2n}$, $\forall n\in\mathbb{N}$. (g) $(a - b)^2 = (b - a)^2$. (h) $(a - b)^3 = -(b - a)^3$. (i) $(a - b)^{2n} = (b - a)^{2n}$, $\forall n\in\mathbb{N}$. (j) $(a - b)^{2n+1} = -(b - a)^{2n+1}$, $\forall n\in\mathbb{N}$.
\end{baitoan}

\begin{baitoan}[\cite{Binh_Toan_7_tap_1}, 24., p. 10]
	Tính: (a) $\left(\dfrac{1}{2}\right)^{15}\cdot\left(\dfrac{1}{4}\right)^{20}$. (b) $\left(\dfrac{1}{9}\right)^{25}:\left(\dfrac{1}{3}\right)^{30}$. (c) $\left(\dfrac{1}{16}\right)^3:\left(\dfrac{1}{8}\right)^2$. (d) $(x^3)^2:(x^2)^3$ với $x\ne 0$.\\\mbox{}\hfill{\sf Ans: (a) $\left(\dfrac{1}{2}\right)^{55}$. (b) $\left(\dfrac{1}{3}\right)^{30}$. (c) $\left(\dfrac{1}{2}\right)^6$. (d) $1$.}
\end{baitoan}

\begin{baitoan}[\cite{Binh_Toan_7_tap_1}, 25., p. 10]
	Viết số $64$ dưới dạng $a^n$ với $a\in\mathbb{Z}$. Có bao nhiêu cách viết?
\end{baitoan}

\begin{baitoan}[\cite{Binh_Toan_7_tap_1}, 26., p. 10]
	Rút gọn biểu thức: $A = \dfrac{4^5\cdot 9^4 - 2\cdot 6^9}{2^{10}\cdot 3^8 + 6^8\cdot 20}$.\hfill{\sf Ans: $-\dfrac{1}{3}$.}
\end{baitoan}

\begin{baitoan}[\cite{Binh_Toan_7_tap_1}, 27., p. 10]
	(a) Chứng minh: $2^{10}\approx 10^3$ \& $3^{16}\approx 80^4$. (b) Dùng nhận xét ở (a) để chứng minh $3^{16}\approx 40000000$.	
\end{baitoan}

\begin{baitoan}[\cite{Binh_Toan_7_tap_1}, 28., p. 10]
	Cho $S_n = \sum_{i=1}^{n-1} (-1)^{i-1}i = 1 - 2 + 3 - 4 + \cdots + (-1)^{n-1}n$ với $n\in\mathbb{N}^\star$. Tính $S_{35} + S_{60}$.\hfill{\sf Ans: $-12$.}
\end{baitoan}

\begin{baitoan}[\cite{Binh_Toan_7_tap_1}, 29., p. 10]
	Cho $A = 1 - 5 + 9 - 13 + 17 - 21 + 25 - \cdots$ ($n$ số hạng, giá trị tuyệt đối của số sau lớn hơn giá trị tuyệt đối của số hạng trước $4$ đơn vị, các dấu $+$ \& $-$ xen kẽ). (a) Tính $A$ theo $n$. (b) Viết số hạng thứ $n$ của biểu thức $A$ theo $n$ (chú ý dùng lũy thừa để biểu thị dấu của số hạng đó).\\\mbox{}\hfill{\sf Ans: (a) $A = -2n$ với $n$ chẵn, $A = 2n - 1$ với $n$ lẻ. (b) $(-1)^{n-1}(4n - 3) = (-1)^{n+1}(4n - 3)$.}
\end{baitoan}

\begin{baitoan}[\cite{Binh_Toan_7_tap_1}, 30., p. 11]
	Với giá trị nào của các chữ thì các biểu thức sau có giá trị là số $0$, số dương, số âm? (a) $P = \dfrac{a^2b}{c}$. (b) $Q = \dfrac{x^3}{yz}$.\hfill{\sf Ans: (a) $P = 0\Leftrightarrow a = 0$, $c\ne 0$ hoặc $b = 0$, $c\ne 0$. $P > 0\Leftrightarrow a\ne 0$, $b$ \& $c$ cùng dấu. $P < 0\Leftrightarrow a\ne 0$, $b$ \& $c$ trái dấu. (b) $Q = 0\Leftrightarrow x = 0$, $y\ne 0$, $z\ne 0$. $Q > 0\Leftrightarrow$ trong $x,y,z$ hoặc cả 3 số cùng dương hoặc có 2 số âm \& 1 số dương. $Q < 0\Leftrightarrow$ trong $x,y,z$ hoặc cả 3 số cùng âm hoặc có 1 số âm \& 2 số dương.}
\end{baitoan}

\begin{baitoan}[\cite{Binh_Toan_7_tap_1}, 31., p. 11]
	Cho $2$ số hữu tỷ $a$ \& $b$ trái dấu trong đó $|a| = b^5$. Xác định dấu của mỗi số.\hfill{\sf Ans: $a < 0$, $b > 0$.}
\end{baitoan}

\begin{baitoan}[\cite{Binh_Toan_7_tap_1}, 32., p. 11]
	Viết các số sau dưới dạng lũy thừa của $2$: $16,64,1,\dfrac{1}{32},\dfrac{1}{8},0.5,0.25$.
\end{baitoan}

\begin{baitoan}[\cite{Binh_Toan_7_tap_1}, 33., p. 11]
	(a) Viết các số sau thành lũy thừa với số mũ âm: $\dfrac{1}{1000000},0.00000002$. (b) Viết các số sau dưới dạng số thập phân: $10^{-7}$, $2.5\cdot 10^{-6}$.
\end{baitoan}

\begin{baitoan}[\cite{Binh_Toan_7_tap_1}, 34., p. 11]
	Tính xem $A$ gấp mấy lần $B$: (a) $A = 3.4\cdot 10^{-8}$, $B = 34\cdot 10^{-9}$. (b) $A = 10^{-4} + 10^{-3} + 10^{-2}$, $B = 10^{-9}$.\\\mbox{}\hfill{\sf Ans: (a) $A = B$. (b) $A = 0.0111 = 11 100 000B$.}
\end{baitoan}

\begin{baitoan}[\cite{Binh_Toan_7_tap_1}, 35., p. 11]
	So sánh: (a) $\left(-\dfrac{1}{16}\right)^{100}$ \& $\left(-\dfrac{1}{2}\right)^{500}$. (b) $(-32)^9$ \& $(-18)^{13}$. (c) $a = 2^{100}$, $b = 3^{75}$, $c = 5^{50}$.\\\mbox{}\hfill{\sf Ans: (a) $\left(-\dfrac{1}{16}\right)^{100} > \left(-\dfrac{1}{2}\right)^{500}$. (b) $(-32)^9 > (-18)^{13}$. (c) $a = 16^{25} < c = 25^{25} < b = 27^{25}$.}
\end{baitoan}

\begin{baitoan}[\cite{Binh_Toan_7_tap_1}, 36., p. 11]
	Trong các câu sau, câu nào đúng với mọi $a\in\mathbb{Q}$? (a) Nếu $a < 0$ thì $a^2 > 0$. (b) Nếu $a^2 > 0$ thì $a > 0$. (c) Nếu $a < 0$ thì $a^2 > a$. (d) Nếu $a^2 > a$ thì $a > 0$. (e) Nếu $a^2 > a$ thì $a < 0$.\hfill{\sf Ans: (a) Đ. (b) S. (c) Đ. (d) S. (e) S.}
\end{baitoan}

\begin{baitoan}[\cite{Binh_Toan_7_tap_1}, 37., p. 11]
	(a) Cho $a^m = a^n$ ($a\in\mathbb{Q}$, $m,n\in\mathbb{N}$). Tìm $m,n$. (b) Cho $a^m > a^n$ ($a\in\mathbb{Q}$, $a > 0$, $m,n\in\mathbb{N}$). So sánh $m$ \& $n$.\hfill{\sf Ans: (a) Nếu $a = 0$: $\forall m,n\in\mathbb{N}^\star$. Nếu $a = 1$: $\forall m,n\in\mathbb{N}$. Nếu $a = -1$, thì $m$ \& $n$ là các số chẵn tùy ý hoặc các số lẻ tùy ý. Nếu $a\ne 0$, $a\ne\pm 1$ thì $m = n$. (b) Nếu $a > 1$ thì $m > n$. Nếu $0 < a < 1$ thì $m < n$.}
\end{baitoan}

\begin{baitoan}[\cite{Binh_Toan_7_tap_1}, 38., p. 11]
	Tìm $x\in\mathbb{Q}$, biết: (a) $(2x - 1)^4 = 81$. (b) $(x - 1)^5 = -32$. (c) $(2x - 1)^6 = (2x - 1)^8$.\\\mbox{}\hfill{\sf Ans: (a) $-1$, $2$. (b) $-1$. (c) $0$, $\dfrac{1}{2}$, $1$.}
\end{baitoan}

\begin{baitoan}[\cite{Binh_Toan_7_tap_1}, 39., p. 11]
	Tìm $x\in\mathbb{N}$, biết: (a) $5^x + 5^{x+2} = 650$. (b) $3^{x-1} + 5\cdot 3^{x-1} = 162$.\hfill{\sf Ans: (a) $2$. (b) $4$.}
\end{baitoan}

\begin{baitoan}[\cite{Binh_Toan_7_tap_1}, 40., p. 11]
	Tìm $x,y\in\mathbb{N}$, biết: (a) $2^{x+1}\cdot 3^y = 12^x$. (b) $10^x:5^y = 20^y$.	(c) $2^x = 4^{y-1}$ \& $27^y = 3^{x+8}$.\\\mbox{}\hfill{\sf Ans: (a) $x = y = 1$. (b) $x = 2y$. (c) $x = 10$, $y = 6$.}
\end{baitoan}

\begin{baitoan}[\cite{Binh_Toan_7_tap_1}, 41., p. 11]
	Tìm $a,b,c\in\mathbb{Q}$, biết: (a) $ab = \dfrac{3}{5}$, $bc = \dfrac{4}{5}$, $ca = \dfrac{3}{4}$.	(b) $a(a + b + c) = -12$, $b(a + b + c) = 18$, $c(a + b + c) = 30$. (c) $ab = c$, $bc = 4a$, $ac = 9b$.\hfill{\sf Ans: (a) $(a,b,c)\in\left\{\left(\dfrac{3}{4},\dfrac{4}{5},1\right),\left(-\dfrac{3}{4},-\dfrac{4}{5},-1\right)\right\}$. (b) $(a,b,c)\in\{(-2,3,5),(2,-3,-5)\}$. (c) $(a,b,c)\in\{(0,0,0),(3,2,6),(-3,-2,6),(3,-2,-6),(-3,2,-6)\}$.}
\end{baitoan}

\begin{baitoan}[\cite{Binh_Toan_7_tap_1}, 42., p. 12]
	Cho $a,b,c,d,e\in\mathbb{N}$ thỏa mãn $a^b = b^c = c^d = d^e = e^a$. Chứng minh $a = b = c = d = e$.
\end{baitoan}

\begin{baitoan}[\cite{Binh_Toan_7_tap_1}, 43., p. 12]
	Cho $A = \prod_{i=2}^{100} \left(\dfrac{1}{i^2} - 1\right) = \left(\dfrac{1}{2^2} - 1\right)\left(\dfrac{1}{3^2} - 1\right)\left(\dfrac{1}{4^2} - 1\right)\cdots\left(\dfrac{1}{100^2} - 1\right)$. So sánh $A$ với $-\dfrac{1}{2}$.
\end{baitoan}

\begin{baitoan}[\cite{Binh_Toan_7_tap_1}, 44., p. 12]
	Rút gọn $A = \sum_{i=1}^{100} (-1)^i2^i = 2^{100} - 2^{99} + 2^{98} - 2^{97} + \cdots + 2^2 - 2$.\hfill{\sf Ans: $\dfrac{2^{101} - 2}{3}$.}
\end{baitoan}

\begin{baitoan}[\cite{Binh_Toan_7_tap_1}, 45., p. 12]
	Rút gọn $B = \sum_{i=0}^{100} (-1)^i3^i = 3^{100} - 3^{99} + 3^{98} - 3^{97} + \cdots + 3^2 - 3 + 1$.\hfill{\sf Ans: $\dfrac{1 + 3^{101}}{4}$.}
\end{baitoan}

\begin{baitoan}[\cite{Binh_Toan_7_tap_1}, 46., p. 12]
	Cho $C = \sum_{i=1}^{99} \dfrac{1}{3^i} = \dfrac{1}{3} + \dfrac{1}{3^2} + \cdots + \dfrac{1}{3^{99}}$. Chứng minh $C < \dfrac{1}{2}$.
\end{baitoan}

\begin{baitoan}[\cite{Binh_Toan_7_tap_1}, 47., p. 12]
	Chứng minh $\dfrac{3}{1^2\cdot 2^2} + \dfrac{5}{2^2\cdot 3^2} + \dfrac{7}{3^2\cdot 4^2} + \cdots + \dfrac{19}{9^2\cdot 10^2} < 1$.
\end{baitoan}

\begin{baitoan}[\cite{Binh_Toan_7_tap_1}, 48., p. 12]
	Chứng minh $\sum_{i=1}^{100} \dfrac{i}{3^i} = \dfrac{1}{3} + \dfrac{2}{3^2} + \dfrac{3}{3^3} + \cdots + \dfrac{100}{3^{100}} < \dfrac{3}{4}$.
\end{baitoan}

\begin{baitoan}[\cite{Binh_Toan_7_tap_1}, 49., p. 12]
	Ta không có $2^m + 2^n = 2^{m+n}$, $\forall m,n\in\mathbb{N}^\star$. Nhưng có những số nguyên dương $m,n$ có tính chất trên. Tìm các số đó.\hfill{\sf Ans: $m = n = 1$.}
\end{baitoan}

\begin{baitoan}[\cite{Binh_Toan_7_tap_1}, 50., p. 12]
	Tìm $m,n\in\mathbb{N}^\star$ sao cho $2^m - 2^n = 256$.\hfill{\sf Ans: $m = 9$, $n = 8$.}
\end{baitoan}

\begin{baitoan}[\cite{Binh_Toan_7_tap_1}, 51., p. 12]
	Cho 1 bảng vuông $3\times 3$ ô. Trong mỗi ô của bảng viết số $1$ hoặc số $-1$. Gọi $d_i$ là tích các số trên dòng $i$ ($i = 1,2,3$), $c_k$ là tích các số trên cột $k$ ($k = 1,2,3$). (a) Chứng minh không thể xảy ra $d_1 + d_2 + d_3 + c_1 + c_2 + c_3 = 0$. (b) Xét bài toán trên đối với bảng vuông $n\times n$.	
\end{baitoan}

\begin{baitoan}[\cite{Binh_Toan_7_tap_1}, 52., p. 12]
	Cho $n$ số $x_1,\ldots,x_n$, mỗi số bằng $1$ hoặc $-1$. Biết tổng của $n$ tích $x_1x_2$, $x_2x_3$, $x_3x_4,\ldots,x_nx_1$ bằng $0$. Chứng minh $n\divby4$.
\end{baitoan}

%------------------------------------------------------------------------------%

\subsection{Order of Operations. Change Side Rule -- Thứ Tự Thực Hiện Các Phép Tính. Quy Tắc Chuyển Vế}

\begin{baitoan}[\cite{Tuyen_Toan_7}, Ví dụ 9, p. 11]
	Tính: $A = \dfrac{2}{5} - \left(\dfrac{7}{10}\right)^2:\dfrac{28}{25} + \left(\dfrac{1}{2}\right)^3\cdot(-3)$.\hfill{\sf Ans: $-\dfrac{33}{80}$.}
\end{baitoan}

\begin{baitoan}[\cite{Tuyen_Toan_7}, Ví dụ 10, p. 11]
	Tính: $B = \dfrac{3 + \dfrac{1}{6} - \dfrac{2}{5}}{5 - \dfrac{1}{6} + \dfrac{7}{10}} - \dfrac{3}{2}$.\hfill{\sf Ans: $-1$.}
\end{baitoan}

\begin{baitoan}[\cite{Tuyen_Toan_7}, Ví dụ 11, p. 12]
	Tìm $x$ biết: $\dfrac{3}{7}\left(x - \dfrac{14}{9}\right) = -\dfrac{11}{7}\left(x + \dfrac{14}{11}\right)$.\hfill{\sf Ans: $-\dfrac{2}{3}$.}
\end{baitoan}

\begin{baitoan}[\cite{Tuyen_Toan_7}, 34., p. 12]
	Tính: $2\dfrac{1}{8}:1\dfrac{11}{40} - \left(2^4 - 7\dfrac{13}{18}\right):11\dfrac{1}{27}$.\hfill{\sf Ans: $\dfrac{11}{12}$.}
\end{baitoan}

\begin{baitoan}[\cite{Tuyen_Toan_7}, 35., p. 12]
	Tính: $1\dfrac{13}{15}\cdot\dfrac{3}{4} - \left[\dfrac{2^3}{4^2 - 1} + \left(\dfrac{1}{2}\right)^2\right]\cdot\dfrac{24}{47}$.\hfill{\sf Ans: $1$.}
\end{baitoan}

\begin{baitoan}[\cite{Tuyen_Toan_7}, 36., p. 12]
	Tính giá trị của biểu thức: $A = \dfrac{2 - \dfrac{1}{3} + 2^{-2}}{2 + \dfrac{1}{6} - 2^{-2}} - 2^0$.\hfill{\sf Ans: $0$.}
\end{baitoan}

\begin{baitoan}[\cite{Tuyen_Toan_7}, 37., p. 12]
	Tìm $x$ biết: $2\dfrac{1}{4}\cdot\left(x - 7\dfrac{1}{3}\right) = 1.5$.\hfill{\sf Ans: $8$.}
\end{baitoan}

\begin{baitoan}[\cite{Tuyen_Toan_7}, 38., p. 12]
	Tìm $x$ biết: $\left(12\dfrac{7}{18} - 10\dfrac{13}{18}\right):x - 1\dfrac{7}{33}:\dfrac{8}{11} = 1\dfrac{2}{3}$.\hfill{\sf Ans: $\dfrac{1}{2}$.}
\end{baitoan}

\begin{baitoan}[\cite{Tuyen_Toan_7}, 39., p. 12]
	Cho biểu thức $A = \dfrac{12}{17}:\dfrac{5}{51} - \dfrac{8}{35}\cdot 7$. (a) Tính giá trị của biểu thức $A$. (b) Đặt thêm dấu ngoặc để biểu thức $A$ có giá trị là $48.8$.\hfill{\sf Ans: (a) $\dfrac{28}{5}$. (b) $48.8$.}
\end{baitoan}

\begin{baitoan}[\cite{Tuyen_Toan_7}, 40., p. 12]
	Ông Phú gửi tiết kiệm $100$ triệu đồng tại 1 ngân hàng với kỳ hạn 1 năm, lãi suất $5$\%\emph{\texttt{/}}năm. Hết thời hạn 1 năm, tiền lãi gộp vào sổ tiền gửi ban đầu \& lại gửi theo thể thức cũ. Cứ như thế sau 3 năm thì số tiền cả gốc lẫn lãi là bao nhiêu?\hfill{\sf Ans: $115.7625$ triệu.}
\end{baitoan}

%------------------------------------------------------------------------------%

\section{Miscellaneous}

\begin{baitoan}[$\mathbb{N}^\star\subset\mathbb{N}\subset\mathbb{Z}\subset\mathbb{Q}\subset\mathbb{R}\subset\mathbb{C}$]
	Viết tập hợp theo nhiều cách nhất có thể: (a) Tập hợp các số tự nhiên. (b) Tập hợp các số tự nhiên khác $0$. (c) Tập hợp các số nguyên{\tt/}nguyên dương{\tt/}nguyên âm{\tt/}nguyên không âm{\tt/}nguyên không dương. (d) Tập hợp các phân số{\tt/}phân số dương{\tt/}phân số âm{\tt/}phân số không âm{\tt/}phân số không dương. (e) Tập hợp các số thập phân{\tt/}số thập phân dương{\tt/}số thập phân âm{\tt/}số thập phân không âm{\tt/}số thập phân không dương. (f) Tập hợp các phân số thập phân{\tt/}phân số thập phân dương{\tt/}phân số thập phân âm{\tt/}phân số thập phân không âm{\tt/}phân số thập phân không dương. (g) Tập hợp các số hữu tỷ{\tt/}số hữu tỷ dương{\tt/}số hữu tỷ âm{\tt/}số hữu tỷ không âm{\tt/}số hữu tỷ không dương. (h) Tập hợp các số thập phân hữu hạn{\tt/}số thập phân hữu hạn dương{\tt/}số thập phân hữu hạn âm{\tt/}số thập phân hữu hạn không âm{\tt/}số thập phân hữu hạn không dương. (i) Tập hợp các số thập phân vô hạn tuần hoàn{\tt/}số thập phân vô hạn tuần hoàn dương{\tt/}số thập phân vô hạn tuần hoàn âm{\tt/}số thập phân vô hạn tuần hoàn không âm{\tt/}số thập phân vô hạn tuần hoàn không dương. (j) Tập hợp các số thập phân vô hạn không tuần hoàn{\tt/}số thập phân vô hạn không tuần hoàn dương{\tt/}số thập phân vô hạn không tuần hoàn âm{\tt/}số thập phân vô hạn không tuần hoàn không âm{\tt/}số thập phân vô hạn không tuần hoàn không dương.  (k) Tập hợp các số thực{\tt/}số thực dương{\tt/}số thực âm{\tt/}số thực không âm{\tt/}số thực không dương. (l) Tập hợp các số vô tỷ{\tt/}số vô tỷ dương{\tt/}số vô tỷ âm{\tt/}số vô tỷ không dương{\tt/}số vô tỷ không âm. (m) Tập hợp các số phức{\tt/}số thuần thực{\tt/}số thuần ảo.
\end{baitoan}

%------------------------------------------------------------------------------%

\printbibliography[heading=bibintoc]

\end{document}