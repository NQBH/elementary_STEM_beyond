\documentclass{article}
\usepackage[backend=biber,natbib=true,style=alphabetic,maxbibnames=50]{biblatex}
\addbibresource{/home/nqbh/reference/bib.bib}
\usepackage[utf8]{vietnam}
\usepackage{tocloft}
\renewcommand{\cftsecleader}{\cftdotfill{\cftdotsep}}
\usepackage[colorlinks=true,linkcolor=blue,urlcolor=red,citecolor=magenta]{hyperref}
\usepackage{amsmath,amssymb,amsthm,float,graphicx,mathtools,tipa}
\usepackage{enumitem}
\setlist{leftmargin=4mm}
\allowdisplaybreaks
\newtheorem{assumption}{Assumption}
\newtheorem{baitoan}{}
\newtheorem{cauhoi}{Câu hỏi}
\newtheorem{conjecture}{Conjecture}
\newtheorem{corollary}{Corollary}
\newtheorem{dangtoan}{Dạng toán}
\newtheorem{definition}{Definition}
\newtheorem{dinhly}{Định lý}
\newtheorem{dinhnghia}{Định nghĩa}
\newtheorem{example}{Example}
\newtheorem{ghichu}{Ghi chú}
\newtheorem{hequa}{Hệ quả}
\newtheorem{hypothesis}{Hypothesis}
\newtheorem{lemma}{Lemma}
\newtheorem{luuy}{Lưu ý}
\newtheorem{nhanxet}{Nhận xét}
\newtheorem{notation}{Notation}
\newtheorem{note}{Note}
\newtheorem{principle}{Principle}
\newtheorem{problem}{Problem}
\newtheorem{proposition}{Proposition}
\newtheorem{question}{Question}
\newtheorem{remark}{Remark}
\newtheorem{theorem}{Theorem}
\newtheorem{vidu}{Ví dụ}
\usepackage[left=1cm,right=1cm,top=5mm,bottom=5mm,footskip=4mm]{geometry}
\def\labelitemii{$\circ$}
\DeclareRobustCommand{\divby}{%
	\mathrel{\vbox{\baselineskip.65ex\lineskiplimit0pt\hbox{.}\hbox{.}\hbox{.}}}%
}

\title{Problem: Set $\mathbb{Z}$ of Integers -- Bài Tập: Tập Hợp Số Nguyên $\mathbb{Z}$}
\author{Nguyễn Quản Bá Hồng\footnote{Independent Researcher, Ben Tre City, Vietnam\\e-mail: \texttt{nguyenquanbahong@gmail.com}; website: \url{https://nqbh.github.io}.}}
\date{\today}

\begin{document}
\maketitle
\begin{abstract}
	Last updated version: \href{https://github.com/NQBH/elementary_STEM_beyond/blob/main/elementary_mathematics/grade_6/natural/divisibility/problem/NQBH_divisibility_problem.pdf}{GitHub{\tt/}NQBH{\tt/}hobby{\tt/}elementary mathematics{\tt/}grade 6{\tt/}natural{\tt/}divisibility{\tt/}problem[pdf]}.\footnote{\textsc{url}: \url{https://github.com/NQBH/elementary_STEM_beyond/blob/main/elementary_mathematics/grade_6/natural/divisibility/problem/NQBH_divisibility_problem.pdf}.} [\href{https://github.com/NQBH/elementary_STEM_beyond/blob/main/elementary_mathematics/grade_6/natural/divisibility/problem/NQBH_divisibility_problem.tex}{\TeX}]\footnote{\textsc{url}: \url{https://github.com/NQBH/elementary_STEM_beyond/blob/main/elementary_mathematics/grade_6/natural/divisibility/problem/NQBH_divisibility_problem.tex}.}. 
\end{abstract}
\tableofcontents

%------------------------------------------------------------------------------%

\section{$\pm$ on $\mathbb{R}$. Bracket Rule -- Phép $\pm$ Các Số Nguyên. Quy Tắc Dấu Ngoặc}

\begin{baitoan}[\cite{Trong_Toan_6_2021}, 9., p. 59]
	Tính hợp lý: (a) $152 + (-73) - (-18) - 127$; (b) $7 + 8 + (-9) + (-10)$.
\end{baitoan}

\begin{baitoan}[\cite{Trong_Toan_6_2021}, 10., p. 59]
	Tính giá trị của biểu thức $(-156) - x$ khi: (a) $x = -26$; (b) $x = 76$; (c) $x = (-28) - (-143)$.
\end{baitoan}

\begin{baitoan}[\cite{Trong_Toan_6_2021}, 11., p. 59]
	Thay mỗi dấu $\star$ bằng 1 chữ số thích hợp: (a) $(-\overline{6\star}) + (-34) = -100$; (b) $(-789) + \overline{2\star\star} = -515$.
\end{baitoan}

\begin{baitoan}[\cite{Trong_Toan_6_2021}, 12., p. 59]
	Liệt kê các phần tử của tập hợp sau rồi tính tổng của chúng: (a) $A = \{x\in\mathbb{Z}|- 5 < x < 5\}$; (b) $B = \{x\in\mathbb{Z}|-7\le x < 1\}$.
\end{baitoan}

\begin{baitoan}[Mở rộng \cite{Trong_Toan_6_2021}, 12., p. 59]
	Cho trước $a,b\in\mathbb{Z}$. Liệt kê các phần tử của tập hợp sau rồi tính tổng của chúng: (a) $A = \{x\in\mathbb{Z}|a < x < b\}$; (b) $B = \{x\in\mathbb{Z}|a\le x < b\}$; (c) $C = \{x\in\mathbb{Z}|a < x\le b\}$; (d) $D = \{x\in\mathbb{Z}|a\le x\le b\}$; trong các trường hợp: (1) $a\ge b$; (2) $0 < a < b$; (3) $a < 0 < b$; (4) $a < b < 0$.
\end{baitoan}

\begin{baitoan}[\cite{Binh_boi_duong_Toan_6_tap_1}, H1, p. 54]
	{\rm Đ{\tt/}S?} (a) Tổng của 1 số nguyên dương với 1 số nguyên âm là 1 số nguyên âm. (b) Tổng của 1 số nguyên dương với 1 số nguyên âm là 1 số nguyên dương. (c) Tổng của 1 số nguyên dương với 1 số nguyên âm là số $0$.
\end{baitoan}

\begin{baitoan}[\cite{Binh_boi_duong_Toan_6_tap_1}, H2, p. 54]
	Archimedes là nhà bác học vĩ đại người Hy Lạp, ông sinh năm {\rm287 TCN} \& mất năm {\rm212 TCN}. Hỏi Archimedes sống thọ bao nhiêu tuổi?
\end{baitoan}

\begin{baitoan}[\cite{Binh_boi_duong_Toan_6_tap_1}, H3, p. 55]
	Cho $12$ quả bóng có ghi số \& chia thành $4$ rổ: Rổ 1: $-3,-2,19$. Rổ 2: $9,6,-2$. Rổ 3:$-5,25,-7$. Rổ 4: $-1,22,-9$.
\end{baitoan}

\begin{baitoan}[\cite{Binh_boi_duong_Toan_6_tap_1}, VD1, p. 55]
	Chứng minh $a - b$ \& $b - a$ là 2 số đối nhau.
\end{baitoan}

\begin{baitoan}[\cite{Binh_boi_duong_Toan_6_tap_1}, VD2, p. 55]
	1 tòa nhà ở Thành phố Hồ Chí Minh có $25$ tầng được đánh số các tầng theo thứ tự cao dần là $0$ (tầng trệt)), $1,2,3,\ldots,24$ \& $3$ tầng hầm được đánh số là B1, B2, B3. 1 thang máy đang ở tầng $14$, nó đi lên $3$ tầng rồi đi xuống $19$ tầng. Hỏi thang máy dừng lại ở tầng mấy?
\end{baitoan}

\begin{baitoan}[\cite{Binh_boi_duong_Toan_6_tap_1}, VD3, p. 56]
	Tính hợp lý: (a) $A = 49 + (-27 + 10 - 49 + 87)$. (b) $B = 1 + 2 - 3 - 4 + 5 + 6 - 7 - 8 + \ldots - 99 - 100 + 101$.
\end{baitoan}

\begin{baitoan}[\cite{Binh_boi_duong_Toan_6_tap_1}, VD4, p. 56]
	Tính hợp lý: (a) $A = 78 - [29 + (78 - 129)]$.
\end{baitoan}

\begin{baitoan}[\cite{Binh_boi_duong_Toan_6_tap_1}, VD5, p. 56]
	Chứng minh: $(a - b) - (b + c) + (c - a) - (a - b - c) = -(a + b - c)$.
\end{baitoan}

\begin{baitoan}[\cite{Binh_boi_duong_Toan_6_tap_1}, VD6, p. 56]
	Tìm chữ số $a$ biết $-\overline{a5} + (-92) = -157$.
\end{baitoan}

\begin{baitoan}[\cite{Binh_boi_duong_Toan_6_tap_1}, VD7, p. 57]
	Tìm $x\in\mathbb{Z}$ biết: (a) $(-x + 42) - 38 = -68 + 12$. (b) $-129 - (35 - x) = 55$.
\end{baitoan}

\begin{baitoan}[\cite{Binh_boi_duong_Toan_6_tap_1}, 8.1., p. 57]
	Tính hợp lý: (a) $(367 - 24) + (133 - 76)$. (b) $(338 - 635) - (165 - 162)$. (c) $-418 - \{-346 - 218 - [-146 - (-285) + 2015]\}$.
\end{baitoan}

\begin{baitoan}[\cite{Binh_boi_duong_Toan_6_tap_1}, 8.2., p. 57]
	Tính hợp lý: (a) $(-3) + 8 + (-13) + 18 + \ldots + (-53) + 58$. (b) $(-40) + (-39) + \cdots + 33 + 34 + 35$.
\end{baitoan}

\begin{baitoan}[\cite{Binh_boi_duong_Toan_6_tap_1}, 8.3., p. 57]
	Tìm giá trị của biểu thức: (a) $x + (-53)$ biết $x = -27$. (b) $-x + (-182)$ biết $x = -237$.
\end{baitoan}

\begin{baitoan}[\cite{Binh_boi_duong_Toan_6_tap_1}, 8.4., p. 57]
	Rút gọn biểu thức: (a) $A = -(45 + x) - (-24 - x) + (-55 - x)$. (b) $B = x - 42 - [(13 + x) - (17 - x)]$. (c) $C = -(20 + x) - [17 + (-x)]$.
\end{baitoan}

\begin{baitoan}[\cite{Binh_boi_duong_Toan_6_tap_1}, 8.5., p. 57]
	Tính $x - y$ biết điểm $x$ \& điểm $y$ đều cách điểm $0$ là $5$ đơn vị.
\end{baitoan}

\begin{baitoan}[\cite{Binh_boi_duong_Toan_6_tap_1}, 8.6., p. 57]
	Tính tổng tất cả các số nguyên $x$ thỏa mãn: (a) $-11\le x < 15$.
\end{baitoan}

\begin{baitoan}[\cite{Binh_boi_duong_Toan_6_tap_1}, 8.7., p. 57]
	Tìm chữ số $a,b\in\mathbb{N}$ biết: (a) $56 + (-\overline{a8}) = -32$. (b) $-\overline{ab7} - 45 = -172$.
\end{baitoan}

\begin{baitoan}[\cite{Binh_boi_duong_Toan_6_tap_1}, 8.8., p. 57]
	Tìm $x\in\mathbb{Z}$ biết: (a) $x + (-42) = 92 + (-52)$. (b) $x - 27 = -48 - (-72)$.
\end{baitoan}

\begin{baitoan}[\cite{Binh_boi_duong_Toan_6_tap_1}, 8.9., p. 57]
	Tìm $x\in\mathbb{Z}$ biết: (a) $57 + (7 - 32) = 319 - (x + 319)$. (b) $(76 - x) - (67 - x) = 9 - (-2 + x)$. (c) $x - \{34 - [26 + (-66 - x)]\} = 27 - \{43 + [25 - (20 - x)]\}$.
\end{baitoan}

\begin{baitoan}[\cite{Binh_boi_duong_Toan_6_tap_1}, 8.10., p. 58]
	Chứng minh đẳng thức: (a) $(a + b) - (c - d) - (a + d) = b - c$. (b) $(a - b) - (d - b) - (c - d) = a - c$.
\end{baitoan}

\begin{baitoan}[\cite{Binh_boi_duong_Toan_6_tap_1}, 8.11., p. 58]
	Cho $A = -a + b - c$, $B = a - b + c$ với $a,b,c\in\mathbb{Z}$. Chứng minh $A,B$ là 2 số đối nhau.
\end{baitoan}

\begin{baitoan}[\cite{Binh_boi_duong_Toan_6_tap_1}, 8.12., p. 58]
	Tìm $x\in\mathbb{Z}$ biết: (a) $(-2) + 4 + (-6) + 8 + \ldots + x = 2014$. (b) $1 + (-4) + 7 + (-10) + \ldots + (-x) = -3000$.
\end{baitoan}

\begin{baitoan}[\cite{Binh_boi_duong_Toan_6_tap_1}, 8.13., p. 58]
	Cho $a + b = 1$. Tính $S = -(-a + b - c) + (-c - b - a) - (a - b)$.
\end{baitoan}

\begin{baitoan}[\cite{Binh_boi_duong_Toan_6_tap_1}, 8.14., p. 58]
	Viết tất cả các số nguyên lớn hơn $-51$ nhưng nhỏ hơn $51$ theo 1 thứ tự bất kỳ. Sau đó cứ mỗi số cộng với thứ tự của nó sẽ được 1 tổng. Tính tổng tất cả các số nhận được.
\end{baitoan}

\begin{conjecture}[Goldbach conjecture -- Giả thuyết Goldbach]
	Mọi số nguyên dương chẵn lớn hơn $2$ đều có thể viết dưới dưới dạng tổng của 2 số nguyên tố.
\end{conjecture}

\begin{baitoan}[\cite{Binh_boi_duong_Toan_6_tap_1}, p. 58]
	(a) Cho $30$ số nguyên thỏa mãn: Tổng của $6$ số bất kỳ trong các số đó đều là 1 số âm. Chứng minh tổng của $30$ số nguyên đã cho cũng là 1 số âm. (b) Kết quả còn đúng không nếu thay $30$ số bởi $31$ số? (c${}^\star$) Kết quả còn đúng không nếu thay $30$ số bởi $a\in\mathbb{N}^\star$ số \& thay $6$ số bởi $b\in\mathbb{N}^\star$ số?
\end{baitoan}

\begin{baitoan}[\cite{Tuyen_Toan_6}, 187., p. 38]
	Tính tổng các số nguyên $x$ biết: $-17\le x\le18$.
\end{baitoan}

\begin{baitoan}[\cite{Tuyen_Toan_6}, 188., p. 38]
	Cho $S_1 = 1 + (-3) + 5 + (-7) + \cdots + 17$, $S_2 = -2 + 4 + (-6) + 8 + \cdots + (-18)$. Tính $S_1 + S_2$.
\end{baitoan}

\begin{baitoan}[\cite{Tuyen_Toan_6}, 189., p. 38]
	Cho $x\in\{-3,-2,-1,0,1,2,\ldots,10\}$, $y\in\{-1,0,1,2,\ldots,5\}$. Biết $x + y = 3$, tìm $x,y$.
\end{baitoan}

\begin{baitoan}[\cite{Tuyen_Toan_6}, 190., p. 38]
	1 thủ quỹ ghi số tiền thu chi trong ngày (đơn vị là nghìn đồng) như sau: $+7250,+13485,-10964,+5000$, $-1380,+24750,-9771$. Đầu ngày trong két có $500$ (nghìn đồng). Hỏi cuối ngày trong két có bao nhiêu?
\end{baitoan}

\begin{baitoan}[\cite{Tuyen_Toan_6}, 191., p. 38]
	Chứng minh số đối của tổng 2 số bằng tổng 2 số đối của chúng.
\end{baitoan}

\begin{baitoan}[Mở rộng \cite{Tuyen_Toan_6}, 191., p. 38]
	Chứng minh số đối của tổng $n$ số bằng tổng $n$ số đối của chúng với $n\in\mathbb{N}^\star$ cho trước.
\end{baitoan}

\begin{baitoan}[\cite{Tuyen_Toan_6}, 192., p. 38]
	Cho $18$ số nguyên sao cho tổng của $6$ số bất kỳ trong các số đó đều là 1 số âm. Giải thích vì sao tổng của $18$ số đó cũng là 1 số âm. Bài toán còn đúng không nếu thay $18$ số bởi $19$ số?
\end{baitoan}

\begin{baitoan}[\cite{Tuyen_Toan_6}, 192., p. 38]
	Cho trước $m,n\in\mathbb{N}^\star$. Cho $m$ số nguyên sao cho tổng của $n$ số bất kỳ trong các số đó đều là 1 số âm. Tổng của $m$ số đó có là 1 số âm hay không? Biện luận theo $m,n$.
\end{baitoan}

\begin{baitoan}[\cite{Tuyen_Toan_6}, 193., p. 38]
	Cho $x = \pm5,y = \pm11$. Tính $x + y$.
\end{baitoan}

\begin{baitoan}[\cite{Tuyen_Toan_6}, 194., p. 38]
	Cho $x = \pm7,y = \pm20$. Tính $x - y$.
\end{baitoan}

\begin{baitoan}[\cite{Tuyen_Toan_6}, 195., p. 38]
	Cho $x,y\in\mathbb{Z},-3\le x\le3,-5\le y\le5$. Biết $x - y = 2$, tìm $x,y$.
\end{baitoan}

\begin{baitoan}[\cite{Tuyen_Toan_6}, 196., p. 38]
	Cho $x\in\{-2,-1,0,1,\ldots,11\}$, $y\in\{-89,-88,-87,\ldots,-1,0,1\}$. Tìm giá trị lớn nhất (GTLN hoặc $\max$) \& giá trị nhỏ nhất (GTNN hoặc $\min$) của hiệu $x - y$.
\end{baitoan}

\begin{baitoan}[\cite{Tuyen_Toan_6}, 197., p. 38]
	Quan sát các số sau \& các số còn thiếu (?) để tìm giá trị của $x$:
	\begin{table}[H]
		\centering
		\begin{tabular}{ccccccc}
			$40$ &  & $32$ &  & $21$ &  &  $15$ \\
			& $8$ &  & ? &  & $6$ &  \\
			&  & ? &  & ? &  &  \\
			&  &  & $x$ &  &  &  \\
		\end{tabular}
	\end{table}
\end{baitoan}

\begin{baitoan}[\cite{Binh_Toan_6_tap_1}, VD49, p. 42]
	Tìm $x\in\mathbb{Z}$, biết $10 = 10 + 9 + 8 + \cdots + x$, trong đó vế phải là tổng các số nguyên liên tiếp viết theo thứ tự giảm dần.
\end{baitoan}

\begin{baitoan}[\cite{Binh_Toan_6_tap_1}, 251., p. 42]
	Tìm tổng của số nguyên âm nhỏ nhất có 1 chữ số \& số nguyên dương lớn nhất có 1 chữ số.
\end{baitoan}

\begin{baitoan}[\cite{Binh_Toan_6_tap_1}, 252., p. 42]
	Điền vào chỗ trống cho đúng: (a) Số đối của 1 số nguyên âm là 1 số $\ldots$ (b) 2 số nguyên đối nhau thì có giá trị tuyệt đối $\ldots$ (c) 2 số nguyên có giá trị tuyệt đối bằng nhau thì $\ldots$ (d) Số $\ldots$ thì nhỏ hơn số đối của nó. (e) Nếu $a\ldots$ thì $-a > 0$. (f) Nếu $a < 0$ thì $|a| = \ldots$ (g) Nếu $a < 0$ thì $a + |a| = \ldots$
\end{baitoan}

\begin{baitoan}[\cite{Binh_Toan_6_tap_1}, 253., p. 43]
	Tìm $x\in\mathbb{Z}$ biết: (a) $x + 13 = 5$. (b) $x - 1 = -9$. (c) $25 - |x| = 10$. (d) $|x - 2| + 7 = 12$. (e) $x + 4$ là số nguyên dương nhỏ nhất. (f) $10 - x$ là số nguyên âm lớn nhất.
\end{baitoan}

\begin{baitoan}[\cite{Binh_Toan_6_tap_1}, 254., p. 43]
	(a) Cho bảng vuông $3\times 3$ ô:
	\begin{table}[H]
		\centering
		\begin{tabular}{|c|c|c|}
			\hline
			$-8$ & $7$ &  \\
			\hline
			$\ \ 5$ &  & $\ \ 9$ \\
			\hline
			& $5$ & $-6$ \\
			\hline
		\end{tabular}
	\end{table}
	\noindent Điền số vào các ô trống sao cho tổng các số ở 3 dòng 1,2,3 lần lượt bằng $-5,11,1$. Tính tổng các số ở mỗi cột. (b) Cho bảng vuông $3\times 3$ ô. Có thể điền được hay không 9 số nguyên vào 9 ô của bảng sao cho tổng các số ở 3 dòng lần lượt bằng $5,-3,2$ \& tổng các số ở 3 cột lần lượt bằng $-1,2,2$?
\end{baitoan}

\begin{baitoan}[\cite{Binh_Toan_6_tap_1}, 255., p. 43]
	(a) Có $10$ ô liên tiếp trong đó ô đầu tiên ghi số $6$, ô thứ $8$ ghi số $-4$. Điền số vào các ô trống để tổng 3 số ở 3 ô liền nhau bằng $0$. (b) 1 bảng vuông $4\times 4$ ô có 2 ô ở góc trên ghi số $-3$ \& $2$. Điền số vào các ô còn lại, sao cho tổng 2 số ở 2 ô liền nhau thì bằng nhau (2 ô liền nhau là 2 ô có 1 cạnh chung).	
\end{baitoan}

\begin{baitoan}[\cite{Binh_Toan_6_tap_1}, 256., p. 43]
	Tìm $x\in\mathbb{Z}$ biết $x + (x + 1) + (x + 2) + \cdots + 19 + 20 = 20$, trong đó vế trái là tổng các số nguyên liên tiếp viết theo thứ tự tăng dần.
\end{baitoan}

\begin{baitoan}[\cite{Binh_Toan_6_tap_1}, 257., p. 43]
	Tìm $a\in\mathbb{Z}$ sao cho: (a) $a > -a$. (b) $a = -a$. (c) $a < -a$.
\end{baitoan}

\begin{baitoan}[\cite{Binh_Toan_6_tap_1}, 258., p. 43]
	Tìm $a,b,c\in\mathbb{Z}$ biết: $a + b = 11$, $b + c = 3$, $c + a = 2$.
\end{baitoan}

\begin{baitoan}[\cite{Binh_Toan_6_tap_1}, 259., p. 43]
	Tìm $a,b,c,d\in\mathbb{Z}$ biết $a + b + c + d = 1$, $a + c + d = 2$, $a + b + d = 3$, $a + b + c = 4$.
\end{baitoan}

\begin{baitoan}[\cite{Binh_Toan_6_tap_1}, 260., p. 43]
	Cho $\sum_{i=1}^{51} x_i = x_1 + x_2 + \cdots + x_{50} + x_{51} = 0$ \& $x_1 + x_2 = x_3 + x_4 = \cdots = x_{47} + x_{48} = x_{49} + x_{50} = x_{50} + x_{51} = 1$. Tính $x_{50}$.
\end{baitoan}

\begin{baitoan}[\cite{Tuyen_Toan_6}, VD42, p. 39]
	Cho $a$ là 1 số nguyên âm, còn $b,c\in\mathbb{Z}$. Chứng minh số $M = (-a + b) - (b + c - a) + (c - a)$ là 1 số nguyên dương.
\end{baitoan}

\begin{baitoan}[\cite{Tuyen_Toan_6}, 198., p. 39]
	Tính hợp lý: (a) $-2021 + (-22 + 87 + 2021)$; (b) $1152 - (374 + 1152) + (-65 + 374)$.
\end{baitoan}

\begin{baitoan}[\cite{Tuyen_Toan_6}, 199., p. 39]
	Đặt dấu ngoặc 1 cách thích hợp để tính các tổng đại số sau: (a) $942 - 2567 + 2563 - 1942$; (b) $13 - 12 + 11 + 10 - 9 + 8 - 7 - 6 + 5 - 4 + 3 + 2 - 1$.
\end{baitoan}

\begin{baitoan}[\cite{Tuyen_Toan_6}, 200., p. 39]
	Tìm $x\in\mathbb{Z}$ thỏa: (a) $461 + (x - 45) = 387$; (b) $11 - (-53 + x) = 97$; (c) $-(x + 84) + 213 = -16$.
\end{baitoan}

\begin{baitoan}[\cite{Tuyen_Toan_6}, 201., p. 39]
	Chứng minh: $-(-a + b + c) + (b + c - 1) = (b - c + 6) - (7 - a + b) + c$, $\forall a,b,c\in\mathbb{Z}$.
\end{baitoan}

\begin{baitoan}[\cite{Tuyen_Toan_6}, 202., p. 40]
	Cho $a,b,c\in\mathbb{Z}$ \& $A = a + b - 5$, $B = -b - c + 1$, $C = b - c - 4$, $D = b - a$. Chứng minh $A + B = C - D$.
\end{baitoan}

\begin{baitoan}[\cite{Tuyen_Toan_6}, 203., p. 40]
	Cho $a,b,c\in\mathbb{Z},a > b,S = -(a - b - c) + (-c + b + a) - (a + b)$. Chứng minh $S$ là 1 số nguyên âm.
\end{baitoan}

\begin{baitoan}[\cite{Tuyen_Toan_6}, 204., p. 40]
	Viết $5$ số nguyên vào $5$ đỉnh của 1 ngôi sao 5 cánh sao cho tổng của 2 số tại 2 đỉnh liền nhau luôn bằng $-6$. Tìm $5$ số nguyên đó.
\end{baitoan}

\begin{baitoan}[\cite{Tuyen_Toan_6}, 205., p. 40]
	Cho $1001$ số tự nhiên từ $1$ đến $1001$ sắp xếp theo thứ tự tùy ý. Lấy số thứ nhất trừ đi $1$, lấy số thứ 2 trừ đi $2$, lấy số thứ 3 trừ đi $3$, $\ldots$, lấy số thứ 1001 trừ đi $1001$. Tính tổng của $1001$ số mới.
\end{baitoan}

%------------------------------------------------------------------------------%

\section{Operator $\cdot$ on $\mathbb{Z}$ -- Phép $\cdot$ Số Nguyên}

\begin{baitoan}[\cite{Tuyen_Toan_6}, VD40, p. 37]
	Tính tổng $S = (-351) + (-74) + 51 + (-126) + 149$.
\end{baitoan}

\begin{baitoan}[\cite{Tuyen_Toan_6}, VD41, p. 38]
	Với $a,b\in\mathbb{Z}$, chứng minh $a - b$ \& $b - a$ là 2 số đối nhau.
\end{baitoan}

\begin{baitoan}[\cite{Tuyen_Toan_6}, 186., p. 38]
	Tính nhanh: (a) $-37 + 54 + (-70) + (-163) + 246$; (b) $-359 + 181 + (-123) + 350 + (-172)$; (c) $-69 + 53 + 46 + (-94) + (-14) + 78$.
\end{baitoan}

\begin{baitoan}[\cite{Binh_boi_duong_Toan_6_tap_1}, H1, p. 59]
	3 bạn Egg, Chicken, Bee cùng tham gia 1 trò chơi, mỗi người được tặng trước $100$ điểm. Với mỗi câu trả lời đúng, người chơi được $200$ điểm, mỗi câu trả lời sai được $-100$ điểm (bị trừ đi $100$ điểm). Sau $10$ câu hỏi, Egg trả lời đúng $5$ câu, sai $5$ câu; Chicken trả lời đúng $6$ câu, sai $4$ câu; Bee trả lời đúng $4$ câu, sai $6$ câu. Hỏi số điểm của mỗi bạn đạt được là bao nhiêu? Ai là người có số điểm cao nhất?
\end{baitoan}

\begin{baitoan}[\cite{Binh_boi_duong_Toan_6_tap_1}, H2, p. 60]
	{\rm Đ{\tt/}S?} (a) $a^2\Rightarrow a > 0$. (b) $a^2 = 0\Rightarrow a = 0$. (c) $a^2 > a\Rightarrow a < 0$. (d) $a^2 > a\Rightarrow a > 1$. (e) $a < 0\Rightarrow a^2 > a$.
\end{baitoan}

\begin{baitoan}[\cite{Binh_boi_duong_Toan_6_tap_1}, VD1, p. 60]
	1 xí nghiệp sản xuất giày có chế độ thưởng--phạt hàng tháng như sau: Làm ra 1 sản phẩm đạt chất lượng được thưởng $50000$ đồng. Làm ra 1 sản phẩm không đạt chất lượng bị phạt $40000$ đồng. Tháng này, chị Lan làm được $45$ sản phẩm đạt chất lượng \& $5$ sản phẩm không đạt chất lượng. Hỏi chị Lan nhận được bao nhiêu tiền thưởng--phạt?
\end{baitoan}

\begin{baitoan}[\cite{Binh_boi_duong_Toan_6_tap_1}, VD2, p. 60]
	Tính hợp lý: (a) $A = (162 - 62)\cdot(-27) + 73\cdot(-36 - 64)$. (b) $B = 39\cdot46 - 39\cdot76 + 30\cdot(-61)$. (c) $C = 25\cdot(75 - 49) + 75\cdot(49 - 25)$.
\end{baitoan}

\begin{baitoan}[\cite{Binh_boi_duong_Toan_6_tap_1}, VD3, p. 61]
	Bỏ dấu ngoặc rồi rút gọn biểu thức $A = (a + 1)(b - 2) - (ab - 2)$.
\end{baitoan}

\begin{baitoan}[\cite{Binh_boi_duong_Toan_6_tap_1}, VD4, p. 61]
	Tìm $x\in\mathbb{Z}$ thỏa: (a) $(x + 3)(2 - x) = 0$. (b) $(2x - 7)^2 = 25$. (c) $(1 - 3x)^3 = -8$.
\end{baitoan}

\begin{baitoan}[\cite{Binh_boi_duong_Toan_6_tap_1}, VD5, p. 62]
	Tìm $x\in\mathbb{Z}$ thỏa: $(x + 2)(x - 3) < 0$.
\end{baitoan}

\begin{baitoan}[\cite{Binh_boi_duong_Toan_6_tap_1}, VD6, p. 62]
	Tìm $a,b\in\mathbb{Z}$ thỏa: $ab = 18$ \& $a + b = -11$.
\end{baitoan}

\begin{baitoan}[\cite{Binh_boi_duong_Toan_6_tap_1}, 9.1., p. 62]
	Tính hợp lý: (a) $(-4)\cdot125\cdot(-2)\cdot8\cdot(-5)\cdot25$. (b) $(-154)\cdot67 + 154\cdot(-33) - 46$.
\end{baitoan}

\begin{baitoan}[\cite{Binh_boi_duong_Toan_6_tap_1}, 9.2., p. 62]
	Tính giá trị của biểu thức: (a) $A = 7a^2b^3$ với $a = 1$, $b = -1$. (b) $B = -9a^2b^4$ với $a = -2$, $b = -1$.
\end{baitoan}

\begin{baitoan}[\cite{Binh_boi_duong_Toan_6_tap_1}, 9.3., p. 62]
	Tính giá trị của biểu thức: (a) $ax + ay + bx + by$ biết $a + b = -5$, $x + y = 13$. (b) $ax + ay - bx - by$ biết $a - b = 6$, $x + y = -16$.
\end{baitoan}

\begin{baitoan}[\cite{Binh_boi_duong_Toan_6_tap_1}, 9.4., p. 62]
	Cho $a,b,c\in\mathbb{Z}$. Chứng minh: $a(b - c) - b(c + a) = -c(a + b)$
\end{baitoan}

\begin{baitoan}[\cite{Binh_boi_duong_Toan_6_tap_1}, 9.5., p. 62]
	Tìm $x\in\mathbb{Z}$ thỏa: (a) $5(3 - x) + 2(x - 7) = -14$. (b) $(x + 17)(25 - x) = 0$.
\end{baitoan}

\begin{baitoan}[\cite{Binh_boi_duong_Toan_6_tap_1}, 9.6., p. 62]
	Tìm $x\in\mathbb{Z}$ thỏa: (a) $(3x^2 + 2)(25 - x^2) = 0$. (b) $(x^2 - 1)(9 + 2x^2) = 0$.
\end{baitoan}

\begin{baitoan}[\cite{Binh_boi_duong_Toan_6_tap_1}, 9.7., p. 63]
	Tìm $x\in\mathbb{Z}$ thỏa: (a) $(x - 5)(8 - x) > 0$. (b) $(x^2 - 15)(x^2 - 21) < 0$.
\end{baitoan}

\begin{baitoan}[\cite{Binh_boi_duong_Toan_6_tap_1}, 9.8., p. 63]
	Tìm $x,y\in\mathbb{Z}$ thỏa: (a) $xy = -20$. (b) $(2x - 1)(4y + 2) = -30$.
\end{baitoan}

\begin{baitoan}[\cite{Binh_boi_duong_Toan_6_tap_1}, 9.9., p. 63]
	Cho $106$ số nguyên trong đó tích của $7$ số bất kỳ là 1 số âm. Chứng minh tích của tất cả $106$ số đó là 1 số dương.
\end{baitoan}

\begin{baitoan}[\cite{Binh_boi_duong_Toan_6_tap_1}, 9.10., p. 63]
	Tìm $x,y\in\mathbb{Z}$ thỏa: (a) $x + xy + y = 9$. (b) $xy + 3x - 2y = 17$.
\end{baitoan}

\begin{baitoan}[\cite{Binh_boi_duong_Toan_6_tap_1}, 9.11., p. 63]
	Chicken lấy tuổi của mình viết sau tuổi của bố thì được 1 số gồm 4 chữ số. Chicken lấy số này trừ đi hiệu số tuổi của bố \& con thì được kết quả là $4289$. Tìm số tuổi của 2 bố con Chicken.
\end{baitoan}

\begin{baitoan}[\cite{Binh_boi_duong_Toan_6_tap_1}, p. 63, Lũy thừa của số nguyên âm]
	Với $a\in\mathbb{Z}$, $a > 0$, $n\in\mathbb{N}$, chứng minh: (a) Lũy thừa bậc chẵn của 1 số nguyên âm là 1 số nguyên dương: $(-a)^{2n} = a^{2n}$. (b) Lũy thừa bậc lẻ của 1 số nguyên âm là 1 số nguyên âm: $(-a)^{2n + 1} = -a^{2n + 1}$.
\end{baitoan}

%------------------------------------------------------------------------------%

\printbibliography[heading=bibintoc]

\end{document}