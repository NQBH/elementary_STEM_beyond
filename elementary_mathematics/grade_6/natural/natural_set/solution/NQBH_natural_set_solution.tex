\documentclass{article}
\usepackage[backend=biber,natbib=true,style=alphabetic,maxbibnames=50]{biblatex}
\addbibresource{/home/nqbh/reference/bib.bib}
\usepackage[utf8]{vietnam}
\usepackage{tocloft}
\renewcommand{\cftsecleader}{\cftdotfill{\cftdotsep}}
\usepackage[colorlinks=true,linkcolor=blue,urlcolor=red,citecolor=magenta]{hyperref}
\usepackage{amsmath,amssymb,amsthm,float,graphicx,mathtools}
\usepackage{enumitem}
\setlist{leftmargin=4mm}
\allowdisplaybreaks
\newtheorem{assumption}{Assumption}
\newtheorem{baitoan}{}
\newtheorem{cauhoi}{Câu hỏi}
\newtheorem{conjecture}{Conjecture}
\newtheorem{corollary}{Corollary}
\newtheorem{dangtoan}{Dạng toán}
\newtheorem{definition}{Definition}
\newtheorem{dinhly}{Định lý}
\newtheorem{dinhnghia}{Định nghĩa}
\newtheorem{example}{Example}
\newtheorem{ghichu}{Ghi chú}
\newtheorem{hequa}{Hệ quả}
\newtheorem{hypothesis}{Hypothesis}
\newtheorem{kyhieu}{Ký hiệu}
\newtheorem{lemma}{Lemma}
\newtheorem{luuy}{Lưu ý}
\newtheorem{nhanxet}{Nhận xét}
\newtheorem{notation}{Notation}
\newtheorem{note}{Note}
\newtheorem{principle}{Principle}
\newtheorem{problem}{Problem}
\newtheorem{proposition}{Proposition}
\newtheorem{question}{Question}
\newtheorem{remark}{Remark}
\newtheorem{theorem}{Theorem}
\newtheorem{vidu}{Ví dụ}
\usepackage[left=1cm,right=1cm,top=5mm,bottom=5mm,footskip=4mm]{geometry}
\def\labelitemii{$\circ$}
\DeclareRobustCommand{\divby}{%
	\mathrel{\vbox{\baselineskip.65ex\lineskiplimit0pt\hbox{.}\hbox{.}\hbox{.}}}%
}

\title{Problem {\it\&} Solution: Set $\mathbb{N}$ of Naturals\\Bài Tập {\it\&} Lời Giải: Tập Hợp $\mathbb{N}$ Các Số Tự Nhiên}
\author{Nguyễn Quản Bá Hồng\footnote{Independent Researcher, Ben Tre City, Vietnam\\e-mail: \texttt{nguyenquanbahong@gmail.com}; website: \url{https://nqbh.github.io}.}}
\date{\today}

\begin{document}
\maketitle
\begin{abstract}
	Last updated version:
	\begin{itemize}
		\item Problem: \href{https://github.com/NQBH/elementary_STEM_beyond/blob/main/elementary_mathematics/grade_6/natural/natural_set/problem/NQBH_natural_set_problem.pdf}{GitHub{\tt/}NQBH{\tt/}elementary STEM \& beyond{\tt/}elementary mathematics{\tt/}grade 6{\tt/}natural{\tt/}natural set $\mathbb{N}${\tt/}problem: set $\mathbb{N}$ of naturals}.\footnote{\textsc{url}: \url{https://github.com/NQBH/elementary_STEM_beyond/blob/main/elementary_mathematics/grade_6/natural/natural_set/problem/NQBH_natural_set_problem.pdf}.} [\href{https://github.com/NQBH/elementary_STEM_beyond/blob/main/elementary_mathematics/grade_6/natural/natural_set/problem/NQBH_natural_set_problem.tex}{\TeX}]\footnote{\textsc{url}: \url{https://github.com/NQBH/elementary_STEM_beyond/blob/main/elementary_mathematics/grade_6/natural/natural_set/problem/NQBH_natural_set_problem.tex}.}
		\item Solution: \href{https://github.com/NQBH/elementary_STEM_beyond/blob/main/elementary_mathematics/grade_6/natural/natural_set/solution/NQBH_natural_set_solution.pdf}{GitHub{\tt/}NQBH{\tt/}elementary STEM \& beyond{\tt/}elementary mathematics{\tt/}grade 6{\tt/}natural{\tt/}natural set $\mathbb{N}${\tt/}problem \& solution: set $\mathbb{N}$ of naturals}.\footnote{\textsc{url}: \url{https://github.com/NQBH/elementary_STEM_beyond/blob/main/elementary_mathematics/grade_6/natural/natural_set/solution/NQBH_natural_set_solution.pdf}.} [\href{https://github.com/NQBH/elementary_STEM_beyond/blob/main/elementary_mathematics/grade_6/natural/natural_set/solution/NQBH_natural_set_solution.tex}{\TeX}]\footnote{\textsc{url}: \url{https://github.com/NQBH/elementary_STEM_beyond/blob/main/elementary_mathematics/grade_6/natural/natural_set/solution/NQBH_natural_set_solution.tex}.}
	\end{itemize}	
\end{abstract}
\tableofcontents

%------------------------------------------------------------------------------%

\section{Set -- Tập Hợp}
\fbox{\bf1} Tập hợp là 1 khái niệm cơ bản của Toán học. Đặt tên cho các tập hợp bằng các chữ cái in hoa, e.g., $A,B,C,\ldots$. \fbox{\bf2} Để viết 1 tập hợp, thường có 2 cách: (a) Liệt kê các phần tử của tập hợp; (b) Chỉ ra tính chất đặc trưng cho các phần tử của tập hợp đó. \fbox{\bf3} Khi liệt kê các phần tử của 1 tập hợp, mỗi phần tử \textit{chỉ liệt kê đúng 1 lần, thứ tự liệt kê tùy ý}. \fbox{\bf4} 1 tập hợp có thể có 1 phần tử, có nhiều phần tử, có vô số phần tử, cũng có thể không có phần tử nào, gọi là tập rỗng, ký hiệu là $\emptyset\coloneqq\{\}$. \fbox{\bf5} Cách viết tập hợp bằng phương pháp liệt kê các phần tử thích hợp cho các tập hợp có kích thước nhỏ, hoặc các tập hợp lớn nhưng có quy luật của các phần tử có thể nhận thấy 1 cách rõ ràng \& dễ dàng (e.g., $A = \{0,1,2,\ldots,100\}$ có quy luật: các phần tử là các số tự nhiên liên tiếp, $B = \{1,3,5,\ldots,99\}$ có quy luật: các phần tử là các số tự nhiên lẻ liên tiếp) \& có thể sử dụng dấu $\ldots$ để ám chỉ quy luật đó; còn cách viết tập hợp bằng phương pháp đặt trưng của các phần tử thích hợp với các tập hợp có kích thước lớn, có 1 hay nhiều quy luật, đặc biệt là các quy luật không dễ dàng mô tả bằng phương thức $\ldots$ như cách liệt kê. \fbox{\bf6} Mỗi số tự nhiên được biểu diễn bởi 1 điểm trên tia số. Điểm biểu diễn số tự nhiên $a$ trên tia số gọi là \textit{điểm} $a$. \fbox{\bf7} Trong hệ La Mã, dùng 7 ký hiệu: I, V, X, L, C, D, M với giá trị tương ứng trong hệ thập phân lần lượt là: $1,5,10,50,100,500,1000$.

\begin{dinhnghia}[Số phần tử của tập hợp]
	Với tập hợp $A$ bất kỳ có hữu hạn phần tử, {\rm số phần tử} của $A$ được ký hiệu là $|A|$.
\end{dinhnghia}

\begin{nhanxet}
	Với mọi tập hợp $A$ có hữu hạn phần tử, $|A|\in\mathbb{N}$ nên $|A| < \infty$. Với mọi tập hợp $A$ có vô hạn phần tử, $|A|\notin\mathbb{N}$, e.g., $|\mathbb{N}| = |\mathbb{N}^\star| =  \infty$. Để đo số lượng phần tử của các tập hợp có vô hạn phần tử (đếm được hoặc không đếm được), các nhà toán học sử dụng khái niệm {\rm lực lượng (cardinality)} của tập hợp (xem, e.g., {\rm\cite{Halmos1960, Halmos1974}}).
\end{nhanxet}

\begin{baitoan}[\cite{Tuyen_Toan_6}, VD1, p. 4]
	Cho 2 tập hợp: $A = \{6;7;8;9;10\}$, $B = \{x;9;7;10;y\}$. (a) Viết tập hợp $A$ bằng cách chỉ ra tính chất đặc trưng cho các phần tử của nó. (b) Điền $\in,\notin$: $9\square A$, $x\square A$, $y\square B$. (c) Tìm $x,y$ để $A = B$.
\end{baitoan}

\begin{proof}[Giải]
	(a) 2 cách viết: $A = \{x\in\mathbb{N}|5 < x < 11\} = \{x\in\mathbb{N}|6\le x\le10\}$. (b) $9\in A$, $x\notin A$, $y\in B$. (c) $A = B\Leftrightarrow\{x,y\} = \{6,8\}\Leftrightarrow(x = 6\land y = 8)\lor(x = 8\land y = 6)$.
\end{proof}

\begin{kyhieu}[$\land,\lor$]
	Ký hiệu logical and $\land$ nghĩa là ``và'', ``and'', còn ký hiệu logical or $\lor$ nghĩa là ``hoặc'', ``or''.
\end{kyhieu}

\begin{luuy}
	Trong lời giải (c), có 2 đáp số vì thứ tự liệt kê của 1 tập hợp không quan trọng. Hơn nữa, nếu sử dụng bộ sắp thứ tự (pair of numbers) thì cũng có thể lý luận $\{x,y\} = \{6,8\}\Leftrightarrow(x,y) = (6,8)\lor(x,y) = (8,6)$. Tổng quát hơn, $\{x,y\} = \{a,b\}\Leftrightarrow(x,y) = (a,b)\lor(x,y) = (b,a)$, $\forall a,b\in\mathbb{N}$, $a\ne b$. Việc mở rộng lên $n$ số tự nhiên hoàn toàn tương tự: $\{x_1,x_2,\ldots,x_n\} = \{a_1,a_2,\ldots,a_n\}\Leftrightarrow(x_1,x_2,\ldots,x_n) = (a_1,a_2,\ldots,a_n)$ \& các hoán vị (i.e., đổi chỗ), $\forall a_i\in\mathbb{N}$, $i = 1,2,\ldots,n$ là $n$ số tự nhiên đôi một khác nhau (điều kiện khác nhau đảm bảo tập hợp $\{a_1,a_2,\ldots,a_n\}$ có đúng $n$ phần tử chứ không bị ``teo'' lại nếu có chứa các phần tử bằng nhau.
\end{luuy}

\begin{baitoan}[\cite{Tuyen_Toan_6}, 2., p. 5]
	(a) Viết tập hợp $M$ các chữ cái của chữ {\rm``NGANG''}. (b) Với tất cả các phần tử của tập hợp $M$, viết thành 1 chữ thuộc loại danh từ (không sử dụng thêm dấu).
\end{baitoan}

\begin{proof}[Giải]
	(a) $M = \{{\rm N,G,A}\}$. (b) NGA (nước Nga, Russia), GAN (lá gan -- 1 bộ phận trong tiêu hóa của cơ thể động vật), GANG (gang, cast iron, là 1 nhóm vật liệu hợp kim sắt-carbon có hàm lượng carbon $> 2.14\%$, xem, e.g., \href{https://vi.wikipedia.org/wiki/Gang}{Wikipedia{\tt/}gang}), NGAN (con ngan, 1 loại gia cầm, xem \href{https://vi.wikipedia.org/wiki/Ngan_nh%C3%A0}{Wikipedia{\tt/}ngan nhà}, \href{https://vi.wikipedia.org/wiki/Ngan_c%E1%BB%8F}{Wikipedia{\tt/}ngan cỏ}).
\end{proof}

\begin{baitoan}[\cite{Tuyen_Toan_6}, 3., p. 5]
	Viết tập hợp $P$ tên các tỉnh tiếp giáp với Thủ đô Hà Nội.
\end{baitoan}

\begin{proof}[Giải]
	$P = \{\mbox{Bắc Giang, Bắc Ninh, Hà Nam, Hòa Bình, Hưng Yên, Phú Thọ, Thái Nguyên, Vĩnh Phúc}\}$.
\end{proof}

\begin{baitoan}[\cite{Tuyen_Toan_6}, 4., p. 5]
	Cho tập hợp $A = \{a,b\}$, $B = \{1,2,3\}$. Viết tất cả các tập hợp có $3$ phần tử trong đó $1$ phần tử thuộc tập hợp $A$, $2$ phần tử thuộc tập hợp $B$.
\end{baitoan}

\begin{proof}[Giải]
	Có 6 tập hợp: $\{a,1,2\},\{a,1,3\},\{a,2,3\},\{b,1,2\},\{b,1,3\},\{b,2,3\}$.
\end{proof}

\begin{baitoan}[\cite{Tuyen_Toan_6}, 5., p. 5]
	Cho các tập hợp: $P$ là tập hợp các số tự nhiên $x$ mà $x + 3\le10$, $Q$ là tập hợp các số tự nhiên $x$ mà $x\cdot3 = 5$, $R$ là tập hợp các số tự nhiên $x$ mà $x\cdot3 = 0$, $S$ là tập hợp các số tự nhiên $x$ mà $x\cdot3\le24$. (a) Tập hợp nào là tập hợp rỗng? (b) Tập hợp nào có đúng 1 phần tử? (c) 2 tập hợp nào bằng nhau?
\end{baitoan}

\begin{proof}[Giải]
	$P = \{x\in\mathbb{N}|x + 3\le10\} = \{0,1,2,3,4,5,6,7\}$, $Q = \{x\in\mathbb{N}|x\cdot3 = 5\} = \emptyset$ (vì nghiệm duy nhất của phương trình $x\cdot3 = 5$ là $x = \frac{5}{3}\notin\mathbb{N}$, nghiệm này là số hữu tỷ nhưng không phải là số nguyên hay số tự nhiên), $R = \{x\in\mathbb{N}|x\cdot3 = 0\} = \{0\}$, $S = \{x\in\mathbb{N}|x\cdot3\le24\} = \{0,1,2,3,4,5,6,7,8\}$. (a) Tập hợp $Q$ là tập hợp rỗng. (b) Tập hợp $R$ có 1 phần tử. (c) Không có 2 tập hợp nào bằng nhau.
\end{proof}

\begin{baitoan}
	Viết tập hợp: (a) Tập các màu sắc của cầu vồng. (b) Tập hợp các huyện của tỉnh Bến Tre. (c) Tập hợp các châu lục trên Trái Đất. (d) Tập hợp các hành tinh trong Hệ Mặt Trời.
\end{baitoan}

\begin{proof}[Giải]
	(a) $A = \{\mbox{đỏ, cam, vàng, xanh lá, xanh lam, chàm, tím}\}$. (b) Bến Tre có 09 đơn vị hành chính cấp huyện trực thuộc, bao gồm các huyện: Ba Tri, Bình Đại, Châu Thành, Chợ Lách, Giồng Trôm, Mỏ Cày Bắc, Mỏ Cày Nam, Thạnh Phú \& thành phố Bến Tre (xem, e.g., \href{https://vi.wikipedia.org/wiki/B%E1%BA%BFn_Tre}{Wikipedia{\tt/}Bến Tre}), nên $B = \{$Ba Tri, Bình Đại, Châu Thành, Chợ Lách, Giồng Trôm, Mỏ Cày Bắc, Mỏ Cày Nam, Thạnh Phú, thành phố Bến Tre$\}$. (c) $C = \{$Á, Phi, Nam Mỹ, Bắc Mỹ, Âu, Úc, Nam Cực$\}$. (d) Xem, e.g., \href{https://vi.wikipedia.org/wiki/H%E1%BB%87_M%E1%BA%B7t_Tr%E1%BB%9Di}{Wikipedia{\tt/}hệ Mặt Trời}: $D = \{$Sao Thủy, Sao Kim, Trái Đất, Sao Hỏa, Sao Mộc, Sao Thổ, Sao Thiên Vương, Sao Hải Vương$\}$.
\end{proof}

%------------------------------------------------------------------------------%

\section{Set $\mathbb{N}$ of Natural Numbers -- Tập hợp $\mathbb{N}$ Các Số Tự Nhiên}

\begin{baitoan}[\cite{Binh_boi_duong_Toan_6_tap_1}, VD1, p. 8]
	Nhiệt độ thay đổi theo giờ trong 1 ngày tháng 8 \& 1 ngày tháng 9 lần lượt được ghi lại:
	\begin{table}[H]
		\centering
		\begin{tabular}{|c|c|c|c|c|c|}
			\hline
			Thời điểm trong ngày & 6:00 & 9:00 & 12:00 & 14:00 & 17:00 \\
			\hline
			Nhiệt độ (${}^\circ$) & 24 & 25 & 27 & 26 & 24 \\
			\hline
		\end{tabular}
	\end{table}
	\begin{table}[H]
		\centering
		\begin{tabular}{|c|c|c|c|c|c|}
			\hline
			Thời điểm trong ngày & 6:00 & 9:00 & 12:00 & 14:00 & 17:00 \\
			\hline
			Nhiệt độ (${}^\circ$) & 23 & 24 & 26 & 25 & 22 \\
			\hline
		\end{tabular}
	\end{table}
	\noindent(a) Viết 2 tập hợp A, B gồm các giá trị nhiệt độ của mỗi bảng trên. (b) Viết tập hợp C gồm các phần tử thuộc tập hợp A mà không thuộc tập hợp B. (c) Viết tập hợp D gồm các phần tử thuộc tập hợp B mà không thuộc tập hợp A. (d) Viết tập hợp E gồm các phần tử thuộc cả 2 tập hợp A \& B. (e) Viết tập hợp F gồm các phần tử thuộc tập hợp A hoặc thuộc tập hợp B.
\end{baitoan}

\begin{proof}[Giải]
	(a) $A = \{24,25,26,27\}$, $B = \{22,23,24,25,26\}$. (b) $C = A\backslash B = \{27\}$. (c) $D = B\backslash A = \{22,23\}$. (d) $E = A\cap B = \{24,25,26\}$. (e) $F = \{22,23,24,25,26,27\}$.
\end{proof}

\begin{luuy}
	Khi viết tập hợp bằng cách liệt kê, mỗi phần tử chỉ được viết 1 lần.
\end{luuy}

\begin{luuy}
	Tập hợp C gồm các phần tử thuộc tập hợp A, trừ các phần tử của A mà cũng thuộc B. Trên biểu đồ Venn, tập hợp C có minh hoa là miền gạch chéo. Ký hiệu $C = A\backslash B$ (đọc là: C là {\rm hiệu} của A, B). Tương tự, tập hợp $D = B\backslash A$ (đọc là: D là hiệu của B \& A). Tập hợp E gồm các phần tử chung của 2 tập hợp A, B, ký hiệu $E = A\cap B$ (đọc là: E là {\rm giao} của A, B). Tập hợp F gồm các phần thử hoặc thuộc A, hoặc thuộc B, ký hiệu $F = A\cup B$ (đọc là: F là {\rm hợp} của A, B). Hiển nhiên cũng có $F = C\cup D\cup E$.
\end{luuy}

\begin{baitoan}[\cite{Binh_boi_duong_Toan_6_tap_1}, VD2, p. 9]
	Cho A là tập hợp các số tự nhiên chẵn có 3 chữ số. Hỏi A có bao nhiêu phần tử?
\end{baitoan}

\begin{proof}[1st giải]
	Khi liệt kê các phần tử của tập hợp $A$ theo giá trị tăng dần, ta được 1 dãy số cách đều có khoảng cách là 2 đơn vị: $100,102,104,\ldots,998$. Số phần tử của tập hợp $A$ bằng số số hạng của dãy số cách đều: $(998 - 100):2 + 1 = 898:2 + 1 = 449 + 1 = 450$.
\end{proof}

\begin{proof}[2nd giải]
	Các số thỏa mãn có dạng $\overline{abc}$ với $a\in\{1,2,\ldots,9\}$ nên có 9 cách chọn, $b\in\{0,1,2,\ldots,9\}$ nên có 10 cách chọn, $c$ chẵn nên $c\in\{0,2,4,6,8\}$ nên có 5 cách chọn. Theo quy tắc nhân, số số thỏa mãn bằng $9\cdot10\cdot5 = 45\cdot10 = 450$.
\end{proof}

\begin{baitoan}
	Với $n\in\mathbb{N}^\star$, cho $A_n$ là tập hợp các số tự nhiên chẵn có $n$ chữ số, $B_n$ là tập hợp các số tự nhiên lẻ có $n$ chữ số. Hỏi $A,B$ có bao nhiêu phần tử?
\end{baitoan}

\begin{proof}[1st giải (sử dụng biểu diễn thập phân)]
	Khi liệt kê các phần tử của tập hợp $A_n$ theo giá trị tăng dần, ta được 1 dãy số cách đều có khoảng cách là 2 đơn vị: $1\underbrace{0\ldots0}_{n-1},1\underbrace{0\ldots0}_{n-2}2,1\underbrace{0\ldots0}_{n-2}4,\ldots,\underbrace{9\ldots9}_{n-1}6,\underbrace{9\ldots9}_{n-1}8$. Số phần tử của tập hợp $A_n$ bằng số số hạng của dãy số cách đều: $(\underbrace{9\ldots9}_{n-1}8 - 1\underbrace{0\ldots0}_{n-1}):2 + 1 = 8\underbrace{9\ldots9}_{n-2}8:2 + 1 = \underbrace{4\ldots4}_{n-1}9 + 1 = 45\underbrace{0\ldots0}_{n-2} = 45\cdot10^{n-2}$. Tương tự, Khi liệt kê các phần tử của tập hợp $B_n$ theo giá trị tăng dần, ta được 1 dãy số cách đều có khoảng cách là 2 đơn vị: $1\underbrace{0\ldots0}_{n-2}1,1\underbrace{0\ldots0}_{n-2}3,1\underbrace{0\ldots0}_{n-2}5,\ldots,\underbrace{9\ldots9}_{n-1}7,\underbrace{9\ldots9}_n$. Số phần tử của tập hợp $B_n$ bằng số số hạng của dãy số cách đều: $(\underbrace{9\ldots9}_n - 1\underbrace{0\ldots0}_{n-2}1):2 + 1 = 8\underbrace{9\ldots9}_{n-2}8:2 + 1 = \underbrace{4\ldots4}_{n-1}9 + 1 = 45\underbrace{0\ldots0}_{n-2} = 45\cdot10^{n-2}$. Vậy $|A_n| = |B_n| = 45\cdot10^{n-2}$, $\forall n\in\mathbb{N}^\star$.
\end{proof}

\begin{nhanxet}
	Để tính số phần tử của tập hợp $B_n$ dựa vào số phần tử của tập hợp $A_n$, có thể nhận xét: nếu tăng mỗi phần tử của $A_n$ thêm 1 đơn vị sẽ được 1 phần tử của $B$, \& ngược lại, nếu giảm mỗi phần tử của $B$ đi 1 đơn vị sẽ được 1 phần tử của $A_n$, nên số phần tử của 2 tập hợp $A,B$ phải bằng nhau, i.e., $|B_n| = |A_n| = 45\cdot10^{n-2}$, $\forall n\in\mathbb{N}^\star$.
\end{nhanxet}

\begin{proof}[2nd giải (sử dụng công thức lũy thừa)]
	Khi liệt kê các phần tử của tập hợp $A_n$ theo giá trị tăng dần, ta được 1 dãy số cách đều có khoảng cách là 2 đơn vị: $10^{n-1},10^{n-1} + 2,10^{n-1} + 4,\ldots,10^n - 2$. Số phần tử của tập hợp $A_n$ bằng số số hạng của dãy số cách đều: $(10^n - 2 - 10^{n-1}):2 + 1 = (10^n - 10^{n-1}):2 - 1 + 1 = \frac{1}{2}(10^n - 10^{n-1})$. Tương tự, khi liệt kê các phần tử của tập hợp $B_n$ theo giá trị tăng dần, ta được 1 dãy số cách đều có khoảng cách là 2 đơn vị: $10^{n-1} + 1,10^{n-1} + 3,10^{n-1} + 5,\ldots,10^n - 1$. Số phần tử của tập hợp $B_n$ bằng số số hạng của dãy số cách đều: $[10^n - 1 - (10^{n-1} + 1)]:2 + 1 = (10^n - 10^{n-1} - 2):2 + 1 = (10^n - 10^{n-1}):2 - 1 + 1 = \frac{1}{2}(10^n - 10^{n-1})$. Vậy $|A_n| = |B_n| = \frac{1}{2}(10^n - 10^{n-1})$, $\forall n\in\mathbb{N}^\star$.
\end{proof}

\begin{proof}[3rd giải (sử dụng phương pháp đếm trong tổ hợp)]
	Các phần tử của tập hợp $A_n$ có dạng $\overline{a_{n-1}\ldots a_1a_0}$ với $a_{n-1}\in\{1,2,\ldots,9\}$ nên có 9 cách chọn, $a_i\in\{0,1,2,\ldots,9\}$ nên có 10 cách chọn, $\forall i = 1,2,\ldots,n - 2$, $a_0$ chẵn nên $a_0\in\{0,2,4,6,8\}$ nên có 5 cách chọn. Theo quy tắc nhân, $|A_n| = 9\cdot10^{n-2}\cdot5 = 45\cdot10^{n-2}$. Tương tự, các phần tử của tập hợp $B$ có dạng $\overline{a_{n-1}\ldots a_1a_0}$ với $a_{n-1}\in\{1,2,\ldots,9\}$ nên có 9 cách chọn, $a_i\in\{0,1,2,\ldots,9\}$ nên có 10 cách chọn, $\forall i = 1,2,\ldots,n - 2$, $a_0$ lẻ nên $a_0\in\{1,3,5,7,9\}$ nên có 5 cách chọn. Theo quy tắc nhân, $|B_n| = 9\cdot10^{n-2}\cdot5 = 45\cdot10^{n-2}$. Vậy $|A_n| = |B_n| = 45\cdot10^{n-2}$, $\forall n\in\mathbb{N}^\star$.
\end{proof}

\begin{baitoan}[\cite{Binh_boi_duong_Toan_6_tap_1}, VD3, p. 9]
	Cho A là tập hợp các số tự nhiên lẻ lớn hơn $3$ \& không lớn hơn $99$. (a) Viết tập hợp A bằng cách chỉ ra tính chất đặc trưng của các phần tử. (b) Giả sử các phần tử của A được viết theo giá trị tăng dần. Tìm phần tử thứ $23$ của A.
\end{baitoan}

\begin{proof}[1st giải]
	(a) $A = \{n\in\mathbb{N}|n\not{\divby}\ 2,\,3 < n\le99\}$. (b) Khi được viết theo giá trị tăng dần, giá trị các phần tử của $A$ tạo thành 1 dãy số cách đều tăng dần: $5,7,9,\ldots,99$. Giả sử phần tử thứ 23 của $A$ là $x$, có: $(x - 5):2 + 1 = 23\Leftrightarrow(x - 5):2 = 22\Leftrightarrow x - 5 = 22\cdot2 = 44\Leftrightarrow x = 44 + 5 = 49$.
\end{proof}

\begin{proof}[2nd giải]
	(a) $A = \{n\in\mathbb{N}|n\not{\divby}\ 2,\,3 < n\le99\}$. (b) Số hạng thứ $n$ của dãy số có dạng tổng quát là $5 + 2(n - 1)$ nên số hạng thứ 23 là $5 + 2(23 - 1) = 49$.
\end{proof}

\begin{luuy}
	Tập hợp A có $(99 - 5):2 + 1 = 48$ phần tử nên A có phần tử thứ $23$. Tổng quát, với tập hợp A bất kỳ, ta phải đảm bảo phần tử thứ $n$ có nằm trong tập đã cho hay không nhờ bất đẳng thức về chỉ số của phần tử: $n\le|A|$, trong đó $|A|$ là số phần tử của tập hợp $A$. Nếu $n > |A|$ thì tập hợp A không có ``chỗ'' để chứa phần tử này nên phần tử thứ $n$ này của tập hợp A không có nghĩa, i.e., không tồn tại. Với (b), có thể viết tập hợp A dưới dạng liệt kê các phần tử cho tới phần tử thứ $23$ nhưng nhược điểm là ta phải liệt kê được tất cả các phần tử đứng trước phần tử cần tìm, nên nếu bài toán yêu cầu tìm phần tử ở vị trí càng lớn thì sẽ càng khó khăn. Tóm lại, thay vì liệt kê các phần tử của 1 tập hợp, nên thử tìm quy luật hoặc công thức tổng quát của các phần tử (nếu có) của tập hợp ấy để thuận tiện tính toán.
\end{luuy}

\begin{baitoan}[\cite{Binh_boi_duong_Toan_6_tap_1}, VD4, p. 10]
	Để đánh số các trang sách (bắt đầu từ trang $1$) của 1 cuốn sách có $2015$ trang thì cần dùng bao nhiêu chữ số?
\end{baitoan}

\begin{proof}[Giải]
	Chia các số trang của cuốn sách thành 4 nhóm: Nhóm các số có 1 chữ số (từ trang 1--9) nên số chữ số cần dùng là 9. Nhóm các số có 2 chữ số (từ trang 10--99) có số trang sách: $(99 - 10):1 + 1 = 90$ nên số chữ số cần dùng: $90\cdot2 = 180$. Nhóm các số có 3 chữ số (từ trang 100--999) có số trang sách: $(999 - 100):1 + 1 = 900$ nên số chữ số cần dùng: $900\cdot3 = 2700$. Nhóm các số có 4 chữ số (từ trang 1000--2015) có số trang sách: $(2015 - 1000):1 + 1 = 1016$ nên số chữ số cần dùng: $1016\cdot4 = 4064$. Suy ra tổng số chữ số cần dùng để đánh số trang của cuốn sách đó: $9 + 180 + 2700 + 4064 = 6953$.
\end{proof}

\begin{baitoan}
	Để đánh số các trang sách (bắt đầu từ trang $1$) của 1 cuốn sách có $a\in\mathbb{N}^\star$ trang, với $a$ có $n$ chữ số, thì cần dùng bao nhiêu chữ số?
\end{baitoan}

\begin{proof}[Giải]
	Chia các số trang của cuốn sách thành $n$ nhóm: $\forall i = 1,2,\ldots,n - 1$, nhóm các số có $i$ chữ số (từ trang $10^{i-1}$ đến trang $10^i - 1$) có số trang sách: $(10^i - 1 - 10^{i-1}):1 + 1 = 10^i - 10^{i-1}$ nên số chữ số cần dùng: $i(10^i - 10^{i-1})$, \& nhóm các số có $n$ chữ số (từ trang $10^{n-1}$ đến $a$) có số trang sách: $(a - 10^{n-1}):1 + 1 = a - 10^{n-1} + 1$ nên số chữ số cần dùng: $n(a - 10^{n-1} + 1)$. Suy ra tổng số chữ số cần dùng để đánh số trang của cuốn sách đó:
	\begin{align*}
		&\sum_{i=1}^{n-1} i(10^i - 10^{i-1}) + n(a - 10^{n-1} + 1) = \sum_{i=1}^{n-1} i10^i - \sum_{i=1}^{n-1} i10^{i-1} + n(a + 1) - n10^{n-1}\\
		= &\sum_{i=1}^{n-1} i10^i - \sum_{i=0}^{n-2} (i + 1)10^i + n(a + 1) - n\cdot10^{n-1} = \sum_{i=1}^{n-2} [i - (i + 1)]10^i + (n - 1)10^{n-1} - 1\cdot10^0 + n(a + 1) - n\cdot10^{n-1}\\
		= &-\sum_{i=1}^{n-2} 10^i + n10^{n-1} - n10^{n-1} - 10^{n-1} - 1 + n(a + 1) = -10\sum_{i=0}^{n-3} 10^i - 10^{n-1} - 1 + n(a + 1)\\
		= &-10\dfrac{10^{n-2} - 1}{10 - 1} - 10^{n-1} - 1 + n(a + 1) = -\dfrac{10}{9}(10^{n-2} - 1) - 10^{n-1} - 1 + n(a + 1) = -\dfrac{10^{n-1}}{9} + \dfrac{10}{9} - 10^{n-1} - 1 + n(a + 1)\\
		= &-\left(\dfrac{1}{9} + 1\right)10^{n-1} + \dfrac{1}{9} + n(a + 1) = n(a + 1) - \dfrac{10^n - 1}{9}.
	\end{align*}
	Vậy cần dùng $n(a + 1) - \dfrac{10^n - 1}{9}$ chữ số để đánh số các trang của cuốn sáchđó.
\end{proof}

\begin{baitoan}[\cite{Binh_boi_duong_Toan_6_tap_1}, VD5, p. 10]
	Biết người ta đã dùng đúng $6793$ chữ số để đánh số trang của 1 cuốn sách (bắt đầu từ trang $1$), hỏi cuốn sách đó có bao nhiêu trang?
\end{baitoan}

\begin{baitoan}[\cite{Binh_boi_duong_Toan_6_tap_1}, VD6, p. 11]
	Gọi A là tập hợp các số tự nhiên có 2 chữ số mà chữ số hàng chục lớn hơn chữ số hàng đơn vị; B là tập hợp các số tự nhiên có 2 chữ số mà chữ số hàng chục nhỏ hơn chữ số hàng đơn vị. So sánh số phần tử của 2 tập hợp A, B.
\end{baitoan}

\begin{baitoan}[\cite{Binh_boi_duong_Toan_6_tap_1}, VD7, p. 12]
	Tìm 1 số có 2 chữ số biết khi viết thêm chữ số $0$ vào giữa 2 chữ số của số đó thì được số mới gấp $6$ lần số đã cho.
\end{baitoan}

\begin{baitoan}[\cite{Binh_boi_duong_Toan_6_tap_1}, VD8, p. 12]
	Tìm số có 3 chữ số biết nếu viết thêm chữ số $1$ vào trước số đó thì được số mới gấp $9$ lần số ban đầu.	
\end{baitoan}

\begin{baitoan}[\cite{Binh_boi_duong_Toan_6_tap_1}, p. 13]
	Ngày mùng 2 tháng 9 năm $\overline{abcd}$, tại quảng trường Ba Đình lịch sử, Chủ tịch Hồ Chí Minh đã đọc bản tuyên ngôn độc lập khai sinh nước Việt Nam Dân chủ Cộng hòa (nay là nước Cộng hòa xã hội chủ nghĩa Việt Nam). Năm $\overline{abcd}$ là năm nào? Biết $a$ là phần tử nhỏ nhất trong tập hợp $ \mathbb{N}^\star$, $b$ là chữ số lớn nhất, $c,d$ là 2 số tự nhiên liên tiếp \& $c + d = b$.
\end{baitoan}

\begin{baitoan}[\cite{Binh_boi_duong_Toan_6_tap_1}, 1.1., p. 13]
	Cho tập hợp $A = \{1,3,5\}$. {\rm Đ{\tt/}S?} (a) $1\in A$. (b) $\{1\}\in A$. (c) $3\notin A$. (d) $5\notin A$.
\end{baitoan}

\begin{proof}[Giải]
	(a) Đ. (b) S. Phải là $1\in A$ hoặc $\{1\}\subset A$ với ký hiệu $A\subset B$ nghĩa là: tập hợp $A$ là tập hợp con của tập hợp $B$. (c) S. Phải là $3\in A$. (d) S. Phải là $5\in A$.
\end{proof}

\begin{baitoan}[\cite{Binh_boi_duong_Toan_6_tap_1}, 1.2., p. 13]
	Cho 2 tập hợp: $A = \{1,2,3,4,5\}$, $B = \{3,5,7,9\}$. (a) Mỗi tập hợp trên có bao nhiêu phần tử? (b) Viết các tập hợp trên bằng cách chỉ ra tính chất đặc trưng của các phần tử.
\end{baitoan}

\begin{proof}[Giải]
	(a) Tập hợp $A$ có 5 phần tử. Tập hợp $B$ có 4 phần tử. (b) $A = \{n\in\mathbb{N}^\star|n < 6\}$, $B = \{n\in\mathbb{N}^\star|n\mbox{  là chữ số lẻ lớn hơn 1}\} = \{n\in\mathbb{N}^\star|n\not{\divby}\ 2,\,n\le9\}$.
\end{proof}

\begin{baitoan}[\cite{Binh_boi_duong_Toan_6_tap_1}, 1.3., p. 13]
	Viết các tập hợp sau \& cho biết mỗi tập hợp đó có bao nhiêu phần tử. (a) Tập hợp A các số tự nhiên $x$ thỏa $12 - x = 5$. (b) Tập hợp B các số tự nhiên $y$ thỏa $7 + y = 21$. (c) Tập hợp C các số tự nhiên $z$ mà $z\cdot0 = 0$.
\end{baitoan}

\begin{proof}[Giải]
	(a) $12 - x = 5\Leftrightarrow x = 12 - 5 = 7\Rightarrow A = \{7\}$ có 1 phần tử. (b) $7 + y = 21\Leftrightarrow y = 21 - 7 = 14\Rightarrow B = \{14\}$ có 1 phần tử. (c) Có $0z = 0$, $\forall z\in\mathbb{N}\Rightarrow C = \mathbb{N}$ có vô số (vô hạn đếm được{\tt/}infinitely countable) phần tử.
\end{proof}

\begin{baitoan}[\cite{Binh_boi_duong_Toan_6_tap_1}, 1.4., p. 14]
	Tính số phần tử của tập hợp: (a) $A = \{2,4,6,\ldots,98\}$. (b) $B = \{6,10,14,18,22,\ldots,70\}$.
\end{baitoan}

\begin{proof}[Giải]
	(a) $|A| = (98 - 2):2 + 1 = 96:2 + 1 = 48 + 1 = 49$. (b) $|B| = (70 - 6):4 + 1 = 64:4 + 1 = 16 + 1 = 17$.
\end{proof}

\begin{baitoan}[\cite{Binh_boi_duong_Toan_6_tap_1}, 1.5., p. 14]
	Cho dãy số $2,5,8,11,14,\ldots$. (a) Nêu quy luật của dãy số trên. (b) Viết tập hợp B gồm $5$ số hạng tiếp theo của dãy số trên. (c) Tính tổng $100$ số hạng đầu tiên của dãy số.
\end{baitoan}

\begin{proof}[Giải]
	(a) Quy luật: Dãy số cách đều với khoảng cách là 3 đơn vị. (b) $B = \{17,20,23,26,29\}$. (c) Gọi số hạng thứ 100 của dãy là $n$, $n$ thỏa $(n - 2):3 + 1 = 100\Leftrightarrow(n - 2):3 = 100 - 1 = 99\Leftrightarrow n - 2 = 99\cdot3 = 297\Leftrightarrow n = 297 + 2 = 299$. Tổng 100 số hạng đầu của dãy: $(2 + 299)\cdot100:2 = 301\cdot50 = 15050$.
\end{proof}

\begin{baitoan}[Cấp số cộng]
	Cho $a\in\mathbb{N}$, $b,n\in\mathbb{N}^\star$. Xét dãy số $a,a + b,a + 2b,a + 3b,\ldots$. (a) Nêu quy luật của dãy số trên. (b) Tính tổng $n$ số hạng đầu tiên của dãy số.
\end{baitoan}
	
\begin{proof}[1st giải]
	(a) Quy luật: Dãy số cách đều với khoảng cách là $n$ đơn vị. (b) Số hạng thứ $n$ của dãy là $a + (n - 1)b$. Tổng $n$ số hạng đầu của dãy: $[a + a + (n - 1)b]\cdot n:2 = [2a + (n - 1)b]\cdot n:2 = an + \dfrac{bn(n - 1)}{2}$.
\end{proof}

\begin{proof}[2nd giải]
	(b) Số hạng thứ $n$ của dãy là $a + (n - 1)b$. Tổng $n$ số hạng đầu của dãy: $\sum_{i=0}^{n-1} a + ib = \sum_{i=0}^{n-1} a + \sum_{i=0}^{n-1} ib = (n - 1 - 0 + 1)a + b \sum_{i=0}^{n-1} i = na + b\dfrac{(n - 1)n}{2} = an + \dfrac{bn(n - 1)}{2}$.
\end{proof}
\noindent$\star$ \textsf{Công thức tính tổng $n$ số tự nhiên khác $0$, i.e., $n$ số nguyên dương đầu tiên}:
\begin{align*}
	\boxed{\sum_{i=1}^n i = 1 + 2 + \cdots + n = \dfrac{n(n + 1)}{2},\ \forall n\in\mathbb{N}.}
\end{align*}

\begin{baitoan}[\cite{Binh_boi_duong_Toan_6_tap_1}, 1.6., p. 14]
	Viết lại mỗi tập hợp bằng cách liệt kê các phần tử: $A = \{x\in\mathbb{N}|{x\not{\divby}\ 2},\,30 < x < 50\}$. (b) $B = \{x\in\mathbb{N}|x\divby5,\,x\divby2,\,10 < x\le90\}$.
\end{baitoan}

\begin{proof}[Giải]
	(a) $A = \{31,33,35,37,39,41,43,45,47,49\}$. $\forall x\in B$, vì $x\divby2$ \& $x\divby5$ mà $\mbox{\rm ƯCLN}(2,5) = 1$ nên suy ra $x\divby{\rm BCNN}(2,5)$, i.e., $x\divby10$, suy ra $B = \{20,30,40,50,60,70,80,90\}$.
\end{proof}

\begin{baitoan}[\cite{Binh_boi_duong_Toan_6_tap_1}, 1.7., p. 13]
	Thực hiện yêu cầu phòng chống dịch Covid-19, tại 1 trường trung học, vào đầu giờ sáng trước khi vào lớp, các học sinh đều được yêu cầu khử khuẩn tay \& đo thân nhiệt. Kết quả đo thân nhiệt tại lớp 6H:
	\begin{table}[H]
		\centering
		\begin{tabular}{|c|l|}
			\hline
			Tổ & Thân nhiệt (${}^\circ$C) \\
			\hline
			1 & 36, 36.5, 37, 36, 35.5, 37, 36.5, 36, 35.5, 36, 36.5, 37 \\
			\hline
			2 & 36.5, 37, 35.5, 36, 36, 35.5, 37, 36, 36.5, 36, 35.5, 36 \\
			\hline
			3 & 37, 36.5, 36, 35.5, 35, 36, 35.5, 35, 36, 35, 36.5, 36 \\
			\hline
			4 & 36, 35.5, 36, 36, 35.5, 36.5, 35, 36.5, 36, 35.5, 36.5, 36 \\
			\hline
		\end{tabular}
	\end{table}
	\noindent(a) Gọi A, B, C, D lần lượt là tập hợp gồm các phần tử là thân nhiệt của các bạn tổ 1, tổ 2, tổ 3, tổ 4. Viết các tập hợp A, B, C, D theo cách liệt kê các phần tử (mỗi phần tử chỉ liệt kê 1 lần). (b) Tìm cặp tập hợp bằng nhau trong các tập hợp A, B, C, D. Dùng ký hiệu ``$=$'' để thể hiện mối quan hệ đó.
\end{baitoan}

\begin{proof}[Giải]
	(a) $A = \{35.5,36,36.5,37\}$, $B = \{35.5,36,36.5,37\}$, $C = \{35,35.5,36,36.5,37\}$, $D = \{35,35.5,36,37\}$. (b) $A = B$.
\end{proof}

\begin{baitoan}[\cite{Binh_boi_duong_Toan_6_tap_1}, 1.8., p. 14]
	Với 3 chữ số La Mã I, V, X có thể viết được bao nhiêu số La Mã mà mỗi chữ số chỉ xuất hiện 1 lần? Số nhỏ nhất là số nào? Số lớn nhất là số nào?
\end{baitoan}

\begin{proof}[Giải]
	(a) Có thể viết được 10 số: I, V, X, IV, VI, IX, XI, XV, XIV, XVI. Số nhỏ nhất: I: 1. Số lớn nhất: XVI: 16.
\end{proof}

\begin{baitoan}[\cite{Binh_boi_duong_Toan_6_tap_1}, 1.9., p. 14]
	Tìm số có 3 chữ số, biết nếu viết các chữ số theo thứ tự ngược lại thì được số mới nhỏ hơn số ban đầu $792$ đơn vị.
\end{baitoan}

\begin{proof}[Giải]
	Gọi số cần tìm là $\overline{abc}$, $a,b,c\in\{0,1,2,\ldots,9\}$, $ac\ne0$. Có $\overline{abc} - \overline{cba} = 792\Leftrightarrow100a + 10b + c - (100c + 10b + a) = 792\Leftrightarrow99(a - c) = 792\Leftrightarrow a - c = 792:99 = 8\Rightarrow a = 9$, $c = 1$. Vậy các số cần tìm là $\overline{9b1}$, $\forall b = 0,1,2,\ldots,9$, i.e., $901,902,903,904,905,906,907,908,909$.
\end{proof}

\begin{baitoan}
	(a) Tìm tất cả các giá trị có thể có của $n$ biết nếu viết các chữ số của 1 số có 3 chữ số theo thứ tự ngược lại thì được số mới nhỏ hơn số ban đầu $n$ đơn vị. (b) Với mỗi giá trị của $n$ vừa tìm được, tìm tất cả các số có 3 chữ số thỏa mãn.
\end{baitoan}

\begin{proof}[Giải]
	(a) Các số có 3 chữ số có dạng là $\overline{abc}$, $a,b,c\in\{0,1,2,\ldots,9\}$, $ac\ne0$ (điều kiện $c\ne0$ là cần thiết để viết đảo ngược số $\overline{abc}$ lại). Có $\overline{abc} - \overline{cba} = n\Leftrightarrow100a + 10b + c - (100c + 10b + a) = n\Leftrightarrow99(a - c) = n$ mà $a,c\in\{1,2,\ldots,9\}$ nên suy ra $a - c\in\{-8,-7,\ldots,7,8\} = \{0,\pm1,\pm2,\ldots,\pm8\}\Rightarrow n = 99(a - c)\in A\coloneqq\{0\cdot99,\pm1\cdot99,\pm2\cdot99,\ldots,\pm8\cdot99\} = \{0,\pm99,\pm198,\pm297,\pm396,\pm495,\pm594,\pm693,\pm792\}$. (b) Gọi $B$ là tập hợp các số có 3 chữ số thỏa mãn $\overline{abc} - \overline{cba} = n$. Với mỗi $n\in A$, theo câu (a), tương ứng với 1 trong các trường hợp sau: Nếu $n = 0\Rightarrow a - c = 0\Rightarrow a = c\Rightarrow B = \{\overline{aba}|a = 1,\ldots,9,\,b = 0,1,\ldots,9\}$. Nếu $n = -99\Rightarrow a - c = -1\Rightarrow B = \{\overline{1b2},\overline{2b3},\overline{3b4},\overline{4b5},\overline{5b6},\overline{6b7},\overline{7b8},\overline{8b9}|b = 0,1,\ldots,9\}$. Nếu $n = 99\Rightarrow a - c = 1\Rightarrow B = \{\overline{2b1},\overline{3b2},\overline{4b3},\overline{5b4},\overline{6b5},\overline{7b6},\overline{8b7},\overline{9b8}|b = 0,1,\ldots,9\}$. Nếu $n = -198\Rightarrow a - c = -2\Rightarrow B = \{\overline{1b3},\overline{2b4},\overline{3b5},\overline{4b6},\overline{5b7},\overline{6b8},\overline{7b9}|b = 0,1,\ldots,9\}$. Nếu $n = 198\Rightarrow a - c = 2\Rightarrow B = \{\overline{3b1},\overline{4b2},\overline{5b3},\overline{6b4},\overline{7b5},\overline{8b6},\overline{9b7}|b = 0,1,\ldots,9\}$. Nếu $n = -297\Rightarrow a - c = -3\Rightarrow B = \{\overline{1b4},\overline{2b5},\overline{3b6},\overline{4b7},\overline{5b8},\overline{6b9}|b = 0,1,\ldots,9\}$. Nếu $n = 297\Rightarrow a - c = 3\Rightarrow B = \{\overline{4b1},\overline{5b2},\overline{6b3},\overline{7b4},\overline{8b5},\overline{9b6}|b = 0,1,\ldots,9\}$. Nếu $n = -396\Rightarrow a - c = -4\Rightarrow B = \{\overline{1b5},\overline{2b6},\overline{3b7},\overline{4b8},\overline{5b9}|b = 0,1,\ldots,9\}$. Nếu $n = 396\Rightarrow a - c = 4\Rightarrow B = \{\overline{5b1},\overline{6b2},\overline{7b3},\overline{8b4},\overline{9b5}|b = 0,1,\ldots,9\}$. Nếu $n = -495\Rightarrow a - c = -5\Rightarrow B = \{\overline{1b6},\overline{2b7},\overline{3b8},\overline{4b9}|b = 0,1,\ldots,9\}$. Nếu $n = 495\Rightarrow a - c = 5\Rightarrow B = \{\overline{6b1},\overline{7b2},\overline{8b3},\overline{9b4}|b = 0,1,\ldots,9\}$. Nếu $n = -594\Rightarrow a - c = -6\Rightarrow B = \{\overline{1b7},\overline{2b8},\overline{3b9}|b = 0,1,\ldots,9\}$. Nếu $n = 594\Rightarrow a - c = 6\Rightarrow B = \{\overline{7b1},\overline{8b2},\overline{9b3}|b = 0,1,\ldots,9\}$. Nếu $n = -693\Rightarrow a - c = -7\Rightarrow B = \{\overline{1b8},\overline{2b9}|b = 0,1,\ldots,9\}$. Nếu $n = 693\Rightarrow a - c = 7\Rightarrow B = \{\overline{8b1},\overline{9b2}|b = 0,1,\ldots,9\}$. Nếu $n = -792\Rightarrow a - c = -8\Rightarrow B = \{\overline{1b9}|b = 0,1,\ldots,9\}$. Nếu $n = 792\Rightarrow a - c = 8\Rightarrow B = \{\overline{9b1}|b = 0,1,\ldots,9\}$.
\end{proof}

\begin{baitoan}[\cite{Binh_boi_duong_Toan_6_tap_1}, 1.10., p. 14]
	Có bao nhiêu số tự nhiên có 2 chữ số mà: (a) Trong số đó có ít nhất 1 chữ số $9$? (b) Trong số đó chữ số hàng chục bé hơn chữ số hàng đơn vị? (c) Trong số đó chữ số hàng chục gấp đôi chữ số hàng đơn vị?
\end{baitoan}

\begin{proof}[1st giải]
	(a) 18 số gồm 17 số có 1 chữ số 9: $19,29,39,49,59,69,79,89,90,91,92,93,94,95,96,97,98$ \& 1 số có 2 chữ số 9: 99. (b) 
\end{proof}

\begin{proof}[2nd giải]
	(a) Các số cần tìm có dạng $\overline{ab}$ với $a,b\in\{0,1,2,\ldots,9\}$ với $a = 9$ hoặc $b = 9$. Nếu $a = 9$ thì $b\in\{0,1,2,\ldots,9\}$ có 10 cách chọn. Nếu $b = 9$ thì $a\in\{1,2,\ldots,9\}$ có 9 cách chọn. Nhưng số 99 được tính 2 lần nên tổng cộng có $10 + 9 - 1 = 18$ số thỏa mãn. (b) Các số cần tìm có dạng $\overline{ab}$
\end{proof}

\begin{baitoan}[\cite{Binh_boi_duong_Toan_6_tap_1}, 1.11., p. 14]
	Cho 1 số có 2 chữ số. Nếu viết thêm chữ số 2 vào bên trái \& bên phải số đó ta được số mới gấp $32$ lần số đã cho. Tìm số đã cho.
\end{baitoan}

\begin{baitoan}[\cite{Binh_boi_duong_Toan_6_tap_1}, 1.12., p. 14]
	Mẹ mua cho Hà 1 quyển sổ tay có $358$ trang. Để tiện theo dõi, Hà đánh số trang từ $1$ đến $358$. Hỏi Hà đã phải viết bao nhiêu chữ số để đánh số trang hết cuốn sổ tay đó.
\end{baitoan}

\begin{baitoan}[\cite{Binh_boi_duong_Toan_6_tap_1}, 1.13., p. 14]
	Viết liền nhau các số tự nhiên $123456789101112\ldots$ (a) Hỏi các chữ số hàng đơn vị của các số $49,217,2401$ đứng ở vị trí thứ bao nhiêu kể từ trái sang phải? (b) Chữ số viết ở vị trí thứ $427$ là chữ số nào?
\end{baitoan}

\begin{baitoan}[\cite{Binh_boi_duong_Toan_6_tap_1}, 1.14., p. 15]
	Cho 4 chữ số $a,b,c,d$ đôi một khác nhau \& khác $0$. Tập hợp các số tự nhiên có 3 chữ số gồm 3 trong 4 chữ số $a,b,c,d$ có bao nhiêu phần tử?
\end{baitoan}

\begin{baitoan}[\cite{Binh_boi_duong_Toan_6_tap_1}, 1.15., p. 15]
	Mỗi tập hợp sau đây có bao nhiêu phần tử? (a) Tập hợp các số có 2 chữ số. (b) Tập hợp các số có 2 chữ số được lập nên từ 2 số khác nhau. (c) Tập hợp các số có 3 chữ số được lập nên tử 3 chữ số đôi một khác nhau.
\end{baitoan}

\begin{baitoan}[\cite{Binh_boi_duong_Toan_6_tap_1}, 1.16., p. 15]
	Tổng kết đợt thi đua, lớp 6A có $35$ bạn được 1 điểm $10$ trở lên, $21$ bạn được từ 2 điểm $10$ trở lên, $18$ bạn được 3 điểm $10$ trở lên, $5$ bạn được $4$ điểm $10$. Biết không có ai được trên $4$ điểm $10$, hỏi trong đợt thi đua đó lớp 6A có bao nhiêu điểm $10$?
\end{baitoan}

\begin{baitoan}[\cite{Binh_boi_duong_Toan_6_tap_1}, 1.17., p. 15]
	Tìm số tự nhiên có 4 chữ số, chữ số hàng đơn vị là $9$. Nếu chuyển chữ số hàng đơn vị lên đầu thì được 1 số mới lớn hơn số đã cho $2889$ đơn vị.
\end{baitoan}

\begin{baitoan}[\cite{Binh_boi_duong_Toan_6_tap_1}, 1.18., p. 15]
	Hiệu của 2 số tự nhiên là $53$. Chữ số hàng đơn vị của số bị trừ là $8$. Nếu bỏ chữ số hàng đơn vị của số bị trừ ta được số trừ. Tìm 2 số đó.
\end{baitoan}

\begin{baitoan}[\cite{Binh_boi_duong_Toan_6_tap_1}, 1.19., p. 15]
	Tìm 1 số có 5 chữ số biết nếu viết chữ số $7$ đằng trước số đó thì được số lớn gấp $5$ lần số có được bằng cách viết thêm chữ số $7$ vào đằng sau số đó.
\end{baitoan}

\begin{baitoan}[\cite{Binh_boi_duong_Toan_6_tap_1}, 1.20., p. 15]
	1 số gồm 3 chữ số tận cùng là chữ số $9$, nếu chuyển chữ số $9$ đó lên đầu thì được 1 số mới mà khi chia cho số cũ thì được thương là $3$ dư $61$. Tìm số đó.
\end{baitoan}

\begin{baitoan}[\cite{Binh_boi_duong_Toan_6_tap_1}, p. 15]
	Trong 1 lớp học, tất cả học sinh nam đều tham gia vào các câu lạc bộ thể thao: bóng đá, bóng chuyền, cầu lông. Biết có $7$ học sinh tham gia câu lạc bộ bóng đá, $6$ học sinh tham gia câu lạc bộ bóng chuyền, $5$ học sinh tham gia câu lạc bộ cầu lông; trong số đó có $4$ học sinh tham gia cả 2 câu lạc bộ bóng đá \& bóng chuyền, $3$ học sinh tham gia cả 2 câu lạc bộ bóng đá \& cầu lông, $2$ học sinh tham gia cả 2 câu lạc bộ bóng chuyền \& cầu lông, $1$ học sinh tham gia cả 3 câu lạc bộ. Hỏi lớp đó có bao nhiêu học sinh nam?
\end{baitoan}

\begin{baitoan}[\cite{Tuyen_Toan_6}, VD2, p. 6]
	Phố Hàng Ngang là 1 trong các phố cổ của Hà Nội. Các nhà được đánh số liên tục, dãy lẻ $1,3,5,7,\ldots,61$; dãy chẵn $2,4,6,\ldots,64$. (a) Bên số nhà chẵn, trong 1 phòng gác nhỏ, chủ tịch Hồ Chí Minh đã khởi thảo bản Tuyên Ngôn Độc Lập khai sinh cho nước Việt Nam Dân Chủ Cộng Hòa. Ngôi nhà có căn phòng đó là nhà thứ $24$ kể từ đầu phố (số $2$). Hỏi ngôi nhà này có số nào? (b) Bên số nhà lẻ chữ số nào được dùng nhiều nhất? Chữ số nào chưa được dùng đến? (c) Phải dùng tất cả bao nhiêu chữ số để ghi số nhà của phố này?
\end{baitoan}

\begin{proof}[Giải]
	(a) Ngôi nhà thứ 24 bên dãy số nhà chẵn có số $2\cdot24 = 48$. (b) Bên số lẻ chữ số 1 dùng tới 7 lần ở hàng đơn vị (nhiều nhất so với các chữ số khác); dùng tới 5 lần ở hàng chục (không kém so với các chữ số khác). Vậy chữ số 1 được dùng nhiều nhất (12 lần). Chữ số 0 không dùng ở hàng đơn vị cũng như ở hàng chục. Chữ số 8 không dùng ở hàng đơn vị còn ở hàng chục thì chưa dùng tới. Vậy bên số lẻ thì chữ số 0 \& chữ số 8 chưa được dùng đến. (c) Tạm chưa tính nhà 64 thì phố này có 62 nhà từ nhà $1,2,3,\ldots,62$. Trong dãy số này có 9 số có 1 chữ số \& có $62 - 9 = 53$ số có 2 chữ số. Số chữ số cần dùng để viết các số này trừ nhà 64 ra là $9\cdot1 + 53\cdot2 = 9 + 106 = 115$. Thêm 2 chữ số của nhà 64, tổng các chữ số cần dùng là $115 + 2 = 117$ chữ số.
\end{proof}

\begin{luuy}
	Công thức tính số chữ số cần dùng để ghi các số tự nhiên liên tiếp: Gọi số các số có $1$ chữ số là $a_1$, số các số có $2$ chữ số là $a_2$, $\ldots$,  số các số có $n$ chữ số là $a_n$ (hay viết gọn: gọi số các số có $i$ chữ số là $a_i$, $\forall i = 1,2,\ldots,n$) thì số chữ số cần dùng là: $S = \sum_{i=1}^n ia_i = a_1 + 2a_2 + \cdots + na_n$.
\end{luuy}

\begin{baitoan}[\cite{Tuyen_Toan_6}, 6., p. 6]
	Viết tập hợp $4$ số tự nhiên liên tiếp lớn hơn $94$ nhưng không quá $100$.
\end{baitoan}

\begin{proof}[Giải]
	Có 3 tập hợp thỏa mãn: $\{95,96,97,98\},\{96,97,98,99\},\{97,98,99,100\}$.
\end{proof}

\begin{baitoan}[\cite{Tuyen_Toan_6}, 7., p. 6]
	(a) Có bao nhiêu số tự nhiên nhỏ hơn $20$? (b) Có bao nhiêu số tự nhiên nhỏ hơn $n\in\mathbb{N}$? (c) Có bao nhiêu số tự nhiên chẵn nhỏ hơn $n\in\mathbb{N}$? (d) Có bao nhiêu số tự nhiên lẻ nhỏ hơn $n\in\mathbb{N}$?
\end{baitoan}

\begin{proof}[Giải]
	(a) Có 20 số tự nhiên nhỏ hơn 20: $0,1,2,\ldots,19$. (b) Có $n$ số tự nhiên nhỏ hơn $n$: $0,1,2,\ldots,n - 1$. (c) Xét 2 trường hợp: Nếu $n$ chẵn thì có $\frac{n}{2}$ số tự nhiên chẵn nhỏ hơn $n$: $0,2,\ldots,n - 2$. Nếu $n$ lẻ thì có $\frac{n + 1}{2}$ số tự nhiên chẵn nhỏ hơn $n$: $0,2,\ldots,n - 1$. (d) Xét 2 trường hợp: Nếu $n$ chẵn thì có $\frac{n}{2}$ số tự nhiên lẻ nhỏ hơn $n$: $1,3,\ldots,n - 1$. Nếu $n$ lẻ thì có $\frac{n + 1}{2}$ số tự nhiên lẻ nhỏ hơn $n$: $1,3,\ldots,n - 2$.
\end{proof}

\begin{baitoan}[\cite{Tuyen_Toan_6}, 8., p. 7]
	(a) Có bao nhiêu số có $4$ chữ số mà cả $4$ chữ số đều giống nhau? (b) Có bao nhiêu số có $4$ chữ số? (c) Có bao nhiêu số có $n$ chữ số, với $n\in\mathbb{N}$?
\end{baitoan}

\begin{proof}[1st giải]
	(a) Có 9 số: $1111,2222,\ldots,9999$. (b) Có $9999 - 1000 + 1 = 9000$ số có 4 chữ số: $1000,1001,\ldots,9999$. (c) Có $\underbrace{99\ldots9}_n - 1\underbrace{00\ldots0}_{n-1} + 1 = 9\underbrace{00\ldots0}_{n-1}$ số có $n$ chữ số: $1\underbrace{00\ldots0}_{n-1},1\underbrace{00\ldots0}_{n-2}1,\ldots,\underbrace{99\ldots9}_n$.
\end{proof}

\begin{proof}[2nd giải]
	(c) Các số có $n$ chữ số: $10^{n-1},10^{n-1} + 1,\ldots,10^n - 1$. Suy ra có $10^n - 1 - 10^{n-1} + 1 = 10^n - 10^{n-1}$ số có $n$ chữ số.
\end{proof}

\begin{baitoan}[\cite{Tuyen_Toan_6}, 9., p. 7]
	Đèn hướng dẫn giao thông liên tục sáng màu xanh hoặc đỏ kế tiếp nhau. Bảng hiện số của đèn có 2 chữ số liên tục thay đổi theo từng giây. Hỏi trong 1 phút xe bị dừng vì đèn đỏ thì đèn có: (a) Bao nhiêu lần thay đổi các số? (b) Bao nhiêu lần thay đổi các chữ số?
\end{baitoan}

\begin{proof}[Giải]
	(a) Trong 1 phút $=$ 60 giây có 60 lần thay đổi số. (b) Hàng đơn vị thay đổi 60 lần, hàng chục thay đổi 6 lần ($6\to5\to4\to3\to2\to1\to0$) nên tổng cộng có $60 + 6 = 66$ lần thay đổi chữ số.
\end{proof}

\begin{baitoan}[\cite{Tuyen_Toan_6}, 10., p. 7]
	Tìm $3$ số tự nhiên $a,b,c$ biết chúng thỏa mãn đồng thời 3 điều kiện: $a < b < c$, $101\le a\le103$, $101 < c < 104$.
\end{baitoan}

\begin{proof}[1st giải]
	Từ điều kiện $101\le a\le103$ \& $a\in\mathbb{N}$, suy ra $a\in\{101,102,103\}$. Từ điều kiện $101 < c < 104$ \& $c\in\mathbb{N}$, suy ra $c\in\{102,103\}$. Mặt khác, kết hợp 2 điều vừa tập giá trị vừa thu được với điều kiện $a < b < c$, suy ra $a = 101$, $b = 102$, $c = 103$.
\end{proof}

\begin{proof}[2nd giải]
	Vì $a < b < c$ \& $a,b,c\in\mathbb{N}$ nên $c\ge b + 1\ge a + 2$ (1). Kết hợp (1) với điều kiện $101\le a$, suy ra $c\ge a + 2\ge101 + 2 = 103$ (2). Kết hợp (2) với điều kiện $c < 104$, suy ra $c = 103$. Dấu ``$=$'' xảy ra $\Leftrightarrow c = b + 1 = a + 2\Leftrightarrow(a,b,c) = (101,102,103)$.
\end{proof}

\begin{baitoan}[\cite{Tuyen_Toan_6}, 11., p. 7]
	Cho số $4321$. Viết thêm chữ số $9$ xen giữa các chữ số của nó để được 1 số: (a) Lớn nhất có thể được. (b) Nhỏ nhất có thể được.
\end{baitoan}

\begin{proof}[Giải]
	(a) Số lớn nhất có thể: 49321. (b) Số nhỏ nhất có thể: 43291.
\end{proof}

\begin{baitoan}[\cite{Tuyen_Toan_6}, 12., p. 7]
	Với $9$ que diêm, sắp xếp thành 1 số La Mã: (a) Có giá trị lớn nhất. (b) Có giá trị nhỏ nhất.
\end{baitoan}

\begin{proof}[Giải]
	(a) MMI: 2001. (b) XXVIII: 28.
\end{proof}

\begin{baitoan}[\cite{Tuyen_Toan_6}, 13., p. 7]
	Có $13$ que diêm sắp xếp như sau: $\rm XII - V = VII$. (a) Đẳng thức trên đúng hay sai? (b) Đổi chỗ chỉ 1 que diêm để được 1 đẳng thức đúng.
\end{baitoan}

\begin{proof}[Giải]
	(a) Đẳng thức XII $-$ V = VII đúng vì $12 - 5 = 7$. (b) XI $-$ IV = VII: $11 - 4 = 7$, hoặc XII $-$ VI = VI: $12 - 6 = 6$.
\end{proof}

\begin{baitoan}[\cite{Binh_Toan_6_tap_1}, VD1, p. 4]
	Viết các tập hợp sau rồi tìm số phần tử của mỗi tập hợp đó: (a) Tập hợp $A$ các số tự nhiên $x$ mà $8:x = 2$. (b) Tập hợp $B$ các số tự nhiên $x$ mà $x + 3 < 5$. (c) Tập hợp $C$ các số tự nhiên $x$ mà $x - 2 = x + 2$. (d) Tập hợp $D$ các số tự nhiên $x$ mà $x:2 = x:4$. (e) Tập hợp $E$ các số tự nhiên $x$ mà $x + 0 = x$.
\end{baitoan}

\begin{proof}[Giải]
	(a) $A = \{4\}$ có 1 phần tử. (b) $B = \{0,1\}$ có 2 phần tử. (c) $C = \emptyset$ không có phần tử nào. (d) $D = \{0\}$ có 1 phần tử. (e) $E = \{0,1,2,\ldots\} = \mathbb{N}$ có vô hạn đếm được số phần tử.
\end{proof}

\begin{baitoan}[\cite{Binh_Toan_6_tap_1}, VD2, p. 5]
	Viết các tập hợp sau bằng cách liệt kê các phần tử của nó: (a) Tập hợp $A$ các số tự nhiên có 2 chữ số, trong đó chữ số hàng chục lớn hơn chữ số hàng đơn vị là $2$. (b) Tập hợp $B$ các số tự nhiên có 3 chữ số mà tổng các chữ số bằng $3$.
\end{baitoan}

\begin{proof}[Giải]
	(a) $A = \{20,31,42,53,64,75,86,97\}$. (b) $B = \{102,111,120,201,210,300\}$.
\end{proof}

\begin{baitoan}[\cite{Binh_Toan_6_tap_1}, VD3, p. 5]
	Tìm số tự nhiên có 5 chữ số, biết nếu viết thêm chữ số $2$ vào đằng sau số đó thì được số lớn gấp 3 lần số có được bằng cách viết thêm chữ số $2$ vào đằng trước số đó.
\end{baitoan}

\begin{proof}[1st giải]
	Gọi số phải tìm là $\overline{abcde}$, có phép nhân: $\overline{2abcde}\cdot3 = \overline{abcde2}$. Lần lượt tìm từng chữ số ở số bị nhân từ phải sang trái: $3e$ tận cùng $2\Rightarrow e = 4$, có: $3\cdot4 = 12$, nhớ 1 sang hàng chục. $3d + 1$ tận cùng $4\Rightarrow d = 1$. $3c$ tận cùng $1\Rightarrow c = 7$, có $3\cdot7 = 21$, nhớ 2 sang hàng nghìn. $3b + 2$ tận cùng $7\Rightarrow b = 5$, có $3\cdot5 = 15$, nhớ 1 sang hàng chục nghìn. $3a + 1$ tận cùng $5\Rightarrow a = 8$, có $3\cdot8 = 24$, nhớ 2 sang hàng trăm nghìn. $3\cdot2 + 2 = 8$. Tổng hợp lại: $285714\cdot3 = 857142$. Vậy số đó là 85714.
\end{proof}

\begin{proof}[2nd giải]
	Gọi số cần tìm là $x = \overline{abcde}$, có $\overline{abcde2} = 3\overline{2abcde}\Leftrightarrow10x + 2 = 3(200000 + x)\Leftrightarrow10x + 2 = 600000 + 3x\Leftrightarrow7x = 599998\Leftrightarrow x = 599998:7 = 85714$. Vậy số đó là 85714.
\end{proof}

\begin{baitoan}[\cite{Binh_Toan_6_tap_1}, Mở rộng VD3, p. 5]
	(a) Tìm số tự nhiên nhỏ nhất có chữ số đầu tiên ở bên trái là $2$, khi chuyển chữ số $2$ này xuống cuối cùng thì số đó tăng gấp 3 lần. (b) Tìm tất cả các số tự nhiên thỏa mãn câu (a).
\end{baitoan}

\begin{proof}[Giải]
	Gọi số tự nhiên cần tìm là $\overline{2a_na_{n-1}\ldots a_1a_0}$, đặt $x\coloneqq\overline{a_na_{n-1}\ldots a_1a_0}$, có $3\overline{2a_na_{n-1}\ldots a_1a_0} = \overline{a_na_{n-1}\ldots a_1a_02}\Leftrightarrow3(2\cdot10^n + x) = 10x + 2\Leftrightarrow7x = 6\cdot10^n - 2\Leftrightarrow x = \dfrac{6\cdot10^n - 2}{7}$. Nếu $n = 1$, $x = \dfrac{58}{7}$. Nếu $n = 2$, $x = \dfrac{598}{7}$. Nếu $n = 3$, $x = \dfrac{5998}{7}$. Nếu $n = 4$, $x = \dfrac{59998}{7}$. Nếu $n = 5$, $x = \dfrac{599998}{7} = 85714$. Vậy số tự nhiên nhỏ nhất thỏa mãn là 285714. (b) 
\end{proof}

\begin{baitoan}[\cite{Binh_Toan_6_tap_1}, Mở rộng VD3, p. 6]
	Tìm số tự nhiên có 5 chữ số, biết nếu viết thêm 1 chữ số vào đằng sau số đó thì được số lớn gấp 3 lần số có được nếu viết thêm chính chữ số ấy vào đằng trước số đó.
\end{baitoan}

\begin{baitoan}
	Tìm số tự nhiên sao cho khi thêm 1 chữ số vào đằng sau số đó thì được số lớn gấp $a\in\mathbb{N}^\star$ lần số có được nếu viết thêm chính chữ số ấy vào đằng trước số đó.
\end{baitoan}

\begin{baitoan}
	Tìm số tự nhiên sao cho khi thêm 1 chữ số vào đằng trước số đó thì được số lớn gấp $a\in\mathbb{N}^\star$ lần số có được nếu viết thêm chính chữ số ấy vào đằng sau số đó.
\end{baitoan}

\begin{baitoan}[\cite{Binh_Toan_6_tap_1}, 2., p. 6]
	Xác định các tập hợp sau bằng cách chỉ ra tính chất đặc trưng của các phần tử thuộc tập hợp đó: (a) $A = \{1,3,5,7,\ldots,49\}$. (b) $B = \{11,22,33,44,\ldots,99\}$. (c) $C = \{\mbox{tháng } 1,\mbox{tháng } 3,\mbox{tháng } 5,\mbox{tháng } 7,\mbox{tháng } 8,\mbox{tháng } 10,\mbox{tháng } 12\}$.
\end{baitoan}

\begin{proof}[Giải]
	(a) $A$ là tập hợp các số lẻ nhỏ hơn 50: $A = \{n\in\mathbb{N}|n\mbox{ không chia hết cho } 2,\,n < 50\} = \{n\in\mathbb{N}|n\not{\divby}\ 2,\,n < 50\} = \{n\in\mathbb{N}|n\equiv 1\mod2,\,n < 50\}$. (b) $B$ là tập hợp các số tự nhiên có 2 chữ số giống nhau: $B = \{\overline{aa}|a\in\{1,2,\ldots,9\}\} = \{\overline{aa}|a\in\mathbb{N},1\le a\le9\}$ hoặc $B$ là tập hợp các số tự nhiên có 2 chữ số \& chia hết cho 11: $B = \{n\in\mathbb{N}|n\divby11,\,10\le n\le99\}$. (c) $C$ là tập hợp các tháng có 31 ngày của năm dương lịch.
\end{proof}

\begin{baitoan}[\cite{Binh_Toan_6_tap_1}, 3., p. 6]
	Tìm tập hợp các số tự nhiên $x$ sao cho: (a) $x + 3 = 4$. (b) $8 - x = 5$. (c) $x:2 = 0$. (d) $0:x = 0$. (e) $5x = 12$.
\end{baitoan}

\begin{proof}[Giải]
	(a) $A = \{1\}$. (b) $B = \{3\}$. (c) $C = \{0\}$. (d) $D = \mathbb{N}^\star$. (e) $E = \emptyset$.
\end{proof}

\begin{baitoan}[\cite{Binh_Toan_6_tap_1}, 4., p. 6]
	Tìm $a,b\in\mathbb{N}$ sao cho $12 < a < b < 16$.
\end{baitoan}

\begin{proof}[Giải]
	$a,b\in\mathbb{N}$ \& $12 < a < b < 16\Leftrightarrow13\le a < b\le15$. $(a,b)\in\{(13,14),(13,15),(14,15)\}$.
\end{proof}

\begin{baitoan}[\cite{Binh_Toan_6_tap_1}, 5., p. 6]
	Viết các số tự nhiên có 4 chữ số trong đó có 2 chữ số $3$, 1 chữ số $2$, 1 chữ số $1$.
\end{baitoan}

\begin{proof}[Giải]
	Có 12 số thỏa mãn: Chữ số 1 đứng đầu: 1233, 1323, 1332. Chữ số 2 đứng đầu: 2133, 2313, 2331. Chữ số 3 đứng đầu: 3123, 3132, 3213, 3231, 3312, 3321.
\end{proof}

\begin{baitoan}[\cite{Binh_Toan_6_tap_1}, 6., p. 6]
	Với cả 2 chữ số {\rm I} \& {\rm X}, viết được bao nhiêu số La Mã? (Mỗi chữ số có thể viết nhiều lần, nhưng không viết liên tiếp quá 3 lần).
\end{baitoan}

\begin{proof}[Giải]
	Có 13 chữ số thỏa mãn: Các số chứa 1 chữ số X: IX, XI, XII, XIII. Các số chứa 2 chữ số X: XIX, XXI, XXII, XXIII. Các số chứa 3 chữ số X: XXIX, XXXI, XXXII, XXXIII. Các số chứa 4 chữ số X: XXXIX.
\end{proof}

\begin{baitoan}[\cite{Binh_Toan_6_tap_1}, 7., pp. 6--7]
	(a) Dùng 3 que diêm, xếp được các số La Mã nào? (b) Để viết các số La Mã từ $4000$ trở lên, e.g. số $19520$, người ta viết {\rm XIXmDXX} (chữ {\rm m} biểu thị \emph{1 nghìn}, m là chữ đầu của từ \emph{mille}, tiếng Latin là 1 nghìn). Viết 2 số sau bằng chữ số La Mã: $7203, 121512$.
\end{baitoan}

\begin{proof}[Giải]
	(a) Xếp được 7 số La Mã: III: 3, IV: 4, VI: 6, IX: 9, XI: 11, LI: 51, C: 100. (b) VIImCCIII: 7203, CXXImDXII: 121512.
\end{proof}

\begin{baitoan}[\cite{Binh_Toan_6_tap_1}, 8., p. 7]
	Tìm số tự nhiên có tận cùng bằng $3$, biết rằng nếu xóa chữ số hàng đơn vị thì số đó giảm đi $1992$ đơn vị.
\end{baitoan}

\begin{proof}[Giải]
	Gọi số cần tìm có dạng $\overline{x3}$, $x\in\mathbb{N}^\star$, có $\overline{x3} - x = 1992\Leftrightarrow10x + 3 - x = 1992\Leftrightarrow9x = 1992 - 3 = 1989\Leftrightarrow x = 1989:9 = 221$. Vậy số cần tìm là $2213$.
\end{proof}

\begin{baitoan}[\cite{Binh_Toan_6_tap_1}, 9., p. 7]
	Tìm số tự nhiên có 6 chữ số, biết rằng chữ số hàng đơn vị là $4$ \& nếu chuyển chữ số đó lên hàng đầu tiên thì số đó tăng gấp 4 lần.
\end{baitoan}

\begin{proof}[Giải]
	Số cần tìm có dạng $\overline{abcde4}$, đặt $x\coloneqq\overline{abcde}$, có: $4\overline{abcde4} = \overline{4abcde}\Leftrightarrow4(10x + 4) = 4\cdot10^5 + x\Leftrightarrow39x = 4\cdot10^5 - 4\cdot4 = 399984\Leftrightarrow x = 399984:39 = 10256$. Vậy số cần tìm là 102564.
\end{proof}

\begin{baitoan}[\cite{Binh_Toan_6_tap_1}, 10., p. 7]
	Cho 4 chữ số $a,b,c,d$ khác nhau \& khác $0$. Lập số tự nhiên lớn nhất \& số tự nhiên nhỏ nhất có 4 chữ số gồm cả 4 chữ số ấy. Tổng của 2 số này bằng $11330$. Tìm tổng các chữ số $a + b + c + d$.
\end{baitoan}

\begin{proof}[1st giải]
	Không mất tính tổng quát (without loss of generality, abbr., w.l.o.g.), giả sử $a > b > c > d > 0$. Số lớn nhất \& số nhỏ nhất có thể lập được từ cả 4 chữ số này lần lượt là $\overline{abcd},\overline{dcba}$. Có $\overline{abcd} + \overline{dcba} = 11330$. Từ hàng đơn vị suy ra $a + d = 10$, viết 0 nhớ 1. Từ hàng chục \& hàng trăm suy ra $b + c = 12$ (vì nếu $b + c = 2$ thì hàng trăm của tổng phải bằng 2 thay vì 3, mâu thuẫn với giả thiết $\overline{abcd} + \overline{dcba} = 11330$). Suy ra $a + b + c + d = (a + d) + (b + c) = 10 + 12 = 22$.
\end{proof}

\begin{proof}[2nd giải]
	Giả sử $a > b > c > d > 0$. Số lớn nhất \& số nhỏ nhất có thể lập được từ cả 4 chữ số này lần lượt là $\overline{abcd},\overline{dcba}$. Có $\overline{abcd} + \overline{dcba} = 11330\Leftrightarrow1000a + 100b + 10c + d + 1000d + 100c + 10b + a = 1001(a + d) + 110(b + c) = 11330$. Nếu $a + d\ge11$ thì $11330 = 1001(a + d) + 110(b + c)\ge1001\cdot11 + 110(3 + 2) 11561 > 11330$, vô lý. Nếu $a + d\le9$ thì $110(b + c) = 11330 - 1001(a + d)\ge11330 - 1001\cdot9 = 2321\Rightarrow b + c\ge2321:110 = 21.1 > 20$, vô lý vì $b\le8$, $c\le7$ nên $b + c\le 8 + 7 = 15$. Suy ra $a + d = 10$, thay vào đẳng thức $1001(a + d) + 110(b + c) = 11330$ được: $110(b + c) = 11330 - 1001\cdot10 = 1320\Leftrightarrow b + c = 1320:110 = 12$. Suy ra $a + b + c + d = (a + d) + (b + c) = 10 + 12 = 22$.
\end{proof}

\begin{baitoan}[\cite{Binh_Toan_6_tap_1}, 11., p. 7]
	Cho 3 chữ số $a,b,c$ sao cho $0 < a < b < c$. (a) Viết tập hợp $A$ các số tự nhiên có 3 chữ số gồm cả 3 chữ số $a,b,c$. (b) Biết tổng 2 số nhỏ nhất trong tập hợp $A$ bằng $488$. Tìm 3 chữ số $a,b,c$ nói trên.
\end{baitoan}

\begin{proof}[1st giải]
	(a) $A = \{\overline{abc},\overline{acb},\overline{bac},\overline{bca},\overline{cab},\overline{cba}\}$. (b) 2 số nhỏ nhất trong tập $A$ là $\overline{abc},\overline{acb}$, có $\overline{abc} + \overline{acb} = 488$. Xét phép cộng ở cột hàng đơn vị \& cột hàng chục, ta thấy $c + b$ không có nhớ nên $b + c = 8$, $a + a = 4\Leftrightarrow$ suy ra $a = 2$. Vì $2 = a < b < c$ \& $b + c = 8$, suy ra $b = 3$, $c = 5$. Vậy $a = 2$, $b = 3$, $c = 5$.
\end{proof}

\begin{proof}[2nd giải]
	(a) $A = \{\overline{abc},\overline{acb},\overline{bac},\overline{bca},\overline{cab},\overline{cba}\}$. (b) 2 số nhỏ nhất trong tập $A$ là $\overline{abc},\overline{acb}$, có $\overline{abc} + \overline{acb} = 488\Leftrightarrow200a + 11(b + c) = 488$. Nếu $a\ge3$ thì $488 = 200a + 11(b + c) > 200\cdot3 = 600 > 488$, vô lý. Nếu $a = 1$ thì $11(b + c) = 488 - 200 = 288\Leftrightarrow b + c = 288:11 = 26.(18)\notin\mathbb{Z}$, vô lý. Suy ra $a = 2$, thay vào đẳng thức $200a + 11(b + c) = 488$ được $11(b + c) = 488 - 200\cdot2 = 88\Leftrightarrow b + c = 88:11 = 8$, mà $2 = a < b < c$ \& $b + c = 8$, suy ra $b = 3$, $c = 5$. Vậy $a = 2$, $b = 3$, $c = 5$.
\end{proof}

\begin{baitoan}[\cite{Binh_Toan_6_tap_1}, 12., p. 7]
	Tìm 3 chữ số khác nhau \& khác $0$, biết rằng nếu dùng cả 3 chữ số này lập thành các số tự nhiên có 3 chữ số thì 2 số lớn nhất có tổng bằng $1444$.
\end{baitoan}

\begin{proof}[1st giải]
	Gọi 3 chữ số phải tìm là $a,b,c\in\{1,2,\ldots,9\}$, $a > b > c > 0$. 2 số lớn nhất lập bởi 3 chữ số này là $\overline{abc},\overline{acb}$, có $\overline{abc} + \overline{acb} = 1444$. So sánh 2 cột đơn vị \& cột hàng chục, ta thấy phép cộng $c + b$ không có nhớ nên $b + c = 4$, mà $b > c > 0$ nên $b = 3$, $c = 1$. Xét cột hàng trăm: $a + a  = 14\Leftrightarrow2a = 14\Leftrightarrow a = 7$. Vậy 3 chữ số phải tìm là $7,3,1$.
\end{proof}

\begin{proof}[2nd giải]
	Gọi 3 chữ số phải tìm là $a,b,c\in\{1,2,\ldots,9\}$, $a > b > c > 0$. 2 số lớn nhất lập bởi 3 chữ số này là $\overline{abc},\overline{acb}$, có $\overline{abc} + \overline{acb} = 1444\Leftrightarrow100a + 10b + c + 100a + 10c + b = 200a + 11(b + c) = 1444$. Nếu $a\ge8$ thì $1444 = 200a + 11(b + c) > 200\cdot8 = 1600 > 1444$ vô lý. Nếu $a\le6$, vì $a > b > c > 0$ nên $b\le5$, $c\le4$, suy ra $1444 = 200a + 11(b + c)\le200\cdot6 + 11(5 + 4) = 1299 < 1444$ vô lý. Suy ra $a = 7$. Thay vì phương trình, được $11(b + c) = 1444 - 200\cdot7 = 44\Leftrightarrow b + c = 4$ mà $b > c > 0$ nên suy ra $b = 3$, $c = 1$. Vậy 3 chữ số phải tìm là $7,3,1$.
\end{proof}

\begin{baitoan}[Even vs. odd -- Chẵn vs. lẻ]
	Viết tập hợp theo nhiều cách nhất có thể: (a) Tập hợp các số tự nhiên chẵn. (b) Tập hợp các số tự nhiên lẻ. (c) Tập hợp các số nguyên dương chẵn. (d) Tập hợp các số nguyên dương lẻ. (e) Tập hợp các số nguyên chẵn. (f) Tập hợp các số nguyên lẻ.
\end{baitoan}

\begin{proof}[Giải]
	(a) Tập hợp các số tự nhiên chẵn $A = \{0,2,4,\ldots\} = \{n\in\mathbb{N}|n \mbox{ chia hết cho }2\} = \{n\in\mathbb{N}|n\divby2\} = \{n\in\mathbb{N}|n\equiv0\mod2\} = \{2n|n\in\mathbb{N}\} = {\rm B}(2)\cap\mathbb{N}$. (b) Tập hợp các số tự nhiên lẻ $B = \{1,3,5,\ldots\} = \{n\in\mathbb{N}|n \mbox{ không chia hết cho }2\} = \{n\in\mathbb{N}|n\not{\divby}\ 2\} = \{n\in\mathbb{N}|n\equiv1\mod2\} = \{2n + 1|n\in\mathbb{N}\} = \mathbb{N}\backslash{\rm B}(2)$. (c) Tập hợp các số nguyên dương chẵn $C = \{2,4,6,\ldots\} = \{n\in\mathbb{N}^\star|n \mbox{ chia hết cho }2\} = \{n\in\mathbb{N}^\star|n\divby2\} = \{n\in\mathbb{N}^\star|n\equiv0\mod2\} = \{2n|n\in\mathbb{N}^\star\} = {\rm B}(2)\cap\mathbb{N}\backslash\{0\} = {\rm B}(2)\cap\mathbb{N}^\star$. (d) Tập hợp các số nguyên dương lẻ $D\equiv B = \{1,3,5,\ldots\} = \{n\in\mathbb{N}|n \mbox{ không chia hết cho }2\} = \{n\in\mathbb{N}|n\not{\divby}\ 2\} = \{n\in\mathbb{N}|n\equiv1\mod2\} = \{2n + 1|n\in\mathbb{N}\} = \mathbb{N}\backslash{\rm B}(2)$. (e) Tập hợp các số nguyên chẵn $E = \{\ldots,-4,-2,0,2,4,\ldots\} = \{0,\pm2,\pm4,\ldots\} = \{n\in\mathbb{Z}|n \mbox{ chia hết cho }2\} = \{n\in\mathbb{Z}|n\divby2\} = \{n\in\mathbb{Z}|n\equiv0\mod2\} = \{2n|n\in\mathbb{Z}\} = {\rm B}(2)$.\footnote{Trong tài liệu này, ký hiệu ${\rm B}(n)$ dùng để ký hiệu tập hợp tất cả các \textit{bội số nguyên} của $n\in\mathbb{Z}$ bất kỳ, phân biệt với tập hợp ${\rm B}(n)\cap\mathbb{N}$ (chỉ giới hạn trên tập $\mathbb{N}$ các số tự nhiên) các \textit{bội số tự nhiên} của $n$. Theo định nghĩa này, hiển nhiên ta có đẳng thức ${\rm B}(n)\equiv{\rm B}(n)\cap\mathbb{Z}$, $\forall n\in\mathbb{Z}$.} (f) Tập hợp các số nguyên lẻ $F = \{\ldots,-5,-3,-1,1,3,5,\ldots\} = \{n\in\mathbb{Z}|n \mbox{ không chia hết cho }2\} = \{n\in\mathbb{Z}|n\not{\divby}\ 2\} = \{n\in\mathbb{Z}|n\equiv1\mod2\} = \{2n + 1|n\in\mathbb{Z}\} = \mathbb{Z}\backslash{\rm B}(2)$.
\end{proof}

\begin{baitoan}[1-sided bounded subset of $\mathbb{N}$ -- Tập con của $\mathbb{N}$ chỉ bị chặn 1 phía]
	Cho $a\in\mathbb{N}$ cho trước. Viết các tập hợp sau theo nhiều cách nhất có thể: (a) Tập hợp các số tự nhiên nhỏ hơn $a$. (b) Tập hợp các số tự nhiên lớn hơn $a$. (c) Tập hợp các số tự nhiên nhỏ hơn hoặc bằng $a$. (d) Tập hợp các số tự nhiên lớn hơn hoặc bằng $a$. (e) Tập hợp các số tự nhiên chẵn nhỏ hơn $a$. (f) Tập hợp các số tự nhiên chẵn lớn hơn $a$. (g) Tập hợp các số tự nhiên chẵn nhỏ hơn hoặc bằng $a$. (h) Tập hợp các số tự nhiên chẵn lớn hơn hoặc bằng $a$. (i) Tập hợp các số tự nhiên lẻ nhỏ hơn $a$. (j) Tập hợp các số tự nhiên lẻ lớn hơn $a$. (k) Tập hợp các số tự nhiên lẻ nhỏ hơn hoặc bằng $a$. (l) Tập hợp các số tự nhiên lẻ lớn hơn hoặc bằng $a$.
\end{baitoan}

\begin{proof}[Giải]
	(a) Tập hợp các số tự nhiên nhỏ hơn $a$: $A(a) = \{n\in\mathbb{N}|n < a\} = \{0,1,\ldots,a - 1\} = [0,a)\cap\mathbb{N} = (-\infty,a)\cap\mathbb{N}$. Chú ý $A(0) = \emptyset$, $A(1) = \{0\}$. (b) Tập hợp các số tự nhiên lớn hơn $a$: $B(a) = \{n\in\mathbb{N}|n > a\} = \{a + 1,a + 2,\ldots\} = (a,\infty)\cap\mathbb{N} = [a + 1,\infty)\cap\mathbb{N}$. Chú ý $B(0) = \mathbb{N}^\star$. (c) Tập hợp các số tự nhiên nhỏ hơn hoặc bằng $a$: $C(a) = \{n\in\mathbb{N}|n\le a\} = \{0,1,\ldots,a - 1,a\} = [0,a]\cap\mathbb{N} = [0,a + 1)\cap\mathbb{N} = (-\infty,a]\cap\mathbb{N} = (-\infty,a + 1)\cap\mathbb{N}$. Chú ý $C(0) = \{0\}$, $C(1) = \{0,1\}$. (d) Tập hợp các số tự nhiên lớn hơn hoặc bằng $a$: $D(a) = \{n\in\mathbb{N}|n\ge a\} = \{a,a + 1,a + 2,\ldots\} = (a - 1),\infty)\cap\mathbb{N} = [a,\infty)\cap\mathbb{N}$. (e) Tập hợp các số tự nhiên chẵn nhỏ hơn $a$. (f) Tập hợp các số tự nhiên chẵn lớn hơn $a$. (g) Tập hợp các số tự nhiên chẵn nhỏ hơn hoặc bằng $a$. (h) Tập hợp các số tự nhiên chẵn lớn hơn hoặc bằng $a$. (i) Tập hợp các số tự nhiên lẻ nhỏ hơn $a$. (j) Tập hợp các số tự nhiên lẻ lớn hơn $a$. (k) Tập hợp các số tự nhiên lẻ nhỏ hơn hoặc bằng $a$. (l) Tập hợp các số tự nhiên lẻ lớn hơn hoặc bằng $a$.
\end{proof}

\begin{baitoan}[2-sided bounded subset of $\mathbb{N}$ -- Tập con của $\mathbb{N}$ bị chặn cả 2 phía]
	Với $a,b\in\mathbb{N}$ cho trước. Viết các tập hợp sau theo nhiều cách nhất có thể: (a) Tập hợp các số tự nhiên lớn hơn $a$ \& nhỏ hơn $b$. (b) Tập hợp các số tự nhiên lớn hơn hoặc bằng $a$ \& nhỏ hơn $b$. (c) Tập hợp các số tự nhiên lớn hơn $a$ \& nhỏ hơn hoặc bằng $b$. (d) Tập hợp các số tự nhiên lớn hơn hoặc bằng $a$ \& nhỏ hơn hoặc bằng $b$. (e) Tập hợp các số tự nhiên chẵn lớn hơn $a$ \& nhỏ hơn $b$. (f) Tập hợp các số tự nhiên chẵn lớn hơn hoặc bằng $a$ \& nhỏ hơn $b$. (g) Tập hợp các số tự nhiên chẵn lớn hơn $a$ \& nhỏ hơn hoặc bằng $b$. (h) Tập hợp các số tự nhiên chẵn lớn hơn hoặc bằng $a$ \& nhỏ hơn hoặc bằng $b$. (i) Tập hợp các số tự nhiên lẻ lớn hơn $a$ \& nhỏ hơn $b$. (j) Tập hợp các số tự nhiên lẻ lớn hơn hoặc bằng $a$ \& nhỏ hơn $b$. (k) Tập hợp các số tự nhiên lẻ lớn hơn $a$ \& nhỏ hơn hoặc bằng $b$. (l) Tập hợp các số tự nhiên lẻ lớn hơn hoặc bằng $a$ \& nhỏ hơn hoặc bằng $b$.
\end{baitoan}

\begin{baitoan}[Divisibility vs. Indivisibility (division with remainders) -- Chia hết vs. chia không hết{\tt/}có dư]
	Với $b,r,m,n\in\mathbb{N}$ cho trước (i.e., số tự nhiên khác $0$), viết tập hợp theo 2 cách: (a) Tập hợp các số tự nhiên chia hết cho $b$. (b) Tập hợp các số tự nhiên chia hết cho $b$ \& nhỏ hơn{\tt/}lớn hơn{\tt/}nhỏ hơn hoặc bằng{\tt/}lớn hơn hoặc bằng $m$. (c) Tập hợp các số tự nhiên chia cho $b$ dư $r$. (d) Tập hợp các số tự nhiên chia cho $b$ dư $r$ \& lớn hơn{\tt/}lớn hơn hoặc bằng $m$ \& nhỏ hơn{\tt/}nhỏ hơn hoặc bằng $n$.
\end{baitoan}

\begin{baitoan}[$n$-digit natural number -- Số tự nhiên có $n$ chữ số]
	Viết tập hợp theo nhiều cách nhất có thể: (a) Tập hợp các số tự nhiên có 1 chữ số. (b) Tập hợp các số tự nhiên có 2 chữ số. (c) Tập hợp các số tự nhiên có 3 chữ số. (d) Tập hợp các số tự nhiên có $n$ chữ số, với $n$ là 1 số tự nhiên cho trước.
\end{baitoan}

\begin{baitoan}[Biểu diễn thập phân $n$-digit natural number -- Số tự nhiên có $n$ chữ số]
	Viết biễu diễn thập phân của các số tự nhiên có: (a) 1 chữ số. (b) 2 chữ số. (c) 3 chữ số. (d) 4 chữ số. (e) 5 chữ số. (f) 6 chữ số. (g) 7 chữ số. (h) 8 chữ số. (i) 9 chữ số. (j) 10 chữ số. (k) $n$ chữ số, với $n\in\mathbb{N}$ cho trước.
\end{baitoan}

\begin{proof}[Giải]
	(a) $a\in\{0,1,2,\ldots,9\}$. (b) $\overline{ab} = 10a + b$, $\forall a,b\in\{0,1,2,\ldots,9\}$, $a\ne0$. (c) $\overline{abc} = 100a + 10b + c = 10^2a + 10b + c$, $\forall a,b,c\in\{0,1,2,\ldots,9\}$, $a\ne0$. (d) (Nghìn) $\overline{abcd} = 1000a + 100b + 10c + d = 10^3a + 10^2b + 10c + d$, $\forall a,b,c,d\in\{0,1,2,\ldots,9\}$, $a\ne0$. (e) (Vạn) $\overline{abcde} = 10000a + 1000b + 100c + 10d + e = 10^4a + 10^3b + 10^2c + 10d + e$, $\forall a,b,c,d,e\in\{0,1,2,\ldots,9\}$, $a\ne0$. (f) $\overline{abcdef} = 100000a + 10000b + 1000c + 100d + 10e + f = 10^5a + 10^4b + 10^3c + 10^2d + 10e + f$, $\forall a,b,c,d,e,f\in\{0,1,2,\ldots,9\}$, $a\ne0$. (g) (Triệu) $\overline{abcdefg} = 1000000a + 100000b + 10000c + 1000d + 100e + 10f + g = 10^6a + 10^5b + 10^4c + 10^3d + 10^2e + 10f + g$, $\forall a,b,c,d,e,f,g\in\{0,1,2,\ldots,9\}$, $a\ne0$. (h) $\overline{abcdefgh} = 10000000a + 1000000b + 100000c + 10000d + 1000e + 100f + 10g + h = 10^7a + 10^6b + 10^5c + 10^4d + 10^3e + 10^2f + 10g + h$, $\forall a,b,c,d,e,f,g,h\in\{0,1,2,\ldots,9\}$, $a\ne0$. (i) $\overline{abcdefghi} = 100000000a + 10000000b + 1000000c + 100000d + 10000e + 1000f + 100g + 10h + i = 10^8a + 10^7b + 10^6c + 10^5d + 10^4e + 10^3f + 10^2g + 10h + i$, $\forall a,b,c,d,e,f,g,h,i\in\{0,1,2,\ldots,9\}$, $a\ne0$. (j) (Tỷ) $\overline{abcdefghij} = 1000000000a + 100000000b + 10000000c + 1000000d + 100000e + 10000f + 1000g + 100h + 10i + j = 10^9a + 10^8b + 10^7c + 10^6d + 10^5e + 10^4f + 10^3g + 10^2h + 10i + j$, $\forall a,b,c,d,e,f,g,h,i,j\in\{0,1,2,\ldots,9\}$, $a\ne0$. (k) Tổng quát $\overline{a_na_{n-1}\ldots a_1a_0} = 10^na_n + 10^{n-1}a_{n-1} + \cdots + 10^2a_2 + 10a_1 + a_0$, $\forall n\in\mathbb{N}^\star$, $\forall a_i\in\{0,1,2,\ldots,9\}$, $\forall i = 0,1,\ldots,n$, $a_n\ne0$.
\end{proof}
Dạng biểu diễn thập phân tổng quát của số tự nhiên $x\in\mathbb{N}$ bất kỳ:
\begin{align*}
	\boxed{x = \overline{a_na_{n-1}\ldots a_1a_0} = \sum_{i=0}^n 10^ia_i = 10^na_n + 10^{n-1}a_{n-1} + \cdots + 10^2a_2 + 10a_1 + a_0},\\\forall n\in\mathbb{N},\,\forall a_i\in\{0,1,2,\ldots,9\},\,\forall i = 0,1,\ldots,n,\,a_n\ne0.
\end{align*}
Chỉ cần thêm dấu $\pm$, ta suy ra dạng biểu diễn thập phân tổng quát của số nguyên $x\in\mathbb{Z}$ bất kỳ:
\begin{align*}
	\boxed{x = {\rm sgn}(x)\overline{a_na_{n-1}\ldots a_1a_0} = {\rm sgn}(x)\sum_{i=0}^n 10^ia_i = {\rm sgn}(x)\left(10^na_n + 10^{n-1}a_{n-1} + \cdots + 10^2a_2 + 10a_1 + a_0\right)},\\\forall n\in\mathbb{N},\,\forall a_i\in\{0,1,2,\ldots,9\},\,\forall i = 0,1,\ldots,n,\,a_n\ne0,
\end{align*}
trong đó ${\rm sgn}(x)$ là \textit{hàm dấu} (sign) của 1 số, được định nghĩa bởi công thức sau:
\begin{equation*}
	{\rm sgn}(x) = \left\{\begin{split}
		-&1,&&\mbox{nếu } x < 0,\\
		&0,&&\mbox{nếu } x = 0,\\
		&1,&&\mbox{nếu } x > 0.
	\end{split}\right.
\end{equation*}

\begin{baitoan}
	Chứng minh: (a) Trong 2 số tự nhiên có số chữ số khác nhau: Số nào có nhiều chữ số hơn thì lớn hơn, số nào có ít chữ số hơn thì nhỏ hơn. (b) Trong 2 số tự nhiên có cùng số chữ số, nếu trong cặp chữ số khác nhau đầu tiên từ trái sang phải, số nào có chữ số tương ứng trong cặp đó lớn hơn thì lớn hơn.
\end{baitoan}

\begin{proof}[Giải]
	
\end{proof}

\begin{baitoan}[$\mathbb{N}^\star\subset\mathbb{N}\subset\mathbb{Z}\subset\mathbb{Q}\subset\mathbb{R}\subset\mathbb{C}$]
	Viết tập hợp theo nhiều cách nhất có thể: (a) Tập hợp các số tự nhiên. (b) Tập hợp các số tự nhiên khác $0$. (c) Tập hợp các số nguyên{\tt/}nguyên dương{\tt/}nguyên âm{\tt/}nguyên không âm{\tt/}nguyên không dương. (d) Tập hợp các phân số{\tt/}phân số dương{\tt/}phân số âm{\tt/}phân số không âm{\tt/}phân số không dương. (e) Tập hợp các số thập phân{\tt/}số thập phân dương{\tt/}số thập phân âm{\tt/}số thập phân không âm{\tt/}số thập phân không dương. (f) Tập hợp các phân số thập phân{\tt/}phân số thập phân dương{\tt/}phân số thập phân âm{\tt/}phân số thập phân không âm{\tt/}phân số thập phân không dương. (g) Tập hợp các số hữu tỷ{\tt/}số hữu tỷ dương{\tt/}số hữu tỷ âm{\tt/}số hữu tỷ không âm{\tt/}số hữu tỷ không dương. (h) Tập hợp các số thập phân hữu hạn{\tt/}số thập phân hữu hạn dương{\tt/}số thập phân hữu hạn âm{\tt/}số thập phân hữu hạn không âm{\tt/}số thập phân hữu hạn không dương. (i) Tập hợp các số thập phân vô hạn tuần hoàn{\tt/}số thập phân vô hạn tuần hoàn dương{\tt/}số thập phân vô hạn tuần hoàn âm{\tt/}số thập phân vô hạn tuần hoàn không âm{\tt/}số thập phân vô hạn tuần hoàn không dương. (j) Tập hợp các số thập phân vô hạn không tuần hoàn{\tt/}số thập phân vô hạn không tuần hoàn dương{\tt/}số thập phân vô hạn không tuần hoàn âm{\tt/}số thập phân vô hạn không tuần hoàn không âm{\tt/}số thập phân vô hạn không tuần hoàn không dương.  (k) Tập hợp các số thực{\tt/}số thực dương{\tt/}số thực âm{\tt/}số thực không âm{\tt/}số thực không dương. (l) Tập hợp các số vô tỷ{\tt/}số vô tỷ dương{\tt/}số vô tỷ âm{\tt/}số vô tỷ không dương{\tt/}số vô tỷ không âm. (m) Tập hợp các số phức{\tt/}số thuần thực{\tt/}số thuần ảo.
\end{baitoan}

%------------------------------------------------------------------------------%

\printbibliography[heading=bibintoc]

\end{document}