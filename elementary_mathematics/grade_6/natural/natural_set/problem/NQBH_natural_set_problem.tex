\documentclass{article}
\usepackage[backend=biber,natbib=true,style=alphabetic,maxbibnames=50]{biblatex}
\addbibresource{/home/nqbh/reference/bib.bib}
\usepackage[utf8]{vietnam}
\usepackage{tocloft}
\renewcommand{\cftsecleader}{\cftdotfill{\cftdotsep}}
\usepackage[colorlinks=true,linkcolor=blue,urlcolor=red,citecolor=magenta]{hyperref}
\usepackage{amsmath,amssymb,amsthm,float,graphicx,mathtools}
\allowdisplaybreaks
\newtheorem{assumption}{Assumption}
\newtheorem{baitoan}{}
\newtheorem{cauhoi}{Câu hỏi}
\newtheorem{conjecture}{Conjecture}
\newtheorem{corollary}{Corollary}
\newtheorem{dangtoan}{Dạng toán}
\newtheorem{definition}{Definition}
\newtheorem{dinhly}{Định lý}
\newtheorem{dinhnghia}{Định nghĩa}
\newtheorem{example}{Example}
\newtheorem{ghichu}{Ghi chú}
\newtheorem{hequa}{Hệ quả}
\newtheorem{hypothesis}{Hypothesis}
\newtheorem{lemma}{Lemma}
\newtheorem{luuy}{Lưu ý}
\newtheorem{nhanxet}{Nhận xét}
\newtheorem{notation}{Notation}
\newtheorem{note}{Note}
\newtheorem{principle}{Principle}
\newtheorem{problem}{Problem}
\newtheorem{proposition}{Proposition}
\newtheorem{question}{Question}
\newtheorem{remark}{Remark}
\newtheorem{theorem}{Theorem}
\newtheorem{vidu}{Ví dụ}
\usepackage[left=1cm,right=1cm,top=5mm,bottom=5mm,footskip=4mm]{geometry}
\def\labelitemii{$\circ$}
\DeclareRobustCommand{\divby}{%
	\mathrel{\vbox{\baselineskip.65ex\lineskiplimit0pt\hbox{.}\hbox{.}\hbox{.}}}%
}

\title{Problem: Set -- Bài Tập: Tập Hợp}
\author{Nguyễn Quản Bá Hồng\footnote{Independent Researcher, Ben Tre City, Vietnam\\e-mail: \texttt{nguyenquanbahong@gmail.com}; website: \url{https://nqbh.github.io}.}}
\date{\today}

\begin{document}
\maketitle
\tableofcontents

%------------------------------------------------------------------------------%

\section{Set -- Tập Hợp}

\begin{baitoan}[\cite{Tuyen_Toan_6}, Ví dụ 1, p. 4]
	Cho 2 tập hợp: $A = \{6;7;8;9;10\}$, $B = \{x;9;7;10;y\}$. (a) Viết tập hợp $A$ bằng cách chỉ ra tính chất đặc trưng cho các phần tử của nó. (b) Điền $\in,\notin$: $9\square A$, $x\square A$, $y\square B$. (c) Tìm $x,y$ để $A = B$.
\end{baitoan}

\begin{baitoan}[\cite{Tuyen_Toan_6}, 2., p. 5]
	(a) Viết tập hợp $M$ các chữ cái của chữ ``NGANG''. (b) Với tất cả các phần tử của tập hợp $M$, viết thành 1 chữ thuộc loại danh từ (không sử dụng thêm dấu).
\end{baitoan}

\begin{baitoan}[\cite{Tuyen_Toan_6}, 4., p. 5]
	Cho tập hợp $A = \{a;b\}$, $B = \{1;2;3\}$. Viết tất cả các tập hợp có $3$ phần tử trong đó $1$ phần tử thuộc tập hợp $A$, $2$ phần tử thuộc tập hợp $B$.
\end{baitoan}

\begin{baitoan}[\cite{Tuyen_Toan_6}, 5., p. 5]
	Cho các tập hợp: (a) $P$ là tập hợp các số tự nhiên $x$ mà $x + 3\le10$, $Q$ là tập hợp các số tự nhiên $x$ mà $x\cdot3 = 5$, $R$ là tập hợp các số tự nhiên $x$ mà $x\cdot3 = 0$, $S$ là tập hợp các số tự nhiên $x$ mà $x\cdot3\le24$. (a) Tập hợp nào là tập hợp rỗng? (b) Tập hợp nào có đúng 1 phần tử? (c) 2 tập hợp nào bằng nhau?
\end{baitoan}

\begin{baitoan}
	Viết tập hợp: (a) Tập các màu sắc của cầu vồng. (b) Tập hợp các huyện của tỉnh Bến Tre. (c) Tập hợp các châu lục trên Trái Đất. (d) Tập hợp các hành tinh trong Hệ Mặt Trời.
\end{baitoan}

\begin{baitoan}[$\mathbb{N}^\star\subset\mathbb{N}\subset\mathbb{Z}\subset\mathbb{Q}\subset\mathbb{R}\subset\mathbb{C}$]
	Viết tập hợp theo nhiều cách nhất có thể: (a) Tập hợp các số tự nhiên. (b) Tập hợp các số nguyên dương, i.e., tập hợp các số tự nhiên khác $0$. (c) Tập hợp các số nguyên. (d) Tập hợp các số nguyên âm. (e) Tập hợp các số nguyên không âm. (f) Tập hợp các số nguyên không dương. (g) Tập hợp các phân số, i.e., tập hợp các số hữu tỷ. (h) Tập hợp các số hữu tỷ dương. (i) Tập hợp các số hữu tỷ âm. (j) Tập hợp các số thực. (k) Tập hợp các số thực âm. (l) Tập hợp các số thực dương. (m) Tập hợp các số thực không âm. (n) Tập hợp các thực không dương. (o) Tập hợp các số vô tỷ. (p) Tập hợp các số vô tỷ dương. (q) Tập hợp các số vô tỷ âm. (r) Tập hợp các số phức.
\end{baitoan}

%------------------------------------------------------------------------------%

\section{Set $\mathbb{N}$ of Natural Numbers -- Tập hợp $\mathbb{N}$ Các Số Tự Nhiên}

\begin{baitoan}[\cite{Binh_boi_duong_Toan_6_tap_1}, Ví dụ 1, p. 8]
	Nhiệt độ thay đổi theo giờ trong 1 ngày tháng 8 \& 1 ngày tháng 9 lần lượt được ghi lại:
	\begin{table}[H]
		\centering
		\begin{tabular}{|c|c|c|c|c|c|}
			\hline
			Thời điểm trong ngày & 6:00 & 9:00 & 12:00 & 14:00 & 17:00 \\
			\hline
			Nhiệt độ (${}^\circ$) & 24 & 25 & 27 & 26 & 24 \\
			\hline
		\end{tabular}
	\end{table}
	\begin{table}[H]
		\centering
		\begin{tabular}{|c|c|c|c|c|c|}
			\hline
			Thời điểm trong ngày & 6:00 & 9:00 & 12:00 & 14:00 & 17:00 \\
			\hline
			Nhiệt độ (${}^\circ$) & 23 & 24 & 26 & 25 & 22 \\
			\hline
		\end{tabular}
	\end{table}
	\noindent(a) Viết 2 tập hợp A, B gồm các giá trị nhiệt độ của mỗi bảng trên. (b) Viết tập hợp C gồm các phần tử thuộc tập hợp A mà không thuộc tập hợp B. (c) Viết tập hợp D gồm các phần tử thuộc tập hợp B mà không thuộc tập hợp A. (d) Viết tập hợp E gồm các phần tử thuộc cả 2 tập hợp A \& B. (e) Viết tập hợp F gồm các phần tử thuộc tập hợp A hoặc thuộc tập hợp B.
\end{baitoan}

\begin{baitoan}[\cite{Binh_boi_duong_Toan_6_tap_1}, Ví dụ 2, p. 9]
	Cho A là tập hợp các số tự nhiên chẵn có 3 chữ số. Hỏi A có bao nhiêu phần tử?
\end{baitoan}

\begin{baitoan}[\cite{Binh_boi_duong_Toan_6_tap_1}, Ví dụ 3, p. 9]
	Cho A là tập hợp các số tự nhiên lẻ lớn hơn $3$ \& không lớn hơn $99$. (a) Viết tập hợp A bằng cách chỉ ra tính chất đặc trưng của các phần tử. (b) Giả sử các phần tử của A được viết theo giá trị tăng dần. Tìm phần tử thứ $23$ của A.
\end{baitoan}

\begin{baitoan}[\cite{Binh_boi_duong_Toan_6_tap_1}, Ví dụ 4, p. 10]
	Để đánh số các trang sách (bắt đầu từ trang $1$) của 1 cuốn sách của $2015$ trang thì cần dùng bao nhiêu chữ số?
\end{baitoan}

\begin{baitoan}[\cite{Binh_boi_duong_Toan_6_tap_1}, Ví dụ 5, p. 10]
	Biết người ta đã dùng đúng $6793$ chữ số để đánh số trang của 1 cuốn sách (bắt đầu từ trang $1$), hỏi cuốn sách đó có bao nhiêu trang?
\end{baitoan}

\begin{baitoan}[\cite{Binh_boi_duong_Toan_6_tap_1}, Ví dụ 6, p. 11]
	Gọi A là tập hợp các số tự nhiên có 2 chữ số mà chữ số hàng chục lớn hơn chữ số hàng đơn vị; B là tập hợp các số tự nhiên có 2 chữ số mà chữ số hàng chục nhỏ hơn chữ số hàng đơn vị. So sánh số phần tử của 2 tập hợp A, B.
\end{baitoan}

\begin{baitoan}[\cite{Binh_boi_duong_Toan_6_tap_1}, Ví dụ 7, p. 12]
	Tìm 1 số có 2 chữ số biết khi viết thêm chữ số $0$ vào giữa 2 chữ số của số đó thì được số mới gấp $6$ lần số đã cho.
\end{baitoan}

\begin{baitoan}[\cite{Binh_boi_duong_Toan_6_tap_1}, Ví dụ 8, p. 12]
	Tìm số có 3 chữ số biết nếu viết thêm chữ số $1$ vào trước số đó thì được số mới gấp $9$ lần số ban đầu.	
\end{baitoan}

\begin{baitoan}[\cite{Binh_boi_duong_Toan_6_tap_1}, p. 13]
	Ngày mùng 2 tháng 9 năm $\overline{abcd}$, tại quảng trường Ba Đình lịch sử, Chủ tịch Hồ Chí Minh đã đọc bản tuyên ngôn độc lập khai sinh nước Việt Nam Dân chủ Cộng hòa (nay là nước Cộng hòa xã hội chủ nghĩa Việt Nam). Năm $\overline{abcd}$ là năm nào? Biết $a$ là phần tử nhỏ nhất trong tập hợp $ \mathbb{N}^\star$, $b$ là chữ số lớn nhất, $c,d$ là 2 số tự nhiên liên tiếp \& $c + d = b$.
\end{baitoan}

\begin{baitoan}[\cite{Binh_boi_duong_Toan_6_tap_1}, 1.1., p. 13]
	Cho tập hợp $A = \{1,3,5\}$. {\rm Đ{\tt/}S?} (a) $1\in A$. (b) $\{1\}\in A$. (c) $3\notin A$. (d) $5\notin A$.
\end{baitoan}

\begin{baitoan}[\cite{Binh_boi_duong_Toan_6_tap_1}, 1.2., p. 13]
	Cho 2 tập hợp: $A = \{1,2,3,4,5\}$, $B = \{3,5,7,9\}$. (a) Mỗi tập hợp trên có bao nhiêu phần tử? (b) Viết các tập hợp trên bằng cách chỉ ra tính chất đặc trưng của các phần tử.
\end{baitoan}

\begin{baitoan}[\cite{Binh_boi_duong_Toan_6_tap_1}, 1.3., p. 13]
	Viết các tập hợp sau \& cho biết mỗi tập hợp đó có bao nhiêu phần tử. (a) Tập hợp A các số tự nhiên $x$ thỏa $12 - x = 5$. (b) Tập hợp B các số tự nhiên $y$ thỏa $7 + y = 21$. (c) Tập hợp C các số tự nhiên $z$ mà $z\cdot0 = 0$.
\end{baitoan}

\begin{baitoan}[\cite{Binh_boi_duong_Toan_6_tap_1}, 1.4., p. 14]
	Tính số phần tử của tập hợp: (a) $A = \{2,4,6,\ldots,98\}$. (b) $B = \{6,10,14,18,22,\ldots,70\}$.
\end{baitoan}

\begin{baitoan}[\cite{Binh_boi_duong_Toan_6_tap_1}, 1.5., p. 14]
	Cho dãy số $2,5,8,11,14,\ldots$. (a) Nêu quy luật của dãy số trên. (b) Viết tập hợp B gồm $5$ số hạng tiếp theo của dãy số trên. (c) Tính tổng $100$ số hạng đầu tiên của dãy số.
\end{baitoan}

\begin{baitoan}[\cite{Binh_boi_duong_Toan_6_tap_1}, 1.6., p. 14]
	Viết lại mỗi tập hợp bằng cách liệt kê các phần tử: $A = \{x\in\mathbb{N}|{x\not{\divby}\ 2},\,30 < x < 50\}$. (b) $B = \{x\in\mathbb{N}|x\divby5,\,x\divby2,\,10 < x\le90\}$.
\end{baitoan}

\begin{baitoan}[\cite{Binh_boi_duong_Toan_6_tap_1}, 1.7., p. 13]
	Thực hiện yêu cầu phòng chống dịch Covid-19, tại 1 trường trung học, vào đầu giờ sáng trước khi vào lớp, các học sinh đều được yêu cầu khử khuẩn tay \& đo thân nhiệt. Kết quả đo thân nhiệt tại lớp 6H:
	\begin{table}[H]
		\centering
		\begin{tabular}{|c|l|}
			\hline
			Tổ & Thân nhiệt (${}^\circ$C) \\
			\hline
			1 & 36, 36.5, 37, 36, 35.5, 37, 36.5, 36, 35.5, 36, 36.5, 37 \\
			\hline
			2 & 36.5, 37, 35.5, 36, 36, 35.5, 37, 36, 36.5, 36, 35.5, 36 \\
			\hline
			3 & 37, 36.5, 36, 35.5, 35, 36, 35.5, 35, 36, 35, 36.5, 36 \\
			\hline
			4 & 36, 35.5, 36, 36, 35.5, 36.5, 35, 36.5, 36, 35.5, 36.5, 36 \\
			\hline
		\end{tabular}
	\end{table}
	\noindent(a) Gọi A, B, C, D lần lượt là tập hợp gồm các phần tử là thân nhiệt của các bạn tổ 1, tổ 2, tổ 3, tổ 4. Viết các tập hợp A, B, C, D theo cách liệt kê các phần tử (mỗi phần tử chỉ liệt kê 1 lần). (b) Tìm cặp tập hợp bằng nhau trong các tập hợp A, B, C, D. Dùng ký hiệu ``$=$'' để thể hiện mối quan hệ đó.
\end{baitoan}

\begin{baitoan}[\cite{Binh_boi_duong_Toan_6_tap_1}, 1.8., p. 14]
	Với 3 chữ số La Mã I, V, X có thể viết được bao nhiêu số La Mã mà mỗi chữ số chỉ xuất hiện 1 lần? Số nhỏ nhất là số nào? Số lớn nhất là số nào?
\end{baitoan}

\begin{baitoan}[\cite{Binh_boi_duong_Toan_6_tap_1}, 1.9., p. 14]
	Tìm số có 3 chữ số, biết nếu viết các chữ số theo thứ tự ngược lại thì được số mới nhỏ hơn số ban đầu $792$ đơn vị.
\end{baitoan}

\begin{baitoan}[\cite{Binh_boi_duong_Toan_6_tap_1}, 1.10., p. 14]
	Có bao nhiêu số tự nhiên có 2 chữ số mà: (a) Trong số đó có ít nhất 1 chữ số $9$? (b) Trong số đó chữ số hàng chục bé hơn chữ số hàng đơn vị? (c) Trong số đó chữ số hàng chục gấp đôi chữ số hàng đơn vị?
\end{baitoan}

\begin{baitoan}[\cite{Binh_boi_duong_Toan_6_tap_1}, 1.11., p. 14]
	Cho 1 số có 2 chữ số. Nếu viết thêm chữ số 2 vào bên trái \& bên phải số đó ta được số mới gấp $32$ lần số đã cho. Tìm số đã cho.
\end{baitoan}

\begin{baitoan}[\cite{Binh_boi_duong_Toan_6_tap_1}, 1.12., p. 14]
	Mẹ mua cho Hà 1 quyển sổ tay có $358$ trang. Để tiện theo dõi, Hà đánh số trang từ $1$ đến $358$. Hỏi Hà đã phải viết bao nhiêu chữ số để đánh số trang hết cuốn sổ tay đó.
\end{baitoan}

\begin{baitoan}[\cite{Binh_boi_duong_Toan_6_tap_1}, 1.13., p. 14]
	Viết liền nhau các số tự nhiên $123456789101112\ldots$ (a) Hỏi các chữ số hàng đơn vị của các số $49,217,2401$ đứng ở vị trí thứ bao nhiêu kể từ trái sang phải? (b) Chữ số viết ở vị trí thứ $427$ là chữ số nào?
\end{baitoan}

\begin{baitoan}[\cite{Binh_boi_duong_Toan_6_tap_1}, 1.14., p. 15]
	Cho 4 chữ số $a,b,c,d$ đôi một khác nhau \& khác $0$. Tập hợp các số tự nhiên có 3 chữ số gồm 3 trong 4 chữ số $a,b,c,d$ có bao nhiêu phần tử?
\end{baitoan}

\begin{baitoan}[\cite{Binh_boi_duong_Toan_6_tap_1}, 1.15., p. 15]
	Mỗi tập hợp sau đây có bao nhiêu phần tử? (a) Tập hợp các số có 2 chữ số. (b) Tập hợp các số có 2 chữ số được lập nên từ 2 số khác nhau. (c) Tập hợp các số có 3 chữ số được lập nên từ 3 chữ số đôi một khác nhau.
\end{baitoan}

\begin{baitoan}[\cite{Binh_boi_duong_Toan_6_tap_1}, 1.16., p. 15]
	Tổng kết đợt thi đua, lớp 6A có $35$ bạn được 1 điểm $10$ trở lên, $21$ bạn được từ 2 điểm $10$ trở lên, $18$ bạn được 3 điểm $10$ trở lên, $5$ bạn được $4$ điểm $10$. Biết không có ai được trên $4$ điểm $10$, hỏi trong đợt thi đua đó lớp 6A có bao nhiêu điểm $10$?
\end{baitoan}

\begin{baitoan}[\cite{Binh_boi_duong_Toan_6_tap_1}, 1.17., p. 15]
	Tìm số tự nhiên có 4 chữ số, chữ số hàng đơn vị là $9$. Nếu chuyển chữ số hàng đơn vị lên đầu thì được 1 số mới lớn hơn số đã cho $2889$ đơn vị.
\end{baitoan}

\begin{baitoan}[\cite{Binh_boi_duong_Toan_6_tap_1}, 1.18., p. 15]
	Hiệu của 2 số tự nhiên là $53$. Chữ số hàng đơn vị của số bị trừ là $8$. Nếu bỏ chữ số hàng đơn vị của số bị trừ ta được số trừ. Tìm 2 số đó.
\end{baitoan}

\begin{baitoan}[\cite{Binh_boi_duong_Toan_6_tap_1}, 1.19., p. 15]
	Tìm 1 số có 5 chữ số biết nếu viết chữ số $7$ đằng trước số đó thì được số lớn gấp $5$ lần số có được bằng cách viết thêm chữ số $7$ vào đằng sau số đó.
\end{baitoan}

\begin{baitoan}[\cite{Binh_boi_duong_Toan_6_tap_1}, 1.20., p. 15]
	1 số gồm 3 chữ số tận cùng là chữ số $9$, nếu chuyển chữ số $9$ đó lên đầu thì được 1 số mới mà khi chia cho số cũ thì được thương là $3$ dư $61$. Tìm số đó.
\end{baitoan}

\begin{baitoan}[\cite{Binh_boi_duong_Toan_6_tap_1}, p. 15]
	Trong 1 lớp học, tất cả học sinh nam đều tham gia vào các câu lạc bộ thể thao: bóng đá, bóng chuyền, cầu lông. Biết có $7$ học sinh tham gia câu lạc bộ bóng đá, $6$ học sinh tham gia câu lạc bộ bóng chuyền, $5$ học sinh tham gia câu lạc bộ cầu lông; trong số đó có $4$ học sinh tham gia cả 2 câu lạc bộ bóng đá \& bóng chuyền, $3$ học sinh tham gia cả 2 câu lạc bộ bóng đá \& cầu lông, $2$ học sinh tham gia cả 2 câu lạc bộ bóng chuyền \& cầu lông, $1$ học sinh tham gia cả 3 câu lạc bộ. Hỏi lớp đó có bao nhiêu học sinh nam?
\end{baitoan}

\begin{baitoan}[\cite{Tuyen_Toan_6}, Ví dụ 2, p. 6]
	Phố Hàng Ngang là 1 trong các phố cổ của Hà Nội. Các nhà được đánh số liên tục, dãy lẻ $1,3,5,7,\ldots,61$; dãy chẵn $2,4,6,\ldots,64$. (a) Bên số nhà chẵn, trong 1 phòng gác nhỏ, chủ tịch Hồ Chí Minh đã khởi thảo bản Tuyên Ngôn Độc Lập khai sinh cho nước Việt Nam Dân Chủ Cộng Hòa. Ngôi nhà có căn phòng đó là nhà thứ $24$ kể từ đầu phố (số $2$). Hỏi ngôi nhà này có số nào? (b) Bên số nhà lẻ chữ số nào được dùng nhiều nhất? Chữ số nào chưa được dùng đến? (c) Phải dùng tất cả bao nhiêu chữ số để ghi số nhà của phố này?
\end{baitoan}

\begin{baitoan}[\cite{Tuyen_Toan_6}, 6., p. 6]
	Viết tập hợp $4$ số tự nhiên liên tiếp lớn hơn $94$ nhưng không quá $100$.
\end{baitoan}

\begin{baitoan}[\cite{Tuyen_Toan_6}, 7., p. 6]
	(a) Có bao nhiêu số tự nhiên nhỏ hơn $20$? (b) Có bao nhiêu số tự nhiên nhỏ hơn $n\in\mathbb{N}$? (c) Có bao nhiêu số tự nhiên chẵn nhỏ hơn $n\in\mathbb{N}$? (d) Có bao nhiêu số tự nhiên lẻ nhỏ hơn $n\in\mathbb{N}$?
\end{baitoan}

\begin{baitoan}[\cite{Tuyen_Toan_6}, 8., p. 7]
	(a) Có bao nhiêu số có $4$ chữ số mà cả $4$ chữ số đều giống nhau? (b) Có bao nhiêu số có $4$ chữ số? (c) Có bao nhiêu số có $n$ chữ số, với $n\in\mathbb{N}$?
\end{baitoan}

\begin{baitoan}[\cite{Tuyen_Toan_6}, 9., p. 7]
	Đèn hướng dẫn giao thông liên tục sáng màu xanh hoặc đỏ kế tiếp nhau. Bảng hiện số của đèn có 2 chữ số liên tục thay đổi theo từng giây. Hỏi trong 1 phút xe bị dừng vì đèn đỏ thì đèn có: (a) Bao nhiêu lần thay đổi các số? (b) Bao nhiêu lần thay đổi các chữ số?
\end{baitoan}

\begin{baitoan}[\cite{Tuyen_Toan_6}, 10., p. 7]
	Tìm $3$ số tự nhiên $a,b,c$ biết chúng thỏa mãn đồng thời 3 điều kiện: $a < b < c$, $101\le a\le103$, $101 < c < 104$.
\end{baitoan}

\begin{baitoan}[\cite{Tuyen_Toan_6}, 11., p. 7]
	Cho số $4321$. Viết thêm chữ số $9$ xen giữa các chữ số của nó để được 1 số: (a) Lớn nhất có thể được. (b) Nhỏ nhất có thể được.
\end{baitoan}

\begin{baitoan}[\cite{Tuyen_Toan_6}, 12., p. 7]
	Với $9$ que diêm, sắp xếp thành 1 số La Mã: (a) Có giá trị lớn nhất. (b) Có giá trị nhỏ nhất.
\end{baitoan}

\begin{baitoan}[\cite{Tuyen_Toan_6}, 13., p. 7]
	Có $13$ que diêm sắp xếp như sau: $\rm XII - V = VII$. (a) Đẳng thức trên đúng hay sai? (b) Đổi chỗ chỉ 1 que diêm để được 1 đẳng thức đúng.
\end{baitoan}

\begin{baitoan}[\cite{Binh_Toan_6_tap_1}, Ví dụ 1, p. 4]
	Viết các tập hợp sau rồi tìm số phần tử của mỗi tập hợp đó: (a) Tập hợp $A$ các số tự nhiên $x$ mà $8:x = 2$. (b) Tập hợp $B$ các số tự nhiên $x$ mà $x + 3 < 5$. (c) Tập hợp $C$ các số tự nhiên $x$ mà $x - 2 = x + 2$. (d) Tập hợp $D$ các số tự nhiên $x$ mà $x:2 = x:4$. (e) Tập hợp $E$ các số tự nhiên $x$ mà $x + 0 = x$.
\end{baitoan}

\begin{baitoan}[\cite{Binh_Toan_6_tap_1}, Ví dụ 2, p. 5]
	Viết các tập hợp sau bằng cách liệt kê các phần tử của nó: (a) Tập hợp $A$ các số tự nhiên có 2 chữ số, trong đó chữ số hàng chục lớn hơn chữ số hàng đơn vị là $2$. (b) Tập hợp $B$ các số tự nhiên có 3 chữ số mà tổng các chữ số bằng $3$.
\end{baitoan}

\begin{baitoan}[\cite{Binh_Toan_6_tap_1}, Ví dụ 3, p. 5]
	Tìm số tự nhiên có 5 chữ số, biết nếu viết thêm chữ số $2$ vào đằng sau số đó thì được số lớn gấp 3 lần số có được bằng cách viết thêm chữ số $2$ vào đằng trước số đó.
\end{baitoan}

\begin{baitoan}[\cite{Binh_Toan_6_tap_1}, Mở rộng Ví dụ 3, p. 5]
	Tìm số tự nhiên nhỏ nhất có chữ số đầu tiên ở bên trái là $2$, khi chuyển chữ số $2$ này xuống cuối cùng thì số đó tăng gấp 3 lần.\hfill{\sf Ans:} $285714$.
\end{baitoan}

\begin{baitoan}[\cite{Binh_Toan_6_tap_1}, Mở rộng Ví dụ 3, p. 6]
	Tìm số tự nhiên có 5 chữ số, biết nếu viết thêm 1 chữ số vào đằng sau số đó thì được số lớn gấp 3 lần số có được nếu viết thêm chính chữ số ấy vào đằng trước số đó.\hfill{\sf Ans:} $85714$.
\end{baitoan}

\begin{baitoan}[\cite{Binh_Toan_6_tap_1}, \textbf{2.}, p. 6]
	Xác định các tập hợp sau bằng cách chỉ ra tính chất đặc trưng của các phần tử thuộc tập hợp đó: (a) $A = \{1,3,5,7,\ldots,49\}$. (b) $B = \{11,22,33,44,\ldots,99\}$. (c) $C = \{\mbox{tháng } 1,\mbox{tháng } 3,\mbox{tháng } 5,\mbox{tháng } 7,\mbox{tháng } 8,\mbox{tháng } 10,\mbox{tháng } 12\}$.
\end{baitoan}

\begin{baitoan}[\cite{Binh_Toan_6_tap_1}, \textbf{3.}, p. 6]
	Tìm tập hợp các số tự nhiên $x$ sao cho: (a) $x + 3 = 4$. (b) $8 - x = 5$. (c) $x:2 = 0$. (d) $0:x = 0$. (e) $5x = 12$.
\end{baitoan}

\begin{baitoan}[\cite{Binh_Toan_6_tap_1}, \textbf{4.}, p. 6]
	Tìm $a,b\in\mathbb{N}$ sao cho $12 < a < b < 16$.
\end{baitoan}

\begin{baitoan}[\cite{Binh_Toan_6_tap_1}, \textbf{5.}, p. 6]
	Viết các số tự nhiên có 4 chữ số trong đó có 2 chữ số $3$, 1 chữ số $2$, 1 chữ số $1$.
\end{baitoan}

\begin{baitoan}[\cite{Binh_Toan_6_tap_1}, \textbf{6.}, p. 6]
	Với cả 2 chữ số I \& X, viết được bao nhiêu số La Mã? (Mỗi chữ số có thể viết nhiều lần, nhưng không viết liên tiếp quá 3 lần).
\end{baitoan}

\begin{baitoan}[\cite{Binh_Toan_6_tap_1}, \textbf{7.}, pp. 6--7]
	(a) Dùng 3 que diêm, xếp được các số La Mã nào? (b) Để viết các số La Mã từ 4000 trở lên, e.g. số 19520, người ta viết XIXmDXX (chữ m biểu thị \emph{1 nghìn}, m là chữ đầu của từ \emph{mille}, tiếng Latin là 1 nghìn). Hãy viết các số sau bằng chữ số La Mã: 7203, 121512.
\end{baitoan}

\begin{baitoan}[\cite{Binh_Toan_6_tap_1}, \textbf{8.}, p. 7]
	Tìm số tự nhiên có tận cùng bằng $3$, biết rằng nếu xóa chữ số hàng đơn vị thì số đó giảm đi $1992$ đơn vị.
\end{baitoan}

\begin{baitoan}[\cite{Binh_Toan_6_tap_1}, \textbf{9.}, p. 7]
	Tìm số tự nhiên có 6 chữ số, biết rằng chữ số hàng đơn vị là $4$ \& nếu chuyển chữ số đó lên hàng đầu tiên thì số đó tăng gấp 4 lần.
\end{baitoan}

\begin{baitoan}[\cite{Binh_Toan_6_tap_1}, \textbf{10.}, p. 7]
	Cho 4 chữ số $a,b,c,d$ khác nhau \& khác $0$. Lập số tự nhiên lớn nhất \& số tự nhiên nhỏ nhất có 4 chữ số gồm cả 4 chữ số ấy. Tổng của 2 số này bằng $11330$. Tìm tổng các chữ số $a + b + c + d$.
\end{baitoan}

\begin{baitoan}[\cite{Binh_Toan_6_tap_1}, \textbf{11.}, p. 7]
	Cho 3 chữ số $a,b,c$ sao cho $0 < a < b < c$. (a) Viết tập hợp $A$ các số tự nhiên có 3 chữ số gồm cả 3 chữ số $a,b,c$. (b) Biết tổng 2 số nhỏ nhất trong tập hợp $A$ bằng $488$. Tìm 3 chữ số $a,b,c$ nói trên.
\end{baitoan}

\begin{baitoan}[\cite{Binh_Toan_6_tap_1}, \textbf{12.}, p. 7]
	Tìm 3 chữ số khác nhau \& khác $0$, biết rằng nếu dùng cả 3 chữ số này lập thành các số tự nhiên có 3 chữ số thì 2 số lớn nhất có tổng bằng $1444$.
\end{baitoan}

\begin{baitoan}[Even vs. odd -- Chẵn vs. lẻ]
	Viết tập hợp theo nhiều cách nhất có thể: (a) Tập hợp các số tự nhiên chẵn. (b) Tập hợp các số tự nhiên lẻ. (c) Tập hợp các số nguyên dương chẵn. (d) Tập hợp các số nguyên dương lẻ. (e) Tập hợp các số nguyên chẵn. (f) Tập hợp các số nguyên lẻ.
\end{baitoan}

\begin{baitoan}
	Với $b,r$ là 2 số nguyên dương cho trước (i.e., số tự nhiên khác $0$), viết tập hợp theo 2 cách: (a) Tập hợp các số tự nhiên chia hết cho $b$. (b) Tập hợp các số tự nhiên chia cho $b$ dư $r$.
\end{baitoan}

\begin{baitoan}[Tập con của $\mathbb{N}$ chỉ bị chặn 1 phía]
	Cho $a$ là 1 số tự nhiên cho trước. Viết các tập hợp sau theo nhiều cách nhất có thể: (a) Tập hợp các số tự nhiên nhỏ hơn $a$. (b) Tập hợp các số tự nhiên lớn hơn $a$. (c) Tập hợp các số tự nhiên nhỏ hơn hoặc bằng $a$. (d) Tập hợp các số tự nhiên lớn hơn hoặc bằng $a$. (e) Tập hợp các số tự nhiên chẵn nhỏ hơn $a$. (f) Tập hợp các số tự nhiên chẵn lớn hơn $a$. (g) Tập hợp các số tự nhiên chẵn nhỏ hơn hoặc bằng $a$. (h) Tập hợp các số tự nhiên chẵn lớn hơn hoặc bằng $a$. (i) Tập hợp các số tự nhiên lẻ nhỏ hơn $a$. (j) Tập hợp các số tự nhiên lẻ lớn hơn $a$. (k) Tập hợp các số tự nhiên lẻ nhỏ hơn hoặc bằng $a$. (l) Tập hợp các số tự nhiên lẻ lớn hơn hoặc bằng $a$.
\end{baitoan}

\begin{baitoan}[Tập con của $\mathbb{N}$ bị chặn cả 2 phía]
	Với $a,b$ là 2 số tự nhiên cho trước. Viết các tập hợp sau theo nhiều cách nhất có thể: (a) Tập hợp các số tự nhiên lớn hơn $a$ \& nhỏ hơn $b$. (b) Tập hợp các số tự nhiên lớn hơn hoặc bằng $a$ \& nhỏ hơn $b$. (c) Tập hợp các số tự nhiên lớn hơn $a$ \& nhỏ hơn hoặc bằng $b$. (d) Tập hợp các số tự nhiên lớn hơn hoặc bằng $a$ \& nhỏ hơn hoặc bằng $b$. (e) Tập hợp các số tự nhiên chẵn lớn hơn $a$ \& nhỏ hơn $b$. (f) Tập hợp các số tự nhiên chẵn lớn hơn hoặc bằng $a$ \& nhỏ hơn $b$. (g) Tập hợp các số tự nhiên chẵn lớn hơn $a$ \& nhỏ hơn hoặc bằng $b$. (h) Tập hợp các số tự nhiên chẵn lớn hơn hoặc bằng $a$ \& nhỏ hơn hoặc bằng $b$. (i) Tập hợp các số tự nhiên lẻ lớn hơn $a$ \& nhỏ hơn $b$. (j) Tập hợp các số tự nhiên lẻ lớn hơn hoặc bằng $a$ \& nhỏ hơn $b$. (k) Tập hợp các số tự nhiên lẻ lớn hơn $a$ \& nhỏ hơn hoặc bằng $b$. (l) Tập hợp các số tự nhiên lẻ lớn hơn hoặc bằng $a$ \& nhỏ hơn hoặc bằng $b$.
\end{baitoan}

\begin{baitoan}
	Viết tập hợp theo nhiều cách nhất có thể: (a) Tập hợp các số tự nhiên có 1 chữ số. (b) Tập hợp các số tự nhiên có 2 chữ số. (c) Tập hợp các số tự nhiên có 3 chữ số. (d) Tập hợp các số tự nhiên có $n$ chữ số, với $n$ là 1 số tự nhiên cho trước.
\end{baitoan}

\begin{baitoan}
	Viết biễu diễn thập phân của các số tự nhiên có: (a) 1 chữ số. (b) 2 chữ số. (c) 3 chữ số. (d) 4 chữ số. (e) 5 chữ số. (f) 6 chữ số. (g) 7 chữ số. (h) (d) 8 chữ số. (i) 9 chữ số. (j) (d) 10 chữ số. (k) $n$ chữ số, với $n\in\mathbb{N}$ cho trước. 
\end{baitoan}

\begin{baitoan}
	Chứng minh: (a) Trong 2 số tự nhiên có số chữ số khác nhau: Số nào có nhiều chữ số hơn thì lớn hơn, số nào có ít chữ số hơn thì nhỏ hơn. (b) Trong 2 số tự nhiên có cùng số chữ số, nếu trong cặp chữ số khác nhau đầu tiên từ trái sang phải, số nào có chữ số tương ứng trong cặp đó lớn hơn thì lớn hơn.
\end{baitoan}

%------------------------------------------------------------------------------%

\printbibliography[heading=bibintoc]
	
\end{document}