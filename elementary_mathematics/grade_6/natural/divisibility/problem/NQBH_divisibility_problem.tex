\documentclass{article}
\usepackage[backend=biber,natbib=true,style=alphabetic,maxbibnames=50]{biblatex}
\addbibresource{/home/nqbh/reference/bib.bib}
\usepackage[utf8]{vietnam}
\usepackage{tocloft}
\renewcommand{\cftsecleader}{\cftdotfill{\cftdotsep}}
\usepackage[colorlinks=true,linkcolor=blue,urlcolor=red,citecolor=magenta]{hyperref}
\usepackage{amsmath,amssymb,amsthm,float,graphicx,mathtools,tipa}
\allowdisplaybreaks
\newtheorem{assumption}{Assumption}
\newtheorem{baitoan}{}
\newtheorem{cauhoi}{Câu hỏi}
\newtheorem{conjecture}{Conjecture}
\newtheorem{corollary}{Corollary}
\newtheorem{dangtoan}{Dạng toán}
\newtheorem{definition}{Definition}
\newtheorem{dinhly}{Định lý}
\newtheorem{dinhnghia}{Định nghĩa}
\newtheorem{example}{Example}
\newtheorem{ghichu}{Ghi chú}
\newtheorem{hequa}{Hệ quả}
\newtheorem{hypothesis}{Hypothesis}
\newtheorem{lemma}{Lemma}
\newtheorem{luuy}{Lưu ý}
\newtheorem{nhanxet}{Nhận xét}
\newtheorem{notation}{Notation}
\newtheorem{note}{Note}
\newtheorem{principle}{Principle}
\newtheorem{problem}{Problem}
\newtheorem{proposition}{Proposition}
\newtheorem{question}{Question}
\newtheorem{remark}{Remark}
\newtheorem{theorem}{Theorem}
\newtheorem{vidu}{Ví dụ}
\usepackage[left=1cm,right=1cm,top=5mm,bottom=5mm,footskip=4mm]{geometry}
\def\labelitemii{$\circ$}
\DeclareRobustCommand{\divby}{%
	\mathrel{\vbox{\baselineskip.65ex\lineskiplimit0pt\hbox{.}\hbox{.}\hbox{.}}}%
}

\title{Problem: Divisibility -- Bài Tập: Tính Chia Hết}
\author{Nguyễn Quản Bá Hồng\footnote{Independent Researcher, Ben Tre City, Vietnam\\e-mail: \texttt{nguyenquanbahong@gmail.com}; website: \url{https://nqbh.github.io}.}}
\date{\today}

\begin{document}
\maketitle
\begin{abstract}
	Last updated version: \href{https://github.com/NQBH/hobby/blob/master/elementary_mathematics/grade_6/natural/natural_calculus/problem/NQBH_natural_calculus_problem.pdf}{GitHub{\tt/}NQBH{\tt/}hobby{\tt/}elementary mathematics{\tt/}grade 6{\tt/}natural{\tt/}natural calculus{\tt/}problem: calculus on set $\mathbb{N}$ of naturals [pdf]}.\footnote{\textsc{url}: \url{https://github.com/NQBH/hobby/blob/master/elementary_mathematics/grade_6/natural/natural_calculus/problem/NQBH_natural_calculus_problem.pdf}.} [\href{https://github.com/NQBH/hobby/blob/master/elementary_mathematics/grade_6/natural/natural_calculus/problem/NQBH_natural_calculus_problem.tex}{\TeX}]\footnote{\textsc{url}: \url{https://github.com/NQBH/hobby/blob/master/elementary_mathematics/grade_6/natural/natural_calculus/problem/NQBH_natural_calculus_problem.tex}.}. 
\end{abstract}
\tableofcontents

%------------------------------------------------------------------------------%

\begin{itemize}\sf
	\item \textbf{divisible} [a] {\tt/}\textipa{d@'vIz@bl}{\tt/} [not before noun] \textit{divisible (by something)} that can be divided, usually with nothing remaining. {\sc opposite}: \textbf{indivisible}.
	\item \textbf{divisibility}  [n] [uncountable] {\tt/}\textipa{d@'vIz@bIl@ti}{\tt/}.
\end{itemize}

\section{Divisibility of Sum -- Tính Chất Chia Hết của Tổng}

\begin{baitoan}[\cite{Binh_boi_duong_Toan_6_tap_1}, H1, p. 24]
	{\rm Đ{\tt/}S? (a) $127\cdot5 + 40\divby5$. (b) $13\cdot48 + 12 + 17\divby6$. (c) $3\cdot300 - 12\divby9$. (d) $49 + 62\cdot7\divby7$.}
\end{baitoan}

\begin{baitoan}[\cite{Binh_boi_duong_Toan_6_tap_1}, H2, p. 24]
	Khi chia số $a$ cho số $b$, $a,b\in\mathbb{N}^\star$, $a > b$ ta được số dư là $r$. Khi đó: {\sf A.} $a + r\divby b$. {\sf B.} $a - r\divby b$. {\sf C.} $a + b\divby r$. {\sf D.} $a - b\divby r$.
\end{baitoan}

\begin{baitoan}[\cite{Binh_boi_duong_Toan_6_tap_1}, H3, p. 24]
	Tìm số tự nhiên $x$ có 1 chữ số thỏa $121 + x\divby11$.
\end{baitoan}

\begin{baitoan}[\cite{Binh_boi_duong_Toan_6_tap_1}, Ví dụ 1, p. 25]
	Không tính các tổng \& hiệu, xét xem các tổng \& hiệu sau có chia hết cho $12$ không? Vì sao? (a) $600\cdot37 - 144$. (b) $96 + 34 + 48$.
\end{baitoan}

\begin{baitoan}[\cite{Binh_boi_duong_Toan_6_tap_1}, Ví dụ 2, p. 25]
	Không tính ra kết quả, xét xem tổng $84 + 37 + 23$ có chia hết cho $12$ không? Vì sao?
\end{baitoan}

\begin{baitoan}[\cite{Binh_boi_duong_Toan_6_tap_1}, Ví dụ 3, p. 25]
	Chứng minh trong 3 số tự nhiên liên tiếp có 1 số chia hết cho $3$.
\end{baitoan}

\begin{baitoan}[\cite{Binh_boi_duong_Toan_6_tap_1}, Mở rộng Ví dụ 4, p. 25]
	Với $n\in\mathbb{N}^\star$ bất kỳ. Chứng minh: (a) Trong $n$ số tự nhiên liên tiếp luôn có 1 số chia hết cho $n$. (b) Tích của $n$ số tự nhiên liên tiếp là 1 số chia hết cho $n$.
\end{baitoan}

\begin{baitoan}[\cite{Binh_boi_duong_Toan_6_tap_1}, Ví dụ 4, p. 26]
	Chứng minh tổng của 3 số tự nhiên liên tiếp là 1 số chia hết cho $3$.
\end{baitoan}

\begin{baitoan}
	Với $n\in\mathbb{N}^\star$ bất kỳ. Liệu tổng của $n$ số tự nhiên liên tiếp có chia hết cho $n$ không?
\end{baitoan}

\begin{baitoan}[\cite{Binh_boi_duong_Toan_6_tap_1}, Ví dụ 5, p. 26]
	Chứng minh: (a) $\overline{ab} - \overline{ba}\divby9$ với $a > b$. (b) Nếu $\overline{ab} + \overline{cd}\divby11$ thì $\overline{abcd}\divby11$.
\end{baitoan}

\begin{baitoan}[\cite{Binh_boi_duong_Toan_6_tap_1}, Ví dụ 6, p. 26]
	Cho $A = 15 + 30 + 37 + x$ với $x\in\mathbb{N}$. Tìm điều kiện của $x$ để: (a) $A\divby3$. (b) $A\not{\divby}\ 9$.
\end{baitoan}

\begin{baitoan}[\cite{Binh_boi_duong_Toan_6_tap_1}, Ví dụ 7, p. 26]
	Tìm $n\in\mathbb{N}$ để: (a) $n + 4\divby n$. (b) $5n - 6\divby n$ với $n > 1$. (c) $143 - 12n\divby n$ với $n < 12$.
\end{baitoan}

\begin{baitoan}[\cite{Binh_boi_duong_Toan_6_tap_1}, Ví dụ 8, p. 27]
	Tìm $n\in\mathbb{N}$ để: (a) $n + 9\divby n + 4$. (b) $3n + 40\divby n + 4$. (c) $5n + 2\divby2n + 9$.
\end{baitoan}

\begin{baitoan}[\cite{Binh_boi_duong_Toan_6_tap_1}, 3.1., p. 27]
	Cho $A = 2\cdot5\cdot9\cdot13 + 84$. Hỏi $A$ có chia hết cho $3$, cho $6$, cho $9$, cho $13$ không? Vì sao?
\end{baitoan}

\begin{baitoan}[\cite{Binh_boi_duong_Toan_6_tap_1}, 3.2., p. 27]
	Chứng minh tổng 5 số chẵn liên tiếp là 1 số chia hết cho $10$.
\end{baitoan}

\begin{baitoan}[\cite{Binh_boi_duong_Toan_6_tap_1}, 3.3., p. 27]
	Khi chia số tự nhiên $a$ cho $27$, ta được số dư là $15$. Hỏi số $a$ có chia hết cho $3$, cho $9$ không? Vì sao?
\end{baitoan}

\begin{baitoan}[\cite{Binh_boi_duong_Toan_6_tap_1}, 3.4., p. 27]
	Chứng minh mọi số tự nhiên có 3 chữ số giống nhau đều chia hết cho $37$.
\end{baitoan}

\begin{baitoan}[\cite{Binh_boi_duong_Toan_6_tap_1}, 3.5., p. 28]
	Chứng minh: (a) $\sum_{i=0}^{101} 5^i = 1 + 5 + 5^2 + 5^3 + \cdots + 5^{101}\divby6$. (b) $\sum_{i=1}^{100} 2^i = 2 + 2^2 + 2^3 + \cdots + 2^{100}$ vừa chia hết cho $31$, vừa chia hết cho $5$.
\end{baitoan}

\begin{baitoan}[\cite{Binh_boi_duong_Toan_6_tap_1}, 3.6., p. 28]
	Chứng minh: (a) Nếu $\overline{abc} - \overline{def}\divby11$ thì $\overline{abcdef}\divby11$. (b) Nếu $\overline{abc}\divby8$ thì $4a + 2b + c\divby8$.
\end{baitoan}

\begin{baitoan}[\cite{Binh_boi_duong_Toan_6_tap_1}, 3.7., p. 28]
	Tìm chữ số $a$ biết $\overline{21a21a21a}\divby31$.
\end{baitoan}

\begin{baitoan}[\cite{Binh_boi_duong_Toan_6_tap_1}, 3.8., p. 28]
	Tìm $n\in\mathbb{N}$ sao cho: (a) $n + 21\divby n$. (b) $18 - 2n\divby n$ với $n < 9$. (c) $6n - 9\divby n$ với $n\ge2$.
\end{baitoan}

\begin{baitoan}[\cite{Binh_boi_duong_Toan_6_tap_1}, 3.9., p. 28]
	Tìm $n\in\mathbb{N}$ sao cho: (a) $n + 15\divby n - 3$ với $n > 5$. (b) $18 - 2n\divby n + 3$ với $n\le9$. (c) $3n + 13\divby2n + 3$ với $n\ge1$.
\end{baitoan}

\begin{baitoan}[\cite{Binh_boi_duong_Toan_6_tap_1}, 3.10., p. 28]
	Cho $a,b\in\mathbb{N}$. Chứng minh nếu $7a + 2b$ \& $31a + 9b$ cùng chia hết cho $2015$ thì $a,b$ cũng chia hết cho $2015$.
\end{baitoan}

\begin{baitoan}[\cite{Binh_boi_duong_Toan_6_tap_1}, p. 28]
	Chứng minh: (a) Tích 2 số tự nhiên liên tiếp là 1 số chẵn. (b) Tích 3 số tự nhiên liên tiếp luôn chia hết cho $6$. (c) Tích của $n$ số tự nhiên liên tiếp bất kỳ luôn chia hết cho $n! = \prod_{i=1}^n i = 1\cdot2\cdot3\cdots n$, $\forall n\in\mathbb{N}^\star$.
\end{baitoan}

\begin{baitoan}[\cite{Binh_boi_duong_Toan_6_tap_1}, p. 28]
	Với $n\in\mathbb{N}^\star$. (a) Khi nào thì tổng của $n$ số tự nhiên liên tiếp bất kỳ chia hết cho $n$? (b) Khi nào thì tổng của $n$ số tự nhiên chẵn liên tiếp bất kỳ chia hết cho $n$? (c) Khi nào thì tổng của $n$ số tự nhiên lẻ liên tiếp bất kỳ chia hết cho $n$?
\end{baitoan}

%------------------------------------------------------------------------------%

\section{Divisibility Rule -- Dấu Hiệu Chia Hết}

%------------------------------------------------------------------------------%

\section{Miscellaneous}

%------------------------------------------------------------------------------%

\printbibliography[heading=bibintoc]

\end{document}