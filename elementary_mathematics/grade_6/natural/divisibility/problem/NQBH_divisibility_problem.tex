\documentclass{article}
\usepackage[backend=biber,natbib=true,style=alphabetic,maxbibnames=50]{biblatex}
\addbibresource{/home/nqbh/reference/bib.bib}
\usepackage[utf8]{vietnam}
\usepackage{tocloft}
\renewcommand{\cftsecleader}{\cftdotfill{\cftdotsep}}
\usepackage[colorlinks=true,linkcolor=blue,urlcolor=red,citecolor=magenta]{hyperref}
\usepackage{amsmath,amssymb,amsthm,float,graphicx,mathtools,tipa}
\usepackage{enumitem}
\setlist{leftmargin=4mm}
\allowdisplaybreaks
\newtheorem{assumption}{Assumption}
\newtheorem{baitoan}{}
\newtheorem{cauhoi}{Câu hỏi}
\newtheorem{conjecture}{Conjecture}
\newtheorem{corollary}{Corollary}
\newtheorem{dangtoan}{Dạng toán}
\newtheorem{definition}{Definition}
\newtheorem{dinhly}{Định lý}
\newtheorem{dinhnghia}{Định nghĩa}
\newtheorem{example}{Example}
\newtheorem{ghichu}{Ghi chú}
\newtheorem{hequa}{Hệ quả}
\newtheorem{hypothesis}{Hypothesis}
\newtheorem{lemma}{Lemma}
\newtheorem{luuy}{Lưu ý}
\newtheorem{nhanxet}{Nhận xét}
\newtheorem{notation}{Notation}
\newtheorem{note}{Note}
\newtheorem{principle}{Principle}
\newtheorem{problem}{Problem}
\newtheorem{proposition}{Proposition}
\newtheorem{question}{Question}
\newtheorem{remark}{Remark}
\newtheorem{theorem}{Theorem}
\newtheorem{vidu}{Ví dụ}
\usepackage[left=1cm,right=1cm,top=5mm,bottom=5mm,footskip=4mm]{geometry}
\def\labelitemii{$\circ$}
\DeclareRobustCommand{\divby}{%
	\mathrel{\vbox{\baselineskip.65ex\lineskiplimit0pt\hbox{.}\hbox{.}\hbox{.}}}%
}

\title{Problem: Divisibility -- Bài Tập: Tính Chia Hết}
\author{Nguyễn Quản Bá Hồng\footnote{Independent Researcher, Ben Tre City, Vietnam\\e-mail: \texttt{nguyenquanbahong@gmail.com}; website: \url{https://nqbh.github.io}.}}
\date{\today}

\begin{document}
\maketitle
\begin{abstract}
	Last updated version: \href{https://github.com/NQBH/elementary_STEM_beyond/blob/main/elementary_mathematics/grade_6/natural/divisibility/problem/NQBH_divisibility_problem.pdf}{GitHub{\tt/}NQBH{\tt/}hobby{\tt/}elementary mathematics{\tt/}grade 6{\tt/}natural{\tt/}divisibility{\tt/}problem[pdf]}.\footnote{\textsc{url}: \url{https://github.com/NQBH/elementary_STEM_beyond/blob/main/elementary_mathematics/grade_6/natural/divisibility/problem/NQBH_divisibility_problem.pdf}.} [\href{https://github.com/NQBH/elementary_STEM_beyond/blob/main/elementary_mathematics/grade_6/natural/divisibility/problem/NQBH_divisibility_problem.tex}{\TeX}]\footnote{\textsc{url}: \url{https://github.com/NQBH/elementary_STEM_beyond/blob/main/elementary_mathematics/grade_6/natural/divisibility/problem/NQBH_divisibility_problem.tex}.}. 
\end{abstract}
\tableofcontents

%------------------------------------------------------------------------------%

\begin{itemize}\sf
	\item \textbf{divisible} [a] {\tt/}\textipa{d@'vIz@bl}{\tt/} [not before noun] \textit{divisible (by something)} that can be divided, usually with nothing remaining. {\sc opposite}: \textbf{indivisible}.
	\item \textbf{divisibility}  [n] [uncountable] {\tt/}\textipa{d@'vIz@bIl@ti}{\tt/}.
\end{itemize}

\section{Divisibility of Sum, Difference, Product -- Tính Chất Chia Hết của Tổng, Hiệu, Tích}

\begin{baitoan}[\cite{Binh_boi_duong_Toan_6_tap_1}, H1, p. 24]
	{\rm Đ{\tt/}S? (a) $127\cdot5 + 40\divby5$. (b) $13\cdot48 + 12 + 17\divby6$. (c) $3\cdot300 - 12\divby9$. (d) $49 + 62\cdot7\divby7$.}
\end{baitoan}

\begin{baitoan}[\cite{Binh_boi_duong_Toan_6_tap_1}, H2, p. 24]
	Khi chia số $a$ cho số $b$, $a,b\in\mathbb{N}^\star$, $a > b$ ta được số dư là $r$. Khi đó: {\sf A.} $a + r\divby b$. {\sf B.} $a - r\divby b$. {\sf C.} $a + b\divby r$. {\sf D.} $a - b\divby r$.
\end{baitoan}

\begin{baitoan}[\cite{Binh_boi_duong_Toan_6_tap_1}, H3, p. 24]
	Tìm số tự nhiên $x$ có 1 chữ số thỏa $121 + x\divby11$.
\end{baitoan}

\begin{baitoan}[\cite{Binh_boi_duong_Toan_6_tap_1}, VD1, p. 25]
	Không tính các tổng \& hiệu, xét xem các tổng \& hiệu sau có chia hết cho $12$ không? Vì sao? (a) $600\cdot37 - 144$. (b) $96 + 34 + 48$.
\end{baitoan}

\begin{baitoan}[\cite{Binh_boi_duong_Toan_6_tap_1}, VD2, p. 25]
	Không tính ra kết quả, xét xem tổng $84 + 37 + 23$ có chia hết cho $12$ không? Vì sao?
\end{baitoan}

\begin{baitoan}[\cite{Binh_boi_duong_Toan_6_tap_1}, VD3, p. 25]
	Chứng minh trong 3 số tự nhiên liên tiếp có 1 số chia hết cho $3$.
\end{baitoan}

\begin{baitoan}[\cite{Binh_boi_duong_Toan_6_tap_1}, Mở rộng VD4, p. 25]
	Với $n\in\mathbb{N}^\star$ bất kỳ. Chứng minh: (a) Trong $n$ số tự nhiên liên tiếp luôn có 1 số chia hết cho $n$. (b) Tích của $n$ số tự nhiên liên tiếp là 1 số chia hết cho $n$.
\end{baitoan}

\begin{baitoan}[\cite{Binh_boi_duong_Toan_6_tap_1}, VD4, p. 26]
	Chứng minh tổng của 3 số tự nhiên liên tiếp là 1 số chia hết cho $3$.
\end{baitoan}

\begin{baitoan}
	Với $n\in\mathbb{N}^\star$ bất kỳ. Liệu tổng của $n$ số tự nhiên liên tiếp có chia hết cho $n$ không?
\end{baitoan}

\begin{baitoan}[\cite{Binh_boi_duong_Toan_6_tap_1}, VD5, p. 26]
	Chứng minh: (a) $\overline{ab} - \overline{ba}\divby9$ với $a > b$. (b) Nếu $\overline{ab} + \overline{cd}\divby11$ thì $\overline{abcd}\divby11$.
\end{baitoan}

\begin{baitoan}[\cite{Binh_boi_duong_Toan_6_tap_1}, VD6, p. 26]
	Cho $A = 15 + 30 + 37 + x$ với $x\in\mathbb{N}$. Tìm điều kiện của $x$ để: (a) $A\divby3$. (b) $A\not{\divby}\ 9$.
\end{baitoan}

\begin{baitoan}[\cite{Binh_boi_duong_Toan_6_tap_1}, VD7, p. 26]
	Tìm $n\in\mathbb{N}$ để: (a) $n + 4\divby n$. (b) $5n - 6\divby n$ với $n > 1$. (c) $143 - 12n\divby n$ với $n < 12$.
\end{baitoan}

\begin{baitoan}[\cite{Binh_boi_duong_Toan_6_tap_1}, VD8, p. 27]
	Tìm $n\in\mathbb{N}$ để: (a) $n + 9\divby n + 4$. (b) $3n + 40\divby n + 4$. (c) $5n + 2\divby2n + 9$.
\end{baitoan}

\begin{baitoan}[\cite{Binh_boi_duong_Toan_6_tap_1}, 3.1., p. 27]
	Cho $A = 2\cdot5\cdot9\cdot13 + 84$. Hỏi $A$ có chia hết cho $3$, cho $6$, cho $9$, cho $13$ không? Vì sao?
\end{baitoan}

\begin{baitoan}[\cite{Binh_boi_duong_Toan_6_tap_1}, 3.2., p. 27]
	Chứng minh tổng 5 số chẵn liên tiếp là 1 số chia hết cho $10$.
\end{baitoan}

\begin{baitoan}[\cite{Binh_boi_duong_Toan_6_tap_1}, 3.3., p. 27]
	Khi chia số tự nhiên $a$ cho $27$, ta được số dư là $15$. Hỏi số $a$ có chia hết cho $3$, cho $9$ không? Vì sao?
\end{baitoan}

\begin{baitoan}[\cite{Binh_boi_duong_Toan_6_tap_1}, 3.4., p. 27]
	Chứng minh mọi số tự nhiên có 3 chữ số giống nhau đều chia hết cho $37$.
\end{baitoan}

\begin{baitoan}[\cite{Binh_boi_duong_Toan_6_tap_1}, 3.5., p. 28]
	Chứng minh: (a) $\sum_{i=0}^{101} 5^i = 1 + 5 + 5^2 + 5^3 + \cdots + 5^{101}\divby6$. (b) $\sum_{i=1}^{100} 2^i = 2 + 2^2 + 2^3 + \cdots + 2^{100}$ vừa chia hết cho $31$, vừa chia hết cho $5$.
\end{baitoan}

\begin{baitoan}[\cite{Binh_boi_duong_Toan_6_tap_1}, 3.6., p. 28]
	Chứng minh: (a) Nếu $\overline{abc} - \overline{def}\divby11$ thì $\overline{abcdef}\divby11$. (b) Nếu $\overline{abc}\divby8$ thì $4a + 2b + c\divby8$.
\end{baitoan}

\begin{baitoan}[\cite{Binh_boi_duong_Toan_6_tap_1}, 3.7., p. 28]
	Tìm chữ số $a$ biết $\overline{21a21a21a}\divby31$.
\end{baitoan}

\begin{baitoan}[\cite{Binh_boi_duong_Toan_6_tap_1}, 3.8., p. 28]
	Tìm $n\in\mathbb{N}$ sao cho: (a) $n + 21\divby n$. (b) $18 - 2n\divby n$ với $n < 9$. (c) $6n - 9\divby n$ với $n\ge2$.
\end{baitoan}

\begin{baitoan}[\cite{Binh_boi_duong_Toan_6_tap_1}, 3.9., p. 28]
	Tìm $n\in\mathbb{N}$ sao cho: (a) $n + 15\divby n - 3$ với $n > 5$. (b) $18 - 2n\divby n + 3$ với $n\le9$. (c) $3n + 13\divby2n + 3$ với $n\ge1$.
\end{baitoan}

\begin{baitoan}[\cite{Binh_boi_duong_Toan_6_tap_1}, 3.10., p. 28]
	Cho $a,b\in\mathbb{N}$. Chứng minh nếu $7a + 2b$ \& $31a + 9b$ cùng chia hết cho $2015$ thì $a,b$ cũng chia hết cho $2015$.
\end{baitoan}

\begin{baitoan}[\cite{Binh_boi_duong_Toan_6_tap_1}, p. 28]
	Chứng minh: (a) Tích 2 số tự nhiên liên tiếp là 1 số chẵn. (b) Tích 3 số tự nhiên liên tiếp luôn chia hết cho $6$. (c) Tích của $n$ số tự nhiên liên tiếp bất kỳ luôn chia hết cho $n! = \prod_{i=1}^n i = 1\cdot2\cdot3\cdots n$, $\forall n\in\mathbb{N}^\star$.
\end{baitoan}

\begin{baitoan}[\cite{Binh_boi_duong_Toan_6_tap_1}, p. 28]
	Với $n\in\mathbb{N}^\star$. (a) Khi nào thì tổng của $n$ số tự nhiên liên tiếp bất kỳ chia hết cho $n$? (b) Khi nào thì tổng của $n$ số tự nhiên chẵn liên tiếp bất kỳ chia hết cho $n$? (c) Khi nào thì tổng của $n$ số tự nhiên lẻ liên tiếp bất kỳ chia hết cho $n$?
\end{baitoan}

\begin{baitoan}[\cite{Tuyen_Toan_6}, VD21, p. 22]
	Cho $a\divby m,b\divby m$. Chứng minh $k_1a + k_2b\divby m$.
\end{baitoan}

\begin{baitoan}[\cite{Tuyen_Toan_6}, VD21, p. 22]
	Chứng minh: (a) Nếu $a\divby m,b\divby m,a + b + c\divby m$ thì $c\divby m$. (b) Nếu $a\divby m,b\divby m,a + b + c\not{\divby}\ m$ thì $c\not{\divby}\ m$.
\end{baitoan}

\begin{baitoan}[\cite{Tuyen_Toan_6}, 90., p. 22]
	Chứng minh $\forall n\in\mathbb{N}$, $60n + 45\divby15$ nhưng $60n + 45\not{\divby}\ 30$
\end{baitoan}

\begin{baitoan}[\cite{Tuyen_Toan_6}, 91., p. 22]
	Cho $A = 2\cdot4\cdot6\cdot8\cdot10\cdot12 + 40$. A có chia hết cho $6$, cho $8$, cho $5$ không?
\end{baitoan}

\begin{baitoan}[\cite{Tuyen_Toan_6}, 92., p. 22]
	Cho $A = 23! + 19! - 15!$. Chứng minh: (a) $A\divby11$. (b) $A\divby10$.
\end{baitoan}

\begin{baitoan}[\cite{Tuyen_Toan_6}, 93., p. 23]
	Chứng minh tổng của 3 số tự nhiên liên tiếp thì chia hết cho $3$ còn tổng của $4$ số tự nhiên liên tiếp thì không chia hết cho $4$.
\end{baitoan}

\begin{baitoan}[\cite{Tuyen_Toan_6}, 94., p. 23]
	Chứng minh tổng của $5$ số chẵn liên tiếp thì chia hết cho $10$ còn tổng của $5$ số lẻ liên tiếp chia cho $10$ dư $5$.
\end{baitoan}

\begin{baitoan}[\cite{Tuyen_Toan_6}, 95., p. 23]
	Cho 4 số tự nhiên không chia hết cho $5$, khi chia cho $5$ được các số dư khác nhau. Chứng minh tổng của 4 số này chia hết cho $5$.
\end{baitoan}

\begin{baitoan}[\cite{Tuyen_Toan_6}, 96., p. 23]
	Cho $A = \sum_{i=0}^{11} 3^i = 1 + 3 + 3^2 + \cdots + 3^{11}$. Chứng minh: (a) $A\divby13$. (b) $A\divby40$.
\end{baitoan}

\begin{baitoan}[\cite{Tuyen_Toan_6}, 97., p. 23]
	Chứng minh: (a) Tích của 2 số tự nhiên liên tiếp thì chia hết cho $2$. (b) Tích của 3 số tự nhiên liên tiếp thì chia hết cho $3$.
\end{baitoan}

\begin{baitoan}[\cite{Tuyen_Toan_6}, 98., p. 23]
	Chứng minh không có số tự nhiên nào chia cho $15$ dư $6$ còn chia cho $9$ dư $1$.
\end{baitoan}

\begin{baitoan}[\cite{Tuyen_Toan_6}, 99., p. 23]
	Tìm $n\in\mathbb{N}$ thỏa: (a) $n + 4\divby n$. (b) $3n + 7\divby n$. (c) $27 - 5n\divby n$.
\end{baitoan}

\begin{baitoan}[\cite{Tuyen_Toan_6}, 100., p. 23]
	Tìm $n\in\mathbb{N}$ thỏa: (a) $n + 6\divby n + 2$. (b) $2n + 3\divby n - 2$. (c) $3n + 1\divby11 - 2n$.
\end{baitoan}

\begin{baitoan}[\cite{Tuyen_Toan_6}, 101., p. 23]
	Cho $10^k - 1\divby19$ với $k > 1$. Chứng minh: (a) $10^{2k} - 1\divby19$. (b) $10^{3k} - 1\divby19$.
\end{baitoan}

\begin{baitoan}[\cite{Binh_Toan_6_tap_1}, VD21, p. 22]
	Chứng minh: (a) $\overline{ab} + \overline{ba}\divby11$. (b) $\overline{ab} - \overline{ba}\divby9$ với $a > b$.
\end{baitoan}

\begin{baitoan}[\cite{Binh_Toan_6_tap_1}, VD22, p. 23]
	Quan sát các ví dụ: $14 + 19 = 33\divby11,1419\divby11$, $6 + 49 = 55\divby11,649\divby11$. Rút ra nhận xét \& chứng minh nhận xét ấy.
\end{baitoan}

\begin{baitoan}[\cite{Binh_Toan_6_tap_1}, VD23, p. 23]
	Cho số $\overline{abc}\divby27$. Chứng minh $\overline{bca}\divby27$.
\end{baitoan}

\begin{baitoan}[\cite{Binh_Toan_6_tap_1}, VD24, p. 23]
	Tìm số tự nhiên có 3 chữ số biết số đó chia hết cho $18$ \& các chữ số của nó nếu sắp xếp từ nhỏ đến lớn thì tỷ lệ với $1:2:3$.
\end{baitoan}

\begin{baitoan}[\cite{Binh_Toan_6_tap_1}, 118., p. 23]
	Có thể chọn được 5 số trong dãy số sau để tổng của chúng bằng $70$ không? (a) $1,2,\ldots,30$. (b) $1,3,5,\ldots,29$.
\end{baitoan}

\begin{baitoan}[\cite{Binh_Toan_6_tap_1}, 119., p. 23]
	Cho 9 số: $1,3,5,7,9,11,13,15,17$. Có thể phân chia được hay không 9 số trên thành 2 nhóm sao cho: (a) Tổng các số thuộc nhóm I gấp đôi tổng các số thuộc nhóm II? (b) Tổng các số thuộc nhóm I bằng tổng các số thuộc nhóm II?
\end{baitoan}

\begin{baitoan}[\cite{Binh_Toan_6_tap_1}, 120., p. 23]
	(a) Có 3 số tự nhiên nào mà tổng của chúng tận cùng bằng $4$, tích của chúng tận cùng bằng $1$ không? (b) Có tồn tại hay không 4 số tự nhiên mà tổng của chúng \& tích của chúng đều là số lẻ?
\end{baitoan}

\begin{baitoan}[\cite{Binh_Toan_6_tap_1}, 121., p. 23]
	Chứng minh không tồn tại $a,b,c\in\mathbb{N}$ thỏa:
	\begin{equation*}
		\left\{\begin{split}
			abc + a &= 333,\\
			abc + b &= 335,\\
			abc + c &= 341.
		\end{split}\right.
	\end{equation*}
\end{baitoan}

\begin{baitoan}[\cite{Binh_Toan_6_tap_1}, 122., p. 23]
	1 lớp học có $6$ tổ, số người của mỗi tổ bằng nhau. Trong 1 bài kiểm tra, tất cả học sinh đều được điểm $7$ hoặc $8$. Tổng số điểm của cả lớp là $350$. Tính số học sinh của lớp, số học sinh đạt từng loại điểm.
\end{baitoan}

\begin{baitoan}[\cite{Binh_Toan_6_tap_1}, 123., p. 24]
	Khối 6 của 1 trường có $366$ học sinh, gồm $8$ lớp. Mỗi lớp gồm 1 số tổ, mỗi tổ có $9$ người hoặc $10$ người. Biết số tổ của các lớp đều bằng nhau, tính số tổ có $9$ người, số tổ có $10$ người của cả khối.
\end{baitoan}

\begin{baitoan}[\cite{Binh_Toan_6_tap_1}, 124., p. 24]
	(a) Chứng minh nếu viết thêm vào đằng sau 1 số tự nhiên có 2 chữ số số gồm chính 2 chữ số ấy viết theo thứ tự ngược lại thì được 1 số chia hết cho $11$. (b) Mở rộng (a) cho số tự nhiên có 3 chữ số.
\end{baitoan}

\begin{baitoan}[\cite{Binh_Toan_6_tap_1}, 125., p. 24]
	Chứng minh nếu $\overline{ab} = 2\overline{cd}$ thì $\overline{abcd}\divby67$.
\end{baitoan}

\begin{baitoan}[\cite{Binh_Toan_6_tap_1}, 126., p. 24]
	Chứng minh: (a) $A = \overline{abcabc}$, $A\divby7,A\divby11,A\divby13$. (b) $B = \overline{abcdef}$, $B\divby23,B\divby29$, biết $\overline{abc} = 2\overline{def}$.
\end{baitoan}

\begin{baitoan}[\cite{Binh_Toan_6_tap_1}, 127., p. 24]
	Chứng minh nếu $\overline{ab} + \overline{cd} + \overline{ef}\divby11$ thì $\overline{abcdef}\divby11$.
\end{baitoan}

\begin{baitoan}[\cite{Binh_Toan_6_tap_1}, 128., p. 24]
	(a) Cho $\overline{abc} + \overline{def}\divby37$. Chứng minh $\overline{abcdef}\divby37$. (b) Cho $\overline{abc} + \overline{def}\divby7$. Chứng minh $\overline{abcdef}\divby7$. (c) Cho 8 số tự nhiên có 3 chữ số. Chứng minh trong 8 số đó, tồn tại 2 số mà khi viết liên tiếp nhau thì tạo thành 1 số có 6 chữ số chia hết cho $7$.
\end{baitoan}

\begin{baitoan}[\cite{Binh_Toan_6_tap_1}, 129., p. 24]
	Tìm chữ số $a$ biết $\overline{20a20a20a}\divby7$.
\end{baitoan}

\begin{baitoan}[\cite{Binh_Toan_6_tap_1}, 130., p. 24]
	Cho 3 chữ số khác nhau \& khác $0$. Lập tất cả các số tự nhiên có 3 chữ số gồm cả 3 chữ số ấy. Chứng minh tổng của chúng chia hết cho $6,37$.
\end{baitoan}

\begin{baitoan}[\cite{Binh_Toan_6_tap_1}, 131., p. 24]
	Có $x,y\in\mathbb{N}$ thỏa $(x + y)(x - y) = 1002$ không?
\end{baitoan}

\begin{baitoan}[\cite{Binh_Toan_6_tap_1}, 132., p. 24]
	Tìm $n\in\mathbb{N}$ nhỏ nhất sao cho ta có cách thêm $n$ chữ số vào sau số đó để được 1 số chia hết cho $39$.
\end{baitoan}

\begin{baitoan}[\cite{Binh_Toan_6_tap_1}, 133., p. 24]
	Tìm $a\in\mathbb{N}$ có 2 chữ số sao cho nếu viết $a$ tiếp sau số $1999$ thì ta được 1 số chia hết cho $37$.
\end{baitoan}

\begin{baitoan}[\cite{Binh_Toan_6_tap_1}, 134., p. 24]
	Cho $n\in\mathbb{N}$. Chứng minh: (a) $(n + 10)(n + 15)\divby2$. (b) $A = n(n + 1)(n + 2)$, $A\divby2,A\divby3$. (c) $B = n(n + 1)(2n + 1)$, $B\divby2,B\divby3$.
\end{baitoan}

\begin{baitoan}[\cite{Binh_Toan_6_tap_1}, 135., p. 24]
	Tìm $a,b\in\mathbb{N}$ thỏa $a\divby b,b\divby a$.
\end{baitoan}

\begin{baitoan}[\cite{Binh_Toan_6_tap_1}, 136., p. 24]
	1 học sinh viết các số tự nhiên từ $1$ đến $\overline{abc}$. Bạn đó phải viết tất cả $m$ chữ số. Biết $m\divby\overline{abc}$, tìm $\overline{abc}$.
\end{baitoan}

\begin{baitoan}[\cite{Binh_Toan_6_tap_1}, 137., p. 24]
	Cho 9 số tự nhiên viết theo thứ tự giảm dần từ $9$ đến $1$: $987654321$. Có thể đặt được hay không 1 số dấu $+$ hoặc $-$ vào giữa các số đó để kết quả của phép tính bằng: (a) $5$. (b) $6$?
\end{baitoan}

\begin{baitoan}[\cite{Binh_Toan_6_tap_1}, 138., p. 25]
	Cho tổng $1 + 2 + \cdots + 9$. Xóa 2 số bất kỳ rồi thay bằng hiệu của chúng \& cứ làm như vậy nhiều lần. Có cách nào làm cho kết quả cuối cùng bằng $0$ không?
\end{baitoan}

\begin{baitoan}[\cite{Binh_Toan_6_tap_1}, 139., p. 25]
	Chứng minh tổng các số ghi trên vé xổ số có 6 chữ số mà tổng 3 chữ số đầu bằng tổng 3 chữ số cuối thì chia hết cho $13$ (các chữ số đầu có thể bằng $0$).
\end{baitoan}

\begin{baitoan}[\cite{Binh_Toan_6_tap_1}, 140., p. 25]
	(a) $n\in\mathbb{N}$ chia cho $54$ dư $17$. Tìm số dư lớn nhất khi chia $n$ cho $162$. (b) $n\in\mathbb{N}$ chia cho $802$ dư $502$. Tìm số dư nhỏ nhất khi chia $n$ choa $2005$.
\end{baitoan}

\begin{baitoan}[\cite{Binh_Toan_6_tap_1}, 141., p. 25]
	Tìm $a,b\in\mathbb{N}$ nhỏ nhất lớn hơn $1$ sao cho $a^7 = b^8$.
\end{baitoan}

%------------------------------------------------------------------------------%

\section{Divisibility Rule -- Dấu Hiệu Chia Hết}

\begin{baitoan}[\cite{Binh_boi_duong_Toan_6_tap_1}, H1, p. 29]
	Nối cột để được kết quả đúng.
	\begin{table}[H]
		\centering
		\begin{tabular}{|l|l|}
			\hline
			(a) $230 + 175$ & (1) chia hết cho 2 nhưng không chia hết cho 5. \\
			\hline
			(b) $2070 - 590$ & (2) chia hết cho 5 nhưng không chia hết cho 2. \\
			\hline
			(c) $747 + 350$ & (3) chia hết cho cả 2 \& 5. \\
			\hline
			& (4) không chia hết cho cả 2 \& 5. \\
			\hline
		\end{tabular}
	\end{table}
\end{baitoan}

\begin{baitoan}[\cite{Binh_boi_duong_Toan_6_tap_1}, H2, p. 30]
	Khi giải bài toán: ``Thêm 1 chữ số vào bên phải \& 1 chữ số vào bên trái số $2015$ để được 1 số mới chia hết cho cả $2,3,5$.'' Tìm kết quả sai: {\sf A.} $120150$. {\sf B.} $420150$. {\sf C.} $620150$. {\sf D.} $720150$.
\end{baitoan}

\begin{baitoan}[\cite{Binh_boi_duong_Toan_6_tap_1}, Mở rộng H2, p. 29]
	Thêm 1 chữ số vào bên phải \& 1 chữ số vào bên trái số $2015$ để được 1 số mới chia hết cho cả $2,3,5$. Tìm tất cả các cặp số có thể thêm vào.
\end{baitoan}

\begin{baitoan}[\cite{Binh_boi_duong_Toan_6_tap_1}, H3, p. 30]
	Trong khoảng từ $1010$ đến $1975$ có bao nhiêu số chia hết cho $3$?
\end{baitoan}

\begin{baitoan}[\cite{Binh_boi_duong_Toan_6_tap_1}, H4, p. 30]
	Thay các chữ cái khác nhau bởi các chữ số khác nhau: $\rm HANOI + HANOI + HANOI = \overline{TT221}$.
\end{baitoan}

\begin{baitoan}[\cite{Binh_boi_duong_Toan_6_tap_1}, VD1, p. 30]
	2 bạn Egg \& Chicken đi mua $18$ gói bánh \& $12$ gói kẹo để chuẩn bị cho buổi liên hoan lớp. Egg đưa cho cô bán hàng 3 tờ tiền, mỗi tờ có mệnh giá $50000$ đồng \& được trả lại $22000$ đồng. Thấy vậy, Chicken liền nói: ``Cô tính sai rồi!'' Chicken đúng hay sai? Vì sao?
\end{baitoan}

\begin{baitoan}[\cite{Binh_boi_duong_Toan_6_tap_1}, VD2, p. 30]
	Chứng minh $(n + 29)(n + 30)\divby2$, $\forall n\in\mathbb{N}$.
\end{baitoan}

\begin{baitoan}[Tính chia hết cho 2 của 1 tích]
	(a) Với $a,b\in\mathbb{N}$ thỏa điều kiện nào thì $(n + a)(n + b)\divby2$, $\forall n\in\mathbb{N}$? (b) Với $a,b,c\in\mathbb{N}$ thỏa điều kiện nào thì $(n + a)(n + b)\divby2$, $\forall n\in\mathbb{N}$? (c) Cho $n\in\mathbb{N}^\star$. Với $a_1,a_2,\ldots,a_n\in\mathbb{N}$ thỏa điều kiện nào thì $\prod_{i=1}^n (m + a_i) = (m + a_1)(m + a_2)\cdots(m + a_n)\divby2$, $\forall n\in\mathbb{N}$?
\end{baitoan}

\begin{baitoan}[Tính chia hết cho 3 của 1 tích]
	(a) Với $a,b\in\mathbb{N}$ thỏa điều kiện nào thì $(n + a)(n + b)\divby3$, $\forall n\in\mathbb{N}$? (b) Với $a,b,c\in\mathbb{N}$ thỏa điều kiện nào thì $(n + a)(n + b)\divby3$, $\forall n\in\mathbb{N}$? (c) Cho $n\in\mathbb{N}^\star$. Với $a_1,a_2,\ldots,a_n\in\mathbb{N}$ thỏa điều kiện nào thì $\prod_{i=1}^n (m + a_i) = (m + a_1)(m + a_2)\cdots(m + a_n)\divby3$, $\forall n\in\mathbb{N}$?
\end{baitoan}

\begin{baitoan}[\cite{Binh_boi_duong_Toan_6_tap_1}, VD3, p. 31]
	Chứng minh $39^{2015} + 11^{2016}\divby10$.
\end{baitoan}

\begin{baitoan}
	Với $a,b\in\mathbb{N}$ thỏa điều kiện nào thì: (a) $39^a + 11^b\divby10$? (b) $(\overline{a_ma_{m-1}\ldots a_19})^a + (\overline{b_nb_{n-1}\ldots b_11})^b\divby10$ với $a_i,b_j\in\{0,1,2,\ldots,9\}$, $\forall i = 1,2,\ldots,m$, $\forall j = 1,2,\ldots,n$, $a_nb_m\ne0$?
\end{baitoan}

\begin{baitoan}[\cite{Binh_boi_duong_Toan_6_tap_1}, VD4, p. 31]
	Thay dấu $+$ hoặc $-$ vào các dấu $\star$ trong dãy tính sau để được kết quả là 1 số chia hết cho $2$: $10\star9\star8\star7\star6\star5\star4\star3\star2\star1$.
\end{baitoan}

\begin{baitoan}
	Thay dấu $+$ hoặc $-$ vào các dấu $\star$ trong dãy tính sau để được kết quả là 1 số chia hết cho $2$: $n\star(n - 1)\star(n - 2)\star\cdots3\star2\star1$ với $n\in\mathbb{N}$.
\end{baitoan}

\begin{baitoan}[\cite{Binh_boi_duong_Toan_6_tap_1}, VD5, p. 32]
	Viết các số tự nhiên liên tiếp từ $10$ đến $99$ ta được số $A$. Hỏi $A$ có chia hết cho $9$ không? Vì sao?
\end{baitoan}

\begin{baitoan}
	Cho $n\in\mathbb{N}^\star$. Viết các số tự nhiên liên tiếp từ $10^n$ (số tự nhiên nhỏ nhất có $n + 1$ chữ số) đến $10^{n+1} - 1$ (số tự nhiên lớn nhất có $n + 1$ chữ số) ta được số $A$. Hỏi $A$ có chia hết cho $9$ không? Vì sao?
\end{baitoan}

\begin{baitoan}[\cite{Binh_boi_duong_Toan_6_tap_1}, VD6, p. 32]
	Tìm 2 chữ số $x,y$ biết: (a) $\overline{38x5y}$ chia hết cho $2,5,9$. (b) $\overline{12x3y}\divby45$.
\end{baitoan}

\begin{baitoan}[\cite{Binh_boi_duong_Toan_6_tap_1}, VD7, p. 32]
	Thay $a,b$ bằng các chữ số thích hợp để số $ \overline{2a83b}$ chia hết cho $3$ \& chia cho $5$ dư $1$.
\end{baitoan}

\begin{baitoan}[\cite{Binh_boi_duong_Toan_6_tap_1}, VD8, p. 33]
	Tìm 2 số tự nhiên chia hết cho $9$, biết tổng của chúng bằng $\overline{35\star1}$ \& số lớn gấp đôi số bé.
\end{baitoan}

\begin{baitoan}[\cite{Binh_boi_duong_Toan_6_tap_1}, VD9, p. 33]
	Tìm chữ số $a$ sao cho $\overline{95a14}\divby11$.
\end{baitoan}

\begin{baitoan}[\cite{Binh_boi_duong_Toan_6_tap_1}, 4.1., p. 33]
	Từ 3 trong 5 chữ số $2,5,7,8,0$, ghép thành số có 3 chữ số khác nhau thỏa mãn 1 trong các điều kiện: (a) Là số lớn nhất chia hết cho $2$. (b) Là số nhỏ nhất chia hết cho $2$. (c) Là số lớn nhất chia hết cho $5$. (d) Là số nhỏ nhất chia hết cho $5$. (e) Là số lớn nhất chia hết cho $9$. (f) Là số nhỏ nhất chia hết cho $9$. (g) Là số lớn nhất chia hết cho $3$. (h) Là số nhỏ nhất chia hết cho $3$.
\end{baitoan}

\begin{baitoan}[\cite{Binh_boi_duong_Toan_6_tap_1}, 4.2., p. 33]
	Dùng 3 trong 4 số $,2,4,6,8$, viết tất cả các số tự nhiên có 3 chữ số chia hết cho cả 3 số $2,3,9$.
\end{baitoan}

\begin{baitoan}[\cite{Binh_boi_duong_Toan_6_tap_1}, 4.3., p. 33]
	Có $10$ mẩu que lần lượt dài {\rm1 cm, 2cm, 3cm, $\ldots$, 10 cm}. Hỏi có thể dùng cả $10$ mẫu que đó để xếp thành 1 tam giác có 3 cạnh bằng nhau không?
\end{baitoan}

\begin{baitoan}[\cite{Binh_boi_duong_Toan_6_tap_1}, 4.4., p. 33]
	Chứng minh: (a) $10^{2015} + 8\divby18$. (b) $10^{21} + 20\divby6$.
\end{baitoan}

\begin{baitoan}[\cite{Binh_boi_duong_Toan_6_tap_1}, 4.5., p. 33]
	Chứng minh $(n + 11)(n + 12)\divby2$, $\forall n\in\mathbb{N}$.
\end{baitoan}

\begin{baitoan}[\cite{Binh_boi_duong_Toan_6_tap_1}, 4.6., p. 33]
	Chứng minh tích của 3 số tự nhiên chẵn liên tiếp chia hết cho $48$.
\end{baitoan}

\begin{baitoan}[\cite{Binh_boi_duong_Toan_6_tap_1}, 4.7., p. 33]
	Tìm số tự nhiên có 5 chữ số, các chữ số giống nhau, biết số đó chia cho $5$ dư $4$ \& chia hết cho $2$.
\end{baitoan}

\begin{baitoan}[\cite{Binh_boi_duong_Toan_6_tap_1}, 4.8., p. 34]
	Tìm 2 chữ số $x,y$ biết: (a) $\overline{2x98y}$ chia hết cho $2,3,5$. (b) $\overline{43xy5}\divby45$. (c) $\overline{21x7y}$ chia hết cho $5,18$.
\end{baitoan}

\begin{baitoan}[\cite{Binh_boi_duong_Toan_6_tap_1}, 4.9., p. 34]
	Tìm chữ số $a$ để $\overline{aaaaa96}$ chia hết cho cả $3$ \& $8$.
\end{baitoan}

\begin{baitoan}[\cite{Binh_boi_duong_Toan_6_tap_1}, 4.10., p. 34]
	Tìm chữ số $a$ để $\overline{1aaa1}\divby11$.
\end{baitoan}

\begin{baitoan}[\cite{Binh_boi_duong_Toan_6_tap_1}, 4.11., p. 34]
	Cho $a\in\mathbb{N}$. Đổi chỗ các chữ số của $a$ để được số $b$ gấp $3$ lần số $a$. Chứng minh $a\divby27$.
\end{baitoan}

\begin{baitoan}[\cite{Binh_boi_duong_Toan_6_tap_1}, 4.12., p. 34]
	Cho $n\in\mathbb{N}^\star$. Chứng minh: (a) $6^n - 1\divby5$. (b) $10^n + 18n - 1\divby27$.
\end{baitoan}

\begin{baitoan}[\cite{Binh_boi_duong_Toan_6_tap_1}, 4.13., p. 34]
	Tìm 2 chữ số $a,b$ sao cho: (a) $\overline{71ab}$ chia hết cho $9$, cho $2$, \& chia cho $5$ dư $3$. (b) $\overline{15a3b}$ chia hết cho $2$, chia hết cho $9$, \& chia cho $5$ dư $4$.
\end{baitoan}

\begin{baitoan}[\cite{Binh_boi_duong_Toan_6_tap_1}, 4.14., p. 34]
	Tìm 2 số tự nhiên liên tiếp có 2 chữ số, biết 1 số chia hết cho $4$, số kia chia hết cho $25$.
\end{baitoan}

\begin{baitoan}[\cite{Binh_boi_duong_Toan_6_tap_1}, 4.15., p. 34]
	Tìm số tự nhiên có 4 chữ số sao cho khi nhân số đó với $9$ ta được số mới gồm chính các chữ số của số ấy nhưng viết theo thứ tự ngược lại.
\end{baitoan}

\begin{baitoan}[\cite{Binh_boi_duong_Toan_6_tap_1}, 4.16., p. 34, Thái Lan]
	Nếu đem số $31513$ \& số $34369$ chia cho cùng 1 số có 3 chữ số thì 2 phép chia có số dư bằng nhau. Tìm số dư của 2 phép chia đó.
\end{baitoan}

\begin{baitoan}[\cite{Binh_boi_duong_Toan_6_tap_1}, 4.17., p. 34]
	Chứng minh hiệu của 1 số \& tổng các chữ số của nó chia hết cho $9$.
\end{baitoan}

\begin{dinhnghia}[Hàm tổng các chữ số]
	Ký hiệu $S(n)$ là tổng các chữ số của $n\in\mathbb{N}$.
\end{dinhnghia}

\begin{baitoan}[\cite{Binh_boi_duong_Toan_6_tap_1}, 4.18., p. 34]
	 Tìm $n\in\mathbb{N}$ biết $n + S(n) = 88$.
\end{baitoan}

\begin{baitoan}[\cite{Tuyen_Toan_6}, VD23, p. 23]
	Chứng minh: $9^{2n} - 1$ chia hết cho $2$ \& $5,\forall n\in\mathbb{N}$.
\end{baitoan}

\begin{baitoan}[\cite{Tuyen_Toan_6}, VD24, p. 24]
	Cho số $A = \overline{76a23}$. (a) Tìm chữ số $a$ để $A\divby9$. (b) Trong các giá trị vừa tìm được của $a$, có giá trị nào để $A\divby11$ không?
\end{baitoan}

\begin{baitoan}[\cite{Tuyen_Toan_6}, VD25, p. 24]
	Theo dương lịch cứ $4$ năm lại có 1 năm nhuận, i.e., năm chia hết cho $4$. Tuy nhiên các năm có tận cùng bằng 2 chữ số $0$ chỉ được coi là năm nhuận nếu chúng cũng chia hết cho $400$. Tính số năm nhuận từ năm $2000$--$2100$.
\end{baitoan}

\begin{baitoan}[\cite{Tuyen_Toan_6}, 102., p. 24]
	Cho $n\in\mathbb{N}$. Chứng minh $6^n - 1\divby5$.
\end{baitoan}

\begin{baitoan}[\cite{Tuyen_Toan_6}, 103., p. 24]
	Cho $n\in\mathbb{N}$. Chứng minh $5^n - 1\divby4$.
\end{baitoan}

\begin{baitoan}[\cite{Tuyen_Toan_6}, 104., p. 24]
	Chứng minh: (a) $942^{60} - 351^{37}\divby5$. (b) $99^5 - 98^4 + 97^3 - 96^2$ chia hết cho $2$ \& $5$.
\end{baitoan}

\begin{baitoan}[\cite{Tuyen_Toan_6}, 105., p. 24]
	Có 2 số tự nhiên nào mà tổng bằng $3456$ \& số lớn gấp $4$ lần số nhỏ không?
\end{baitoan}

\begin{baitoan}[\cite{Tuyen_Toan_6}, 106., p. 24]
	Cho $a,b\in\mathbb{N}$. Hỏi số $ab(a + b)$ có tận cùng bằng $9$ không?
\end{baitoan}

\begin{baitoan}[\cite{Tuyen_Toan_6}, 107., p. 24]
	Cho $n\in\mathbb{N}$, $A = n^2 + n + 1$. Chứng minh $A\not{\divby}\ 4$, $A\not{\divby}\ 5$.
\end{baitoan}

\begin{baitoan}[\cite{Tuyen_Toan_6}, 108., p. 24]
	Cho số $\overline{abc}\not{\divby}\ 3$. Phải viết số này liên tiếp nhau mấy lần để được 1 số chia hết cho $3$?
\end{baitoan}

\begin{baitoan}[\cite{Tuyen_Toan_6}, 109., p. 24]
	1 số tự nhiên có chữ số đầu tiên lớn hơn chữ số hàng đơn vị. Khi viết số đó theo thứ tự ngược lại thì được 1 số mới kém số cũ là 1 trong 3 số $2020,2021,2022$. Hiệu của chúng là số nào trong 3 số đó?
\end{baitoan}

\begin{baitoan}[\cite{Tuyen_Toan_6}, 110., p. 24]
	Cho biểu thức $A = 1494\cdot1495\cdot1496$. Không thực hiện phép tính, chứng minh: (a) $A\divby180$. (b) $A\divby495$.
\end{baitoan}

\begin{baitoan}[\cite{Tuyen_Toan_6}, 111., p. 24]
	Chứng minh $\forall n\in\mathbb{N}$: (a) $10^n - 1\divby9$. (b) $10^n + 8\divby9$.
\end{baitoan}

\begin{baitoan}[\cite{Tuyen_Toan_6}, 112., p. 25]
	Chứng minh hiệu của 1 số \& tổng các chữ số của nó thì chia hết cho $9$.
\end{baitoan}

\begin{baitoan}[\cite{Tuyen_Toan_6}, 113., p. 25]
	Cho số $A = 8n + \underbrace{1\ldots1}_n$, với $n\in\mathbb{N}^\star$. Chứng minh $A\divby9$.
\end{baitoan}

\begin{baitoan}[\cite{Tuyen_Toan_6}, 114., p. 25]
	Lấy 1 mảnh giấy cắt ra làm $4$ mảnh nhỏ. Lấy 1 mảnh bất kỳ cắt ra  thành $4$ mảnh khác. Cứ thế tiếp tục nhiều lần. (a) Hỏi khi ngừng cắt theo quy luật trên thì có thể được tất cả $60$ mảnh giấy nhỏ không? (b) Phải cắt tất cả bao nhiêu mảnh giấy theo quy luật trên để được tất cả $52$ mảnh giấy nhỏ?
\end{baitoan}

\begin{baitoan}[\cite{Tuyen_Toan_6}, 115., p. 25]
	Chọn bất kỳ $90$ số trong $100$ số tự nhiên từ $1$--$100$ thì có ít nhất bao nhiêu số chia hết cho $9$?
\end{baitoan}

\begin{baitoan}[\cite{Binh_Toan_6_tap_1}, VD25, p. 25]
	Tìm $a\in\mathbb{N}$ có 4 chữ số, $a\divby5,a\divby27$, biết 2 chữ số giữa của $a$ là $97$.
\end{baitoan}

\begin{baitoan}[\cite{Binh_Toan_6_tap_1}, VD26, p. 25]
	2 số tự nhiên $a,2a$ đều có tổng các chữ số bằng $k$. Chứng minh $a\divby9$.
\end{baitoan}

\begin{baitoan}[\cite{Binh_Toan_6_tap_1}, VD27, p. 25]
	Chứng minh số gồm $27$ chữ số $1$ thì chia hết cho $27$.
\end{baitoan}

\begin{baitoan}[\cite{Binh_Toan_6_tap_1}, VD28, p. 26]
	Tìm $a\in\mathbb{N}$ nhỏ nhất sao cho tổng các chữ số của $a$ \& tổng các chữ số của $a + 1$ đều chia hết cho $5$.
\end{baitoan}

\begin{baitoan}[\cite{Binh_Toan_6_tap_1}, VD29, p. 26]
	Cho số tự nhiên $\overline{ab}$ bằng $3$ lần tích các chữ số của nó. (a) Chứng minh $b\divby a$. (b) Giả sử $b = ka$, với $k\in\mathbb{N}$. Chứng minh $k$ là ước của $10$. (c) Tìm các số $\overline{ab}$ thỏa mãn.
\end{baitoan}

\begin{baitoan}[\cite{Binh_Toan_6_tap_1}, VD30, p. 27]
	Tìm $a\in\mathbb{N}$ có 2 chữ số biết $a$ chia hết cho tích các chữ số của nó.
\end{baitoan}

\begin{baitoan}[\cite{Binh_Toan_6_tap_1}, VD31, p. 27]
	Cho $a\in\mathbb{N}$. Gọi $b$ là tổng các chữ số của $a$, $c$ là tổng các chữ số của $b$. Biết $a + 3b + 6c = 1395$. (a) Chứng minh $a\divby9$. (b) Tìm $a$.
\end{baitoan}

\begin{baitoan}[\cite{Binh_Toan_6_tap_1}, 142., p. 27]
	Cho $a = 13! - 11!$. Tìm số dư khi chia $a$ cho $2,5,155$.
\end{baitoan}

\begin{baitoan}[\cite{Binh_Toan_6_tap_1}, 143., p. 27]
	Tổng các số tự nhiên từ $1$ đến $154$ có chia hết cho $2,5$ không?
\end{baitoan}

\begin{baitoan}[\cite{Binh_Toan_6_tap_1}, 144., p. 27]
	Cho $a = \sum_{i=0}^9 11^i = 1 + 11 + \cdots + 11^8 + 11^9$. Chứng minh $a\divby5$.
\end{baitoan}

\begin{baitoan}[\cite{Binh_Toan_6_tap_1}, 145., p. 27]
	Chứng minh $n^2 + n + 6\not{\divby}\ 5$, $\forall n\in\mathbb{N}$.
\end{baitoan}

\begin{baitoan}[\cite{Binh_Toan_6_tap_1}, 146., p. 28]
	Trong các số tự nhiên nhỏ hơn $1000$, có bao nhiêu số chia hết cho $2$ nhưng không chia hết cho $5$?
\end{baitoan}

\begin{baitoan}[\cite{Binh_Toan_6_tap_1}, 147., p. 28]
	Dùng cả $10$ chữ số từ $0$--$9$, viết thành số tự nhiên: (a) nhỏ nhất chia hết cho $4$. (b) lớn nhất chia hết cho $4$.
\end{baitoan}

\begin{baitoan}[\cite{Binh_Toan_6_tap_1}, 148., p. 28]
	Dùng $10$ chữ số khác nhau, viết số chia hết cho $8$ có $10$ chữ số sao cho số đó có giá trị: (a) lớn nhất. (b) nhỏ nhất.
\end{baitoan}

\begin{baitoan}[\cite{Binh_Toan_6_tap_1}, 149., p. 28]
	Dùng $10$ chữ số khác nhau, viết số chia hết cho $25$ có $10$ chữ số sao cho số đó có giá trị: (a) nhỏ nhất. (b) lớn nhất.
\end{baitoan}

\begin{baitoan}[\cite{Binh_Toan_6_tap_1}, 150., p. 28]
	Tìm 2 số tự nhiên liên tiếp nhỏ nhất sao cho tổng các chữ số của mỗi số đều: (a) chia hết cho $8$. (b) chia hết cho $17$.
\end{baitoan}

\begin{baitoan}[\cite{Binh_Toan_6_tap_1}, 151., p. 28]
	Tìm các số tự nhiên chia cho $4$ dư $1$, còn chia cho $25$ thì dư $3$.
\end{baitoan}

\begin{baitoan}[\cite{Binh_Toan_6_tap_1}, 152., p. 28]
	Tìm các số tự nhiên chia cho $8$ dư $3$, còn chia cho $125$ thì dư $12$.
\end{baitoan}

\begin{baitoan}[\cite{Binh_Toan_6_tap_1}, 153., p. 28]
	Có phép trừ 2 số tự nhiên nào mà số trừ gấp 3 lần hiệu \& số bị trừ bằng $1030$ không?
\end{baitoan}

\begin{baitoan}[\cite{Binh_Toan_6_tap_1}, 154., p. 28]
	Điền các chữ số thích hợp vào dấu $\star$ sao cho: (a) $521\star\divby8$. (b) $2\star8\star7\divby9$, biết chữ số hàng chục lớn hơn chữ số hàng nghìn là $2$.
\end{baitoan}

\begin{baitoan}[\cite{Binh_Toan_6_tap_1}, 155., p. 28]
	Tìm 2 chữ số $a,b$ sao cho: (a) $a - b = 4,\overline{7a5b1}\divby3$. (b) $a - b = 6,\overline{4a7} + \overline{1b5}\divby9$.
\end{baitoan}

\begin{baitoan}[\cite{Binh_Toan_6_tap_1}, 156., p. 28]
	Tìm số tự nhiên có 3 chữ số, chia hết cho $5,9$, biết chữ số hàng chục bằng trung bình cộng của 2 chữ số kia.
\end{baitoan}

\begin{baitoan}[\cite{Binh_Toan_6_tap_1}, 157., p. 28]
	Tìm 2 số tự nhiên chia hết cho $9$, biết: (a) Tổng của chúng bằng $\star657$ \& hiệu của chúng bằng $5\star91$. (b) Tổng của chúng bằng $513\star$ \& số lớn gấp đôi số nhỏ.
\end{baitoan}

\begin{baitoan}[\cite{Binh_Toan_6_tap_1}, 158., p. 28]
	An làm phép tính trừ trong đó số bị trừ là số có 3 chữ số, số trừ là số gồm chính 3 chữ số ấy viết theo thứ tự ngược lại. An tính được hiệu bằng $188$. Chứng minh An đã tính sai.
\end{baitoan}

\begin{baitoan}[\cite{Binh_Toan_6_tap_1}, 159., p. 28]
	Tìm $a\in\mathbb{N}$ có 3 chữ số, chia hết cho $45$, biết hiệu giữa số đó \& số gồm chính 3 chữ số ấy viết theo thứ tự ngược lại bằng $297$.
\end{baitoan}

\begin{baitoan}[\cite{Binh_Toan_6_tap_1}, 160., p. 28]
	Tìm $a\in\mathbb{N}$  mà khi xóa chữ số tận cùng của nó thì số ấy giảm đi $1415$ đơn vị.
\end{baitoan}

\begin{baitoan}[\cite{Binh_Toan_6_tap_1}, 161., p. 28]
	1 số tự nhiên gọi là {\rm số hào hiệp} nếu nó chia hết cho mỗi chữ số của nó \& chia hết cho tổng các chữ số của nó, e.g., $12$ là số hào hiệp, $15$ không là số hào hiệp. Tìm số hào hiệp nhỏ nhất chia hết cho $11$.
\end{baitoan}

\begin{baitoan}[\cite{Binh_Toan_6_tap_1}, 162., p. 28]
	Tìm số tự nhiên nhỏ nhất tạo thành bởi 2 chữ số $3,7$ mà chia hết cho $3$ \& chia hết cho $7$.
\end{baitoan}

\begin{baitoan}[\cite{Binh_Toan_6_tap_1}, 163., p. 29]
	Tìm $x\in\mathbb{N}$ biết tổng các chữ số của $x$ bằng $y$, tổng các chữ số của $y$ bằng $z$ \& $x + y + z = 60$.
\end{baitoan}

\begin{baitoan}[\cite{Binh_Toan_6_tap_1}, 164., p. 29]
	Tìm $a\in\mathbb{N}$ biết $a + 6b + 9c = 1551$, trong đó $b$ là tổng các chữ số của $a$, $c$ là tổng các chữ số của $b$.
\end{baitoan}

\begin{baitoan}[\cite{Binh_Toan_6_tap_1}, 165., p. 29]
	Chứng minh: (a) $10^{28} + 8\divby72$. (b) $8^8 + 2^{20}\divby17$.
\end{baitoan}

\begin{baitoan}[\cite{Binh_Toan_6_tap_1}, 166., p. 29]
	(a) Cho $A = \sum_{i=1}^{60} 2^i = 2 + 2^2 + \cdots + 2^{60}$. Chứng minh $A$ chia hết cho $3,7,15$. (b) Cho $B = 3 + 3^3 + 3^5 + \cdots + 3^{1991}$. Chứng minh $B$ chia hết cho $13,41$.
\end{baitoan}

\begin{baitoan}[\cite{Binh_Toan_6_tap_1}, 167., p. 29]
	Chứng minh: (a) $2n + \underbrace{1\ldots1}_{n}\divby3$. (b) $10^n + 18n - 1\divby27$. (c) $10^n + 72n -1\divby81$.
\end{baitoan}

\begin{baitoan}[\cite{Binh_Toan_6_tap_1}, 168., p. 29]
	Chứng minh: (a) Số gồm $81$ chữ số $1$ thì chia hết cho $81$. (b) Số gồm $27$ nhóm chữ số $10$ thì chia hết cho $27$.
\end{baitoan}

\begin{baitoan}[\cite{Binh_Toan_6_tap_1}, 169., p. 29]
	2 số tự nhiên $a,4a$ có tổng các chữ số bằng nhau. Chứng minh $a\divby3$.
\end{baitoan}

\begin{baitoan}[\cite{Binh_Toan_6_tap_1}, 170., p. 29]
	(a) Tổng các chữ số của $3^{100}$ viết trong hệ thập phân có thể bằng $459$ không? (b) Tổng các chữ số của $3^{1000}$ là $a$, tổng các chữ số của $a$ là $b$, tổng các chữ số của $b$ là $c$. Tính $c$.
\end{baitoan}

\begin{baitoan}[\cite{Binh_Toan_6_tap_1}, 171., p. 29]
	Cho $a,b\in\mathbb{N}$ tùy ý có số dư trong phép chia cho $9$ lần lượt là $r_1,r_2$. Chứng minh $r_1r_2$ \& $ab$ có cùng số dư trong phép chia cho $9$.
\end{baitoan}

\begin{baitoan}[\cite{Binh_Toan_6_tap_1}, 172., p. 29]
	1 số tự nhiên chia hết cho $4$ có 3 chữ số đều chẵn, khác nhau, \& khác $0$. Chứng minh tồn tại cách đổi vị trí các chữ số để được 1 số mới chia hết cho $4$.
\end{baitoan}

\begin{baitoan}[\cite{Binh_Toan_6_tap_1}, 173., p. 29]
	Tìm số $\overline{abcd}$ , biết số đó chia hết cho tích 2 số $\overline{ab}\cdot\overline{cd}$.
\end{baitoan}

\begin{baitoan}[\cite{Binh_Toan_6_tap_1}, 174., p. 29]
	Tìm $a\in\mathbb{N}$ có 5 chữ số, biết số đó bằng $45$ lần tích các chữ số của $a$.
\end{baitoan}

\begin{baitoan}[\cite{Binh_Toan_6_tap_1}, 175., p. 29]
	1 cửa hàng có $6$ hòm hàng với khối lượng {\rm316 kg, 327 kg, 336 kg, 338 kg, 349 kg, 351 kg}. Cửa hàng đó đã bán $5$ hòm, trong đó khối lượng hàng bán buổi sáng gấp $4$ lần khối lượng hàng bán buổi chiều. Hỏi hòm còn lại là hòm nào?
\end{baitoan}

\begin{baitoan}[\cite{Binh_Toan_6_tap_1}, 176., p. 29]
	Từ 4 chữ số $1,2,3,4$, lập tất cả các số tự nhiên có 4 chữ số gồm cả 4 chữ số ấy. Trong các số đó, có tồn tại 2 số nào mà 1 số chia hết cho số còn lại không?
\end{baitoan}

\begin{baitoan}[\cite{Binh_Toan_6_tap_1}, 177., p. 29]
	Chứng minh trong tất cả các số tự nhiên khác nhau có 7 chữ số lập bởi cả 7 chữ số $1,2,3,4,5,6,7$, không có 2 số nào mà 1 số chia hết cho số còn lại.
\end{baitoan}

\begin{baitoan}[\cite{Binh_Toan_6_tap_1}, 178., pp. 29--30]
	Cho 4 số tự nhiên không nhất thiết khác nhau. Nếu cộng 3 số trong 4 số đó, ta được 4 tổng, trong đó 3 tổng là $54,55,59$, tổng thứ 4 bằng 1 trong 3 tổng trên. (a) Tính tổng thứ 4. (b) Tính tổng của 4 số đã cho. (c) Tính 4 số đã cho.
\end{baitoan}

\begin{baitoan}
	Tìm \& chứng minh dấu hiệu chia hết cho $11$.
\end{baitoan}

\begin{baitoan}[Tích 2 số tự nhiên liên tiếp]
	Cho $n\in\mathbb{N}$. Xét tích 2 số tự nhiên liên tiếp $A_2(n) = n(n + 1)$. (a) Chứng minh $A_2(n)\divby2$, $\forall n\in\mathbb{N}$. (b) Chứng minh tổng của $n$ số chẵn dương đầu tiên bằng $A_2(n)$. (c) Tìm điều kiện của $n$ để $A_2(n)$ chia hết cho $4,8,16,\ldots,2^m$ với $m\in\mathbb{N}^\star$.
\end{baitoan}

\begin{baitoan}[Tích 2 số tự nhiên chẵn liên tiếp]
	Cho $n\in\mathbb{N}$. Xét tích 2 số tự nhiên chẵn liên tiếp $E_2(n) = 2n(2n + 2)$. (a) Chứng minh $E_2(n)\divby8$, $\forall n\in\mathbb{N}$. (b) Chứng minh tổng của $n$ số chẵn dương đầu tiên bằng $4E_2(n)$. (c) Tìm điều kiện của $n$ để $E_2(n)$ chia hết cho $2^m$ với $m\in\mathbb{N}^\star$.
\end{baitoan}

\begin{baitoan}[Tích 3 số tự nhiên liên tiếp]
	Cho $n\in\mathbb{N}$. Xét tích 3 số tự nhiên liên tiếp $A_3(n) = n(n + 1)(n + 2)$. (a) Chứng minh $A_3(n)\divby6$ với mọi $n$ lẻ. (b) Chứng minh $A_3(n)\divby24$ với mọi $n$ chẵn. (c${}^\star$) Tìm điều kiện của $n$ để $A_3(n)$ chia hết cho $2^a\cdot3^b$ với $a,b\in\mathbb{N}^\star$.
\end{baitoan}

\begin{baitoan}[Tích 4 số tự nhiên liên tiếp]
	Cho $n\in\mathbb{N}$. Xét tích 4 số tự nhiên liên tiếp $A_4(n) = n(n + 1)(n + 2)(n + 3)$. (a) Chứng minh $A_4(n)\divby24$, $\forall n\in\mathbb{N}$. (b${}^\star$) Tìm điều kiện của $n$ để $A_4(n)$ chia hết cho $2^a\cdot3^b$ với $a,b\in\mathbb{N}^\star$.
\end{baitoan}

%------------------------------------------------------------------------------%

\section{Miscellaneous}

%------------------------------------------------------------------------------%

\printbibliography[heading=bibintoc]

\end{document}