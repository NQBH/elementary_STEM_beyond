\documentclass{article}
\usepackage[backend=biber,natbib=true,style=alphabetic,maxbibnames=50]{biblatex}
\addbibresource{/home/nqbh/reference/bib.bib}
\usepackage[utf8]{vietnam}
\usepackage{tocloft}
\renewcommand{\cftsecleader}{\cftdotfill{\cftdotsep}}
\usepackage[colorlinks=true,linkcolor=blue,urlcolor=red,citecolor=magenta]{hyperref}
\usepackage{amsmath,amssymb,amsthm,float,graphicx,mathtools,tipa}
\usepackage{enumitem}
\setlist{leftmargin=4mm}
\allowdisplaybreaks
\newtheorem{assumption}{Assumption}
\newtheorem{baitoan}{}
\newtheorem{cauhoi}{Câu hỏi}
\newtheorem{conjecture}{Conjecture}
\newtheorem{corollary}{Corollary}
\newtheorem{dangtoan}{Dạng toán}
\newtheorem{definition}{Definition}
\newtheorem{dinhly}{Định lý}
\newtheorem{dinhnghia}{Định nghĩa}
\newtheorem{example}{Example}
\newtheorem{ghichu}{Ghi chú}
\newtheorem{hequa}{Hệ quả}
\newtheorem{hypothesis}{Hypothesis}
\newtheorem{lemma}{Lemma}
\newtheorem{luuy}{Lưu ý}
\newtheorem{nhanxet}{Nhận xét}
\newtheorem{notation}{Notation}
\newtheorem{note}{Note}
\newtheorem{principle}{Principle}
\newtheorem{problem}{Problem}
\newtheorem{proposition}{Proposition}
\newtheorem{question}{Question}
\newtheorem{remark}{Remark}
\newtheorem{theorem}{Theorem}
\newtheorem{vidu}{Ví dụ}
\usepackage[left=1cm,right=1cm,top=5mm,bottom=5mm,footskip=4mm]{geometry}
\def\labelitemii{$\circ$}
\DeclareRobustCommand{\divby}{%
	\mathrel{\vbox{\baselineskip.65ex\lineskiplimit0pt\hbox{.}\hbox{.}\hbox{.}}}%
}

\title{Problem: Divisibility -- Bài Tập: Tính Chia Hết}
\author{Nguyễn Quản Bá Hồng\footnote{Independent Researcher, Ben Tre City, Vietnam\\e-mail: \texttt{nguyenquanbahong@gmail.com}; website: \url{https://nqbh.github.io}.}}
\date{\today}

\begin{document}
\maketitle
\begin{abstract}
	Last updated version: \href{https://github.com/NQBH/elementary_STEM_beyond/blob/main/elementary_mathematics/grade_6/natural/divisibility/problem/NQBH_divisibility_problem.pdf}{GitHub{\tt/}NQBH{\tt/}hobby{\tt/}elementary mathematics{\tt/}grade 6{\tt/}natural{\tt/}divisibility{\tt/}problem[pdf]}.\footnote{\textsc{url}: \url{https://github.com/NQBH/elementary_STEM_beyond/blob/main/elementary_mathematics/grade_6/natural/divisibility/problem/NQBH_divisibility_problem.pdf}.} [\href{https://github.com/NQBH/elementary_STEM_beyond/blob/main/elementary_mathematics/grade_6/natural/divisibility/problem/NQBH_divisibility_problem.tex}{\TeX}]\footnote{\textsc{url}: \url{https://github.com/NQBH/elementary_STEM_beyond/blob/main/elementary_mathematics/grade_6/natural/divisibility/problem/NQBH_divisibility_problem.tex}.}. 
\end{abstract}
\tableofcontents

%------------------------------------------------------------------------------%

\begin{itemize}\sf
	\item \textbf{divisible} [a] {\tt/}\textipa{d@'vIz@bl}{\tt/} [not before noun] \textit{divisible (by something)} that can be divided, usually with nothing remaining. {\sc opposite}: \textbf{indivisible}.
	\item \textbf{divisibility}  [n] [uncountable] {\tt/}\textipa{d@'vIz@bIl@ti}{\tt/}.
\end{itemize}

\section{Divisibility of Sum -- Tính Chất Chia Hết của Tổng}

\begin{baitoan}[\cite{Binh_boi_duong_Toan_6_tap_1}, H1, p. 24]
	{\rm Đ{\tt/}S? (a) $127\cdot5 + 40\divby5$. (b) $13\cdot48 + 12 + 17\divby6$. (c) $3\cdot300 - 12\divby9$. (d) $49 + 62\cdot7\divby7$.}
\end{baitoan}

\begin{baitoan}[\cite{Binh_boi_duong_Toan_6_tap_1}, H2, p. 24]
	Khi chia số $a$ cho số $b$, $a,b\in\mathbb{N}^\star$, $a > b$ ta được số dư là $r$. Khi đó: {\sf A.} $a + r\divby b$. {\sf B.} $a - r\divby b$. {\sf C.} $a + b\divby r$. {\sf D.} $a - b\divby r$.
\end{baitoan}

\begin{baitoan}[\cite{Binh_boi_duong_Toan_6_tap_1}, H3, p. 24]
	Tìm số tự nhiên $x$ có 1 chữ số thỏa $121 + x\divby11$.
\end{baitoan}

\begin{baitoan}[\cite{Binh_boi_duong_Toan_6_tap_1}, Ví dụ 1, p. 25]
	Không tính các tổng \& hiệu, xét xem các tổng \& hiệu sau có chia hết cho $12$ không? Vì sao? (a) $600\cdot37 - 144$. (b) $96 + 34 + 48$.
\end{baitoan}

\begin{baitoan}[\cite{Binh_boi_duong_Toan_6_tap_1}, Ví dụ 2, p. 25]
	Không tính ra kết quả, xét xem tổng $84 + 37 + 23$ có chia hết cho $12$ không? Vì sao?
\end{baitoan}

\begin{baitoan}[\cite{Binh_boi_duong_Toan_6_tap_1}, Ví dụ 3, p. 25]
	Chứng minh trong 3 số tự nhiên liên tiếp có 1 số chia hết cho $3$.
\end{baitoan}

\begin{baitoan}[\cite{Binh_boi_duong_Toan_6_tap_1}, Mở rộng Ví dụ 4, p. 25]
	Với $n\in\mathbb{N}^\star$ bất kỳ. Chứng minh: (a) Trong $n$ số tự nhiên liên tiếp luôn có 1 số chia hết cho $n$. (b) Tích của $n$ số tự nhiên liên tiếp là 1 số chia hết cho $n$.
\end{baitoan}

\begin{baitoan}[\cite{Binh_boi_duong_Toan_6_tap_1}, Ví dụ 4, p. 26]
	Chứng minh tổng của 3 số tự nhiên liên tiếp là 1 số chia hết cho $3$.
\end{baitoan}

\begin{baitoan}
	Với $n\in\mathbb{N}^\star$ bất kỳ. Liệu tổng của $n$ số tự nhiên liên tiếp có chia hết cho $n$ không?
\end{baitoan}

\begin{baitoan}[\cite{Binh_boi_duong_Toan_6_tap_1}, Ví dụ 5, p. 26]
	Chứng minh: (a) $\overline{ab} - \overline{ba}\divby9$ với $a > b$. (b) Nếu $\overline{ab} + \overline{cd}\divby11$ thì $\overline{abcd}\divby11$.
\end{baitoan}

\begin{baitoan}[\cite{Binh_boi_duong_Toan_6_tap_1}, Ví dụ 6, p. 26]
	Cho $A = 15 + 30 + 37 + x$ với $x\in\mathbb{N}$. Tìm điều kiện của $x$ để: (a) $A\divby3$. (b) $A\not{\divby}\ 9$.
\end{baitoan}

\begin{baitoan}[\cite{Binh_boi_duong_Toan_6_tap_1}, Ví dụ 7, p. 26]
	Tìm $n\in\mathbb{N}$ để: (a) $n + 4\divby n$. (b) $5n - 6\divby n$ với $n > 1$. (c) $143 - 12n\divby n$ với $n < 12$.
\end{baitoan}

\begin{baitoan}[\cite{Binh_boi_duong_Toan_6_tap_1}, Ví dụ 8, p. 27]
	Tìm $n\in\mathbb{N}$ để: (a) $n + 9\divby n + 4$. (b) $3n + 40\divby n + 4$. (c) $5n + 2\divby2n + 9$.
\end{baitoan}

\begin{baitoan}[\cite{Binh_boi_duong_Toan_6_tap_1}, 3.1., p. 27]
	Cho $A = 2\cdot5\cdot9\cdot13 + 84$. Hỏi $A$ có chia hết cho $3$, cho $6$, cho $9$, cho $13$ không? Vì sao?
\end{baitoan}

\begin{baitoan}[\cite{Binh_boi_duong_Toan_6_tap_1}, 3.2., p. 27]
	Chứng minh tổng 5 số chẵn liên tiếp là 1 số chia hết cho $10$.
\end{baitoan}

\begin{baitoan}[\cite{Binh_boi_duong_Toan_6_tap_1}, 3.3., p. 27]
	Khi chia số tự nhiên $a$ cho $27$, ta được số dư là $15$. Hỏi số $a$ có chia hết cho $3$, cho $9$ không? Vì sao?
\end{baitoan}

\begin{baitoan}[\cite{Binh_boi_duong_Toan_6_tap_1}, 3.4., p. 27]
	Chứng minh mọi số tự nhiên có 3 chữ số giống nhau đều chia hết cho $37$.
\end{baitoan}

\begin{baitoan}[\cite{Binh_boi_duong_Toan_6_tap_1}, 3.5., p. 28]
	Chứng minh: (a) $\sum_{i=0}^{101} 5^i = 1 + 5 + 5^2 + 5^3 + \cdots + 5^{101}\divby6$. (b) $\sum_{i=1}^{100} 2^i = 2 + 2^2 + 2^3 + \cdots + 2^{100}$ vừa chia hết cho $31$, vừa chia hết cho $5$.
\end{baitoan}

\begin{baitoan}[\cite{Binh_boi_duong_Toan_6_tap_1}, 3.6., p. 28]
	Chứng minh: (a) Nếu $\overline{abc} - \overline{def}\divby11$ thì $\overline{abcdef}\divby11$. (b) Nếu $\overline{abc}\divby8$ thì $4a + 2b + c\divby8$.
\end{baitoan}

\begin{baitoan}[\cite{Binh_boi_duong_Toan_6_tap_1}, 3.7., p. 28]
	Tìm chữ số $a$ biết $\overline{21a21a21a}\divby31$.
\end{baitoan}

\begin{baitoan}[\cite{Binh_boi_duong_Toan_6_tap_1}, 3.8., p. 28]
	Tìm $n\in\mathbb{N}$ sao cho: (a) $n + 21\divby n$. (b) $18 - 2n\divby n$ với $n < 9$. (c) $6n - 9\divby n$ với $n\ge2$.
\end{baitoan}

\begin{baitoan}[\cite{Binh_boi_duong_Toan_6_tap_1}, 3.9., p. 28]
	Tìm $n\in\mathbb{N}$ sao cho: (a) $n + 15\divby n - 3$ với $n > 5$. (b) $18 - 2n\divby n + 3$ với $n\le9$. (c) $3n + 13\divby2n + 3$ với $n\ge1$.
\end{baitoan}

\begin{baitoan}[\cite{Binh_boi_duong_Toan_6_tap_1}, 3.10., p. 28]
	Cho $a,b\in\mathbb{N}$. Chứng minh nếu $7a + 2b$ \& $31a + 9b$ cùng chia hết cho $2015$ thì $a,b$ cũng chia hết cho $2015$.
\end{baitoan}

\begin{baitoan}[\cite{Binh_boi_duong_Toan_6_tap_1}, p. 28]
	Chứng minh: (a) Tích 2 số tự nhiên liên tiếp là 1 số chẵn. (b) Tích 3 số tự nhiên liên tiếp luôn chia hết cho $6$. (c) Tích của $n$ số tự nhiên liên tiếp bất kỳ luôn chia hết cho $n! = \prod_{i=1}^n i = 1\cdot2\cdot3\cdots n$, $\forall n\in\mathbb{N}^\star$.
\end{baitoan}

\begin{baitoan}[\cite{Binh_boi_duong_Toan_6_tap_1}, p. 28]
	Với $n\in\mathbb{N}^\star$. (a) Khi nào thì tổng của $n$ số tự nhiên liên tiếp bất kỳ chia hết cho $n$? (b) Khi nào thì tổng của $n$ số tự nhiên chẵn liên tiếp bất kỳ chia hết cho $n$? (c) Khi nào thì tổng của $n$ số tự nhiên lẻ liên tiếp bất kỳ chia hết cho $n$?
\end{baitoan}

%------------------------------------------------------------------------------%

\section{Divisibility Rule -- Dấu Hiệu Chia Hết}

\begin{baitoan}[\cite{Binh_boi_duong_Toan_6_tap_1}, H1, p. 29]
	Nối cột để được kết quả đúng.
	\begin{table}[H]
		\centering
		\begin{tabular}{|l|l|}
			\hline
			(a) $230 + 175$ & (1) chia hết cho 2 nhưng không chia hết cho 5. \\
			\hline
			(b) $2070 - 590$ & (2) chia hết cho 5 nhưng không chia hết cho 2. \\
			\hline
			(c) $747 + 350$ & (3) chia hết cho cả 2 \& 5. \\
			\hline
			& (4) không chia hết cho cả 2 \& 5. \\
			\hline
		\end{tabular}
	\end{table}
\end{baitoan}

\begin{baitoan}[\cite{Binh_boi_duong_Toan_6_tap_1}, H2, p. 30]
	Khi giải bài toán: ``Thêm 1 chữ số vào bên phải \& 1 chữ số vào bên trái số $2015$ để được 1 số mới chia hết cho cả $2,3,5$.'' Tìm kết quả sai: {\sf A.} $120150$. {\sf B.} $420150$. {\sf C.} $620150$. {\sf D.} $720150$.
\end{baitoan}

\begin{baitoan}[\cite{Binh_boi_duong_Toan_6_tap_1}, Mở rộng H2, p. 29]
	Thêm 1 chữ số vào bên phải \& 1 chữ số vào bên trái số $2015$ để được 1 số mới chia hết cho cả $2,3,5$. Tìm tất cả các cặp số có thể thêm vào.
\end{baitoan}

\begin{baitoan}[\cite{Binh_boi_duong_Toan_6_tap_1}, H3, p. 30]
	Trong khoảng từ $1010$ đến $1975$ có bao nhiêu số chia hết cho $3$?
\end{baitoan}

\begin{baitoan}[\cite{Binh_boi_duong_Toan_6_tap_1}, H4, p. 30]
	Thay các chữ cái khác nhau bởi các chữ số khác nhau: $\rm HANOI + HANOI + HANOI = \overline{TT221}$.
\end{baitoan}

\begin{baitoan}[\cite{Binh_boi_duong_Toan_6_tap_1}, Ví dụ 1, p. 30]
	2 bạn Egg \& Chicken đi mua $18$ gói bánh \& $12$ gói kẹo để chuẩn bị cho buổi liên hoan lớp. Egg đưa cho cô bán hàng 3 tờ tiền, mỗi tờ có mệnh giá $50000$ đồng \& được trả lại $22000$ đồng. Thấy vậy, Chicken liền nói: ``Cô tính sai rồi!'' Chicken đúng hay sai? Vì sao?
\end{baitoan}

\begin{baitoan}[\cite{Binh_boi_duong_Toan_6_tap_1}, Ví dụ 2, p. 30]
	Chứng minh $(n + 29)(n + 30)\divby2$, $\forall n\in\mathbb{N}$.
\end{baitoan}

\begin{baitoan}[Tính chia hết cho 2 của 1 tích]
	(a) Với $a,b\in\mathbb{N}$ thỏa điều kiện nào thì $(n + a)(n + b)\divby2$, $\forall n\in\mathbb{N}$? (b) Với $a,b,c\in\mathbb{N}$ thỏa điều kiện nào thì $(n + a)(n + b)\divby2$, $\forall n\in\mathbb{N}$? (c) Cho $n\in\mathbb{N}^\star$. Với $a_1,a_2,\ldots,a_n\in\mathbb{N}$ thỏa điều kiện nào thì $\prod_{i=1}^n (m + a_i) = (m + a_1)(m + a_2)\cdots(m + a_n)\divby2$, $\forall n\in\mathbb{N}$?
\end{baitoan}

\begin{baitoan}[Tính chia hết cho 3 của 1 tích]
	(a) Với $a,b\in\mathbb{N}$ thỏa điều kiện nào thì $(n + a)(n + b)\divby3$, $\forall n\in\mathbb{N}$? (b) Với $a,b,c\in\mathbb{N}$ thỏa điều kiện nào thì $(n + a)(n + b)\divby3$, $\forall n\in\mathbb{N}$? (c) Cho $n\in\mathbb{N}^\star$. Với $a_1,a_2,\ldots,a_n\in\mathbb{N}$ thỏa điều kiện nào thì $\prod_{i=1}^n (m + a_i) = (m + a_1)(m + a_2)\cdots(m + a_n)\divby3$, $\forall n\in\mathbb{N}$?
\end{baitoan}

\begin{baitoan}[\cite{Binh_boi_duong_Toan_6_tap_1}, Ví dụ 3, p. 31]
	Chứng minh $39^{2015} + 11^{2016}\divby10$.
\end{baitoan}

\begin{baitoan}
	Với $a,b\in\mathbb{N}$ thỏa điều kiện nào thì: (a) $39^a + 11^b\divby10$? (b) $\overline{a_ma_{m-1}\ldots a_19}^a + \overline{b_nb_{n-1}\ldots b_11}^b\divby10$ với $a_i,b_j\in\{0,1,2,\ldots,9\}$, $\forall i = 1,2,\ldots,m$, $\forall j = 1,2,\ldots,n$, $a_nb_m\ne0$?
\end{baitoan}

\begin{baitoan}[\cite{Binh_boi_duong_Toan_6_tap_1}, Ví dụ 4, p. 31]
	Thay dấu $+$ hoặc $-$ vào các dấu $\star$ trong dãy tính sau để được kết quả là 1 số chia hết cho $2$: $10\star9\star8\star7\star6\star5\star4\star3\star2\star1$.
\end{baitoan}

\begin{baitoan}
	Thay dấu $+$ hoặc $-$ vào các dấu $\star$ trong dãy tính sau để được kết quả là 1 số chia hết cho $2$: $n\star(n - 1)\star(n - 2)\star\cdots3\star2\star1$ với $n\in\mathbb{N}$.
\end{baitoan}

\begin{baitoan}[\cite{Binh_boi_duong_Toan_6_tap_1}, Ví dụ 5, p. 32]
	Viết các số tự nhiên liên tiếp từ $10$ đến $99$ ta được số $A$. Hỏi $A$ có chia hết cho $9$ không? Vì sao?
\end{baitoan}

\begin{baitoan}
	Cho $n\in\mathbb{N}^\star$. Viết các số tự nhiên liên tiếp từ $10^n$ (số tự nhiên nhỏ nhất có $n + 1$ chữ số) đến $10^{n+1} - 1$ (số tự nhiên lớn nhất có $n + 1$ chữ số) ta được số $A$. Hỏi $A$ có chia hết cho $9$ không? Vì sao?
\end{baitoan}

\begin{baitoan}[\cite{Binh_boi_duong_Toan_6_tap_1}, Ví dụ 6, p. 32]
	Tìm 2 chữ số $x,y$ biết: (a) $\overline{38x5y}$ chia hết cho $2,5,9$. (b) $\overline{12x3y}\divby45$.
\end{baitoan}

\begin{baitoan}[\cite{Binh_boi_duong_Toan_6_tap_1}, Ví dụ 7, p. 32]
	Thay $a,b$ bằng các chữ số thích hợp để số $ \overline{2a83b}$ chia hết cho $3$ \& chia cho $5$ dư $1$.
\end{baitoan}

\begin{baitoan}[\cite{Binh_boi_duong_Toan_6_tap_1}, Ví dụ 8, p. 33]
	Tìm 2 số tự nhiên chia hết cho $9$, biết tổng của chúng bằng $\overline{35\star1}$ \& số lớn gấp đôi số bé.
\end{baitoan}

\begin{baitoan}[\cite{Binh_boi_duong_Toan_6_tap_1}, Ví dụ 9, p. 33]
	Tìm chữ số $a$ sao cho $\overline{95a14}\divby11$.
\end{baitoan}

\begin{baitoan}[\cite{Binh_boi_duong_Toan_6_tap_1}, 4.1., p. 33]
	Từ 3 trong 5 chữ số $2,5,7,8,0$, ghép thành số có 3 chữ số khác nhau thỏa mãn 1 trong các điều kiện: (a) Là số lớn nhất chia hết cho $2$. (b) Là số nhỏ nhất chia hết cho $2$. (c) Là số lớn nhất chia hết cho $5$. (d) Là số nhỏ nhất chia hết cho $5$. (e) Là số lớn nhất chia hết cho $9$. (f) Là số nhỏ nhất chia hết cho $9$. (g) Là số lớn nhất chia hết cho $3$. (h) Là số nhỏ nhất chia hết cho $3$.
\end{baitoan}

\begin{baitoan}[\cite{Binh_boi_duong_Toan_6_tap_1}, 4.2., p. 33]
	Dùng 3 trong 4 số $,2,4,6,8$, viết tất cả các số tự nhiên có 3 chữ số chia hết cho cả 3 số $2,3,9$.
\end{baitoan}

\begin{baitoan}[\cite{Binh_boi_duong_Toan_6_tap_1}, 4.3., p. 33]
	Có $10$ mẩu que lần lượt dài {\rm1 cm, 2cm, 3cm, $\ldots$, 10 cm}. Hỏi có thể dùng cả $10$ mẫu que đó để xếp thành 1 tam giác có 3 cạnh bằng nhau không?
\end{baitoan}

\begin{baitoan}[\cite{Binh_boi_duong_Toan_6_tap_1}, 4.4., p. 33]
	Chứng minh: (a) $10^{2015} + 8\divby18$. (b) $10^{21} + 20\divby6$.
\end{baitoan}

\begin{baitoan}[\cite{Binh_boi_duong_Toan_6_tap_1}, 4.5., p. 33]
	Chứng minh $(n + 11)(n + 12)\divby2$, $\forall n\in\mathbb{N}$.
\end{baitoan}

\begin{baitoan}[\cite{Binh_boi_duong_Toan_6_tap_1}, 4.6., p. 33]
	Chứng minh tích của 3 số tự nhiên chẵn liên tiếp chia hết cho $48$.
\end{baitoan}

\begin{baitoan}[\cite{Binh_boi_duong_Toan_6_tap_1}, 4.7., p. 33]
	Tìm số tự nhiên có 5 chữ số, các chữ số giống nhau, biết số đó chia cho $5$ dư $4$ \& chia hết cho $2$.
\end{baitoan}

\begin{baitoan}[\cite{Binh_boi_duong_Toan_6_tap_1}, 4.8., p. 34]
	Tìm 2 chữ số $x,y$ biết: (a) $\overline{2x98y}$ chia hết cho $2,3,5$. (b) $\overline{43xy5}\divby45$. (c) $\overline{21x7y}$ chia hết cho $5,18$.
\end{baitoan}

\begin{baitoan}[\cite{Binh_boi_duong_Toan_6_tap_1}, 4.9., p. 34]
	Tìm chữ số $a$ để $\overline{aaaaa96}$ chia hết cho cả $3$ \& $8$.
\end{baitoan}

\begin{baitoan}[\cite{Binh_boi_duong_Toan_6_tap_1}, 4.10., p. 34]
	Tìm chữ số $a$ để $\overline{1aaa1}\divby11$.
\end{baitoan}

\begin{baitoan}[\cite{Binh_boi_duong_Toan_6_tap_1}, 4.11., p. 34]
	Cho $a\in\mathbb{N}$. Đổi chỗ các chữ số của $a$ để được số $b$ gấp $3$ lần số $a$. Chứng minh $a\divby27$.
\end{baitoan}

\begin{baitoan}[\cite{Binh_boi_duong_Toan_6_tap_1}, 4.12., p. 34]
	Cho $n\in\mathbb{N}^\star$. Chứng minh: (a) $6^n - 1\divby5$. (b) $10^n + 18n - 1\divby27$.
\end{baitoan}

\begin{baitoan}[\cite{Binh_boi_duong_Toan_6_tap_1}, 4.13., p. 34]
	Tìm 2 chữ số $a,b$ sao cho: (a) $\overline{71ab}$ chia hết cho $9$, cho $2$, \& chia cho $5$ dư $3$. (b) $\overline{15a3b}$ chia hết cho $2$, chia hết cho $9$, \& chia cho $5$ dư $4$.
\end{baitoan}

\begin{baitoan}[\cite{Binh_boi_duong_Toan_6_tap_1}, 4.14., p. 34]
	Tìm 2 số tự nhiên liên tiếp có 2 chữ số, biết 1 số chia hết cho $4$, số kia chia hết cho $25$.
\end{baitoan}

\begin{baitoan}[\cite{Binh_boi_duong_Toan_6_tap_1}, 4.15., p. 34]
	Tìm số tự nhiên có 4 chữ số sao cho khi nhân số đó với $9$ ta được số mới gồm chính các chữ số của số ấy nhưng viết theo thứ tự ngược lại.
\end{baitoan}

\begin{baitoan}[\cite{Binh_boi_duong_Toan_6_tap_1}, 4.16., p. 34, Thái Lan]
	Nếu đem số $31513$ \& số $34369$ chia cho cùng 1 số có 3 chữ số thì 2 phép chia có số dư bằng nhau. Tìm số dư của 2 phép chia đó.
\end{baitoan}

\begin{baitoan}[\cite{Binh_boi_duong_Toan_6_tap_1}, 4.17., p. 34]
	Chứng minh hiệu của 1 số \& tổng các chữ số của nó chia hết cho $9$.
\end{baitoan}

\begin{dinhnghia}[Hàm tổng các chữ số]
	Ký hiệu $S(n)$ là tổng các chữ số của $n\in\mathbb{N}$.
\end{dinhnghia}

\begin{baitoan}[\cite{Binh_boi_duong_Toan_6_tap_1}, 4.18., p. 34]
	 Tìm $n\in\mathbb{N}$ biết $n + S(n) = 88$.
\end{baitoan}

\begin{baitoan}
	Tìm \& chứng minh dấu hiệu chia hết cho $11$.
\end{baitoan}

\begin{baitoan}[Tích 2 số tự nhiên liên tiếp]
	Cho $n\in\mathbb{N}$. Xét tích 2 số tự nhiên liên tiếp $A_2(n) = n(n + 1)$. (a) Chứng minh $A_2(n)\divby2$, $\forall n\in\mathbb{N}$. (b) Chứng minh tổng của $n$ số chẵn dương đầu tiên bằng $A_2(n)$. (c) Tìm điều kiện của $n$ để $A_2(n)$ chia hết cho $4,8,16,\ldots,2^m$ với $m\in\mathbb{N}^\star$.
\end{baitoan}

\begin{baitoan}[Tích 2 số tự nhiên chẵn liên tiếp]
	Cho $n\in\mathbb{N}$. Xét tích 2 số tự nhiên chẵn liên tiếp $E_2(n) = 2n(2n + 2)$. (a) Chứng minh $E_2(n)\divby8$, $\forall n\in\mathbb{N}$. (b) Chứng minh tổng của $n$ số chẵn dương đầu tiên bằng $4E_2(n)$. (c) Tìm điều kiện của $n$ để $E_2(n)$ chia hết cho $2^m$ với $m\in\mathbb{N}^\star$.
\end{baitoan}

\begin{baitoan}[Tích 3 số tự nhiên liên tiếp]
	Cho $n\in\mathbb{N}$. Xét tích 3 số tự nhiên liên tiếp $A_3(n) = n(n + 1)(n + 2)$. (a) Chứng minh $A_3(n)\divby6$ với mọi $n$ lẻ. (b) Chứng minh $A_3(n)\divby24$ với mọi $n$ chẵn. (c${}^\star$) Tìm điều kiện của $n$ để $A_3(n)$ chia hết cho $2^a\cdot3^b$ với $a,b\in\mathbb{N}^\star$.
\end{baitoan}

\begin{baitoan}[Tích 4 số tự nhiên liên tiếp]
	Cho $n\in\mathbb{N}$. Xét tích 4 số tự nhiên liên tiếp $A_4(n) = n(n + 1)(n + 2)(n + 3)$. (a) Chứng minh $A_4(n)\divby24$, $\forall n\in\mathbb{N}$. (b${}^\star$) Tìm điều kiện của $n$ để $A_4(n)$ chia hết cho $2^a\cdot3^b$ với $a,b\in\mathbb{N}^\star$.
\end{baitoan}

%------------------------------------------------------------------------------%

\section{Miscellaneous}

%------------------------------------------------------------------------------%

\printbibliography[heading=bibintoc]

\end{document}