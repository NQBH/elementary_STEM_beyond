\documentclass{article}
\usepackage[backend=biber,natbib=true,style=alphabetic,maxbibnames=50]{biblatex}
\addbibresource{/home/nqbh/reference/bib.bib}
\usepackage[utf8]{vietnam}
\usepackage{tocloft}
\renewcommand{\cftsecleader}{\cftdotfill{\cftdotsep}}
\usepackage[colorlinks=true,linkcolor=blue,urlcolor=red,citecolor=magenta]{hyperref}
\usepackage{amsmath,amssymb,amsthm,float,graphicx,mathtools}
\usepackage{enumitem}
\setlist{leftmargin=4mm}
\allowdisplaybreaks
\newtheorem{assumption}{Assumption}
\newtheorem{baitoan}{}
\newtheorem{cauhoi}{Câu hỏi}
\newtheorem{conjecture}{Conjecture}
\newtheorem{corollary}{Corollary}
\newtheorem{dangtoan}{Dạng toán}
\newtheorem{definition}{Definition}
\newtheorem{dinhly}{Định lý}
\newtheorem{dinhnghia}{Định nghĩa}
\newtheorem{example}{Example}
\newtheorem{ghichu}{Ghi chú}
\newtheorem{hequa}{Hệ quả}
\newtheorem{hypothesis}{Hypothesis}
\newtheorem{lemma}{Lemma}
\newtheorem{luuy}{Lưu ý}
\newtheorem{nhanxet}{Nhận xét}
\newtheorem{notation}{Notation}
\newtheorem{note}{Note}
\newtheorem{principle}{Principle}
\newtheorem{problem}{Problem}
\newtheorem{proposition}{Proposition}
\newtheorem{question}{Question}
\newtheorem{remark}{Remark}
\newtheorem{theorem}{Theorem}
\newtheorem{vidu}{Ví dụ}
\usepackage[left=1cm,right=1cm,top=5mm,bottom=5mm,footskip=4mm]{geometry}
\def\labelitemii{$\circ$}
\DeclareRobustCommand{\divby}{%
	\mathrel{\vbox{\baselineskip.65ex\lineskiplimit0pt\hbox{.}\hbox{.}\hbox{.}}}%
}

\title{Problem: Calculus on Set $\mathbb{N}$ of Naturals\\Bài Tập: Các Phép Tính Trên Tập Hợp Các Số Tự Nhiên}
\author{Nguyễn Quản Bá Hồng\footnote{Independent Researcher, Ben Tre City, Vietnam\\e-mail: \texttt{nguyenquanbahong@gmail.com}; website: \url{https://nqbh.github.io}.}}
\date{\today}

\begin{document}
\maketitle
\begin{abstract}
	Last updated version: \href{https://github.com/NQBH/hobby/blob/master/elementary_mathematics/grade_6/natural/natural_calculus/problem/NQBH_natural_calculus_problem.pdf}{GitHub{\tt/}NQBH{\tt/}hobby{\tt/}elementary mathematics{\tt/}grade 6{\tt/}natural{\tt/}natural calculus{\tt/}problem: calculus on set $\mathbb{N}$ of naturals [pdf]}.\footnote{\textsc{url}: \url{https://github.com/NQBH/hobby/blob/master/elementary_mathematics/grade_6/natural/natural_calculus/problem/NQBH_natural_calculus_problem.pdf}.} [\href{https://github.com/NQBH/hobby/blob/master/elementary_mathematics/grade_6/natural/natural_calculus/problem/NQBH_natural_calculus_problem.tex}{\TeX}]\footnote{\textsc{url}: \url{https://github.com/NQBH/hobby/blob/master/elementary_mathematics/grade_6/natural/natural_calculus/problem/NQBH_natural_calculus_problem.tex}.}. 
\end{abstract}
\tableofcontents

%------------------------------------------------------------------------------%

\section{Basic Calculus on $\mathbb{N}$ -- Phép $\pm,\cdot,:$ trên $\mathbb{N}$}

\begin{baitoan}[\cite{Binh_boi_duong_Toan_6_tap_1}, H1, p. 17]
	(a) Viết đủ 6 số $1,2,3,4,5,6$ vào 3 đỉnh \& 3 trung điểm của 3 cạnh của 1 tam giác sao cho tổng 3 số trên mỗi cạnh bằng $10$. (b${}^\star$) Có bao nhiêu cách tất cả?
\end{baitoan}

\begin{baitoan}[\cite{Binh_boi_duong_Toan_6_tap_1}, H2, p. 17]
	{\rm Đ{\tt/}S?} Cho $a,b,c,m,n,p\in\mathbb{N}^\star$ thỏa $a + m = b + n = c + p = a + b + c$. (a) $m + n > p$. (b) $n + p < m$. (c) $p + m > n$. (d) $m + n + p = a + b + c$. (e) $m + n + p = 2(a + b + c)$. (f) $m,n,p$ là độ dài 3 cạnh của 1 tam giác.
\end{baitoan}

\begin{baitoan}[\cite{Binh_boi_duong_Toan_6_tap_1}, H3, p. 17]
	Tính $3^4:3 + 2^3:2^2$.
\end{baitoan}

\begin{baitoan}[\cite{Binh_boi_duong_Toan_6_tap_1}, H4, p. 17]
	{\rm Đ{\tt/}S?} (a) $(15 + 5)(4 + 1) = 20\cdot5 = 100$. (b) $5 + 20\cdot4 = 25\cdot4 = 100$.
\end{baitoan}

\begin{baitoan}[\cite{Binh_boi_duong_Toan_6_tap_1}, H5, p. 17]
	Tính: (a) Hiệu của 2 số lẻ mà giữa chúng có $10$ số chẵn. (b) Hiệu của 2 số lẻ mà giữa chúng có $10$ số lẻ. (c) Hiệu của 2 số chẵn mà giữa chúng có $5$ số chẵn. (d) Hiệu của 2 số chẵn mà giữa chúng có $5$ số lẻ.
\end{baitoan}

\begin{baitoan}[\cite{Binh_boi_duong_Toan_6_tap_1}, VD1, p. 17]
	Tính hợp lý: (a) $A = 27\cdot36 + 73\cdot99 + 27\cdot14 - 49\cdot73$. (b) $B = (4^5\cdot10\cdot5^6 + 25^5\cdot2^8):(2^8\cdot5^4 + 5^7\cdot2^5)$.
\end{baitoan}

\begin{baitoan}[\cite{Binh_boi_duong_Toan_6_tap_1}, VD2, p. 18]
	Egg \& Chicken cùng ra cửa hàng mua sách. Tổng số tiền ban đầu của 2 bạn là $78000$ đồng. Egg mua hết $32000$ đồng, Chicken mua hết $14000$ đồng. Khi đó số tiền còn lại của 2 bạn bằng nhau. Hỏi ban đầu mỗi bạn có bao nhiêu tiền?
\end{baitoan}

\begin{baitoan}[\cite{Binh_boi_duong_Toan_6_tap_1}, VD3, p. 18]
	So sánh: (a) $(4 + 5)^2$ \& $4^2 + 5^2$. (b) $2^{30}$ \& $3^{20}$.
\end{baitoan}

\begin{baitoan}[\cite{Binh_boi_duong_Toan_6_tap_1}, VD4, p. 19]
	Tế bào lớn lên đến 1 kích thước nhất định thì phân chia. Quá trình đó diễn ra như sau: Đầu tiên từ $1$ nhân hình thành $2$ nhân, tách xa nhau. Sau đó chất tế bào được phân chia, xuất hiện $1$ vách ngăn, ngăn đôi tế bào cũ thành $2$ tế bào con. Các tế bào con tiếp tục lớn lên cho đến khi bằng tế bào mẹ. Các tế bào này lại tiếp tục phân chia thành $4$, rồi thành $8,\ldots$ tế bào. Từ $1$ tế bào ban đầu, tìm số tế bào có được sau lần phân chia thứ $5$, thứ $8$, thứ $10$, thứ $n$.
\end{baitoan}

\begin{baitoan}[\cite{Binh_boi_duong_Toan_6_tap_1}, VD5, p. 19]
	Tìm $x\in\mathbb{N}$ thỏa: (a) $149 - (35:x + 3)\cdot17 = 13$. (b) $\overline{1x32} + \overline{7x8} + \overline{4x} = \overline{200x}$.
\end{baitoan}

\begin{baitoan}[\cite{Binh_boi_duong_Toan_6_tap_1}, VD6, p. 20]
	Tìm $x\in\mathbb{N}$ thỏa: (a) $(3x - 2)^3 = 2\cdot32$. (b) $5^{x+1} - 5^x = 500$.
\end{baitoan}

\begin{baitoan}[\cite{Binh_boi_duong_Toan_6_tap_1}, VD7, p. 20]
	Tìm các số mũ tự nhiên $n$ sao cho lũy thừa $3^n$ thỏa mãn điều kiện $25 < 3^n < 260$.
\end{baitoan}

\begin{baitoan}[\cite{Binh_boi_duong_Toan_6_tap_1}, VD8, p. 21]
	Tìm số chia \& số bị chia nhỏ nhất có thương số là $6$ \& số dư là $13$.
\end{baitoan}

\begin{baitoan}[\cite{Binh_boi_duong_Toan_6_tap_1}, 2.1., p. 21]
	Tính hợp lý: (a) $21\cdot(271 + 29) + 79\cdot(271 + 29)$. (b) $1 + 2 - 3 - 4 + 5 + 6 - 7 - 8 + \cdots - 499 - 500 + 501 + 502$.
\end{baitoan}

\begin{baitoan}[\cite{Binh_boi_duong_Toan_6_tap_1}, 2.2., p. 21]
	Quan hệ về cường độ của các nốt nhạc: 1 nốt tròn bằng 2 nốt trắng, 1 nốt trắng bằng 2 nốt đen, 1 nốt đen bằng 2 nốt móc, 1 nốt móc bằng 2 nốt móc đôi, 1 nốt móc đoi bằng 2 nốt móc 3, 1 nốt móc 3 bằng 2 nốt móc 4. Dùng lũy thừa của 1 số tự nhiên để diễn tả mối quan hệ về cường độ giữa: (a) Nốt tròn \& nốt đen. (b) Nốt tròn \& nốt móc 4. (c) Nốt trắng \& nốt móc đôi.
\end{baitoan}

\begin{baitoan}[\cite{Binh_boi_duong_Toan_6_tap_1}, 2.3., p. 21]
	Tính giá trị của biểu thức: $A = 3ab^2 -  \dfrac{a^3}{d} + c$ với $a = 3$, $b = 5$, $c = 7$, $d = 1$.
\end{baitoan}

\begin{baitoan}[\cite{Binh_boi_duong_Toan_6_tap_1}, 2.4., p. 21]
	So sánh: (a) $243^7$ \& $9^{10}\cdot27^5$. (b) $15^{15}$ \& $81^3\cdot125^5$. (c) $78^{15} - 78^{12}$ \& $78^{12} - 78^9$.
\end{baitoan}

\begin{baitoan}[\cite{Binh_boi_duong_Toan_6_tap_1}, 2.5., p. 21]
	Tìm $x\in\mathbb{N}$ thỏa: (a) $121:11 - (4x + 5):3 = 4$. (b) $2 + 4 + 6 + \cdots + x = 2450$ với $x$ là số tự nhiên chẵn.
\end{baitoan}

\begin{baitoan}[\cite{Binh_boi_duong_Toan_6_tap_1}, 2.6., p. 21]
	Tìm $x\in\mathbb{N}$ thỏa: (a) $(3x - 7)^5 = 32$. (b) $(4x - 1)^3 = 27\cdot125$.
\end{baitoan}

\begin{baitoan}[\cite{Binh_boi_duong_Toan_6_tap_1}, 2.7., p. 21]
	Cho 3 số $5,7,9$. Tìm tổng tất cả các số khác nhau viết bằng cả 3 số đó, mỗi chữ số dùng 1 lần.
\end{baitoan}

\begin{baitoan}[\cite{Binh_boi_duong_Toan_6_tap_1}, 2.8., p. 21]
	Tích của 2 số là $476$. Nếu thêm $22$ đơn vị vào 1 số thì tích của 2 số là $850$. Tìm 2 số đó.
\end{baitoan}

\begin{baitoan}[\cite{Binh_boi_duong_Toan_6_tap_1}, 2.9., p. 21]
	Hiệu của 2 số là $12$. Nếu tăng số bị trừ lên $2$ lần, giữ nguyên số trừ thì hiệu của chúng là $49$. Tìm 2 số đó.
\end{baitoan}

\begin{baitoan}[\cite{Binh_boi_duong_Toan_6_tap_1}, 2.10., p. 21]
	Tìm 2 số tự nhiên có thương bằng $7$. Nếu giảm số bị chia đi $124$ đơn vị thì thương của chúng bằng $3$.
\end{baitoan}

\begin{baitoan}[\cite{Binh_boi_duong_Toan_6_tap_1}, 2.11., p. 21]
	Rút gọn biểu thức: (a) $10\cdot\dfrac{4^6\cdot9^5 + 6^9\cdot120}{8^4\cdot3^{12} - 6^{11}}$. (b) $\sum_{i=0}^{50} 2^i = 1 + 2 + 2^2 + 2^3 + 2^4 + \cdots + 2^{49} + 2^{50}$. (c) $5 + 5^3 + 5^5 + \cdots + 5^{47} + 5^{49}$.
\end{baitoan}

\begin{baitoan}[\cite{Binh_boi_duong_Toan_6_tap_1}, 2.12., p. 22]
	Cho $\sum_{i=0}^{2000} 3^i = 1 + 3 + 3^2 + 3^3 + \cdots + 3^{1999} + 3^{2000}$. Chứng minh $A\divby13$.
\end{baitoan}

\begin{baitoan}[\cite{Binh_boi_duong_Toan_6_tap_1}, 2.13., p. 22]
	Tìm $x\in\mathbb{N}$ thỏa: (a) $2^x + 2^{x + 1} = 96$. (b) $3^{4x + 4} = 81^{x + 3}$.
\end{baitoan}

\begin{baitoan}[\cite{Binh_boi_duong_Toan_6_tap_1}, 2.14., p. 22]
	Tìm $x\in\mathbb{N}$ thỏa: (a) $(x - 5)^7 = (x - 5)^9$. (b) $x^{2015} = x^{2016}$.
\end{baitoan}

\begin{baitoan}[\cite{Binh_boi_duong_Toan_6_tap_1}, 2.15., p. 22]
	Tìm các số tự nhiên $x$ biết lũy thừa $5^{2x - 3}$ thỏa mãn điều kiện $100 < 5^{2x - 3}\le5^9$.
\end{baitoan}

\begin{baitoan}[\cite{Binh_boi_duong_Toan_6_tap_1}, 2.16., p. 22]
	Trong 1 phép chia, số bị chia bằng $69$, số dư bằng $3$. Tìm số chia \& thương.
\end{baitoan}

\begin{baitoan}[\cite{Binh_boi_duong_Toan_6_tap_1}, 2.17., p. 22]
	Tổng của 3 số là $124$. Nếu lấy số thứ nhất chia cho số thứ 2 hoặc lấy số thứ 2 chia cho số thứ 3 đều được thương là $3$ \& dư $4$. Tìm 3 số đó.
\end{baitoan}

\begin{baitoan}[\cite{Binh_boi_duong_Toan_6_tap_1}, 2.18., p. 22]
	(a) Khi chia 1 số cho $54$ thì được số dư là $49$. Nếu chia số đó cho $18$ thì thương \& số dư thay đổi thế nào? (b) Khi chia 1 số $a\in\mathbb{N}$ cho $b\in\mathbb{N}^\star$ thì được số dư là $r\in\mathbb{N}$. Nếu chia số đó cho $c\in\mathbb{N}^\star$ là 1 ước của $b$ thì thương \& số dư thay đổi thế nào?
\end{baitoan}

\begin{baitoan}[\cite{Binh_boi_duong_Toan_6_tap_1}, 2.19., p. 22]
	Tìm số bị chia \& số chia nhỏ nhất để được thương là $8$ \& dư là $45$.
\end{baitoan}

\begin{baitoan}[\cite{Binh_boi_duong_Toan_6_tap_1}, 2.20., p. 22]
	Tổng của 2 số bằng $36000$. Chia số lớn cho số nhỏ được thương bằng $4$ \& còn dư $940$. Tìm 2 số đó.
\end{baitoan}

\begin{baitoan}[\cite{Binh_boi_duong_Toan_6_tap_1}, 2.21., p. 22, Hạt thóc \& bàn cờ vua]
	1 nhà thông thái muốn thuyết phục vua Shilhram (Ấn Độ) về tầm quan trọng của mỗi người dân trong vương quốc. Vì thế, ông ta phát minh ra bàn cờ vua để thể hiện 1 vương quốc bao gồm vua, hoàng hậu, giáo sĩ trưởng, hiệp sĩ, \& quân lính, mọi thành phố đều quan trọng. Nhà vua trở nên cực kỳ thích \& ra lệnh cho mọi người trong vương quốc chơi cờ vua. Shilhram tuyên bố sẽ ban tặng nhà thông thái bất cứ số vàng \& bạc nào mà ông ta muốn, nhưng nhà thông thái không muốn nhận phần thưởng như vậy. Nhà thông thái cùng với nhà vua đi đến bàn cờ \& nhờ ông đặt 1 hạt thóc lên ô vuông đầu tiên, 2 hạt lên ô thứ 2, 4 hạt lên ô vuông thứ 3 \& xin với nhà vua tổng số hạt thóc được đặt theo cách như vậy đến ô cuối cùng của bàn cờ (số hạt đặt vào ô sau gấp đôi số hạt đặt vào ô trước). Nhà vua cảm thấy bị xúc phạm nhưng ông vẫn ra lệnh cho người hầu làm theo ước muốn của nhà thông thái. Những người hầu tuyệt vọng báo lại rằng số lượng hạt thóc dùng để thưởng cho nhà thông thái theo cách như vậy là không đủ. Nhà vua hiểu ngay nhà thông thái muốn dạy ông bài học thứ 2. Giống như quân tốt trong cờ vua, không nên đánh giá thấp những thứ nhỏ bé trong cuộc sống. Tính số hạt thóc mà nhà thông thái yêu cầu nhà vua Shilhram thưởng cho mình.
\end{baitoan}

\begin{baitoan}[\cite{Binh_boi_duong_Toan_6_tap_1}, p. 23]
	Có $79$ số tự nhiên, trong đó tổng của $13$ số bất kỳ đều là 1 số lẻ. Hỏi tổng của $79$ số tự nhiên đó là số lẻ hay số chẵn?
\end{baitoan}

\begin{baitoan}[\cite{Tuyen_Toan_6}, VD3, p. 8]
	1 học sinh khi nhân 1 số với $31$ đã đặt các tích riêng thẳng hàng như trong phép cộng nên tích đã giảm đi $540$ đơn vị so với tích đúng. Tìm tích đúng.
\end{baitoan}

\begin{baitoan}[\cite{Tuyen_Toan_6}, VD4, p. 8]
	Cho 2 số không chia hết cho $3$, khi chia cho $3$ được các số dư khác nhau. Chứng minh tổng của 2 số đó chia hết cho $3$.
\end{baitoan}

\begin{baitoan}[\cite{Tuyen_Toan_6}, 14., p. 9]
	Tính hợp lý: (a) $38 + 41 + 117 + 159 + 62$. (b) $73 + 86 + 978 + 914 + 3022$. (c) $341\cdot67 + 341\cdot16 + 659\cdot83$. (d) $42\cdot53 + 47\cdot156 - 47\cdot114$.
\end{baitoan}

\begin{baitoan}[\cite{Tuyen_Toan_6}, 15., p. 9]
	Tính giá trị của biểu thức: (a) $A = (100 - 1)\cdot(100 - 2)\cdots(100 - n)$ với $n\in\mathbb{N}^\star$ \& tích trên có đúng $100$ thừa số. (b) $B = 13a + 19b + 4a - 2b$ với $a + b = 100$.
\end{baitoan}

\begin{baitoan}[\cite{Tuyen_Toan_6}, 16., p. 9]
	Không tính giá trị cụ thể, so sánh giá trị 2 biểu thức: (a) $A = 199\cdot201$ \& $B = 200\cdot200$. (b) $C = 35\cdot53 - 18$ \& $D = 35 + 53\cdot34$.
\end{baitoan}

\begin{baitoan}[\cite{Tuyen_Toan_6}, 17., p. 9]
	Tính hợp lý: (a) $(44\cdot52\cdot60):(11\cdot13\cdot15)$. (b) $123\cdot456456 - 456\cdot123123$. (c) $(98\cdot7676 - 9898\cdot76):(2021\cdot2022\cdot2023\cdots2030)$.
\end{baitoan}

\begin{baitoan}[\cite{Tuyen_Toan_6}, 18., p. 9]
	Tìm giá trị nhỏ nhất của biểu thức: $A = 2021 - 1021:(999 - x)$.
\end{baitoan}

\begin{baitoan}[\cite{Tuyen_Toan_6}, 20., p. 9]
	Tìm số hạng thứ 5, thứ $n$ của dãy số: (a) $2,3,7,25,\ldots$. (b) $8,30,72,140,\ldots$.
\end{baitoan}

\begin{baitoan}[\cite{Tuyen_Toan_6}, 21., p. 9]
	Tìm $x$: (a) $(x + 74) - 318 = 200$. (b) $3636:(12x - 91) = 36$. (c) $(x:23 + 45)\cdot67 = 8911$.
\end{baitoan}

\begin{baitoan}[\cite{Tuyen_Toan_6}, 22., p. 9]
	1 nong tằm là $5$ nong kén. 1 nong kén là $9$ nén tơ. Hỏi muốn được $540$ nén tơ thì phải chăn bao nhiêu nong tằm?
\end{baitoan}

\begin{baitoan}[\cite{Tuyen_Toan_6}, 23., p. 9]
	2 số tự nhiên $a,b$ chia cho $m$ có cùng số dư, $a\ge b$. Chứng tỏ $a - b$ chia hết cho $m$.
\end{baitoan}

\begin{baitoan}[\cite{Tuyen_Toan_6}, 24., p. 9]
	Trong 1 phép chia có số bị chia là $155$, số dư là $12$. Tìm số chia \& thương.
\end{baitoan}

\begin{baitoan}[\cite{Tuyen_Toan_6}, 25., p. 9]
	Viết tập hợp $A$ các số tự nhiên $x$ biết lấy $x$ chia cho $12$ ta được thương bằng số dư.
\end{baitoan}

\begin{baitoan}[\cite{Tuyen_Toan_6}, 26., p. 10]
	Chia $129$ cho 1 số ta được số dư là $10$. Chia $61$ cho số đó ta cũng được số dư là $10$. Tìm số chia.
\end{baitoan}

\begin{baitoan}[\cite{Tuyen_Toan_6}, 27., p. 10]
	Cho 3 chữ số $a,b,c$ khác nhau \& khác $0$. Với cùng cả 3 chữ số này có thể lập được bao nhiêu số có 3 chữ số?
\end{baitoan}

\begin{baitoan}[\cite{Tuyen_Toan_6}, 28., p. 10]
	Cho 4 chữ số khác nhau \& khác $0$. (a) Với cùng cả 4 chữ số này có thể lập được bao nhiêu số có 4 chữ số? (b) Có thể lập được bao nhiêu số có 2 chữ số khác nhau trong 4 chữ số đã cho?
\end{baitoan}

\begin{baitoan}[\cite{Tuyen_Toan_6}, 29., p. 10]
	Cho 4 chữ số khác nhau trong đó có 1 chữ số $0$. Với cùng cả 4 chữ số này có thể lập được bao nhiêu số có 4 chữ số?
\end{baitoan}

\begin{baitoan}[\cite{Tuyen_Toan_6}, 30., p. 10]
	Anh Bách đi mua bánh kẹo tại 1 siêu thị, thanh toán bằng 1 phiếu mua hàng trị giá $100000$ đồng. Siêu thị không trả lại số tiền thừa. Giúp anh Bách chọn mua vừa hết số tiền ghi trong phiếu. Bảng giá 1 số mặt hàng có bán:
	\begin{table}[H]
		\centering
		\begin{tabular}{|c|l|c|r|}
			\hline
			STT & Tên hàng & Đơn vị & Giá bán \\
			\hline
			1 & Bánh đậu xanh & Hộp & 31 500 đồng \\
			\hline
			2 & Bánh bông lan & Gói & 23 500 đồng \\
			\hline
			3 & Bánh gạo & Gói & 19 000 đồng \\
			\hline
			4 & Bánh gạo chiên & Gói & 17 800 đồng \\
			\hline
			5 & Bánh quy & Gói & 13 500 đồng \\
			\hline
			6 & Bánh xốp & Gói & 5 300 đồng \\
			\hline
			7 & Kẹo hương dâu & Gói & 2 500 đồng \\
			\hline
		\end{tabular}
	\end{table}
\end{baitoan}

\begin{baitoan}[\cite{Binh_Toan_6_tap_1}, VD4, p. 7]
	Cho $A = 137\cdot454 + 206$, $B = 453\cdot138 - 110$. Không tính giá trị của $A$ \& $B$, chứng tỏ $A = B$.
\end{baitoan}

\begin{baitoan}[\cite{Binh_Toan_6_tap_1}, VD5, p. 8]
	Tìm kết quả của phép nhân: $A = \underbrace{33\ldots3}_{50}\cdot\underbrace{99\ldots9}_{50}$.
\end{baitoan}

\begin{baitoan}[\cite{Binh_Toan_6_tap_1}, VD6, p. 8]
	Tổng của 2 số tự nhiên khác nhau gấp 3 hiệu của chúng. Tìm thương của 2 số tự nhiên đó.
\end{baitoan}

\begin{baitoan}[\cite{Binh_Toan_6_tap_1}, VD7, p. 8]
	Khi chia số tự nhiên $a$ cho $54$, ta được số dư là $38$. Chia số $a$ cho $18$, ta được thương là $14$ \& còn dư. Tìm $a$.
\end{baitoan}

\begin{baitoan}[\cite{Binh_Toan_6_tap_1}, VD8, p. 8]
	Tìm 2 số tự nhiên lớn hơn $0$ sao cho tích của 2 số ấy gấp đôi tổng của chúng.
\end{baitoan}

\begin{baitoan}[\cite{Binh_Toan_6_tap_1}, VD9, p. 9]
	Điền các số tự nhiên $1,2,3,4,5$ vào các dấu $*$ để kết quả phép tính bằng $6$:  $*+*-*\cdot*:*$.
\end{baitoan}

\begin{baitoan}[\cite{Binh_Toan_6_tap_1}, VD10, p. 9]
	Giá tiền $7$ quyển vở nhiều hơn giá tiền $8$ bút chì. Hỏi giá tiền $8$ quyển vở \& giá tiền $9$ bút chì, đằng nào nhiều hơn?
\end{baitoan}

\begin{baitoan}[\cite{Binh_Toan_6_tap_1}, VD11, p. 9]
	Cho 6 số tự nhiên khác nhau có tổng bằng $50$. Chứng minh trong 6 số đó tồn tại 3 số có tổng lớn hơn hoặc bằng $30$.	
\end{baitoan}

\begin{baitoan}[\cite{Binh_Toan_6_tap_1}, 13., p. 10]
	Có thể viết được hay không 9 số vào 1 bảng vuông $3\times 3$, sao cho: Tổng các số trong 3 dòng theo thứ tự bằng $352, 463, 541$; tổng các số trong 3 cột theo thứ tự bằng $335, 687, 234$?
\end{baitoan}

\begin{baitoan}[\cite{Binh_Toan_6_tap_1}, 14., p. 10]
	Cho 9 số xếp vào 9 ô thành 1 hàng ngang, trong đó số đầu tiên là $4$, số cuối cùng là $8$, \& tổng 3 số ở 3 ô liền nhau bất kỳ bằng $17$. Tìm 9 số đó.
\end{baitoan}

\begin{baitoan}[\cite{Binh_Toan_6_tap_1}, 15., p. 10]
	Tìm số có $3$ chữ số, biết chữ số hàng trăm gấp $4$ lần chữ số hàng đơn vị \& nếu viết số ấy theo thứ tự ngược lại thì nó giảm đi $594$ đơn vị.
\end{baitoan}

\begin{baitoan}[\cite{Binh_Toan_6_tap_1}, 16., p. 10]
	Thay các dấu * bởi các chữ số thích hợp: $**** - *** = **$ biết số bị trừ, số trừ \& hiệu đều không đổi nếu đọc mỗi số từ phải sang trái.
\end{baitoan}

\begin{baitoan}[\cite{Binh_Toan_6_tap_1}, 19., p. 10]
	Hiệu của 2 số là $4$. Nếu tăng 1 số gấp $3$ lần, giữ nguyên số kia thì hiệu của chúng bằng $60$. Tìm 2 số đó.
\end{baitoan}

\begin{baitoan}[\cite{Binh_Toan_6_tap_1}, 20., p. 10]
	Cho số $123456789$. Đặt 1 số dấu ``$+$'' \& ``$-$'' vào giữa các chữ số để kết quả của phép tính bằng $100$.
\end{baitoan}

\begin{baitoan}[\cite{Binh_Toan_6_tap_1}, 21., p. 10]
	Cho số $987654321$. Đặt 1 số dấu ``$+$'' \& ``$-$'' vào giữa các chữ số để kết quả của phép tính bằng: $100,99$.
\end{baitoan}

\begin{baitoan}[\cite{Binh_Toan_6_tap_1}, 22., p. 10]
	Tìm giá trị lớn nhất của hiệu $\overline{bd} - \overline{ac}$ biết $a < b < c < d$.
\end{baitoan}

\begin{baitoan}[\cite{Binh_Toan_6_tap_1}, 23., p. 10]
	Tìm 6 chữ số khác nhau $a,b,c,d,e,f$ sao cho $A = \overline{abc} - \overline{def}$ có giá trị nhỏ nhất.
\end{baitoan}

\begin{baitoan}[\cite{Binh_Toan_6_tap_1}, 24., p. 11]
	Cho 6 chữ số khác nhau $a,b,c,d,e,f$. Gọi $A = \overline{abc} + \overline{bcd} + \overline{cde} + \overline{def}$. Tìm giá trị lớn nhất \& giá trị nhỏ nhất của $A$.
\end{baitoan}

\begin{baitoan}[\cite{Binh_Toan_6_tap_1}, 25., p. 11]
	Tìm 2 số biết tổng của chúng gấp $5$ lần hiệu của chúng, tích của chúng gấp $24$ lần hiệu của chúng.
\end{baitoan}

\begin{baitoan}[\cite{Binh_Toan_6_tap_1}, 26., p. 11]
	Tìm 2 số biết tổng của chúng gấp $7$ lần hiệu của chúng, còn tích của chúng gấp $192$ lần hiệu của chúng.
\end{baitoan}

\begin{baitoan}[\cite{Binh_Toan_6_tap_1}, 27., p. 11]
	Tích của 2 số là $6210$. Nếu giảm 1 thừa số đi $7$ đơn vị thì tích mới là $5265$. Tìm các thừa số của tích.
\end{baitoan}

\begin{baitoan}[\cite{Binh_Toan_6_tap_1}, 28., p. 11]
	Bảo làm 1 phép nhân, trong đó số nhân là $102$. Nhưng khi viết số nhân, Bảo đã quên không viết chữ số $0$ nên tích bị giảm đi $21870$ đơn vị so với tích đúng. Tìm số bị nhân của phép nhân đó.
\end{baitoan}

\begin{baitoan}[\cite{Binh_Toan_6_tap_1}, 29., p. 11]
	1 học sinh nhân $78$ với số nhân là số có 2 chữ số, trong đó chữ số hàng chục gấp $3$ lần chữ số hàng đơn vị. Do nhầm lẫn bạn đó viết đổi thứ tự 2 chữ số của số nhân, nên tích giảm đi $2808$ đơn vị so với tích đúng. Tìm số nhân đúng.
\end{baitoan}

\begin{baitoan}[\cite{Binh_Toan_6_tap_1}, 30., p. 11]
	1 học sinh nhân 1 số với $463$. Vì bạn đó viết các chữ số tận cùng của các tích riêng ở cùng 1 cột nên tích bằng $30524$. Tìm số bị nhân.
\end{baitoan}

\begin{baitoan}[\cite{Binh_Toan_6_tap_1}, 31., p. 11]
	Chứng minh hiệu sau có thể viết được thành 1 tích của 2 thừa số bằng nhau: $11111111 - 2222$.
\end{baitoan}

\begin{baitoan}[\cite{Binh_Toan_6_tap_1}, 32., p. 11]
	Chỉ ra 2 số khác nhau sao cho nếu nhân mỗi số với $7$ thì ta được kết quả là các số gồm toàn các chữ số $9$.
\end{baitoan}

\begin{baitoan}[\cite{Binh_Toan_6_tap_1}, 33., p. 11]
	Tìm kết quả của phép nhân sau: $\underbrace{3\ldots 3}_{50}\cdot\underbrace{3\ldots 3}_{50}$.
\end{baitoan}

\begin{baitoan}[\cite{Binh_Toan_6_tap_1}, 34., p. 11]
	Tìm tổng các chữ số của tích: (a) $A = \underbrace{88\ldots8}_{21}\cdot\underbrace{99\ldots9}_{21}$. (b) $B = \underbrace{99\ldots9}_{94}\cdot\underbrace{44\ldots4}_{94}$.
\end{baitoan}

\begin{baitoan}[\cite{Binh_Toan_6_tap_1}, 35., p. 11]
	Chứng minh các số sau có thể viết được thành 1 tích của 2 số tự nhiên liên tiếp: $111222$, $444222$.
\end{baitoan}

\begin{baitoan}[\cite{Binh_Toan_6_tap_1}, 36., p. 11]
	Tìm 2 số tự nhiên có thương bằng $35$ biết nếu số bị chia tăng thêm $1056$ đơn vị thì thương bằng $57$.
\end{baitoan}

\begin{baitoan}[\cite{Binh_Toan_6_tap_1}, 37., p. 11]
	Tìm số bị chia \& số chia biết thương bằng $6$, số dư bằng $49$, tổng của số bị chia, số chia, \& số dư bằng $595$.
\end{baitoan}

\begin{baitoan}[\cite{Binh_Toan_6_tap_1}, 38., p. 11]
	1 phép chia có thương bằng $4$, số dư bằng $25$. Tổng của số bị chia, số chia \& số dư bằng $210$. Tìm số bị chia \& số chia.
\end{baitoan}

\begin{baitoan}[\cite{Binh_Toan_6_tap_1}, 39., p. 11]
	Trong hội trường có $680$ người ngồi. Tất cả có $25$ dãy ghế, mỗi dãy ghế có $30$ chỗ ngồi. Ít nhất có bao nhiêu dãy ghế có số chỗ ngồi như nhau?
\end{baitoan}

\begin{baitoan}[\cite{Binh_Toan_6_tap_1}, 40., p. 12]
	(a) Trong 1 năm, có ít nhất \& nhiều nhất bao nhiêu ngày Chủ nhật? (b) Ngày 1.1 năm nay rơi vào ngày Chủ nhật. Ngày 1.1 năm sau rơi vào ngày thứ mấy?
\end{baitoan}

\begin{baitoan}[\cite{Binh_Toan_6_tap_1}, 41., p. 12]
	Tháng 8 của 1 năm có 4 ngày thứ Năm \& 5 ngày thứ 4. Hỏi ngày đầu tiên của tháng đó là ngày thứ mấy?
\end{baitoan}

\begin{baitoan}[\cite{Binh_Toan_6_tap_1}, 42., p. 12]
	Ngày 19.8.2002 vào ngày thứ Hai. Tính xem ngày 19.8.1945 vào ngày nào trong tuần?
\end{baitoan}

\begin{baitoan}[\cite{Binh_Toan_6_tap_1}, 43., p. 12]
	Tìm thương của 1 phép nhân, biết nếu thêm $15$ vào số bị chia \& thêm $5$ vào số chia thì thương \& số dư không đổi.
\end{baitoan}

\begin{baitoan}[\cite{Binh_Toan_6_tap_1}, 44., p. 12]
	Tìm thương của 1 phép chia, biết nếu tăng số bị chia $90$ đơn vị, tăng số chia $6$ đơn vị thì thương \& số dư không đổi.
\end{baitoan}

\begin{baitoan}[\cite{Binh_Toan_6_tap_1}, 45., p. 12]
	Tìm thương của 1 phép chia, biết nếu tăng số bị chia $73$ đơn vị, tăng số chia $4$ đơn vị thì thương không đổi, còn số dư tăng $5$ đơn vị.
\end{baitoan}

\begin{baitoan}[\cite{Binh_Toan_6_tap_1}, 46., p. 12]
	Xác định phép chia, biết số bị chia, số chia, thương \& số dư là 4 số trong các số sau: (a) $3,4,16,64,256,772$. (b) $2,3,9,27,81,243,567$.
\end{baitoan}

\begin{baitoan}[\cite{Binh_Toan_6_tap_1}, 47., p. 12]
	Khi chia 1 số tự nhiên gồm 3 chữ số như nhau cho 1 số tự nhiên gồm 3 chữ số như nhau, ta được thương là $2$ \& còn dư. Nếu xóa 1 chữ số ở số bị chia \& xóa 1 chữ số ở số chia thì thương của phép chia vẫn bằng $3$ nhưng số dư giảm hơn trước là $100$. Tìm số bị chia \& số chia lúc đầu.
\end{baitoan}

\begin{baitoan}[\cite{Binh_Toan_6_tap_1}, 48., p. 12]
	Trong 1 phép chia có dư, số bị chia gồm 4 chữ số như nhau, số chia gồm 3 chữ số như nhau, thương bằng $13$ \& còn dư. Nếu xóa 1 chữ số ở số bị chia, xóa 1 chữ số ở số chia thì thương không đổi, còn số dư giảm hơn trước là $100$ đơn vị. Tìm số bị chia \& số chia lúc đầu.
\end{baitoan}

\subsection{Combination of Calculus -- Phối Hợp Các Phép Tính}

\begin{baitoan}[\cite{Binh_Toan_6_tap_1}, 49., p. 12]
	Tính nhanh: (a) $19\cdot64 + 76\cdot34$. (b) $35\cdot12 + 65\cdot13$. (c) $136\cdot68 + 16\cdot272$. (d) $(2 + 4 + 6 + \cdots + 100)\cdot(36\cdot333 - 108\cdot111)$. (e) $19991999\cdot1998 - 19981998\cdot1999$.
\end{baitoan}

\begin{baitoan}[\cite{Binh_Toan_6_tap_1}, 50., p. 12]
	Không tính cụ thể các giá trị của $A$ \& $B$, cho biết số nào lớn hơn \& lớn hơn bao nhiêu? (a) $A = 1998\cdot1998$, $B = 1996\cdot2000$. (b) $A = 2000\cdot2000$, $B = 1990\cdot2010$. (c) $A = 25\cdot33 - 10$, $B = 31\cdot26 + 10$. (d) $A = 32\cdot53 - 31$, $B = 53\cdot31 + 32$.
\end{baitoan}
Bài toán trên có thể được tổng quát như sau:

\begin{baitoan}
	Cho $n,k\in\mathbb{N}^\star$. Không tính cụ thể các giá trị của $A_i$ \& $B_i$, $i = 1,2$, cho biết số nào lớn hơn \& lớn hơn bao nhiêu? (a) $A_1 = n\cdot n = n^2$, $B_1 = (n - k)(n + k)$. (b) $A_2 = n^3$, $B_2 = (n - k)n(n + k)$.
\end{baitoan}

\begin{proof}[Giải]
	(a) Khai triển $B_1$ hoặc dùng \textit{hằng đẳng thức đáng nhớ} $(a + b)(a - b) = a^2 - b^2$, $\forall a,b\in\mathbb{R}$, ta có $B_1 = (n - k)(n + k) = n^2 + nk - kn - k^2 = n^2 - k^2 = A_1 - k^2 < A_1$ do $k\ge 1$. Vậy $A_1 > B_1$ \& lớn hơn 1 lượng bằng $k^2$. (b) Nhận thấy $A_2 = nA_1$, $B_2 = nB_1$, nên $A_2 - B_2 = n(A_1 - B_1) = nk^2 > 0$.
\end{proof}
Bài toán trên có thể được tổng quát hơn nữa như sau, ý tưởng giải vẫn là sử dụng \textit{nhiều lần} hằng đẳng thức $(a + b)(a - b) = a^2 - b^2$, $\forall a,b\in\mathbb{R}$, 1 cách thích hợp:

\begin{baitoan}
	Cho $m,n,k\in\mathbb{N}^\star$. Không tính cụ thể các giá trị của $A_i$ \& $B_i$, $i = 1,2$, cho biết số nào lớn hơn \& lớn hơn bao nhiêu? (a) $A_1 = n^{2m}$, $B_1 = (n - mk)(n - (m - 1)k)\cdots(n - k)(n + k)\cdots(n + mk)$. (b) $A_2 = n^{2m + 1}$, $B_2 = (n - mk)(n - (m - 1)k)\cdots(n - k)n(n + k)\cdots(n + mk)$.
\end{baitoan}
Khi $m = 1$, bài toán tổng quát này trở thành bài toán trước đó.

\begin{baitoan}[\cite{Binh_Toan_6_tap_1}, 51., p. 12]
	Tìm thương của phép chia sau mà không tính kết quả cụ thể của số bị chia \& số chia: $\dfrac{37\cdot13 - 13}{24 + 37\cdot12}$.
\end{baitoan}

\begin{baitoan}[\cite{Binh_Toan_6_tap_1}, 52., p. 13]
	Tính: (a) $A = \dfrac{\sum_{i=1}^{101} i}{\sum_{i=1}^{101} (-1)^{i+1}i} = \dfrac{101 + 100 + 99 + 98 + \cdots + 3 + 2 + 1}{101 - 100 + 99 - 98 + \cdots + 3 - 2 + 1}$. (b) $B = \dfrac{3737\cdot43 - 4343\cdot37}{2 + 4 + 6 + \cdots + 100}$.
\end{baitoan}

\begin{baitoan}[\cite{Binh_Toan_6_tap_1}, 53., p. 13]
	Vận dụng tính chất các phép tính để tìm các kết quả bằng cách nhanh chóng: (a) $1990\cdot1990 - 1992\cdot1988$. (b) $(1374\cdot57 + 687\cdot86):(26\cdot13 + 74\cdot14)$. (c) $(124\cdot237 + 152):(870 + 235\cdot122)$. (d) $\dfrac{423134\cdot846267 - 423133}{846267\cdot423133 + 423134}$.
\end{baitoan}

\begin{baitoan}[\cite{Binh_Toan_6_tap_1}, 54., p. 13]
	Tìm $a\in\mathbb{N}$ biết: (a) $697:\dfrac{15a + 364}{a} = 17$. (b) $92\cdot4 - 27 = \dfrac{a + 350}{a} + 315$.
\end{baitoan}

\begin{baitoan}[\cite{Binh_Toan_6_tap_1}, 55., p. 13]
	Tìm $x\in\mathbb{N}$ biết: (a) $720:[41 - (2x - 5)] = 40$. (b) $(x + 1) + (x + 2) + (x + 3) + \cdots + (x + 100) = 5750$.
\end{baitoan}

\begin{baitoan}[\cite{Binh_Toan_6_tap_1}, 56., p. 13]
	Cho số $12345678$. Đặt các dấu phép tính \& dấu ngoặc để kết quả của phép tính bằng $9$.
\end{baitoan}

\begin{baitoan}[\cite{Binh_Toan_6_tap_1}, 57., p. 13]
	Viết 5 dãy tính có kết quả bằng $100$, với 6 chữ số $5$ cùng với dấu các phép tính (\& dấu ngoặc nếu cần).
\end{baitoan}

\begin{baitoan}[\cite{Binh_Toan_6_tap_1}, 58., p. 13]
	(a) Viết dãy tính có kết quả bằng $100$, với 5 chữ số như nhau cùng với dấu các phép tính (\& dấu ngoặc nếu cần). (b) Cũng hỏi như vậy với 6 chữ số khác nhau.
\end{baitoan}

\begin{baitoan}[\cite{Binh_Toan_6_tap_1}, 59., p. 13]
	(a) Tính (kết quả khá đặc biệt): $1\cdot8 + 1$, $12\cdot8 + 2$, $123\cdot8 + 3$, $1234\cdot8 + 4$. (b) Viết tiếp 4 dòng nữa theo quy luật trên.
\end{baitoan}

\begin{baitoan}[\cite{Binh_Toan_6_tap_1}, 60., p. 13]
	Điền các số $1,2,3,4,5$ vào các dấu $*$ để kết quả của phép tính bằng $3$: $*+*-*\cdot*:*$.
\end{baitoan}

\subsection{Use Inequalities -- Sử Dụng Bất Đẳng Thức}

\begin{baitoan}[\cite{Binh_Toan_6_tap_1}, 61., p. 13]
	Giá tiền 1 quyển sách, 6 quyển vở, 3 chiếc bút là $7700$ đồng, giá tiền 8 quyển sách, 6 quyển vở, 6 chiếc bút là $16000$ đồng. So sánh giá tiền 1 quyển sách \& 1 quyển vở.
\end{baitoan}

\begin{baitoan}[\cite{Binh_Toan_6_tap_1}, 62., p. 13]
	Viết liên tiếp các số tự nhiên từ $1$ đến $15$, được: $A = 1234\ldots1415$. Xóa đi 15 chữ số của số $A$ để các chữ số còn lại (vẫn giữ nguyên thứ tự như trước) tạo thành: (a) Số nhỏ nhất. (b) Số lớn nhất.
\end{baitoan}

\begin{baitoan}[\cite{Binh_Toan_6_tap_1}, 63., p. 14]
	Cho số $123\ldots20$, i.e., viết liên tiếp các số tự nhiên từ $1$ đến $20$. Xóa đi $20$ chữ số để số còn lại có giá trị: (a) Nhỏ nhất. (b) Lớn nhất.
\end{baitoan}

\begin{baitoan}[\cite{Binh_Toan_6_tap_1}, 64., p. 14]
	Tìm giá trị nhỏ nhất của hiệu giữa 1 số tự nhiên có 2 chữ số với tổng các chữ số của nó.
\end{baitoan}

\begin{baitoan}[\cite{Binh_Toan_6_tap_1}, 65., p. 14]
	Tìm số chia \& số dư biết số bị chia bằng $112$, thương bằng $5$.
\end{baitoan}

\begin{baitoan}[\cite{Binh_Toan_6_tap_1}, 66., p. 14]
	Tìm số chia \& số dư biết số bị chia bằng $813$, thương bằng $15$, số dư gồm 2 chữ số như nhau.
\end{baitoan}

\begin{baitoan}[\cite{Binh_Toan_6_tap_1}, 67., p. 14]
	Tìm số chia \& số dư của phép chia $542$ cho 1 số tự nhiên, biết thương bằng $12$.
\end{baitoan}

\begin{baitoan}[\cite{Binh_Toan_6_tap_1}, 68., p. 14]
	1 học sinh trong 5 năm học từ lớp 5 đến lớp 9 đã qua $31$ kỳ thi, trong đó số kỳ thi ở năm sau nhiều hơn số kỳ thi ở năm trước, \& số kỳ thi ở năm cuối gấp 3 lần số kỳ thi ở năm đầu. Hỏi học sinh đó thi bao nhiêu kỳ ở năm thứ 4?
\end{baitoan}

\begin{baitoan}[\cite{Binh_Toan_6_tap_1}, 69., p. 14]
	Tìm 2 số tự nhiên sao cho tổng của 2 số ấy bằng tích của chúng.
\end{baitoan}

\begin{baitoan}[\cite{Binh_Toan_6_tap_1}, 70., p. 14]
	Tìm 3 số tự nhiên khác $0$ biết tổng của 3 số ấy bằng tích của chúng.
\end{baitoan}

%------------------------------------------------------------------------------%

\section{Exponentiation on $\mathbb{N}$ -- Lũy Thừa với Số Mũ Tự Nhiên}

\begin{dinhnghia}[Số chính phương]
	{\rm Số chính phương} là số bằng bình phương của 1 số tự nhiên, i.e., $a$ là số chính phương $\Leftrightarrow a = n^2$ với $n\in\mathbb{N}$ nào đó.
\end{dinhnghia}

\begin{baitoan}[\cite{Tuyen_Toan_6}, VD5, p. 11]
	Chứng minh tổng $\sum_{i=1}^5 i^3 = 1^3 + 2^3 + 3^3 + 4^3 + 5^3$ là 1 số chính phương.
\end{baitoan}

\begin{baitoan}[\cite{Tuyen_Toan_6}, VD6, p. 11]
	Tìm $x\in\mathbb{N}$ biết $2\cdot3^x = 162$.
\end{baitoan}

\begin{baitoan}[\cite{Tuyen_Toan_6}, VD7, p. 11]
	Tìm $x\in\mathbb{N}$ biết $(x + 2)^4 = 5^2\cdot25$.
\end{baitoan}

\begin{baitoan}[\cite{Tuyen_Toan_6}, 31., p. 11]
	Trong các số $2^4,3^4,4^2,4^3,99^0,0^{99},1^n$ với $n\in\mathbb{N}^\star$, các số nào bằng nhau? Số nào nhỏ nhất? Số nào lớn nhất?
\end{baitoan}

\begin{baitoan}[\cite{Tuyen_Toan_6}, 32., p. 11]
	Kiểm tra đẳng thức $152 - 5^3 = 10^2$ đúng hay sai. Nếu sai, di chuyển 1 chữ số đến vị trí khác để được đẳng thức đúng.
\end{baitoan}

\begin{baitoan}[\cite{Tuyen_Toan_6}, 33., p. 11]
	Chứng minh mỗi tổng{\tt/}hiệu sau là 1 số chính phương: (a) $3^2 + 4^2$. (b) $13^2 - 5^2$. (c) $1^3 + 2^3 + 3^3 + 4^3$.
\end{baitoan}

\begin{baitoan}[\cite{Tuyen_Toan_6}, 34., pp. 11--12]
	Viết các tổng{\tt/}hiệu sau dưới dạng 1 lũy thừa với số mũ lớn hơn $1$. (a) $17^2 - 15^2$. (b) $4^3 - 2^3 + 5^2$. 
\end{baitoan}

\begin{baitoan}[\cite{Tuyen_Toan_6}, 35., p. 12]
	Viết số $729$ dưới dạng 1 lũy thừa với 3 cơ số khác nhau \& số mũ lớn hơn $1$.
\end{baitoan}

\begin{baitoan}[\cite{Tuyen_Toan_6}, 36., p. 12]
	Viết các tích{\tt/}thương sau dưới dạng lũy thừa của 1 số: (a) $2^5\cdot8^4$. (b) $25^6\cdot125^3$. (c) $625^5:25^7$. (d) $12^3\cdot3^3$.
\end{baitoan}

\begin{baitoan}[\cite{Tuyen_Toan_6}, 37., p. 12]
	Tính $6^{3^1},3^{2^3},7^{1^{2^{3^4}}},2020^{3^{0^{1^0}}}$.
\end{baitoan}

\begin{baitoan}[\cite{Tuyen_Toan_6}, 38., p. 12]
	Tìm $x\in\mathbb{N}$ biết: (a) $(3x - 2)^3 = 64$. (b) $(2x + 5)^4 = 3^4\cdot5^4$.
\end{baitoan}

\begin{baitoan}[\cite{Tuyen_Toan_6}, 39., p. 12]
	Tìm $x\in\mathbb{N}$ biết: (a) $5^x + 5^{x + 2} = 650$. (b) $3^{x + 4} = 9^{2x - 1}$.
\end{baitoan}

\begin{baitoan}[\cite{Tuyen_Toan_6}, 40., p. 12]
	Tìm $x\in\mathbb{N}$ biết: (a) $2^x - 15 = 17$. (b) $(7x - 11)^3 = 2^5\cdot5^2 + 200$.
\end{baitoan}

\begin{baitoan}[\cite{Tuyen_Toan_6}, 41., p. 12]
	Tìm $x\in\mathbb{N}$ biết: (a) $x^{10} = 1^x$. (b) $x^{10} = x$. (c) $(2x - 15)^5 = (2x - 15)^3$.
\end{baitoan}

\begin{baitoan}[\cite{Tuyen_Toan_6}, 42., p. 12]
	Tìm $m,n\in\mathbb{N}$ thỏa $2^m + 2^n = 40$.
\end{baitoan}

\begin{baitoan}[\cite{Tuyen_Toan_6}, 43., p. 12]
	Số $4^6\cdot5^{14}$ có bao nhiêu chữ số nếu viết trong hệ thập phân ở dạng thông thường (không có số mũ)?
\end{baitoan}

\begin{baitoan}[\cite{Tuyen_Toan_6}, 44., p. 12]
	Trong âm nhạc, về trường đột thì: 1 nốt tròn bằng 2 nốt trắng, 1 nốt trắng bằng 2 nốt đen, 1 nốt đen bằng 2 nốt móc đơn, 1 nốt móc đơn bằng 2 nốt móc kép, 1 nốt móc kép bằng 2 nốt móc 3, 1 nốt móc 3 bằng 2 nốt móc 4. Dùng lũy thừa của 1 số tự nhiên để: (a) Diễn tả mối quan hệ về trường độ giữa nốt tròn với các nốt nhạc khác. (b) Cho biết nốt nhạc có trường độ gấp $8$ lần nốt móc 3 là nốt nhạc nào?
\end{baitoan}

\begin{baitoan}[\cite{Tuyen_Toan_6}, 45., p. 12, Phân bào]
	Tế bào lớn lên đến 1 kích thước nhất định thì phân chia thành $2$ tế bào con. Mỗi tế bào con tiếp tục lớn lên cho đến khi bằng tế bào mẹ, sau đó phân chia thành $2$ tế bào, quá trình cứ thế tiếp tục. Cho biết: (a) Số tế bào con sau lần phân chia thứ $3$, thứ $5$, thứ $n\in\mathbb{N}^\star$. Viết kết quả dưới dạng lũy thừa cơ số $2$. (b) Sau mấy lần phân chia thì số tế bào con là $128$?
\end{baitoan}
Về phân bào, see, e.g., \href{https://vi.wikipedia.org/wiki/Ph%C3%A2n_b%C3%A0o}{Wikipedia{\tt/}phân bào} \& \href{https://en.wikipedia.org/wiki/Spindle_apparatus}{Wikipedia{\tt/}spindle apparatus}.

\begin{baitoan}[\cite{Binh_Toan_6_tap_1}, VD12, p. 14]
	Không dùng máy tính, chứng minh $A = 215216217\cdot218218220$ là số có $17$ chữ số.
\end{baitoan}

\begin{baitoan}[\cite{Binh_Toan_6_tap_1}, VD13, p. 14]
	Sử dụng nhận xét $2^{10} = 1024\approx10^3$, chứng minh $2^{64}$ có vào khoảng $20$ chữ số.
\end{baitoan}

\begin{baitoan}[\cite{Binh_Toan_6_tap_1}, VD14, p. 15]
	Chứng minh $A = 4 + \sum_{i=2}^{20} 2^i = 4 + 2^2 + 2^3 + 2^4 + \cdots + 2^{20}$.
\end{baitoan}

\begin{baitoan}[\cite{Binh_Toan_6_tap_1}, VD15, p. 15]
	So sánh $63^7$ \& $16^{12}$.
\end{baitoan}

\begin{baitoan}[\cite{Binh_Toan_6_tap_1}, VD16, p. 15]
	(a) Với 3 chữ số $2$, viết thành 1 số tự nhiên có giá trị lớn nhất. (b) Cũng hỏi như vậy đối với 3 chữ số $4$. (c) Cũng hỏi như vậy đối với 3 chữ số $a\in\mathbb{N}$.
\end{baitoan}

\begin{baitoan}[\cite{Binh_Toan_6_tap_1}, VD17, p. 16]
	Số $2^2$ \& $5^2$ viết liền nhau được số $425$ có 3 chữ số, số $2^3$ \& $5^3$ viết liền nhau được số $8125$ có 4 chữ số, số $2^4$ \& $5^4$ viết liền nhau được số $16625$ có 5 chữ số. Chứng minh số $2^{1991}$ \& $5^{1991}$ viết liền nhau được số có $1992$ chữ số.
\end{baitoan}

\begin{baitoan}[\cite{Binh_Toan_6_tap_1}, 71., p. 16]
	Tính: (a) $4^{10}\cdot8^{15}$. (b) $4^{15}\cdot5^{30}$. (c) $27^{16}:9^{10}$. (d) $A = \dfrac{72^3\cdot54^2}{108^4}$. (e) $B = \dfrac{3^{10}\cdot11 + 3^{10}\cdot5}{3^9\cdot2^4}$.
\end{baitoan}

\begin{baitoan}[\cite{Binh_Toan_6_tap_1}, 72., p. 16]
	Tính giá trị của biểu thức: (a) $\dfrac{2^{10}\cdot13 + 2^{10}\cdot65}{2^8\cdot104}$.\\(b) $(1 + 2 + 3 + \cdots + 100)(1^2 + 2^2 + 3^2 + \cdots + 10^2)(65\cdot111 - 13\cdot15\cdot37)$.
\end{baitoan}

\begin{baitoan}[\cite{Binh_Toan_6_tap_1}, 73., p. 16]
	Tìm $x\in\mathbb{N}$ biết: (a) $2^x\cdot4 = 128$. (b) $x^{15} = x$. (c) $(2x + 1)^3 = 125$. (d) $(x - 5)^4 = (x - 5)^6$.
\end{baitoan}

\begin{baitoan}[\cite{Binh_Toan_6_tap_1}, 74., p. 16]
	Cho $A = \sum_{i=1}^{100} 3^i = 3 + 3^2 + 3^3 + \cdots + 3^{100}$. Tìm $n\in\mathbb{N}$ biết $2A + 3 = 3^n$.
\end{baitoan}

\begin{baitoan}[\cite{Binh_Toan_6_tap_1}, 75., p. 16]
	Tính tổng của $100$ số: $\sum_{i=1}^{100} \underbrace{9}_i = 9+ 99 + 999 + \cdots + \underbrace{99\ldots9}_{100}$.
\end{baitoan}

\begin{baitoan}[\cite{Binh_Toan_6_tap_1}, 76., p. 16]
	Tìm số tự nhiên có 3 chữ số, biết bình phương của chữ số hàng chục bằng tích của 2 chữ số kia \& số tự nhiên đó trừ đi số gồm 3 chữ số ấy viết theo thứ tự ngược lại bằng $495$.
\end{baitoan}

\begin{baitoan}[\cite{Binh_Toan_6_tap_1}, 77., pp. 16--17]
	(a) Viết dãy tính có kết quả bằng $1000000$, với 5 chữ số như nhau cùng với dấu các phép tính \& dấu ngoặc nếu cần. (b) Cũng hỏi như vậy với 6 chữ số khác nhau.
\end{baitoan}

\begin{baitoan}[\cite{Binh_Toan_6_tap_1}, 78., p. 17]
	So sánh $A$ \& $B$: (a) $A = \sum_{i=1}^{1000} i = 1 + 2 + 3 + \cdots + 1000$, $B = 11! = \prod_{i=1}^{11} = 1\cdot2\cdot3\cdots11$. (b) $A = 20! = \prod_{i=1}^{20} = 1\cdot2\cdot3\cdots20$, $B = \sum_{i=1}^{1000000} i = 1 + 2 + 3 + \cdots + 1000000$.
\end{baitoan}

\begin{baitoan}[\cite{Binh_Toan_6_tap_1}, 79., p. 17]
	So sánh: (a) $3^{500}$ \& $7^{300}$. (b) $8^5$ \& $3\cdot4^7$. (c) $99^{20}$ \& $9999^{10}$. (d) $202^{303}$ \& $303^{202}$. (e) $3^{21}$ \& $2^{31}$. (f) $11^{1979}$ \& $37^{1320}$. (g) $10^10$ \& $48\cdot50^5$. (h) $1990^{10} + 1990^9 + 1991^{10}$.
\end{baitoan}

\begin{baitoan}[\cite{Binh_Toan_6_tap_1}, 80., p. 17]
	So sánh: (a) $5^{299}$ \& $3^{501}$. (b) $3^{23}$ \& $5^{15}$. (c) $127^{23}$ \& $513^{18}$.
\end{baitoan}

\begin{baitoan}[\cite{Binh_Toan_6_tap_1}, 81., p. 17]
	 Chứng minh: $5^{27} < 2^{63} < 5^{28}$.
\end{baitoan}

\begin{baitoan}[\cite{Binh_Toan_6_tap_1}, 82., p. 17]
	Viết liền nhau các kết quả của các lũy thừa $4^{50}$ \& $25^{50}$, ta được 1 số tự nhiên có bao nhiêu chữ số?
\end{baitoan}

\begin{baitoan}[\cite{Binh_Toan_6_tap_1}, 83., p. 17]
	Tìm số tự nhiên có 4 chữ số biết số đó có thể phân tích thành tích của 2 thừa số có tổng bằng $100$ \& 1 trong 2 thừa số ấy có dạng $a^a$.
\end{baitoan}

%------------------------------------------------------------------------------%

\subsection{Compare Exponentiations -- So Sánh Các Lũy Thừa}

\begin{baitoan}[\cite{Tuyen_Toan_6}, VD8, p. 13]
	So sánh $3^7$ \& $2^{11}$.
\end{baitoan}

\begin{baitoan}[\cite{Tuyen_Toan_6}, VD9, p. 13]
	So sánh $16^{19}$ \& $8^{25}$.
\end{baitoan}

\begin{baitoan}[\cite{Tuyen_Toan_6}, VD10, p. 13]
	So sánh $3^{4040}$ \& $2^{6060}$.
\end{baitoan}

\begin{baitoan}[\cite{Tuyen_Toan_6}, 46., p. 14]
	So sánh: (a) $27^{11}$ \& $81^8$. (b) $625^5$ \& $125^7$.
\end{baitoan}

\begin{baitoan}[\cite{Tuyen_Toan_6}, 47., p. 14]
	So sánh: (a) $5^{36}$ \& $11^{24}$. (b) $3^{2n}$ \& $2^{3n}$, $\forall n\in\mathbb{N}^\star$.
\end{baitoan}

\begin{baitoan}[\cite{Tuyen_Toan_6}, 48., p. 14]
	So sánh $A = 2\cdot3^{54}$ \& $B = 6\cdot5^{32}$.
\end{baitoan}

\begin{baitoan}[\cite{Tuyen_Toan_6}, 49., p. 14]
	Chứng minh: $5^{60n} < 2^{140n} < 3^{100n}$, $\forall n\in\mathbb{N}^\star$.
\end{baitoan}

\begin{baitoan}[\cite{Tuyen_Toan_6}, 50., p. 14]
	Sắp xếp 3 số $3^{539},7^{308},2^{847}$ theo thứ tự tăng dần.
\end{baitoan}

\begin{baitoan}[\cite{Tuyen_Toan_6}, 51., p. 14]
	So sánh: (a) $5^{75}$ \& $7^{60}$. (b) $3^{21}$ \& $2^{31}$.
\end{baitoan}

\begin{baitoan}[\cite{Tuyen_Toan_6}, 52., p. 14]
	So sánh: (a) $5^{23}$ \& $6\cdot5^{22}$. (b) $7\cdot2^{13}$ \& $2^{16}$. (c) $21^{15}$ \& $27^5\cdot49^8$.
\end{baitoan}

\begin{baitoan}[\cite{Tuyen_Toan_6}, 53., p. 14]
	So sánh: (a) $199^{20}$ \& $2003^{15}$. (b) $3^{39}$ \& $11^{21}$.
\end{baitoan}

\begin{baitoan}[\cite{Tuyen_Toan_6}, 54., p. 14]
	So sánh 2 hiệu $A = 72^{45} - 72^{44}$ \& $B = 72^{44} - 72^{43}$.
\end{baitoan}

\begin{baitoan}[\cite{Tuyen_Toan_6}, 55., p. 14]
	Tìm $x\in\mathbb{N}$ biết: (a) $16^x < 128^4$. (b) $5^x\cdot5^{x + 1}\cdot5^{x + 2}\le1\underbrace{00\ldots0}_{18}:2^{18}$.
\end{baitoan}

\begin{baitoan}[\cite{Tuyen_Toan_6}, 56., p. 14]
	Tìm $n\in\mathbb{N}$ biết $2^5\cdot3^n\cdot3^{n + 2}\le32\cdot3^6\cdot3^4$.
\end{baitoan}

\begin{baitoan}[\cite{Tuyen_Toan_6}, 57., p. 14]
	So sánh $A = \sum_{i=0}^9 2^i = 1 + 2 + 2^2 + 2^3 + \cdots + 2^9$ \& $B = 5\cdot2^8$.
\end{baitoan}

\begin{baitoan}[\cite{Tuyen_Toan_6}, 58., p. 14]
	Viết số lớn nhất bằng cách dùng $3$ chữ số $1,2,3$ với điều kiện mỗi chữ số chỉ dùng 1 lần.
\end{baitoan}

%------------------------------------------------------------------------------%

\subsection{Last Digit of Products \& Exponentiations -- Chữ Số Tận Cùng của Các Tích \& Lũy Thừa}

\begin{baitoan}[\cite{Tuyen_Toan_6}, VD11, p. 15]
	Cho tổng $A = 9531^m + 246^n$ với $m,n\in\mathbb{N}^\star$. Hỏi tổng $A$ có phải là số chính phương không?
\end{baitoan}

\begin{baitoan}[\cite{Tuyen_Toan_6}, VD12, p. 15]
	Cho $B = 559^{361} - 7^{202}$. Chứng minh $B\divby10$.
\end{baitoan}

\begin{baitoan}[\cite{Tuyen_Toan_6}, VD13, pp. 15--16]
	Ngoài Dương lịch, Âm lịch, còn ghi lịch theo hệ đếm Can Chi, e.g., Nhâm Ngọ, Quý Mùi, Giáp Thân, $\ldots$ Chữ thứ nhất chỉ hàng Can, chữ thứ 2 chỉ hàng Chi. Có $10$ Can là: Giáp, Ất, Bính, Đinh, Mậu, Kỷ, Canh, Tân, Nhâm, Quý. Có $12$ Chi là: Tý, Sửu, Dần, Mão, Thìn, Tỵ, Ngọ, Mùi, Thân, Dậu, Tuất, Hợi. Muốn tìm hàng Can của 1 năm ta chỉ cần xét chữ số tận cùng của năm dương lịch rồi đối chiếu với bảng:
	\begin{table}[H]
		\centering
		\begin{tabular}{|c|c|c|c|c|c|c|c|c|c|c|}
			\hline
			Hàng can & Giáp & Ất & Bính & Đinh & Mậu & Kỷ & Canh & Tân & Nhâm & Quý \\
			\hline
			Chữ số tận cùng của năm dương lịch & 4 & 5 & 6 & 7 & 8 & 9 & 0 & 1 & 2 & 3 \\
			\hline
		\end{tabular}
	\end{table}
	Muốn tìm hàng Chi của 1 năm ta dùng công thức
	\begin{align*}
		\mbox{Hàng Chi} = \mbox{Số dư của }\frac{\mbox{năm} - 4}{12} + 1.
	\end{align*}
	Rồi đối chiếu kết quả với bảng:
	\begin{table}[H]
		\centering
		\begin{tabular}{|c|c|c|c|c|c|c|c|c|c|c|c|c|}
			\hline
			Hàng chi & Tý & Sửu & Dần & Mão & Thìn & Tỵ & Ngọ & Mùi & Thân & Dậu & Tuất & Hợi \\
			\hline
			Mã số & 1 & 2 & 3 & 4 & 5 & 6 & 7 & 8 & 9 & 10 & 11 & 12 \\
			\hline
		\end{tabular}
	\end{table}
	Năm 2010 kỷ niệm $1000$ năm Thăng Long--Hà Nội, tính xem năm đó là năm nào theo hệ đếm Can Chi? Chú ý: Vì năm dương lịch không trùng hoàn toàn với năm âm lịch nên đối với 2 tháng đầu của năm dương lịch thì còn phải chỉnh chút ít.
\end{baitoan}

\begin{baitoan}[\cite{Tuyen_Toan_6}, 59., p. 16]
	Tìm chữ số tận cùng của tổng $A = 1\cdot3\cdot5\cdots99 + 2\cdot4\cdot6\cdots98$.
\end{baitoan}

\begin{baitoan}[\cite{Tuyen_Toan_6}, 60., p. 16]
	Có 5 số tự nhiên nào mà tích bằng $2021$ \& tổng có tận cùng bằng $8$ không?
\end{baitoan}

\begin{baitoan}[\cite{Tuyen_Toan_6}, 61., p. 16]
	Tích các số lẻ liên tiếp có tận cùng là $7$. Hỏi tích đó có bao nhiêu thừa số?
\end{baitoan}

\begin{baitoan}[\cite{Tuyen_Toan_6}, 62., p. 16]
	Tìm chữ số tận cùng của các lũy thừa: $87^{32},58^{33},23^{35},74^{30},49^{31}$.
\end{baitoan}

\begin{baitoan}[\cite{Tuyen_Toan_6}, 63., p. 16]
	Chia mỗi số sau cho $100$ được số dư là bao nhiêu? (a) $7^{2025}$. (b) $6^{1202}$.
\end{baitoan}

\begin{baitoan}[\cite{Tuyen_Toan_6}, 64., p. 16]
	Tìm chữ số tận cùng của các số sau: (a) $234^{5^{6^7}}$. (b) $6^{1202}$.
\end{baitoan}

\begin{baitoan}[\cite{Tuyen_Toan_6}, 65., p. 16]
	Chứng minh các tổng hoặc hiệu sau không chia hết cho $10$: (a) $A = 98\cdot96\cdot94\cdot92 - 91\cdot93\cdot95\cdot97$. (b) $B(m,n) = 405^n + 2^{405} + m^2$, với $\forall m,n\in\mathbb{N}$, $n\ne0$.
\end{baitoan}

\begin{baitoan}[\cite{Tuyen_Toan_6}, 66., p. 17]
	Cho $P = \left(\prod_{i=1}^{10} 2^i\right)\left(\prod_{i=1}^7 5^{2i}\right) = (2\cdot2^2\cdot2^3\cdots2^{10})(5^2\cdot5^4\cdot5^6\cdots5^{14})$ tận cùng bằng bao nhiêu chữ số $0$?
\end{baitoan}

\begin{baitoan}[\cite{Tuyen_Toan_6}, 67., p. 17]
	Cho $S = \sum_{i=0}^{30} 3^i = 1 + 3 + 3^2 + 3^3 + \cdots + 3^{30}$. Tìm chữ số tận cùng của $S$, từ đó suy ra $S$ không phải là số chính phương.
\end{baitoan}

\begin{baitoan}[\cite{Tuyen_Toan_6}, 68., p. 17]
	Nước Việt Nam Dân chủ Cộng hòa ra đời ngay sau Cách mạng tháng 8 năm 1945. Trong hệ đếm Can Chi năm 1945 là năm nào?
\end{baitoan}

\begin{baitoan}[\cite{Tuyen_Toan_6}, 69., p. 17]
	Chiến thắng Đống Đa giải phóng Thăng Long vào mùa xuân năm 1789. Năm đó là năm nào trong hệ đếm Can Chi?
\end{baitoan}

%------------------------------------------------------------------------------%

\section{Problem on Naturals -- Bài Toán về Số Tự Nhiên}

\begin{baitoan}[\cite{Binh_Toan_6_tap_1}, VD18, p. 17]
	Tuổi anh hiện nay gấp 3 lần tuổi em trước kia, lúc anh bằng tuổi em hiện nay. Khi tuổi em bằng tuổi anh hiện nay thì tổng số tuổi của 2 người sẽ là $28$. Tính tuổi của mỗi người hiện nay.
\end{baitoan}

\begin{baitoan}[\cite{Binh_Toan_6_tap_1}, VD19, p. 18]
	3 ôtô chở tổng cộng $50$ chuyến, gồm $118$ tấn hàng. Mỗi chuyến, xe thứ nhất chở $2$ tấn, xe thứ 2 chở $2.5$ tấn, xe thứ 3 chở $3$ tấn. Hỏi mỗi xe chở bao nhiêu chuyến biết số chuyến xe thứ nhất gấp rưỡi số chuyến xe thứ 2?
\end{baitoan}

\begin{baitoan}[\cite{Binh_Toan_6_tap_1}, VD20, p. 19]
	Anh Lâm nói: ``Năm 1990, tuổi mình đúng bằng tổng các chữ số của năm sinh''. Tính xem anh Lâm sinh năm nào.
\end{baitoan}

\subsection{Sum, difference, \& ratio -- Tìm các số biết tổng \& các hiệu, biết tổng (hiệu) \& các tỷ số}

\begin{baitoan}[\cite{Binh_Toan_6_tap_1}, 84., p. 19]
	Lương Thế Vinh là nhà toán học nổi tiếng của nước ta thời xưa. Năm sinh của ông rất đặc biệt, đó là 1 số có 4 chữ số, chữ số hàng nghìn bằng chữ số hàng đơn vị, chữ số hàng trăm bằng chữ số hàng chục \& tổng của 4 chữ số bằng $10$. Tính năm sinh của ông.
\end{baitoan}

\begin{baitoan}[\cite{Binh_Toan_6_tap_1}, 85., p. 19]
	Tìm 4 số tự nhiên chẵn liên tiếp có tổng bằng $5420$.
\end{baitoan}

\begin{baitoan}[\cite{Binh_Toan_6_tap_1}, 86., p. 19]
	Tìm 3 số biết: Tổng của số thứ nhất \& số thứ 2 bằng $56$, tổng của số thứ 2 \& số thứ 3 bằng $64$, tổng của số thứ 3 \& số thứ nhất bằng $78$.
\end{baitoan}

\begin{baitoan}[\cite{Binh_Toan_6_tap_1}, 87., p. 19]
	Tìm 3 số tự nhiên lẻ liên tiếp biết tổng của số lớn nhất \& số nhỏ nhất bằng $114$.
\end{baitoan}

\begin{baitoan}[\cite{Binh_Toan_6_tap_1}, 88., p. 19]
	2 ngăn sách lúc đầu có tổng cộng $118$ cuốn. Sau khi lấy đi $8$ cuốn ở ngăn {\rm I}, thêm $10$ cuốn vào ngăn {\rm II} thì số sách ở ngăn {\rm II} gấp đôi số sách ở ngăn {\rm I}. Tính số sách ở mỗi ngăn lúc đầu.
\end{baitoan}

\begin{baitoan}[\cite{Binh_Toan_6_tap_1}, 89., p. 19]
	Tìm số tự nhiên tận cùng bằng $7$ biết sau khi xóa chữ số $7$ đó thì số ấy giảm đi $484$ đơn vị.
\end{baitoan}

\begin{baitoan}[\cite{Binh_Toan_6_tap_1}, 90., p. 19]
	Hiệu của 2 số bằng $1217$. Nếu tăng số trừ gấp 4 lần thì được số lớn hơn số bị trừ là $376$. Tìm số bị trừ, số trừ.
\end{baitoan}

\begin{baitoan}[\cite{Binh_Toan_6_tap_1}, 91., p. 19]
	1 vườn hình chữ nhật có chu vi {\rm356 m}. Tính chiều dài \& chiều rộng của vườn biết nếu viết thêm chữ số $1$ vào trước số đo chiều dài thì được số đo chiều dài.
\end{baitoan}

\begin{baitoan}[\cite{Binh_Toan_6_tap_1}, 92., pp. 19--20]
	Bài toán cổ Hy Lạp:
	\begin{center}
		Lừa \& ngựa thồ hàng ra chợ,\\Ngựa thở than mình chở quá nhiều.\\Lừa rằng: ``Anh chớ lắm điều!\\Tôi đây mới bị chất nhiều làm sao!\\Anh đưa tôi 1 bao mang bớt\\Thì tôi thồ nhiều gấp đôi anh\\Chính tôi phải trút cho anh\\1 bao gánh đỡ mới thành bằng nhau''.
	\end{center}
	Hỏi lừa, ngựa chở mấy bao?
\end{baitoan}

\begin{baitoan}[\cite{Binh_Toan_6_tap_1}, 93., p. 20]
	Tìm số bị chia \& số chia của 1 phép chia, biết thương bằng $6$, số dư bằng $49$, tổng của số bị chia, số chia \& số dư bằng $595$.
\end{baitoan}

\begin{baitoan}[\cite{Binh_Toan_6_tap_1}, 94., p. 20]
	Tìm số tự nhiên biết nếu viết thêm chữ số $2$ vào sau chữ số hàng đơn vị thì số ấy tăng thêm $2000$ đơn vị.
\end{baitoan}

\begin{baitoan}[\cite{Binh_Toan_6_tap_1}, 95., p. 20]
	Mẹ hơn con $28$ tuổi. Sau $5$ năm nữa, tuổi mẹ gấp 3 tuổi con. Tính tuổi mẹ \& con hiện nay.
\end{baitoan}

\begin{baitoan}[\cite{Binh_Toan_6_tap_1}, 96., p. 20]
	Con $10$ tuổi, bố $40$ tuổi. Sau mấy năm nữa, tuổi bố gấp 3 tuổi con?
\end{baitoan}

\begin{baitoan}[\cite{Binh_Toan_6_tap_1}, 97., p. 20]
	Năm 2000, bố $40$ tuổi, Mai $11$ tuổi, em Nam $5$ tuổi. Đến năm nào, tuổi bố bằng tổng số tuổi của 2 chị em?
\end{baitoan}

\begin{baitoan}[\cite{Binh_Toan_6_tap_1}, 98., p. 20]
	Năm 2000, mẹ $36$ tuổi, 2 con $7$ tuổi \& $12$ tuổi. Bắt đầu từ năm nào, tuổi mẹ ít hơn tổng số tuổi của 2 con?
\end{baitoan}

\begin{baitoan}[\cite{Binh_Toan_6_tap_1}, 99., p. 20]
	Anh hơn em $3$ tuổi. Tuổi anh hiện nay gấp rưỡi tuổi em, lúc anh bằng tuổi em hiện nay. Tính tuổi hiện nay của mỗi người.
\end{baitoan}

\subsection{Additional hypothesis -- Giải toán bằng phương pháp giả thiết tạm}

\begin{baitoan}[\cite{Binh_Toan_6_tap_1}, 100., p. 20]
	1 số học sinh xếp hàng $12$ thì thừa $5$ học sinh, còn xếp hàng $15$ cũng thừa $5$ học sinh \& ít hơn trước là $4$ hàng. Tính số học sinh.
\end{baitoan}

\begin{baitoan}[\cite{Binh_Toan_6_tap_1}, 101., p. 20]
	Có 1 số học sinh \& 1 số thuyền. Nếu xếp $4$ học sinh 1 thuyền thì thừa $3$ học sinh chưa có chỗ. Nếu xếp $5$ học sinh 1 thuyền thì thừa ra 1 thuyền. Tính số học sinh \& số thuyền.
\end{baitoan}

\begin{baitoan}[\cite{Binh_Toan_6_tap_1}, 102., p. 20]
	An vào cửa hàng mua $12$ vở \& $4$ bút chì hết $36000$ đồng. Bích mua $8$ vở \& $5$ bút chì cùng loại hết $27500$ đồng. Tính giá 1 quyển vở, giá 1 bút chì.
\end{baitoan}

\begin{baitoan}[\cite{Binh_Toan_6_tap_1}, 103., p. 20]
	Trong 1 đợt trồng cây, lớp 6A trồng $26$ cây, lớp 6B trồng $29$ cây, lớp 6C trồng $32$ cây. Số cây lớp $6D$ trồng nhiều hơn trung bình cộng của 4 lớp là $3$ cây. Tính số cây lớp 6D trồng.
\end{baitoan}

\begin{baitoan}[\cite{Binh_Toan_6_tap_1}, 104., p. 20]
	Bơm nước vào 1 bể: dùng máy {\rm I} trong $30$ phút, dùng máy {\rm II} trong $20$ phút. Tính xem trong mỗi phút mỗi máy bơm được bao nhiêu lít nước, biết mỗi phút máy {\rm II} bơm được nhiều hơn máy {\rm I} là {\rm50 L} \& tổng cộng 2 máy bơm được {\rm21000 L} nước?
\end{baitoan}

\begin{baitoan}[\cite{Binh_Toan_6_tap_1}, 105., p. 20]
	1 tổ may phải may $1800$ chiếc cả quần \& áo trong $13$ giờ. Trong $8$ giờ đầu tổ may áo \& trong thời gian còn lại tổ may quần. Biết trong $1$ giờ, tổ may được số áo nhiều hơn số quần là $30$ chiếc. Tính số áo \& số quần tổ đã may.
\end{baitoan}

\begin{baitoan}[\cite{Binh_Toan_6_tap_1}, 106., p. 21]
	Đố:
	\begin{center}
		Quýt, cam 17 quả tươi\\Đem chia cho 100 người cùng vui.\\Chia 3 mỗi quả quýt rồi,\\Còn cam mỗi quả chia 10 vừa xinh.\\Trăm người, trăm miếng ngọt lành,\\Quýt, cam mỗi loại tính rành là bao?
	\end{center}
\end{baitoan}

\begin{baitoan}[\cite{Binh_Toan_6_tap_1}, 107., p. 21]
	1 đội bóng thi đấu $25$ trận, chỉ có thắng \& hòa, mỗi trận thắng được $3$ điểm, mỗi trận hòa được $1$ điểm, kết quả đội đó được $59$ điểm. Tính số trận thắng, số trận hòa của đội bóng.
\end{baitoan}

\begin{baitoan}[\cite{Binh_Toan_6_tap_1}, 108., p. 21]
	(a) 1 cuộc thi có $20$ câu hỏi, mỗi câu trả lời đúng được $5$ điểm, mỗi câu trả lời sai bị trừ đi $1$ điểm. 1 đội học sinh dự thi đạt $52$ điểm. Hỏi đội đó trả lời đúng mấy câu, sai mấy câu? (b) 1 người làm gia công $45$ sản phẩm, mỗi chiếc làm đúng quy cách được $800$ đồng, mỗi chiếc làm sai quy cách phải đền $1200$ đồng. Tính ra người đó được lĩnh $30000$ đồng. Hỏi người đó làm bao nhiêu sản phẩm đúng quy cách?
\end{baitoan}

\begin{baitoan}[\cite{Binh_Toan_6_tap_1}, 109., p. 21]
	1 câu lạc bộ có $22$ chiếc ghế gồm 3 loại: ghế 3 chân, ghế 4 chân, ghế 6 chân. Tính số ghế mỗi loại, biết tổng số chân ghế bằng $100$ \& số ghế 6 chân gấp đôi số ghế 3 chân.
\end{baitoan}

\begin{baitoan}[\cite{Binh_Toan_6_tap_1}, 110., p. 21]
	1 số tiền trị giá $224000$ đồng gồm các loại tiền $5000$ đồng, $2000$ đồng, $1000$ đồng, tổng cộng $130$ tờ. Biết số tờ $1000$ đồng gấp 5 số tờ $5000$ đồng. Tính số tờ tiền mỗi loại.
\end{baitoan}

\begin{baitoan}[\cite{Binh_Toan_6_tap_1}, 111., p. 21]
	Có $25$ gói đường gồm 3 loại: gói $5$ lạng, gói $2$ lạng, gói $1$ lạng, có khối lượng tổng cộng là $56$ lạng. Biết số gói $1$ lạng gấp đôi số gói $5$ lạng. Tính số gói mỗi loại.
\end{baitoan}

\begin{baitoan}[\cite{Binh_Toan_6_tap_1}, 112., p. 21]
	1 hộp có thể chứa được vừa vặn $25$ gói bánh hoặc $30$ gói kẹo. Xếp $28$ gói cả bánh \& kẹo thì vừa đầy hộp đó. Biết giá tiền bánh \& kẹo đều bằng nhau \& bằng $36000$ đồng. Tính giá 1 gói bánh, 1 gói kẹo.
\end{baitoan}

\subsection{Selection method -- Toán giải bằng phương pháp lựa chọn}

\begin{baitoan}[\cite{Binh_Toan_6_tap_1}, 113., p. 21]
	Tìm số tự nhiên có 3 chữ số, biết tổng 6 số tự nhiên có 2 chữ số lập bởi 2 trong 3 chữ số ấy gấp đôi số phải tìm.
\end{baitoan}

\begin{baitoan}[\cite{Binh_Toan_6_tap_1}, 114., p. 21]
	(a) Tìm 3 chữ số khác nhau \& khác $0$, biết tổng các số tự nhiên có 3 chữ số gồm cả 3 chữ số ấy bằng $1554$. (b) Tìm 3 chữ số khác nhau \& khác $0$, biết tổng các số tự nhiên có 3 chữ số gồm cả 3 chữ số ấy bằng $2886$, còn hiệu giữa số lớn nhất \& số nhỏ nhất bằng $495$. (c) Có 3 tờ bìa ghi các số $23,79$, \& $\overline{ab}$. Xếp 3 tờ bìa thành 1 hàng thì được số có 6 chữ số. Cộng tất cả các số 6 chữ số đó lại (bằng cách đổi chỗ các tờ bìa) thì được $2989896$. Tìm số $\overline{ab}$.
\end{baitoan}

\begin{baitoan}[\cite{Binh_Toan_6_tap_1}, 115., p. 21]
	Có 5 người cân theo từng cặp 2 người. Số cân nặng (tính bằng {\rm kg}) trong $10$ lượt cân xếp từ lớn đến nhỏ là: $129,125,124,13,122,121,120,118,116,114$. Tính cân nặng của mỗi người.
\end{baitoan}

\begin{baitoan}[\cite{Binh_Toan_6_tap_1}, 116., p. 22]
	Bé Mai nhận thấy nếu ghép tuổi mình với tuổi của em Thu thì được tuổi của bà. Biết tổng số tuổi của 3 bà cháu là $85$. Tính tuổi mỗi người.
\end{baitoan}

\begin{baitoan}[\cite{Binh_Toan_6_tap_1}, 117., p. 22, Đố vui: những năm sinh đặc biệt]
	Ngày đầu năm 1991, bác Nam hỏi anh Việt:
	
	- Năm nay cháu bao nhiêu tuổi rồi?
	
	- Tuổi cháu năm nay đúng bằng tổng các chữ số của năm sinh -- Anh Việt trả lời.
	
	Thế mà bác Nam tính ngay ra tuổi của anh Việt. Bác gật gù nói:
	
	- Lúc bác bằng tuổi cháu hiện nay, bác đang tham gia kháng chiến chống Pháp, \& năm ấy cũng có tổng các chữ số bằng tuổi cháu.
	
	Anh Việt cũng tính đúng tuổi của bác Nam. Hỏi anh Việt \& bác Nam sinh năm nào?
\end{baitoan}

%------------------------------------------------------------------------------%

\section{Order of Operations -- Thứ Tự Thực Hiện Các Phép Tính}

\begin{baitoan}[\cite{Tuyen_Toan_6}, VD14, p. 17]
	Dùng 5 chữ số $1$ \& dấu của các phép tính kể cả dấu ngoặc để viết thành 1 biểu thức có giá trị bằng $100$.
\end{baitoan}

\begin{baitoan}[\cite{Tuyen_Toan_6}, VD15, p. 17]
	1 quyển sách giá khoa có $172$ trang. Hỏi phải dùng tất cả bao nhiêu chữ số để đánh số các trang của quyển sách này?
\end{baitoan}

\begin{baitoan}[\cite{Tuyen_Toan_6}, VD16, p. 18]
	Cho $S = 8 + 12 + 16 + \cdots + 96 +100$. (a) Tổng này có bao nhiêu số hạng? (b) Tính $S$.
\end{baitoan}

\begin{baitoan}[\cite{Tuyen_Toan_6}, 70., p. 18]
	Tính: (a) $[400 - (40:2^3 + 3\cdot5^3)]:5$. (b) $(37 + 18)\cdot\{3250 - 15^2\cdot[(4^4 - 2^5):16]\}$.
\end{baitoan}

\begin{baitoan}[\cite{Tuyen_Toan_6}, 71., p. 18]
	Tính: (a) $(10^2 + 11^2 + 12^2):(13^2 + 14^2)$. (b) $9! - 8! - 7!\cdot8^2$. (c) $\dfrac{(3\cdot4\cdot2^{16})^2}{11\cdot2^{13}\cdot4^{11} - 16^9}$.
\end{baitoan}

\begin{baitoan}[\cite{Tuyen_Toan_6}, 72., p. 18]
	Tìm $x\in\mathbb{N}$ biết: (a) $(19x + 2\cdot5^2):14 = (13 - 8)^2 - 4^2$. (b) $2\cdot3^x = 10\cdot3^{12} + 8\cdot27^4$. 
\end{baitoan}

\begin{baitoan}[\cite{Tuyen_Toan_6}, 73., p. 18]
	Tìm $x\in\mathbb{N}$ biết: (a) $(5^2\cdot23 - 5^2\cdot13)x = 6\cdot5^3$. (b) $x^2 - [666:(24 + 13)] = 7$.
\end{baitoan}

\begin{baitoan}[\cite{Tuyen_Toan_6}, 74., p. 18]
	Dùng 6 chữ số $1$ cùng với dấu của các phép tính \& dấu ngoặc (nếu cần) để viết thành 1 biểu thức có giá trị là $100$.
\end{baitoan}

\begin{baitoan}[\cite{Tuyen_Toan_6}, 75., p. 18]
	Với 6 chữ số $3$ cũng với dấu của các phép tính \& dấu ngoặc (nếu cần), viết 1 biểu thức có giá trị là $1000000$.
\end{baitoan}

\begin{baitoan}[\cite{Tuyen_Toan_6}, 76., p. 18]
	Cho biểu thức $252 - 84:21 + 7$. (a) Tính giá trị của biểu thức đó. (b) Nếu dùng thêm dấu ngoặc thì có thể được các giá trị nào khác?
\end{baitoan}

\begin{baitoan}[\cite{Tuyen_Toan_6}, 77., p. 18]
	Cho $S = 7 + 10 + 13 + \cdots + 97 + 100$. (a) Tổng trên có bao nhiêu số hạng? (b) Tìm số hạng thứ $22$. (c) Tính $S$.
\end{baitoan}

\begin{baitoan}[\cite{Tuyen_Toan_6}, 78., p. 18]
	Cho $A$ là tập hợp các số tự nhiên không vượt quá $150$, chia cho $7$ dư $3$. $A = \{x\in\mathbb{N}|x = 7q + 3,\ q\in\mathbb{N},\,x\le150\}$. (a) Liệt kê các phần tử của $A$ thành 1 dãy số từ nhỏ đến lớn. (b) Tính tổng các phần tử của $A$.
\end{baitoan}

\begin{baitoan}[\cite{Tuyen_Toan_6}, 79., p. 18]
	1 quyển sách có $366$ trang. Để đánh số các trang của quyển sách này phải dùng bao nhiêu chữ số?
\end{baitoan}

\begin{baitoan}[\cite{Tuyen_Toan_6}, 80., p. 18]
	Để đánh số các trang của 1 quyển sách phải dùng tất cả $600$ chữ số. Hỏi quyển sách đó có bao nhiêu trang?
\end{baitoan}

\begin{baitoan}[\cite{Tuyen_Toan_6}, 81., p. 19]
	Viết liền nhau dãy các số tự nhiên bắt đầu từ $1$: $1,2,3,\ldots$ Hỏi chữ số thứ $659$ là chữ số nào?
\end{baitoan}

%------------------------------------------------------------------------------%

\section{Cân \& Đong với 1 Số Điều Kiện Hạn Chế}

\begin{baitoan}[\cite{Tuyen_Toan_6}, VD17, p. 19]
	Trong 9 gói hàng có 1 gói nhẹ hơn các gói kia chút ít, còn lại nặng như nhau. Với 1 chiếc cân 2 đĩa \& không có quả cân nào, cân chỉ 2 lần tìm được gói hàng nhẹ hơn.
\end{baitoan}

\begin{baitoan}[Tổng quát \cite{Tuyen_Toan_6}, VD17, p. 19]
	Trong $3^n$ gói hàng có 1 gói nhẹ hơn các gói kia chút ít, còn lại nặng như nhau. Với 1 chiếc cân 2 đĩa \& không có quả cân nào, cân chỉ $n$ lần tìm được gói hàng nhẹ hơn.
\end{baitoan}

\begin{baitoan}[\cite{Tuyen_Toan_6}, VD18, p. 19]
	Có 10 thùng đựng các gói muối trong đó có 9 thùng đựng các gói muối {\rm100 g}, chỉ có 1 thùng đựng các gói không đúng quy cách, mỗi gói chỉ có {\rm90 g}. Làm thế nào chỉ cân 1 lần mà xác định được thùng có các gói muối {\rm90 g}?
\end{baitoan}

\begin{baitoan}[\cite{Tuyen_Toan_6}, VD19, p. 20]
	Có 1 bình {\rm4 L} \& 1 bình {\rm5 L}. Làm thế nào để đong được đúng {\rm3 L} nước từ 1 bể nước?
\end{baitoan}

\begin{baitoan}[\cite{Tuyen_Toan_6}, VD20, p. 20]
	Có 1 bình {\rm5 L} \& 1 bình {\rm3 L}, làm thế nào để đong được đúng {\rm7 L} nước từ 1 bể nước?
\end{baitoan}

\begin{baitoan}[\cite{Tuyen_Toan_6}, 82., p. 21]
	Có {\rm1500 g} đường. Làm thế nào để lấy ra được {\rm250 g} với 1 chiếc cân 2 đĩa \& 1 quả cân {\rm100 g}.
\end{baitoan}

\begin{baitoan}[\cite{Tuyen_Toan_6}, 83., p. 21]
	Trong $27$ chiếc nhẫn có 1 chiếc nặng hơn các chiếc kia chút ít, các chiếc còn lại nặng như nhau. Với 1 chiếc cân 2 đĩa \& không có quả cân, làm thế nào để cân đúng 3 lần xác định được chiếc nhẫn nặng hơn đó.
\end{baitoan}

\begin{baitoan}[\cite{Tuyen_Toan_6}, 84., p. 21]
	Trong $26$ chiếc tẩy có 1 chiếc nặng hơn các chiếc còn lại chút ít, các chiếc còn lại nặng như nhau. Với chiếc cân đĩa \& không có quả cân, làm thế nào để cân đúng $3$ lần xác định được chiếc tẩy nặng hơn đó.
\end{baitoan}

\begin{baitoan}[\cite{Tuyen_Toan_6}, 85., p. 21]
	Phải dùng ít nhất bao nhiêu quả cân để cân tất cả các vật có khối lượng là 1 số tự nhiên từ {\rm1 g} đến {\rm100 g} (các vật chỉ đặt trên 1 đĩa cân).
\end{baitoan}

\begin{baitoan}[\cite{Tuyen_Toan_6}, 86., p. 21]
	1 thùng nước lọc còn hơn {\rm7 L}. Làm thế nào để lấy ra được đúng {\rm4 L} chỉ bằng 1 bình loại {\rm5 L} \& 1 bình loại {\rm3 L}?
\end{baitoan}

\begin{baitoan}[\cite{Tuyen_Toan_6}, 87., p. 21]
	1 thùng dầu còn hơn {\rm10 L}. Dùng 1 can {\rm7 L} \& 1 can {\rm5 L} để lấy ra đúng {\rm8 L} dầu.
\end{baitoan}

\begin{baitoan}[\cite{Tuyen_Toan_6}, 88., p. 21]
	1 thùng có {\rm16 L} nước. Dùng 1 can {\rm7 L} \& 1 can {\rm3 L} để chia {\rm16 L} nước làm 2 phần bằng nhau.
\end{baitoan}

\begin{baitoan}[\cite{Tuyen_Toan_6}, 89., p. 21]
	1 can A đựng {\rm12 L} nước. Dùng 1 can B loại {\rm7 L} \& 1 can C loại {\rm5 L} để chia {\rm12 L} trong can A thành 3 phần: {\rm3 L, 4 L, 5 L}.
\end{baitoan}

%------------------------------------------------------------------------------%

\section{Bài Toán Thực Tế}

\begin{baitoan}[\cite{TLCT_THCS_Toan_6_so_hoc}, VD2.1, p. 14]
	Ông Toàn đi công tác trở về nhà thì chiếc đồng hồ lên dây cót của ông đã đứng. Ông lên dây cót, vặn kim đồng hồ chỉ {\rm8:00} rồi sang ngay nhà bạn gần đó để chơi \& hỏi giờ. Trên đường đi, ông phát hiện mình không mang theo đồng hồ. Do đó ông đã ghi lại lúc vừa đến nhà bạn là {\rm8:20} \& lúc bắt đầu rời nhà bạn để về nhà mình là {\rm8:50}. Khi về đến nhà, ông thấy đồng hồ của mình chỉ {\rm8:50}. Hỏi ông phải chỉnh đồng hồ của mình để kim đồng hồ chỉ mấy giờ?
\end{baitoan}

\begin{baitoan}[\cite{TLCT_THCS_Toan_6_so_hoc}, VD2.2, p. 14]
	An về nghỉ hè ở quê trong 1 số ngày, trong đó có $10$ ngày mưa. Biết có $11$ buổi sáng không mưa, có $9$ buổi chiều không mưa \& không bao giờ trời mưa cả sáng lẫn chiều. Hỏi An về nghỉ ở quê trong bao nhiêu ngày?
\end{baitoan}

\begin{baitoan}[\cite{TLCT_THCS_Toan_6_so_hoc}, VD2.3, p. 15]
	1 số học sinh dự thi học sinh giỏi toán. Nếu xếp $25$ học sinh 1 phòng thì thừa $5$ học sinh chưa có chỗ. Nếu xếp $28$ học sinh 1 phòng thì thừa 1 phòng. Tính số học sinh dự thi.
\end{baitoan}

\begin{baitoan}[\cite{TLCT_THCS_Toan_6_so_hoc}, VD2.4, p. 15]
	Trong 1 bảng đấu loại bóng đá, có 4 đội thi đấu vòng tròn 1 lượt: đội thắng được $3$ điểm, đội hòa được $1$ điểm, đội thua được $0$ điểm. Tổng số điểm của 4 đội khi kết thúc vòng đấu bảng là $16$ điểm. Tính số trận hòa.
\end{baitoan}

\begin{baitoan}[\cite{TLCT_THCS_Toan_6_so_hoc}, VD2.5, p. 16]
	1 câu lạc bộ lúc đầu có 1 thành viên, sau 1 tháng thì thành viên đó phải tìm thêm 2 thành viên mới. Cứ như vậy, mỗi thành viên (cả cũ lẫn mới) sau 1 tháng phải tìm được thêm 2 thành viên mới. Nếu kế hoạch phát triển hội viên như trên được thực hiện thì số thành viên của câu lạc bộ đó là bao nhiêu: (a) Sau 6 tháng? (b) Sau 12 tháng?
\end{baitoan}

\begin{baitoan}[\cite{TLCT_THCS_Toan_6_so_hoc}, VD2.6, p. 16]
	Tính số sách Toán bán được trong mỗi ngày của 1 cửa hàng biết số sách đã bán ra như sau: Thứ 2, thứ 3, thứ 4: $115$ quyển. Thứ 4, thứ 5: $85$ quyển. Thứ 3, thứ 5: $90$ quyển. Thứ 2, thứ 6: $70$ quyển. Thứ 5, thứ 6: $80$ quyển.
\end{baitoan}

\begin{baitoan}[\cite{TLCT_THCS_Toan_6_so_hoc}, VD2.7, p. 17, Bài toán bò ăn cỏ của Newton]
	Trên 1 cánh đồng cỏ, cỏ mọc đều như nhau \& lớn đều như nhau. Biết $70$ con bò ăn hết số cỏ có sẵn \& số cỏ mọc thêm trên cánh đồng ấy trong $24$ ngày, nếu có $30$ con bò thì chúng ăn hết cỏ trong $60$ ngày. (a) Gọi số cỏ 1 con bò ăn trong 1 ngày là 1 bó. Hỏi số cỏ mọc thêm trên cánh đồng trong $36$ ngày là bao nhiêu bó? (b) Bao nhiêu con bò sẽ ăn hết cỏ của cánh đồng trong $96$ ngày?
\end{baitoan}

\begin{baitoan}[\cite{TLCT_THCS_Toan_6_so_hoc}, 2.1., p. 18]
	1 cửa hàng mua 1 xe ôtô giá $1500$ triệu đồng, đem cho thuê $20$ tuần với giá cho thuê $30$ triệu đồng 1 tuần. Phí bảo hiểm cửa hàng phải nộp là $80$ triệu đồng, chi phí sửa chữa hết $120$ triệu đồng. Sau đó cửa hàng bán chiếc xe với giá $1300$ triệu đồng. Tính lợi nhận của thương vụ này.
\end{baitoan}

\begin{baitoan}[\cite{TLCT_THCS_Toan_6_so_hoc}, 2.2., p. 18]
	1 vòng xích có đường kính ngoài là {\rm40 mm}, độ dày của kim loại là {\rm3 mm}. Có $10$ vòng xích được nối với nhau. Tính chiều dài lớn nhất của dây xích.
\end{baitoan}

\begin{baitoan}[\cite{TLCT_THCS_Toan_6_so_hoc}, 2.3., p. 18]
	An \& Bích làm việc tại cùng 1 nhà máy. Cứ sau $9$ ngày làm việc thì An nghỉ $1$ ngày. Cứ sau $6$ ngày làm việc thì Bích nghỉ $1$ ngày. Hôm nay là ngày nghỉ của An \& ngày mai là ngày nghỉ của Bích. Hỏi sau ít nhất bao lâu (kể từ hôm nay) thì cả 2 người sẽ có cùng ngày nghỉ?
\end{baitoan}

\begin{baitoan}[\cite{TLCT_THCS_Toan_6_so_hoc}, 2.4., p. 18]
	Có $10$ người, tuổi của mỗi người là 1 số tự nhiên. Tổng số tuổi của $9$ người trong $10$ người đó là $82,83,84,85,87,89,90,91,92$. Tìm tuổi của người trẻ nhất, tuổi của người già nhất.	
\end{baitoan}

\begin{baitoan}[\cite{TLCT_THCS_Toan_6_so_hoc}, 2.5., p. 18]
	Ta gọi {\rm số đối xứng} là số mà viết các chữ số của nó theo thứ tự ngược lại lẫn được chính số đó, e.g., $353,1221,\ldots$ Đồng hồ đo quãng đường của 1 xe máy chỉ số $15951$. Tìm số đối xứng nhỏ nhất tiếp theo xuất hiện trên mặt đồng hồ.	
\end{baitoan}

\begin{baitoan}[\cite{TLCT_THCS_Toan_6_so_hoc}, 2.6., p. 18]
	Có 1 số con mèo chui vào chuồng bồ câu. Đếm trong chuồng thấy tổng cộng có $34$ cái đầu \& $80$ cái chân. Tính số mèo.
\end{baitoan}

\begin{baitoan}[\cite{TLCT_THCS_Toan_6_so_hoc}, 2.7., p. 18]
	Ở 1 bến cảng có $15$ con tàu, mỗi con tàu có $3$ cột buồm hoặc $5$ cột buồm, tổng cộng có $61$ cột buồm. Hỏi có bao nhiêu con tàu có $5$ cột buồm?
\end{baitoan}

\begin{baitoan}[\cite{TLCT_THCS_Toan_6_so_hoc}, 2.8., p. 18]
	Đội tuyển của 1 trường dự 1 cuộc thi đấu được chia đều thành 6 nhóm, mỗi học sinh dự thi đạt $8$ điểm hoặc $10$ điểm. Tổng số điểm của cả đội là $160$ điểm. Tính số học sinh đạt điểm $10$.
\end{baitoan}

\begin{baitoan}[\cite{TLCT_THCS_Toan_6_so_hoc}, 2.9., p. 18]
	Có $64$ bạn tham gia giải bóng bàn theo thể thức đấu loại trực tiếp. Những người được chọn ở mỗi vòng chia thành từng nhóm 2 người, 2 người trong nhóm đấu với nhau 1 trận để chọn lấy 1 người. Tìm số trận đấu ở: (a) Vòng 1. (b) Vòng 5.
\end{baitoan}

\begin{baitoan}[\cite{TLCT_THCS_Toan_6_so_hoc}, 2.10., p. 18]
	Tâm có $5$ tờ tiền mệnh giá $2000$ đồng \& $4$ tờ tiền mệnh giá $5000$ đồng. Tâm có bao nhiêu cách khác nhau để trả tiền bằng cách dùng 1 hoặc cả 2 loại tiền trên?
\end{baitoan}

\begin{baitoan}[\cite{TLCT_THCS_Toan_6_so_hoc}, 2.11., p. 19]
	Có $40$ bạn lớp 6A \& $30$ bạn lớp 6B xếp hàng đôi để vào tham quan Viện bảo tàng. Gọi $a$ là số trường hợp 2 bạn lớp 6A xếp cùng hàng đôi, gọi $b$ là số trường hợp 2 bạn lớp 6B xếp cùng hàng đôi. So sánh $a,b$, số nào lớn hơn, \& lớn hơn bao nhiêu?
\end{baitoan}

\begin{baitoan}[\cite{TLCT_THCS_Toan_6_so_hoc}, 2.12., p. 19]
	Tờ lịch của 1 tháng:
	\begin{table}[H]
		\centering
		\begin{tabular}{|c|c|c|c|c|c|c|}
			\hline
			S & M & T & W & Th & F & Sa \\
			\hline
			&  &  &  &  & 1 & 2 \\
			3 & 4 & 5 & 6 & 7 & 8 & 9 \\
			10 & 11 & 12 & 13 & 14 & 15 & 16 \\
			17 & 18 & 19 & 20 & 21 & 22 & 23 \\
			24 & 25 & 26 & 27 & 28 & 29 & 30 \\
			\hline
		\end{tabular}
	\end{table}
	\noindent Biết 1 bảng $3\times3$ của 1 tờ lịch khác có tổng 9 số trong bảng là $162$. (a) Tính số ở chính giữa của bảng đó. (b) Lập bảng $3\times3$ đó.
\end{baitoan}

%------------------------------------------------------------------------------%

\section{Miscellaneous}

\begin{baitoan}[\cite{TLCT_THCS_Toan_6_so_hoc}, VD1.1, p. 6]
	Xét $100$ số tự nhiên đầu tiên $0,1,2,\ldots,99$. Tìm $k\in\mathbb{N}$ sao cho trong $100$ số này, có nhiều nhất các số có tổng các chữ số bằng $k$.
\end{baitoan}

\begin{baitoan}[Mở rộng \cite{TLCT_THCS_Toan_6_so_hoc}, VD1.1, p. 6]
	Xét $n$ số tự nhiên đầu tiên $0,1,2,\ldots,n - 1$. Tìm $k\in\mathbb{N}$ sao cho trong $n$ số này, có nhiều nhất các số có tổng các chữ số bằng $k$.
\end{baitoan}

\begin{baitoan}[\cite{TLCT_THCS_Toan_6_so_hoc}, VD1.2, p. 6]
	Tính: (a) $a = 234\cdot\underbrace{9\ldots9}_{50}$. (b) $b = \underbrace{1\ldots1}_{100}\cdot3456$.
\end{baitoan}

\begin{baitoan}[\cite{TLCT_THCS_Toan_6_so_hoc}, VD1.3, p. 7]
	Tính hiệu $a - b$ biết $a = 1\cdot2 + 2\cdot3 + 3\cdot4 + \cdots + 98\cdot99$, $b = 1^2 + 2^2 + 3^2 + \cdots + 98^2$.
\end{baitoan}

\begin{baitoan}[\cite{TLCT_THCS_Toan_6_so_hoc}, VD1.4, p. 7]
	Tìm $x,y\in\mathbb{R}$ thỏa $2^x + 2^y = 20$.
\end{baitoan}

\begin{baitoan}[\cite{TLCT_THCS_Toan_6_so_hoc}, VD1.5, p. 8]
	Trong 1 phép chia có dư, số bị chia bằng $24$, thương bằng $3$. Tìm số chia \& số dư.
\end{baitoan}

\begin{baitoan}[\cite{TLCT_THCS_Toan_6_so_hoc}, VD1.6, p. 8]
	Tìm số tự nhiên có 2 chữ số biết nếu viết thêm chữ số $4$ vào trước chữ số hàng chục thì được số $a$, nếu viết thêm chữ số $8$ vào sau chữ số hàng đơn vị thì được số $b$, trong đó $b$ gấp đôi $a$.
\end{baitoan}

\begin{baitoan}[\cite{TLCT_THCS_Toan_6_so_hoc}, VD1.7, p. 9]
	Điền chữu số thỏa mãn cả 2 phép cộng: $\rm one + one + one + one = four$, $\rm four + one = five$.
\end{baitoan}

\begin{baitoan}[\cite{TLCT_THCS_Toan_6_so_hoc}, VD1.8, p. 10]
	Tìm chữ số tận cùng của $3^{2015}$.
\end{baitoan}

\begin{baitoan}[\cite{TLCT_THCS_Toan_6_so_hoc}, VD1.9, p. 11]
	Tìm 2 chữ số tận cùng của $6^{2011}$.
\end{baitoan}

\begin{baitoan}[\cite{TLCT_THCS_Toan_6_so_hoc}, 1.1., p. 11]
	Tính giá trị của biểu thức (tính nhanh nếu có thể): (a) $215\cdot62 + 42 - 52\cdot215$. (b) $14\cdot29 + 14\cdot71 + (1 + 2 + 3 + \cdots + 99)(199199\cdot198 - 198198\cdot199)$.
\end{baitoan}

\begin{baitoan}[\cite{TLCT_THCS_Toan_6_so_hoc}, 1.2., p. 11]
	Đánh số trang của 1 cuốn sách bằng dãy số tự nhiên $1,2,3,\ldots$ (a) Nếu quyển sách có $180$ trang thì phải viết tất cả bao nhiêu chữ số? (b) Nếu phải viết tất cả $327$ chữ số thì quyển sách có bao nhiêu trang?
\end{baitoan}

\begin{baitoan}[\cite{TLCT_THCS_Toan_6_so_hoc}, 1.3., p. 11]
	Tìm 2 số tự nhiên biết tổng của chúng gấp 3 hiệu của chúng \& bằng tích của chúng.
\end{baitoan}

\begin{baitoan}[\cite{TLCT_THCS_Toan_6_so_hoc}, 1.4., p. 11]
	Tìm số tự nhiên lớn nhất có $3$ chữ biết khi chia nó cho $69$ thì thương \& số dư bằng nhau.
\end{baitoan}

\begin{baitoan}[\cite{TLCT_THCS_Toan_6_so_hoc}, 1.5., p. 11]
	Tìm số dư của phép chia số $\underbrace{1\ldots1}_{100}$ cho $1001$.
\end{baitoan}

\begin{baitoan}[\cite{TLCT_THCS_Toan_6_so_hoc}, 1.6., p. 11]
	Điền chữ số thích hợp để $\overline{8aba} + \overline{c25d} = \overline{d52c}$.
\end{baitoan}

\begin{baitoan}[\cite{TLCT_THCS_Toan_6_so_hoc}, 1.7., p. 11]
	Điền chữ số $\overline{abc} + \overline{bca}$ sao cho tổng trên là lớn nhất \& $a,b,c$ nhận các giá trị $1,2,3$ (không nhất thiết tương ứng).
\end{baitoan}

\begin{baitoan}[\cite{TLCT_THCS_Toan_6_so_hoc}, 1.8., p. 12]
	Điền chữ số thích hợp để $\overline{aa} + \overline{bb} + \overline{cc} = \overline{bac}$.
\end{baitoan}

\begin{baitoan}[\cite{TLCT_THCS_Toan_6_so_hoc}, 1.9., p. 12]
	Tìm chữ số thích hợp để $\overline{xyz1} = \overline{ab}\cdot\overline{ba}$.
\end{baitoan}

\begin{baitoan}[\cite{TLCT_THCS_Toan_6_so_hoc}, 1.10., p. 12]
	Có bao nhiêu số tự nhiên có 2 chữ số mà chữ số hàng chục nhỏ hơn chữ số hàng đơn vị?
\end{baitoan}

\begin{baitoan}[\cite{TLCT_THCS_Toan_6_so_hoc}, 1.11., p. 12]
	Có bao nhiêu số tự nhiên có 3 chữ số, trong đó có ít nhất 2 chữ số giống nhau?
\end{baitoan}

\begin{baitoan}[\cite{TLCT_THCS_Toan_6_so_hoc}, 1.12., p. 12]
	Trong các số tự nhiên từ $1$ đến $500$, có bao nhiêu số có ít nhất 1 chữ số $5$?
\end{baitoan}

\begin{baitoan}[\cite{TLCT_THCS_Toan_6_so_hoc}, 1.13., p. 12]
	Trên $n\in\mathbb{N}^\star$ ô vuông cách đều nhau của 1 đường tròn, ghi $n$ số tự nhiên liên tiếp theo chiều kim đồng hồ. Biết ô ghi số $12$ đối diện với ô ghi số $60$. Tính $n$.
\end{baitoan}

\begin{baitoan}[\cite{TLCT_THCS_Toan_6_so_hoc}, 1.14., p. 12]
	Tìm $n\in\mathbb{N}^\star$ biết có đúng $100$ số lẻ nằm giữa $n$ \& $2n$.
\end{baitoan}

\begin{baitoan}[\cite{TLCT_THCS_Toan_6_so_hoc}, 1.15., p. 12]
	Ghép 2 chữ số $1$, 2 chữ số $2$, 2 chữ số $3$ làm thành 1 số có $6$ chữ số. Tìm số có $6$ chữ số ấy biết 2 chữ số $1$ cách nhau 1 chữ số, 2 chữ số $2$ cách nhau 2 chữ số, 2 chữ số $3$ cách nhau 3 chữ số.
\end{baitoan}

\begin{baitoan}[\cite{TLCT_THCS_Toan_6_so_hoc}, 1.18., p. 12]
	Chia $a\in\mathbb{N}$ cho $72$ thì dư $69$. Chia số $a$ cho $18$ thì thương bằng số dư. Tìm $a$.
\end{baitoan}

\begin{baitoan}[\cite{TLCT_THCS_Toan_6_so_hoc}, 1.19., p. 13]
	Xét phép chia $a\in\mathbb{N}$ cho $b\in\mathbb{N}^\star$, có $a = bq + r$, $0\le r < b$. Nếu $r = 0$ thì $q$ gọi là {\rm thương đúng} của phép chia. Nếu $r\ne0$ thì $q$ gọi là {\rm thương hụt} của phép chia. Ký hiệu $[a:b]$ là thương đúng hoặc thương hụt của phép chia $a$ cho $b$. Tính: (a) $[32:4],[61:4]$. (b) $[800:5] + [800:5^2] + [800:5^3] + [800:5^4]$.
\end{baitoan}

\begin{baitoan}[\cite{TLCT_THCS_Toan_6_so_hoc}, 1.20., p. 13]
	Điền chữ số thích hợp để $\overline{84**}:47 = \overline{*8*}$.
\end{baitoan}

\begin{baitoan}[\cite{TLCT_THCS_Toan_6_so_hoc}, 1.21., p. 13]
	Tính: (a) $(2^9\cdot16 + 2^9\cdot34):2^{10}$. (b) $(3^4\cdot57 - 9^2\cdot21):3^5$.
\end{baitoan}

\begin{baitoan}[\cite{TLCT_THCS_Toan_6_so_hoc}, 1.22., p. 13]
	Cho biết $\sum_{i=1}^9 i^3 = 1^3 + 2^3 + \cdots + 9^3 = 2025$. Tính $2^3 + 4^3 + 6^3 + \cdots + 18^3$.
\end{baitoan}

\begin{baitoan}[\cite{TLCT_THCS_Toan_6_so_hoc}, 1.23., p. 13]
	Cho $a = \sum_{i=1}^{10} 2^i = 2 + 2^2 + 2^3 + \cdots + 2^{10}$. Không tính giá trị của biểu thức $a$, chứng minh $a + 2 = 2^11$.
\end{baitoan}

\begin{baitoan}[\cite{TLCT_THCS_Toan_6_so_hoc}, 1.24., p. 13]
	Tìm $x\in\mathbb{N}$ thỏa $(2x + 1)^2 = 625$.
\end{baitoan}

\begin{baitoan}[\cite{TLCT_THCS_Toan_6_so_hoc}, 1.25., p. 13]
	Quan sát $11 - 2 = 9 = 3^2$, $1111 - 22 = 1089 = 33^2$. Chứng minh $\underbrace{1\ldots1}_{2n} - \underbrace{2\ldots2}_n$ là số chính phương.
\end{baitoan}

\begin{baitoan}[\cite{TLCT_THCS_Toan_6_so_hoc}, 1.26., p. 13]
	Tìm chữ số tận cùng: (a) $7^{35} - 4^{31}$. (b) $2^{1930}\cdot9^{1945}$.
\end{baitoan}

\begin{baitoan}[\cite{TLCT_THCS_Toan_6_so_hoc}, 1.27., p. 13]
	Tìm 2 chữ số tận cùng: (a) $351^{2011}$. (b) $218^{218}$.
\end{baitoan}

%------------------------------------------------------------------------------%

\printbibliography[heading=bibintoc]

\end{document}