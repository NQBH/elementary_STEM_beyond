\documentclass{article}
\usepackage[backend=biber,natbib=true,style=alphabetic,maxbibnames=50]{biblatex}
\addbibresource{/home/nqbh/reference/bib.bib}
\usepackage[utf8]{vietnam}
\usepackage{tocloft}
\renewcommand{\cftsecleader}{\cftdotfill{\cftdotsep}}
\usepackage[colorlinks=true,linkcolor=blue,urlcolor=red,citecolor=magenta]{hyperref}
\usepackage{amsmath,amssymb,amsthm,float,graphicx,mathtools,tipa}
\usepackage{enumitem}
\setlist{leftmargin=4mm}
\allowdisplaybreaks
\newtheorem{assumption}{Assumption}
\newtheorem{baitoan}{}
\newtheorem{cauhoi}{Câu hỏi}
\newtheorem{conjecture}{Conjecture}
\newtheorem{corollary}{Corollary}
\newtheorem{dangtoan}{Dạng toán}
\newtheorem{definition}{Definition}
\newtheorem{dinhly}{Định lý}
\newtheorem{dinhnghia}{Định nghĩa}
\newtheorem{example}{Example}
\newtheorem{ghichu}{Ghi chú}
\newtheorem{hequa}{Hệ quả}
\newtheorem{hypothesis}{Hypothesis}
\newtheorem{lemma}{Lemma}
\newtheorem{luuy}{Lưu ý}
\newtheorem{nhanxet}{Nhận xét}
\newtheorem{notation}{Notation}
\newtheorem{note}{Note}
\newtheorem{principle}{Principle}
\newtheorem{problem}{Problem}
\newtheorem{proposition}{Proposition}
\newtheorem{question}{Question}
\newtheorem{remark}{Remark}
\newtheorem{theorem}{Theorem}
\newtheorem{vidu}{Ví dụ}
\usepackage[left=1cm,right=1cm,top=5mm,bottom=5mm,footskip=4mm]{geometry}
\def\labelitemii{$\circ$}
\DeclareRobustCommand{\divby}{%
	\mathrel{\vbox{\baselineskip.65ex\lineskiplimit0pt\hbox{.}\hbox{.}\hbox{.}}}%
}

\title{Problem: Prime, Composite, GCD, {\it\&} LCM\\Bài Tập: Số Nguyên Tố, Hợp Số, ƯCLN, {\it\&} BCNN}
\author{Nguyễn Quản Bá Hồng\footnote{Independent Researcher, Ben Tre City, Vietnam\\e-mail: \texttt{nguyenquanbahong@gmail.com}; website: \url{https://nqbh.github.io}.}}
\date{\today}

\begin{document}
\maketitle
\begin{abstract}
	Last updated version: \href{https://github.com/NQBH/elementary_STEM_beyond/blob/main/elementary_mathematics/grade_6/natural/divisibility/problem/NQBH_divisibility_problem.pdf}{GitHub{\tt/}NQBH{\tt/}hobby{\tt/}elementary mathematics{\tt/}grade 6{\tt/}natural{\tt/}divisibility{\tt/}problem[pdf]}.\footnote{\textsc{url}: \url{https://github.com/NQBH/elementary_STEM_beyond/blob/main/elementary_mathematics/grade_6/natural/divisibility/problem/NQBH_divisibility_problem.pdf}.} [\href{https://github.com/NQBH/elementary_STEM_beyond/blob/main/elementary_mathematics/grade_6/natural/divisibility/problem/NQBH_divisibility_problem.tex}{\TeX}]\footnote{\textsc{url}: \url{https://github.com/NQBH/elementary_STEM_beyond/blob/main/elementary_mathematics/grade_6/natural/divisibility/problem/NQBH_divisibility_problem.tex}.}. 
\end{abstract}
\tableofcontents

%------------------------------------------------------------------------------%M

\section{Prime. Composite -- Số Nguyên Tố. Hợp Số}

\begin{baitoan}[\cite{Binh_boi_duong_Toan_6_tap_1}, H1, p. 36]
	Egg có $54$ viên bi \& muốn chia đều số bi đó vào các hộp. Tìm tất cả các cách chia thỏa mãn.
\end{baitoan}

\begin{baitoan}[\cite{Binh_boi_duong_Toan_6_tap_1}, H2, p. 36]
	(a) Số nào có phân tích ra thừa số nguyên tố là $2^3\cdot3^2\cdot7$. (b) Phân tích $2160$ ra thừa số nguyên tố.
\end{baitoan}

\begin{baitoan}[\cite{Binh_boi_duong_Toan_6_tap_1}, H3, p. 36]
	Tìm chữ số $a$ để $\overline{17a}$ là số nguyên tố.
\end{baitoan}

\begin{baitoan}[\cite{Binh_boi_duong_Toan_6_tap_1}, H4, p. 36]
	{\rm Đ{\tt/}S?} Ký hiệu $P$ là tập hợp các số nguyên tố. (a) $19\in P$. (b) $\{3,5,7\}\in P$. (c) $\{71,73\}\in P$. (d) $6\cdot7\cdot8\cdot9 - 5\cdot7\cdot11\in P$. (e) Mọi số nguyên tố đều có tận cùng là số lẻ.
\end{baitoan}

\begin{baitoan}[\cite{Binh_boi_duong_Toan_6_tap_1}, Ví dụ 1, p. 37]
	Cho 1 phép chia có số bị chia bằng $236$ \& số dư bằng $15$. Tìm số chia \& thương.
\end{baitoan}

\begin{baitoan}[\cite{Binh_boi_duong_Toan_6_tap_1}, Ví dụ 2, p. 37]
	Có bao nhiêu số là bội của $6$ trong khoảng từ $72$ đến $2016$?
\end{baitoan}

\begin{baitoan}[\cite{Binh_boi_duong_Toan_6_tap_1}, Ví dụ 3, p. 37]
	Tìm $x\in\mathbb{N}$ sao cho $42\divby(2x + 5)$.
\end{baitoan}

\begin{baitoan}[\cite{Binh_boi_duong_Toan_6_tap_1}, Ví dụ 4, p. 38]
	Tìm số nguyên tố $p$ sao cho $p + 2$ \& $p + 4$ cũng là 2 số nguyên tố.
\end{baitoan}

\begin{baitoan}[\cite{Binh_boi_duong_Toan_6_tap_1}, Ví dụ 5, p. 38]
	Cho $p > 3$ \& $2p + 1$ là 2 số nguyên tố. Hỏi $4p + 1$ là số nguyên tố hay hợp số.
\end{baitoan}

\begin{baitoan}[\cite{Binh_boi_duong_Toan_6_tap_1}, Ví dụ 6, p. 39]
	Tìm số nguyên tố, biết số đó bằng tổng của 2 số nguyên tố \& cũng bằng hiệu của 2 số nguyên tố khác.
\end{baitoan}

\begin{luuy}
	$2$ là số nguyên tố chẵn duy nhất.
\end{luuy}

\begin{baitoan}[\cite{Binh_boi_duong_Toan_6_tap_1}, Ví dụ 7, p. 39]
	Phân t ích ra thừa số nguyên tố: (a) $2016^7$. (b) $30\cdot4\cdot1975$.
\end{baitoan}

\begin{baitoan}[\cite{Binh_boi_duong_Toan_6_tap_1}, Ví dụ 8, p. 39]
	Tìm $n\in\mathbb{N}^\star$ thỏa $2 + 4 + 6 + \cdots + 2n = 870$.
\end{baitoan}

\begin{baitoan}[\cite{Binh_boi_duong_Toan_6_tap_1}, Ví dụ 9, p. 40]
	Tìm $n\in\mathbb{N}^\star$ sao cho $p = (n - 2)(n^2 + n - 5)$ là số nguyên tố.
\end{baitoan}

\begin{baitoan}[\cite{Binh_boi_duong_Toan_6_tap_1}, 5.1., p. 40]
	Tìm tập hợp các số tự nhiên vừa là bội của $9$, vừa là ước của $72$.
\end{baitoan}

\begin{baitoan}[\cite{Binh_boi_duong_Toan_6_tap_1}, 5.2., p. 40]
	Tìm $x\in\mathbb{N}^\star$ thỏa: (a) $x - 1$ là ước của $24$. (b) $36$ là bội của $2x + 1$.
\end{baitoan}

\begin{baitoan}[\cite{Binh_boi_duong_Toan_6_tap_1}, 5.3., p. 40]
	Tìm $x,y\in\mathbb{N}^\star$ thỏa $(2x + 1)(y - 3) = 15$.
\end{baitoan}

\begin{baitoan}[\cite{Binh_boi_duong_Toan_6_tap_1}, 5.4., p. 40]
	Phân tích ra thừa số nguyên tố: (a) $1\cdot12\cdot78$. (b) $1930^8$.
\end{baitoan}

\begin{baitoan}[\cite{Binh_boi_duong_Toan_6_tap_1}, 5.5., p. 40]
	Chứng minh nếu $p$ là 1 số nguyên tố lớn hơn $3$ thì $(p - 1)(p + 1)$ chia hết cho $3$ \& cho $8$.
\end{baitoan}

\begin{baitoan}[\cite{Binh_boi_duong_Toan_6_tap_1}, 5.6., p. 40]
	Tìm chữ số $a$ để $\overline{23a}$ là số nguyên tố.
\end{baitoan}

\begin{baitoan}[\cite{Binh_boi_duong_Toan_6_tap_1}, 5.7., p. 40]
	Tìm số tự nhiên nhỏ nhất có đúng $18$ ước số.
\end{baitoan}

\begin{baitoan}[\cite{Binh_boi_duong_Toan_6_tap_1}, 5.8., p. 40]
	Chứng minh: Nếu 1 số tự nhiên có 3 chữ số tận cùng là $104$ thì số đó có ít nhất $4$ ước số.
\end{baitoan}

\begin{baitoan}[\cite{Binh_boi_duong_Toan_6_tap_1}, 5.9., p. 40]
	Tìm 2 số nguyên tố có tổng bằng $309$.
\end{baitoan}

\begin{baitoan}[\cite{Binh_boi_duong_Toan_6_tap_1}, 5.10., p. 40]
	Tìm số nguyên tố $p$ sao cho $p + 4,p + 8$ cũng là 2 số nguyên tố.
\end{baitoan}

\begin{baitoan}[\cite{Binh_boi_duong_Toan_6_tap_1}, 5.11., p. 40]
	Tìm số nguyên tố $p$ sao cho $p + 6,p + 8,p + 12,p + 14$ cũng là 4 số nguyên tố.
\end{baitoan}

\begin{baitoan}[\cite{Binh_boi_duong_Toan_6_tap_1}, 5.12., p. 40]
	Cho $ptố > 3$ \& $p + 4$ là 2 số nguyên tố. Chứng minh $p + 8$ là hợp số.
\end{baitoan}

\begin{baitoan}[\cite{Binh_boi_duong_Toan_6_tap_1}, 5.13., p. 40]
	Số $3^2 + 3^4 + 3^6 + \cdots + 3^{2012}$ là số nguyên tố hay hợp số?
\end{baitoan}

\begin{baitoan}[\cite{Binh_boi_duong_Toan_6_tap_1}, 5.14., p. 40]
	2 số nguyên tố được gọi là {\rm sinh đôi} nếu chúng là 2 số nguyên tố \& là 2 số lẻ liên tiếp, e.g., $3$ \& $5$, $11$ \& $13,\ldots$. Chứng minh số tự nhiên lớn hơn $4$ \& nằm giữa 2 số nguyên tố sinh đôi thì chia hết cho $6$.
\end{baitoan}

\begin{baitoan}[\cite{Binh_boi_duong_Toan_6_tap_1}, 5.15., p. 41]
	Tìm 3 số tự nhiên lẻ liên tiếp đều là số nguyên tố.
\end{baitoan}

\begin{baitoan}[\cite{Binh_boi_duong_Toan_6_tap_1}, 5.16., p. 41]
	Tìm $n\in\mathbb{N}^\star$ thỏa $1 + 3 + 5 + \cdots + (2n + 1) = 169$.
\end{baitoan}

\begin{baitoan}[\cite{Binh_boi_duong_Toan_6_tap_1}, 5.17., p. 41]
	Biết số $\overline{abc}$ khi phân tích ra t hừa số nguyên tố có thừa số $3$ \& thừa số $7$. Chứng minh số $a + 19b + 4c$ cũng có tính chất đó.
\end{baitoan}

\begin{baitoan}[\cite{Binh_boi_duong_Toan_6_tap_1}, 5.18., p. 41]
	Tìm chữ số $a$ sao cho số $\overline{aaa}$ là tổng của các số tự nhiên liên tiếp từ $1$ đến số $n$ nào đó.
\end{baitoan}

\begin{baitoan}
	Chứng minh tập hợp các số nguyên tố có vô hạn phần tử \& không có số nguyên tố lớn nhất.
\end{baitoan}
\noindent\textit{Hint.} Giả sử phản chứng: chỉ có hữu hạn số nguyên tố $p_1 < p_2 < \cdots < p_n$. Chứng minh $p\coloneqq\prod_{i=1}^n p_i + 1 = p_1p_2\cdots p_n + 1$ là 1 số nguyên tố lớn hơn mỗi số nguyên tố $p_i$, $\forall i\in\mathbb{N}$.

%------------------------------------------------------------------------------%

\section{Greatest Common Divisor. Least Common Multiple -- Ước Chung Lớn Nhất. Bội Chung Nhỏ Nhất}

%------------------------------------------------------------------------------%

\section{Miscellaneous}

%------------------------------------------------------------------------------%

\printbibliography[heading=bibintoc]

\end{document}