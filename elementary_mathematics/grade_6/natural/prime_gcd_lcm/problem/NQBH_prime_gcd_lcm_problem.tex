\documentclass{article}
\usepackage[backend=biber,natbib=true,style=alphabetic,maxbibnames=50]{biblatex}
\addbibresource{/home/nqbh/reference/bib.bib}
\usepackage[utf8]{vietnam}
\usepackage{tocloft}
\renewcommand{\cftsecleader}{\cftdotfill{\cftdotsep}}
\usepackage[colorlinks=true,linkcolor=blue,urlcolor=red,citecolor=magenta]{hyperref}
\usepackage{amsmath,amssymb,amsthm,float,graphicx,mathtools,tipa}
\usepackage{enumitem}
\setlist{leftmargin=4mm}
\allowdisplaybreaks
\newtheorem{assumption}{Assumption}
\newtheorem{baitoan}{}
\newtheorem{cauhoi}{Câu hỏi}
\newtheorem{conjecture}{Conjecture}
\newtheorem{corollary}{Corollary}
\newtheorem{dangtoan}{Dạng toán}
\newtheorem{definition}{Definition}
\newtheorem{dinhly}{Định lý}
\newtheorem{dinhnghia}{Định nghĩa}
\newtheorem{example}{Example}
\newtheorem{ghichu}{Ghi chú}
\newtheorem{hequa}{Hệ quả}
\newtheorem{hypothesis}{Hypothesis}
\newtheorem{lemma}{Lemma}
\newtheorem{luuy}{Lưu ý}
\newtheorem{nhanxet}{Nhận xét}
\newtheorem{notation}{Notation}
\newtheorem{note}{Note}
\newtheorem{principle}{Principle}
\newtheorem{problem}{Problem}
\newtheorem{proposition}{Proposition}
\newtheorem{question}{Question}
\newtheorem{remark}{Remark}
\newtheorem{theorem}{Theorem}
\newtheorem{vidu}{Ví dụ}
\usepackage[left=1cm,right=1cm,top=5mm,bottom=5mm,footskip=4mm]{geometry}
\def\labelitemii{$\circ$}
\DeclareRobustCommand{\divby}{%
	\mathrel{\vbox{\baselineskip.65ex\lineskiplimit0pt\hbox{.}\hbox{.}\hbox{.}}}%
}

\title{Problem: Prime, Composite, GCD, {\it\&} LCM\\Bài Tập: Số Nguyên Tố, Hợp Số, ƯCLN, {\it\&} BCNN}
\author{Nguyễn Quản Bá Hồng\footnote{Independent Researcher, Ben Tre City, Vietnam\\e-mail: \texttt{nguyenquanbahong@gmail.com}; website: \url{https://nqbh.github.io}.}}
\date{\today}

\begin{document}
\maketitle
\begin{abstract}
	Last updated version: \href{https://github.com/NQBH/elementary_STEM_beyond/blob/main/elementary_mathematics/grade_6/natural/prime_gcd_lcm/problem/NQBH_prime_gcd_lcm_problem.pdf}{GitHub{\tt/}NQBH{\tt/}hobby{\tt/}elementary mathematics{\tt/}grade 6{\tt/}natural{\tt/}prime, gcd, lcm{\tt/}problem[pdf]}.\footnote{\textsc{url}: \url{https://github.com/NQBH/elementary_STEM_beyond/blob/main/elementary_mathematics/grade_6/natural/prime_gcd_lcm/problem/NQBH_prime_gcd_lcm_problem.pdf}.} [\href{https://github.com/NQBH/elementary_STEM_beyond/blob/main/elementary_mathematics/grade_6/natural/prime_gcd_lcm/problem/NQBH_prime_gcd_lcm_problem.tex}{\TeX}]\footnote{\textsc{url}: \url{https://github.com/NQBH/elementary_STEM_beyond/blob/main/elementary_mathematics/grade_6/natural/prime_gcd_lcm/problem/NQBH_prime_gcd_lcm_problem.tex}.}. 
\end{abstract}
\tableofcontents

%------------------------------------------------------------------------------%

\section{Prime. Composite -- Số Nguyên Tố. Hợp Số}

\begin{baitoan}[\cite{Binh_boi_duong_Toan_6_tap_1}, H1, p. 36]
	Egg có $54$ viên bi \& muốn chia đều số bi đó vào các hộp. Tìm tất cả các cách chia thỏa mãn.
\end{baitoan}

\begin{baitoan}[\cite{Binh_boi_duong_Toan_6_tap_1}, H2, p. 36]
	(a) Số nào có phân tích ra thừa số nguyên tố là $2^3\cdot3^2\cdot7$. (b) Phân tích $2160$ ra thừa số nguyên tố.
\end{baitoan}

\begin{baitoan}[\cite{Binh_boi_duong_Toan_6_tap_1}, H3, p. 36]
	Tìm chữ số $a$ để $\overline{17a}$ là số nguyên tố.
\end{baitoan}

\begin{baitoan}[\cite{Binh_boi_duong_Toan_6_tap_1}, H4, p. 36]
	{\rm Đ{\tt/}S?} Ký hiệu $P$ là tập hợp các số nguyên tố. (a) $19\in P$. (b) $\{3,5,7\}\in P$. (c) $\{71,73\}\in P$. (d) $6\cdot7\cdot8\cdot9 - 5\cdot7\cdot11\in P$. (e) Mọi số nguyên tố đều có tận cùng là số lẻ.
\end{baitoan}

\begin{baitoan}[\cite{Binh_boi_duong_Toan_6_tap_1}, VD1, p. 37]
	Cho 1 phép chia có số bị chia bằng $236$ \& số dư bằng $15$. Tìm số chia \& thương.
\end{baitoan}

\begin{baitoan}[\cite{Binh_boi_duong_Toan_6_tap_1}, VD2, p. 37]
	Có bao nhiêu số là bội của $6$ trong khoảng từ $72$ đến $2016$?
\end{baitoan}

\begin{baitoan}[\cite{Binh_boi_duong_Toan_6_tap_1}, VD3, p. 37]
	Tìm $x\in\mathbb{N}$ sao cho $42\divby(2x + 5)$.
\end{baitoan}

\begin{baitoan}[\cite{Binh_boi_duong_Toan_6_tap_1}, VD4, p. 38]
	Tìm số nguyên tố $p$ sao cho $p + 2$ \& $p + 4$ cũng là 2 số nguyên tố.
\end{baitoan}

\begin{baitoan}[\cite{Binh_boi_duong_Toan_6_tap_1}, VD5, p. 38]
	Cho $p > 3$ \& $2p + 1$ là 2 số nguyên tố. Hỏi $4p + 1$ là số nguyên tố hay hợp số.
\end{baitoan}

\begin{baitoan}[\cite{Binh_boi_duong_Toan_6_tap_1}, VD6, p. 39]
	Tìm số nguyên tố bằng tổng của 2 số nguyên tố \& cũng bằng hiệu của 2 số nguyên tố khác.
\end{baitoan}

\begin{baitoan}[\cite{Binh_boi_duong_Toan_6_tap_1}, VD7, p. 39]
	Phân tích ra thừa số nguyên tố: (a) $2016^7$. (b) $30\cdot4\cdot1975$.
\end{baitoan}

\begin{baitoan}[\cite{Binh_boi_duong_Toan_6_tap_1}, VD8, p. 39]
	Tìm $n\in\mathbb{N}^\star$ thỏa $2 + 4 + 6 + \cdots + 2n = 870$.
\end{baitoan}

\begin{baitoan}[\cite{Binh_boi_duong_Toan_6_tap_1}, VD9, p. 40]
	Tìm $n\in\mathbb{N}^\star$ sao cho $p = (n - 2)(n^2 + n - 5)$ là số nguyên tố.
\end{baitoan}

\begin{baitoan}[\cite{Binh_boi_duong_Toan_6_tap_1}, 5.1., p. 40]
	Tìm tập hợp các số tự nhiên vừa là bội của $9$, vừa là ước của $72$.
\end{baitoan}

\begin{baitoan}[\cite{Binh_boi_duong_Toan_6_tap_1}, 5.2., p. 40]
	Tìm $x\in\mathbb{N}^\star$ thỏa: (a) $x - 1$ là ước của $24$. (b) $36$ là bội của $2x + 1$.
\end{baitoan}

\begin{baitoan}[\cite{Binh_boi_duong_Toan_6_tap_1}, 5.3., p. 40]
	Tìm $x,y\in\mathbb{N}^\star$ thỏa $(2x + 1)(y - 3) = 15$.
\end{baitoan}

\begin{baitoan}[\cite{Binh_boi_duong_Toan_6_tap_1}, 5.4., p. 40]
	Phân tích ra thừa số nguyên tố: (a) $1\cdot12\cdot78$. (b) $1930^8$.
\end{baitoan}

\begin{baitoan}[\cite{Binh_boi_duong_Toan_6_tap_1}, 5.5., p. 40]
	Chứng minh nếu $p$ là 1 số nguyên tố lớn hơn $3$ thì $(p - 1)(p + 1)$ chia hết cho $3$ \& cho $8$.
\end{baitoan}

\begin{baitoan}[\cite{Binh_boi_duong_Toan_6_tap_1}, 5.6., p. 40]
	Tìm chữ số $a$ để $\overline{23a}$ là số nguyên tố.
\end{baitoan}

\begin{baitoan}[\cite{Binh_boi_duong_Toan_6_tap_1}, 5.7., p. 40]
	Tìm số tự nhiên nhỏ nhất có đúng $18$ ước số.
\end{baitoan}

\begin{baitoan}[\cite{Binh_boi_duong_Toan_6_tap_1}, 5.8., p. 40]
	Chứng minh: Nếu 1 số tự nhiên có 3 chữ số tận cùng là $104$ thì số đó có ít nhất $4$ ước số.
\end{baitoan}

\begin{baitoan}[\cite{Binh_boi_duong_Toan_6_tap_1}, 5.9., p. 40]
	Tìm 2 số nguyên tố có tổng bằng $309$.
\end{baitoan}

\begin{baitoan}[\cite{Binh_boi_duong_Toan_6_tap_1}, 5.10., p. 40]
	Tìm số nguyên tố $p$ sao cho $p + 4,p + 8$ cũng là 2 số nguyên tố.
\end{baitoan}

\begin{baitoan}[\cite{Binh_boi_duong_Toan_6_tap_1}, 5.11., p. 40]
	Tìm số nguyên tố $p$ sao cho $p + 6,p + 8,p + 12,p + 14$ cũng là 4 số nguyên tố.
\end{baitoan}

\begin{baitoan}[\cite{Binh_boi_duong_Toan_6_tap_1}, 5.12., p. 40]
	Cho $ptố > 3$ \& $p + 4$ là 2 số nguyên tố. Chứng minh $p + 8$ là hợp số.
\end{baitoan}

\begin{baitoan}[\cite{Binh_boi_duong_Toan_6_tap_1}, 5.13., p. 40]
	Số $3^2 + 3^4 + 3^6 + \cdots + 3^{2012}$ là số nguyên tố hay hợp số?
\end{baitoan}

\begin{baitoan}[\cite{Binh_boi_duong_Toan_6_tap_1}, 5.14., p. 40]
	2 số nguyên tố được gọi là {\rm sinh đôi} nếu chúng là 2 số nguyên tố \& là 2 số lẻ liên tiếp, e.g., $3$ \& $5$, $11$ \& $13,\ldots$. Chứng minh số tự nhiên lớn hơn $4$ \& nằm giữa 2 số nguyên tố sinh đôi thì chia hết cho $6$.
\end{baitoan}

\begin{baitoan}[\cite{Binh_boi_duong_Toan_6_tap_1}, 5.15., p. 41]
	Tìm 3 số tự nhiên lẻ liên tiếp đều là số nguyên tố.
\end{baitoan}

\begin{baitoan}[\cite{Binh_boi_duong_Toan_6_tap_1}, 5.16., p. 41]
	Tìm $n\in\mathbb{N}^\star$ thỏa $1 + 3 + 5 + \cdots + (2n + 1) = 169$.
\end{baitoan}

\begin{baitoan}[\cite{Binh_boi_duong_Toan_6_tap_1}, 5.17., p. 41]
	Biết số $\overline{abc}$ khi phân tích ra t hừa số nguyên tố có thừa số $3$ \& thừa số $7$. Chứng minh số $a + 19b + 4c$ cũng có tính chất đó.
\end{baitoan}

\begin{baitoan}[\cite{Binh_boi_duong_Toan_6_tap_1}, 5.18., p. 41]
	Tìm chữ số $a$ sao cho số $\overline{aaa}$ là tổng của các số tự nhiên liên tiếp từ $1$ đến số $n$ nào đó.
\end{baitoan}

\begin{baitoan}
	Chứng minh tập hợp các số nguyên tố có vô hạn phần tử \& không có số nguyên tố lớn nhất.
\end{baitoan}
\noindent\textit{Hint.} Giả sử phản chứng: chỉ có hữu hạn số nguyên tố $p_1 < p_2 < \cdots < p_n$. Chứng minh $p\coloneqq\prod_{i=1}^n p_i + 1 = p_1p_2\cdots p_n + 1$ là 1 số nguyên tố lớn hơn mỗi số nguyên tố $p_i$, $\forall i\in\mathbb{N}$.

\begin{baitoan}[\cite{Tuyen_Toan_6}, VD26, p. 25]
	Tìm số nguyên tố $a$ để $4a + 11$ là số nguyên tố nhỏ hơn $30$.
\end{baitoan}

\begin{baitoan}[\cite{Tuyen_Toan_6}, VD27, p. 25]
	Cho $A = \sum_{i=1}^{100} 5^i = 5 + 5^2 + \cdots + 5^{100}$. (a) Hỏi A là số nguyên tố hay hợp số? (b) Số A có phải là số chính phương không?
\end{baitoan}

\begin{baitoan}[\cite{Tuyen_Toan_6}, VD28, p. 2]
	Tính cạnh của 1 hình vuông có diện tích $\rm5929\ m^2$.
\end{baitoan}

\begin{baitoan}[\cite{Tuyen_Toan_6}, 116., p. 26]
	Phân loại số nguyên tố, hợp số: (a) $A = 1\cdot3\cdot5\cdot7\cdots13 + 20$. (b) $B = 147\cdot247\cdot347 - 13$.
\end{baitoan}

\begin{baitoan}[\cite{Tuyen_Toan_6}, 117., p. 26]
	Tìm số bị chia \& thương trong phép chia: $9\star\star:17 = \star\star$. Biết thương là 1 số nguyên tố.
\end{baitoan}

\begin{baitoan}[\cite{Tuyen_Toan_6}, 118., p. 26]
	Cho $a,n\in\mathbb{N}^\star$. Biết $a^n\divby5$. Chứng minh $a^2 + 150\divby25$.
\end{baitoan}

\begin{baitoan}[\cite{Tuyen_Toan_6}, 119., p. 26]
	(a) Cho $n\in\mathbb{N}$, $n\not{\divby}\ 3$. Chứng minh $n^2$ chia cho $3$ dư $1$. (b) Cho $p$ là 1 số nguyên tố lớn hơn $3$. Hỏi $p^2 + 2021$ là số nguyên tố hay hợp số?
\end{baitoan}

\begin{baitoan}[\cite{Tuyen_Toan_6}, 120., p. 26]
	Cho $n\in\mathbb{N}$, $n > 2$, $n\not{\divby}\ 3$. Chứng minh 2 số $n^2\pm1$ không thể đồng thời là 2 số nguyên tố.
\end{baitoan}

\begin{baitoan}[\cite{Tuyen_Toan_6}, 121., p. 26]
	Cho $p > 3,p + 8$ đều là số nguyên tố. Hỏi $p + 100$ là số nguyên tố hay hợp số?
\end{baitoan}

\begin{baitoan}[\cite{Tuyen_Toan_6}, 122., p. 26]
	Phân tích ra thừa số nguyên tố bằng cách hợp lý nhất: (a) $700,9000,210000$. (b) $500,1600,18000$.
\end{baitoan}

\begin{baitoan}[\cite{Tuyen_Toan_6}, 123., p. 26]
	Đếm số ước số của: $90,540,3675$.
\end{baitoan}

\begin{baitoan}[\cite{Tuyen_Toan_6}, 124., p. 26]
	Tìm: (a) 2 số tự nhiên liên tiếp có tích bằng $1260$. (b) 3 số tự nhiên liên tiếp có tích bằng $3360$.
\end{baitoan}

\begin{baitoan}[\cite{Tuyen_Toan_6}, 125., p. 26]
	Tìm: (a) 3 số chẵn liên tiếp có tích bằng $5760$. (b) 3 số lẻ liên tiếp có tích bằng $19575$.
\end{baitoan}

\begin{baitoan}[\cite{Tuyen_Toan_6}, 126., p. 26]
	Tính cạnh của 1 hình lập phương biết thể tích của nó là $\rm1728\ cm^3$.
\end{baitoan}

\begin{baitoan}[\cite{Tuyen_Toan_6}, 127., p. 27]
	Chứng minh 1 số tự nhiên $\ne0$ có số lượng các ước là 1 số lẻ $\Leftrightarrow$ số tự nhiên đó là số chính phương.
\end{baitoan}

\begin{baitoan}[\cite{Tuyen_Toan_6}, 128., p. 27]
	Tìm $n\in\mathbb{N}^\star$ thỏa: (a) $2 + 4 + 6 + \cdots + 2n = 210$. (b) $1 + 3 + 5 + \cdots + (2n - 1) = 225$.
\end{baitoan}

\begin{baitoan}[\cite{Binh_Toan_6_tap_1}, VD32, p. 30]
	Điền các chữ số thích hợp trong phép phân tích ra thừa số nguyên tố: $\overline{abcd} = e\overline{fcga} = en\overline{abc} = enc\overline{ncf} = \ldots$
\end{baitoan}

\begin{baitoan}[\cite{Binh_Toan_6_tap_1}, VD33, p. 30]
	Tìm số nguyên tố $p$ sao cho $p + 2,p + 4$ cũng là 2 số nguyên tố.
\end{baitoan}

\begin{baitoan}[\cite{Binh_Toan_6_tap_1}, VD34, p. 31]
	1 số nguyên tố $p$ chia cho $42$ có số dư $r$ là hợp số. Tìm số dư $r$.
\end{baitoan}

\begin{baitoan}[\cite{Binh_Toan_6_tap_1}, VD35, p. 31]
	Tìm $n\in\mathbb{N}^\star$ nhỏ nhất sao cho $n! + 1$ là hợp số.
\end{baitoan}

\begin{baitoan}[\cite{Binh_Toan_6_tap_1}, 180., p. 31]
	(a) Đếm số số nguyên tố nhỏ hơn $100$. (b) Tính tổng tất cả các số nguyên tố nhỏ hơn $100$.
\end{baitoan}

\begin{baitoan}[\cite{Binh_Toan_6_tap_1}, 181., p. 31]
	Tổng của 3 số nguyên tố bằng $1012$. Tìm số nhỏ nhất trong 3 số nguyên tố đó.
\end{baitoan}

\begin{baitoan}[\cite{Binh_Toan_6_tap_1}, 182., p. 31]
	Tìm 4 số nguyên tố liên tiếp, sao cho tổng của chúng là số nguyên tố.
\end{baitoan}

\begin{baitoan}[\cite{Binh_Toan_6_tap_1}, 183., p. 31]
	Tổng của 2 số nguyên tố có thể bằng $2003$ không?
\end{baitoan}

\begin{baitoan}[\cite{Binh_Toan_6_tap_1}, 184., p. 31]
	Tìm 2 số tự nhiên sao cho tổng \& tích của chúng đều là số nguyên tố.
\end{baitoan}

\begin{baitoan}[\cite{Binh_Toan_6_tap_1}, 185., p. 31]
	Trong 1 cuộc phỏng vấn tuyển nhân viên làm việc ở Tập đoàn Microsoft của Mỹ, 1 ứng viên nhận được câu hỏi: Tìm số tiếp theo trong dãy $4,6,12,18,30,42,60,\ldots$ Nhờ có kiến thức về số nguyên tố, ứng viên đã trả lời đúng. Số tiếp theo của dãy là số nào?
\end{baitoan}

\begin{baitoan}[\cite{Binh_Toan_6_tap_1}, 186., p. 31]
	Phân loại số nguyên tố \& hợp số: (a) $a = \underbrace{1\ldots1}_{2001}, b = \underbrace{1\ldots1}_{2000}, c = 1010101, d = 1112111, e = \sum_{i=1}^{100} i! = 1! + 2! + \cdots + 100!, f = 3\cdot5\cdot7\cdot9 - 28, g = 311141111$.
\end{baitoan}

\begin{baitoan}[\cite{Binh_Toan_6_tap_1}, 187., p. 31]
	Tìm số nguyên tố có 3 chữ số biết nếu viết số đó theo thứ tự ngược lại thì ta được 1 số là lập phương của 1 số tự nhiên.
\end{baitoan}

\begin{baitoan}[\cite{Binh_Toan_6_tap_1}, 188., p. 31]
	Tìm số tự nhiên có 4 chữ số, chữ số hàng nghìn bằng chữ số hàng đơn vị, chữ số hàng trăm bằng chữ số hàng chục, \& số đó viết được dưới dạng tích của 3 số nguyên tố liên tiếp.
\end{baitoan}

\begin{baitoan}[\cite{Binh_Toan_6_tap_1}, 189., p. 32]
	Tìm số nguyên tố $p$ sao cho các số sau cũng là số nguyên tố: (a) $p + 2,p + 10$. (b) $p + 10,p + 20$. (c) $p + 2,p + 6,p + 8,p + 12,p + 14$.
\end{baitoan}

\begin{baitoan}[\cite{Binh_Toan_6_tap_1}, 190., p. 32]
	Tìm số nguyên tố biết số đó bằng tổng của 2 số nguyên tố \& bằng hiệu của 2 số nguyên tố.
\end{baitoan}

\begin{baitoan}[\cite{Binh_Toan_6_tap_1}, 191., p. 32]
	Cho 3 số nguyên tố lớn hơn $3$, trong đó số sau lớn hơn số trước là $d$ đơn vị. Chứng minh $d\divby6$.
\end{baitoan}

\begin{baitoan}[\cite{Binh_Toan_6_tap_1}, 192., p. 32]
	2 số nguyên tố gọi là {\rm sinh đôi} nếu chúng là 2 số nguyên tố lẻ liên tiếp. Chứng minh 1 số tự nhiên lớn hơn $3$ nằm giữa 2 số nguyên tố sinh đôi thì chia hết cho $6$.
\end{baitoan}

\begin{baitoan}[\cite{Binh_Toan_6_tap_1}, 193., p. 32]
	Cho $p > 3$ là số nguyên tố. Biết $p + 2$ cũng là số nguyên tố. Chứng minh $p + 1\divby6$.
\end{baitoan}

\begin{baitoan}[\cite{Binh_Toan_6_tap_1}, 194., p. 32]
	Cho $p > 3,p + 4$ là 2 số nguyên tố. Chứng minh $p + 8$ là hợp số.
\end{baitoan}

\begin{baitoan}[\cite{Binh_Toan_6_tap_1}, 195., p. 32]
	Cho $p,8p - 1$ là 2 số nguyên tố. Chứng minh $8p + 1$ là hợp số.
\end{baitoan}

\begin{baitoan}[\cite{Binh_Toan_6_tap_1}, 196., p. 32]
	1 ngày đầu năm 2002, Huy viết thư hỏi ngày sinh của Long \& nhận được thư trả lời: Mình sinh ngày $a$, tháng $b$, năm $1900 + c$, \& đến nay $d$ tuổi. Biết $abcd = 59007$. Huy đã tính được ngày sinh của anh Long \& kịp viết thư mừng sinh nhật bạn. Tìm ngày sinh của Long.
\end{baitoan}

\begin{baitoan}[\cite{Binh_Toan_6_tap_1}, 197., p. 32]
	1 số nguyên tố chia cho $30$ có số dư là $r$. Tìm $r$ biết $r$ không là số nguyên tố.
\end{baitoan}

\begin{baitoan}[\cite{Binh_Toan_6_tap_1}, 198., p. 32]
	Chứng minh: (a) Số $17$ không viết được dưới dạng tổng của 3 hợp số khác nhau. (b) Mọi số lẻ lớn hơn $17$ đều viết được dưới dạng tổng của 3 hợp số khác nhau.
\end{baitoan}

\begin{baitoan}[\cite{Binh_Toan_6_tap_1}, 199., p. 32]
	Tuổi trung bình của 8 người là $15$, trong đó tuổi mỗi người đều là số nguyên tố. Trong 4 người nhiều tuổi nhất, có 3 người $19$ tuổi. Tuổi trung bình của người nhiều tuổi thứ 4 \& thứ 5 là $11$. Tính tuổi của người nhiều tuổi nhất.
\end{baitoan}

\begin{baitoan}[\cite{TLCT_THCS_Toan_6_so_hoc}, VD4.1, p. 29]
	$30,17$ chia cho $a\in\mathbb{N},a\ne1$ đều dư $r$. Tìm $a,r$.
\end{baitoan}

\begin{baitoan}[\cite{TLCT_THCS_Toan_6_so_hoc}, VD4.2, p. 29]
	Có hơn $20$ học sinh xếp thành 1 vòng tròn. Khi đếm theo chiều kim đồng hồ, bắt đầu từ số $1$, thì 2 số $24,900$ rơi vào cùng 1 học sinh. Có ít nhất bao nhiêu học sinh?
\end{baitoan}

\begin{baitoan}[\cite{TLCT_THCS_Toan_6_so_hoc}, VD4.3, p. 29]
	Tìm $n\in\mathbb{N}^\star$ biết $\sum_{i=1}^n i = 1 + 2 + \cdots + n = 378$.
\end{baitoan}

\begin{baitoan}[\cite{TLCT_THCS_Toan_6_so_hoc}, VD4.4, p. 30]
	Cho tích $800$ số tự nhiên từ $1$ đến $800$: $A = \prod_{i=1}^{800} i = 1\cdot2\cdots800$. (a) Dạng phân tích của A ra thừa số nguyên tố chứa thừa số $5$ với số mũ bao nhiêu? (b) A tận cùng bằng bao nhiêu chữ số $0$?
\end{baitoan}

\begin{baitoan}[\cite{TLCT_THCS_Toan_6_so_hoc}, VD4.5, p. 31]
	Chứng minh chỉ có duy nhất 1 bộ 3 số nguyên tố mà hiệu của 2 số liên tiếp bằng $4$.
\end{baitoan}

\begin{baitoan}[\cite{TLCT_THCS_Toan_6_so_hoc}, VD4.6, p. 31]
	Tìm số tự nhiên nhỏ nhất có $8$ ước số.
\end{baitoan}

\begin{baitoan}[\cite{TLCT_THCS_Toan_6_so_hoc}, VD4.7, p. 31]
	Viết mỗi số sau thành 1 tổng của các hợp số sao cho số số hạng của tổng là nhiều nhất: $100,101,102,103$.
\end{baitoan}

\begin{baitoan}[\cite{TLCT_THCS_Toan_6_so_hoc}, 4.1., p. 32]
	Trong 1 tháng, có 3 ngày chủ nhật là 3 số nguyên tố. Ngày 15 của tháng đó là ngày thứ mấy?
\end{baitoan}

\begin{baitoan}[\cite{TLCT_THCS_Toan_6_so_hoc}, 4.2., p. 32]
	Viết liên tiếp các số tự nhiên từ $1$ đến $99$, ta được 1 số A. A là số nguyên tố hay hợp số?
\end{baitoan}

\begin{baitoan}[\cite{TLCT_THCS_Toan_6_so_hoc}, 4.3., p. 32]
	Xét tính chính phương: (a) $\sum_{i=1}^{20} 2^i = 2 + 2^2 + 2^3 + \cdots + 2^{20}$. (b) $10^{15} + 8$.
\end{baitoan}

\begin{baitoan}[\cite{TLCT_THCS_Toan_6_so_hoc}, 4.4., p. 32]
	Cho $p,p + 14$ là 2 số nguyên tố. Chứng minh $p + 7$ là hợp số.
\end{baitoan}

\begin{baitoan}[\cite{TLCT_THCS_Toan_6_so_hoc}, 4.5., p. 32]
	Cho $p,p + 20,p + 40$ là 3 số nguyên tố. Chứng minh $p + 80$ là số nguyên tố.
\end{baitoan}

\begin{baitoan}[\cite{TLCT_THCS_Toan_6_so_hoc}, 4.6., p. 32]
	Tìm số nguyên tố $p$ sao cho $p + 6,p + 12,p + 18,p + 24$ cũng là 4 số nguyên tố.
\end{baitoan}

\begin{baitoan}[\cite{TLCT_THCS_Toan_6_so_hoc}, 4.7., p. 32]
	Tìm số nguyên tố nhỏ hơn $200$, biết khi chia nó cho $60$ thì số dư là hợp số.
\end{baitoan}

\begin{baitoan}[\cite{TLCT_THCS_Toan_6_so_hoc}, 4.8., p. 32]
	Chứng minh số $\underbrace{1\ldots1}_{10}2\underbrace{1\ldots1}_{10}$ là hợp số.
\end{baitoan}

\begin{baitoan}[\cite{TLCT_THCS_Toan_6_so_hoc}, 4.9., p. 32]
	Tìm 3 số tự nhiên liên tiếp có tích bằng $13800$.
\end{baitoan}

\begin{baitoan}[\cite{TLCT_THCS_Toan_6_so_hoc}, 4.10., p. 32]
	Tìm $n\in\mathbb{N}$ biết: (a) $2n + 1\divby n - 3$. (b) $n^2 + 3\divby n + 1$.
\end{baitoan}

\begin{baitoan}[\cite{TLCT_THCS_Toan_6_so_hoc}, 4.11., p. 33]
	Tìm $n\in\mathbb{N}$ biết: (a) $\sum_{i=1}^n i = 1 + 2 + \cdots + n = 231$. (b) $\sum_{i=1}^n 2i - 1 = 1 + 3 + 5 + \cdots + (2n - 1) = 169$.
\end{baitoan}

\begin{baitoan}[\cite{TLCT_THCS_Toan_6_so_hoc}, 4.12., p. 33]
	Tìm 2 số tự nhiên không chia hết cho $10$ \& có tích bằng $10000$.
\end{baitoan}

\begin{baitoan}[\cite{TLCT_THCS_Toan_6_so_hoc}, 4.13., p. 33]
	Chi tính tổng các số tự nhiên liên tiếp từ $1$ đến $n$ \& nhận thấy tổng đó chia hết cho $29$. Hoàng tính tổng các số tự nhiên từ $1$ đến $m$ \& cũng nhận thấy tổng đó chia hết cho $29$. Tìm $m,n$ biết $m < n < 50$.
\end{baitoan}

\begin{baitoan}[\cite{TLCT_THCS_Toan_6_so_hoc}, 4.14., p. 33]
	Chứng minh tồn tại $99$ số tự nhiên liên tiếp đều là hợp số.
\end{baitoan}

\begin{baitoan}[\cite{TLCT_THCS_Toan_6_so_hoc}, 4.15., p. 33]
	Tìm số tự nhiên lớn nhất có 2 chữ số: (a) Có ít ước số nhất. (b) Có $12$ ước số.
\end{baitoan}

\begin{baitoan}[\cite{TLCT_THCS_Toan_6_so_hoc}, 4.16., p. 33]
	Tìm số tự nhiên nhỏ nhất: (a) Có $7$ ước số. (b) Có $15$ ước số.
\end{baitoan}

\begin{baitoan}[\cite{TLCT_THCS_Toan_6_so_hoc}, 4.17., p. 33]
	Tìm số tự nhiên nhỏ nhất chỉ chứa các thừa số nguyên tố $2,5$, biết khi chia nó cho $2$ thì được 1 số chính phương.
\end{baitoan}

\begin{baitoan}[\cite{TLCT_THCS_Toan_6_so_hoc}, 4.18., p. 33]
	Tìm số tự nhiên nhỏ nhất khác $0$ sao cho khi chia nó cho $2$ thì được 1 số chính phương, khi chia nó cho $3$ thì được lập phương của 1 số tự nhiên.
\end{baitoan}

\begin{baitoan}[\cite{TLCT_THCS_Toan_6_so_hoc}, 4.19., p. 33]
	$n\in\mathbb{N}$ chỉ chứa 2 thừa số nguyên tố. Biết $n^2$ có $21$ ước số. Tính số ước số của $n^3,n^4,n^k$, $\forall k\in\mathbb{N}$.
\end{baitoan}

%------------------------------------------------------------------------------%

\section{Divisor \& Multiple -- Ước \& Bội}

\begin{baitoan}[\cite{Binh_Toan_6_tap_1}, VD36, p. 33]
	Tìm số chia \& thương của 1 phép chia có số bị chia bằng $145$, số dư bằng $12$ biết thương khác $1$.
\end{baitoan}

\begin{baitoan}[\cite{Binh_Toan_6_tap_1}, VD37, p. 33]
	Tìm số tự nhiên có 2 chữ số khác nhau sao cho nếu xóa bất kỳ chữ số nào của nó thì số nhận được vẫn là ước của số ban đầu.
\end{baitoan}

\begin{baitoan}[\cite{Binh_Toan_6_tap_1}, VD38, p. 33]
	1 tổ sản xuất được thưởng $840$ nghìn đồng. Số tiền thưởng chia đều cho số người trong tổ. Sau khi chia xong, tổ phát hiện đã bỏ sót không chia cho 1 người vắng mặt, do đó mỗi người được chia đã góp $2$ nghìn đồng \& kết quả là người vắng mặt cũng được nhận số tiền như những người có mặt. Tính số tiền mỗi người đã được thưởng (số tiền đó là 1 số tự nhiên với đơn vị nghìn đồng).
\end{baitoan}

\begin{baitoan}[\cite{Binh_Toan_6_tap_1}, VD39, p. 33]
	Trong 1 buổi họp mặt của 2 câu lạc bộ A \& B, mỗi người bắt tay 1 lần với tất cả những người còn lại. Tính số người của mỗi câu lạc bộ, biết có tất cả $496$ cái bắt tay, trong đó có $241$ cái bắt tay của 2 người trong cùng 1 câu lạc bộ.
\end{baitoan}

\begin{baitoan}[\cite{Binh_Toan_6_tap_1}, VD40, p. 34]
	Tìm 5 số tự nhiên khác nhau, biết khi nhân từng cặp 2 số thì tích nhỏ nhất bằng $28$, tích lớn nhất bằng $240$ \& 1 tích khác bằng $128$.
\end{baitoan}

\begin{baitoan}[\cite{Binh_Toan_6_tap_1}, VD41, p. 34]
	Viết số $108$ dưới dạng tổng các số tự nhiên liên tiếp lớn hơn $0$.
\end{baitoan}

\begin{baitoan}[\cite{Binh_Toan_6_tap_1}, 200., p. 35]
	Tìm $x,y\in\mathbb{N}$ sao cho: (a) $(2x + 1)(y - 3) = 10$. (b) $(3x - 2)(2y - 3) = 1$. (c) $(x + 1)(2y - 1) = 12$. (d) $x + 6 = y(x - 1)$. (e) $x - 3 = y(x + 2)$.
\end{baitoan}

\begin{baitoan}[\cite{Binh_Toan_6_tap_1}, 201., p. 35]
	1 phép chia số tự nhiên có số bị chia bằng $3193$. Tìm số chia \& thương của phép chia đó, biết số chia có 2 chữ số.
\end{baitoan}

\begin{baitoan}[\cite{Binh_Toan_6_tap_1}, 202., p. 35]
	Tìm số chia của 1 phép chia, biết: Số bị chia bằng $236$, số dư bằng $15$, số chia là số tự nhiên có 2 chữ số.
\end{baitoan}	

\begin{baitoan}[\cite{Binh_Toan_6_tap_1}, 203., p. 35]
	Tìm ước của $161$ trong khoảng từ $10$ đến $150$.
\end{baitoan}

\begin{baitoan}[\cite{Binh_Toan_6_tap_1}, 204., p. 35]
	Tìm 2 số tự nhiên liên tiếp có tích bằng $600$.
\end{baitoan}

\begin{baitoan}[\cite{Binh_Toan_6_tap_1}, 205., p. 35]
	Tìm 3 số tự nhiên liên tiếp có tích bằng $2730$.
\end{baitoan}

\begin{baitoan}[\cite{Binh_Toan_6_tap_1}, 206., p. 35]
	Tìm 3 số lẻ liên tiếp có tích bằng $12075$.
\end{baitoan}

\begin{baitoan}[\cite{Binh_Toan_6_tap_1}, 207., p. 35]
	Có 1 số số tự nhiên khác nhau được viết trên bảng. Tích của 2 số nhỏ nhất là $16$, tích của 2 số lớn nhất là $225$. Tính tổng của tất cả các số tự nhiên đó.
\end{baitoan}

\begin{baitoan}[\cite{Binh_Toan_6_tap_1}, 208., p. 35]
	Trên 1 tấm bia có các vòng tròn tính điểm là $18,23,28,33,38$. Muốn trúng thưởng, phải bắn 1 số phát tên để đạt đúng $100$ điểm. Hỏi phải bắn bao nhiêu phát tên \& vào những vòng nào?
\end{baitoan}

\begin{baitoan}[\cite{Binh_Toan_6_tap_1}, 209., p. 35]
	1 tờ hóa đơn bị dây mực, chỗ dây mực biểu thị bởi dấu $\star$. Phục hồi lại các chữ số bị dây mực (dấu $\star$ thay cho 1 hay nhiều chữ số). Giá mua 1 hộp bút: $3200$ đồng. Số hộp bút đã bán: $\star$ chiếc. Giá bán 1 hộp bút: $\star00$ đồng. Thành tiền: $107300$ đồng.
\end{baitoan}

\begin{baitoan}[\cite{Binh_Toan_6_tap_1}, 210., p. 36]
	Tìm $n\in\mathbb{N}$, biết: $\sum_{i=1}^n i = 1 + 2 + 3 + \cdots + n = 820$.
\end{baitoan}

\begin{baitoan}[\cite{Binh_Toan_6_tap_1}, 211., p. 36]
	Viết số $100$ dưới dạng tổng các số lẻ liên tiếp.
\end{baitoan}

\begin{baitoan}[\cite{Binh_Toan_6_tap_1}, 212., p. 36]
	Tân \& Hùng gặp nhau trong hội nghị học sinh giỏi Toán. Tân hỏi số nhà Hùng, Hùng trả lời: - Nhà mình ở chính giữa phố, đoạn phố ấy có tổng các số nhà bằng $161$. Nghĩ 1 chút, Tâm nói: - Bạn ở số nhà $23$ chứ gì! Hỏi Tân đã tìm ra như thế nào?
\end{baitoan}

\begin{baitoan}[\cite{Binh_Toan_6_tap_1}, 213., p. 36]
	Tìm $a,b,c,d\in\{1,2,\ldots,99\}$ sao cho $b\divby a,b\divby c,c\divby a,b\divby d,c\divby d$ \& $a$ có {\rm GTLN} trong các giá trị $a$ có thể nhận được.
\end{baitoan}

\begin{baitoan}[\cite{Binh_Toan_6_tap_1}, 214., p. 36]
	Tìm $n\in\mathbb{N}$, sao cho: (a) $n + 4\divby n + 1$. (b) $n^2 + 4\divby n + 2$. (c) $13n\divby n - 1$.
\end{baitoan}

\begin{baitoan}[\cite{Binh_Toan_6_tap_1}, 215., p. 36]
	Tìm số tự nhiên có 3 chữ số, biết nó tăng gấp $n$ lần nếu cộng mỗi chữ số của nó với $n$ ($n\in\mathbb{N}$, có thể gồm 1 hoặc nhiều chữ số).
\end{baitoan}

\begin{baitoan}[\cite{Binh_Toan_6_tap_1}, 216., p. 36]
	2 công ty A \& B năm trước có số nhân viên bằng nhau. Năm sau, công ty A tuyển thêm số nhân viên mới bằng 4 lần số nhân viên cũ, còn công ty B cho nghỉ việc 5 nhân viên, do đó số nhân viên công ty A là bội của số nhân viên công ty B. Hỏi năm trước mỗi công ty có nhiều nhất bao nhiêu nhân viên?
\end{baitoan}

%------------------------------------------------------------------------------%

\section{Common Divisor. Least Common Divisor -- Ước Chung. Ước Chung Lớn Nhất}

\begin{baitoan}[\cite{Binh_Toan_6_tap_1}, VD42, p. 36]
	Tìm $a\in\mathbb{N}$ biết $264$ chia cho $a$ dư $24$, còn $363$ chia cho $a$ dư $43$.
\end{baitoan}

\begin{baitoan}[\cite{Binh_Toan_6_tap_1}, VD44, p. 37]
	Trên 1 hành tinh, các cư dân chia 1 ngày đêm thành $a$ giờ, chia 1 giờ thành $b$ phút, chia 1 phút thành $c$ giây ($a,b,c\in\mathbb{N}$). Biết 1 ngày đêm có $620$ phút, mỗi giờ có $899$ giây. Hỏi trên hành tinh đó, mỗi ngày đêm gồm bao nhiêu giây?
\end{baitoan}

\begin{baitoan}[\cite{Binh_Toan_6_tap_1}, 217., p. 37]
	Tìm $a\in\mathbb{N}$, biết $398$ chia cho $a$ thì dư $38$, còn $450$ chia cho $a$ thì dư $18$.
\end{baitoan}

\begin{baitoan}[\cite{Binh_Toan_6_tap_1}, 218., p. 37]
	Tìm $a\in\mathbb{N}$, biết $350$ chia cho $a$ thì dư $14$, còn $320$ chia cho $a$ thì dư $26$.
\end{baitoan}

\begin{baitoan}[\cite{Binh_Toan_6_tap_1}, 219., p. 37]
	Có $100$ quyển vở \& $90$ bút chì được thưởng đều cho 1 số học sinh, còn lại $4$ quyển vở \& $18$ bút chì không đủ chia đều. Tính số học sinh được thưởng.
\end{baitoan}

\begin{baitoan}[\cite{Binh_Toan_6_tap_1}, 220., p. 37]
	Phần thưởng cho học sinh của 1 lớp học gồm $128$ vở, $48$ bút chì, $192$ nhãn vở. Có thể chia được nhiều nhất thành bao nhiêu phần thưởng như nhau, mỗi phần thưởng gồm bao nhiêu vở, bút chì, nhãn vở?
\end{baitoan}

\begin{baitoan}[\cite{Binh_Toan_6_tap_1}, 221., p. 37]
	3 khối $6,7,8$ theo thứ tự có $300$ học sinh, $276$ học sinh, $252$ học sinh xếp hàng dọc để diễu hành sao cho số hàng dọc của mỗi khối như nhau. Có thể xếp nhiều nhất thành mấy hàng dọc để mỗi khối đều không có ai lẻ hàng? Khi đó ở mỗi khối có bao nhiêu hàng ngang?
\end{baitoan}

\begin{baitoan}[\cite{Binh_Toan_6_tap_1}, 222., p. 37]
	Người ta muốn chia $200$ bút bi, $240$ bút chì, $320$ tẩy thành 1 số phần thưởng như nhau. Hỏi có thể chia được nhiều nhất thành bao nhiêu phần thưởng, mỗi phần thưởng có bao nhiêu bút bi, bút chì, tẩy?
\end{baitoan}

\begin{baitoan}[\cite{Binh_Toan_6_tap_1}, 223., p. 38]
	Các số $1620$ \& $1410$ chia cho số tự nhiên $a$ có 3 chữ số cùng được số dư là $r$. Tìm $a$ \& $r$.
\end{baitoan}

\begin{baitoan}[\cite{Binh_Toan_6_tap_1}, 224., p. 38]
	Tìm số chia \& thương của 1 phép chia số tự nhiên có số bị chia bằng $9578$ \& các số dư liên tiếp là $5,3,2$.
\end{baitoan}

%------------------------------------------------------------------------------%

\section{Multiple. Common Multiple -- Bội. Bội Chung}

\begin{baitoan}[\cite{Binh_Toan_6_tap_1}, VD45, p. 38]
	Tìm số tự nhiên $a$ nhỏ nhất sao cho chia $a$ cho $3$, cho $5$, cho $7$ được số dư theo thứ tự là $2,3,4$.
\end{baitoan}

\begin{baitoan}[\cite{Binh_Toan_6_tap_1}, VD46, p. 38]
	1 số tự nhiên chia cho $3$ thì dư $1$, chia cho $4$ thì dư $2$, chia cho $5$ thì dư $3$, chia cho $6$ thì dư $4$, \& chia hết cho $13$. (a) Tìm số nhỏ nhất có tính chất trên. (b) Tìm dạng chung của tất cả các số có tính chất trên.
\end{baitoan}

\begin{baitoan}[\cite{Binh_Toan_6_tap_1}, VD47, p. 39]
	3 người mua 3 chiếc ô tô cùng loại với cùng 1 giá. Ông A đặt cọc $130$ triệu đồng, mỗi tháng trả $18$ triệu đồng thì trả xong. Ông B đặt cọc $100$ triệu đồng, mỗi tháng trả $24$ triệu đồng thì trả xong. Ông C đặt cọc $60$ triệu đồng, mỗi tháng trả $28$ triệu đồng thì trả xong. Tính giá mỗi chiếc ô tô, biết giá ô tô chưa đến $900$ triệu đồng.
\end{baitoan}

\begin{baitoan}[\cite{Binh_Toan_6_tap_1}, 225., p. 39]
	Tìm các bội chung của $40,60,126$ \& nhỏ hơn $6000$.
\end{baitoan}

\begin{baitoan}[\cite{Binh_Toan_6_tap_1}, 226., p. 39]
	1 cuộc thi chạy tiếp sức theo vòng tròn gồm nhiều chặng. Biết chu vi đường tròn là $330$\emph{m}, mỗi chặng dài $75$\emph{m}, địa điểm xuất phát \& kết thúc cùng 1 chỗ. Hỏi cuộc thi có ít nhất mấy chặng?
\end{baitoan}

\begin{baitoan}[\cite{Binh_Toan_6_tap_1}, 227., p. 39]
	3 ô tô cùng khởi hành 1 lúc từ 1 bến. Thời gian cả đi lẫn về của xe thứ nhất là $40$ phút, của xe thứ 2 là $50$ phút, của xe thứ 3 là $30$ phút. Khi trở về bến, mỗi xe đều nghỉ $10$ phút rồi tiếp tục chạy. Hỏi sau ít nhất bao lâu: (a) Xe thứ nhất \& xe thứ 2 cùng rời bến? (b) Xe thứ 2 \& xe thứ 3 cùng rời bến? (c) Cả 3 xe cùng rời bến?
\end{baitoan}

\begin{baitoan}[\cite{Binh_Toan_6_tap_1}, 228., p. 39]
	1 đơn vị bộ đội khi xếp hàng $20,25,30$ đều dư $15$, nhưng xếp hàng $41$ thì vừa đủ. Tính số người của đơn vị đó biết số người chưa đến $1000$.
\end{baitoan}

\begin{baitoan}[\cite{Binh_Toan_6_tap_1}, 229., p. 39]
	1 chiếc xe đạp xiếc có chu vi bánh xe lớn $21$\emph{dm}, chu vi bánh xe nhỏ $9$\emph{dm}. Hiện nay van của 2 bánh xe đều ở vị trí thấp nhất. Hỏi xe phải lăn bao nhiêu mét nữa thì 2 van của 2 bánh xe lại ở vị trí thấp nhất?
\end{baitoan}

\begin{baitoan}[\cite{Binh_Toan_6_tap_1}, 230., p. 39]
	Tìm $n\in\mathbb{N}$ có 3 chữ số sao cho $n + 6\divby7,n + 7\divby8,n + 8\divby9$.
\end{baitoan}

\begin{baitoan}[\cite{Binh_Toan_6_tap_1}, 231., p. 39]
	Tìm số tự nhiên có 3 chữ số, sao cho chia nó cho $17$, cho $25$ được các số dư theo thứ tự là $8$ \& $16$.
\end{baitoan}

\begin{baitoan}[\cite{Binh_Toan_6_tap_1}, 232., p. 39]
	Tìm $n\in\mathbb{N}$ lớn nhất có 3 chữ số, sao cho $n$ chia cho $8$ thì dư $7$, chia cho $31$ thì dư $28$.
\end{baitoan}

\begin{baitoan}[\cite{Binh_Toan_6_tap_1}, 233., p. 40]
	Tìm số tự nhiên nhỏ hơn $500$, sao cho chia nó cho $15$, cho $35$ được các số dư theo thứ tự là $8$ \& $13$.
\end{baitoan}

\begin{baitoan}[\cite{Binh_Toan_6_tap_1}, 234., p. 40]
	(a) Tìm số tự nhiên lớn nhất có 3 chữ số, sao cho chia nó cho $2$, cho $3$, cho $4$, cho $5$, cho $6$ ta được các số dư theo thứ tự là $1,2,3,4,5$. (b) Tìm dạng chung của các số tự nhiên $a$ chia cho $4$ dư $3$, chia cho $5$ thì dư $4$, chia cho $6$ thì dư $5$, chia hết cho $13$.
\end{baitoan}

\begin{baitoan}[\cite{Binh_Toan_6_tap_1}, 235., p. 40]
	Tìm số tự nhiên nhỏ nhất chia cho $8$ dư $6$, chia cho $12$ dư $10$, chia cho $15$ dư $13$ \& chia hết cho $13$.
\end{baitoan}

\begin{baitoan}[\cite{Binh_Toan_6_tap_1}, 236., p. 40]
	Tìm số tự nhiên nhỏ nhất chia cho $8,10,15,20$ theo thứ tự dư $5,7,12,17$ \& chia hết cho $41$.
\end{baitoan}

\begin{baitoan}[\cite{Binh_Toan_6_tap_1}, 237., p. 40]
	Chị Mai xếp bánh (ít hơn $100$ chiếc) vào các đĩa. Nếu mỗi đĩa xếp $8$ bánh thì có $1$ đĩa chỉ có $3$ chiếc bánh. Nếu mỗi đĩa xếp $7$ chiếc bánh thì có $1$ đĩa chỉ có $5$ chiếc bánh. Nếu mỗi đĩa xếp $3$ chiếc bánh thì có $1$ đĩa chỉ có $1$ chiếc bánh. Tìm số bánh.
\end{baitoan}

\begin{baitoan}[\cite{Binh_Toan_6_tap_1}, 238., p. 40]
	7 người có 7 mảnh đất diện tích bằng nhau. Người thứ nhất trồng 1 cây cam. Người thứ 2 trồng $2$ cây cam. Người thứ 3 trồng $3$ cây cam. $\ldots$ Người thứ 7 trồng $7$ cây cam. Điều đặc biệt là ai cũng thấy các cây cam của mình có số quả bằng nhau. Ngoài ra số cam của mỗi người không chênh lệch nhiều nên sau khi người thứ 7 cho người thứ 2,3,4,5,6 mỗi người $1$ quả cam thì cả 7 người đều có số cam bằng nhau. Tính số cam trên cây của mỗi người lúc đầu, biết không có cây cam nào có hơn $200$ quả.
\end{baitoan}

\begin{baitoan}[\cite{Binh_Toan_6_tap_1}, 239., p. 40]
	Tìm số tự nhiên nhỏ nhất chia cho $5$, cho $7$, cho $9$ có số dư theo thứ tự là $3,4,5$.
\end{baitoan}

\begin{baitoan}[\cite{Binh_Toan_6_tap_1}, 240., p. 40]
	Tìm số tự nhiên nhỏ nhất chia cho $3$, cho $4$, cho $5$ có số dư theo thứ tự là $1,3,1$.
\end{baitoan}

\begin{baitoan}[\cite{Binh_Toan_6_tap_1}, 241., p. 40]
	Trên đoạn đường dài $4800$\emph{m} có các cột điện trồng cách nhau $60$\emph{m}, nay trồn lại cách nhau $80$\emph{m}. Hỏi có bao nhiêu cột không phải trồng lại, biết ở cả 2 đầu đoạn đường đều có cột điện?
\end{baitoan}

\begin{baitoan}[\cite{Binh_Toan_6_tap_1}, 242., p. 40]
	3 con tàu cập bến theo lịch như sau: Tàu I cứ $15$ ngày thì cập bến, tàu II cứ $20$ ngày thì cập bến, tàu III cứ $12$ ngày thì cập bến. Lần đầu cả 3 tàu cùng cập bến vào ngày thứ 6. Hỏi sau đó ít nhất bao lâu, cả 3 tàu lại cùng cập bến vào ngày thứ 6?
\end{baitoan}

\begin{baitoan}[\cite{Binh_Toan_6_tap_1}, 243., p. 40]
	Nếu xếp 1 số sách vào từng túi $10$ cuốn thì vừa hết, vào từng túi $12$ cuốn thì thừa $2$ cuốn, vào từng túi $18$ cuốn thì thừa $8$ cuốn. Biết số sách trong khoảng từ $715$ đến $1000$, tính số sách đó.
\end{baitoan}

\begin{baitoan}[\cite{Binh_Toan_6_tap_1}, 244., p. 40]
	2 lớp 6A, 6B cùng thu nhặt 1 số giấy vụn bằng nhau. Trong lớp 6A, 1 bạn thu được $26$\emph{kg}, còn lại mỗi bạn thu $11$\emph{kg}. Trong lớp 6B, 1 bạn thu được $25$\emph{kg}, còn lại mỗi bạn thu $10$\emph{kg}. Tính số học sinh mỗi lớp, biết số giấy mỗi lớp thu được trong khoảng từ $200$\emph{kg} đến $300$\emph{kg}.
\end{baitoan}

\begin{baitoan}[\cite{Binh_Toan_6_tap_1}, 245., p. 41]
	1 thiết bị điện tử phát ra tiếng kêu ``bíp' sau mỗi $60$ giây, 1 thiết bị điện tử khác phát ra tiếng kêu ``bíp'' sau mỗi $62$ giây. Cả 2 thiết bị này đều phát ra tiếng ``bíp'' lúc $10:00$. Tính thời điểm để cả 2 cùng phát ra tiếng ``bíp'' tiếp theo.
\end{baitoan}

\begin{baitoan}[\cite{Binh_Toan_6_tap_1}, 246., p. 41]
	Có 2 chiếc đồng hồ (có kim giờ \& kim phút). Trong 1 ngày, chiếc thứ nhất chạy nhanh $2$ phút, chiếc thứ 2 chạy chậm $3$ phút. Cả 2 đồng hồ được lấy lại theo giờ chính xác. Hỏi sau ít nhất bao nhiêu lâu, cả 2 đồng hồ lại cùng chỉ giờ chính xác?
\end{baitoan}

%------------------------------------------------------------------------------%

\section{Greatest Common Divisor. Least Common Multiple -- Ước Chung Lớn Nhất. Bội Chung Nhỏ Nhất}

\begin{baitoan}[\cite{Binh_boi_duong_Toan_6_tap_1}, H1, p. 43]
	1 thửa ruộng hình chữ nhật có chiều dài {\rm72 m}, chiều rộng {\rm40 m}. Chicken muốn chia thửa ruộng thành các mảnh đất hình vuông bằng nhau để trồng các loại ngũ cốc. Tính độ dài lớn nhất của hình vuông mà Chicken có thể chia.
\end{baitoan}

\begin{baitoan}[\cite{Binh_boi_duong_Toan_6_tap_1}, H2, p. 43]
	Có 4 thuyền A, B, C, D. Thuyền A cứ $5$ ngày cập bến 1 lần, thuyền B cứ $6$ ngày cập bến  1 lần, thuyền C cứ $8$ ngày cập bến 1 lần \& thuyền D cứ $10$ ngày cập bến 1 lần. Egg nhẩm tính: Nếu ngày hôm nay cả 4 thuyền cùng cập bến thì: (a) Sau ít nhất $a$ ngày nữa, thuyền A cùng cập bến với thuyền D. (b) Sau ít nhất $b$ ngày nữa, thuyền B cùng cập bến với thuyền C. (c) Sau ít nhất $c$ ngày nữa, thuyền B cùng cập bến với thuyền D. (d) Sau ít nhất $d$ ngày nữa, cả 4 thuyền sẽ cùng cập bến lần thứ 2. Tìm $a,b,c,d$.
\end{baitoan}

\begin{baitoan}[\cite{Binh_boi_duong_Toan_6_tap_1}, VD1, p. 43]
	Tìm $\mbox{\rm ƯC}(48,60)$, ${\rm BC}(4,14)$.
\end{baitoan}

\begin{baitoan}[\cite{Binh_boi_duong_Toan_6_tap_1}, VD2, p. 44]
	Tìm $a\in\mathbb{N}$ biết chia $264$ cho $a$ thì dư $24$, còn khi chia $363$ cho $a$ thì được dư là $43$.
\end{baitoan}

\begin{baitoan}[\cite{Binh_boi_duong_Toan_6_tap_1}, VD3, p. 44]
	Tìm số tự nhiên nhỏ nhất có 4 chữ số biết khi chia số đó cho $18,24,30$ thì có số dư lần lượt là $13,19,25$.
\end{baitoan}

\begin{baitoan}[\cite{Binh_boi_duong_Toan_6_tap_1}, VD4, p. 44]
	Tìm $a,b\in\mathbb{N}$ thỏa $a + b = 336$ \& $\mbox{\rm ƯCLN}(a,b) = 24$.
\end{baitoan}

\begin{baitoan}[\cite{Binh_boi_duong_Toan_6_tap_1}, VD5, p. 45]
	Tìm $a,b\in\mathbb{N}$ thỏa $\mbox{\rm ƯCLN}(a,b) = 24$ \& ${\rm BCNN}(a,b) = 36$.
\end{baitoan}

\begin{baitoan}[\cite{Binh_boi_duong_Toan_6_tap_1}, VD6, p. 45]
	Cho $n\in\mathbb{N}^\star$. Chứng minh: $\mbox{\rm ƯCLN}(2n + 5,3n + 7) = 1$.
\end{baitoan}

\begin{baitoan}[\cite{Binh_boi_duong_Toan_6_tap_1}, VD7, p. 46]
	Học sinh khối 6 của 1 trường khi xếp hàng $12$, hàng $15$ hay hàng $18$ thì đều vừa đủ hàng. Tính số học sinh khối 6 của trường đó biết số học sinh này nằm trong khoảng từ $500$ đến $600$ học sinh.
\end{baitoan}

\begin{baitoan}[\cite{Binh_boi_duong_Toan_6_tap_1}, VD8, p. 46]
	1 lớp học có $28$ học sinh nam \& $24$ học sinh nữ. Khi tham gia lao động, {\rm GVCN} muốn chia lớp thành các nhóm sao cho số học sinh nam \& số học sinh nữ được chia đều vào các nhóm. Hỏi {\rm GVCN} có bao nhiêu cách chia nhóm? Cách chia nào có số học sinh trong mỗi nhóm ít nhất?
\end{baitoan}

\begin{baitoan}[\cite{Binh_boi_duong_Toan_6_tap_1}, 6.1., p. 47]
	Tìm $\mbox{\rm ƯC}(54,120,180)$, ${\rm BC}(21,84)$.
\end{baitoan}

\begin{baitoan}[\cite{Binh_boi_duong_Toan_6_tap_1}, 6.2., p. 47]
	1 số chia cho $21$ dư $2$ \& chia cho $12$ dư $5$. Hỏi số đó chia cho $84$ thì dư bao nhiêu?
\end{baitoan}

\begin{baitoan}[\cite{Binh_boi_duong_Toan_6_tap_1}, 6.3., p. 47]
	Tìm $a\in\mathbb{N}$ thỏa mãn: $a\divby7$ \& $a$ chia cho $4$ hoặc $6$ đều dư $3$ biết $a < 350$.
\end{baitoan}

\begin{baitoan}[\cite{Binh_boi_duong_Toan_6_tap_1}, 6.4., p. 47]
	Tìm số tự nhiên lớn nhất có 3 chữ số sao cho chia nó cho $3$, cho $4$, cho $5$ ta được 3 số dư theo thứ tự là $2,3,4$.
\end{baitoan}

\begin{baitoan}[\cite{Binh_boi_duong_Toan_6_tap_1}, 6.5., p. 47]
	Cho $\mbox{\rm ƯCLN}(a,b) = 1$. Chứng minh: (a) $\mbox{\rm ƯCLN}(a,a - b) = 1$ với $a > b$. (b) $\mbox{\rm ƯCLN}(ab,a + b) = 1$.
\end{baitoan}

\begin{baitoan}[\cite{Binh_boi_duong_Toan_6_tap_1}, 6.6., p. 47]
	Cho $n\in\mathbb{N}$. Chứng minh: (a) $\mbox{\rm ƯCLN}(3n + 13,3n + 14) = 1$. (b) $\mbox{\rm ƯCLN}(3n +5,6n + 9) = 1$.
\end{baitoan}

\begin{baitoan}[\cite{Binh_boi_duong_Toan_6_tap_1}, 6.7., p. 47]
	1 lớp học có $27$ học sinh nam \& $18$ học sinh nữ. Có bao nhiêu cách chia lớp đó thành các tổ sao cho số học sinh nam \& số học sinh nữ được chia đều vào các tổ? Biết số tổ lớn hơn $1$.
\end{baitoan}

\begin{baitoan}[\cite{Binh_boi_duong_Toan_6_tap_1}, 6.8., p. 47]
	1 đơn vị bộ đội khi xếp hàng, mỗi hàng có $20$ người, hoặc $25$ người, hoặc $30$ người đều thừa $15$ người. Nếu xếp mỗi hàng $41$ người thì vừa đủ (không có hàng nào thiếu, không có ai ở ngoài hàng). Hỏi đơn vị có bao nhiêu người biết số người của đơn vị chưa đến $1000$?
\end{baitoan}

\begin{baitoan}[\cite{Binh_boi_duong_Toan_6_tap_1}, 6.9., p. 47]
	Tổng số học sinh khối 6 của 1 trường có khoảng từ $235$ đến $250$ em, khi chia cho $3$ thì dư $2$, chia cho $4$ thì dư $3$, chia cho $5$ thì dư $4$, chia cho $6$ thì dư $5$, chia cho $10$ thì dư $9$. Tìm số học sinh của khối 6.
\end{baitoan}

\begin{baitoan}[\cite{Binh_boi_duong_Toan_6_tap_1}, 6.10., p. 47]
	1 trường tổ chức cho học sinh đi tham quan bằng ôtô. Nếu xếp $27$ hay $36$ học sinh lên 1 ôtô thì đều thấy thừa ra $11$ học sinh. Tính số học sinh đi tham quan biết số học sinh đó có khoảng từ $400$ đến $450$ em.
\end{baitoan}

\begin{baitoan}[\cite{Binh_boi_duong_Toan_6_tap_1}, 6.11., p. 47]
	Cho 2 số nguyên tố cùng nhau $a,b$. Chứng minh 2 số $13a + 4b$ \& $15a + 7b$ hoặc nguyên tố cùng nhau hoặc có 1 ước chung là $31$.
\end{baitoan}

\begin{baitoan}[\cite{Binh_boi_duong_Toan_6_tap_1}, 6.12., p. 47]
	Cho $a,b\in\mathbb{N}$ không nguyên tố cùng nhau thỏa $a = 2n + 3$, $b = 3n + 1$ với $n\in\mathbb{N}$. Tìm $\mbox{\rm ƯCLN}(a,b)$.
\end{baitoan}

\begin{baitoan}[\cite{Binh_boi_duong_Toan_6_tap_1}, 6.13., p. 47]
	Tìm $a,b\in\mathbb{N}$ biết: (a) $5a = 13b$ \& $\mbox{\rm ƯCLN}(a,b) = 48$. (b) ${\rm BCNN}(a,b) = 360$ \& $ab = 6480$. (c) $a + b = 40$ \& ${\rm BCNN}(a,b) = 7\mbox{\rm ƯCLN}(a,b)$.
\end{baitoan}

\begin{baitoan}[\cite{Binh_boi_duong_Toan_6_tap_1}, 6.14., p. 47]
	Tìm $a,b\in\mathbb{N}$ biết $a + 2b = 48$ \& $\mbox{\rm ƯCLN}(a,b) + 3{\rm BCNN}(a,b) = 114$.
\end{baitoan}

\begin{baitoan}[\cite{Binh_boi_duong_Toan_6_tap_1}, 6.15., p. 47]
	Tìm $a,b\in\mathbb{N}$ biết $\mbox{\rm ƯCLN}(a,b) + {\rm BCNN}(a,b) = 21$.
\end{baitoan}

\begin{baitoan}[\cite{Binh_boi_duong_Toan_6_tap_1}, 6.16., p. 47]
	Cho $a = 123456789$ \& $b = 987654321$. Tìm $\mbox{\rm ƯCLN}(a,b)$.
\end{baitoan}

\begin{baitoan}[\cite{Binh_boi_duong_Toan_6_tap_1}, 6.17., p. 47, Thừa Thiên -- Huế 2007]
	Tìm 4 chữ số $a,b,c,d$ sao cho 4 số $a,\overline{ad},\overline{cd},\overline{abcd}$ là 4 số chính phương.
\end{baitoan}

\begin{baitoan}[\cite{Binh_boi_duong_Toan_6_tap_1}, p. 48, 1 số tính chất mở rộng về ƯCLN \& BCNN]
	Chứng minh: (a) $\mbox{\rm ƯCLN}(a,b,c) = \mbox{\rm ƯCLN}(\mbox{\rm ƯCLN}(a,b),c)$. (b) ${\rm BCNN}(a,b,c) = {\rm BCNN}({\rm BCNN}(a,b),c)$. (c) $\mbox{\rm ƯCLN}(ma,mb) = m\mbox{\rm ƯCLN}(a,b)$. (d) ${\rm BCNN}(ma,mb) = m{\rm BCNN}(a,b)$. (e) $\mbox{\rm ƯCLN}(a + kb,b) = \mbox{\rm ƯCLN}(a,b)$. (f) Nếu $ab\divby m$ \& $\mbox{\rm ƯCLN}(b,m) = 1$ thì $a\divby m$. (g) Nếu $a\divby m$ \& $a\divby n$ thì $a\divby{\rm BCNN}(m,n)$. (h) Nếu $a\divby m$, $a\divby n$, \& $\mbox{\rm ƯCLN}(m,n) = 1$ thì $a\divby mn$.
\end{baitoan}

\begin{baitoan}[\cite{Tuyen_Toan_6}, VD29, p. 27]
	Tìm $b\in\mathbb{N}$ biết khia chia $326$ cho $b$ thì dư $11$ còn chia $533$ cho $b$ thì dư $13$.
\end{baitoan}

\begin{baitoan}[\cite{Tuyen_Toan_6}, VD30, p. 28]
	Chứng minh 2 số tự nhiên liên tiếp là 2 số nguyên tố cùng nhau.
\end{baitoan}

\begin{baitoan}[\cite{Tuyen_Toan_6}, 129., p. 28]
	Tìm {\rm ƯCLN, ƯC} của 3 số $432,504,720$.
\end{baitoan}

\begin{baitoan}[\cite{Tuyen_Toan_6}, 130., p. 28]
	Tìm $x\in\mathbb{N}$ lớn nhất sao cho $x + 495,195 - x$ đều chia hết cho $x$.
\end{baitoan}

\begin{baitoan}[\cite{Tuyen_Toan_6}, 131., p. 28]
	1 căn phòng hình chữ nhật có kích thước $630\times480$ {\rm cm} được lát loại gạch hình vuông. Muốn cho 2 hàng gạch cuối cùng sát với 2 bức tường liên tiếp không bị cắt xén thì kích thước lớn nhất của viên gạch là bao nhiêu? Với loại gạch này thì cần bao nhiêu viên gạch để lát cả căn phòng?
\end{baitoan}

\begin{baitoan}[\cite{Tuyen_Toan_6}, 132., p. 28]
	2 lớp 6A, 6B cùng tham gia góp sách truyện để xây dựng thư viện. Mỗi học sinh góp số quyển sách như nhau. Tổng kết lại, lớp 6A góp được $36$ quyển, lớp 6B góp được $39$ quyển. Hỏi mỗi lớp có bao nhiêu bạn góp sách xây dựng thư viện?
\end{baitoan}

\begin{baitoan}[\cite{Tuyen_Toan_6}, 133., p. 28]
	Chứng minh các số sau nguyên tố cùng nhau: (a) 2 số lẻ liên tiếp. (b) $2n + 5,3n + 7$, với $n\in\mathbb{N}$.
\end{baitoan}

\begin{baitoan}[\cite{Tuyen_Toan_6}, 134., p. 28]
	Cho $(a,b) = 1$. Chứng minh $(a,a - b) = 1$. (b) $(ab,a + b) = 1$.
\end{baitoan}

\begin{baitoan}[\cite{Tuyen_Toan_6}, 135., p. 28]
	Cho $a,b$ là 2 số tự nhiên không nguyên tố cùng nhau, $a = 4n + 3,b = 5n + 1$, $n\in\mathbb{N}$. Tìm $(a,b)$.
\end{baitoan}

\begin{baitoan}[\cite{Tuyen_Toan_6}, 136., p. 28]
	{\rm ƯCLN} của 2 số là $45$. Số lớn là $270$. Tìm số nhỏ.
\end{baitoan}

\begin{baitoan}[\cite{Tuyen_Toan_6}, 137., p. 28]
	Tìm 2 số biết tổng của chúng là $162$ \& {\rm ƯCLN} của chúng là $18$.
\end{baitoan}

\begin{baitoan}[\cite{Tuyen_Toan_6}, 138., p. 28]
	Tìm 2 số tự nhiên nhỏ hơn $200$ biết hiệu của chúng là $90$ \& {\rm ƯCLN} của chúng là $15$.
\end{baitoan}

\begin{baitoan}[\cite{Tuyen_Toan_6}, 139., p. 28]
	Tìm 2 số biết tích của chúng là $8748$ \& {\rm ƯCLN} của chúng là $27$.
\end{baitoan}

\begin{baitoan}[\cite{Tuyen_Toan_6}, 140., p. 28]
	$a\in\mathbb{N}$ \& $5$ lần số $a$ có tổng các chữ số như nhau. Chứng minh $a\divby9$.
\end{baitoan}

\begin{baitoan}[\cite{Tuyen_Toan_6}, 141., p. 28]
	Có $64$ người đi tham quan bằng 2 loại xe: loại $12$ chỗ ngồi \& loại $7$ chỗ ngồi (không kể lái xe). Biết số người đi vừa đủ số ghế ngồi, tính số xe mỗi loại.
\end{baitoan}

\begin{baitoan}[\cite{Tuyen_Toan_6}, VD31, p. 29]
	Tìm số tự nhiên nhỏ nhất có $3$ chữ số chia cho $18,30,45$ có số dư lần lượt là $8,20,35$.
\end{baitoan}

\begin{baitoan}[\cite{Tuyen_Toan_6}, VD32, p. 30]
	Trong tiết học thể dục hôm nay, thầy giáo cho các bạn trong lớp xếp hàng $4$, hàng $6$, hàng $9$, mỗi lần đều thấy thừa $2$ học sinh. Thầy không cần biết mỗi lần có bao nhiêu hàng, thầy nói ngay số học sinh có mặt là $38$. Giải thích cách tính nhẩm của thầy.
\end{baitoan}

\begin{baitoan}[\cite{Tuyen_Toan_6}, 142., p. 30]
	1 xe lăn dành cho người khuyết tật có chu vi bánh trước là {\rm63 cm}, chu vi bánh sau là {\rm186 cm}. Người ta đánh dấu 2 điểm tiếp đất của 2 bánh xe này. Hỏi mỗi bánh xe phải lăn ít nhất bao nhiêu vòng thì 2 điểm đã đánh dấu lại cùng tiếp đất 1 lúc?
\end{baitoan}

\begin{baitoan}[\cite{Tuyen_Toan_6}, 143., p. 30]
	3 học sinh mỗi người mua 1 loại bút. Giá tiền 3 loại lần lượt là $4800$ đồng, $6000$ đồng, $8000$ đồng. Biết số tiền phải trả như nhau, hỏi mỗi học sinh mua ít nhất bao nhiêu bút?
\end{baitoan}

\begin{baitoan}[\cite{Tuyen_Toan_6}, 144., p. 30]
	Tìm các bội chung lớn hơn $5000$ nhưng nhỏ hơn $10000$ của 3 số $126,140,180$.
\end{baitoan}

\begin{baitoan}[\cite{Tuyen_Toan_6}, 145., p. 30]
	1 số tự nhiên chia cho $12,18,21$ đều dư $5$. Tìm số đó biết nó xấp xỉ $1000$.
\end{baitoan}

\begin{baitoan}[\cite{Tuyen_Toan_6}, 146., p. 30]
	Khối 6 của 1 trường có chưa tới $400$ học sinh. Khi xếp hàng $10,12,15$ đều dư $3$ nhưng nếu xếp hàng $11$ thì không dư. Tính số học sinh khối 6.
\end{baitoan}

\begin{baitoan}[\cite{Tuyen_Toan_6}, 147., p. 30]
	Tìm $a,b\in\mathbb{N}$ biết ${\rm BCNN}(a,b) = 300,\mbox{\rm ƯCLN}(a,b) = 15$.
\end{baitoan}

\begin{baitoan}[\cite{Tuyen_Toan_6}, 148., p. 30]
	Tìm $a,b\in\mathbb{N}$ biết tích của chúng là $2940$ \& {\rm BCNN} của chúng là $210$.
\end{baitoan}

\begin{baitoan}[\cite{Tuyen_Toan_6}, 149., p. 30]
	Tìm $a,b\in\mathbb{N}$ biết tổng của {\rm BCNN} với {\rm ƯCLN} của chúng là $15$.
\end{baitoan}

\begin{baitoan}[\cite{Tuyen_Toan_6}, 150., p. 30]
	Tìm $a\in\mathbb{N}$ nhỏ nhất có 3 chữ số sao cho chia cho $11$ thì dư $5$, chia cho $13$ thì dư $8$.
\end{baitoan}

\begin{baitoan}[\cite{Tuyen_Toan_6}, 151., p. 30]
	Tìm $x\in\mathbb{N}$ nhỏ nhất sao cho $x$ chia cho $3$ dư $2$, $x$ chia cho $5$ dư $3$, $x$ chia cho $7$ dư $4$.
\end{baitoan}

\begin{baitoan}[\cite{Tuyen_Toan_6}, 152., p. 30]
	Chứng minh nếu $a$ là 1 số lẻ không chia hết cho $3$ thì $a^2 - 1\divby6$.
\end{baitoan}

\begin{baitoan}[\cite{Tuyen_Toan_6}, 153., p. 30]
	Chứng minh tích của $5$ số tự nhiên liên tiếp chia hết cho $120$.
\end{baitoan}

\begin{baitoan}[\cite{TLCT_THCS_Toan_6_so_hoc}, VD5.1, p. 35]
	Tìm $n\in\mathbb{N}$ lớn nhất sao cho khi chia $364,414,539$ cho $n$, ta được 3 số dư bằng nhau.
\end{baitoan}

\begin{baitoan}[\cite{TLCT_THCS_Toan_6_so_hoc}, VD5.2, p. 36]
	Tìm $n\in\mathbb{N},n < 30$ để 2 số $3n + 4,5n + 1$ có ước chung khác $1$.
\end{baitoan}

\begin{baitoan}[\cite{TLCT_THCS_Toan_6_so_hoc}, VD5.3, p. 36]
	Tổng của 5 số tự nhiên bằng $156$. {\rm ƯCLN} của chúng có thể nhận {\rm GTLN} bằng bao nhiêu?
\end{baitoan}

\begin{baitoan}[\cite{TLCT_THCS_Toan_6_so_hoc}, VD5.4, p. 36]
	Có 3 đèn tín hiệu, chúng phát sáng cùng lúc vào {\rm8:00}. Đèn thứ nhất có $4$ phút phát sáng 1 lần. Thời gian đầu tiên để cả 3 đèn cùng phát sáng sau {\rm12:00} là lúc mấy giờ?
\end{baitoan}

\begin{baitoan}[\cite{TLCT_THCS_Toan_6_so_hoc}, VD5.5, p. 37]
	Điền các chữ số thích hợp vào dấu $\star$ để số $A = \overline{679\star\star\star}$ chia hết cho tất cả 4 số $5,6,7,9$.
\end{baitoan}

\begin{baitoan}[\cite{TLCT_THCS_Toan_6_so_hoc}, VD5.6, p. 37]
	Tìm $a,b\in\mathbb{N}$ thỏa $\mbox{\rm ƯCLN}(a,b) = 12,{\rm BCNN}(a,b) = 240$.
\end{baitoan}

\begin{baitoan}[\cite{TLCT_THCS_Toan_6_so_hoc}, 5.1., p. 38]
	Tìm ước chung của 3 số $1820,3080,4900$ trong khoảng từ $40$ đến $100$.
\end{baitoan}

\begin{baitoan}[\cite{TLCT_THCS_Toan_6_so_hoc}, 5.2., p. 38]
	Tìm {\rm ƯCLN} của $121212,181818$.
\end{baitoan}

\begin{baitoan}[\cite{TLCT_THCS_Toan_6_so_hoc}, 5.3., p. 38]
	Tìm {\rm ƯCLN} bằng cách dùng thuật toán Euclide: (a) $11111,1111$. (b) $342,266$.
\end{baitoan}

\begin{baitoan}[\cite{TLCT_THCS_Toan_6_so_hoc}, 5.4., p. 38]
	Tìm $a\in\mathbb{N}$ biết $296$ chia cho $a$ thì dư $16$, còn $230$ chia cho $a$ thì dư $10$.
\end{baitoan}

\begin{baitoan}[\cite{TLCT_THCS_Toan_6_so_hoc}, 5.5., p. 38]
	Chứng minh cặp số nguyên tố cùng nhau $\forall n\in\mathbb{N}$: (a) $2n + 1,6n + 5$. (b) $3n + 2,5n + 3$.
\end{baitoan}

\begin{baitoan}[\cite{TLCT_THCS_Toan_6_so_hoc}, 5.6., p. 38]
	Tìm $n\in\mathbb{N}$ để $3n + 1\divby7$.
\end{baitoan}

\begin{baitoan}[\cite{TLCT_THCS_Toan_6_so_hoc}, 5.7., p. 38]
	Tìm $n\in\mathbb{N}$ để $2n + 1,7n + 2$ nguyên tố cùng nhau.
\end{baitoan}

\begin{baitoan}[\cite{TLCT_THCS_Toan_6_so_hoc}, 5.8., p. 38]
	Tìm $a,b\in\mathbb{N}$ thỏa: (a) $a + b = 96,\mbox{\rm ƯCLN}(a,b) = 12$. (b) $a + b = 72,\mbox{\rm ƯCLN}(a,b) = 8$.
\end{baitoan}

\begin{baitoan}[\cite{TLCT_THCS_Toan_6_so_hoc}, 5.9., p. 38]
	Tìm $a,b\in\mathbb{N},a,b < 200$ thỏa: (a) $a - b = 96,\mbox{\rm ƯCLN}(a,b) = 16$. (b) $a - b = 90,\mbox{\rm ƯCLN}(a,b) = 15$.
\end{baitoan}

\begin{baitoan}[\cite{TLCT_THCS_Toan_6_so_hoc}, 5.10., p. 38]
	Tìm $a,b\in\mathbb{N}$ thỏa: (a) $ab = 448,\mbox{\rm ƯCLN}(a,b) = 4$. (b) $ab = 294,\mbox{\rm ƯCLN}(a,b) = 7$.
\end{baitoan}

\begin{baitoan}[\cite{TLCT_THCS_Toan_6_so_hoc}, 5.11., p. 38]
	Tìm $a,b\in\mathbb{N}$ thỏa $\mbox{\rm ƯCLN}(a,b) = 10,{\rm BCNN}(a,b) = 120$.
\end{baitoan}

\begin{baitoan}[\cite{TLCT_THCS_Toan_6_so_hoc}, 5.12., p. 38]
	Tìm $n\in\mathbb{N}$ biết trong 3 số $6,16,n$, bất cứ số nào cũng là ước của tích 2 số kia.
\end{baitoan}

\begin{baitoan}[\cite{TLCT_THCS_Toan_6_so_hoc}, 5.13., p. 38]
	Tuấn \& Tú ở cùng 1 nhà \& làm việc tại 2 công ty khác nhau. Tuấn cứ $10$ ngày lại đi trực 1 lần, Tú cứ $15$ ngày lại đi trực 1 lần. 2 người cùng trực vào ngày Chủ nhật 1.1.2012. Hỏi trong năm 2012, 2 người cùng trực vào 1 ngày Chủ nhật mấy lần?
\end{baitoan}

\begin{baitoan}[\cite{TLCT_THCS_Toan_6_so_hoc}, 5.14., p. 39]
	Có 2 chiếc đồng hồ. Trong 1 ngày, chiếc thứ nhất chạy nhanh $10$ phút, chiếc thứ 2 chạy chậm $6$ phút. Cả 2 đồng hồ được lấy lại theo giờ chính xác. Sau ít nhất bao lâu, cả 2 đồng hồ lại cùng chỉ giờ chính xác?
\end{baitoan}

\begin{baitoan}[\cite{TLCT_THCS_Toan_6_so_hoc}, 5.15., p. 39]
	Tìm số tự nhiên lớn nhất có 3 chữ số, biết chia nó cho $10$ thì dư $3$, chia nó cho $12$ thì dư $5$, chia nó cho $15$ thì dư $8$ \& nó chia hết cho $19$.
\end{baitoan}

\begin{baitoan}[\cite{TLCT_THCS_Toan_6_so_hoc}, 5.16., p. 39]
	Tìm $a,b,c\in\mathbb{N}^\star$ nhỏ nhất sao cho $16a = 25b = 30c$.
\end{baitoan}

\begin{baitoan}[\cite{TLCT_THCS_Toan_6_so_hoc}, 5.17., p. 39]
	Tìm số tự nhiên nhỏ nhất để khi chia cho $5,8,12$ thì số dư lần lượt là $2,6,8$.
\end{baitoan}

\begin{baitoan}[\cite{TLCT_THCS_Toan_6_so_hoc}, 5.18., p. 39]
	Điền chữ số thích hợp vào dấu $\star$ để $\overline{456\star\star}$ chia hết cho tất cả 3 số $4,5,6$.
\end{baitoan}

\begin{baitoan}[\cite{TLCT_THCS_Toan_6_so_hoc}, 5.19., p. 39]
	Chứng minh tích của 4 số tự nhiên liên tiếp thì chia hết cho $24$.
\end{baitoan}

\begin{baitoan}[\cite{TLCT_THCS_Toan_6_so_hoc}, 5.20., p. 39]
	Tìm $a,b\in\mathbb{N}$ biết: (a) ${\rm BCNN}(a,b) + \mbox{\rm ƯCLN}(a,b) = 19$. (b) ${\rm BCNN}(a,b) - \mbox{\rm ƯCLN}(a,b) = 3$.
\end{baitoan}

%------------------------------------------------------------------------------%

\section{Nguyên Lý Dirichlet \& Bài Toán Chia Hết}

\begin{baitoan}[\cite{Tuyen_Toan_6}, VD33, p. 31]
	Cho $7$ số tự nhiên bất kỳ. Chứng minh bao giờ cũng có thể chọn ra 2 số mà hiệu của chúng chia hết cho $6$.
\end{baitoan}

\begin{baitoan}[\cite{Tuyen_Toan_6}, VD34, p. 32]
	Cho 3 số lẻ. Chứng minh tồn tại 2 số có tổng hay hiệu chia hết cho $8$.
\end{baitoan}

\begin{baitoan}[\cite{Tuyen_Toan_6}, VD35, p. 32]
	Chứng minh có 1 số tự nhiên gồm toàn chữ số $3$ chia hết cho $41$.
\end{baitoan}

\begin{baitoan}[\cite{Tuyen_Toan_6}, 154., p. 33]
	Chứng minh trong $11$ số tự nhiên bất kỳ bao giờ cũng có ít nhất $2$ số có chữ số tận cùng giống nhau.
\end{baitoan}

\begin{baitoan}[\cite{Tuyen_Toan_6}, 155., p. 33]
	Chứng minh tồn tại 1 bội của $13$ gồm toàn chữ số $2$.
\end{baitoan}

\begin{baitoan}[\cite{Tuyen_Toan_6}, 156., p. 33]
	Chứng minh có thể tìm được 1 số có dạng $987987\cdots987$ chia hết cho $2021$.
\end{baitoan}

\begin{baitoan}[\cite{Tuyen_Toan_6}, 157., p. 33]
	Cho dãy số $10,10^2,10^3,\ldots,10^{20}$. Chứng minh tồn tại 1 số chia cho $19$ dư $1$.
\end{baitoan}

\begin{baitoan}[\cite{Tuyen_Toan_6}, 158., p. 33]
	Chứng minh tồn tại 1 bội số là bội của $19$ có tổng các chữ số bằng $19$.
\end{baitoan}

\begin{baitoan}[\cite{Tuyen_Toan_6}, 159., p. 33]
	Cho 3 số nguyên tố lớn hơn $3$. Chứng minh tồn tại 2 số có tổng hoặc hiệu chia hết cho $12$.
\end{baitoan}

\begin{baitoan}[\cite{Tuyen_Toan_6}, 160., p. 33]
	Chứng minh trong 3 số tự nhiên bất kỳ luôn chọn được 2 số có tổng chia hết cho $2$.
\end{baitoan}

\begin{baitoan}[\cite{Tuyen_Toan_6}, 161., p. 33]
	Chứng minh trong 7 số tự nhiên bất kỳ ta luôn chọn được 4 số có tổng chia hết cho $4$.
\end{baitoan}

\begin{baitoan}[\cite{Tuyen_Toan_6}, 162., p. 33]
	Cho 5 số tự nhiên bất kỳ, chứng minh luôn chọn được 3 số có tổng chia hết cho $3$.
\end{baitoan}

\begin{baitoan}[\cite{Tuyen_Toan_6}, 163., p. 33]
	Cho 5 số tự nhiên lẻ bất kỳ, chứng minh luôn chọn được 4 số có tổng chia hết cho $4$.
\end{baitoan}

\begin{baitoan}[\cite{Tuyen_Toan_6}, 164., p. 33]
	Viết 6 số tự nhiên vào 6 mặt của 1 con xúc xắc. Chứng minh khi ta gieo mặt xúc xắc xuống mặt bàn thì trong 5 mặt có thể nhìn thấy bao giờ cũng tìm được 1 hay nhiều mặt để tổng các số trên mặt đó chia hết cho $5$.
\end{baitoan}

%------------------------------------------------------------------------------%

\section{Miscellaneous}

\begin{baitoan}[\cite{Tuyen_Toan_6}, VD36, p. 33]
	Chứng minh tích các ước của $50$ là $50^3$.
\end{baitoan}

\begin{baitoan}[\cite{Tuyen_Toan_6}, VD3, p. 33]
	1 bệnh nhân được bác sĩ chỉ định nhỏ thuốc đau mắt, $4$ giờ 1 lần \& nhỏ tai $6$ giờ 1 lần. Nếu bệnh nhân này vừa nhỏ thuốc đau mắt vừa nhỏ tai vào lúc {\rm6:00} hằng ngày thì ít nhất đến mấy giờ bệnh nhân này lại vừa nhỏ mắt vừa nhỏ tai?
\end{baitoan}

\begin{baitoan}[\cite{Tuyen_Toan_6}, 165., p. 34]
	Tính hợp lý: (a) $\underbrace{19 + 19 + \cdots + 19}_{23} + \underbrace{77 + 77 + \cdots + 77}_{19}$. (b) $1000!(456\cdot789789 - 789\cdot456456)$.
\end{baitoan}

\begin{baitoan}[\cite{Tuyen_Toan_6}, 166., p. 34]
	Tìm $x\in\mathbb{N}$ thỏa: (a) $x + (x + 1) + (x + 2) + \cdots + (x + 30) = 1240$. (b) $1 + 2 + \cdots + x = 210$.
\end{baitoan}

\begin{baitoan}[\cite{Tuyen_Toan_6}, 167., p. 34]
	Chứng minh: (a) $10^n + 5^3\divby9$. (b) $43^{43} - 17^{17}\divby10$. (c) $A = \underbrace{5\ldots5}_{2n}$, $A\divby11$ nhưng $A\not{\divby}\ 125$.
\end{baitoan}

\begin{baitoan}[\cite{Tuyen_Toan_6}, 168., p. 34]
	Tìm $a\in\mathbb{N}$ nhỏ nhất sao cho $a$ chia cho $17$ dư $5$, $a$ chia cho $19$ dư $12$.
\end{baitoan}

\begin{baitoan}[\cite{Tuyen_Toan_6}, 169., p. 34]
	Tìm $a\in\mathbb{N}$ biết $355$ chia cho $a$ dư $13$ \& số $836$ chia cho $a$ dư $8$.
\end{baitoan}

\begin{baitoan}[\cite{Tuyen_Toan_6}, 170., p. 34]
	$a\in\mathbb{N}$ chia cho $7$ dư $5$, chia cho $13$ dư $4$. Tìm số dư khi $a$ chia cho $91$.
\end{baitoan}

\begin{baitoan}[\cite{Tuyen_Toan_6}, 171., p. 34]
	Biết ngày 1.2.2023 là thứ 4. (a) Hỏi ngày 1.3.2023, 1.4.2023 là thứ mấy? (b) Ngày 1.2.2024 là thứ mấy?
\end{baitoan}

\begin{baitoan}[\cite{Tuyen_Toan_6}, 172., p. 34]
	Tính tổng: (a) $A = \sum_{i=0}^{20} 2^i = 1 + 2 + 2^2 + \cdots + 2^{20}$. (b) $B = 4 + 4^3 + 4^5 + \cdots + 4^{49}$.
\end{baitoan}

\begin{baitoan}[\cite{Tuyen_Toan_6}, 173., p. 34]
	Cho $A = \sum_{i=1}^{24} 4^i = 4 + 4^2 + \cdots + 4^{24}$. Chứng minh $A\divby20,A\divby21,A\divby420$.
\end{baitoan}

\begin{baitoan}[\cite{Tuyen_Toan_6}, 174., p. 34]
	Cho $n = 29k$ với $k\in\mathbb{N}$. Với giá trị nào của $k$ thì $n$ là: (a) Số nguyên tố. (b) Hợp số. (c) Không phải là số nguyên tố cũng không phải là hợp số?
\end{baitoan}

\begin{baitoan}[\cite{Tuyen_Toan_6}, 175., p. 34]
	Cho $a$ là 1 hợp số, khi phân tích ra thừa số nguyên tố chỉ chứa 2 thừa số nguyên tố khác nhau là $p_1,p_2$. Biết $a^3$ có tất cả $40$ ước, tính số ước của $a^2$.
\end{baitoan}

\begin{baitoan}[\cite{Tuyen_Toan_6}, 176., p. 34]
	Cho 3 số $12,18,27$. (a) Tìm số lớn nhất có 3 chữ số chia hết cho 3 số đó. (b) Tìm số nhỏ nhất có 4 chữ số chia cho mỗi số đó đều dư $1$. (c) Tìm số nhỏ nhất có 4 chữ số chia cho $12$ dư $10$, chia cho $18$ dư $16$, chia cho $27$ dư $25$.
\end{baitoan}

%------------------------------------------------------------------------------%

\printbibliography[heading=bibintoc]

\end{document}