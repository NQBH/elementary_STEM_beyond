\documentclass{article}
\usepackage[backend=biber,natbib=true,style=alphabetic,maxbibnames=50]{biblatex}
\addbibresource{/home/nqbh/reference/bib.bib}
\usepackage[utf8]{vietnam}
\usepackage{tocloft}
\renewcommand{\cftsecleader}{\cftdotfill{\cftdotsep}}
\usepackage[colorlinks=true,linkcolor=blue,urlcolor=red,citecolor=magenta]{hyperref}
\usepackage{amsmath,amssymb,amsthm,float,graphicx,mathtools,tipa}
\usepackage{enumitem}
\setlist{leftmargin=4mm}
\allowdisplaybreaks
\newtheorem{assumption}{Assumption}
\newtheorem{baitoan}{}
\newtheorem{cauhoi}{Câu hỏi}
\newtheorem{conjecture}{Conjecture}
\newtheorem{corollary}{Corollary}
\newtheorem{dangtoan}{Dạng toán}
\newtheorem{definition}{Definition}
\newtheorem{dinhly}{Định lý}
\newtheorem{dinhnghia}{Định nghĩa}
\newtheorem{example}{Example}
\newtheorem{ghichu}{Ghi chú}
\newtheorem{hequa}{Hệ quả}
\newtheorem{hypothesis}{Hypothesis}
\newtheorem{lemma}{Lemma}
\newtheorem{luuy}{Lưu ý}
\newtheorem{nhanxet}{Nhận xét}
\newtheorem{notation}{Notation}
\newtheorem{note}{Note}
\newtheorem{principle}{Principle}
\newtheorem{problem}{Problem}
\newtheorem{proposition}{Proposition}
\newtheorem{question}{Question}
\newtheorem{remark}{Remark}
\newtheorem{theorem}{Theorem}
\newtheorem{vidu}{Ví dụ}
\usepackage[left=1cm,right=1cm,top=5mm,bottom=5mm,footskip=4mm]{geometry}
\def\labelitemii{$\circ$}
\DeclareRobustCommand{\divby}{%
	\mathrel{\vbox{\baselineskip.65ex\lineskiplimit0pt\hbox{.}\hbox{.}\hbox{.}}}%
}

\title{Problem: Prime, Composite, GCD, {\it\&} LCM\\Bài Tập: Số Nguyên Tố, Hợp Số, ƯCLN, {\it\&} BCNN}
\author{Nguyễn Quản Bá Hồng\footnote{Independent Researcher, Ben Tre City, Vietnam\\e-mail: \texttt{nguyenquanbahong@gmail.com}; website: \url{https://nqbh.github.io}.}}
\date{\today}

\begin{document}
\maketitle
\begin{abstract}
	Last updated version: \href{https://github.com/NQBH/elementary_STEM_beyond/blob/main/elementary_mathematics/grade_6/natural/divisibility/problem/NQBH_divisibility_problem.pdf}{GitHub{\tt/}NQBH{\tt/}hobby{\tt/}elementary mathematics{\tt/}grade 6{\tt/}natural{\tt/}divisibility{\tt/}problem[pdf]}.\footnote{\textsc{url}: \url{https://github.com/NQBH/elementary_STEM_beyond/blob/main/elementary_mathematics/grade_6/natural/divisibility/problem/NQBH_divisibility_problem.pdf}.} [\href{https://github.com/NQBH/elementary_STEM_beyond/blob/main/elementary_mathematics/grade_6/natural/divisibility/problem/NQBH_divisibility_problem.tex}{\TeX}]\footnote{\textsc{url}: \url{https://github.com/NQBH/elementary_STEM_beyond/blob/main/elementary_mathematics/grade_6/natural/divisibility/problem/NQBH_divisibility_problem.tex}.}. 
\end{abstract}
\tableofcontents

%------------------------------------------------------------------------------%

\section{Prime. Composite -- Số Nguyên Tố. Hợp Số}

\begin{baitoan}[\cite{Binh_boi_duong_Toan_6_tap_1}, H1, p. 36]
	Egg có $54$ viên bi \& muốn chia đều số bi đó vào các hộp. Tìm tất cả các cách chia thỏa mãn.
\end{baitoan}

\begin{baitoan}[\cite{Binh_boi_duong_Toan_6_tap_1}, H2, p. 36]
	(a) Số nào có phân tích ra thừa số nguyên tố là $2^3\cdot3^2\cdot7$. (b) Phân tích $2160$ ra thừa số nguyên tố.
\end{baitoan}

\begin{baitoan}[\cite{Binh_boi_duong_Toan_6_tap_1}, H3, p. 36]
	Tìm chữ số $a$ để $\overline{17a}$ là số nguyên tố.
\end{baitoan}

\begin{baitoan}[\cite{Binh_boi_duong_Toan_6_tap_1}, H4, p. 36]
	{\rm Đ{\tt/}S?} Ký hiệu $P$ là tập hợp các số nguyên tố. (a) $19\in P$. (b) $\{3,5,7\}\in P$. (c) $\{71,73\}\in P$. (d) $6\cdot7\cdot8\cdot9 - 5\cdot7\cdot11\in P$. (e) Mọi số nguyên tố đều có tận cùng là số lẻ.
\end{baitoan}

\begin{baitoan}[\cite{Binh_boi_duong_Toan_6_tap_1}, Ví dụ 1, p. 37]
	Cho 1 phép chia có số bị chia bằng $236$ \& số dư bằng $15$. Tìm số chia \& thương.
\end{baitoan}

\begin{baitoan}[\cite{Binh_boi_duong_Toan_6_tap_1}, Ví dụ 2, p. 37]
	Có bao nhiêu số là bội của $6$ trong khoảng từ $72$ đến $2016$?
\end{baitoan}

\begin{baitoan}[\cite{Binh_boi_duong_Toan_6_tap_1}, Ví dụ 3, p. 37]
	Tìm $x\in\mathbb{N}$ sao cho $42\divby(2x + 5)$.
\end{baitoan}

\begin{baitoan}[\cite{Binh_boi_duong_Toan_6_tap_1}, Ví dụ 4, p. 38]
	Tìm số nguyên tố $p$ sao cho $p + 2$ \& $p + 4$ cũng là 2 số nguyên tố.
\end{baitoan}

\begin{baitoan}[\cite{Binh_boi_duong_Toan_6_tap_1}, Ví dụ 5, p. 38]
	Cho $p > 3$ \& $2p + 1$ là 2 số nguyên tố. Hỏi $4p + 1$ là số nguyên tố hay hợp số.
\end{baitoan}

\begin{baitoan}[\cite{Binh_boi_duong_Toan_6_tap_1}, Ví dụ 6, p. 39]
	Tìm số nguyên tố bằng tổng của 2 số nguyên tố \& cũng bằng hiệu của 2 số nguyên tố khác.
\end{baitoan}

\begin{luuy}
	$2$ là số nguyên tố chẵn duy nhất.
\end{luuy}

\begin{baitoan}[\cite{Binh_boi_duong_Toan_6_tap_1}, Ví dụ 7, p. 39]
	Phân tích ra thừa số nguyên tố: (a) $2016^7$. (b) $30\cdot4\cdot1975$.
\end{baitoan}

\begin{baitoan}[\cite{Binh_boi_duong_Toan_6_tap_1}, Ví dụ 8, p. 39]
	Tìm $n\in\mathbb{N}^\star$ thỏa $2 + 4 + 6 + \cdots + 2n = 870$.
\end{baitoan}

\begin{baitoan}[\cite{Binh_boi_duong_Toan_6_tap_1}, Ví dụ 9, p. 40]
	Tìm $n\in\mathbb{N}^\star$ sao cho $p = (n - 2)(n^2 + n - 5)$ là số nguyên tố.
\end{baitoan}

\begin{baitoan}[\cite{Binh_boi_duong_Toan_6_tap_1}, 5.1., p. 40]
	Tìm tập hợp các số tự nhiên vừa là bội của $9$, vừa là ước của $72$.
\end{baitoan}

\begin{baitoan}[\cite{Binh_boi_duong_Toan_6_tap_1}, 5.2., p. 40]
	Tìm $x\in\mathbb{N}^\star$ thỏa: (a) $x - 1$ là ước của $24$. (b) $36$ là bội của $2x + 1$.
\end{baitoan}

\begin{baitoan}[\cite{Binh_boi_duong_Toan_6_tap_1}, 5.3., p. 40]
	Tìm $x,y\in\mathbb{N}^\star$ thỏa $(2x + 1)(y - 3) = 15$.
\end{baitoan}

\begin{baitoan}[\cite{Binh_boi_duong_Toan_6_tap_1}, 5.4., p. 40]
	Phân tích ra thừa số nguyên tố: (a) $1\cdot12\cdot78$. (b) $1930^8$.
\end{baitoan}

\begin{baitoan}[\cite{Binh_boi_duong_Toan_6_tap_1}, 5.5., p. 40]
	Chứng minh nếu $p$ là 1 số nguyên tố lớn hơn $3$ thì $(p - 1)(p + 1)$ chia hết cho $3$ \& cho $8$.
\end{baitoan}

\begin{baitoan}[\cite{Binh_boi_duong_Toan_6_tap_1}, 5.6., p. 40]
	Tìm chữ số $a$ để $\overline{23a}$ là số nguyên tố.
\end{baitoan}

\begin{baitoan}[\cite{Binh_boi_duong_Toan_6_tap_1}, 5.7., p. 40]
	Tìm số tự nhiên nhỏ nhất có đúng $18$ ước số.
\end{baitoan}

\begin{baitoan}[\cite{Binh_boi_duong_Toan_6_tap_1}, 5.8., p. 40]
	Chứng minh: Nếu 1 số tự nhiên có 3 chữ số tận cùng là $104$ thì số đó có ít nhất $4$ ước số.
\end{baitoan}

\begin{baitoan}[\cite{Binh_boi_duong_Toan_6_tap_1}, 5.9., p. 40]
	Tìm 2 số nguyên tố có tổng bằng $309$.
\end{baitoan}

\begin{baitoan}[\cite{Binh_boi_duong_Toan_6_tap_1}, 5.10., p. 40]
	Tìm số nguyên tố $p$ sao cho $p + 4,p + 8$ cũng là 2 số nguyên tố.
\end{baitoan}

\begin{baitoan}[\cite{Binh_boi_duong_Toan_6_tap_1}, 5.11., p. 40]
	Tìm số nguyên tố $p$ sao cho $p + 6,p + 8,p + 12,p + 14$ cũng là 4 số nguyên tố.
\end{baitoan}

\begin{baitoan}[\cite{Binh_boi_duong_Toan_6_tap_1}, 5.12., p. 40]
	Cho $ptố > 3$ \& $p + 4$ là 2 số nguyên tố. Chứng minh $p + 8$ là hợp số.
\end{baitoan}

\begin{baitoan}[\cite{Binh_boi_duong_Toan_6_tap_1}, 5.13., p. 40]
	Số $3^2 + 3^4 + 3^6 + \cdots + 3^{2012}$ là số nguyên tố hay hợp số?
\end{baitoan}

\begin{baitoan}[\cite{Binh_boi_duong_Toan_6_tap_1}, 5.14., p. 40]
	2 số nguyên tố được gọi là {\rm sinh đôi} nếu chúng là 2 số nguyên tố \& là 2 số lẻ liên tiếp, e.g., $3$ \& $5$, $11$ \& $13,\ldots$. Chứng minh số tự nhiên lớn hơn $4$ \& nằm giữa 2 số nguyên tố sinh đôi thì chia hết cho $6$.
\end{baitoan}

\begin{baitoan}[\cite{Binh_boi_duong_Toan_6_tap_1}, 5.15., p. 41]
	Tìm 3 số tự nhiên lẻ liên tiếp đều là số nguyên tố.
\end{baitoan}

\begin{baitoan}[\cite{Binh_boi_duong_Toan_6_tap_1}, 5.16., p. 41]
	Tìm $n\in\mathbb{N}^\star$ thỏa $1 + 3 + 5 + \cdots + (2n + 1) = 169$.
\end{baitoan}

\begin{baitoan}[\cite{Binh_boi_duong_Toan_6_tap_1}, 5.17., p. 41]
	Biết số $\overline{abc}$ khi phân tích ra t hừa số nguyên tố có thừa số $3$ \& thừa số $7$. Chứng minh số $a + 19b + 4c$ cũng có tính chất đó.
\end{baitoan}

\begin{baitoan}[\cite{Binh_boi_duong_Toan_6_tap_1}, 5.18., p. 41]
	Tìm chữ số $a$ sao cho số $\overline{aaa}$ là tổng của các số tự nhiên liên tiếp từ $1$ đến số $n$ nào đó.
\end{baitoan}

\begin{baitoan}
	Chứng minh tập hợp các số nguyên tố có vô hạn phần tử \& không có số nguyên tố lớn nhất.
\end{baitoan}
\noindent\textit{Hint.} Giả sử phản chứng: chỉ có hữu hạn số nguyên tố $p_1 < p_2 < \cdots < p_n$. Chứng minh $p\coloneqq\prod_{i=1}^n p_i + 1 = p_1p_2\cdots p_n + 1$ là 1 số nguyên tố lớn hơn mỗi số nguyên tố $p_i$, $\forall i\in\mathbb{N}$.

\begin{baitoan}[\cite{Tuyen_Toan_6}, VD26, p. 25]
	Tìm số nguyên tố $a$ để $4a + 11$ là số nguyên tố nhỏ hơn $30$.
\end{baitoan}

\begin{baitoan}[\cite{Tuyen_Toan_6}, VD27, p. 25]
	Cho $A = \sum_{i=1}^{100} 5^i = 5 + 5^2 + \cdots + 5^{100}$. (a) Hỏi A là số nguyên tố hay hợp số? (b) Số A có phải là số chính phương không?
\end{baitoan}

\begin{baitoan}[\cite{Tuyen_Toan_6}, VD28, p. 2]
	Tính cạnh của 1 hình vuông có diện tích $\rm5929\ m^2$.
\end{baitoan}

\begin{baitoan}[\cite{Tuyen_Toan_6}, 116., p. 26]
	Phân loại số nguyên tố, hợp số: (a) $A = 1\cdot3\cdot5\cdot7\cdots13 + 20$. (b) $B = 147\cdot247\cdot347 - 13$.
\end{baitoan}

\begin{baitoan}[\cite{Tuyen_Toan_6}, 117., p. 26]
	Tìm số bị chia \& thương trong phép chia: $9\star\star:17 = \star\star$. Biết thương là 1 số nguyên tố.
\end{baitoan}

\begin{baitoan}[\cite{Tuyen_Toan_6}, 118., p. 26]
	Cho $a,n\in\mathbb{N}^\star$. Biết $a^n\divby5$. Chứng minh $a^2 + 150\divby25$.
\end{baitoan}

\begin{baitoan}[\cite{Tuyen_Toan_6}, 119., p. 26]
	(a) Cho $n\in\mathbb{N}$, $n\not{\divby}\ 3$. Chứng minh $n^2$ chia cho $3$ dư $1$. (b) Cho $p$ là 1 số nguyên tố lớn hơn $3$. Hỏi $p^2 + 2021$ là số nguyên tố hay hợp số?
\end{baitoan}

\begin{baitoan}[\cite{Tuyen_Toan_6}, 120., p. 26]
	Cho $n\in\mathbb{N}$, $n > 2$, $n\not{\divby}\ 3$. Chứng minh 2 số $n^2\pm1$ không thể đồng thời là 2 số nguyên tố.
\end{baitoan}

\begin{baitoan}[\cite{Tuyen_Toan_6}, 121., p. 26]
	Cho $p > 3,p + 8$ đều là số nguyên tố. Hỏi $p + 100$ là số nguyên tố hay hợp số?
\end{baitoan}

\begin{baitoan}[\cite{Tuyen_Toan_6}, 122., p. 26]
	Phân tích ra thừa số nguyên tố bằng cách hợp lý nhất: (a) $700,9000,210000$. (b) $500,1600,18000$.
\end{baitoan}

\begin{baitoan}[\cite{Tuyen_Toan_6}, 123., p. 26]
	Đếm số ước số của: $90,540,3675$.
\end{baitoan}

\begin{baitoan}[\cite{Tuyen_Toan_6}, 124., p. 26]
	Tìm: (a) 2 số tự nhiên liên tiếp có tích bằng $1260$. (b) 3 số tự nhiên liên tiếp có tích bằng $3360$.
\end{baitoan}

\begin{baitoan}[\cite{Tuyen_Toan_6}, 125., p. 26]
	Tìm: (a) 3 số chẵn liên tiếp có tích bằng $5760$. (b) 3 số lẻ liên tiếp có tích bằng $19575$.
\end{baitoan}

\begin{baitoan}[\cite{Tuyen_Toan_6}, 126., p. 26]
	Tính cạnh của 1 hình lập phương biết thể tích của nó là $\rm1728\ cm^3$.
\end{baitoan}

\begin{baitoan}[\cite{Tuyen_Toan_6}, 127., p. 27]
	Chứng minh 1 số tự nhiên $\ne0$ có số lượng các ước là 1 số lẻ $\Leftrightarrow$ số tự nhiên đó là số chính phương.
\end{baitoan}

\begin{baitoan}[\cite{Tuyen_Toan_6}, 128., p. 28]
	Tìm $n\in\mathbb{N}^\star$ thỏa: (a) $2 + 4 + 6 + \cdots + 2n = 210$. (b) $1 + 3 + 5 + \cdots + (2n - 1) = 225$.
\end{baitoan}

%------------------------------------------------------------------------------%

\section{Greatest Common Divisor. Least Common Multiple -- Ước Chung Lớn Nhất. Bội Chung Nhỏ Nhất}

\begin{baitoan}[\cite{Binh_boi_duong_Toan_6_tap_1}, H1, p. 43]
	1 thửa ruộng hình chữ nhật có chiều dài {\rm72 m}, chiều rộng {\rm40 m}. Chicken muốn chia thửa ruộng thành các mảnh đất hình vuông bằng nhau để trồng các loại ngũ cốc. Tính độ dài lớn nhất của hình vuông mà Chicken có thể chia.
\end{baitoan}

\begin{baitoan}[\cite{Binh_boi_duong_Toan_6_tap_1}, H2, p. 43]
	Có 4 thuyền A, B, C, D. Thuyền A cứ $5$ ngày cập bến 1 lần, thuyền B cứ $6$ ngày cập bến  1 lần, thuyền C cứ $8$ ngày cập bến 1 lần \& thuyền D cứ $10$ ngày cập bến 1 lần. Egg nhẩm tính: Nếu ngày hôm nay cả 4 thuyền cùng cập bến thì: (a) Sau ít nhất $a$ ngày nữa, thuyền A cùng cập bến với thuyền D. (b) Sau ít nhất $b$ ngày nữa, thuyền B cùng cập bến với thuyền C. (c) Sau ít nhất $c$ ngày nữa, thuyền B cùng cập bến với thuyền D. (d) Sau ít nhất $d$ ngày nữa, cả 4 thuyền sẽ cùng cập bến lần thứ 2. Tìm $a,b,c,d$.
\end{baitoan}

\begin{baitoan}[\cite{Binh_boi_duong_Toan_6_tap_1}, Ví dụ 1, p. 43]
	Tìm $\mbox{\rm ƯC}(48,60)$, ${\rm BC}(4,14)$.
\end{baitoan}

\begin{baitoan}[\cite{Binh_boi_duong_Toan_6_tap_1}, Ví dụ 2, p. 44]
	Tìm $a\in\mathbb{N}$ biết chia $264$ cho $a$ thì dư $24$, còn khi chia $363$ cho $a$ thì được dư là $43$.
\end{baitoan}

\begin{baitoan}[\cite{Binh_boi_duong_Toan_6_tap_1}, Ví dụ 3, p. 44]
	Tìm số tự nhiên nhỏ nhất có 4 chữ số, biết khi chia số đó cho $18,24,30$ thì có số dư lần lượt là $13,19,25$.
\end{baitoan}

\begin{baitoan}[\cite{Binh_boi_duong_Toan_6_tap_1}, Ví dụ 4, p. 44]
	Tìm $a,b\in\mathbb{N}$ thỏa $a + b = 336$ \& $\mbox{\rm ƯCLN}(a,b) = 24$.
\end{baitoan}

\begin{baitoan}[\cite{Binh_boi_duong_Toan_6_tap_1}, Ví dụ 5, p. 45]
	Tìm $a,b\in\mathbb{N}$ thỏa $\mbox{\rm ƯCLN}(a,b) = 24$ \& ${\rm BCNN}(a,b) = 36$.
\end{baitoan}

\begin{baitoan}[\cite{Binh_boi_duong_Toan_6_tap_1}, Ví dụ 6, p. 45]
	Cho $n\in\mathbb{N}^\star$. Chứng minh: $\mbox{\rm ƯCLN}(2n + 5,3n + 7) = 1$.
\end{baitoan}

\begin{baitoan}[\cite{Binh_boi_duong_Toan_6_tap_1}, Ví dụ 7, p. 46]
	Học sinh khối 6 của 1 trường khi xếp hàng $12$, hàng $15$ hay hàng $18$ thì đều vừa đủ hàng. Tính số học sinh khối 6 của trường đó, biết số học sinh này nằm trong khoảng từ $500$ đến $600$ học sinh.
\end{baitoan}

\begin{baitoan}[\cite{Binh_boi_duong_Toan_6_tap_1}, Ví dụ 8, p. 46]
	1 lớp học có $28$ học sinh nam \& $24$ học sinh nữ. Khi tham gia lao động, {\rm GVCN} muốn chia lớp thành các nhóm sao cho số học sinh nam \& số học sinh nữ được chia đều vào các nhóm. Hỏi {\rm GVCN} có bao nhiêu cách chia nhóm? Cách chia nào có số học sinh trong mỗi nhóm ít nhất?
\end{baitoan}

\begin{baitoan}[\cite{Binh_boi_duong_Toan_6_tap_1}, 6.1., p. 47]
	Tìm $\mbox{\rm ƯC}(54,120,180)$, ${\rm BC}(21,84)$.
\end{baitoan}

\begin{baitoan}[\cite{Binh_boi_duong_Toan_6_tap_1}, 6.2., p. 47]
	1 số chia cho $21$ dư $2$ \& chia cho $12$ dư $5$. Hỏi số đó chia cho $84$ thì dư bao nhiêu?
\end{baitoan}

\begin{baitoan}[\cite{Binh_boi_duong_Toan_6_tap_1}, 6.3., p. 47]
	Tìm $a\in\mathbb{N}$ thỏa mãn: $a\divby7$ \& $a$ chia cho $4$ hoặc $6$ đều dư $3$, biết $a < 350$.
\end{baitoan}

\begin{baitoan}[\cite{Binh_boi_duong_Toan_6_tap_1}, 6.4., p. 47]
	Tìm số tự nhiên lớn nhất có 3 chữ số sao cho chia nó cho $3$, cho $4$, cho $5$ ta được 3 số dư theo thứ tự là $2,3,4$.
\end{baitoan}

\begin{baitoan}[\cite{Binh_boi_duong_Toan_6_tap_1}, 6.5., p. 47]
	Cho $\mbox{\rm ƯCLN}(a,b) = 1$. Chứng minh: (a) $\mbox{\rm ƯCLN}(a,a - b) = 1$ với $a > b$. (b) $\mbox{\rm ƯCLN}(ab,a + b) = 1$.
\end{baitoan}

\begin{baitoan}[\cite{Binh_boi_duong_Toan_6_tap_1}, 6.6., p. 47]
	Cho $n\in\mathbb{N}$. Chứng minh: (a) $\mbox{\rm ƯCLN}(3n + 13,3n + 14) = 1$. (b) $\mbox{\rm ƯCLN}(3n +5,6n + 9) = 1$.
\end{baitoan}

\begin{baitoan}[\cite{Binh_boi_duong_Toan_6_tap_1}, 6.7., p. 47]
	1 lớp học có $27$ học sinh nam \& $18$ học sinh nữ. Có bao nhiêu cách chia lớp đó thành các tổ sao cho số học sinh nam \& số học sinh nữ được chia đều vào các tổ? Biết số tổ lớn hơn $1$.
\end{baitoan}

\begin{baitoan}[\cite{Binh_boi_duong_Toan_6_tap_1}, 6.8., p. 47]
	1 đơn vị bộ đội khi xếp hàng, mỗi hàng có $20$ người, hoặc $25$ người, hoặc $30$ người đều thừa $15$ người. Nếu xếp mỗi hàng $41$ người thì vừa đủ (không có hàng nào thiếu, không có ai ở ngoài hàng). Hỏi đơn vị có bao nhiêu người, biết số người của đơn vị chưa đến $1000$?
\end{baitoan}

\begin{baitoan}[\cite{Binh_boi_duong_Toan_6_tap_1}, 6.9., p. 47]
	Tổng số học sinh khối 6 của 1 trường có khoảng từ $235$ đến $250$ em, khi chia cho $3$ thì dư $2$, chia cho $4$ thì dư $3$, chia cho $5$ thì dư $4$, chia cho $6$ thì dư $5$, chia cho $10$ thì dư $9$. Tìm số học sinh của khối 6.
\end{baitoan}

\begin{baitoan}[\cite{Binh_boi_duong_Toan_6_tap_1}, 6.10., p. 47]
	1 trường tổ chức cho học sinh đi tham quan bằng ôtô. Nếu xếp $27$ hay $36$ học sinh lên 1 ôtô thì đều thấy thừa ra $11$ học sinh. Tính số học sinh đi tham quan, biết số học sinh đó có khoảng từ $400$ đến $450$ em.
\end{baitoan}

\begin{baitoan}[\cite{Binh_boi_duong_Toan_6_tap_1}, 6.11., p. 47]
	Cho 2 số nguyên tố cùng nhau $a,b$. Chứng minh 2 số $13a + 4b$ \& $15a + 7b$ hoặc nguyên tố cùng nhau hoặc có 1 ước chung là $31$.
\end{baitoan}

\begin{baitoan}[\cite{Binh_boi_duong_Toan_6_tap_1}, 6.12., p. 47]
	Cho $a,b\in\mathbb{N}$ không nguyên tố cùng nhau thỏa $a = 2n + 3$, $b = 3n + 1$ với $n\in\mathbb{N}$. Tìm $\mbox{\rm ƯCLN}(a,b)$.
\end{baitoan}

\begin{baitoan}[\cite{Binh_boi_duong_Toan_6_tap_1}, 6.13., p. 47]
	Tìm $a,b\in\mathbb{N}$ biết: (a) $5a = 13b$ \& $\mbox{\rm ƯCLN}(a,b) = 48$. (b) ${\rm BCNN}(a,b) = 360$ \& $ab = 6480$. (c) $a + b = 40$ \& ${\rm BCNN}(a,b) = 7\mbox{\rm ƯCLN}(a,b)$.
\end{baitoan}

\begin{baitoan}[\cite{Binh_boi_duong_Toan_6_tap_1}, 6.14., p. 47]
	Tìm $a,b\in\mathbb{N}$ biết $a + 2b = 48$ \& $\mbox{\rm ƯCLN}(a,b) + 3{\rm BCNN}(a,b) = 114$.
\end{baitoan}

\begin{baitoan}[\cite{Binh_boi_duong_Toan_6_tap_1}, 6.15., p. 47]
	Tìm $a,b\in\mathbb{N}$ biết $\mbox{\rm ƯCLN}(a,b) + {\rm BCNN}(a,b) = 21$.
\end{baitoan}

\begin{baitoan}[\cite{Binh_boi_duong_Toan_6_tap_1}, 6.16., p. 47]
	Cho $a = 123456789$ \& $b = 987654321$. Tìm $\mbox{\rm ƯCLN}(a,b)$.
\end{baitoan}

\begin{baitoan}[\cite{Binh_boi_duong_Toan_6_tap_1}, 6.17., p. 47, Thừa Thiên -- Huế 2007]
	Tìm 4 chữ số $a,b,c,d$ sao cho 4 số $a,\overline{ad},\overline{cd},\overline{abcd}$ là 4 số chính phương.
\end{baitoan}

\begin{baitoan}[\cite{Binh_boi_duong_Toan_6_tap_1}, p. 48, 1 số tính chất mở rộng về ƯCLN \& BCNN]
	Chứng minh: (a) $\mbox{\rm ƯCLN}(a,b,c) = \mbox{\rm ƯCLN}(\mbox{\rm ƯCLN}(a,b),c)$. (b) ${\rm BCNN}(a,b,c) = {\rm BCNN}({\rm BCNN}(a,b),c)$. (c) $\mbox{\rm ƯCLN}(ma,mb) = m\mbox{\rm ƯCLN}(a,b)$. (d) ${\rm BCNN}(ma,mb) = m{\rm BCNN}(a,b)$. (e) $\mbox{\rm ƯCLN}(a + kb,b) = \mbox{\rm ƯCLN}(a,b)$. (f) Nếu $ab\divby m$ \& $\mbox{\rm ƯCLN}(b,m) = 1$ thì $a\divby m$. (g) Nếu $a\divby m$ \& $a\divby n$ thì $a\divby{\rm BCNN}(m,n)$. (h) Nếu $a\divby m$, $a\divby n$, \& $\mbox{\rm ƯCLN}(m,n) = 1$ thì $a\divby mn$.
\end{baitoan}

%------------------------------------------------------------------------------%

\section{Miscellaneous}

%------------------------------------------------------------------------------%

\printbibliography[heading=bibintoc]

\end{document}