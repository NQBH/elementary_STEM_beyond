\documentclass{article}
\usepackage[backend=biber,natbib=true,style=alphabetic,maxbibnames=50]{biblatex}
\addbibresource{/home/nqbh/reference/bib.bib}
\usepackage[utf8]{vietnam}
\usepackage{tocloft}
\renewcommand{\cftsecleader}{\cftdotfill{\cftdotsep}}
\usepackage[colorlinks=true,linkcolor=blue,urlcolor=red,citecolor=magenta]{hyperref}
\usepackage{amsmath,amssymb,amsthm,float,graphicx,mathtools,tipa}
\usepackage{enumitem}
\setlist{leftmargin=4mm}
\allowdisplaybreaks
\newtheorem{assumption}{Assumption}
\newtheorem{baitoan}{}
\newtheorem{cauhoi}{Câu hỏi}
\newtheorem{conjecture}{Conjecture}
\newtheorem{corollary}{Corollary}
\newtheorem{dangtoan}{Dạng toán}
\newtheorem{definition}{Definition}
\newtheorem{dinhly}{Định lý}
\newtheorem{dinhnghia}{Định nghĩa}
\newtheorem{example}{Example}
\newtheorem{ghichu}{Ghi chú}
\newtheorem{hequa}{Hệ quả}
\newtheorem{hypothesis}{Hypothesis}
\newtheorem{lemma}{Lemma}
\newtheorem{luuy}{Lưu ý}
\newtheorem{nhanxet}{Nhận xét}
\newtheorem{notation}{Notation}
\newtheorem{note}{Note}
\newtheorem{principle}{Principle}
\newtheorem{problem}{Problem}
\newtheorem{proposition}{Proposition}
\newtheorem{question}{Question}
\newtheorem{remark}{Remark}
\newtheorem{theorem}{Theorem}
\newtheorem{vidu}{Ví dụ}
\usepackage[left=1cm,right=1cm,top=5mm,bottom=5mm,footskip=4mm]{geometry}
\def\labelitemii{$\circ$}
\DeclareRobustCommand{\divby}{%
	\mathrel{\vbox{\baselineskip.65ex\lineskiplimit0pt\hbox{.}\hbox{.}\hbox{.}}}%
}

\title{Problem: Prime, Composite, GCD, {\it\&} LCM\\Bài Tập: Số Nguyên Tố, Hợp Số, ƯCLN, {\it\&} BCNN}
\author{Nguyễn Quản Bá Hồng\footnote{Independent Researcher, Ben Tre City, Vietnam\\e-mail: \texttt{nguyenquanbahong@gmail.com}; website: \url{https://nqbh.github.io}.}}
\date{\today}

\begin{document}
\maketitle
\begin{abstract}
	Last updated version: \href{https://github.com/NQBH/elementary_STEM_beyond/blob/main/elementary_mathematics/grade_6/natural/prime_gcd_lcm/problem/NQBH_prime_gcd_lcm_problem.pdf}{GitHub{\tt/}NQBH{\tt/}hobby{\tt/}elementary mathematics{\tt/}grade 6{\tt/}natural{\tt/}prime, gcd, lcm{\tt/}problem[pdf]}.\footnote{\textsc{url}: \url{https://github.com/NQBH/elementary_STEM_beyond/blob/main/elementary_mathematics/grade_6/natural/prime_gcd_lcm/problem/NQBH_prime_gcd_lcm_problem.pdf}.} [\href{https://github.com/NQBH/elementary_STEM_beyond/blob/main/elementary_mathematics/grade_6/natural/prime_gcd_lcm/problem/NQBH_prime_gcd_lcm_problem.tex}{\TeX}]\footnote{\textsc{url}: \url{https://github.com/NQBH/elementary_STEM_beyond/blob/main/elementary_mathematics/grade_6/natural/prime_gcd_lcm/problem/NQBH_prime_gcd_lcm_problem.tex}.}. 
\end{abstract}
\tableofcontents

%------------------------------------------------------------------------------%

\section{Prime. Composite -- Số Nguyên Tố. Hợp Số}

\begin{baitoan}[\cite{Binh_boi_duong_Toan_6_tap_1}, H1, p. 36]
	Egg có $54$ viên bi \& muốn chia đều số bi đó vào các hộp. Tìm tất cả các cách chia thỏa mãn.
\end{baitoan}

\begin{baitoan}[\cite{Binh_boi_duong_Toan_6_tap_1}, H2, p. 36]
	(a) Số nào có phân tích ra thừa số nguyên tố là $2^3\cdot3^2\cdot7$. (b) Phân tích $2160$ ra thừa số nguyên tố.
\end{baitoan}

\begin{baitoan}[\cite{Binh_boi_duong_Toan_6_tap_1}, H3, p. 36]
	Tìm chữ số $a$ để $\overline{17a}$ là số nguyên tố.
\end{baitoan}

\begin{baitoan}[\cite{Binh_boi_duong_Toan_6_tap_1}, H4, p. 36]
	{\rm Đ{\tt/}S?} Ký hiệu $P$ là tập hợp các số nguyên tố. (a) $19\in P$. (b) $\{3,5,7\}\in P$. (c) $\{71,73\}\in P$. (d) $6\cdot7\cdot8\cdot9 - 5\cdot7\cdot11\in P$. (e) Mọi số nguyên tố đều có tận cùng là số lẻ.
\end{baitoan}

\begin{baitoan}[\cite{Binh_boi_duong_Toan_6_tap_1}, Ví dụ 1, p. 37]
	Cho 1 phép chia có số bị chia bằng $236$ \& số dư bằng $15$. Tìm số chia \& thương.
\end{baitoan}

\begin{baitoan}[\cite{Binh_boi_duong_Toan_6_tap_1}, Ví dụ 2, p. 37]
	Có bao nhiêu số là bội của $6$ trong khoảng từ $72$ đến $2016$?
\end{baitoan}

\begin{baitoan}[\cite{Binh_boi_duong_Toan_6_tap_1}, Ví dụ 3, p. 37]
	Tìm $x\in\mathbb{N}$ sao cho $42\divby(2x + 5)$.
\end{baitoan}

\begin{baitoan}[\cite{Binh_boi_duong_Toan_6_tap_1}, Ví dụ 4, p. 38]
	Tìm số nguyên tố $p$ sao cho $p + 2$ \& $p + 4$ cũng là 2 số nguyên tố.
\end{baitoan}

\begin{baitoan}[\cite{Binh_boi_duong_Toan_6_tap_1}, Ví dụ 5, p. 38]
	Cho $p > 3$ \& $2p + 1$ là 2 số nguyên tố. Hỏi $4p + 1$ là số nguyên tố hay hợp số.
\end{baitoan}

\begin{baitoan}[\cite{Binh_boi_duong_Toan_6_tap_1}, Ví dụ 6, p. 39]
	Tìm số nguyên tố bằng tổng của 2 số nguyên tố \& cũng bằng hiệu của 2 số nguyên tố khác.
\end{baitoan}

\begin{luuy}
	$2$ là số nguyên tố chẵn duy nhất.
\end{luuy}

\begin{baitoan}[\cite{Binh_boi_duong_Toan_6_tap_1}, Ví dụ 7, p. 39]
	Phân tích ra thừa số nguyên tố: (a) $2016^7$. (b) $30\cdot4\cdot1975$.
\end{baitoan}

\begin{baitoan}[\cite{Binh_boi_duong_Toan_6_tap_1}, Ví dụ 8, p. 39]
	Tìm $n\in\mathbb{N}^\star$ thỏa $2 + 4 + 6 + \cdots + 2n = 870$.
\end{baitoan}

\begin{baitoan}[\cite{Binh_boi_duong_Toan_6_tap_1}, Ví dụ 9, p. 40]
	Tìm $n\in\mathbb{N}^\star$ sao cho $p = (n - 2)(n^2 + n - 5)$ là số nguyên tố.
\end{baitoan}

\begin{baitoan}[\cite{Binh_boi_duong_Toan_6_tap_1}, 5.1., p. 40]
	Tìm tập hợp các số tự nhiên vừa là bội của $9$, vừa là ước của $72$.
\end{baitoan}

\begin{baitoan}[\cite{Binh_boi_duong_Toan_6_tap_1}, 5.2., p. 40]
	Tìm $x\in\mathbb{N}^\star$ thỏa: (a) $x - 1$ là ước của $24$. (b) $36$ là bội của $2x + 1$.
\end{baitoan}

\begin{baitoan}[\cite{Binh_boi_duong_Toan_6_tap_1}, 5.3., p. 40]
	Tìm $x,y\in\mathbb{N}^\star$ thỏa $(2x + 1)(y - 3) = 15$.
\end{baitoan}

\begin{baitoan}[\cite{Binh_boi_duong_Toan_6_tap_1}, 5.4., p. 40]
	Phân tích ra thừa số nguyên tố: (a) $1\cdot12\cdot78$. (b) $1930^8$.
\end{baitoan}

\begin{baitoan}[\cite{Binh_boi_duong_Toan_6_tap_1}, 5.5., p. 40]
	Chứng minh nếu $p$ là 1 số nguyên tố lớn hơn $3$ thì $(p - 1)(p + 1)$ chia hết cho $3$ \& cho $8$.
\end{baitoan}

\begin{baitoan}[\cite{Binh_boi_duong_Toan_6_tap_1}, 5.6., p. 40]
	Tìm chữ số $a$ để $\overline{23a}$ là số nguyên tố.
\end{baitoan}

\begin{baitoan}[\cite{Binh_boi_duong_Toan_6_tap_1}, 5.7., p. 40]
	Tìm số tự nhiên nhỏ nhất có đúng $18$ ước số.
\end{baitoan}

\begin{baitoan}[\cite{Binh_boi_duong_Toan_6_tap_1}, 5.8., p. 40]
	Chứng minh: Nếu 1 số tự nhiên có 3 chữ số tận cùng là $104$ thì số đó có ít nhất $4$ ước số.
\end{baitoan}

\begin{baitoan}[\cite{Binh_boi_duong_Toan_6_tap_1}, 5.9., p. 40]
	Tìm 2 số nguyên tố có tổng bằng $309$.
\end{baitoan}

\begin{baitoan}[\cite{Binh_boi_duong_Toan_6_tap_1}, 5.10., p. 40]
	Tìm số nguyên tố $p$ sao cho $p + 4,p + 8$ cũng là 2 số nguyên tố.
\end{baitoan}

\begin{baitoan}[\cite{Binh_boi_duong_Toan_6_tap_1}, 5.11., p. 40]
	Tìm số nguyên tố $p$ sao cho $p + 6,p + 8,p + 12,p + 14$ cũng là 4 số nguyên tố.
\end{baitoan}

\begin{baitoan}[\cite{Binh_boi_duong_Toan_6_tap_1}, 5.12., p. 40]
	Cho $ptố > 3$ \& $p + 4$ là 2 số nguyên tố. Chứng minh $p + 8$ là hợp số.
\end{baitoan}

\begin{baitoan}[\cite{Binh_boi_duong_Toan_6_tap_1}, 5.13., p. 40]
	Số $3^2 + 3^4 + 3^6 + \cdots + 3^{2012}$ là số nguyên tố hay hợp số?
\end{baitoan}

\begin{baitoan}[\cite{Binh_boi_duong_Toan_6_tap_1}, 5.14., p. 40]
	2 số nguyên tố được gọi là {\rm sinh đôi} nếu chúng là 2 số nguyên tố \& là 2 số lẻ liên tiếp, e.g., $3$ \& $5$, $11$ \& $13,\ldots$. Chứng minh số tự nhiên lớn hơn $4$ \& nằm giữa 2 số nguyên tố sinh đôi thì chia hết cho $6$.
\end{baitoan}

\begin{baitoan}[\cite{Binh_boi_duong_Toan_6_tap_1}, 5.15., p. 41]
	Tìm 3 số tự nhiên lẻ liên tiếp đều là số nguyên tố.
\end{baitoan}

\begin{baitoan}[\cite{Binh_boi_duong_Toan_6_tap_1}, 5.16., p. 41]
	Tìm $n\in\mathbb{N}^\star$ thỏa $1 + 3 + 5 + \cdots + (2n + 1) = 169$.
\end{baitoan}

\begin{baitoan}[\cite{Binh_boi_duong_Toan_6_tap_1}, 5.17., p. 41]
	Biết số $\overline{abc}$ khi phân tích ra t hừa số nguyên tố có thừa số $3$ \& thừa số $7$. Chứng minh số $a + 19b + 4c$ cũng có tính chất đó.
\end{baitoan}

\begin{baitoan}[\cite{Binh_boi_duong_Toan_6_tap_1}, 5.18., p. 41]
	Tìm chữ số $a$ sao cho số $\overline{aaa}$ là tổng của các số tự nhiên liên tiếp từ $1$ đến số $n$ nào đó.
\end{baitoan}

\begin{baitoan}
	Chứng minh tập hợp các số nguyên tố có vô hạn phần tử \& không có số nguyên tố lớn nhất.
\end{baitoan}
\noindent\textit{Hint.} Giả sử phản chứng: chỉ có hữu hạn số nguyên tố $p_1 < p_2 < \cdots < p_n$. Chứng minh $p\coloneqq\prod_{i=1}^n p_i + 1 = p_1p_2\cdots p_n + 1$ là 1 số nguyên tố lớn hơn mỗi số nguyên tố $p_i$, $\forall i\in\mathbb{N}$.

\begin{baitoan}[\cite{Tuyen_Toan_6}, VD26, p. 25]
	Tìm số nguyên tố $a$ để $4a + 11$ là số nguyên tố nhỏ hơn $30$.
\end{baitoan}

\begin{baitoan}[\cite{Tuyen_Toan_6}, VD27, p. 25]
	Cho $A = \sum_{i=1}^{100} 5^i = 5 + 5^2 + \cdots + 5^{100}$. (a) Hỏi A là số nguyên tố hay hợp số? (b) Số A có phải là số chính phương không?
\end{baitoan}

\begin{baitoan}[\cite{Tuyen_Toan_6}, VD28, p. 2]
	Tính cạnh của 1 hình vuông có diện tích $\rm5929\ m^2$.
\end{baitoan}

\begin{baitoan}[\cite{Tuyen_Toan_6}, 116., p. 26]
	Phân loại số nguyên tố, hợp số: (a) $A = 1\cdot3\cdot5\cdot7\cdots13 + 20$. (b) $B = 147\cdot247\cdot347 - 13$.
\end{baitoan}

\begin{baitoan}[\cite{Tuyen_Toan_6}, 117., p. 26]
	Tìm số bị chia \& thương trong phép chia: $9\star\star:17 = \star\star$. Biết thương là 1 số nguyên tố.
\end{baitoan}

\begin{baitoan}[\cite{Tuyen_Toan_6}, 118., p. 26]
	Cho $a,n\in\mathbb{N}^\star$. Biết $a^n\divby5$. Chứng minh $a^2 + 150\divby25$.
\end{baitoan}

\begin{baitoan}[\cite{Tuyen_Toan_6}, 119., p. 26]
	(a) Cho $n\in\mathbb{N}$, $n\not{\divby}\ 3$. Chứng minh $n^2$ chia cho $3$ dư $1$. (b) Cho $p$ là 1 số nguyên tố lớn hơn $3$. Hỏi $p^2 + 2021$ là số nguyên tố hay hợp số?
\end{baitoan}

\begin{baitoan}[\cite{Tuyen_Toan_6}, 120., p. 26]
	Cho $n\in\mathbb{N}$, $n > 2$, $n\not{\divby}\ 3$. Chứng minh 2 số $n^2\pm1$ không thể đồng thời là 2 số nguyên tố.
\end{baitoan}

\begin{baitoan}[\cite{Tuyen_Toan_6}, 121., p. 26]
	Cho $p > 3,p + 8$ đều là số nguyên tố. Hỏi $p + 100$ là số nguyên tố hay hợp số?
\end{baitoan}

\begin{baitoan}[\cite{Tuyen_Toan_6}, 122., p. 26]
	Phân tích ra thừa số nguyên tố bằng cách hợp lý nhất: (a) $700,9000,210000$. (b) $500,1600,18000$.
\end{baitoan}

\begin{baitoan}[\cite{Tuyen_Toan_6}, 123., p. 26]
	Đếm số ước số của: $90,540,3675$.
\end{baitoan}

\begin{baitoan}[\cite{Tuyen_Toan_6}, 124., p. 26]
	Tìm: (a) 2 số tự nhiên liên tiếp có tích bằng $1260$. (b) 3 số tự nhiên liên tiếp có tích bằng $3360$.
\end{baitoan}

\begin{baitoan}[\cite{Tuyen_Toan_6}, 125., p. 26]
	Tìm: (a) 3 số chẵn liên tiếp có tích bằng $5760$. (b) 3 số lẻ liên tiếp có tích bằng $19575$.
\end{baitoan}

\begin{baitoan}[\cite{Tuyen_Toan_6}, 126., p. 26]
	Tính cạnh của 1 hình lập phương biết thể tích của nó là $\rm1728\ cm^3$.
\end{baitoan}

\begin{baitoan}[\cite{Tuyen_Toan_6}, 127., p. 27]
	Chứng minh 1 số tự nhiên $\ne0$ có số lượng các ước là 1 số lẻ $\Leftrightarrow$ số tự nhiên đó là số chính phương.
\end{baitoan}

\begin{baitoan}[\cite{Tuyen_Toan_6}, 128., p. 27]
	Tìm $n\in\mathbb{N}^\star$ thỏa: (a) $2 + 4 + 6 + \cdots + 2n = 210$. (b) $1 + 3 + 5 + \cdots + (2n - 1) = 225$.
\end{baitoan}

\begin{baitoan}[\cite{Binh_Toan_6_tap_1}, VD32, p. 30]
	Điền các chữ số thích hợp trong phép phân tích ra thừa số nguyên tố: $\overline{abcd} = e\overline{fcga} = en\overline{abc} = enc\overline{ncf} = \ldots$
\end{baitoan}

\begin{baitoan}[\cite{Binh_Toan_6_tap_1}, VD33, p. 30]
	Tìm số nguyên tố $p$ sao cho $p + 2,p + 4$ cũng là 2 số nguyên tố.
\end{baitoan}

\begin{baitoan}[\cite{Binh_Toan_6_tap_1}, VD34, p. 31]
	1 số nguyên tố $p$ chia cho $42$ có số dư $r$ là hợp số. Tìm số dư $r$.
\end{baitoan}

\begin{baitoan}[\cite{Binh_Toan_6_tap_1}, VD35, p. 31]
	Tìm $n\in\mathbb{N}^\star$ nhỏ nhất sao cho $n! + 1$ là hợp số.
\end{baitoan}

\begin{baitoan}[\cite{Binh_Toan_6_tap_1}, 180., p. 31]
	(a) Đếm số số nguyên tố nhỏ hơn $100$. (b) Tính tổng tất cả các số nguyên tố nhỏ hơn $100$.
\end{baitoan}

\begin{baitoan}[\cite{Binh_Toan_6_tap_1}, 181., p. 31]
	Tổng của 3 số nguyên tố bằng $1012$. Tìm số nhỏ nhất trong 3 số nguyên tố đó.
\end{baitoan}

\begin{baitoan}[\cite{Binh_Toan_6_tap_1}, 182., p. 31]
	Tìm 4 số nguyên tố liên tiếp, sao cho tổng của chúng là số nguyên tố.
\end{baitoan}

\begin{baitoan}[\cite{Binh_Toan_6_tap_1}, 183., p. 31]
	Tổng của 2 số nguyên tố có thể bằng $2003$ không?
\end{baitoan}

\begin{baitoan}[\cite{Binh_Toan_6_tap_1}, 184., p. 31]
	Tìm 2 số tự nhiên sao cho tổng \& tích của chúng đều là số nguyên tố.
\end{baitoan}

\begin{baitoan}[\cite{Binh_Toan_6_tap_1}, 185., p. 31]
	Trong 1 cuộc phỏng vấn tuyển nhân viên làm việc ở Tập đoàn Microsoft của Mỹ, 1 ứng viên nhận được câu hỏi: Tìm số tiếp theo trong dãy $4,6,12,18,30,42,60,\ldots$ Nhờ có kiến thức về số nguyên tố, ứng viên đã trả lời đúng. Số tiếp theo của dãy là số nào?
\end{baitoan}

\begin{baitoan}[\cite{Binh_Toan_6_tap_1}, 186., p. 31]
	Phân loại số nguyên tố \& hợp số: (a) $a = \underbrace{1\ldots1}_{2001}, b = \underbrace{1\ldots1}_{2000}, c = 1010101, d = 1112111, e = \sum_{i=1}^{100} i! = 1! + 2! + \cdots + 100!, f = 3\cdot5\cdot7\cdot9 - 28, g = 311141111$.
\end{baitoan}

\begin{baitoan}[\cite{Binh_Toan_6_tap_1}, 187., p. 31]
	Tìm số nguyên tố có 3 chữ số biết nếu viết số đó theo thứ tự ngược lại thì ta được 1 số là lập phương của 1 số tự nhiên.
\end{baitoan}

\begin{baitoan}[\cite{Binh_Toan_6_tap_1}, 188., p. 31]
	Tìm số tự nhiên có 4 chữ số, chữ số hàng nghìn bằng chữ số hàng đơn vị, chữ số hàng trăm bằng chữ số hàng chục, \& số đó viết được dưới dạng tích của 3 số nguyên tố liên tiếp.
\end{baitoan}

\begin{baitoan}[\cite{Binh_Toan_6_tap_1}, 189., p. 32]
	Tìm số nguyên tố $p$ sao cho các số sau cũng là số nguyên tố: (a) $p + 2,p + 10$. (b) $p + 10,p + 20$. (c) $p + 2,p + 6,p + 8,p + 12,p + 14$.
\end{baitoan}

\begin{baitoan}[\cite{Binh_Toan_6_tap_1}, 190., p. 32]
	Tìm số nguyên tố biết số đó bằng tổng của 2 số nguyên tố \& bằng hiệu của 2 số nguyên tố.
\end{baitoan}

\begin{baitoan}[\cite{Binh_Toan_6_tap_1}, 191., p. 32]
	Cho 3 số nguyên tố lớn hơn $3$, trong đó số sau lớn hơn số trước là $d$ đơn vị. Chứng minh $d\divby6$.
\end{baitoan}

\begin{baitoan}[\cite{Binh_Toan_6_tap_1}, 192., p. 32]
	2 số nguyên tố gọi là {\rm sinh đôi} nếu chúng là 2 số nguyên tố lẻ liên tiếp. Chứng minh 1 số tự nhiên lớn hơn $3$ nằm giữa 2 số nguyên tố sinh đôi thì chia hết cho $6$.
\end{baitoan}

\begin{baitoan}[\cite{Binh_Toan_6_tap_1}, 193., p. 32]
	Cho $p > 3$ là số nguyên tố. Biết $p + 2$ cũng là số nguyên tố. Chứng minh $p + 1\divby6$.
\end{baitoan}

\begin{baitoan}[\cite{Binh_Toan_6_tap_1}, 194., p. 32]
	Cho $p > 3,p + 4$ là 2 số nguyên tố. Chứng minh $p + 8$ là hợp số.
\end{baitoan}

\begin{baitoan}[\cite{Binh_Toan_6_tap_1}, 195., p. 32]
	Cho $p,8p - 1$ là 2 số nguyên tố. Chứng minh $8p + 1$ là hợp số.
\end{baitoan}

\begin{baitoan}[\cite{Binh_Toan_6_tap_1}, 196., p. 32]
	1 ngày đầu năm 2002, Huy viết thư hỏi ngày sinh của Long \& nhận được thư trả lời: Mình sinh ngày $a$, tháng $b$, năm $1900 + c$, \& đến nay $d$ tuổi. Biết $abcd = 59007$. Huy đã tính được ngày sinh của anh Long \& kịp viết thư mừng sinh nhật bạn. Tìm ngày sinh của Long.
\end{baitoan}

\begin{baitoan}[\cite{Binh_Toan_6_tap_1}, 197., p. 32]
	1 số nguyên tố chia cho $30$ có số dư là $r$. Tìm $r$ biết $r$ không là số nguyên tố.
\end{baitoan}

\begin{baitoan}[\cite{Binh_Toan_6_tap_1}, 198., p. 32]
	Chứng minh: (a) Số $17$ không viết được dưới dạng tổng của 3 hợp số khác nhau. (b) Mọi số lẻ lớn hơn $17$ đều viết được dưới dạng tổng của 3 hợp số khác nhau.
\end{baitoan}

\begin{baitoan}[\cite{Binh_Toan_6_tap_1}, 199., p. 32]
	Tuổi trung bình của 8 người là $15$, trong đó tuổi mỗi người đều là số nguyên tố. Trong 4 người nhiều tuổi nhất, có 3 người $19$ tuổi. Tuổi trung bình của người nhiều tuổi thứ 4 \& thứ 5 là $11$. Tính tuổi của người nhiều tuổi nhất.
\end{baitoan}

%------------------------------------------------------------------------------%

\section{Greatest Common Divisor. Least Common Multiple -- Ước Chung Lớn Nhất. Bội Chung Nhỏ Nhất}

\begin{baitoan}[\cite{Binh_boi_duong_Toan_6_tap_1}, H1, p. 43]
	1 thửa ruộng hình chữ nhật có chiều dài {\rm72 m}, chiều rộng {\rm40 m}. Chicken muốn chia thửa ruộng thành các mảnh đất hình vuông bằng nhau để trồng các loại ngũ cốc. Tính độ dài lớn nhất của hình vuông mà Chicken có thể chia.
\end{baitoan}

\begin{baitoan}[\cite{Binh_boi_duong_Toan_6_tap_1}, H2, p. 43]
	Có 4 thuyền A, B, C, D. Thuyền A cứ $5$ ngày cập bến 1 lần, thuyền B cứ $6$ ngày cập bến  1 lần, thuyền C cứ $8$ ngày cập bến 1 lần \& thuyền D cứ $10$ ngày cập bến 1 lần. Egg nhẩm tính: Nếu ngày hôm nay cả 4 thuyền cùng cập bến thì: (a) Sau ít nhất $a$ ngày nữa, thuyền A cùng cập bến với thuyền D. (b) Sau ít nhất $b$ ngày nữa, thuyền B cùng cập bến với thuyền C. (c) Sau ít nhất $c$ ngày nữa, thuyền B cùng cập bến với thuyền D. (d) Sau ít nhất $d$ ngày nữa, cả 4 thuyền sẽ cùng cập bến lần thứ 2. Tìm $a,b,c,d$.
\end{baitoan}

\begin{baitoan}[\cite{Binh_boi_duong_Toan_6_tap_1}, Ví dụ 1, p. 43]
	Tìm $\mbox{\rm ƯC}(48,60)$, ${\rm BC}(4,14)$.
\end{baitoan}

\begin{baitoan}[\cite{Binh_boi_duong_Toan_6_tap_1}, Ví dụ 2, p. 44]
	Tìm $a\in\mathbb{N}$ biết chia $264$ cho $a$ thì dư $24$, còn khi chia $363$ cho $a$ thì được dư là $43$.
\end{baitoan}

\begin{baitoan}[\cite{Binh_boi_duong_Toan_6_tap_1}, Ví dụ 3, p. 44]
	Tìm số tự nhiên nhỏ nhất có 4 chữ số biết khi chia số đó cho $18,24,30$ thì có số dư lần lượt là $13,19,25$.
\end{baitoan}

\begin{baitoan}[\cite{Binh_boi_duong_Toan_6_tap_1}, Ví dụ 4, p. 44]
	Tìm $a,b\in\mathbb{N}$ thỏa $a + b = 336$ \& $\mbox{\rm ƯCLN}(a,b) = 24$.
\end{baitoan}

\begin{baitoan}[\cite{Binh_boi_duong_Toan_6_tap_1}, Ví dụ 5, p. 45]
	Tìm $a,b\in\mathbb{N}$ thỏa $\mbox{\rm ƯCLN}(a,b) = 24$ \& ${\rm BCNN}(a,b) = 36$.
\end{baitoan}

\begin{baitoan}[\cite{Binh_boi_duong_Toan_6_tap_1}, Ví dụ 6, p. 45]
	Cho $n\in\mathbb{N}^\star$. Chứng minh: $\mbox{\rm ƯCLN}(2n + 5,3n + 7) = 1$.
\end{baitoan}

\begin{baitoan}[\cite{Binh_boi_duong_Toan_6_tap_1}, Ví dụ 7, p. 46]
	Học sinh khối 6 của 1 trường khi xếp hàng $12$, hàng $15$ hay hàng $18$ thì đều vừa đủ hàng. Tính số học sinh khối 6 của trường đó biết số học sinh này nằm trong khoảng từ $500$ đến $600$ học sinh.
\end{baitoan}

\begin{baitoan}[\cite{Binh_boi_duong_Toan_6_tap_1}, Ví dụ 8, p. 46]
	1 lớp học có $28$ học sinh nam \& $24$ học sinh nữ. Khi tham gia lao động, {\rm GVCN} muốn chia lớp thành các nhóm sao cho số học sinh nam \& số học sinh nữ được chia đều vào các nhóm. Hỏi {\rm GVCN} có bao nhiêu cách chia nhóm? Cách chia nào có số học sinh trong mỗi nhóm ít nhất?
\end{baitoan}

\begin{baitoan}[\cite{Binh_boi_duong_Toan_6_tap_1}, 6.1., p. 47]
	Tìm $\mbox{\rm ƯC}(54,120,180)$, ${\rm BC}(21,84)$.
\end{baitoan}

\begin{baitoan}[\cite{Binh_boi_duong_Toan_6_tap_1}, 6.2., p. 47]
	1 số chia cho $21$ dư $2$ \& chia cho $12$ dư $5$. Hỏi số đó chia cho $84$ thì dư bao nhiêu?
\end{baitoan}

\begin{baitoan}[\cite{Binh_boi_duong_Toan_6_tap_1}, 6.3., p. 47]
	Tìm $a\in\mathbb{N}$ thỏa mãn: $a\divby7$ \& $a$ chia cho $4$ hoặc $6$ đều dư $3$ biết $a < 350$.
\end{baitoan}

\begin{baitoan}[\cite{Binh_boi_duong_Toan_6_tap_1}, 6.4., p. 47]
	Tìm số tự nhiên lớn nhất có 3 chữ số sao cho chia nó cho $3$, cho $4$, cho $5$ ta được 3 số dư theo thứ tự là $2,3,4$.
\end{baitoan}

\begin{baitoan}[\cite{Binh_boi_duong_Toan_6_tap_1}, 6.5., p. 47]
	Cho $\mbox{\rm ƯCLN}(a,b) = 1$. Chứng minh: (a) $\mbox{\rm ƯCLN}(a,a - b) = 1$ với $a > b$. (b) $\mbox{\rm ƯCLN}(ab,a + b) = 1$.
\end{baitoan}

\begin{baitoan}[\cite{Binh_boi_duong_Toan_6_tap_1}, 6.6., p. 47]
	Cho $n\in\mathbb{N}$. Chứng minh: (a) $\mbox{\rm ƯCLN}(3n + 13,3n + 14) = 1$. (b) $\mbox{\rm ƯCLN}(3n +5,6n + 9) = 1$.
\end{baitoan}

\begin{baitoan}[\cite{Binh_boi_duong_Toan_6_tap_1}, 6.7., p. 47]
	1 lớp học có $27$ học sinh nam \& $18$ học sinh nữ. Có bao nhiêu cách chia lớp đó thành các tổ sao cho số học sinh nam \& số học sinh nữ được chia đều vào các tổ? Biết số tổ lớn hơn $1$.
\end{baitoan}

\begin{baitoan}[\cite{Binh_boi_duong_Toan_6_tap_1}, 6.8., p. 47]
	1 đơn vị bộ đội khi xếp hàng, mỗi hàng có $20$ người, hoặc $25$ người, hoặc $30$ người đều thừa $15$ người. Nếu xếp mỗi hàng $41$ người thì vừa đủ (không có hàng nào thiếu, không có ai ở ngoài hàng). Hỏi đơn vị có bao nhiêu người biết số người của đơn vị chưa đến $1000$?
\end{baitoan}

\begin{baitoan}[\cite{Binh_boi_duong_Toan_6_tap_1}, 6.9., p. 47]
	Tổng số học sinh khối 6 của 1 trường có khoảng từ $235$ đến $250$ em, khi chia cho $3$ thì dư $2$, chia cho $4$ thì dư $3$, chia cho $5$ thì dư $4$, chia cho $6$ thì dư $5$, chia cho $10$ thì dư $9$. Tìm số học sinh của khối 6.
\end{baitoan}

\begin{baitoan}[\cite{Binh_boi_duong_Toan_6_tap_1}, 6.10., p. 47]
	1 trường tổ chức cho học sinh đi tham quan bằng ôtô. Nếu xếp $27$ hay $36$ học sinh lên 1 ôtô thì đều thấy thừa ra $11$ học sinh. Tính số học sinh đi tham quan biết số học sinh đó có khoảng từ $400$ đến $450$ em.
\end{baitoan}

\begin{baitoan}[\cite{Binh_boi_duong_Toan_6_tap_1}, 6.11., p. 47]
	Cho 2 số nguyên tố cùng nhau $a,b$. Chứng minh 2 số $13a + 4b$ \& $15a + 7b$ hoặc nguyên tố cùng nhau hoặc có 1 ước chung là $31$.
\end{baitoan}

\begin{baitoan}[\cite{Binh_boi_duong_Toan_6_tap_1}, 6.12., p. 47]
	Cho $a,b\in\mathbb{N}$ không nguyên tố cùng nhau thỏa $a = 2n + 3$, $b = 3n + 1$ với $n\in\mathbb{N}$. Tìm $\mbox{\rm ƯCLN}(a,b)$.
\end{baitoan}

\begin{baitoan}[\cite{Binh_boi_duong_Toan_6_tap_1}, 6.13., p. 47]
	Tìm $a,b\in\mathbb{N}$ biết: (a) $5a = 13b$ \& $\mbox{\rm ƯCLN}(a,b) = 48$. (b) ${\rm BCNN}(a,b) = 360$ \& $ab = 6480$. (c) $a + b = 40$ \& ${\rm BCNN}(a,b) = 7\mbox{\rm ƯCLN}(a,b)$.
\end{baitoan}

\begin{baitoan}[\cite{Binh_boi_duong_Toan_6_tap_1}, 6.14., p. 47]
	Tìm $a,b\in\mathbb{N}$ biết $a + 2b = 48$ \& $\mbox{\rm ƯCLN}(a,b) + 3{\rm BCNN}(a,b) = 114$.
\end{baitoan}

\begin{baitoan}[\cite{Binh_boi_duong_Toan_6_tap_1}, 6.15., p. 47]
	Tìm $a,b\in\mathbb{N}$ biết $\mbox{\rm ƯCLN}(a,b) + {\rm BCNN}(a,b) = 21$.
\end{baitoan}

\begin{baitoan}[\cite{Binh_boi_duong_Toan_6_tap_1}, 6.16., p. 47]
	Cho $a = 123456789$ \& $b = 987654321$. Tìm $\mbox{\rm ƯCLN}(a,b)$.
\end{baitoan}

\begin{baitoan}[\cite{Binh_boi_duong_Toan_6_tap_1}, 6.17., p. 47, Thừa Thiên -- Huế 2007]
	Tìm 4 chữ số $a,b,c,d$ sao cho 4 số $a,\overline{ad},\overline{cd},\overline{abcd}$ là 4 số chính phương.
\end{baitoan}

\begin{baitoan}[\cite{Binh_boi_duong_Toan_6_tap_1}, p. 48, 1 số tính chất mở rộng về ƯCLN \& BCNN]
	Chứng minh: (a) $\mbox{\rm ƯCLN}(a,b,c) = \mbox{\rm ƯCLN}(\mbox{\rm ƯCLN}(a,b),c)$. (b) ${\rm BCNN}(a,b,c) = {\rm BCNN}({\rm BCNN}(a,b),c)$. (c) $\mbox{\rm ƯCLN}(ma,mb) = m\mbox{\rm ƯCLN}(a,b)$. (d) ${\rm BCNN}(ma,mb) = m{\rm BCNN}(a,b)$. (e) $\mbox{\rm ƯCLN}(a + kb,b) = \mbox{\rm ƯCLN}(a,b)$. (f) Nếu $ab\divby m$ \& $\mbox{\rm ƯCLN}(b,m) = 1$ thì $a\divby m$. (g) Nếu $a\divby m$ \& $a\divby n$ thì $a\divby{\rm BCNN}(m,n)$. (h) Nếu $a\divby m$, $a\divby n$, \& $\mbox{\rm ƯCLN}(m,n) = 1$ thì $a\divby mn$.
\end{baitoan}

\begin{baitoan}[\cite{Tuyen_Toan_6}, VD29, p. 27]
	Tìm $b\in\mathbb{N}$ biết khia chia $326$ cho $b$ thì dư $11$ còn chia $533$ cho $b$ thì dư $13$.
\end{baitoan}

\begin{baitoan}[\cite{Tuyen_Toan_6}, VD30, p. 28]
	Chứng minh 2 số tự nhiên liên tiếp là 2 số nguyên tố cùng nhau.
\end{baitoan}

\begin{baitoan}[\cite{Tuyen_Toan_6}, 129., p. 28]
	Tìm {\rm ƯCLN, ƯC} của 3 số $432,504,720$.
\end{baitoan}

\begin{baitoan}[\cite{Tuyen_Toan_6}, 130., p. 28]
	Tìm $x\in\mathbb{N}$ lớn nhất sao cho $x + 495,195 - x$ đều chia hết cho $x$.
\end{baitoan}

\begin{baitoan}[\cite{Tuyen_Toan_6}, 131., p. 28]
	1 căn phòng hình chữ nhật có kích thước $630\times480$ {\rm cm} được lát loại gạch hình vuông. Muốn cho 2 hàng gạch cuối cùng sát với 2 bức tường liên tiếp không bị cắt xén thì kích thước lớn nhất của viên gạch là bao nhiêu? Với loại gạch này thì cần bao nhiêu viên gạch để lát cả căn phòng?
\end{baitoan}

\begin{baitoan}[\cite{Tuyen_Toan_6}, 132., p. 28]
	2 lớp 6A, 6B cùng tham gia góp sách truyện để xây dựng thư viện. Mỗi học sinh góp số quyển sách như nhau. Tổng kết lại, lớp 6A góp được $36$ quyển, lớp 6B góp được $39$ quyển. Hỏi mỗi lớp có bao nhiêu bạn góp sách xây dựng thư viện?
\end{baitoan}

\begin{baitoan}[\cite{Tuyen_Toan_6}, 133., p. 28]
	Chứng minh các số sau nguyên tố cùng nhau: (a) 2 số lẻ liên tiếp. (b) $2n + 5,3n + 7$, với $n\in\mathbb{N}$.
\end{baitoan}

\begin{baitoan}[\cite{Tuyen_Toan_6}, 134., p. 28]
	Cho $(a,b) = 1$. Chứng minh $(a,a - b) = 1$. (b) $(ab,a + b) = 1$.
\end{baitoan}

\begin{baitoan}[\cite{Tuyen_Toan_6}, 135., p. 28]
	Cho $a,b$ là 2 số tự nhiên không nguyên tố cùng nhau, $a = 4n + 3,b = 5n + 1$, $n\in\mathbb{N}$. Tìm $(a,b)$.
\end{baitoan}

\begin{baitoan}[\cite{Tuyen_Toan_6}, 136., p. 28]
	{\rm ƯCLN} của 2 số là $45$. Số lớn là $270$. Tìm số nhỏ.
\end{baitoan}

\begin{baitoan}[\cite{Tuyen_Toan_6}, 137., p. 28]
	Tìm 2 số biết tổng của chúng là $162$ \& {\rm ƯCLN} của chúng là $18$.
\end{baitoan}

\begin{baitoan}[\cite{Tuyen_Toan_6}, 138., p. 28]
	Tìm 2 số tự nhiên nhỏ hơn $200$ biết hiệu của chúng là $90$ \& {\rm ƯCLN} của chúng là $15$.
\end{baitoan}

\begin{baitoan}[\cite{Tuyen_Toan_6}, 139., p. 28]
	Tìm 2 số biết tích của chúng là $8748$ \& {\rm ƯCLN} của chúng là $27$.
\end{baitoan}

\begin{baitoan}[\cite{Tuyen_Toan_6}, 140., p. 28]
	$a\in\mathbb{N}$ \& $5$ lần số $a$ có tổng các chữ số như nhau. Chứng minh $a\divby9$.
\end{baitoan}

\begin{baitoan}[\cite{Tuyen_Toan_6}, 141., p. 28]
	Có $64$ người đi tham quan bằng 2 loại xe: loại $12$ chỗ ngồi \& loại $7$ chỗ ngồi (không kể lái xe). Biết số người đi vừa đủ số ghế ngồi, tính số xe mỗi loại.
\end{baitoan}

\begin{baitoan}[\cite{Tuyen_Toan_6}, VD31, p. 29]
	Tìm số tự nhiên nhỏ nhất có $3$ chữ số chia cho $18,30,45$ có số dư lần lượt là $8,20,35$.
\end{baitoan}

\begin{baitoan}[\cite{Tuyen_Toan_6}, VD32, p. 30]
	Trong tiết học thể dục hôm nay, thầy giáo cho các bạn trong lớp xếp hàng $4$, hàng $6$, hàng $9$, mỗi lần đều thấy thừa $2$ học sinh. Thầy không cần biết mỗi lần có bao nhiêu hàng, thầy nói ngay số học sinh có mặt là $38$. Giải thích cách tính nhẩm của thầy.
\end{baitoan}

\begin{baitoan}[\cite{Tuyen_Toan_6}, 142., p. 30]
	1 xe lăn dành cho người khuyết tật có chu vi bánh trước là {\rm63 cm}, chu vi bánh sau là {\rm186 cm}. Người ta đánh dấu 2 điểm tiếp đất của 2 bánh xe này. Hỏi mỗi bánh xe phải lăn ít nhất bao nhiêu vòng thì 2 điểm đã đánh dấu lại cùng tiếp đất 1 lúc?
\end{baitoan}

\begin{baitoan}[\cite{Tuyen_Toan_6}, 143., p. 30]
	3 học sinh mỗi người mua 1 loại bút. Giá tiền 3 loại lần lượt là $4800$ đồng, $6000$ đồng, $8000$ đồng. Biết số tiền phải trả như nhau, hỏi mỗi học sinh mua ít nhất bao nhiêu bút?
\end{baitoan}

\begin{baitoan}[\cite{Tuyen_Toan_6}, 144., p. 30]
	Tìm các bội chung lớn hơn $5000$ nhưng nhỏ hơn $10000$ của 3 số $126,140,180$.
\end{baitoan}

\begin{baitoan}[\cite{Tuyen_Toan_6}, 145., p. 30]
	1 số tự nhiên chia cho $12,18,21$ đều dư $5$. Tìm số đó biết nó xấp xỉ $1000$.
\end{baitoan}

\begin{baitoan}[\cite{Tuyen_Toan_6}, 146., p. 30]
	Khối 6 của 1 trường có chưa tới $400$ học sinh. Khi xếp hàng $10,12,15$ đều dư $3$ nhưng nếu xếp hàng $11$ thì không dư. Tính số học sinh khối 6.
\end{baitoan}

\begin{baitoan}[\cite{Tuyen_Toan_6}, 147., p. 30]
	Tìm $a,b\in\mathbb{N}$ biết ${\rm BCNN}(a,b) = 300,\mbox{\rm ƯCLN}(a,b) = 15$.
\end{baitoan}

\begin{baitoan}[\cite{Tuyen_Toan_6}, 148., p. 30]
	Tìm $a,b\in\mathbb{N}$ biết tích của chúng là $2940$ \& {\rm BCNN} của chúng là $210$.
\end{baitoan}

\begin{baitoan}[\cite{Tuyen_Toan_6}, 149., p. 30]
	Tìm $a,b\in\mathbb{N}$ biết tổng của {\rm BCNN} với {\rm ƯCLN} của chúng là $15$.
\end{baitoan}

\begin{baitoan}[\cite{Tuyen_Toan_6}, 150., p. 30]
	Tìm $a\in\mathbb{N}$ nhỏ nhất có 3 chữ số sao cho chia cho $11$ thì dư $5$, chia cho $13$ thì dư $8$.
\end{baitoan}

\begin{baitoan}[\cite{Tuyen_Toan_6}, 151., p. 30]
	Tìm $x\in\mathbb{N}$ nhỏ nhất sao cho $x$ chia cho $3$ dư $2$, $x$ chia cho $5$ dư $3$, $x$ chia cho $7$ dư $4$.
\end{baitoan}

\begin{baitoan}[\cite{Tuyen_Toan_6}, 152., p. 30]
	Chứng minh nếu $a$ là 1 số lẻ không chia hết cho $3$ thì $a^2 - 1\divby6$.
\end{baitoan}

\begin{baitoan}[\cite{Tuyen_Toan_6}, 153., p. 30]
	Chứng minh tích của $5$ số tự nhiên liên tiếp chia hết cho $120$.
\end{baitoan}

%------------------------------------------------------------------------------%

\section{Nguyên Lý Dirichlet \& Bài Toán Chia Hết}

\begin{baitoan}[\cite{Tuyen_Toan_6}, VD33, p. 31]
	Cho $7$ số tự nhiên bất kỳ. Chứng minh bao giờ cũng có thể chọn ra 2 số mà hiệu của chúng chia hết cho $6$.
\end{baitoan}

\begin{baitoan}[\cite{Tuyen_Toan_6}, VD34, p. 32]
	Cho 3 số lẻ. Chứng minh tồn tại 2 số có tổng hay hiệu chia hết cho $8$.
\end{baitoan}

\begin{baitoan}[\cite{Tuyen_Toan_6}, VD35, p. 32]
	Chứng minh có 1 số tự nhiên gồm toàn chữ số $3$ chia hết cho $41$.
\end{baitoan}

\begin{baitoan}[\cite{Tuyen_Toan_6}, 154., p. 33]
	Chứng minh trong $11$ số tự nhiên bất kỳ bao giờ cũng có ít nhất $2$ số có chữ số tận cùng giống nhau.
\end{baitoan}

\begin{baitoan}[\cite{Tuyen_Toan_6}, 155., p. 33]
	Chứng minh tồn tại 1 bội của $13$ gồm toàn chữ số $2$.
\end{baitoan}

\begin{baitoan}[\cite{Tuyen_Toan_6}, 156., p. 33]
	Chứng minh có thể tìm được 1 số có dạng $987987\cdots987$ chia hết cho $2021$.
\end{baitoan}

\begin{baitoan}[\cite{Tuyen_Toan_6}, 157., p. 33]
	Cho dãy số $10,10^2,10^3,\ldots,10^{20}$. Chứng minh tồn tại 1 số chia cho $19$ dư $1$.
\end{baitoan}

\begin{baitoan}[\cite{Tuyen_Toan_6}, 158., p. 33]
	Chứng minh tồn tại 1 bội số là bội của $19$ có tổng các chữ số bằng $19$.
\end{baitoan}

\begin{baitoan}[\cite{Tuyen_Toan_6}, 159., p. 33]
	Cho 3 số nguyên tố lớn hơn $3$. Chứng minh tồn tại 2 số có tổng hoặc hiệu chia hết cho $12$.
\end{baitoan}

\begin{baitoan}[\cite{Tuyen_Toan_6}, 160., p. 33]
	Chứng minh trong 3 số tự nhiên bất kỳ luôn chọn được 2 số có tổng chia hết cho $2$.
\end{baitoan}

\begin{baitoan}[\cite{Tuyen_Toan_6}, 161., p. 33]
	Chứng minh trong 7 số tự nhiên bất kỳ ta luôn chọn được 4 số có tổng chia hết cho $4$.
\end{baitoan}

\begin{baitoan}[\cite{Tuyen_Toan_6}, 162., p. 33]
	Cho 5 số tự nhiên bất kỳ, chứng minh luôn chọn được 3 số có tổng chia hết cho $3$.
\end{baitoan}

\begin{baitoan}[\cite{Tuyen_Toan_6}, 163., p. 33]
	Cho 5 số tự nhiên lẻ bất kỳ, chứng minh luôn chọn được 4 số có tổng chia hết cho $4$.
\end{baitoan}

\begin{baitoan}[\cite{Tuyen_Toan_6}, 164., p. 33]
	Viết 6 số tự nhiên vào 6 mặt của 1 con xúc xắc. Chứng minh khi ta gieo mặt xúc xắc xuống mặt bàn thì trong 5 mặt có thể nhìn thấy bao giờ cũng tìm được 1 hay nhiều mặt để tổng các số trên mặt đó chia hết cho $5$.
\end{baitoan}

%------------------------------------------------------------------------------%

\section{Miscellaneous}

\begin{baitoan}[\cite{Tuyen_Toan_6}, VD36, p. 33]
	Chứng minh tích các ước của $50$ là $50^3$.
\end{baitoan}

\begin{baitoan}[\cite{Tuyen_Toan_6}, VD3, p. 33]
	1 bệnh nhân được bác sĩ chỉ định nhỏ thuốc đau mắt, $4$ giờ 1 lần \& nhỏ tai $6$ giờ 1 lần. Nếu bệnh nhân này vừa nhỏ thuốc đau mắt vừa nhỏ tai vào lúc {\rm6:00} hằng ngày thì ít nhất đến mấy giờ bệnh nhân này lại vừa nhỏ mắt vừa nhỏ tai?
\end{baitoan}

\begin{baitoan}[\cite{Tuyen_Toan_6}, 165., p. 34]
	Tính hợp lý: (a) $\underbrace{19 + 19 + \cdots + 19}_{23} + \underbrace{77 + 77 + \cdots + 77}_{19}$. (b) $1000!(456\cdot789789 - 789\cdot456456)$.
\end{baitoan}

\begin{baitoan}[\cite{Tuyen_Toan_6}, 166., p. 34]
	Tìm $x\in\mathbb{N}$ thỏa: (a) $x + (x + 1) + (x + 2) + \cdots + (x + 30) = 1240$. (b) $1 + 2 + \cdots + x = 210$.
\end{baitoan}

\begin{baitoan}[\cite{Tuyen_Toan_6}, 167., p. 34]
	Chứng minh: (a) $10^n + 5^3\divby9$. (b) $43^{43} - 17^{17}\divby10$. (c) $A = \underbrace{5\ldots5}_{2n}$, $A\divby11$ nhưng $A\not{\divby}\ 125$.
\end{baitoan}

\begin{baitoan}[\cite{Tuyen_Toan_6}, 168., p. 34]
	Tìm $a\in\mathbb{N}$ nhỏ nhất sao cho $a$ chia cho $17$ dư $5$, $a$ chia cho $19$ dư $12$.
\end{baitoan}

\begin{baitoan}[\cite{Tuyen_Toan_6}, 169., p. 34]
	Tìm $a\in\mathbb{N}$ biết $355$ chia cho $a$ dư $13$ \& số $836$ chia cho $a$ dư $8$.
\end{baitoan}

\begin{baitoan}[\cite{Tuyen_Toan_6}, 170., p. 34]
	$a\in\mathbb{N}$ chia cho $7$ dư $5$, chia cho $13$ dư $4$. Tìm số dư khi $a$ chia cho $91$.
\end{baitoan}

\begin{baitoan}[\cite{Tuyen_Toan_6}, 171., p. 34]
	Biết ngày 1.2.2023 là thứ 4. (a) Hỏi ngày 1.3.2023, 1.4.2023 là thứ mấy? (b) Ngày 1.2.2024 là thứ mấy?
\end{baitoan}

\begin{baitoan}[\cite{Tuyen_Toan_6}, 172., p. 34]
	Tính tổng: (a) $A = \sum_{i=0}^{20} 2^i = 1 + 2 + 2^2 + \cdots + 2^{20}$. (b) $B = 4 + 4^3 + 4^5 + \cdots + 4^{49}$.
\end{baitoan}

\begin{baitoan}[\cite{Tuyen_Toan_6}, 173., p. 34]
	Cho $A = \sum_{i=1}^{24} 4^i = 4 + 4^2 + \cdots + 4^{24}$. Chứng minh $A\divby20,A\divby21,A\divby420$.
\end{baitoan}

\begin{baitoan}[\cite{Tuyen_Toan_6}, 174., p. 34]
	Cho $n = 29k$ với $k\in\mathbb{N}$. Với giá trị nào của $k$ thì $n$ là: (a) Số nguyên tố. (b) Hợp số. (c) Không phải là số nguyên tố cũng không phải là hợp số?
\end{baitoan}

\begin{baitoan}[\cite{Tuyen_Toan_6}, 175., p. 34]
	Cho $a$ là 1 hợp số, khi phân tích ra thừa số nguyên tố chỉ chứa 2 thừa số nguyên tố khác nhau là $p_1,p_2$. Biết $a^3$ có tất cả $40$ ước, tính số ước của $a^2$.
\end{baitoan}

\begin{baitoan}[\cite{Tuyen_Toan_6}, 176., p. 34]
	Cho 3 số $12,18,27$. (a) Tìm số lớn nhất có 3 chữ số chia hết cho 3 số đó. (b) Tìm số nhỏ nhất có 4 chữ số chia cho mỗi số đó đều dư $1$. (c) Tìm số nhỏ nhất có 4 chữ số chia cho $12$ dư $10$, chia cho $18$ dư $16$, chia cho $27$ dư $25$.
\end{baitoan}

%------------------------------------------------------------------------------%

\printbibliography[heading=bibintoc]

\end{document}