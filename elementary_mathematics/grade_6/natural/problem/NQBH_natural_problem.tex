\documentclass{article}
\usepackage[backend=biber,natbib=true,style=alphabetic,maxbibnames=50]{biblatex}
\addbibresource{/home/nqbh/reference/bib.bib}
\usepackage[utf8]{vietnam}
\usepackage{tocloft}
\renewcommand{\cftsecleader}{\cftdotfill{\cftdotsep}}
\usepackage[colorlinks=true,linkcolor=blue,urlcolor=red,citecolor=magenta]{hyperref}
\usepackage{amsmath,amssymb,amsthm,float,graphicx,mathtools,tipa}
\usepackage{enumitem}
\setlist{leftmargin=4mm}
\allowdisplaybreaks
\newtheorem{assumption}{Assumption}
\newtheorem{baitoan}{}
\newtheorem{cauhoi}{Câu hỏi}
\newtheorem{conjecture}{Conjecture}
\newtheorem{corollary}{Corollary}
\newtheorem{dangtoan}{Dạng toán}
\newtheorem{definition}{Definition}
\newtheorem{dinhly}{Định lý}
\newtheorem{dinhnghia}{Định nghĩa}
\newtheorem{example}{Example}
\newtheorem{ghichu}{Ghi chú}
\newtheorem{hequa}{Hệ quả}
\newtheorem{hypothesis}{Hypothesis}
\newtheorem{lemma}{Lemma}
\newtheorem{luuy}{Lưu ý}
\newtheorem{nhanxet}{Nhận xét}
\newtheorem{notation}{Notation}
\newtheorem{note}{Note}
\newtheorem{principle}{Principle}
\newtheorem{problem}{Problem}
\newtheorem{proposition}{Proposition}
\newtheorem{question}{Question}
\newtheorem{remark}{Remark}
\newtheorem{theorem}{Theorem}
\newtheorem{vidu}{Ví dụ}
\usepackage[left=1cm,right=1cm,top=5mm,bottom=5mm,footskip=4mm]{geometry}
\def\labelitemii{$\circ$}
\DeclareRobustCommand{\divby}{%
	\mathrel{\vbox{\baselineskip.65ex\lineskiplimit0pt\hbox{.}\hbox{.}\hbox{.}}}%
}

\title{Problem: Natural -- Bài Tập: Số Tự Nhiên $\mathbb{N}$}
\author{Nguyễn Quản Bá Hồng\footnote{A Scientist {\it\&} Creative Artist Wannabe. E-mail: {\tt nguyenquanbahong@gmail.com}. Bến Tre City, Việt Nam.}}
\date{\today}

\begin{document}
\maketitle
\begin{abstract}
	This text is a part of the series {\it Some Topics in Elementary STEM \& Beyond}:
	
	{\sc url}: \url{https://nqbh.github.io/elementary_STEM}.
	
	Latest version:
	\begin{itemize}
		\item {\it Problem: Natural -- Bài Tập: Số Tự Nhiên $\mathbb{N}$}.
		
		PDF: {\sc url}: \url{https://github.com/NQBH/elementary_STEM_beyond/blob/main/elementary_mathematics/grade_6/natural/problem/NQBH_natural_problem.pdf}.
		
		\TeX: {\sc url}: \url{https://github.com/NQBH/elementary_STEM_beyond/blob/main/elementary_mathematics/grade_6/natural/problem/NQBH_natural_problem.tex}.
		\item {\it Problem \& Solution: Natural -- Bài Tập \& Lời Giải: Số Tự Nhiên $\mathbb{N}$}.
		
		PDF: {\sc url}: \url{https://github.com/NQBH/elementary_STEM_beyond/blob/main/elementary_mathematics/grade_6/natural/problem/NQBH_natural_solution.pdf}.
		
		\TeX: {\sc url}: \url{https://github.com/NQBH/elementary_STEM_beyond/blob/main/elementary_mathematics/grade_6/natural/problem/NQBH_natural_solution.tex}.
	\end{itemize}
\end{abstract}
\tableofcontents

%------------------------------------------------------------------------------%

\section{Set -- Tập Hợp}
\cite[\S1, pp. 6--7]{SBT_Toan_6_Canh_Dieu_tap_1}: 1. 2. 3. 4. 5. 6. 7. 8.

\begin{baitoan}[\cite{Tuyen_Toan_6}, VD1, p. 4]
	Cho 2 tập hợp: $A = \{6;7;8;9;10\}$, $B = \{x;9;7;10;y\}$. (a) Viết tập hợp $A$ bằng cách chỉ ra tính chất đặc trưng cho các phần tử của nó. (b) Điền $\in,\notin$: $9\square A$, $x\square A$, $y\square B$. (c) Tìm $x,y$ để $A = B$.
\end{baitoan}

\begin{baitoan}[\cite{Tuyen_Toan_6}, 2., p. 5]
	(a) Viết tập hợp $M$ các chữ cái của chữ ``NGANG''. (b) Với tất cả các phần tử của tập hợp $M$, viết thành 1 chữ thuộc loại danh từ (không sử dụng thêm dấu).
\end{baitoan}

\begin{baitoan}[\cite{Tuyen_Toan_6}, 4., p. 5]
	Cho tập hợp $A = \{a;b\}$, $B = \{1;2;3\}$. Viết tất cả các tập hợp có $3$ phần tử trong đó $1$ phần tử thuộc tập hợp $A$, $2$ phần tử thuộc tập hợp $B$.
\end{baitoan}

\begin{baitoan}[\cite{Tuyen_Toan_6}, 5., p. 5]
	Cho các tập hợp: (a) $P$ là tập hợp các số tự nhiên $x$ mà $x + 3\le10$, $Q$ là tập hợp các số tự nhiên $x$ mà $x\cdot3 = 5$, $R$ là tập hợp các số tự nhiên $x$ mà $x\cdot3 = 0$, $S$ là tập hợp các số tự nhiên $x$ mà $x\cdot3\le24$. (a) Tập hợp nào là tập hợp rỗng? (b) Tập hợp nào có đúng 1 phần tử? (c) 2 tập hợp nào bằng nhau?
\end{baitoan}

\begin{baitoan}
	Viết tập hợp: (a) Tập các màu sắc của cầu vồng. (b) Tập hợp các huyện của tỉnh Bến Tre. (c) Tập hợp các châu lục trên Trái Đất. (d) Tập hợp các hành tinh trong Hệ Mặt Trời.
\end{baitoan}

\begin{baitoan}[$\mathbb{N}^\star\subset\mathbb{N}\subset\mathbb{Z}\subset\mathbb{Q}\subset\mathbb{R}\subset\mathbb{C}$]
	Viết tập hợp theo nhiều cách nhất có thể: (a) Tập hợp các số tự nhiên. (b) Tập hợp các số nguyên dương, i.e., tập hợp các số tự nhiên khác $0$. (c) Tập hợp các số nguyên. (d) Tập hợp các số nguyên âm. (e) Tập hợp các số nguyên không âm. (f) Tập hợp các số nguyên không dương. (g) Tập hợp các phân số, i.e., tập hợp các số hữu tỷ. (h) Tập hợp các số hữu tỷ dương. (i) Tập hợp các số hữu tỷ âm. (j) Tập hợp các số thực. (k) Tập hợp các số thực âm. (l) Tập hợp các số thực dương. (m) Tập hợp các số thực không âm. (n) Tập hợp các thực không dương. (o) Tập hợp các số vô tỷ. (p) Tập hợp các số vô tỷ dương. (q) Tập hợp các số vô tỷ âm. (r) Tập hợp các số phức.
\end{baitoan}

%------------------------------------------------------------------------------%

\section{Set $\mathbb{N}$ of Natural Numbers -- Tập hợp $\mathbb{N}$ Các Số Tự Nhiên}
\cite[\S2, pp. 8--9]{SBT_Toan_6_Canh_Dieu_tap_1}: 9. 10. 11. 12. 13. 14.

\begin{baitoan}[\cite{Binh_boi_duong_Toan_6_tap_1}, VD1, p. 8]
	Nhiệt độ thay đổi theo giờ trong 1 ngày tháng 8 \& 1 ngày tháng 9 lần lượt được ghi lại:
	\begin{table}[H]
		\centering
		\begin{tabular}{|c|c|c|c|c|c|}
			\hline
			Thời điểm trong ngày & 6:00 & 9:00 & 12:00 & 14:00 & 17:00 \\
			\hline
			Nhiệt độ (${}^\circ$) & 24 & 25 & 27 & 26 & 24 \\
			\hline
		\end{tabular}
	\end{table}
	\begin{table}[H]
		\centering
		\begin{tabular}{|c|c|c|c|c|c|}
			\hline
			Thời điểm trong ngày & 6:00 & 9:00 & 12:00 & 14:00 & 17:00 \\
			\hline
			Nhiệt độ (${}^\circ$) & 23 & 24 & 26 & 25 & 22 \\
			\hline
		\end{tabular}
	\end{table}
	\noindent(a) Viết 2 tập hợp A, B gồm các giá trị nhiệt độ của mỗi bảng trên. (b) Viết tập hợp C gồm các phần tử thuộc tập hợp A mà không thuộc tập hợp B. (c) Viết tập hợp D gồm các phần tử thuộc tập hợp B mà không thuộc tập hợp A. (d) Viết tập hợp E gồm các phần tử thuộc cả 2 tập hợp A \& B. (e) Viết tập hợp F gồm các phần tử thuộc tập hợp A hoặc thuộc tập hợp B.
\end{baitoan}

\begin{baitoan}[\cite{Binh_boi_duong_Toan_6_tap_1}, VD2, p. 9]
	Cho A là tập hợp các số tự nhiên chẵn có 3 chữ số. Hỏi A có bao nhiêu phần tử?
\end{baitoan}

\begin{baitoan}[\cite{Binh_boi_duong_Toan_6_tap_1}, VD3, p. 9]
	Cho A là tập hợp các số tự nhiên lẻ lớn hơn $3$ \& không lớn hơn $99$. (a) Viết tập hợp A bằng cách chỉ ra tính chất đặc trưng của các phần tử. (b) Giả sử các phần tử của A được viết theo giá trị tăng dần. Tìm phần tử thứ $23$ của A.
\end{baitoan}

\begin{baitoan}[\cite{Binh_boi_duong_Toan_6_tap_1}, VD4, p. 10]
	Để đánh số các trang sách (bắt đầu từ trang $1$) của 1 cuốn sách của $2015$ trang thì cần dùng bao nhiêu chữ số?
\end{baitoan}

\begin{baitoan}[\cite{Binh_boi_duong_Toan_6_tap_1}, VD5, p. 10]
	Biết người ta đã dùng đúng $6793$ chữ số để đánh số trang của 1 cuốn sách (bắt đầu từ trang $1$), hỏi cuốn sách đó có bao nhiêu trang?
\end{baitoan}

\begin{baitoan}[\cite{Binh_boi_duong_Toan_6_tap_1}, VD6, p. 11]
	Gọi A là tập hợp các số tự nhiên có 2 chữ số mà chữ số hàng chục lớn hơn chữ số hàng đơn vị; B là tập hợp các số tự nhiên có 2 chữ số mà chữ số hàng chục nhỏ hơn chữ số hàng đơn vị. So sánh số phần tử của 2 tập hợp A, B.
\end{baitoan}

\begin{baitoan}[\cite{Binh_boi_duong_Toan_6_tap_1}, VD7, p. 12]
	Tìm 1 số có 2 chữ số biết khi viết thêm chữ số $0$ vào giữa 2 chữ số của số đó thì được số mới gấp $6$ lần số đã cho.
\end{baitoan}

\begin{baitoan}[\cite{Binh_boi_duong_Toan_6_tap_1}, VD8, p. 12]
	Tìm số có 3 chữ số biết nếu viết thêm chữ số $1$ vào trước số đó thì được số mới gấp $9$ lần số ban đầu.	
\end{baitoan}

\begin{baitoan}[\cite{Binh_boi_duong_Toan_6_tap_1}, p. 13]
	Ngày mùng 2 tháng 9 năm $\overline{abcd}$, tại quảng trường Ba Đình lịch sử, Chủ tịch Hồ Chí Minh đã đọc bản tuyên ngôn độc lập khai sinh nước Việt Nam Dân chủ Cộng hòa (nay là nước Cộng hòa xã hội chủ nghĩa Việt Nam). Năm $\overline{abcd}$ là năm nào? Biết $a$ là phần tử nhỏ nhất trong tập hợp $ \mathbb{N}^\star$, $b$ là chữ số lớn nhất, $c,d$ là 2 số tự nhiên liên tiếp \& $c + d = b$.
\end{baitoan}

\begin{baitoan}[\cite{Binh_boi_duong_Toan_6_tap_1}, 1.1., p. 13]
	Cho tập hợp $A = \{1,3,5\}$. {\rm Đ{\tt/}S?} (a) $1\in A$. (b) $\{1\}\in A$. (c) $3\notin A$. (d) $5\notin A$.
\end{baitoan}

\begin{baitoan}[\cite{Binh_boi_duong_Toan_6_tap_1}, 1.2., p. 13]
	Cho 2 tập hợp: $A = \{1,2,3,4,5\}$, $B = \{3,5,7,9\}$. (a) Mỗi tập hợp trên có bao nhiêu phần tử? (b) Viết các tập hợp trên bằng cách chỉ ra tính chất đặc trưng của các phần tử.
\end{baitoan}

\begin{baitoan}[\cite{Binh_boi_duong_Toan_6_tap_1}, 1.3., p. 13]
	Viết các tập hợp sau \& cho biết mỗi tập hợp đó có bao nhiêu phần tử. (a) Tập hợp A các số tự nhiên $x$ thỏa $12 - x = 5$. (b) Tập hợp B các số tự nhiên $y$ thỏa $7 + y = 21$. (c) Tập hợp C các số tự nhiên $z$ mà $z\cdot0 = 0$.
\end{baitoan}

\begin{baitoan}[\cite{Binh_boi_duong_Toan_6_tap_1}, 1.4., p. 14]
	Tính số phần tử của tập hợp: (a) $A = \{2,4,6,\ldots,98\}$. (b) $B = \{6,10,14,18,22,\ldots,70\}$.
\end{baitoan}

\begin{baitoan}[\cite{Binh_boi_duong_Toan_6_tap_1}, 1.5., p. 14]
	Cho dãy số $2,5,8,11,14,\ldots$. (a) Nêu quy luật của dãy số trên. (b) Viết tập hợp B gồm $5$ số hạng tiếp theo của dãy số trên. (c) Tính tổng $100$ số hạng đầu tiên của dãy số.
\end{baitoan}

\begin{baitoan}[\cite{Binh_boi_duong_Toan_6_tap_1}, 1.6., p. 14]
	Viết lại mỗi tập hợp bằng cách liệt kê các phần tử: $A = \{x\in\mathbb{N}|{x\not{\divby}\ 2},\,30 < x < 50\}$. (b) $B = \{x\in\mathbb{N}|x\divby5,\,x\divby2,\,10 < x\le90\}$.
\end{baitoan}

\begin{baitoan}[\cite{Binh_boi_duong_Toan_6_tap_1}, 1.7., p. 13]
	Thực hiện yêu cầu phòng chống dịch Covid-19, tại 1 trường trung học, vào đầu giờ sáng trước khi vào lớp, các học sinh đều được yêu cầu khử khuẩn tay \& đo thân nhiệt. Kết quả đo thân nhiệt tại lớp 6H:
	\begin{table}[H]
		\centering
		\begin{tabular}{|c|l|}
			\hline
			Tổ & Thân nhiệt (${}^\circ$C) \\
			\hline
			1 & 36, 36.5, 37, 36, 35.5, 37, 36.5, 36, 35.5, 36, 36.5, 37 \\
			\hline
			2 & 36.5, 37, 35.5, 36, 36, 35.5, 37, 36, 36.5, 36, 35.5, 36 \\
			\hline
			3 & 37, 36.5, 36, 35.5, 35, 36, 35.5, 35, 36, 35, 36.5, 36 \\
			\hline
			4 & 36, 35.5, 36, 36, 35.5, 36.5, 35, 36.5, 36, 35.5, 36.5, 36 \\
			\hline
		\end{tabular}
	\end{table}
	\noindent(a) Gọi A, B, C, D lần lượt là tập hợp gồm các phần tử là thân nhiệt của các bạn tổ 1, tổ 2, tổ 3, tổ 4. Viết các tập hợp A, B, C, D theo cách liệt kê các phần tử (mỗi phần tử chỉ liệt kê 1 lần). (b) Tìm cặp tập hợp bằng nhau trong các tập hợp A, B, C, D. Dùng ký hiệu ``$=$'' để thể hiện mối quan hệ đó.
\end{baitoan}

\begin{baitoan}[\cite{Binh_boi_duong_Toan_6_tap_1}, 1.8., p. 14]
	Với 3 chữ số La Mã I, V, X có thể viết được bao nhiêu số La Mã mà mỗi chữ số chỉ xuất hiện 1 lần? Số nhỏ nhất là số nào? Số lớn nhất là số nào?
\end{baitoan}

\begin{baitoan}[\cite{Binh_boi_duong_Toan_6_tap_1}, 1.9., p. 14]
	Tìm số có 3 chữ số, biết nếu viết các chữ số theo thứ tự ngược lại thì được số mới nhỏ hơn số ban đầu $792$ đơn vị.
\end{baitoan}

\begin{baitoan}[\cite{Binh_boi_duong_Toan_6_tap_1}, 1.10., p. 14]
	Có bao nhiêu số tự nhiên có 2 chữ số mà: (a) Trong số đó có ít nhất 1 chữ số $9$? (b) Trong số đó chữ số hàng chục bé hơn chữ số hàng đơn vị? (c) Trong số đó chữ số hàng chục gấp đôi chữ số hàng đơn vị?
\end{baitoan}

\begin{baitoan}[\cite{Binh_boi_duong_Toan_6_tap_1}, 1.11., p. 14]
	Cho 1 số có 2 chữ số. Nếu viết thêm chữ số 2 vào bên trái \& bên phải số đó ta được số mới gấp $32$ lần số đã cho. Tìm số đã cho.
\end{baitoan}

\begin{baitoan}[\cite{Binh_boi_duong_Toan_6_tap_1}, 1.12., p. 14]
	Mẹ mua cho Hà 1 quyển sổ tay có $358$ trang. Để tiện theo dõi, Hà đánh số trang từ $1$ đến $358$. Hỏi Hà đã phải viết bao nhiêu chữ số để đánh số trang hết cuốn sổ tay đó.
\end{baitoan}

\begin{baitoan}[\cite{Binh_boi_duong_Toan_6_tap_1}, 1.13., p. 14]
	Viết liền nhau các số tự nhiên $123456789101112\ldots$ (a) Hỏi các chữ số hàng đơn vị của các số $49,217,2401$ đứng ở vị trí thứ bao nhiêu kể từ trái sang phải? (b) Chữ số viết ở vị trí thứ $427$ là chữ số nào?
\end{baitoan}

\begin{baitoan}[\cite{Binh_boi_duong_Toan_6_tap_1}, 1.14., p. 15]
	Cho 4 chữ số $a,b,c,d$ đôi một khác nhau \& khác $0$. Tập hợp các số tự nhiên có 3 chữ số gồm 3 trong 4 chữ số $a,b,c,d$ có bao nhiêu phần tử?
\end{baitoan}

\begin{baitoan}[\cite{Binh_boi_duong_Toan_6_tap_1}, 1.15., p. 15]
	Mỗi tập hợp sau đây có bao nhiêu phần tử? (a) Tập hợp các số có 2 chữ số. (b) Tập hợp các số có 2 chữ số được lập nên từ 2 số khác nhau. (c) Tập hợp các số có 3 chữ số được lập nên từ 3 chữ số đôi một khác nhau.
\end{baitoan}

\begin{baitoan}[\cite{Binh_boi_duong_Toan_6_tap_1}, 1.16., p. 15]
	Tổng kết đợt thi đua, lớp 6A có $35$ bạn được 1 điểm $10$ trở lên, $21$ bạn được từ 2 điểm $10$ trở lên, $18$ bạn được 3 điểm $10$ trở lên, $5$ bạn được $4$ điểm $10$. Biết không có ai được trên $4$ điểm $10$, hỏi trong đợt thi đua đó lớp 6A có bao nhiêu điểm $10$?
\end{baitoan}

\begin{baitoan}[\cite{Binh_boi_duong_Toan_6_tap_1}, 1.17., p. 15]
	Tìm số tự nhiên có 4 chữ số, chữ số hàng đơn vị là $9$. Nếu chuyển chữ số hàng đơn vị lên đầu thì được 1 số mới lớn hơn số đã cho $2889$ đơn vị.
\end{baitoan}

\begin{baitoan}[\cite{Binh_boi_duong_Toan_6_tap_1}, 1.18., p. 15]
	Hiệu của 2 số tự nhiên là $53$. Chữ số hàng đơn vị của số bị trừ là $8$. Nếu bỏ chữ số hàng đơn vị của số bị trừ ta được số trừ. Tìm 2 số đó.
\end{baitoan}

\begin{baitoan}[\cite{Binh_boi_duong_Toan_6_tap_1}, 1.19., p. 15]
	Tìm 1 số có 5 chữ số biết nếu viết chữ số $7$ đằng trước số đó thì được số lớn gấp $5$ lần số có được bằng cách viết thêm chữ số $7$ vào đằng sau số đó.
\end{baitoan}

\begin{baitoan}[\cite{Binh_boi_duong_Toan_6_tap_1}, 1.20., p. 15]
	1 số gồm 3 chữ số tận cùng là chữ số $9$, nếu chuyển chữ số $9$ đó lên đầu thì được 1 số mới mà khi chia cho số cũ thì được thương là $3$ dư $61$. Tìm số đó.
\end{baitoan}

\begin{baitoan}[\cite{Binh_boi_duong_Toan_6_tap_1}, p. 15]
	Trong 1 lớp học, tất cả học sinh nam đều tham gia vào các câu lạc bộ thể thao: bóng đá, bóng chuyền, cầu lông. Biết có $7$ học sinh tham gia câu lạc bộ bóng đá, $6$ học sinh tham gia câu lạc bộ bóng chuyền, $5$ học sinh tham gia câu lạc bộ cầu lông; trong số đó có $4$ học sinh tham gia cả 2 câu lạc bộ bóng đá \& bóng chuyền, $3$ học sinh tham gia cả 2 câu lạc bộ bóng đá \& cầu lông, $2$ học sinh tham gia cả 2 câu lạc bộ bóng chuyền \& cầu lông, $1$ học sinh tham gia cả 3 câu lạc bộ. Hỏi lớp đó có bao nhiêu học sinh nam?
\end{baitoan}

\begin{baitoan}[\cite{Tuyen_Toan_6}, VD2, p. 6]
	Phố Hàng Ngang là 1 trong các phố cổ của Hà Nội. Các nhà được đánh số liên tục, dãy lẻ $1,3,5,7,\ldots,61$; dãy chẵn $2,4,6,\ldots,64$. (a) Bên số nhà chẵn, trong 1 phòng gác nhỏ, chủ tịch Hồ Chí Minh đã khởi thảo bản Tuyên Ngôn Độc Lập khai sinh cho nước Việt Nam Dân Chủ Cộng Hòa. Ngôi nhà có căn phòng đó là nhà thứ $24$ kể từ đầu phố (số $2$). Hỏi ngôi nhà này có số nào? (b) Bên số nhà lẻ chữ số nào được dùng nhiều nhất? Chữ số nào chưa được dùng đến? (c) Phải dùng tất cả bao nhiêu chữ số để ghi số nhà của phố này?
\end{baitoan}

\begin{baitoan}[\cite{Tuyen_Toan_6}, 6., p. 6]
	Viết tập hợp $4$ số tự nhiên liên tiếp lớn hơn $94$ nhưng không quá $100$.
\end{baitoan}

\begin{baitoan}[\cite{Tuyen_Toan_6}, 7., p. 6]
	(a) Có bao nhiêu số tự nhiên nhỏ hơn $20$? (b) Có bao nhiêu số tự nhiên nhỏ hơn $n\in\mathbb{N}$? (c) Có bao nhiêu số tự nhiên chẵn nhỏ hơn $n\in\mathbb{N}$? (d) Có bao nhiêu số tự nhiên lẻ nhỏ hơn $n\in\mathbb{N}$?
\end{baitoan}

\begin{baitoan}[\cite{Tuyen_Toan_6}, 8., p. 7]
	(a) Có bao nhiêu số có $4$ chữ số mà cả $4$ chữ số đều giống nhau? (b) Có bao nhiêu số có $4$ chữ số? (c) Có bao nhiêu số có $n$ chữ số, với $n\in\mathbb{N}$?
\end{baitoan}

\begin{baitoan}[\cite{Tuyen_Toan_6}, 9., p. 7]
	Đèn hướng dẫn giao thông liên tục sáng màu xanh hoặc đỏ kế tiếp nhau. Bảng hiện số của đèn có 2 chữ số liên tục thay đổi theo từng giây. Hỏi trong 1 phút xe bị dừng vì đèn đỏ thì đèn có: (a) Bao nhiêu lần thay đổi các số? (b) Bao nhiêu lần thay đổi các chữ số?
\end{baitoan}

\begin{baitoan}[\cite{Tuyen_Toan_6}, 10., p. 7]
	Tìm $3$ số tự nhiên $a,b,c$ biết chúng thỏa mãn đồng thời 3 điều kiện: $a < b < c$, $101\le a\le103$, $101 < c < 104$.
\end{baitoan}

\begin{baitoan}[\cite{Tuyen_Toan_6}, 11., p. 7]
	Cho số $4321$. Viết thêm chữ số $9$ xen giữa các chữ số của nó để được 1 số: (a) Lớn nhất có thể được. (b) Nhỏ nhất có thể được.
\end{baitoan}

\begin{baitoan}[\cite{Tuyen_Toan_6}, 12., p. 7]
	Với $9$ que diêm, sắp xếp thành 1 số La Mã: (a) Có giá trị lớn nhất. (b) Có giá trị nhỏ nhất.
\end{baitoan}

\begin{baitoan}[\cite{Tuyen_Toan_6}, 13., p. 7]
	Có $13$ que diêm sắp xếp như sau: $\rm XII - V = VII$. (a) Đẳng thức trên đúng hay sai? (b) Đổi chỗ chỉ 1 que diêm để được 1 đẳng thức đúng.
\end{baitoan}

\begin{baitoan}[\cite{Binh_Toan_6_tap_1}, VD1, p. 4]
	Viết các tập hợp sau rồi tìm số phần tử của mỗi tập hợp đó: (a) Tập hợp $A$ các số tự nhiên $x$ mà $8:x = 2$. (b) Tập hợp $B$ các số tự nhiên $x$ mà $x + 3 < 5$. (c) Tập hợp $C$ các số tự nhiên $x$ mà $x - 2 = x + 2$. (d) Tập hợp $D$ các số tự nhiên $x$ mà $x:2 = x:4$. (e) Tập hợp $E$ các số tự nhiên $x$ mà $x + 0 = x$.
\end{baitoan}

\begin{baitoan}[\cite{Binh_Toan_6_tap_1}, VD2, p. 5]
	Viết các tập hợp sau bằng cách liệt kê các phần tử của nó: (a) Tập hợp $A$ các số tự nhiên có 2 chữ số, trong đó chữ số hàng chục lớn hơn chữ số hàng đơn vị là $2$. (b) Tập hợp $B$ các số tự nhiên có 3 chữ số mà tổng các chữ số bằng $3$.
\end{baitoan}

\begin{baitoan}[\cite{Binh_Toan_6_tap_1}, VD3, p. 5]
	Tìm số tự nhiên có 5 chữ số, biết nếu viết thêm chữ số $2$ vào đằng sau số đó thì được số lớn gấp 3 lần số có được bằng cách viết thêm chữ số $2$ vào đằng trước số đó.
\end{baitoan}

\begin{baitoan}[\cite{Binh_Toan_6_tap_1}, Mở rộng VD3, p. 5]
	Tìm số tự nhiên nhỏ nhất có chữ số đầu tiên ở bên trái là $2$, khi chuyển chữ số $2$ này xuống cuối cùng thì số đó tăng gấp 3 lần.\hfill{\sf Ans:} $285714$.
\end{baitoan}

\begin{baitoan}[\cite{Binh_Toan_6_tap_1}, Mở rộng VD3, p. 6]
	Tìm số tự nhiên có 5 chữ số, biết nếu viết thêm 1 chữ số vào đằng sau số đó thì được số lớn gấp 3 lần số có được nếu viết thêm chính chữ số ấy vào đằng trước số đó.\hfill{\sf Ans:} $85714$.
\end{baitoan}

\begin{baitoan}[\cite{Binh_Toan_6_tap_1}, 2., p. 6]
	Xác định các tập hợp sau bằng cách chỉ ra tính chất đặc trưng của các phần tử thuộc tập hợp đó: (a) $A = \{1,3,5,7,\ldots,49\}$. (b) $B = \{11,22,33,44,\ldots,99\}$. (c) $C = \{\mbox{tháng } 1,\mbox{tháng } 3,\mbox{tháng } 5,\mbox{tháng } 7,\mbox{tháng } 8,\mbox{tháng } 10,\mbox{tháng } 12\}$.
\end{baitoan}

\begin{baitoan}[\cite{Binh_Toan_6_tap_1}, 3., p. 6]
	Tìm tập hợp các số tự nhiên $x$ sao cho: (a) $x + 3 = 4$. (b) $8 - x = 5$. (c) $x:2 = 0$. (d) $0:x = 0$. (e) $5x = 12$.
\end{baitoan}

\begin{baitoan}[\cite{Binh_Toan_6_tap_1}, 4., p. 6]
	Tìm $a,b\in\mathbb{N}$ sao cho $12 < a < b < 16$.
\end{baitoan}

\begin{baitoan}[\cite{Binh_Toan_6_tap_1}, 5., p. 6]
	Viết các số tự nhiên có 4 chữ số trong đó có 2 chữ số $3$, 1 chữ số $2$, 1 chữ số $1$.
\end{baitoan}

\begin{baitoan}[\cite{Binh_Toan_6_tap_1}, 6., p. 6]
	Với cả 2 chữ số I \& X, viết được bao nhiêu số La Mã? (Mỗi chữ số có thể viết nhiều lần, nhưng không viết liên tiếp quá 3 lần).
\end{baitoan}

\begin{baitoan}[\cite{Binh_Toan_6_tap_1}, 7., pp. 6--7]
	(a) Dùng 3 que diêm, xếp được các số La Mã nào? (b) Để viết các số La Mã từ 4000 trở lên, e.g. số 19520, người ta viết XIXmDXX (chữ m biểu thị \emph{1 nghìn}, m là chữ đầu của từ \emph{mille}, tiếng Latin là 1 nghìn). Hãy viết các số sau bằng chữ số La Mã: 7203, 121512.
\end{baitoan}

\begin{baitoan}[\cite{Binh_Toan_6_tap_1}, 8., p. 7]
	Tìm số tự nhiên có tận cùng bằng $3$, biết rằng nếu xóa chữ số hàng đơn vị thì số đó giảm đi $1992$ đơn vị.
\end{baitoan}

\begin{baitoan}[\cite{Binh_Toan_6_tap_1}, 9., p. 7]
	Tìm số tự nhiên có 6 chữ số, biết rằng chữ số hàng đơn vị là $4$ \& nếu chuyển chữ số đó lên hàng đầu tiên thì số đó tăng gấp 4 lần.
\end{baitoan}

\begin{baitoan}[\cite{Binh_Toan_6_tap_1}, 10., p. 7]
	Cho 4 chữ số $a,b,c,d$ khác nhau \& khác $0$. Lập số tự nhiên lớn nhất \& số tự nhiên nhỏ nhất có 4 chữ số gồm cả 4 chữ số ấy. Tổng của 2 số này bằng $11330$. Tìm tổng các chữ số $a + b + c + d$.
\end{baitoan}

\begin{baitoan}[\cite{Binh_Toan_6_tap_1}, 11., p. 7]
	Cho 3 chữ số $a,b,c$ sao cho $0 < a < b < c$. (a) Viết tập hợp $A$ các số tự nhiên có 3 chữ số gồm cả 3 chữ số $a,b,c$. (b) Biết tổng 2 số nhỏ nhất trong tập hợp $A$ bằng $488$. Tìm 3 chữ số $a,b,c$ nói trên.
\end{baitoan}

\begin{baitoan}[\cite{Binh_Toan_6_tap_1}, 12., p. 7]
	Tìm 3 chữ số khác nhau \& khác $0$, biết rằng nếu dùng cả 3 chữ số này lập thành các số tự nhiên có 3 chữ số thì 2 số lớn nhất có tổng bằng $1444$.
\end{baitoan}

\begin{baitoan}[Even vs. odd -- Chẵn vs. lẻ]
	Viết tập hợp theo nhiều cách nhất có thể: (a) Tập hợp các số tự nhiên chẵn. (b) Tập hợp các số tự nhiên lẻ. (c) Tập hợp các số nguyên dương chẵn. (d) Tập hợp các số nguyên dương lẻ. (e) Tập hợp các số nguyên chẵn. (f) Tập hợp các số nguyên lẻ.
\end{baitoan}

\begin{baitoan}
	Với $b,r$ là 2 số nguyên dương cho trước (i.e., số tự nhiên khác $0$), viết tập hợp theo 2 cách: (a) Tập hợp các số tự nhiên chia hết cho $b$. (b) Tập hợp các số tự nhiên chia cho $b$ dư $r$.
\end{baitoan}

\begin{baitoan}[Tập con của $\mathbb{N}$ chỉ bị chặn 1 phía]
	Cho $a$ là 1 số tự nhiên cho trước. Viết các tập hợp sau theo nhiều cách nhất có thể: (a) Tập hợp các số tự nhiên nhỏ hơn $a$. (b) Tập hợp các số tự nhiên lớn hơn $a$. (c) Tập hợp các số tự nhiên nhỏ hơn hoặc bằng $a$. (d) Tập hợp các số tự nhiên lớn hơn hoặc bằng $a$. (e) Tập hợp các số tự nhiên chẵn nhỏ hơn $a$. (f) Tập hợp các số tự nhiên chẵn lớn hơn $a$. (g) Tập hợp các số tự nhiên chẵn nhỏ hơn hoặc bằng $a$. (h) Tập hợp các số tự nhiên chẵn lớn hơn hoặc bằng $a$. (i) Tập hợp các số tự nhiên lẻ nhỏ hơn $a$. (j) Tập hợp các số tự nhiên lẻ lớn hơn $a$. (k) Tập hợp các số tự nhiên lẻ nhỏ hơn hoặc bằng $a$. (l) Tập hợp các số tự nhiên lẻ lớn hơn hoặc bằng $a$.
\end{baitoan}

\begin{baitoan}[Tập con của $\mathbb{N}$ bị chặn cả 2 phía]
	Với $a,b$ là 2 số tự nhiên cho trước. Viết các tập hợp sau theo nhiều cách nhất có thể: (a) Tập hợp các số tự nhiên lớn hơn $a$ \& nhỏ hơn $b$. (b) Tập hợp các số tự nhiên lớn hơn hoặc bằng $a$ \& nhỏ hơn $b$. (c) Tập hợp các số tự nhiên lớn hơn $a$ \& nhỏ hơn hoặc bằng $b$. (d) Tập hợp các số tự nhiên lớn hơn hoặc bằng $a$ \& nhỏ hơn hoặc bằng $b$. (e) Tập hợp các số tự nhiên chẵn lớn hơn $a$ \& nhỏ hơn $b$. (f) Tập hợp các số tự nhiên chẵn lớn hơn hoặc bằng $a$ \& nhỏ hơn $b$. (g) Tập hợp các số tự nhiên chẵn lớn hơn $a$ \& nhỏ hơn hoặc bằng $b$. (h) Tập hợp các số tự nhiên chẵn lớn hơn hoặc bằng $a$ \& nhỏ hơn hoặc bằng $b$. (i) Tập hợp các số tự nhiên lẻ lớn hơn $a$ \& nhỏ hơn $b$. (j) Tập hợp các số tự nhiên lẻ lớn hơn hoặc bằng $a$ \& nhỏ hơn $b$. (k) Tập hợp các số tự nhiên lẻ lớn hơn $a$ \& nhỏ hơn hoặc bằng $b$. (l) Tập hợp các số tự nhiên lẻ lớn hơn hoặc bằng $a$ \& nhỏ hơn hoặc bằng $b$.
\end{baitoan}

\begin{baitoan}
	Viết tập hợp theo nhiều cách nhất có thể: (a) Tập hợp các số tự nhiên có 1 chữ số. (b) Tập hợp các số tự nhiên có 2 chữ số. (c) Tập hợp các số tự nhiên có 3 chữ số. (d) Tập hợp các số tự nhiên có $n$ chữ số, với $n$ là 1 số tự nhiên cho trước.
\end{baitoan}

\begin{baitoan}
	Viết biễu diễn thập phân của các số tự nhiên có: (a) 1 chữ số. (b) 2 chữ số. (c) 3 chữ số. (d) 4 chữ số. (e) 5 chữ số. (f) 6 chữ số. (g) 7 chữ số. (h) (d) 8 chữ số. (i) 9 chữ số. (j) (d) 10 chữ số. (k) $n$ chữ số, với $n\in\mathbb{N}$ cho trước. 
\end{baitoan}

\begin{baitoan}
	Chứng minh: (a) Trong 2 số tự nhiên có số chữ số khác nhau: Số nào có nhiều chữ số hơn thì lớn hơn, số nào có ít chữ số hơn thì nhỏ hơn. (b) Trong 2 số tự nhiên có cùng số chữ số, nếu trong cặp chữ số khác nhau đầu tiên từ trái sang phải, số nào có chữ số tương ứng trong cặp đó lớn hơn thì lớn hơn.
\end{baitoan}

%------------------------------------------------------------------------------%

\section{Basic Calculus on $\mathbb{N}$ -- Phép $\pm,\cdot,:$ trên $\mathbb{N}$}
\cite[\S3, pp. 10--12]{SBT_Toan_6_Canh_Dieu_tap_1}: 15. 16. 17. 18. 19. 20. 21. 22. \cite[\S4, pp. 13--15]{SBT_Toan_6_Canh_Dieu_tap_1}: 23. 24. 25. 26. 27. 28. 29. 30. 31. 32. 33. 34. 35. 36.

\begin{baitoan}[\cite{Binh_boi_duong_Toan_6_tap_1}, H1, p. 17]
	(a) Viết đủ 6 số $1,2,3,4,5,6$ vào 3 đỉnh \& 3 trung điểm của 3 cạnh của 1 tam giác sao cho tổng 3 số trên mỗi cạnh bằng $10$. (b${}^\star$) Có bao nhiêu cách tất cả?
\end{baitoan}

\begin{baitoan}[\cite{Binh_boi_duong_Toan_6_tap_1}, H2, p. 17]
	{\rm Đ{\tt/}S?} Cho $a,b,c,m,n,p\in\mathbb{N}^\star$ thỏa $a + m = b + n = c + p = a + b + c$. (a) $m + n > p$. (b) $n + p < m$. (c) $p + m > n$. (d) $m + n + p = a + b + c$. (e) $m + n + p = 2(a + b + c)$. (f) $m,n,p$ là độ dài 3 cạnh của 1 tam giác.
\end{baitoan}

\begin{baitoan}[\cite{Binh_boi_duong_Toan_6_tap_1}, H3, p. 17]
	Tính $3^4:3 + 2^3:2^2$.
\end{baitoan}

\begin{baitoan}[\cite{Binh_boi_duong_Toan_6_tap_1}, H4, p. 17]
	{\rm Đ{\tt/}S?} (a) $(15 + 5)(4 + 1) = 20\cdot5 = 100$. (b) $5 + 20\cdot4 = 25\cdot4 = 100$.
\end{baitoan}

\begin{baitoan}[\cite{Binh_boi_duong_Toan_6_tap_1}, H5, p. 17]
	Tính: (a) Hiệu của 2 số lẻ mà giữa chúng có $10$ số chẵn. (b) Hiệu của 2 số lẻ mà giữa chúng có $10$ số lẻ. (c) Hiệu của 2 số chẵn mà giữa chúng có $5$ số chẵn. (d) Hiệu của 2 số chẵn mà giữa chúng có $5$ số lẻ.
\end{baitoan}

\begin{baitoan}[\cite{Binh_boi_duong_Toan_6_tap_1}, VD1, p. 17]
	Tính hợp lý: (a) $A = 27\cdot36 + 73\cdot99 + 27\cdot14 - 49\cdot73$. (b) $B = (4^5\cdot10\cdot5^6 + 25^5\cdot2^8):(2^8\cdot5^4 + 5^7\cdot2^5)$.
\end{baitoan}

\begin{baitoan}[\cite{Binh_boi_duong_Toan_6_tap_1}, VD2, p. 18]
	Egg \& Chicken cùng ra cửa hàng mua sách. Tổng số tiền ban đầu của 2 bạn là $78000$ đồng. Egg mua hết $32000$ đồng, Chicken mua hết $14000$ đồng. Khi đó số tiền còn lại của 2 bạn bằng nhau. Hỏi ban đầu mỗi bạn có bao nhiêu tiền?
\end{baitoan}

\begin{baitoan}[\cite{Binh_boi_duong_Toan_6_tap_1}, VD3, p. 18]
	So sánh: (a) $(4 + 5)^2$ \& $4^2 + 5^2$. (b) $2^{30}$ \& $3^{20}$.
\end{baitoan}

\begin{baitoan}[\cite{Binh_boi_duong_Toan_6_tap_1}, VD4, p. 19]
	Tế bào lớn lên đến 1 kích thước nhất định thì phân chia. Quá trình đó diễn ra như sau: Đầu tiên từ $1$ nhân hình thành $2$ nhân, tách xa nhau. Sau đó chất tế bào được phân chia, xuất hiện $1$ vách ngăn, ngăn đôi tế bào cũ thành $2$ tế bào con. Các tế bào con tiếp tục lớn lên cho đến khi bằng tế bào mẹ. Các tế bào này lại tiếp tục phân chia thành $4$, rồi thành $8,\ldots$ tế bào. Từ $1$ tế bào ban đầu, tìm số tế bào có được sau lần phân chia thứ $5$, thứ $8$, thứ $10$, thứ $n$.
\end{baitoan}

\begin{baitoan}[\cite{Binh_boi_duong_Toan_6_tap_1}, VD5, p. 19]
	Tìm $x\in\mathbb{N}$ thỏa: (a) $149 - (35:x + 3)\cdot17 = 13$. (b) $\overline{1x32} + \overline{7x8} + \overline{4x} = \overline{200x}$.
\end{baitoan}

\begin{baitoan}[\cite{Binh_boi_duong_Toan_6_tap_1}, VD6, p. 20]
	Tìm $x\in\mathbb{N}$ thỏa: (a) $(3x - 2)^3 = 2\cdot32$. (b) $5^{x+1} - 5^x = 500$.
\end{baitoan}

\begin{baitoan}[\cite{Binh_boi_duong_Toan_6_tap_1}, VD7, p. 20]
	Tìm các số mũ tự nhiên $n$ sao cho lũy thừa $3^n$ thỏa mãn điều kiện $25 < 3^n < 260$.
\end{baitoan}

\begin{baitoan}[\cite{Binh_boi_duong_Toan_6_tap_1}, VD8, p. 21]
	Tìm số chia \& số bị chia nhỏ nhất có thương số là $6$ \& số dư là $13$.
\end{baitoan}

\begin{baitoan}[\cite{Binh_boi_duong_Toan_6_tap_1}, 2.1., p. 21]
	Tính hợp lý: (a) $21\cdot(271 + 29) + 79\cdot(271 + 29)$. (b) $1 + 2 - 3 - 4 + 5 + 6 - 7 - 8 + \cdots - 499 - 500 + 501 + 502$.
\end{baitoan}

\begin{baitoan}[\cite{Binh_boi_duong_Toan_6_tap_1}, 2.2., p. 21]
	Quan hệ về cường độ của các nốt nhạc: 1 nốt tròn bằng 2 nốt trắng, 1 nốt trắng bằng 2 nốt đen, 1 nốt đen bằng 2 nốt móc, 1 nốt móc bằng 2 nốt móc đôi, 1 nốt móc đoi bằng 2 nốt móc 3, 1 nốt móc 3 bằng 2 nốt móc 4. Dùng lũy thừa của 1 số tự nhiên để diễn tả mối quan hệ về cường độ giữa: (a) Nốt tròn \& nốt đen. (b) Nốt tròn \& nốt móc 4. (c) Nốt trắng \& nốt móc đôi.
\end{baitoan}

\begin{baitoan}[\cite{Binh_boi_duong_Toan_6_tap_1}, 2.3., p. 21]
	Tính giá trị của biểu thức: $A = 3ab^2 -  \dfrac{a^3}{d} + c$ với $a = 3$, $b = 5$, $c = 7$, $d = 1$.
\end{baitoan}

\begin{baitoan}[\cite{Binh_boi_duong_Toan_6_tap_1}, 2.4., p. 21]
	So sánh: (a) $243^7$ \& $9^{10}\cdot27^5$. (b) $15^{15}$ \& $81^3\cdot125^5$. (c) $78^{15} - 78^{12}$ \& $78^{12} - 78^9$.
\end{baitoan}

\begin{baitoan}[\cite{Binh_boi_duong_Toan_6_tap_1}, 2.5., p. 21]
	Tìm $x\in\mathbb{N}$ thỏa: (a) $121:11 - (4x + 5):3 = 4$. (b) $2 + 4 + 6 + \cdots + x = 2450$ với $x$ là số tự nhiên chẵn.
\end{baitoan}

\begin{baitoan}[\cite{Binh_boi_duong_Toan_6_tap_1}, 2.6., p. 21]
	Tìm $x\in\mathbb{N}$ thỏa: (a) $(3x - 7)^5 = 32$. (b) $(4x - 1)^3 = 27\cdot125$.
\end{baitoan}

\begin{baitoan}[\cite{Binh_boi_duong_Toan_6_tap_1}, 2.7., p. 21]
	Cho 3 số $5,7,9$. Tìm tổng tất cả các số khác nhau viết bằng cả 3 số đó, mỗi chữ số dùng 1 lần.
\end{baitoan}

\begin{baitoan}[\cite{Binh_boi_duong_Toan_6_tap_1}, 2.8., p. 21]
	Tích của 2 số là $476$. Nếu thêm $22$ đơn vị vào 1 số thì tích của 2 số là $850$. Tìm 2 số đó.
\end{baitoan}

\begin{baitoan}[\cite{Binh_boi_duong_Toan_6_tap_1}, 2.9., p. 21]
	Hiệu của 2 số là $12$. Nếu tăng số bị trừ lên $2$ lần, giữ nguyên số trừ thì hiệu của chúng là $49$. Tìm 2 số đó.
\end{baitoan}

\begin{baitoan}[\cite{Binh_boi_duong_Toan_6_tap_1}, 2.10., p. 21]
	Tìm 2 số tự nhiên có thương bằng $7$. Nếu giảm số bị chia đi $124$ đơn vị thì thương của chúng bằng $3$.
\end{baitoan}

\begin{baitoan}[\cite{Binh_boi_duong_Toan_6_tap_1}, 2.11., p. 21]
	Rút gọn biểu thức: (a) $10\cdot\dfrac{4^6\cdot9^5 + 6^9\cdot120}{8^4\cdot3^{12} - 6^{11}}$. (b) $\sum_{i=0}^{50} 2^i = 1 + 2 + 2^2 + 2^3 + 2^4 + \cdots + 2^{49} + 2^{50}$. (c) $5 + 5^3 + 5^5 + \cdots + 5^{47} + 5^{49}$.
\end{baitoan}

\begin{baitoan}[\cite{Binh_boi_duong_Toan_6_tap_1}, 2.12., p. 22]
	Cho $\sum_{i=0}^{2000} 3^i = 1 + 3 + 3^2 + 3^3 + \cdots + 3^{1999} + 3^{2000}$. Chứng minh $A\divby13$.
\end{baitoan}

\begin{baitoan}[\cite{Binh_boi_duong_Toan_6_tap_1}, 2.13., p. 22]
	Tìm $x\in\mathbb{N}$ thỏa: (a) $2^x + 2^{x + 1} = 96$. (b) $3^{4x + 4} = 81^{x + 3}$.
\end{baitoan}

\begin{baitoan}[\cite{Binh_boi_duong_Toan_6_tap_1}, 2.14., p. 22]
	Tìm $x\in\mathbb{N}$ thỏa: (a) $(x - 5)^7 = (x - 5)^9$. (b) $x^{2015} = x^{2016}$.
\end{baitoan}

\begin{baitoan}[\cite{Binh_boi_duong_Toan_6_tap_1}, 2.15., p. 22]
	Tìm các số tự nhiên $x$ biết lũy thừa $5^{2x - 3}$ thỏa mãn điều kiện $100 < 5^{2x - 3}\le5^9$.
\end{baitoan}

\begin{baitoan}[\cite{Binh_boi_duong_Toan_6_tap_1}, 2.16., p. 22]
	Trong 1 phép chia, số bị chia bằng $69$, số dư bằng $3$. Tìm số chia \& thương.
\end{baitoan}

\begin{baitoan}[\cite{Binh_boi_duong_Toan_6_tap_1}, 2.17., p. 22]
	Tổng của 3 số là $124$. Nếu lấy số thứ nhất chia cho số thứ 2 hoặc lấy số thứ 2 chia cho số thứ 3 đều được thương là $3$ \& dư $4$. Tìm 3 số đó.
\end{baitoan}

\begin{baitoan}[\cite{Binh_boi_duong_Toan_6_tap_1}, 2.18., p. 22]
	(a) Khi chia 1 số cho $54$ thì được số dư là $49$. Nếu chia số đó cho $18$ thì thương \& số dư thay đổi thế nào? (b) Khi chia 1 số $a\in\mathbb{N}$ cho $b\in\mathbb{N}^\star$ thì được số dư là $r\in\mathbb{N}$. Nếu chia số đó cho $c\in\mathbb{N}^\star$ là 1 ước của $b$ thì thương \& số dư thay đổi thế nào?
\end{baitoan}

\begin{baitoan}[\cite{Binh_boi_duong_Toan_6_tap_1}, 2.19., p. 22]
	Tìm số bị chia \& số chia nhỏ nhất để được thương là $8$ \& dư là $45$.
\end{baitoan}

\begin{baitoan}[\cite{Binh_boi_duong_Toan_6_tap_1}, 2.20., p. 22]
	Tổng của 2 số bằng $36000$. Chia số lớn cho số nhỏ được thương bằng $4$ \& còn dư $940$. Tìm 2 số đó.
\end{baitoan}

\begin{baitoan}[\cite{Binh_boi_duong_Toan_6_tap_1}, 2.21., p. 22, Hạt thóc \& bàn cờ vua]
	1 nhà thông thái muốn thuyết phục vua Shilhram (Ấn Độ) về tầm quan trọng của mỗi người dân trong vương quốc. Vì thế, ông ta phát minh ra bàn cờ vua để thể hiện 1 vương quốc bao gồm vua, hoàng hậu, giáo sĩ trưởng, hiệp sĩ, \& quân lính, mọi thành phố đều quan trọng. Nhà vua trở nên cực kỳ thích \& ra lệnh cho mọi người trong vương quốc chơi cờ vua. Shilhram tuyên bố sẽ ban tặng nhà thông thái bất cứ số vàng \& bạc nào mà ông ta muốn, nhưng nhà thông thái không muốn nhận phần thưởng như vậy. Nhà thông thái cùng với nhà vua đi đến bàn cờ \& nhờ ông đặt 1 hạt thóc lên ô vuông đầu tiên, 2 hạt lên ô thứ 2, 4 hạt lên ô vuông thứ 3 \& xin với nhà vua tổng số hạt thóc được đặt theo cách như vậy đến ô cuối cùng của bàn cờ (số hạt đặt vào ô sau gấp đôi số hạt đặt vào ô trước). Nhà vua cảm thấy bị xúc phạm nhưng ông vẫn ra lệnh cho người hầu làm theo ước muốn của nhà thông thái. Những người hầu tuyệt vọng báo lại rằng số lượng hạt thóc dùng để thưởng cho nhà thông thái theo cách như vậy là không đủ. Nhà vua hiểu ngay nhà thông thái muốn dạy ông bài học thứ 2. Giống như quân tốt trong cờ vua, không nên đánh giá thấp những thứ nhỏ bé trong cuộc sống. Tính số hạt thóc mà nhà thông thái yêu cầu nhà vua Shilhram thưởng cho mình.
\end{baitoan}

\begin{baitoan}[\cite{Binh_boi_duong_Toan_6_tap_1}, p. 23]
	Có $79$ số tự nhiên, trong đó tổng của $13$ số bất kỳ đều là 1 số lẻ. Hỏi tổng của $79$ số tự nhiên đó là số lẻ hay số chẵn?
\end{baitoan}

\begin{baitoan}[\cite{Tuyen_Toan_6}, VD3, p. 8]
	1 học sinh khi nhân 1 số với $31$ đã đặt các tích riêng thẳng hàng như trong phép cộng nên tích đã giảm đi $540$ đơn vị so với tích đúng. Tìm tích đúng.
\end{baitoan}

\begin{baitoan}[\cite{Tuyen_Toan_6}, VD4, p. 8]
	Cho 2 số không chia hết cho $3$, khi chia cho $3$ được các số dư khác nhau. Chứng minh tổng của 2 số đó chia hết cho $3$.
\end{baitoan}

\begin{baitoan}[\cite{Tuyen_Toan_6}, 14., p. 9]
	Tính hợp lý: (a) $38 + 41 + 117 + 159 + 62$. (b) $73 + 86 + 978 + 914 + 3022$. (c) $341\cdot67 + 341\cdot16 + 659\cdot83$. (d) $42\cdot53 + 47\cdot156 - 47\cdot114$.
\end{baitoan}

\begin{baitoan}[\cite{Tuyen_Toan_6}, 15., p. 9]
	Tính giá trị của biểu thức: (a) $A = (100 - 1)\cdot(100 - 2)\cdots(100 - n)$ với $n\in\mathbb{N}^\star$ \& tích trên có đúng $100$ thừa số. (b) $B = 13a + 19b + 4a - 2b$ với $a + b = 100$.
\end{baitoan}

\begin{baitoan}[\cite{Tuyen_Toan_6}, 16., p. 9]
	Không tính giá trị cụ thể, so sánh giá trị 2 biểu thức: (a) $A = 199\cdot201$ \& $B = 200\cdot200$. (b) $C = 35\cdot53 - 18$ \& $D = 35 + 53\cdot34$.
\end{baitoan}

\begin{baitoan}[\cite{Tuyen_Toan_6}, 17., p. 9]
	Tính hợp lý: (a) $(44\cdot52\cdot60):(11\cdot13\cdot15)$. (b) $123\cdot456456 - 456\cdot123123$. (c) $(98\cdot7676 - 9898\cdot76):(2021\cdot2022\cdot2023\cdots2030)$.
\end{baitoan}

\begin{baitoan}[\cite{Tuyen_Toan_6}, 18., p. 9]
	Tìm giá trị nhỏ nhất của biểu thức: $A = 2021 - 1021:(999 - x)$.
\end{baitoan}

\begin{baitoan}[\cite{Tuyen_Toan_6}, 20., p. 9]
	Tìm số hạng thứ 5, thứ $n$ của dãy số: (a) $2,3,7,25,\ldots$. (b) $8,30,72,140,\ldots$.
\end{baitoan}

\begin{baitoan}[\cite{Tuyen_Toan_6}, 21., p. 9]
	Tìm $x$: (a) $(x + 74) - 318 = 200$. (b) $3636:(12x - 91) = 36$. (c) $(x:23 + 45)\cdot67 = 8911$.
\end{baitoan}

\begin{baitoan}[\cite{Tuyen_Toan_6}, 22., p. 9]
	1 nong tằm là $5$ nong kén. 1 nong kén là $9$ nén tơ. Hỏi muốn được $540$ nén tơ thì phải chăn bao nhiêu nong tằm?
\end{baitoan}

\begin{baitoan}[\cite{Tuyen_Toan_6}, 23., p. 9]
	2 số tự nhiên $a,b$ chia cho $m$ có cùng số dư, $a\ge b$. Chứng tỏ $a - b$ chia hết cho $m$.
\end{baitoan}

\begin{baitoan}[\cite{Tuyen_Toan_6}, 24., p. 9]
	Trong 1 phép chia có số bị chia là $155$, số dư là $12$. Tìm số chia \& thương.
\end{baitoan}

\begin{baitoan}[\cite{Tuyen_Toan_6}, 25., p. 9]
	Viết tập hợp $A$ các số tự nhiên $x$ biết lấy $x$ chia cho $12$ ta được thương bằng số dư.
\end{baitoan}

\begin{baitoan}[\cite{Tuyen_Toan_6}, 26., p. 10]
	Chia $129$ cho 1 số ta được số dư là $10$. Chia $61$ cho số đó ta cũng được số dư là $10$. Tìm số chia.
\end{baitoan}

\begin{baitoan}[\cite{Tuyen_Toan_6}, 27., p. 10]
	Cho 3 chữ số $a,b,c$ khác nhau \& khác $0$. Với cùng cả 3 chữ số này có thể lập được bao nhiêu số có 3 chữ số?
\end{baitoan}

\begin{baitoan}[\cite{Tuyen_Toan_6}, 28., p. 10]
	Cho 4 chữ số khác nhau \& khác $0$. (a) Với cùng cả 4 chữ số này có thể lập được bao nhiêu số có 4 chữ số? (b) Có thể lập được bao nhiêu số có 2 chữ số khác nhau trong 4 chữ số đã cho?
\end{baitoan}

\begin{baitoan}[\cite{Tuyen_Toan_6}, 29., p. 10]
	Cho 4 chữ số khác nhau trong đó có 1 chữ số $0$. Với cùng cả 4 chữ số này có thể lập được bao nhiêu số có 4 chữ số?
\end{baitoan}

\begin{baitoan}[\cite{Tuyen_Toan_6}, 30., p. 10]
	Anh Bách đi mua bánh kẹo tại 1 siêu thị, thanh toán bằng 1 phiếu mua hàng trị giá $100000$ đồng. Siêu thị không trả lại số tiền thừa. Giúp anh Bách chọn mua vừa hết số tiền ghi trong phiếu. Bảng giá 1 số mặt hàng có bán:
	\begin{table}[H]
		\centering
		\begin{tabular}{|c|l|c|r|}
			\hline
			STT & Tên hàng & Đơn vị & Giá bán \\
			\hline
			1 & Bánh đậu xanh & Hộp & 31 500 đồng \\
			\hline
			2 & Bánh bông lan & Gói & 23 500 đồng \\
			\hline
			3 & Bánh gạo & Gói & 19 000 đồng \\
			\hline
			4 & Bánh gạo chiên & Gói & 17 800 đồng \\
			\hline
			5 & Bánh quy & Gói & 13 500 đồng \\
			\hline
			6 & Bánh xốp & Gói & 5 300 đồng \\
			\hline
			7 & Kẹo hương dâu & Gói & 2 500 đồng \\
			\hline
		\end{tabular}
	\end{table}
\end{baitoan}

\begin{baitoan}[\cite{Binh_Toan_6_tap_1}, VD4, p. 7]
	Cho $A = 137\cdot454 + 206$, $B = 453\cdot138 - 110$. Không tính giá trị của $A$ \& $B$, chứng tỏ $A = B$.
\end{baitoan}

\begin{baitoan}[\cite{Binh_Toan_6_tap_1}, VD5, p. 8]
	Tìm kết quả của phép nhân: $A = \underbrace{33\ldots3}_{50}\cdot\underbrace{99\ldots9}_{50}$.
\end{baitoan}

\begin{baitoan}[\cite{Binh_Toan_6_tap_1}, VD6, p. 8]
	Tổng của 2 số tự nhiên khác nhau gấp 3 hiệu của chúng. Tìm thương của 2 số tự nhiên đó.
\end{baitoan}

\begin{baitoan}[\cite{Binh_Toan_6_tap_1}, VD7, p. 8]
	Khi chia số tự nhiên $a$ cho $54$, ta được số dư là $38$. Chia số $a$ cho $18$, ta được thương là $14$ \& còn dư. Tìm $a$.
\end{baitoan}

\begin{baitoan}[\cite{Binh_Toan_6_tap_1}, VD8, p. 8]
	Tìm 2 số tự nhiên lớn hơn $0$ sao cho tích của 2 số ấy gấp đôi tổng của chúng.
\end{baitoan}

\begin{baitoan}[\cite{Binh_Toan_6_tap_1}, VD9, p. 9]
	Điền các số tự nhiên $1,2,3,4,5$ vào các dấu $*$ để kết quả phép tính bằng $6$:  $*+*-*\cdot*:*$.
\end{baitoan}

\begin{baitoan}[\cite{Binh_Toan_6_tap_1}, VD10, p. 9]
	Giá tiền $7$ quyển vở nhiều hơn giá tiền $8$ bút chì. Hỏi giá tiền $8$ quyển vở \& giá tiền $9$ bút chì, đằng nào nhiều hơn?
\end{baitoan}

\begin{baitoan}[\cite{Binh_Toan_6_tap_1}, VD11, p. 9]
	Cho 6 số tự nhiên khác nhau có tổng bằng $50$. Chứng minh trong 6 số đó tồn tại 3 số có tổng lớn hơn hoặc bằng $30$.	
\end{baitoan}

\begin{baitoan}[\cite{Binh_Toan_6_tap_1}, 13., p. 10]
	Có thể viết được hay không 9 số vào 1 bảng vuông $3\times 3$, sao cho: Tổng các số trong 3 dòng theo thứ tự bằng $352, 463, 541$; tổng các số trong 3 cột theo thứ tự bằng $335, 687, 234$?
\end{baitoan}

\begin{baitoan}[\cite{Binh_Toan_6_tap_1}, 14., p. 10]
	Cho 9 số xếp vào 9 ô thành 1 hàng ngang, trong đó số đầu tiên là $4$, số cuối cùng là $8$, \& tổng 3 số ở 3 ô liền nhau bất kỳ bằng $17$. Tìm 9 số đó.
\end{baitoan}

\begin{baitoan}[\cite{Binh_Toan_6_tap_1}, 15., p. 10]
	Tìm số có $3$ chữ số, biết chữ số hàng trăm gấp $4$ lần chữ số hàng đơn vị \& nếu viết số ấy theo thứ tự ngược lại thì nó giảm đi $594$ đơn vị.
\end{baitoan}

\begin{baitoan}[\cite{Binh_Toan_6_tap_1}, 16., p. 10]
	Thay các dấu * bởi các chữ số thích hợp: $**** - *** = **$ biết số bị trừ, số trừ \& hiệu đều không đổi nếu đọc mỗi số từ phải sang trái.
\end{baitoan}

\begin{baitoan}[\cite{Binh_Toan_6_tap_1}, 19., p. 10]
	Hiệu của 2 số là $4$. Nếu tăng 1 số gấp $3$ lần, giữ nguyên số kia thì hiệu của chúng bằng $60$. Tìm 2 số đó.
\end{baitoan}

\begin{baitoan}[\cite{Binh_Toan_6_tap_1}, 20., p. 10]
	Cho số $123456789$. Đặt 1 số dấu ``$+$'' \& ``$-$'' vào giữa các chữ số để kết quả của phép tính bằng $100$.
\end{baitoan}

\begin{baitoan}[\cite{Binh_Toan_6_tap_1}, 21., p. 10]
	Cho số $987654321$. Đặt 1 số dấu ``$+$'' \& ``$-$'' vào giữa các chữ số để kết quả của phép tính bằng: $100,99$.
\end{baitoan}

\begin{baitoan}[\cite{Binh_Toan_6_tap_1}, 22., p. 10]
	Tìm giá trị lớn nhất của hiệu $\overline{bd} - \overline{ac}$ biết $a < b < c < d$.
\end{baitoan}

\begin{baitoan}[\cite{Binh_Toan_6_tap_1}, 23., p. 10]
	Tìm 6 chữ số khác nhau $a,b,c,d,e,f$ sao cho $A = \overline{abc} - \overline{def}$ có giá trị nhỏ nhất.
\end{baitoan}

\begin{baitoan}[\cite{Binh_Toan_6_tap_1}, 24., p. 11]
	Cho 6 chữ số khác nhau $a,b,c,d,e,f$. Gọi $A = \overline{abc} + \overline{bcd} + \overline{cde} + \overline{def}$. Tìm giá trị lớn nhất \& giá trị nhỏ nhất của $A$.
\end{baitoan}

\begin{baitoan}[\cite{Binh_Toan_6_tap_1}, 25., p. 11]
	Tìm 2 số biết tổng của chúng gấp $5$ lần hiệu của chúng, tích của chúng gấp $24$ lần hiệu của chúng.
\end{baitoan}

\begin{baitoan}[\cite{Binh_Toan_6_tap_1}, 26., p. 11]
	Tìm 2 số biết tổng của chúng gấp $7$ lần hiệu của chúng, còn tích của chúng gấp $192$ lần hiệu của chúng.
\end{baitoan}

\begin{baitoan}[\cite{Binh_Toan_6_tap_1}, 27., p. 11]
	Tích của 2 số là $6210$. Nếu giảm 1 thừa số đi $7$ đơn vị thì tích mới là $5265$. Tìm các thừa số của tích.
\end{baitoan}

\begin{baitoan}[\cite{Binh_Toan_6_tap_1}, 28., p. 11]
	Bảo làm 1 phép nhân, trong đó số nhân là $102$. Nhưng khi viết số nhân, Bảo đã quên không viết chữ số $0$ nên tích bị giảm đi $21870$ đơn vị so với tích đúng. Tìm số bị nhân của phép nhân đó.
\end{baitoan}

\begin{baitoan}[\cite{Binh_Toan_6_tap_1}, 29., p. 11]
	1 học sinh nhân $78$ với số nhân là số có 2 chữ số, trong đó chữ số hàng chục gấp $3$ lần chữ số hàng đơn vị. Do nhầm lẫn bạn đó viết đổi thứ tự 2 chữ số của số nhân, nên tích giảm đi $2808$ đơn vị so với tích đúng. Tìm số nhân đúng.
\end{baitoan}

\begin{baitoan}[\cite{Binh_Toan_6_tap_1}, 30., p. 11]
	1 học sinh nhân 1 số với $463$. Vì bạn đó viết các chữ số tận cùng của các tích riêng ở cùng 1 cột nên tích bằng $30524$. Tìm số bị nhân.
\end{baitoan}

\begin{baitoan}[\cite{Binh_Toan_6_tap_1}, 31., p. 11]
	Chứng minh hiệu sau có thể viết được thành 1 tích của 2 thừa số bằng nhau: $11111111 - 2222$.
\end{baitoan}

\begin{baitoan}[\cite{Binh_Toan_6_tap_1}, 32., p. 11]
	Chỉ ra 2 số khác nhau sao cho nếu nhân mỗi số với $7$ thì ta được kết quả là các số gồm toàn các chữ số $9$.
\end{baitoan}

\begin{baitoan}[\cite{Binh_Toan_6_tap_1}, 33., p. 11]
	Tìm kết quả của phép nhân sau: $\underbrace{3\ldots 3}_{50}\cdot\underbrace{3\ldots 3}_{50}$.
\end{baitoan}

\begin{baitoan}[\cite{Binh_Toan_6_tap_1}, 34., p. 11]
	Tìm tổng các chữ số của tích: (a) $A = \underbrace{88\ldots8}_{21}\cdot\underbrace{99\ldots9}_{21}$. (b) $B = \underbrace{99\ldots9}_{94}\cdot\underbrace{44\ldots4}_{94}$.
\end{baitoan}

\begin{baitoan}[\cite{Binh_Toan_6_tap_1}, 35., p. 11]
	Chứng minh các số sau có thể viết được thành 1 tích của 2 số tự nhiên liên tiếp: $111222$, $444222$.
\end{baitoan}

\begin{baitoan}[\cite{Binh_Toan_6_tap_1}, 36., p. 11]
	Tìm 2 số tự nhiên có thương bằng $35$ biết nếu số bị chia tăng thêm $1056$ đơn vị thì thương bằng $57$.
\end{baitoan}

\begin{baitoan}[\cite{Binh_Toan_6_tap_1}, 37., p. 11]
	Tìm số bị chia \& số chia biết thương bằng $6$, số dư bằng $49$, tổng của số bị chia, số chia, \& số dư bằng $595$.
\end{baitoan}

\begin{baitoan}[\cite{Binh_Toan_6_tap_1}, 38., p. 11]
	1 phép chia có thương bằng $4$, số dư bằng $25$. Tổng của số bị chia, số chia \& số dư bằng $210$. Tìm số bị chia \& số chia.
\end{baitoan}

\begin{baitoan}[\cite{Binh_Toan_6_tap_1}, 39., p. 11]
	Trong hội trường có $680$ người ngồi. Tất cả có $25$ dãy ghế, mỗi dãy ghế có $30$ chỗ ngồi. Ít nhất có bao nhiêu dãy ghế có số chỗ ngồi như nhau?
\end{baitoan}

\begin{baitoan}[\cite{Binh_Toan_6_tap_1}, 40., p. 12]
	(a) Trong 1 năm, có ít nhất \& nhiều nhất bao nhiêu ngày Chủ nhật? (b) Ngày 1.1 năm nay rơi vào ngày Chủ nhật. Ngày 1.1 năm sau rơi vào ngày thứ mấy?
\end{baitoan}

\begin{baitoan}[\cite{Binh_Toan_6_tap_1}, 41., p. 12]
	Tháng 8 của 1 năm có 4 ngày thứ Năm \& 5 ngày thứ 4. Hỏi ngày đầu tiên của tháng đó là ngày thứ mấy?
\end{baitoan}

\begin{baitoan}[\cite{Binh_Toan_6_tap_1}, 42., p. 12]
	Ngày 19.8.2002 vào ngày thứ Hai. Tính xem ngày 19.8.1945 vào ngày nào trong tuần?
\end{baitoan}

\begin{baitoan}[\cite{Binh_Toan_6_tap_1}, 43., p. 12]
	Tìm thương của 1 phép nhân, biết nếu thêm $15$ vào số bị chia \& thêm $5$ vào số chia thì thương \& số dư không đổi.
\end{baitoan}

\begin{baitoan}[\cite{Binh_Toan_6_tap_1}, 44., p. 12]
	Tìm thương của 1 phép chia, biết nếu tăng số bị chia $90$ đơn vị, tăng số chia $6$ đơn vị thì thương \& số dư không đổi.
\end{baitoan}

\begin{baitoan}[\cite{Binh_Toan_6_tap_1}, 45., p. 12]
	Tìm thương của 1 phép chia, biết nếu tăng số bị chia $73$ đơn vị, tăng số chia $4$ đơn vị thì thương không đổi, còn số dư tăng $5$ đơn vị.
\end{baitoan}

\begin{baitoan}[\cite{Binh_Toan_6_tap_1}, 46., p. 12]
	Xác định phép chia, biết số bị chia, số chia, thương \& số dư là 4 số trong các số sau: (a) $3,4,16,64,256,772$. (b) $2,3,9,27,81,243,567$.
\end{baitoan}

\begin{baitoan}[\cite{Binh_Toan_6_tap_1}, 47., p. 12]
	Khi chia 1 số tự nhiên gồm 3 chữ số như nhau cho 1 số tự nhiên gồm 3 chữ số như nhau, ta được thương là $2$ \& còn dư. Nếu xóa 1 chữ số ở số bị chia \& xóa 1 chữ số ở số chia thì thương của phép chia vẫn bằng $3$ nhưng số dư giảm hơn trước là $100$. Tìm số bị chia \& số chia lúc đầu.
\end{baitoan}

\begin{baitoan}[\cite{Binh_Toan_6_tap_1}, 48., p. 12]
	Trong 1 phép chia có dư, số bị chia gồm 4 chữ số như nhau, số chia gồm 3 chữ số như nhau, thương bằng $13$ \& còn dư. Nếu xóa 1 chữ số ở số bị chia, xóa 1 chữ số ở số chia thì thương không đổi, còn số dư giảm hơn trước là $100$ đơn vị. Tìm số bị chia \& số chia lúc đầu.
\end{baitoan}

\subsection{Combination of Calculus -- Phối Hợp Các Phép Tính}

\begin{baitoan}[\cite{Binh_Toan_6_tap_1}, 49., p. 12]
	Tính nhanh: (a) $19\cdot64 + 76\cdot34$. (b) $35\cdot12 + 65\cdot13$. (c) $136\cdot68 + 16\cdot272$. (d) $(2 + 4 + 6 + \cdots + 100)\cdot(36\cdot333 - 108\cdot111)$. (e) $19991999\cdot1998 - 19981998\cdot1999$.
\end{baitoan}

\begin{baitoan}[\cite{Binh_Toan_6_tap_1}, 50., p. 12]
	Không tính cụ thể các giá trị của $A$ \& $B$, cho biết số nào lớn hơn \& lớn hơn bao nhiêu? (a) $A = 1998\cdot1998$, $B = 1996\cdot2000$. (b) $A = 2000\cdot2000$, $B = 1990\cdot2010$. (c) $A = 25\cdot33 - 10$, $B = 31\cdot26 + 10$. (d) $A = 32\cdot53 - 31$, $B = 53\cdot31 + 32$.
\end{baitoan}
Bài toán trên có thể được tổng quát như sau:

\begin{baitoan}
	Cho $n,k\in\mathbb{N}^\star$. Không tính cụ thể các giá trị của $A_i$ \& $B_i$, $i = 1,2$, cho biết số nào lớn hơn \& lớn hơn bao nhiêu? (a) $A_1 = n\cdot n = n^2$, $B_1 = (n - k)(n + k)$. (b) $A_2 = n^3$, $B_2 = (n - k)n(n + k)$.
\end{baitoan}

\begin{proof}[Giải]
	(a) Khai triển $B_1$ hoặc dùng \textit{hằng đẳng thức đáng nhớ} $(a + b)(a - b) = a^2 - b^2$, $\forall a,b\in\mathbb{R}$, ta có $B_1 = (n - k)(n + k) = n^2 + nk - kn - k^2 = n^2 - k^2 = A_1 - k^2 < A_1$ do $k\ge 1$. Vậy $A_1 > B_1$ \& lớn hơn 1 lượng bằng $k^2$. (b) Nhận thấy $A_2 = nA_1$, $B_2 = nB_1$, nên $A_2 - B_2 = n(A_1 - B_1) = nk^2 > 0$.
\end{proof}
Bài toán trên có thể được tổng quát hơn nữa như sau, ý tưởng giải vẫn là sử dụng \textit{nhiều lần} hằng đẳng thức $(a + b)(a - b) = a^2 - b^2$, $\forall a,b\in\mathbb{R}$, 1 cách thích hợp:

\begin{baitoan}
	Cho $m,n,k\in\mathbb{N}^\star$. Không tính cụ thể các giá trị của $A_i$ \& $B_i$, $i = 1,2$, cho biết số nào lớn hơn \& lớn hơn bao nhiêu? (a) $A_1 = n^{2m}$, $B_1 = (n - mk)(n - (m - 1)k)\cdots(n - k)(n + k)\cdots(n + mk)$. (b) $A_2 = n^{2m + 1}$, $B_2 = (n - mk)(n - (m - 1)k)\cdots(n - k)n(n + k)\cdots(n + mk)$.
\end{baitoan}
Khi $m = 1$, bài toán tổng quát này trở thành bài toán trước đó.

\begin{baitoan}[\cite{Binh_Toan_6_tap_1}, 51., p. 12]
	Tìm thương của phép chia sau mà không tính kết quả cụ thể của số bị chia \& số chia: $\dfrac{37\cdot13 - 13}{24 + 37\cdot12}$.
\end{baitoan}

\begin{baitoan}[\cite{Binh_Toan_6_tap_1}, 52., p. 13]
	Tính: (a) $A = \dfrac{\sum_{i=1}^{101} i}{\sum_{i=1}^{101} (-1)^{i+1}i} = \dfrac{101 + 100 + 99 + 98 + \cdots + 3 + 2 + 1}{101 - 100 + 99 - 98 + \cdots + 3 - 2 + 1}$. (b) $B = \dfrac{3737\cdot43 - 4343\cdot37}{2 + 4 + 6 + \cdots + 100}$.
\end{baitoan}

\begin{baitoan}[\cite{Binh_Toan_6_tap_1}, 53., p. 13]
	Vận dụng tính chất các phép tính để tìm các kết quả bằng cách nhanh chóng: (a) $1990\cdot1990 - 1992\cdot1988$. (b) $(1374\cdot57 + 687\cdot86):(26\cdot13 + 74\cdot14)$. (c) $(124\cdot237 + 152):(870 + 235\cdot122)$. (d) $\dfrac{423134\cdot846267 - 423133}{846267\cdot423133 + 423134}$.
\end{baitoan}

\begin{baitoan}[\cite{Binh_Toan_6_tap_1}, 54., p. 13]
	Tìm $a\in\mathbb{N}$ biết: (a) $697:\dfrac{15a + 364}{a} = 17$. (b) $92\cdot4 - 27 = \dfrac{a + 350}{a} + 315$.
\end{baitoan}

\begin{baitoan}[\cite{Binh_Toan_6_tap_1}, 55., p. 13]
	Tìm $x\in\mathbb{N}$ biết: (a) $720:[41 - (2x - 5)] = 40$. (b) $(x + 1) + (x + 2) + (x + 3) + \cdots + (x + 100) = 5750$.
\end{baitoan}

\begin{baitoan}[\cite{Binh_Toan_6_tap_1}, 56., p. 13]
	Cho số $12345678$. Đặt các dấu phép tính \& dấu ngoặc để kết quả của phép tính bằng $9$.
\end{baitoan}

\begin{baitoan}[\cite{Binh_Toan_6_tap_1}, 57., p. 13]
	Viết 5 dãy tính có kết quả bằng $100$, với 6 chữ số $5$ cùng với dấu các phép tính (\& dấu ngoặc nếu cần).
\end{baitoan}

\begin{baitoan}[\cite{Binh_Toan_6_tap_1}, 58., p. 13]
	(a) Viết dãy tính có kết quả bằng $100$, với 5 chữ số như nhau cùng với dấu các phép tính (\& dấu ngoặc nếu cần). (b) Cũng hỏi như vậy với 6 chữ số khác nhau.
\end{baitoan}

\begin{baitoan}[\cite{Binh_Toan_6_tap_1}, 59., p. 13]
	(a) Tính (kết quả khá đặc biệt): $1\cdot8 + 1$, $12\cdot8 + 2$, $123\cdot8 + 3$, $1234\cdot8 + 4$. (b) Viết tiếp 4 dòng nữa theo quy luật trên.
\end{baitoan}

\begin{baitoan}[\cite{Binh_Toan_6_tap_1}, 60., p. 13]
	Điền các số $1,2,3,4,5$ vào các dấu $*$ để kết quả của phép tính bằng $3$: $*+*-*\cdot*:*$.
\end{baitoan}

\subsection{Use Inequalities -- Sử Dụng Bất Đẳng Thức}

\begin{baitoan}[\cite{Binh_Toan_6_tap_1}, 61., p. 13]
	Giá tiền 1 quyển sách, 6 quyển vở, 3 chiếc bút là $7700$ đồng, giá tiền 8 quyển sách, 6 quyển vở, 6 chiếc bút là $16000$ đồng. So sánh giá tiền 1 quyển sách \& 1 quyển vở.
\end{baitoan}

\begin{baitoan}[\cite{Binh_Toan_6_tap_1}, 62., p. 13]
	Viết liên tiếp các số tự nhiên từ $1$ đến $15$, được: $A = 1234\ldots1415$. Xóa đi 15 chữ số của số $A$ để các chữ số còn lại (vẫn giữ nguyên thứ tự như trước) tạo thành: (a) Số nhỏ nhất. (b) Số lớn nhất.
\end{baitoan}

\begin{baitoan}[\cite{Binh_Toan_6_tap_1}, 63., p. 14]
	Cho số $123\ldots20$, i.e., viết liên tiếp các số tự nhiên từ $1$ đến $20$. Xóa đi $20$ chữ số để số còn lại có giá trị: (a) Nhỏ nhất. (b) Lớn nhất.
\end{baitoan}

\begin{baitoan}[\cite{Binh_Toan_6_tap_1}, 64., p. 14]
	Tìm giá trị nhỏ nhất của hiệu giữa 1 số tự nhiên có 2 chữ số với tổng các chữ số của nó.
\end{baitoan}

\begin{baitoan}[\cite{Binh_Toan_6_tap_1}, 65., p. 14]
	Tìm số chia \& số dư biết số bị chia bằng $112$, thương bằng $5$.
\end{baitoan}

\begin{baitoan}[\cite{Binh_Toan_6_tap_1}, 66., p. 14]
	Tìm số chia \& số dư biết số bị chia bằng $813$, thương bằng $15$, số dư gồm 2 chữ số như nhau.
\end{baitoan}

\begin{baitoan}[\cite{Binh_Toan_6_tap_1}, 67., p. 14]
	Tìm số chia \& số dư của phép chia $542$ cho 1 số tự nhiên, biết thương bằng $12$.
\end{baitoan}

\begin{baitoan}[\cite{Binh_Toan_6_tap_1}, 68., p. 14]
	1 học sinh trong 5 năm học từ lớp 5 đến lớp 9 đã qua $31$ kỳ thi, trong đó số kỳ thi ở năm sau nhiều hơn số kỳ thi ở năm trước, \& số kỳ thi ở năm cuối gấp 3 lần số kỳ thi ở năm đầu. Hỏi học sinh đó thi bao nhiêu kỳ ở năm thứ 4?
\end{baitoan}

\begin{baitoan}[\cite{Binh_Toan_6_tap_1}, 69., p. 14]
	Tìm 2 số tự nhiên sao cho tổng của 2 số ấy bằng tích của chúng.
\end{baitoan}

\begin{baitoan}[\cite{Binh_Toan_6_tap_1}, 70., p. 14]
	Tìm 3 số tự nhiên khác $0$ biết tổng của 3 số ấy bằng tích của chúng.
\end{baitoan}

%------------------------------------------------------------------------------%

\section{Exponentiation on $\mathbb{N}$ -- Lũy Thừa với Số Mũ Tự Nhiên}
\cite[\S5, pp. 17--18]{SBT_Toan_6_Canh_Dieu_tap_1}: 37. 38. 39. 40. 41. 42. 43. 44. 45. 46. 47. 48. 49. 

\begin{dinhnghia}[Số chính phương]
	{\rm Số chính phương} là số bằng bình phương của 1 số tự nhiên, i.e., $a$ là số chính phương $\Leftrightarrow a = n^2$ với $n\in\mathbb{N}$ nào đó.
\end{dinhnghia}

\begin{baitoan}[\cite{Tuyen_Toan_6}, VD5, p. 11]
	Chứng minh tổng $\sum_{i=1}^5 i^3 = 1^3 + 2^3 + 3^3 + 4^3 + 5^3$ là 1 số chính phương.
\end{baitoan}

\begin{baitoan}[\cite{Tuyen_Toan_6}, VD6, p. 11]
	Tìm $x\in\mathbb{N}$ biết $2\cdot3^x = 162$.
\end{baitoan}

\begin{baitoan}[\cite{Tuyen_Toan_6}, VD7, p. 11]
	Tìm $x\in\mathbb{N}$ biết $(x + 2)^4 = 5^2\cdot25$.
\end{baitoan}

\begin{baitoan}[\cite{Tuyen_Toan_6}, 31., p. 11]
	Trong các số $2^4,3^4,4^2,4^3,99^0,0^{99},1^n$ với $n\in\mathbb{N}^\star$, các số nào bằng nhau? Số nào nhỏ nhất? Số nào lớn nhất?
\end{baitoan}

\begin{baitoan}[\cite{Tuyen_Toan_6}, 32., p. 11]
	Kiểm tra đẳng thức $152 - 5^3 = 10^2$ đúng hay sai. Nếu sai, di chuyển 1 chữ số đến vị trí khác để được đẳng thức đúng.
\end{baitoan}

\begin{baitoan}[\cite{Tuyen_Toan_6}, 33., p. 11]
	Chứng minh mỗi tổng{\tt/}hiệu sau là 1 số chính phương: (a) $3^2 + 4^2$. (b) $13^2 - 5^2$. (c) $1^3 + 2^3 + 3^3 + 4^3$.
\end{baitoan}

\begin{baitoan}[\cite{Tuyen_Toan_6}, 34., pp. 11--12]
	Viết các tổng{\tt/}hiệu sau dưới dạng 1 lũy thừa với số mũ lớn hơn $1$. (a) $17^2 - 15^2$. (b) $4^3 - 2^3 + 5^2$. 
\end{baitoan}

\begin{baitoan}[\cite{Tuyen_Toan_6}, 35., p. 12]
	Viết số $729$ dưới dạng 1 lũy thừa với 3 cơ số khác nhau \& số mũ lớn hơn $1$.
\end{baitoan}

\begin{baitoan}[\cite{Tuyen_Toan_6}, 36., p. 12]
	Viết các tích{\tt/}thương sau dưới dạng lũy thừa của 1 số: (a) $2^5\cdot8^4$. (b) $25^6\cdot125^3$. (c) $625^5:25^7$. (d) $12^3\cdot3^3$.
\end{baitoan}

\begin{baitoan}[\cite{Tuyen_Toan_6}, 37., p. 12]
	Tính $6^{3^1},3^{2^3},7^{1^{2^{3^4}}},2020^{3^{0^{1^0}}}$.
\end{baitoan}

\begin{baitoan}[\cite{Tuyen_Toan_6}, 38., p. 12]
	Tìm $x\in\mathbb{N}$ biết: (a) $(3x - 2)^3 = 64$. (b) $(2x + 5)^4 = 3^4\cdot5^4$.
\end{baitoan}

\begin{baitoan}[\cite{Tuyen_Toan_6}, 39., p. 12]
	Tìm $x\in\mathbb{N}$ biết: (a) $5^x + 5^{x + 2} = 650$. (b) $3^{x + 4} = 9^{2x - 1}$.
\end{baitoan}

\begin{baitoan}[\cite{Tuyen_Toan_6}, 40., p. 12]
	Tìm $x\in\mathbb{N}$ biết: (a) $2^x - 15 = 17$. (b) $(7x - 11)^3 = 2^5\cdot5^2 + 200$.
\end{baitoan}

\begin{baitoan}[\cite{Tuyen_Toan_6}, 41., p. 12]
	Tìm $x\in\mathbb{N}$ biết: (a) $x^{10} = 1^x$. (b) $x^{10} = x$. (c) $(2x - 15)^5 = (2x - 15)^3$.
\end{baitoan}

\begin{baitoan}[\cite{Tuyen_Toan_6}, 42., p. 12]
	Tìm $m,n\in\mathbb{N}$ thỏa $2^m + 2^n = 40$.
\end{baitoan}

\begin{baitoan}[\cite{Tuyen_Toan_6}, 43., p. 12]
	Số $4^6\cdot5^{14}$ có bao nhiêu chữ số nếu viết trong hệ thập phân ở dạng thông thường (không có số mũ)?
\end{baitoan}

\begin{baitoan}[\cite{Tuyen_Toan_6}, 44., p. 12]
	Trong âm nhạc, về trường độ thì: 1 nốt tròn bằng 2 nốt trắng, 1 nốt trắng bằng 2 nốt đen, 1 nốt đen bằng 2 nốt móc đơn, 1 nốt móc đơn bằng 2 nốt móc kép, 1 nốt móc kép bằng 2 nốt móc 3, 1 nốt móc 3 bằng 2 nốt móc 4. Dùng lũy thừa của 1 số tự nhiên để: (a) Diễn tả mối quan hệ về trường độ giữa nốt tròn với các nốt nhạc khác. (b) Cho biết nốt nhạc có trường độ gấp $8$ lần nốt móc 3 là nốt nhạc nào?
\end{baitoan}

\begin{baitoan}[\cite{Tuyen_Toan_6}, 45., p. 12, Phân bào]
	Tế bào lớn lên đến 1 kích thước nhất định thì phân chia thành $2$ tế bào con. Mỗi tế bào con tiếp tục lớn lên cho đến khi bằng tế bào mẹ, sau đó phân chia thành $2$ tế bào, quá trình cứ thế tiếp tục. Cho biết: (a) Số tế bào con sau lần phân chia thứ $3$, thứ $5$, thứ $n\in\mathbb{N}^\star$. Viết kết quả dưới dạng lũy thừa cơ số $2$. (b) Sau mấy lần phân chia thì số tế bào con là $128$?
\end{baitoan}
Về phân bào, see, e.g., \href{https://vi.wikipedia.org/wiki/Ph%C3%A2n_b%C3%A0o}{Wikipedia{\tt/}phân bào} \& \href{https://en.wikipedia.org/wiki/Spindle_apparatus}{Wikipedia{\tt/}spindle apparatus}.

\begin{baitoan}[\cite{Binh_Toan_6_tap_1}, VD12, p. 14]
	Không dùng máy tính, chứng minh $A = 215216217\cdot218218220$ là số có $17$ chữ số.
\end{baitoan}

\begin{baitoan}[\cite{Binh_Toan_6_tap_1}, VD13, p. 14]
	Sử dụng nhận xét $2^{10} = 1024\approx10^3$, chứng minh $2^{64}$ có vào khoảng $20$ chữ số.
\end{baitoan}

\begin{baitoan}[\cite{Binh_Toan_6_tap_1}, VD14, p. 15]
	Chứng minh $A = 4 + \sum_{i=2}^{20} 2^i = 4 + 2^2 + 2^3 + 2^4 + \cdots + 2^{20}$.
\end{baitoan}

\begin{baitoan}[\cite{Binh_Toan_6_tap_1}, VD15, p. 15]
	So sánh $63^7$ \& $16^{12}$.
\end{baitoan}

\begin{baitoan}[\cite{Binh_Toan_6_tap_1}, VD16, p. 15]
	(a) Với 3 chữ số $2$, viết thành 1 số tự nhiên có giá trị lớn nhất. (b) Cũng hỏi như vậy đối với 3 chữ số $4$. (c) Cũng hỏi như vậy đối với 3 chữ số $a\in\mathbb{N}$.
\end{baitoan}

\begin{baitoan}[\cite{Binh_Toan_6_tap_1}, VD17, p. 16]
	Số $2^2$ \& $5^2$ viết liền nhau được số $425$ có 3 chữ số, số $2^3$ \& $5^3$ viết liền nhau được số $8125$ có 4 chữ số, số $2^4$ \& $5^4$ viết liền nhau được số $16625$ có 5 chữ số. Chứng minh số $2^{1991}$ \& $5^{1991}$ viết liền nhau được số có $1992$ chữ số.
\end{baitoan}

\begin{baitoan}[\cite{Binh_Toan_6_tap_1}, 71., p. 16]
	Tính: (a) $4^{10}\cdot8^{15}$. (b) $4^{15}\cdot5^{30}$. (c) $27^{16}:9^{10}$. (d) $A = \dfrac{72^3\cdot54^2}{108^4}$. (e) $B = \dfrac{3^{10}\cdot11 + 3^{10}\cdot5}{3^9\cdot2^4}$.
\end{baitoan}

\begin{baitoan}[\cite{Binh_Toan_6_tap_1}, 72., p. 16]
	Tính giá trị của biểu thức: (a) $\dfrac{2^{10}\cdot13 + 2^{10}\cdot65}{2^8\cdot104}$.\\(b) $(1 + 2 + 3 + \cdots + 100)(1^2 + 2^2 + 3^2 + \cdots + 10^2)(65\cdot111 - 13\cdot15\cdot37)$.
\end{baitoan}

\begin{baitoan}[\cite{Binh_Toan_6_tap_1}, 73., p. 16]
	Tìm $x\in\mathbb{N}$ biết: (a) $2^x\cdot4 = 128$. (b) $x^{15} = x$. (c) $(2x + 1)^3 = 125$. (d) $(x - 5)^4 = (x - 5)^6$.
\end{baitoan}

\begin{baitoan}[\cite{Binh_Toan_6_tap_1}, 74., p. 16]
	Cho $A = \sum_{i=1}^{100} 3^i = 3 + 3^2 + 3^3 + \cdots + 3^{100}$. Tìm $n\in\mathbb{N}$ biết $2A + 3 = 3^n$.
\end{baitoan}

\begin{baitoan}[\cite{Binh_Toan_6_tap_1}, 75., p. 16]
	Tính tổng của $100$ số: $\sum_{i=1}^{100} \underbrace{9}_i = 9+ 99 + 999 + \cdots + \underbrace{99\ldots9}_{100}$.
\end{baitoan}

\begin{baitoan}[\cite{Binh_Toan_6_tap_1}, 76., p. 16]
	Tìm số tự nhiên có 3 chữ số, biết bình phương của chữ số hàng chục bằng tích của 2 chữ số kia \& số tự nhiên đó trừ đi số gồm 3 chữ số ấy viết theo thứ tự ngược lại bằng $495$.
\end{baitoan}

\begin{baitoan}[\cite{Binh_Toan_6_tap_1}, 77., pp. 16--17]
	(a) Viết dãy tính có kết quả bằng $1000000$, với 5 chữ số như nhau cùng với dấu các phép tính \& dấu ngoặc nếu cần. (b) Cũng hỏi như vậy với 6 chữ số khác nhau.
\end{baitoan}

\begin{baitoan}[\cite{Binh_Toan_6_tap_1}, 78., p. 17]
	So sánh $A$ \& $B$: (a) $A = \sum_{i=1}^{1000} i = 1 + 2 + 3 + \cdots + 1000$, $B = 11! = \prod_{i=1}^{11} = 1\cdot2\cdot3\cdots11$. (b) $A = 20! = \prod_{i=1}^{20} = 1\cdot2\cdot3\cdots20$, $B = \sum_{i=1}^{1000000} i = 1 + 2 + 3 + \cdots + 1000000$.
\end{baitoan}

\begin{baitoan}[\cite{Binh_Toan_6_tap_1}, 79., p. 17]
	So sánh: (a) $3^{500}$ \& $7^{300}$. (b) $8^5$ \& $3\cdot4^7$. (c) $99^{20}$ \& $9999^{10}$. (d) $202^{303}$ \& $303^{202}$. (e) $3^{21}$ \& $2^{31}$. (f) $11^{1979}$ \& $37^{1320}$. (g) $10^10$ \& $48\cdot50^5$. (h) $1990^{10} + 1990^9 + 1991^{10}$.
\end{baitoan}

\begin{baitoan}[\cite{Binh_Toan_6_tap_1}, 80., p. 17]
	So sánh: (a) $5^{299}$ \& $3^{501}$. (b) $3^{23}$ \& $5^{15}$. (c) $127^{23}$ \& $513^{18}$.
\end{baitoan}

\begin{baitoan}[\cite{Binh_Toan_6_tap_1}, 81., p. 17]
	Chứng minh: $5^{27} < 2^{63} < 5^{28}$.
\end{baitoan}

\begin{baitoan}[\cite{Binh_Toan_6_tap_1}, 82., p. 17]
	Viết liền nhau các kết quả của các lũy thừa $4^{50}$ \& $25^{50}$, ta được 1 số tự nhiên có bao nhiêu chữ số?
\end{baitoan}

\begin{baitoan}[\cite{Binh_Toan_6_tap_1}, 83., p. 17]
	Tìm số tự nhiên có 4 chữ số biết số đó có thể phân tích thành tích của 2 thừa số có tổng bằng $100$ \& 1 trong 2 thừa số ấy có dạng $a^a$.
\end{baitoan}

%------------------------------------------------------------------------------%

\subsection{Compare Exponentiations -- So Sánh Các Lũy Thừa}

\begin{baitoan}[\cite{Tuyen_Toan_6}, VD8, p. 13]
	So sánh $3^7$ \& $2^{11}$.
\end{baitoan}

\begin{baitoan}[\cite{Tuyen_Toan_6}, VD9, p. 13]
	So sánh $16^{19}$ \& $8^{25}$.
\end{baitoan}

\begin{baitoan}[\cite{Tuyen_Toan_6}, VD10, p. 13]
	So sánh $3^{4040}$ \& $2^{6060}$.
\end{baitoan}

\begin{baitoan}[\cite{Tuyen_Toan_6}, 46., p. 14]
	So sánh: (a) $27^{11}$ \& $81^8$. (b) $625^5$ \& $125^7$.
\end{baitoan}

\begin{baitoan}[\cite{Tuyen_Toan_6}, 47., p. 14]
	So sánh: (a) $5^{36}$ \& $11^{24}$. (b) $3^{2n}$ \& $2^{3n}$, $\forall n\in\mathbb{N}^\star$.
\end{baitoan}

\begin{baitoan}[\cite{Tuyen_Toan_6}, 48., p. 14]
	So sánh $A = 2\cdot3^{54}$ \& $B = 6\cdot5^{32}$.
\end{baitoan}

\begin{baitoan}[\cite{Tuyen_Toan_6}, 49., p. 14]
	Chứng minh: $5^{60n} < 2^{140n} < 3^{100n}$, $\forall n\in\mathbb{N}^\star$.
\end{baitoan}

\begin{baitoan}[\cite{Tuyen_Toan_6}, 50., p. 14]
	Sắp xếp 3 số $3^{539},7^{308},2^{847}$ theo thứ tự tăng dần.
\end{baitoan}

\begin{baitoan}[\cite{Tuyen_Toan_6}, 51., p. 14]
	So sánh: (a) $5^{75}$ \& $7^{60}$. (b) $3^{21}$ \& $2^{31}$.
\end{baitoan}

\begin{baitoan}[\cite{Tuyen_Toan_6}, 52., p. 14]
	So sánh: (a) $5^{23}$ \& $6\cdot5^{22}$. (b) $7\cdot2^{13}$ \& $2^{16}$. (c) $21^{15}$ \& $27^5\cdot49^8$.
\end{baitoan}

\begin{baitoan}[\cite{Tuyen_Toan_6}, 53., p. 14]
	So sánh: (a) $199^{20}$ \& $2003^{15}$. (b) $3^{39}$ \& $11^{21}$.
\end{baitoan}

\begin{baitoan}[\cite{Tuyen_Toan_6}, 54., p. 14]
	So sánh 2 hiệu $A = 72^{45} - 72^{44}$ \& $B = 72^{44} - 72^{43}$.
\end{baitoan}

\begin{baitoan}[\cite{Tuyen_Toan_6}, 55., p. 14]
	Tìm $x\in\mathbb{N}$ biết: (a) $16^x < 128^4$. (b) $5^x\cdot5^{x + 1}\cdot5^{x + 2}\le1\underbrace{00\ldots0}_{18}:2^{18}$.
\end{baitoan}

\begin{baitoan}[\cite{Tuyen_Toan_6}, 56., p. 14]
	Tìm $n\in\mathbb{N}$ biết $2^5\cdot3^n\cdot3^{n + 2}\le32\cdot3^6\cdot3^4$.
\end{baitoan}

\begin{baitoan}[\cite{Tuyen_Toan_6}, 57., p. 14]
	So sánh $A = \sum_{i=0}^9 2^i = 1 + 2 + 2^2 + 2^3 + \cdots + 2^9$ \& $B = 5\cdot2^8$.
\end{baitoan}

\begin{baitoan}[\cite{Tuyen_Toan_6}, 58., p. 14]
	Viết số lớn nhất bằng cách dùng $3$ chữ số $1,2,3$ với điều kiện mỗi chữ số chỉ dùng 1 lần.
\end{baitoan}

%------------------------------------------------------------------------------%

\subsection{Last Digit of Products \& Exponentiations -- Chữ Số Tận Cùng của Các Tích \& Lũy Thừa}

\begin{baitoan}[\cite{Tuyen_Toan_6}, VD11, p. 15]
	Cho tổng $A = 9531^m + 246^n$ với $m,n\in\mathbb{N}^\star$. Hỏi tổng $A$ có phải là số chính phương không?
\end{baitoan}

\begin{baitoan}[\cite{Tuyen_Toan_6}, VD12, p. 15]
	Cho $B = 559^{361} - 7^{202}$. Chứng minh $B\divby10$.
\end{baitoan}

\begin{baitoan}[\cite{Tuyen_Toan_6}, VD13, pp. 15--16]
	Ngoài Dương lịch, Âm lịch, còn ghi lịch theo hệ đếm Can Chi, e.g., Nhâm Ngọ, Quý Mùi, Giáp Thân, $\ldots$ Chữ thứ nhất chỉ hàng Can, chữ thứ 2 chỉ hàng Chi. Có $10$ Can là: Giáp, Ất, Bính, Đinh, Mậu, Kỷ, Canh, Tân, Nhâm, Quý. Có $12$ Chi là: Tý, Sửu, Dần, Mão, Thìn, Tỵ, Ngọ, Mùi, Thân, Dậu, Tuất, Hợi. Muốn tìm hàng Can của 1 năm ta chỉ cần xét chữ số tận cùng của năm dương lịch rồi đối chiếu với bảng:
	\begin{table}[H]
		\centering
		\begin{tabular}{|c|c|c|c|c|c|c|c|c|c|c|}
			\hline
			Hàng can & Giáp & Ất & Bính & Đinh & Mậu & Kỷ & Canh & Tân & Nhâm & Quý \\
			\hline
			Chữ số tận cùng của năm dương lịch & 4 & 5 & 6 & 7 & 8 & 9 & 0 & 1 & 2 & 3 \\
			\hline
		\end{tabular}
	\end{table}
	\noindent Muốn tìm hàng Chi của 1 năm ta dùng công thức
	\begin{align*}
		\mbox{Hàng Chi} = \mbox{Số dư của }\frac{\mbox{năm} - 4}{12} + 1.
	\end{align*}
	Rồi đối chiếu kết quả với bảng:
	\begin{table}[H]
		\centering
		\begin{tabular}{|c|c|c|c|c|c|c|c|c|c|c|c|c|}
			\hline
			Hàng chi & Tý & Sửu & Dần & Mão & Thìn & Tỵ & Ngọ & Mùi & Thân & Dậu & Tuất & Hợi \\
			\hline
			Mã số & 1 & 2 & 3 & 4 & 5 & 6 & 7 & 8 & 9 & 10 & 11 & 12 \\
			\hline
		\end{tabular}
	\end{table}
	\noindent Năm 2010 kỷ niệm $1000$ năm Thăng Long--Hà Nội, tính xem năm đó là năm nào theo hệ đếm Can Chi? Chú ý: Vì năm dương lịch không trùng hoàn toàn với năm âm lịch nên đối với 2 tháng đầu của năm dương lịch thì còn phải chỉnh chút ít.
\end{baitoan}

\begin{baitoan}[\cite{Tuyen_Toan_6}, 59., p. 16]
	Tìm chữ số tận cùng của tổng $A = 1\cdot3\cdot5\cdots99 + 2\cdot4\cdot6\cdots98$.
\end{baitoan}

\begin{baitoan}[\cite{Tuyen_Toan_6}, 60., p. 16]
	Có 5 số tự nhiên nào mà tích bằng $2021$ \& tổng có tận cùng bằng $8$ không?
\end{baitoan}

\begin{baitoan}[\cite{Tuyen_Toan_6}, 61., p. 16]
	Tích các số lẻ liên tiếp có tận cùng là $7$. Hỏi tích đó có bao nhiêu thừa số?
\end{baitoan}

\begin{baitoan}[\cite{Tuyen_Toan_6}, 62., p. 16]
	Tìm chữ số tận cùng của các lũy thừa: $87^{32},58^{33},23^{35},74^{30},49^{31}$.
\end{baitoan}

\begin{baitoan}[\cite{Tuyen_Toan_6}, 63., p. 16]
	Chia mỗi số sau cho $100$ được số dư là bao nhiêu? (a) $7^{2025}$. (b) $6^{1202}$.
\end{baitoan}

\begin{baitoan}[\cite{Tuyen_Toan_6}, 64., p. 16]
	Tìm chữ số tận cùng của các số sau: (a) $234^{5^{6^7}}$. (b) $6^{1202}$.
\end{baitoan}

\begin{baitoan}[\cite{Tuyen_Toan_6}, 65., p. 16]
	Chứng minh các tổng hoặc hiệu sau không chia hết cho $10$: (a) $A = 98\cdot96\cdot94\cdot92 - 91\cdot93\cdot95\cdot97$. (b) $B(m,n) = 405^n + 2^{405} + m^2$, với $\forall m,n\in\mathbb{N}$, $n\ne0$.
\end{baitoan}

\begin{baitoan}[\cite{Tuyen_Toan_6}, 66., p. 17]
	Cho $P = \left(\prod_{i=1}^{10} 2^i\right)\left(\prod_{i=1}^7 5^{2i}\right) = (2\cdot2^2\cdot2^3\cdots2^{10})(5^2\cdot5^4\cdot5^6\cdots5^{14})$ tận cùng bằng bao nhiêu chữ số $0$?
\end{baitoan}

\begin{baitoan}[\cite{Tuyen_Toan_6}, 67., p. 17]
	Cho $S = \sum_{i=0}^{30} 3^i = 1 + 3 + 3^2 + 3^3 + \cdots + 3^{30}$. Tìm chữ số tận cùng của $S$, từ đó suy ra $S$ không phải là số chính phương.
\end{baitoan}

\begin{baitoan}[\cite{Tuyen_Toan_6}, 68., p. 17]
	Nước Việt Nam Dân chủ Cộng hòa ra đời ngay sau Cách mạng tháng 8 năm 1945. Trong hệ đếm Can Chi năm 1945 là năm nào?
\end{baitoan}

\begin{baitoan}[\cite{Tuyen_Toan_6}, 69., p. 17]
	Chiến thắng Đống Đa giải phóng Thăng Long vào mùa xuân năm 1789. Năm đó là năm nào trong hệ đếm Can Chi?
\end{baitoan}

%------------------------------------------------------------------------------%

\section{Problem on Naturals -- Bài Toán về Số Tự Nhiên}

\begin{baitoan}[\cite{Binh_Toan_6_tap_1}, VD18, p. 17]
	Tuổi anh hiện nay gấp 3 lần tuổi em trước kia, lúc anh bằng tuổi em hiện nay. Khi tuổi em bằng tuổi anh hiện nay thì tổng số tuổi của 2 người sẽ là $28$. Tính tuổi của mỗi người hiện nay.
\end{baitoan}

\begin{baitoan}[\cite{Binh_Toan_6_tap_1}, VD19, p. 18]
	3 ôtô chở tổng cộng $50$ chuyến, gồm $118$ tấn hàng. Mỗi chuyến, xe thứ nhất chở $2$ tấn, xe thứ 2 chở $2.5$ tấn, xe thứ 3 chở $3$ tấn. Hỏi mỗi xe chở bao nhiêu chuyến biết số chuyến xe thứ nhất gấp rưỡi số chuyến xe thứ 2?
\end{baitoan}

\begin{baitoan}[\cite{Binh_Toan_6_tap_1}, VD20, p. 19]
	Anh Lâm nói: ``Năm 1990, tuổi mình đúng bằng tổng các chữ số của năm sinh''. Tính xem anh Lâm sinh năm nào.
\end{baitoan}

\subsection{Sum, difference, \& ratio -- Tìm các số biết tổng \& các hiệu, biết tổng (hiệu) \& các tỷ số}

\begin{baitoan}[\cite{Binh_Toan_6_tap_1}, 84., p. 19]
	Lương Thế Vinh là nhà toán học nổi tiếng của nước ta thời xưa. Năm sinh của ông rất đặc biệt, đó là 1 số có 4 chữ số, chữ số hàng nghìn bằng chữ số hàng đơn vị, chữ số hàng trăm bằng chữ số hàng chục \& tổng của 4 chữ số bằng $10$. Tính năm sinh của ông.
\end{baitoan}

\begin{baitoan}[\cite{Binh_Toan_6_tap_1}, 85., p. 19]
	Tìm 4 số tự nhiên chẵn liên tiếp có tổng bằng $5420$.
\end{baitoan}

\begin{baitoan}[\cite{Binh_Toan_6_tap_1}, 86., p. 19]
	Tìm 3 số biết: Tổng của số thứ nhất \& số thứ 2 bằng $56$, tổng của số thứ 2 \& số thứ 3 bằng $64$, tổng của số thứ 3 \& số thứ nhất bằng $78$.
\end{baitoan}

\begin{baitoan}[\cite{Binh_Toan_6_tap_1}, 87., p. 19]
	Tìm 3 số tự nhiên lẻ liên tiếp biết tổng của số lớn nhất \& số nhỏ nhất bằng $114$.
\end{baitoan}

\begin{baitoan}[\cite{Binh_Toan_6_tap_1}, 88., p. 19]
	2 ngăn sách lúc đầu có tổng cộng $118$ cuốn. Sau khi lấy đi $8$ cuốn ở ngăn {\rm I}, thêm $10$ cuốn vào ngăn {\rm II} thì số sách ở ngăn {\rm II} gấp đôi số sách ở ngăn {\rm I}. Tính số sách ở mỗi ngăn lúc đầu.
\end{baitoan}

\begin{baitoan}[\cite{Binh_Toan_6_tap_1}, 89., p. 19]
	Tìm số tự nhiên tận cùng bằng $7$ biết sau khi xóa chữ số $7$ đó thì số ấy giảm đi $484$ đơn vị.
\end{baitoan}

\begin{baitoan}[\cite{Binh_Toan_6_tap_1}, 90., p. 19]
	Hiệu của 2 số bằng $1217$. Nếu tăng số trừ gấp 4 lần thì được số lớn hơn số bị trừ là $376$. Tìm số bị trừ, số trừ.
\end{baitoan}

\begin{baitoan}[\cite{Binh_Toan_6_tap_1}, 91., p. 19]
	1 vườn hình chữ nhật có chu vi {\rm356 m}. Tính chiều dài \& chiều rộng của vườn biết nếu viết thêm chữ số $1$ vào trước số đo chiều dài thì được số đo chiều dài.
\end{baitoan}

\begin{baitoan}[\cite{Binh_Toan_6_tap_1}, 92., pp. 19--20]
	Bài toán cổ Hy Lạp:
	\begin{center}
		Lừa \& ngựa thồ hàng ra chợ,\\Ngựa thở than mình chở quá nhiều.\\Lừa rằng: ``Anh chớ lắm điều!\\Tôi đây mới bị chất nhiều làm sao!\\Anh đưa tôi 1 bao mang bớt\\Thì tôi thồ nhiều gấp đôi anh\\Chính tôi phải trút cho anh\\1 bao gánh đỡ mới thành bằng nhau''.
	\end{center}
	Hỏi lừa, ngựa chở mấy bao?
\end{baitoan}

\begin{baitoan}[\cite{Binh_Toan_6_tap_1}, 93., p. 20]
	Tìm số bị chia \& số chia của 1 phép chia, biết thương bằng $6$, số dư bằng $49$, tổng của số bị chia, số chia \& số dư bằng $595$.
\end{baitoan}

\begin{baitoan}[\cite{Binh_Toan_6_tap_1}, 94., p. 20]
	Tìm số tự nhiên biết nếu viết thêm chữ số $2$ vào sau chữ số hàng đơn vị thì số ấy tăng thêm $2000$ đơn vị.
\end{baitoan}

\begin{baitoan}[\cite{Binh_Toan_6_tap_1}, 95., p. 20]
	Mẹ hơn con $28$ tuổi. Sau $5$ năm nữa, tuổi mẹ gấp 3 tuổi con. Tính tuổi mẹ \& con hiện nay.
\end{baitoan}

\begin{baitoan}[\cite{Binh_Toan_6_tap_1}, 96., p. 20]
	Con $10$ tuổi, bố $40$ tuổi. Sau mấy năm nữa, tuổi bố gấp 3 tuổi con?
\end{baitoan}

\begin{baitoan}[\cite{Binh_Toan_6_tap_1}, 97., p. 20]
	Năm 2000, bố $40$ tuổi, Mai $11$ tuổi, em Nam $5$ tuổi. Đến năm nào, tuổi bố bằng tổng số tuổi của 2 chị em?
\end{baitoan}

\begin{baitoan}[\cite{Binh_Toan_6_tap_1}, 98., p. 20]
	Năm 2000, mẹ $36$ tuổi, 2 con $7$ tuổi \& $12$ tuổi. Bắt đầu từ năm nào, tuổi mẹ ít hơn tổng số tuổi của 2 con?
\end{baitoan}

\begin{baitoan}[\cite{Binh_Toan_6_tap_1}, 99., p. 20]
	Anh hơn em $3$ tuổi. Tuổi anh hiện nay gấp rưỡi tuổi em, lúc anh bằng tuổi em hiện nay. Tính tuổi hiện nay của mỗi người.
\end{baitoan}

\subsection{Additional hypothesis -- Giải toán bằng phương pháp giả thiết tạm}

\begin{baitoan}[\cite{Binh_Toan_6_tap_1}, 100., p. 20]
	1 số học sinh xếp hàng $12$ thì thừa $5$ học sinh, còn xếp hàng $15$ cũng thừa $5$ học sinh \& ít hơn trước là $4$ hàng. Tính số học sinh.
\end{baitoan}

\begin{baitoan}[\cite{Binh_Toan_6_tap_1}, 101., p. 20]
	Có 1 số học sinh \& 1 số thuyền. Nếu xếp $4$ học sinh 1 thuyền thì thừa $3$ học sinh chưa có chỗ. Nếu xếp $5$ học sinh 1 thuyền thì thừa ra 1 thuyền. Tính số học sinh \& số thuyền.
\end{baitoan}

\begin{baitoan}[\cite{Binh_Toan_6_tap_1}, 102., p. 20]
	An vào cửa hàng mua $12$ vở \& $4$ bút chì hết $36000$ đồng. Bích mua $8$ vở \& $5$ bút chì cùng loại hết $27500$ đồng. Tính giá 1 quyển vở, giá 1 bút chì.
\end{baitoan}

\begin{baitoan}[\cite{Binh_Toan_6_tap_1}, 103., p. 20]
	Trong 1 đợt trồng cây, lớp 6A trồng $26$ cây, lớp 6B trồng $29$ cây, lớp 6C trồng $32$ cây. Số cây lớp $6D$ trồng nhiều hơn trung bình cộng của 4 lớp là $3$ cây. Tính số cây lớp 6D trồng.
\end{baitoan}

\begin{baitoan}[\cite{Binh_Toan_6_tap_1}, 104., p. 20]
	Bơm nước vào 1 bể: dùng máy {\rm I} trong $30$ phút, dùng máy {\rm II} trong $20$ phút. Tính xem trong mỗi phút mỗi máy bơm được bao nhiêu lít nước, biết mỗi phút máy {\rm II} bơm được nhiều hơn máy {\rm I} là {\rm50 L} \& tổng cộng 2 máy bơm được {\rm21000 L} nước?
\end{baitoan}

\begin{baitoan}[\cite{Binh_Toan_6_tap_1}, 105., p. 20]
	1 tổ may phải may $1800$ chiếc cả quần \& áo trong $13$ giờ. Trong $8$ giờ đầu tổ may áo \& trong thời gian còn lại tổ may quần. Biết trong $1$ giờ, tổ may được số áo nhiều hơn số quần là $30$ chiếc. Tính số áo \& số quần tổ đã may.
\end{baitoan}

\begin{baitoan}[\cite{Binh_Toan_6_tap_1}, 106., p. 21]
	Đố:
	\begin{center}
		Quýt, cam 17 quả tươi\\Đem chia cho 100 người cùng vui.\\Chia 3 mỗi quả quýt rồi,\\Còn cam mỗi quả chia 10 vừa xinh.\\Trăm người, trăm miếng ngọt lành,\\Quýt, cam mỗi loại tính rành là bao?
	\end{center}
\end{baitoan}

\begin{baitoan}[\cite{Binh_Toan_6_tap_1}, 107., p. 21]
	1 đội bóng thi đấu $25$ trận, chỉ có thắng \& hòa, mỗi trận thắng được $3$ điểm, mỗi trận hòa được $1$ điểm, kết quả đội đó được $59$ điểm. Tính số trận thắng, số trận hòa của đội bóng.
\end{baitoan}

\begin{baitoan}[\cite{Binh_Toan_6_tap_1}, 108., p. 21]
	(a) 1 cuộc thi có $20$ câu hỏi, mỗi câu trả lời đúng được $5$ điểm, mỗi câu trả lời sai bị trừ đi $1$ điểm. 1 đội học sinh dự thi đạt $52$ điểm. Hỏi đội đó trả lời đúng mấy câu, sai mấy câu? (b) 1 người làm gia công $45$ sản phẩm, mỗi chiếc làm đúng quy cách được $800$ đồng, mỗi chiếc làm sai quy cách phải đền $1200$ đồng. Tính ra người đó được lĩnh $30000$ đồng. Hỏi người đó làm bao nhiêu sản phẩm đúng quy cách?
\end{baitoan}

\begin{baitoan}[\cite{Binh_Toan_6_tap_1}, 109., p. 21]
	1 câu lạc bộ có $22$ chiếc ghế gồm 3 loại: ghế 3 chân, ghế 4 chân, ghế 6 chân. Tính số ghế mỗi loại, biết tổng số chân ghế bằng $100$ \& số ghế 6 chân gấp đôi số ghế 3 chân.
\end{baitoan}

\begin{baitoan}[\cite{Binh_Toan_6_tap_1}, 110., p. 21]
	1 số tiền trị giá $224000$ đồng gồm các loại tiền $5000$ đồng, $2000$ đồng, $1000$ đồng, tổng cộng $130$ tờ. Biết số tờ $1000$ đồng gấp 5 số tờ $5000$ đồng. Tính số tờ tiền mỗi loại.
\end{baitoan}

\begin{baitoan}[\cite{Binh_Toan_6_tap_1}, 111., p. 21]
	Có $25$ gói đường gồm 3 loại: gói $5$ lạng, gói $2$ lạng, gói $1$ lạng, có khối lượng tổng cộng là $56$ lạng. Biết số gói $1$ lạng gấp đôi số gói $5$ lạng. Tính số gói mỗi loại.
\end{baitoan}

\begin{baitoan}[\cite{Binh_Toan_6_tap_1}, 112., p. 21]
	1 hộp có thể chứa được vừa vặn $25$ gói bánh hoặc $30$ gói kẹo. Xếp $28$ gói cả bánh \& kẹo thì vừa đầy hộp đó. Biết giá tiền bánh \& kẹo đều bằng nhau \& bằng $36000$ đồng. Tính giá 1 gói bánh, 1 gói kẹo.
\end{baitoan}

\subsection{Selection method -- Toán giải bằng phương pháp lựa chọn}

\begin{baitoan}[\cite{Binh_Toan_6_tap_1}, 113., p. 21]
	Tìm số tự nhiên có 3 chữ số, biết tổng 6 số tự nhiên có 2 chữ số lập bởi 2 trong 3 chữ số ấy gấp đôi số phải tìm.
\end{baitoan}

\begin{baitoan}[\cite{Binh_Toan_6_tap_1}, 114., p. 21]
	(a) Tìm 3 chữ số khác nhau \& khác $0$, biết tổng các số tự nhiên có 3 chữ số gồm cả 3 chữ số ấy bằng $1554$. (b) Tìm 3 chữ số khác nhau \& khác $0$, biết tổng các số tự nhiên có 3 chữ số gồm cả 3 chữ số ấy bằng $2886$, còn hiệu giữa số lớn nhất \& số nhỏ nhất bằng $495$. (c) Có 3 tờ bìa ghi các số $23,79$, \& $\overline{ab}$. Xếp 3 tờ bìa thành 1 hàng thì được số có 6 chữ số. Cộng tất cả các số 6 chữ số đó lại (bằng cách đổi chỗ các tờ bìa) thì được $2989896$. Tìm số $\overline{ab}$.
\end{baitoan}

\begin{baitoan}[\cite{Binh_Toan_6_tap_1}, 115., p. 21]
	Có 5 người cân theo từng cặp 2 người. Số cân nặng (tính bằng {\rm kg}) trong $10$ lượt cân xếp từ lớn đến nhỏ là: $129,125,124,13,122,121,120,118,116,114$. Tính cân nặng của mỗi người.
\end{baitoan}

\begin{baitoan}[\cite{Binh_Toan_6_tap_1}, 116., p. 22]
	Bé Mai nhận thấy nếu ghép tuổi mình với tuổi của em Thu thì được tuổi của bà. Biết tổng số tuổi của 3 bà cháu là $85$. Tính tuổi mỗi người.
\end{baitoan}

\begin{baitoan}[\cite{Binh_Toan_6_tap_1}, 117., p. 22, Đố vui: những năm sinh đặc biệt]
	Ngày đầu năm 1991, bác Nam hỏi anh Việt:
	
	- Năm nay cháu bao nhiêu tuổi rồi?
	
	- Tuổi cháu năm nay đúng bằng tổng các chữ số của năm sinh -- Anh Việt trả lời.
	
	Thế mà bác Nam tính ngay ra tuổi của anh Việt. Bác gật gù nói:
	
	- Lúc bác bằng tuổi cháu hiện nay, bác đang tham gia kháng chiến chống Pháp, \& năm ấy cũng có tổng các chữ số bằng tuổi cháu.
	
	Anh Việt cũng tính đúng tuổi của bác Nam. Hỏi anh Việt \& bác Nam sinh năm nào?
\end{baitoan}

%------------------------------------------------------------------------------%

\section{Order of Operations -- Thứ Tự Thực Hiện Các Phép Tính}
\cite[\S6, pp. 20--21]{SBT_Toan_6_Canh_Dieu_tap_1}: 50. 51. 52. 53. 54. 55. 56. 57.

\begin{baitoan}[\cite{Tuyen_Toan_6}, VD14, p. 17]
	Dùng 5 chữ số $1$ \& dấu của các phép tính kể cả dấu ngoặc để viết thành 1 biểu thức có giá trị bằng $100$.
\end{baitoan}

\begin{baitoan}[\cite{Tuyen_Toan_6}, VD15, p. 17]
	1 quyển sách giá khoa có $172$ trang. Hỏi phải dùng tất cả bao nhiêu chữ số để đánh số các trang của quyển sách này?
\end{baitoan}

\begin{baitoan}[\cite{Tuyen_Toan_6}, VD16, p. 18]
	Cho $S = 8 + 12 + 16 + \cdots + 96 +100$. (a) Tổng này có bao nhiêu số hạng? (b) Tính $S$.
\end{baitoan}

\begin{baitoan}[\cite{Tuyen_Toan_6}, 70., p. 18]
	Tính: (a) $[400 - (40:2^3 + 3\cdot5^3)]:5$. (b) $(37 + 18)\cdot\{3250 - 15^2\cdot[(4^4 - 2^5):16]\}$.
\end{baitoan}

\begin{baitoan}[\cite{Tuyen_Toan_6}, 71., p. 18]
	Tính: (a) $(10^2 + 11^2 + 12^2):(13^2 + 14^2)$. (b) $9! - 8! - 7!\cdot8^2$. (c) $\dfrac{(3\cdot4\cdot2^{16})^2}{11\cdot2^{13}\cdot4^{11} - 16^9}$.
\end{baitoan}

\begin{baitoan}[\cite{Tuyen_Toan_6}, 72., p. 18]
	Tìm $x\in\mathbb{N}$ biết: (a) $(19x + 2\cdot5^2):14 = (13 - 8)^2 - 4^2$. (b) $2\cdot3^x = 10\cdot3^{12} + 8\cdot27^4$. 
\end{baitoan}

\begin{baitoan}[\cite{Tuyen_Toan_6}, 73., p. 18]
	Tìm $x\in\mathbb{N}$ biết: (a) $(5^2\cdot23 - 5^2\cdot13)x = 6\cdot5^3$. (b) $x^2 - [666:(24 + 13)] = 7$.
\end{baitoan}

\begin{baitoan}[\cite{Tuyen_Toan_6}, 74., p. 18]
	Dùng 6 chữ số $1$ cùng với dấu của các phép tính \& dấu ngoặc (nếu cần) để viết thành 1 biểu thức có giá trị là $100$.
\end{baitoan}

\begin{baitoan}[\cite{Tuyen_Toan_6}, 75., p. 18]
	Với 6 chữ số $3$ cũng với dấu của các phép tính \& dấu ngoặc (nếu cần), viết 1 biểu thức có giá trị là $1000000$.
\end{baitoan}

\begin{baitoan}[\cite{Tuyen_Toan_6}, 76., p. 18]
	Cho biểu thức $252 - 84:21 + 7$. (a) Tính giá trị của biểu thức đó. (b) Nếu dùng thêm dấu ngoặc thì có thể được các giá trị nào khác?
\end{baitoan}

\begin{baitoan}[\cite{Tuyen_Toan_6}, 77., p. 18]
	Cho $S = 7 + 10 + 13 + \cdots + 97 + 100$. (a) Tổng trên có bao nhiêu số hạng? (b) Tìm số hạng thứ $22$. (c) Tính $S$.
\end{baitoan}

\begin{baitoan}[\cite{Tuyen_Toan_6}, 78., p. 18]
	Cho $A$ là tập hợp các số tự nhiên không vượt quá $150$, chia cho $7$ dư $3$. $A = \{x\in\mathbb{N}|x = 7q + 3,\ q\in\mathbb{N},\,x\le150\}$. (a) Liệt kê các phần tử của $A$ thành 1 dãy số từ nhỏ đến lớn. (b) Tính tổng các phần tử của $A$.
\end{baitoan}

\begin{baitoan}[\cite{Tuyen_Toan_6}, 79., p. 18]
	1 quyển sách có $366$ trang. Để đánh số các trang của quyển sách này phải dùng bao nhiêu chữ số?
\end{baitoan}

\begin{baitoan}[\cite{Tuyen_Toan_6}, 80., p. 18]
	Để đánh số các trang của 1 quyển sách phải dùng tất cả $600$ chữ số. Hỏi quyển sách đó có bao nhiêu trang?
\end{baitoan}

\begin{baitoan}[\cite{Tuyen_Toan_6}, 81., p. 19]
	Viết liền nhau dãy các số tự nhiên bắt đầu từ $1$: $1,2,3,\ldots$ Hỏi chữ số thứ $659$ là chữ số nào?
\end{baitoan}

%------------------------------------------------------------------------------%

\section{Cân \& Đong với 1 Số Điều Kiện Hạn Chế}

\begin{baitoan}[\cite{Tuyen_Toan_6}, VD17, p. 19]
	Trong 9 gói hàng có 1 gói nhẹ hơn các gói kia chút ít, còn lại nặng như nhau. Với 1 chiếc cân 2 đĩa \& không có quả cân nào, cân chỉ 2 lần tìm được gói hàng nhẹ hơn.
\end{baitoan}

\begin{baitoan}[Tổng quát \cite{Tuyen_Toan_6}, VD17, p. 19]
	Trong $3^n$ gói hàng có 1 gói nhẹ hơn các gói kia chút ít, còn lại nặng như nhau. Với 1 chiếc cân 2 đĩa \& không có quả cân nào, cân chỉ $n$ lần tìm được gói hàng nhẹ hơn.
\end{baitoan}

\begin{baitoan}[\cite{Tuyen_Toan_6}, VD18, p. 19]
	Có 10 thùng đựng các gói muối trong đó có 9 thùng đựng các gói muối {\rm100 g}, chỉ có 1 thùng đựng các gói không đúng quy cách, mỗi gói chỉ có {\rm90 g}. Làm thế nào chỉ cân 1 lần mà xác định được thùng có các gói muối {\rm90 g}?
\end{baitoan}

\begin{baitoan}[\cite{Tuyen_Toan_6}, VD19, p. 20]
	Có 1 bình {\rm4 L} \& 1 bình {\rm5 L}. Làm thế nào để đong được đúng {\rm3 L} nước từ 1 bể nước?
\end{baitoan}

\begin{baitoan}[\cite{Tuyen_Toan_6}, VD20, p. 20]
	Có 1 bình {\rm5 L} \& 1 bình {\rm3 L}, làm thế nào để đong được đúng {\rm7 L} nước từ 1 bể nước?
\end{baitoan}

\begin{baitoan}[\cite{Tuyen_Toan_6}, 82., p. 21]
	Có {\rm1500 g} đường. Làm thế nào để lấy ra được {\rm250 g} với 1 chiếc cân 2 đĩa \& 1 quả cân {\rm100 g}.
\end{baitoan}

\begin{baitoan}[\cite{Tuyen_Toan_6}, 83., p. 21]
	Trong $27$ chiếc nhẫn có 1 chiếc nặng hơn các chiếc kia chút ít, các chiếc còn lại nặng như nhau. Với 1 chiếc cân 2 đĩa \& không có quả cân, làm thế nào để cân đúng 3 lần xác định được chiếc nhẫn nặng hơn đó.
\end{baitoan}

\begin{baitoan}[\cite{Tuyen_Toan_6}, 84., p. 21]
	Trong $26$ chiếc tẩy có 1 chiếc nặng hơn các chiếc còn lại chút ít, các chiếc còn lại nặng như nhau. Với chiếc cân đĩa \& không có quả cân, làm thế nào để cân đúng $3$ lần xác định được chiếc tẩy nặng hơn đó.
\end{baitoan}

\begin{baitoan}[\cite{Tuyen_Toan_6}, 85., p. 21]
	Phải dùng ít nhất bao nhiêu quả cân để cân tất cả các vật có khối lượng là 1 số tự nhiên từ {\rm1 g} đến {\rm100 g} (các vật chỉ đặt trên 1 đĩa cân).
\end{baitoan}

\begin{baitoan}[\cite{Tuyen_Toan_6}, 86., p. 21]
	1 thùng nước lọc còn hơn {\rm7 L}. Làm thế nào để lấy ra được đúng {\rm4 L} chỉ bằng 1 bình loại {\rm5 L} \& 1 bình loại {\rm3 L}?
\end{baitoan}

\begin{baitoan}[\cite{Tuyen_Toan_6}, 87., p. 21]
	1 thùng dầu còn hơn {\rm10 L}. Dùng 1 can {\rm7 L} \& 1 can {\rm5 L} để lấy ra đúng {\rm8 L} dầu.
\end{baitoan}

\begin{baitoan}[\cite{Tuyen_Toan_6}, 88., p. 21]
	1 thùng có {\rm16 L} nước. Dùng 1 can {\rm7 L} \& 1 can {\rm3 L} để chia {\rm16 L} nước làm 2 phần bằng nhau.
\end{baitoan}

\begin{baitoan}[\cite{Tuyen_Toan_6}, 89., p. 21]
	1 can A đựng {\rm12 L} nước. Dùng 1 can B loại {\rm7 L} \& 1 can C loại {\rm5 L} để chia {\rm12 L} trong can A thành 3 phần: {\rm3 L, 4 L, 5 L}.
\end{baitoan}

%------------------------------------------------------------------------------%

\section{Bài Toán Thực Tế}

\begin{baitoan}[\cite{TLCT_THCS_Toan_6_so_hoc}, VD2.1, p. 14]
	Ông Toàn đi công tác trở về nhà thì chiếc đồng hồ lên dây cót của ông đã đứng. Ông lên dây cót, vặn kim đồng hồ chỉ {\rm8:00} rồi sang ngay nhà bạn gần đó để chơi \& hỏi giờ. Trên đường đi, ông phát hiện mình không mang theo đồng hồ. Do đó ông đã ghi lại lúc vừa đến nhà bạn là {\rm8:20} \& lúc bắt đầu rời nhà bạn để về nhà mình là {\rm8:50}. Khi về đến nhà, ông thấy đồng hồ của mình chỉ {\rm8:50}. Hỏi ông phải chỉnh đồng hồ của mình để kim đồng hồ chỉ mấy giờ?
\end{baitoan}

\begin{baitoan}[\cite{TLCT_THCS_Toan_6_so_hoc}, VD2.2, p. 14]
	An về nghỉ hè ở quê trong 1 số ngày, trong đó có $10$ ngày mưa. Biết có $11$ buổi sáng không mưa, có $9$ buổi chiều không mưa \& không bao giờ trời mưa cả sáng lẫn chiều. Hỏi An về nghỉ ở quê trong bao nhiêu ngày?
\end{baitoan}

\begin{baitoan}[\cite{TLCT_THCS_Toan_6_so_hoc}, VD2.3, p. 15]
	1 số học sinh dự thi học sinh giỏi toán. Nếu xếp $25$ học sinh 1 phòng thì thừa $5$ học sinh chưa có chỗ. Nếu xếp $28$ học sinh 1 phòng thì thừa 1 phòng. Tính số học sinh dự thi.
\end{baitoan}

\begin{baitoan}[\cite{TLCT_THCS_Toan_6_so_hoc}, VD2.4, p. 15]
	Trong 1 bảng đấu loại bóng đá, có 4 đội thi đấu vòng tròn 1 lượt: đội thắng được $3$ điểm, đội hòa được $1$ điểm, đội thua được $0$ điểm. Tổng số điểm của 4 đội khi kết thúc vòng đấu bảng là $16$ điểm. Tính số trận hòa.
\end{baitoan}

\begin{baitoan}[\cite{TLCT_THCS_Toan_6_so_hoc}, VD2.5, p. 16]
	1 câu lạc bộ lúc đầu có 1 thành viên, sau 1 tháng thì thành viên đó phải tìm thêm 2 thành viên mới. Cứ như vậy, mỗi thành viên (cả cũ lẫn mới) sau 1 tháng phải tìm được thêm 2 thành viên mới. Nếu kế hoạch phát triển hội viên như trên được thực hiện thì số thành viên của câu lạc bộ đó là bao nhiêu: (a) Sau 6 tháng? (b) Sau 12 tháng?
\end{baitoan}

\begin{baitoan}[\cite{TLCT_THCS_Toan_6_so_hoc}, VD2.6, p. 16]
	Tính số sách Toán bán được trong mỗi ngày của 1 cửa hàng biết số sách đã bán ra như sau: Thứ 2, thứ 3, thứ 4: $115$ quyển. Thứ 4, thứ 5: $85$ quyển. Thứ 3, thứ 5: $90$ quyển. Thứ 2, thứ 6: $70$ quyển. Thứ 5, thứ 6: $80$ quyển.
\end{baitoan}

\begin{baitoan}[\cite{TLCT_THCS_Toan_6_so_hoc}, VD2.7, p. 17, Bài toán bò ăn cỏ của Newton]
	Trên 1 cánh đồng cỏ, cỏ mọc đều như nhau \& lớn đều như nhau. Biết $70$ con bò ăn hết số cỏ có sẵn \& số cỏ mọc thêm trên cánh đồng ấy trong $24$ ngày, nếu có $30$ con bò thì chúng ăn hết cỏ trong $60$ ngày. (a) Gọi số cỏ 1 con bò ăn trong 1 ngày là 1 bó. Hỏi số cỏ mọc thêm trên cánh đồng trong $36$ ngày là bao nhiêu bó? (b) Bao nhiêu con bò sẽ ăn hết cỏ của cánh đồng trong $96$ ngày?
\end{baitoan}

\begin{baitoan}[\cite{TLCT_THCS_Toan_6_so_hoc}, 2.1., p. 18]
	1 cửa hàng mua 1 xe ôtô giá $1500$ triệu đồng, đem cho thuê $20$ tuần với giá cho thuê $30$ triệu đồng 1 tuần. Phí bảo hiểm cửa hàng phải nộp là $80$ triệu đồng, chi phí sửa chữa hết $120$ triệu đồng. Sau đó cửa hàng bán chiếc xe với giá $1300$ triệu đồng. Tính lợi nhận của thương vụ này.
\end{baitoan}

\begin{baitoan}[\cite{TLCT_THCS_Toan_6_so_hoc}, 2.2., p. 18]
	1 vòng xích có đường kính ngoài là {\rm40 mm}, độ dày của kim loại là {\rm3 mm}. Có $10$ vòng xích được nối với nhau. Tính chiều dài lớn nhất của dây xích.
\end{baitoan}

\begin{baitoan}[\cite{TLCT_THCS_Toan_6_so_hoc}, 2.3., p. 18]
	An \& Bích làm việc tại cùng 1 nhà máy. Cứ sau $9$ ngày làm việc thì An nghỉ $1$ ngày. Cứ sau $6$ ngày làm việc thì Bích nghỉ $1$ ngày. Hôm nay là ngày nghỉ của An \& ngày mai là ngày nghỉ của Bích. Hỏi sau ít nhất bao lâu (kể từ hôm nay) thì cả 2 người sẽ có cùng ngày nghỉ?
\end{baitoan}

\begin{baitoan}[\cite{TLCT_THCS_Toan_6_so_hoc}, 2.4., p. 18]
	Có $10$ người, tuổi của mỗi người là 1 số tự nhiên. Tổng số tuổi của $9$ người trong $10$ người đó là $82,83,84,85,87,89,90,91,92$. Tìm tuổi của người trẻ nhất, tuổi của người già nhất.	
\end{baitoan}

\begin{baitoan}[\cite{TLCT_THCS_Toan_6_so_hoc}, 2.5., p. 18]
	Ta gọi {\rm số đối xứng} là số mà viết các chữ số của nó theo thứ tự ngược lại lẫn được chính số đó, e.g., $353,1221,\ldots$ Đồng hồ đo quãng đường của 1 xe máy chỉ số $15951$. Tìm số đối xứng nhỏ nhất tiếp theo xuất hiện trên mặt đồng hồ.	
\end{baitoan}

\begin{baitoan}[\cite{TLCT_THCS_Toan_6_so_hoc}, 2.6., p. 18]
	Có 1 số con mèo chui vào chuồng bồ câu. Đếm trong chuồng thấy tổng cộng có $34$ cái đầu \& $80$ cái chân. Tính số mèo.
\end{baitoan}

\begin{baitoan}[\cite{TLCT_THCS_Toan_6_so_hoc}, 2.7., p. 18]
	Ở 1 bến cảng có $15$ con tàu, mỗi con tàu có $3$ cột buồm hoặc $5$ cột buồm, tổng cộng có $61$ cột buồm. Hỏi có bao nhiêu con tàu có $5$ cột buồm?
\end{baitoan}

\begin{baitoan}[\cite{TLCT_THCS_Toan_6_so_hoc}, 2.8., p. 18]
	Đội tuyển của 1 trường dự 1 cuộc thi đấu được chia đều thành 6 nhóm, mỗi học sinh dự thi đạt $8$ điểm hoặc $10$ điểm. Tổng số điểm của cả đội là $160$ điểm. Tính số học sinh đạt điểm $10$.
\end{baitoan}

\begin{baitoan}[\cite{TLCT_THCS_Toan_6_so_hoc}, 2.9., p. 18]
	Có $64$ bạn tham gia giải bóng bàn theo thể thức đấu loại trực tiếp. Những người được chọn ở mỗi vòng chia thành từng nhóm 2 người, 2 người trong nhóm đấu với nhau 1 trận để chọn lấy 1 người. Tìm số trận đấu ở: (a) Vòng 1. (b) Vòng 5.
\end{baitoan}

\begin{baitoan}[\cite{TLCT_THCS_Toan_6_so_hoc}, 2.10., p. 18]
	Tâm có $5$ tờ tiền mệnh giá $2000$ đồng \& $4$ tờ tiền mệnh giá $5000$ đồng. Tâm có bao nhiêu cách khác nhau để trả tiền bằng cách dùng 1 hoặc cả 2 loại tiền trên?
\end{baitoan}

\begin{baitoan}[\cite{TLCT_THCS_Toan_6_so_hoc}, 2.11., p. 19]
	Có $40$ bạn lớp 6A \& $30$ bạn lớp 6B xếp hàng đôi để vào tham quan Viện bảo tàng. Gọi $a$ là số trường hợp 2 bạn lớp 6A xếp cùng hàng đôi, gọi $b$ là số trường hợp 2 bạn lớp 6B xếp cùng hàng đôi. So sánh $a,b$, số nào lớn hơn, \& lớn hơn bao nhiêu?
\end{baitoan}

\begin{baitoan}[\cite{TLCT_THCS_Toan_6_so_hoc}, 2.12., p. 19]
	Tờ lịch của 1 tháng:
	\begin{table}[H]
		\centering
		\begin{tabular}{|c|c|c|c|c|c|c|}
			\hline
			S & M & T & W & Th & F & Sa \\
			\hline
			&  &  &  &  & 1 & 2 \\
			3 & 4 & 5 & 6 & 7 & 8 & 9 \\
			10 & 11 & 12 & 13 & 14 & 15 & 16 \\
			17 & 18 & 19 & 20 & 21 & 22 & 23 \\
			24 & 25 & 26 & 27 & 28 & 29 & 30 \\
			\hline
		\end{tabular}
	\end{table}
	\noindent Biết 1 bảng $3\times3$ của 1 tờ lịch khác có tổng 9 số trong bảng là $162$. (a) Tính số ở chính giữa của bảng đó. (b) Lập bảng $3\times3$ đó.
\end{baitoan}

%------------------------------------------------------------------------------%

\section{Miscellaneous}

\begin{baitoan}[\cite{TLCT_THCS_Toan_6_so_hoc}, VD1.1, p. 6]
	Xét $100$ số tự nhiên đầu tiên $0,1,2,\ldots,99$. Tìm $k\in\mathbb{N}$ sao cho trong $100$ số này, có nhiều nhất các số có tổng các chữ số bằng $k$.
\end{baitoan}

\begin{baitoan}[Mở rộng \cite{TLCT_THCS_Toan_6_so_hoc}, VD1.1, p. 6]
	Xét $n$ số tự nhiên đầu tiên $0,1,2,\ldots,n - 1$. Tìm $k\in\mathbb{N}$ sao cho trong $n$ số này, có nhiều nhất các số có tổng các chữ số bằng $k$.
\end{baitoan}

\begin{baitoan}[\cite{TLCT_THCS_Toan_6_so_hoc}, VD1.2, p. 6]
	Tính: (a) $a = 234\cdot\underbrace{9\ldots9}_{50}$. (b) $b = \underbrace{1\ldots1}_{100}\cdot3456$.
\end{baitoan}

\begin{baitoan}[\cite{TLCT_THCS_Toan_6_so_hoc}, VD1.3, p. 7]
	Tính hiệu $a - b$ biết $a = 1\cdot2 + 2\cdot3 + 3\cdot4 + \cdots + 98\cdot99$, $b = 1^2 + 2^2 + 3^2 + \cdots + 98^2$.
\end{baitoan}

\begin{baitoan}[\cite{TLCT_THCS_Toan_6_so_hoc}, VD1.4, p. 7]
	Tìm $x,y\in\mathbb{R}$ thỏa $2^x + 2^y = 20$.
\end{baitoan}

\begin{baitoan}[\cite{TLCT_THCS_Toan_6_so_hoc}, VD1.5, p. 8]
	Trong 1 phép chia có dư, số bị chia bằng $24$, thương bằng $3$. Tìm số chia \& số dư.
\end{baitoan}

\begin{baitoan}[\cite{TLCT_THCS_Toan_6_so_hoc}, VD1.6, p. 8]
	Tìm số tự nhiên có 2 chữ số biết nếu viết thêm chữ số $4$ vào trước chữ số hàng chục thì được số $a$, nếu viết thêm chữ số $8$ vào sau chữ số hàng đơn vị thì được số $b$, trong đó $b$ gấp đôi $a$.
\end{baitoan}

\begin{baitoan}[\cite{TLCT_THCS_Toan_6_so_hoc}, VD1.7, p. 9]
	Điền chữ số thỏa mãn cả 2 phép cộng: $\rm one + one + one + one = four$, $\rm four + one = five$.
\end{baitoan}

\begin{baitoan}[\cite{TLCT_THCS_Toan_6_so_hoc}, VD1.8, p. 10]
	Tìm chữ số tận cùng của $3^{2015}$.
\end{baitoan}

\begin{baitoan}[\cite{TLCT_THCS_Toan_6_so_hoc}, VD1.9, p. 11]
	Tìm 2 chữ số tận cùng của $6^{2011}$.
\end{baitoan}

\begin{baitoan}[\cite{TLCT_THCS_Toan_6_so_hoc}, 1.1., p. 11]
	Tính giá trị của biểu thức (tính nhanh nếu có thể): (a) $215\cdot62 + 42 - 52\cdot215$. (b) $14\cdot29 + 14\cdot71 + (1 + 2 + 3 + \cdots + 99)(199199\cdot198 - 198198\cdot199)$.
\end{baitoan}

\begin{baitoan}[\cite{TLCT_THCS_Toan_6_so_hoc}, 1.2., p. 11]
	Đánh số trang của 1 cuốn sách bằng dãy số tự nhiên $1,2,3,\ldots$ (a) Nếu quyển sách có $180$ trang thì phải viết tất cả bao nhiêu chữ số? (b) Nếu phải viết tất cả $327$ chữ số thì quyển sách có bao nhiêu trang?
\end{baitoan}

\begin{baitoan}[\cite{TLCT_THCS_Toan_6_so_hoc}, 1.3., p. 11]
	Tìm 2 số tự nhiên biết tổng của chúng gấp 3 hiệu của chúng \& bằng tích của chúng.
\end{baitoan}

\begin{baitoan}[\cite{TLCT_THCS_Toan_6_so_hoc}, 1.4., p. 11]
	Tìm số tự nhiên lớn nhất có $3$ chữ biết khi chia nó cho $69$ thì thương \& số dư bằng nhau.
\end{baitoan}

\begin{baitoan}[\cite{TLCT_THCS_Toan_6_so_hoc}, 1.5., p. 11]
	Tìm số dư của phép chia số $\underbrace{1\ldots1}_{100}$ cho $1001$.
\end{baitoan}

\begin{baitoan}[\cite{TLCT_THCS_Toan_6_so_hoc}, 1.6., p. 11]
	Điền chữ số thích hợp để $\overline{8aba} + \overline{c25d} = \overline{d52c}$.
\end{baitoan}

\begin{baitoan}[\cite{TLCT_THCS_Toan_6_so_hoc}, 1.7., p. 11]
	Điền chữ số $\overline{abc} + \overline{bca}$ sao cho tổng trên là lớn nhất \& $a,b,c$ nhận các giá trị $1,2,3$ (không nhất thiết tương ứng).
\end{baitoan}

\begin{baitoan}[\cite{TLCT_THCS_Toan_6_so_hoc}, 1.8., p. 12]
	Điền chữ số thích hợp để $\overline{aa} + \overline{bb} + \overline{cc} = \overline{bac}$.
\end{baitoan}

\begin{baitoan}[\cite{TLCT_THCS_Toan_6_so_hoc}, 1.9., p. 12]
	Tìm chữ số thích hợp để $\overline{xyz1} = \overline{ab}\cdot\overline{ba}$.
\end{baitoan}

\begin{baitoan}[\cite{TLCT_THCS_Toan_6_so_hoc}, 1.10., p. 12]
	Có bao nhiêu số tự nhiên có 2 chữ số mà chữ số hàng chục nhỏ hơn chữ số hàng đơn vị?
\end{baitoan}

\begin{baitoan}[\cite{TLCT_THCS_Toan_6_so_hoc}, 1.11., p. 12]
	Có bao nhiêu số tự nhiên có 3 chữ số, trong đó có ít nhất 2 chữ số giống nhau?
\end{baitoan}

\begin{baitoan}[\cite{TLCT_THCS_Toan_6_so_hoc}, 1.12., p. 12]
	Trong các số tự nhiên từ $1$ đến $500$, có bao nhiêu số có ít nhất 1 chữ số $5$?
\end{baitoan}

\begin{baitoan}[\cite{TLCT_THCS_Toan_6_so_hoc}, 1.13., p. 12]
	Trên $n\in\mathbb{N}^\star$ ô vuông cách đều nhau của 1 đường tròn, ghi $n$ số tự nhiên liên tiếp theo chiều kim đồng hồ. Biết ô ghi số $12$ đối diện với ô ghi số $60$. Tính $n$.
\end{baitoan}

\begin{baitoan}[\cite{TLCT_THCS_Toan_6_so_hoc}, 1.14., p. 12]
	Tìm $n\in\mathbb{N}^\star$ biết có đúng $100$ số lẻ nằm giữa $n$ \& $2n$.
\end{baitoan}

\begin{baitoan}[\cite{TLCT_THCS_Toan_6_so_hoc}, 1.15., p. 12]
	Ghép 2 chữ số $1$, 2 chữ số $2$, 2 chữ số $3$ làm thành 1 số có $6$ chữ số. Tìm số có $6$ chữ số ấy biết 2 chữ số $1$ cách nhau 1 chữ số, 2 chữ số $2$ cách nhau 2 chữ số, 2 chữ số $3$ cách nhau 3 chữ số.
\end{baitoan}

\begin{baitoan}[\cite{TLCT_THCS_Toan_6_so_hoc}, 1.18., p. 12]
	Chia $a\in\mathbb{N}$ cho $72$ thì dư $69$. Chia số $a$ cho $18$ thì thương bằng số dư. Tìm $a$.
\end{baitoan}

\begin{baitoan}[\cite{TLCT_THCS_Toan_6_so_hoc}, 1.19., p. 13]
	Xét phép chia $a\in\mathbb{N}$ cho $b\in\mathbb{N}^\star$, có $a = bq + r$, $0\le r < b$. Nếu $r = 0$ thì $q$ gọi là {\rm thương đúng} của phép chia. Nếu $r\ne0$ thì $q$ gọi là {\rm thương hụt} của phép chia. Ký hiệu $[a:b]$ là thương đúng hoặc thương hụt của phép chia $a$ cho $b$. Tính: (a) $[32:4],[61:4]$. (b) $[800:5] + [800:5^2] + [800:5^3] + [800:5^4]$.
\end{baitoan}

\begin{baitoan}[\cite{TLCT_THCS_Toan_6_so_hoc}, 1.20., p. 13]
	Điền chữ số thích hợp để $\overline{84**}:47 = \overline{*8*}$.
\end{baitoan}

\begin{baitoan}[\cite{TLCT_THCS_Toan_6_so_hoc}, 1.21., p. 13]
	Tính: (a) $(2^9\cdot16 + 2^9\cdot34):2^{10}$. (b) $(3^4\cdot57 - 9^2\cdot21):3^5$.
\end{baitoan}

\begin{baitoan}[\cite{TLCT_THCS_Toan_6_so_hoc}, 1.22., p. 13]
	Cho biết $\sum_{i=1}^9 i^3 = 1^3 + 2^3 + \cdots + 9^3 = 2025$. Tính $2^3 + 4^3 + 6^3 + \cdots + 18^3$.
\end{baitoan}

\begin{baitoan}[\cite{TLCT_THCS_Toan_6_so_hoc}, 1.23., p. 13]
	Cho $a = \sum_{i=1}^{10} 2^i = 2 + 2^2 + 2^3 + \cdots + 2^{10}$. Không tính giá trị của biểu thức $a$, chứng minh $a + 2 = 2^11$.
\end{baitoan}

\begin{baitoan}[\cite{TLCT_THCS_Toan_6_so_hoc}, 1.24., p. 13]
	Tìm $x\in\mathbb{N}$ thỏa $(2x + 1)^2 = 625$.
\end{baitoan}

\begin{baitoan}[\cite{TLCT_THCS_Toan_6_so_hoc}, 1.25., p. 13]
	Quan sát $11 - 2 = 9 = 3^2$, $1111 - 22 = 1089 = 33^2$. Chứng minh $\underbrace{1\ldots1}_{2n} - \underbrace{2\ldots2}_n$ là số chính phương.
\end{baitoan}

\begin{baitoan}[\cite{TLCT_THCS_Toan_6_so_hoc}, 1.26., p. 13]
	Tìm chữ số tận cùng: (a) $7^{35} - 4^{31}$. (b) $2^{1930}\cdot9^{1945}$.
\end{baitoan}

\begin{baitoan}[\cite{TLCT_THCS_Toan_6_so_hoc}, 1.27., p. 13]
	Tìm 2 chữ số tận cùng: (a) $351^{2011}$. (b) $218^{218}$.
\end{baitoan}

%------------------------------------------------------------------------------%

\section{Divisibility of Sum, Difference, Product -- Tính Chất Chia Hết của Tổng, Hiệu, Tích}
\begin{itemize}\sf
	\item \textbf{divisible} [a] {\tt/}\textipa{d@'vIz@bl}{\tt/} [not before noun] \textit{divisible (by something)} that can be divided, usually with nothing remaining. {\sc opposite}: \textbf{indivisible}.
	\item \textbf{divisibility}  [n] [uncountable] {\tt/}\textipa{d@'vIz@bIl@ti}{\tt/}.
\end{itemize}
\cite[\S7, pp. 22--23]{SBT_Toan_6_Canh_Dieu_tap_1}: 58. 59. 60. 61. 62. 63. 64. 65.

\begin{baitoan}[\cite{Trong_Toan_7}, 2, 4, p. 32]
	Tổng nào chia hết cho $6$? $A = 6 + 12 + 120 + 60 + 738,B = 24 + 48 + 31 + 120 + 558,C = 16 + 33 + 8 + 27,D = 54 + 36,E = 66 - 16$.
\end{baitoan}

\begin{baitoan}[\cite{Trong_Toan_7}, 3, p. 32]
	Cho $A = 122 + 45 + 120,B = 5! - 40,C = 1\cdot4\cdot7\cdot5 - 34$. Biểu thức nào chia hết cho $2$, chia hết cho $5$, chia hết cho $10$?
\end{baitoan}

\begin{baitoan}[\cite{Trong_Toan_7}, 53, p. 36]
	Tìm số tự nhiên có 2 chữ số giống nhau biết số ấy chia hết cho $2$ \& khi chia số ấy cho $5$ thì dư $3$.
\end{baitoan}

\begin{baitoan}[\cite{Trong_Toan_7}, 54, p. 36]
	Tìm $n\in\mathbb{N}$ biết $n\divby2,n\divby5,124 < n < 172$.
\end{baitoan}

\begin{baitoan}[\cite{Trong_Toan_7}, 55, p. 36]
	Tìm các số tự nhiên có $3$ chữ số giống nhau biết số ấy chia hết cho $2$ \& chia số ấy cho $5$ thì dư $2$.
\end{baitoan}

\begin{baitoan}[\cite{Binh_boi_duong_Toan_6_tap_1}, H1, p. 24]
	{\rm Đ{\tt/}S? (a) $127\cdot5 + 40\divby5$. (b) $13\cdot48 + 12 + 17\divby6$. (c) $3\cdot300 - 12\divby9$. (d) $49 + 62\cdot7\divby7$.}
\end{baitoan}

\begin{baitoan}[\cite{Binh_boi_duong_Toan_6_tap_1}, H2, p. 24]
	Khi chia số $a$ cho số $b$, $a,b\in\mathbb{N}^\star$, $a > b$ ta được số dư là $r$. Khi đó: {\sf A.} $a + r\divby b$. {\sf B.} $a - r\divby b$. {\sf C.} $a + b\divby r$. {\sf D.} $a - b\divby r$.
\end{baitoan}

\begin{baitoan}[\cite{Binh_boi_duong_Toan_6_tap_1}, H3, p. 24]
	Tìm số tự nhiên $x$ có 1 chữ số thỏa $121 + x\divby11$.
\end{baitoan}

\begin{baitoan}[\cite{Binh_boi_duong_Toan_6_tap_1}, VD1, p. 25]
	Không tính các tổng \& hiệu, xét xem các tổng \& hiệu sau có chia hết cho $12$ không? Vì sao? (a) $600\cdot37 - 144$. (b) $96 + 34 + 48$.
\end{baitoan}

\begin{baitoan}[\cite{Binh_boi_duong_Toan_6_tap_1}, VD2, p. 25]
	Không tính ra kết quả, xét xem tổng $84 + 37 + 23$ có chia hết cho $12$ không? Vì sao?
\end{baitoan}

\begin{baitoan}[\cite{Binh_boi_duong_Toan_6_tap_1}, VD3, p. 25]
	Chứng minh trong 3 số tự nhiên liên tiếp có 1 số chia hết cho $3$.
\end{baitoan}

\begin{baitoan}[\cite{Binh_boi_duong_Toan_6_tap_1}, Mở rộng VD4, p. 25]
	Với $n\in\mathbb{N}^\star$ bất kỳ. Chứng minh: (a) Trong $n$ số tự nhiên liên tiếp luôn có 1 số chia hết cho $n$. (b) Tích của $n$ số tự nhiên liên tiếp là 1 số chia hết cho $n$.
\end{baitoan}

\begin{baitoan}[\cite{Binh_boi_duong_Toan_6_tap_1}, VD4, p. 26]
	Chứng minh tổng của 3 số tự nhiên liên tiếp là 1 số chia hết cho $3$.
\end{baitoan}

\begin{baitoan}
	Với $n\in\mathbb{N}^\star$ bất kỳ. Liệu tổng của $n$ số tự nhiên liên tiếp có chia hết cho $n$ không?
\end{baitoan}

\begin{baitoan}[\cite{Binh_boi_duong_Toan_6_tap_1}, VD5, p. 26, \cite{TLCT_THCS_Toan_6_so_hoc}, 3.2., p. 25]
	Chứng minh: (a) $\overline{ab} - \overline{ba}\divby9$ với $a > b$. (b) Nếu $\overline{ab} + \overline{cd}\divby11$ thì $\overline{abcd}\divby11$.
\end{baitoan}

\begin{baitoan}[\cite{Binh_boi_duong_Toan_6_tap_1}, VD6, p. 26]
	Cho $A = 15 + 30 + 37 + x$ với $x\in\mathbb{N}$. Tìm điều kiện của $x$ để: (a) $A\divby3$. (b) $A\not{\divby}\ 9$.
\end{baitoan}

\begin{baitoan}[\cite{Binh_boi_duong_Toan_6_tap_1}, VD7, p. 26]
	Tìm $n\in\mathbb{N}$ để: (a) $n + 4\divby n$. (b) $5n - 6\divby n$ với $n > 1$. (c) $143 - 12n\divby n$ với $n < 12$.
\end{baitoan}

\begin{baitoan}[\cite{Binh_boi_duong_Toan_6_tap_1}, VD8, p. 27]
	Tìm $n\in\mathbb{N}$ để: (a) $n + 9\divby n + 4$. (b) $3n + 40\divby n + 4$. (c) $5n + 2\divby2n + 9$.
\end{baitoan}

\begin{baitoan}[\cite{Binh_boi_duong_Toan_6_tap_1}, 3.1., p. 27]
	Cho $A = 2\cdot5\cdot9\cdot13 + 84$. Hỏi $A$ có chia hết cho $3$, cho $6$, cho $9$, cho $13$ không? Vì sao?
\end{baitoan}

\begin{baitoan}[\cite{Binh_boi_duong_Toan_6_tap_1}, 3.2., p. 27]
	Chứng minh tổng 5 số chẵn liên tiếp là 1 số chia hết cho $10$.
\end{baitoan}

\begin{baitoan}[\cite{Binh_boi_duong_Toan_6_tap_1}, 3.3., p. 27]
	Khi chia số tự nhiên $a$ cho $27$, ta được số dư là $15$. Hỏi số $a$ có chia hết cho $3$, cho $9$ không? Vì sao?
\end{baitoan}

\begin{baitoan}[\cite{Binh_boi_duong_Toan_6_tap_1}, 3.4., p. 27]
	Chứng minh mọi số tự nhiên có 3 chữ số giống nhau đều chia hết cho $37$.
\end{baitoan}

\begin{baitoan}[\cite{Binh_boi_duong_Toan_6_tap_1}, 3.5., p. 28]
	Chứng minh: (a) $\sum_{i=0}^{101} 5^i = 1 + 5 + 5^2 + 5^3 + \cdots + 5^{101}\divby6$. (b) $\sum_{i=1}^{100} 2^i = 2 + 2^2 + 2^3 + \cdots + 2^{100}$ vừa chia hết cho $31$, vừa chia hết cho $5$.
\end{baitoan}

\begin{baitoan}[\cite{Binh_boi_duong_Toan_6_tap_1}, 3.6., p. 28]
	Chứng minh: (a) Nếu $\overline{abc} - \overline{def}\divby11$ thì $\overline{abcdef}\divby11$. (b) Nếu $\overline{abc}\divby8$ thì $4a + 2b + c\divby8$. (c) Nếu $\overline{a_na_{n-1}\ldots a_2a_1a_0}\divby8$ thì $4a_2 + 2a_1 + a_0\divby8$.
\end{baitoan}

\begin{baitoan}[\cite{Binh_boi_duong_Toan_6_tap_1}, 3.7., p. 28]
	Tìm chữ số $a$ biết $\overline{21a21a21a}\divby31$.
\end{baitoan}

\begin{baitoan}[\cite{Binh_boi_duong_Toan_6_tap_1}, 3.8., p. 28]
	Tìm $n\in\mathbb{N}$ sao cho: (a) $n + 21\divby n$. (b) $18 - 2n\divby n$ với $n < 9$. (c) $6n - 9\divby n$ với $n\ge2$. (d) Mở rộng cho $\mathbb{Z}$.
\end{baitoan}

\begin{baitoan}[\cite{Binh_boi_duong_Toan_6_tap_1}, 3.9., p. 28]
	Tìm $n\in\mathbb{N}$ sao cho: (a) $n + 15\divby n - 3$ với $n > 5$. (b) $18 - 2n\divby n + 3$ với $n\le9$. (c) $3n + 13\divby2n + 3$ với $n\ge1$.
\end{baitoan}

\begin{baitoan}[\cite{Binh_boi_duong_Toan_6_tap_1}, 3.10., p. 28]
	Cho $a,b\in\mathbb{N}$. Chứng minh nếu $7a + 2b$ \& $31a + 9b$ cùng chia hết cho $2015$ thì $a,b$ cũng chia hết cho $2015$.
\end{baitoan}

\begin{baitoan}[\cite{Binh_boi_duong_Toan_6_tap_1}, p. 28]
	Chứng minh: (a) Tích 2 số tự nhiên liên tiếp là 1 số chẵn. (b) Tích 3 số tự nhiên liên tiếp luôn chia hết cho $6$. (c) Tích của $n$ số tự nhiên liên tiếp bất kỳ luôn chia hết cho $n! = \prod_{i=1}^n i = 1\cdot2\cdot3\cdots n$, $\forall n\in\mathbb{N}^\star$.
\end{baitoan}

\begin{baitoan}[\cite{Binh_boi_duong_Toan_6_tap_1}, p. 28]
	Với $n\in\mathbb{N}^\star$. (a) Khi nào thì tổng của $n$ số tự nhiên liên tiếp bất kỳ chia hết cho $n$? (b) Khi nào thì tổng của $n$ số tự nhiên chẵn liên tiếp bất kỳ chia hết cho $n$? (c) Khi nào thì tổng của $n$ số tự nhiên lẻ liên tiếp bất kỳ chia hết cho $n$?
\end{baitoan}

\begin{baitoan}[\cite{Tuyen_Toan_6}, VD21, p. 22]
	Cho $a\divby m,b\divby m$. Chứng minh $k_1a + k_2b\divby m$, $\forall k_1,k_2\in\mathbb{Z}$.
\end{baitoan}

\begin{baitoan}[\cite{Tuyen_Toan_6}, VD22, p. 22]
	Chứng minh: (a) Nếu $a\divby m,b\divby m,a + b + c\divby m$ thì $c\divby m$. (b) Nếu $a\divby m,b\divby m,a + b + c\not{\divby}\ m$ thì $c\not{\divby}\ m$.
\end{baitoan}

\begin{baitoan}[\cite{Tuyen_Toan_6}, 90., p. 22]
	Chứng minh $\forall n\in\mathbb{N}$, $60n + 45\divby15$ nhưng $60n + 45\not{\divby}\ 30$
\end{baitoan}

\begin{baitoan}[\cite{Tuyen_Toan_6}, 91., p. 22]
	Cho $A = 2\cdot4\cdot6\cdot8\cdot10\cdot12 + 40$. A có chia hết cho $6$, cho $8$, cho $5$ không?
\end{baitoan}

\begin{baitoan}[\cite{Tuyen_Toan_6}, 92., p. 22]
	Cho $A = 23! + 19! - 15!$. Chứng minh: (a) $A\divby11$. (b) $A\divby10$.
\end{baitoan}

\begin{baitoan}[\cite{Tuyen_Toan_6}, 93., p. 23]
	Chứng minh tổng của 3 số tự nhiên liên tiếp thì chia hết cho $3$ còn tổng của $4$ số tự nhiên liên tiếp thì không chia hết cho $4$.
\end{baitoan}

\begin{baitoan}[\cite{Tuyen_Toan_6}, 94., p. 23]
	Chứng minh tổng của $5$ số chẵn liên tiếp chia hết cho $10$, tổng của $5$ số lẻ liên tiếp chia cho $10$ dư $5$.
\end{baitoan}

\begin{baitoan}[\cite{Tuyen_Toan_6}, 95., p. 23]
	Cho 4 số tự nhiên không chia hết cho $5$, khi chia cho $5$ được các số dư khác nhau. Chứng minh tổng của 4 số này chia hết cho $5$.
\end{baitoan}

\begin{baitoan}[\cite{Tuyen_Toan_6}, 96., p. 23]
	Cho $A = \sum_{i=0}^{11} 3^i = 1 + 3 + 3^2 + \cdots + 3^{11}$. Chứng minh: (a) $A\divby13$. (b) $A\divby40$.
\end{baitoan}

\begin{baitoan}[\cite{Tuyen_Toan_6}, 97., p. 23]
	Chứng minh: (a) Tích của 2 số tự nhiên liên tiếp thì chia hết cho $2$. (b) Tích của 3 số tự nhiên liên tiếp thì chia hết cho $3$.
\end{baitoan}

\begin{baitoan}[\cite{Tuyen_Toan_6}, 98., p. 23]
	Chứng minh không có số tự nhiên nào chia cho $15$ dư $6$ còn chia cho $9$ dư $1$.
\end{baitoan}

\begin{baitoan}[\cite{Tuyen_Toan_6}, 99., p. 23]
	Tìm $n\in\mathbb{N}$ thỏa: (a) $n + 4\divby n$. (b) $3n + 7\divby n$. (c) $27 - 5n\divby n$.
\end{baitoan}

\begin{baitoan}[\cite{Tuyen_Toan_6}, 100., p. 23]
	Tìm $n\in\mathbb{N}$ thỏa: (a) $n + 6\divby n + 2$. (b) $2n + 3\divby n - 2$. (c) $3n + 1\divby11 - 2n$.
\end{baitoan}

\begin{baitoan}[\cite{Tuyen_Toan_6}, 101., p. 23]
	Cho $10^k - 1\divby19$ với $k > 1$. Chứng minh: (a) $10^{2k} - 1\divby19$. (b) $10^{3k} - 1\divby19$.
\end{baitoan}

\begin{baitoan}[\cite{Binh_Toan_6_tap_1}, VD21, p. 22]
	Chứng minh: (a) $\overline{ab} + \overline{ba}\divby11$. (b) $\overline{ab} - \overline{ba}\divby9$ với $a > b$.
\end{baitoan}

\begin{baitoan}[\cite{Binh_Toan_6_tap_1}, VD22, p. 23]
	Quan sát các ví dụ: $14 + 19 = 33\divby11,1419\divby11$, $6 + 49 = 55\divby11,649\divby11$. Rút ra nhận xét \& chứng minh nhận xét ấy.
\end{baitoan}

\begin{baitoan}[\cite{Binh_Toan_6_tap_1}, VD23, p. 23]
	Cho số $\overline{abc}\divby27$. Chứng minh $\overline{bca}\divby27$.
\end{baitoan}

\begin{baitoan}[\cite{Binh_Toan_6_tap_1}, VD24, p. 23]
	Tìm số tự nhiên có 3 chữ số biết số đó chia hết cho $18$ \& các chữ số của nó nếu sắp xếp từ nhỏ đến lớn thì tỷ lệ với $1:2:3$.
\end{baitoan}

\begin{baitoan}[\cite{Binh_Toan_6_tap_1}, 118., p. 23]
	Có thể chọn được 5 số trong dãy số sau để tổng của chúng bằng $70$ không? (a) $1,2,\ldots,30$. (b) $1,3,5,\ldots,29$.
\end{baitoan}

\begin{baitoan}[\cite{Binh_Toan_6_tap_1}, 119., p. 23]
	Cho 9 số: $1,3,5,7,9,11,13,15,17$. Có thể phân chia được hay không 9 số trên thành 2 nhóm sao cho: (a) Tổng các số thuộc nhóm I gấp đôi tổng các số thuộc nhóm II? (b) Tổng các số thuộc nhóm I bằng tổng các số thuộc nhóm II?
\end{baitoan}

\begin{baitoan}[\cite{Binh_Toan_6_tap_1}, 120., p. 23]
	(a) Có 3 số tự nhiên nào mà tổng của chúng tận cùng bằng $4$, tích của chúng tận cùng bằng $1$ không? (b) Có tồn tại hay không 4 số tự nhiên mà tổng của chúng \& tích của chúng đều là số lẻ?
\end{baitoan}

\begin{baitoan}[\cite{Binh_Toan_6_tap_1}, 121., p. 23]
	Chứng minh không tồn tại $a,b,c\in\mathbb{N}$ thỏa:
	\begin{equation*}
		\left\{\begin{split}
			abc + a &= 333,\\
			abc + b &= 335,\\
			abc + c &= 341.
		\end{split}\right.
	\end{equation*}
\end{baitoan}

\begin{baitoan}[\cite{Binh_Toan_6_tap_1}, 122., p. 23]
	1 lớp học có $6$ tổ, số người của mỗi tổ bằng nhau. Trong 1 bài kiểm tra, tất cả học sinh đều được điểm $7$ hoặc $8$. Tổng số điểm của cả lớp là $350$. Tính số học sinh của lớp, số học sinh đạt từng loại điểm.
\end{baitoan}

\begin{baitoan}[\cite{Binh_Toan_6_tap_1}, 123., p. 24]
	Khối 6 của 1 trường có $366$ học sinh, gồm $8$ lớp. Mỗi lớp gồm 1 số tổ, mỗi tổ có $9$ người hoặc $10$ người. Biết số tổ của các lớp đều bằng nhau, tính số tổ có $9$ người, số tổ có $10$ người của cả khối.
\end{baitoan}

\begin{baitoan}[\cite{Binh_Toan_6_tap_1}, 124., p. 24]
	(a) Chứng minh nếu viết thêm vào đằng sau 1 số tự nhiên có 2 chữ số số gồm chính 2 chữ số ấy viết theo thứ tự ngược lại thì được 1 số chia hết cho $11$. (b) Mở rộng (a) cho số tự nhiên có 3 chữ số.
\end{baitoan}

\begin{baitoan}[\cite{Binh_Toan_6_tap_1}, 125., p. 24]
	Chứng minh nếu $\overline{ab} = 2\overline{cd}$ thì $\overline{abcd}\divby67$.
\end{baitoan}

\begin{baitoan}[\cite{Binh_Toan_6_tap_1}, 126., p. 24]
	Chứng minh: (a) $A = \overline{abcabc}$, $A\divby7,A\divby11,A\divby13$. (b) $B = \overline{abcdef}$, $B\divby23,B\divby29$, biết $\overline{abc} = 2\overline{def}$.
\end{baitoan}

\begin{baitoan}[\cite{Binh_Toan_6_tap_1}, 127., p. 24]
	Chứng minh nếu $\overline{ab} + \overline{cd} + \overline{ef}\divby11$ thì $\overline{abcdef}\divby11$.
\end{baitoan}

\begin{baitoan}[\cite{Binh_Toan_6_tap_1}, 128., p. 24]
	(a) Cho $\overline{abc} + \overline{def}\divby37$. Chứng minh $\overline{abcdef}\divby37$. (b) Cho $\overline{abc} + \overline{def}\divby7$. Chứng minh $\overline{abcdef}\divby7$. (c) Cho 8 số tự nhiên có 3 chữ số. Chứng minh trong 8 số đó, tồn tại 2 số mà khi viết liên tiếp nhau thì tạo thành 1 số có 6 chữ số chia hết cho $7$.
\end{baitoan}

\begin{baitoan}[\cite{Binh_Toan_6_tap_1}, 129., p. 24]
	Tìm chữ số $a$ biết $\overline{20a20a20a}\divby7$.
\end{baitoan}

\begin{baitoan}[\cite{Binh_Toan_6_tap_1}, 130., p. 24]
	Cho 3 chữ số khác nhau \& khác $0$. Lập tất cả các số tự nhiên có 3 chữ số gồm cả 3 chữ số ấy. Chứng minh tổng của chúng chia hết cho $6,37$.
\end{baitoan}

\begin{baitoan}[\cite{Binh_Toan_6_tap_1}, 131., p. 24]
	Có $x,y\in\mathbb{N}$ thỏa $(x + y)(x - y) = 1002$ không?
\end{baitoan}

\begin{baitoan}[\cite{Binh_Toan_6_tap_1}, 132., p. 24]
	Tìm $n\in\mathbb{N}$ nhỏ nhất sao cho ta có cách thêm $n$ chữ số vào sau số đó để được 1 số chia hết cho $39$.
\end{baitoan}

\begin{baitoan}[\cite{Binh_Toan_6_tap_1}, 133., p. 24]
	Tìm $a\in\mathbb{N}$ có 2 chữ số sao cho nếu viết $a$ tiếp sau số $1999$ thì ta được 1 số chia hết cho $37$.
\end{baitoan}

\begin{baitoan}[\cite{Binh_Toan_6_tap_1}, 134., p. 24]
	Cho $n\in\mathbb{N}$. Chứng minh: (a) $(n + 10)(n + 15)\divby2$. (b) $A = n(n + 1)(n + 2)$, $A\divby2,A\divby3$. (c) $B = n(n + 1)(2n + 1)$, $B\divby2,B\divby3$.
\end{baitoan}

\begin{baitoan}[\cite{Binh_Toan_6_tap_1}, 135., p. 24]
	Tìm $a,b\in\mathbb{N}$ thỏa $a\divby b,b\divby a$.
\end{baitoan}

\begin{baitoan}[\cite{Binh_Toan_6_tap_1}, 136., p. 24]
	1 học sinh viết các số tự nhiên từ $1$ đến $\overline{abc}$. Bạn đó phải viết tất cả $m$ chữ số. Biết $m\divby\overline{abc}$, tìm $\overline{abc}$.
\end{baitoan}

\begin{baitoan}[\cite{Binh_Toan_6_tap_1}, 137., p. 24]
	Cho 9 số tự nhiên viết theo thứ tự giảm dần từ $9$ đến $1$: $987654321$. Có thể đặt được hay không 1 số dấu $+$ hoặc $-$ vào giữa các số đó để kết quả của phép tính bằng: (a) $5$. (b) $6$?
\end{baitoan}

\begin{baitoan}[\cite{Binh_Toan_6_tap_1}, 138., p. 25]
	Cho tổng $1 + 2 + \cdots + 9$. Xóa 2 số bất kỳ rồi thay bằng hiệu của chúng \& cứ làm như vậy nhiều lần. Có cách nào làm cho kết quả cuối cùng bằng $0$ không?
\end{baitoan}

\begin{baitoan}[\cite{Binh_Toan_6_tap_1}, 139., p. 25]
	Chứng minh tổng các số ghi trên vé xổ số có 6 chữ số mà tổng 3 chữ số đầu bằng tổng 3 chữ số cuối thì chia hết cho $13$ (các chữ số đầu có thể bằng $0$).
\end{baitoan}

\begin{baitoan}[\cite{Binh_Toan_6_tap_1}, 140., p. 25]
	(a) $n\in\mathbb{N}$ chia cho $54$ dư $17$. Tìm số dư lớn nhất khi chia $n$ cho $162$. (b) $n\in\mathbb{N}$ chia cho $802$ dư $502$. Tìm số dư nhỏ nhất khi chia $n$ choa $2005$.
\end{baitoan}

\begin{baitoan}[\cite{Binh_Toan_6_tap_1}, 141., p. 25]
	Tìm $a,b\in\mathbb{N}$ nhỏ nhất lớn hơn $1$ sao cho $a^7 = b^8$.
\end{baitoan}

\begin{baitoan}[\cite{TLCT_THCS_Toan_6_so_hoc}, p. 20]
	Chứng minh $a\divby b\Rightarrow a^n\divby b^n$, $\forall a,b,n\in\mathbb{N}$, $b\ne0$.
\end{baitoan}

\begin{baitoan}[\cite{TLCT_THCS_Toan_6_so_hoc}, p. 20]
	Chứng minh nếu $ab\divby c$, đồng thời $a,c$ không cùng chia hết cho 1 số tự nhiên khác $1$ nào thì $b\divby c$.
\end{baitoan}

\begin{baitoan}[\cite{TLCT_THCS_Toan_6_so_hoc}, p. 20, dấu hiệu chia hết cho 4, cho 25]
	Chứng minh: (a) Các số có 2 chữ số tận cùng tạo thành số chia hết cho $4$ thì chia hết cho $4$ \& chỉ các số đó mới chia hết cho $4$. (b) Các số có 2 chữ số tận cùng tạo thành số chia hết cho $25$ thì chia hết cho $25$ \& chỉ các số đó mới chia hết cho $25$. (c) Tìm $n\in\mathbb{N}^\star$ biết các số có 2 chữ số tận cùng tạo thành số chia hết cho $n$ thì chia hết cho $n$ \& chỉ các số đó mới chia hết cho $n$.
\end{baitoan}

\begin{baitoan}[\cite{TLCT_THCS_Toan_6_so_hoc}, p. 20, dấu hiệu chia hết cho 8, cho 125]
	Chứng minh: (a) Các số có 3 chữ số tận cùng tạo thành số chia hết cho $8$ thì chia hết cho $8$ \& chỉ các số đó mới chia hết cho $8$. (b) Các số có 3 chữ số tận cùng tạo thành số chia hết cho $125$ thì chia hết cho $125$ \& chỉ các số đó mới chia hết cho $125$. (c) Tìm $n\in\mathbb{N}^\star$ biết các số có 3 chữ số tận cùng tạo thành số chia hết cho $n$ thì chia hết cho $n$ \& chỉ các số đó mới chia hết cho $n$.
\end{baitoan}

\begin{baitoan}[\cite{TLCT_THCS_Toan_6_so_hoc}, p. 21, dấu hiệu chia hết cho 11]
	Chứng minh các số có tổng các chữ số hàng lẻ \& tổng các chữ số hàng chẵn có hiệu chia hết cho $11$ thì chia hết cho $11$ \& chỉ các số đó mới chia hết cho $11$.
\end{baitoan}

\begin{baitoan}[\cite{TLCT_THCS_Toan_6_so_hoc}, VD3.1, p. 20]
	Điền $9$ số tự nhiên từ $1$ đến $9$ vào $8$ đỉnh \& tâm của 1 hình bát giác đều sao cho tổng 3 số thẳng hàng chia hết cho $5$.
\end{baitoan}

\begin{baitoan}[\cite{TLCT_THCS_Toan_6_so_hoc}, VD3.2, p. 21]
	Chứng minh tồn tại 1 số tự nhiên chia hết cho $37$ \& có tổng các chữ số bằng: (a) $27$. (b) $37$.
\end{baitoan}

\begin{baitoan}[\cite{TLCT_THCS_Toan_6_so_hoc}, VD3.3, p. 21]
	Gọi $a$ là tổng của tất cả các số tự nhiên có 2 chữ số. Số $a$ chia hết cho số nào trong các số $2,3,5,7,9$?
\end{baitoan}

\begin{baitoan}[\cite{TLCT_THCS_Toan_6_so_hoc}, VD3.4, p. 22]
	Khi đổi chỗ các chữ số của $a\in\mathbb{N}$, ta được số tự nhiên $b\in\mathbb{N}$ gấp $3$ lần $a$. Chứng minh $a\divby9$.
\end{baitoan}

\begin{baitoan}[\cite{TLCT_THCS_Toan_6_so_hoc}, VD3.5, p. 22]
	Cho $a,d\in\mathbb{N}^\star$. Chứng minh $d = 1$ nếu: (a) $a,2a - 1$ cùng chia hết cho $d$. (b) $a,6a - 1$ cùng chia hết cho $d$.
\end{baitoan}

\begin{baitoan}[\cite{TLCT_THCS_Toan_6_so_hoc}, VD3.6, p. 22]
	Tìm số tự nhiên có 2 chữ số biết số đó gấp đôi tích các chữ số của nó.
\end{baitoan}

\begin{baitoan}[\cite{TLCT_THCS_Toan_6_so_hoc}, VD3.7, p. 24]
	Tìm số tự nhiên lớn nhất có 3 chữ số biết số đó chia hết cho $4$ \& chia cho $25$ thì dư $2$.
\end{baitoan}

\begin{baitoan}[\cite{TLCT_THCS_Toan_6_so_hoc}, VD3.8, p. 24]
	Tìm số tự nhiên có dạng $\overline{aba}$ biết số đó chia hết cho $33$.
\end{baitoan}

\begin{baitoan}[\cite{TLCT_THCS_Toan_6_so_hoc}, VD3.9, p. 24]
	An làm 1 bài thi gồm $20$ câu. Mỗi câu trả lời đúng được $5$ điểm, trả lời sai bị trừ $2$ điểm, bỏ qua không trả lời được $0$ điểm. Trong bài thi, có câu An trả lời sai. Tính số câu trả lời đúng, số câu trả lời sai, số câu bỏ qua không trả lời nếu An được: (a) $60$ điểm. (b) $55$ điểm.
\end{baitoan}

\begin{baitoan}[\cite{TLCT_THCS_Toan_6_so_hoc}, 3.1., p. 25]
	Lấy số $6$, nhân gấp $9$ được $54$, rồi ghép lại thành $546$, số này chia hết cho $7$. (a) Nếu thay số $6$ lần lượt bởi 5 số $1,2,3,4,5$ \& cùng làm như trên thì các số nhận được có chia hết cho $7$ không? (b) Nêu ví dụ để chứng tỏ lấy 1 số khác $6$ rồi cùng làm như trên thì ta có thể được 1 số không chia hết cho $7$.
\end{baitoan}

\begin{baitoan}[\cite{TLCT_THCS_Toan_6_so_hoc}, 3.3., p. 25]
	(a) Chứng minh số có dạng $\overline{abcabc}\divby7$. (b) Điền chữ số để $\overline{31a31a31a}\divby7$.
\end{baitoan}

\begin{baitoan}[\cite{TLCT_THCS_Toan_6_so_hoc}, 3.4., p. 25]
	Dùng 3 trong 4 chữ số $0,3,4,6$, viết tất cả các số tự nhiên có 3 chữ số chia hết cho tất cả 4 số $2,3,5,9$.
\end{baitoan}

\begin{baitoan}[\cite{TLCT_THCS_Toan_6_so_hoc}, 3.5., p. 25]
	Điền chữ số thích hợp: (a) $\overline{abcd0} - \overline{1110n} = \overline{abcd}$. (b) $\overline{3a4b5}\divby9$ biết $a - b = 2$.
\end{baitoan}

\begin{baitoan}[\cite{TLCT_THCS_Toan_6_so_hoc}, 3.6., p. 25]
	Chứng minh số $\overline{1\ldots1}_n - 10n\divby9$.
\end{baitoan}

\begin{baitoan}[\cite{TLCT_THCS_Toan_6_so_hoc}, 3.7., p. 25]
	2 số tự nhiên $a,6a$ có tổng các chữ số như nhau. Chứng minh $a\divby9$.
\end{baitoan}

\begin{baitoan}[\cite{TLCT_THCS_Toan_6_so_hoc}, 3.8., p. 25]
	Gọi $n$ là số tạo bởi các số tự nhiên viết liên tiếp từ $16$ đến $89$. Tìm $k\in\mathbb{N}$ lớn nhất để $n\divby3^k$.
\end{baitoan}

\begin{baitoan}[\cite{TLCT_THCS_Toan_6_so_hoc}, 3.9., p. 26]
	Điền chữ số thích hợp: (a) $\overline{83788}\divby72$. (b) $\overline{24*68*}\divby48$.
\end{baitoan}

\begin{baitoan}[\cite{TLCT_THCS_Toan_6_so_hoc}, 3.10., p. 26]
	Cho số $\overline{abc}\divby4$ có 3 chữ số đều chẵn \& $b\ne0$. Chứng minh số $\overline{bac}\divby4$.
\end{baitoan}

\begin{baitoan}[\cite{TLCT_THCS_Toan_6_so_hoc}, 3.11., p. 26]
	Tìm 2 số tự nhiên liên tiếp có 3 chữ số biết số nhỏ chia hết cho $8$, số lớn chia hết cho $125$.
\end{baitoan}

\begin{baitoan}[\cite{TLCT_THCS_Toan_6_so_hoc}, 3.12., p. 26]
	Dùng $10$ chữ số từ $0$ đến $9$ viết số nhỏ nhất có 10 chữ số \& chia hết cho: (a) $4$. (b) $8$. (c) $25$. (d) $125$.
\end{baitoan}

\begin{baitoan}[\cite{TLCT_THCS_Toan_6_so_hoc}, 3.13., p. 26]
	Điền chữ số thích hợp: (a) $\overline{3*4827}\divby11$. (b) $\overline{*2013*}\divby88$. (c) $3576 - \overline{abc} = \overline{abcd}$.
\end{baitoan}

\begin{baitoan}[\cite{TLCT_THCS_Toan_6_so_hoc}, 3.14., p. 26]
	Xét số tự nhiên $n = \overline{Abcd}$ trong đó $A$ là số nghìn, $A$ có thể bằng $0$, có thể có 1 hoặc có nhiều chữ số. Chứng minh nếu $A + \overline{bcd}\divby37$ thì $n\divby37$.
\end{baitoan}

\begin{baitoan}[\cite{TLCT_THCS_Toan_6_so_hoc}, 3.15., p. 26]
	Tìm số tự nhiên có 2 chữ số biết số đó gấp $6$ lần tích các chữ số của nó.
\end{baitoan}

\begin{baitoan}[\cite{TLCT_THCS_Toan_6_so_hoc}, 3.16., p. 26]
	Cho 4 chữ số $1,2,3,4$. (a) Lập được bao nhiêu số tự nhiên có 4 chữ số gồm cả 4 chữ số này? (b) Chứng minh trong các số lập được ở (a), không có 2 số nào mà 1 số chia hết cho số còn lại.
\end{baitoan}

\begin{baitoan}[\cite{TLCT_THCS_Toan_6_so_hoc}, 3.17., p. 26]
	Chứng minh $\forall n\in\mathbb{N}$: (a) $n(n + 2)(n + 7)\divby3$. (b) $5^n - 1\divby4$. (c) $n^2 + n + 2\not{\divby}\ 5$.
\end{baitoan}

\begin{baitoan}[\cite{TLCT_THCS_Toan_6_so_hoc}, 3.18., p. 26]
	Tìm $n\in\mathbb{N}$ lớn nhất có 3 chữ số sao cho $n^2 - n\divby5$.
\end{baitoan}

\begin{baitoan}[\cite{TLCT_THCS_Toan_6_so_hoc}, 3.19., p. 26]
	Tìm số tự nhiên nhỏ nhất có $9$ chữ số, chia hết cho $9$ \& có các tính chất sau: Nếu xóa 1 chữ số tận cùng thì được số chia hết cho $8$, nếu xóa 2 chữ số tận cùng thì được số chia hết cho $7$, nếu xóa 3 chữ số tận cùng thì được số chia hết cho $6$, nếu xóa 4 chữ số tận cùng thì được số chia hết cho $5$, nếu xóa 5 chữ số tận cùng thì được số chia hết cho $4$, nếu xóa 6 chữ số tận cùng thì được số chia hết cho $3$, nếu xóa 7 chữ số tận cùng thì được số chia hết cho $2$.
\end{baitoan}

\begin{baitoan}[\cite{TLCT_THCS_Toan_6_so_hoc}, 3.20., p. 26]
	2 bạn $A,B$ chơi trò chơi lấy bi trong hộp có $100$ viên bi. Mỗi người lần lượt phải lấy từ $4$ đến $8$ viên bi, người lấy được viên bi cuối cùng là người thắng cuộc. A được đi trước. Nêu cách chơi để A thắng cuộc.
\end{baitoan}

\begin{baitoan}[\cite{TLCT_THCS_Toan_6_so_hoc}, 3.21., p. 26]
	A viết 10 số tự nhiên liên tiếp rồi xóa đi 1 số thì tổng của 9 số còn lại là $490$. Tìm số bị xóa.
\end{baitoan}

\begin{baitoan}[\cite{TLCT_THCS_Toan_6_so_hoc}, 3.22., p. 26]
	Cho dãy số $1,2,3,\ldots,100$. (a) Chứng minh tổng của 3 số liên tiếp bất kỳ trong dãy chia hết cho $3$. (b) Có thể đổi chỗ các số trong dãy đã cho để được 1 dãy mới có tổng của 4 số liên tiếp bất kỳ trong dãy cũng chia hết cho $3$ được không?
\end{baitoan}

\begin{baitoan}[\cite{TLCT_THCS_Toan_6_so_hoc}, 3.23., p. 26]
	A mua 1 số hộp bút \& 1 số quyển vở hết tất cả $100$ nghìn đồng. Biết 1 hộp bút giá $13$ nghìn đồng, 1 quyển vở giá $5$ nghìn đồng. Hỏi A đã mua bao nhiêu hộp bút, bao nhiêu quyển vở?
\end{baitoan}

\begin{baitoan}[\cite{TLCT_THCS_Toan_6_so_hoc}, 3.24., p. 26]
	A có $50$ tờ tiền mệnh giá $1000$ đồng, $50$ tờ tiền mệnh giá $5000$ đồng, $50$ tờ tiền mệnh giá $10000$ đồng. Long cần chọn như thế nào để có $100000$ đồng gồm $18$ tờ tiền, tờ tiền nào cũng có \& số tờ tiền $1000$ đồng là ít nhất.
\end{baitoan}

\begin{baitoan}[\cite{TLCT_THCS_Toan_6_so_hoc}, 3.25., p. 26]
	1 cửa hàng có 5 hộp, mỗi hộp chỉ đựng bát hoặc đĩa. Số lượng bát hoặc đĩa trong 5 hộp là $19,22,32,35,37$ cái. Sau khi bán hết số đĩa trong 1 hộp thì số bát nhiều gấp 4 số đĩa còn lại. Tính số đĩa đã bán, số đĩa tổng cộng.
\end{baitoan}

\begin{baitoan}[\cite{TLCT_THCS_Toan_6_so_hoc}, 3.26., p. 26]
	Tìm $x,y\in\mathbb{N},x,y > 1$ thỏa mãn cả 2 điều kiện là $x + 1\divby y,y + 1\divby x$.
\end{baitoan}

\begin{baitoan}[\cite{Huy_so_hoc}, VD1, p. 6, HSG9 Tp. Thanh Hóa 2017]
	Cho $a,b,c\in\mathbb{Z}^\star$ thỏa $\left(\dfrac{1}{a} + \dfrac{1}{b} + \dfrac{1}{c}\right)^2 = \dfrac{1}{a^2} + \dfrac{1}{b^2} + \dfrac{1}{c^2}$. Chứng minh $a^3 + b^3 + c^3\divby9$.
\end{baitoan}

\begin{baitoan}[\cite{Huy_so_hoc}, VD2, p. 6]
	Cho $x,y,z\in\mathbb{N}^\star$ phân biệt. Chứng minh $(x - y)^5 + (y - z)^5 + (z - x)^5\divby5(x - y)(y - z)(z - x)$.
\end{baitoan}

\begin{baitoan}[\cite{Huy_so_hoc}, VD3, p. 6]
	Cho $n\in\mathbb{N}^\star,n\ge2$. Chứng minh $n^n - n^2 + n - 1\divby(n - 1)^2$.
\end{baitoan}

\begin{dinhnghia}[Giai thừa chẵn{\tt/}lẻ]
	$(2n)!!\coloneqq 1\cdot3\cdot5\cdots(2n - 1),(2n)!!\coloneqq2\cdot4\cdot6\cdots2n$, $\forall n\in\mathbb{N}$.
\end{dinhnghia}

\begin{baitoan}[\cite{Huy_so_hoc}, VD4, p. 7]
	(a) Chứng minh $1985!! + 1986!!\divby1987$. (b) Mở rộng.
\end{baitoan}

\begin{baitoan}[\cite{Huy_so_hoc}, VD5, p. 7]
	Tìm 2 chữ số $a,c$ để: (a) $A = \overline{12a56c}$, $A\divby5,A\divby9$. (b) $B = \overline{a432c}$, $A\divby2,A\divby5,A\divby9$.
\end{baitoan}

\begin{baitoan}[\cite{Huy_so_hoc}, VD6, p. 8]
	Chứng minh: (a) $A = \sum_{i=1}^{60} 2^i = 2 + 2^2 + 2^3 + \cdots + 2^{60}$, $A\divby3,A\divby7,A\divby15$. (b) $B = 3 + 3^3 + 3^5 + \cdots + 3^{1991}$, $B\divby13,B\divby41$. (c) Mở rộng.
\end{baitoan}

\begin{baitoan}[\cite{Huy_so_hoc}, VD7, p. 9]
	Cho $n\in\mathbb{N}^\star$. Chứng minh: (a) $A = 2n + \underbrace{1\ldots1}_n\divby3$. (b) $B = 10^n + 18n - 1\divby27$. (c) $C = 10^n + 72n - 1\divby81$.
\end{baitoan}

\begin{baitoan}[\cite{Huy_so_hoc}, VD8, p. 9]
	Cho $a,b\in\mathbb{N}^\star,a + 1\divby6,b + 2007\divby6$. Chứng minh $4^a + a + b\divby6$.
\end{baitoan}

\begin{baitoan}[\cite{Huy_so_hoc}, VD9, p. 10]
	Giả sử 3 số $\overline{abc},\overline{bca},\overline{cab}$ đều chia hết cho $37$/ Chứng minh $a^3 + b^3 + c^3 - 3abc\divby37$.
\end{baitoan}

\begin{baitoan}[\cite{Huy_so_hoc}, VD10, p. 10]
	Cho $m,n\in\mathbb{N}^\star$ thỏa $2022^m + 1\divby2022^n + 1$. Chứng minh $m\divby n$.
\end{baitoan}

\begin{baitoan}[\cite{Huy_so_hoc}, p. 11]
	Cho $a,b\in\mathbb{N},a\ge2,b\ge2,{\rm ƯCLN}(a,b) = 1$. Chứng minh $a^m + b^m\divby a^n + b^n\Rightarrow m\divby n$.
\end{baitoan}

%------------------------------------------------------------------------------%

\section{Divisibility Rule -- Dấu Hiệu Chia Hết}
\cite[\S8, pp. 24--25]{SBT_Toan_6_Canh_Dieu_tap_1}: 66. 67. 68. 69. 70. 71. 72. 73. 74. 75. 76. \cite[\S9, pp. 27--28]{SBT_Toan_6_Canh_Dieu_tap_1}: 77. 78. 79. 80. 81. 82. 83. 84. 85. 86. 87. 88.

\begin{baitoan}[\cite{Binh_boi_duong_Toan_6_tap_1}, H1, p. 29]
	Nối cột để được kết quả đúng.
	\begin{table}[H]
		\centering
		\begin{tabular}{|l|l|}
			\hline
			(a) $230 + 175$ & (1) chia hết cho 2 nhưng không chia hết cho 5. \\
			\hline
			(b) $2070 - 590$ & (2) chia hết cho 5 nhưng không chia hết cho 2. \\
			\hline
			(c) $747 + 350$ & (3) chia hết cho cả 2 \& 5. \\
			\hline
			& (4) không chia hết cho cả 2 \& 5. \\
			\hline
		\end{tabular}
	\end{table}
\end{baitoan}

\begin{baitoan}[\cite{Binh_boi_duong_Toan_6_tap_1}, H2, p. 30]
	Khi giải bài toán: ``Thêm 1 chữ số vào bên phải \& 1 chữ số vào bên trái số $2015$ để được 1 số mới chia hết cho cả $2,3,5$.'' Tìm kết quả sai: {\sf A.} $120150$. {\sf B.} $420150$. {\sf C.} $620150$. {\sf D.} $720150$.
\end{baitoan}

\begin{baitoan}[\cite{Binh_boi_duong_Toan_6_tap_1}, Mở rộng H2, p. 29]
	Thêm 1 chữ số vào bên phải \& 1 chữ số vào bên trái số $2015$ để được 1 số mới chia hết cho cả $2,3,5$. Tìm tất cả các cặp số có thể thêm vào.
\end{baitoan}

\begin{baitoan}[\cite{Binh_boi_duong_Toan_6_tap_1}, H3, p. 30]
	Trong khoảng từ $1010$ đến $1975$ có bao nhiêu số chia hết cho $3$?
\end{baitoan}

\begin{baitoan}[\cite{Binh_boi_duong_Toan_6_tap_1}, H4, p. 30]
	Thay các chữ cái khác nhau bởi các chữ số khác nhau: $\rm HANOI + HANOI + HANOI = \overline{TT221}$.
\end{baitoan}

\begin{baitoan}[\cite{Binh_boi_duong_Toan_6_tap_1}, VD1, p. 30]
	2 bạn Egg \& Chicken đi mua $18$ gói bánh \& $12$ gói kẹo để chuẩn bị cho buổi liên hoan lớp. Egg đưa cho cô bán hàng 3 tờ tiền, mỗi tờ có mệnh giá $50000$ đồng \& được trả lại $22000$ đồng. Thấy vậy, Chicken liền nói: ``Cô tính sai rồi!'' Chicken đúng hay sai? Vì sao?
\end{baitoan}

\begin{baitoan}[\cite{Binh_boi_duong_Toan_6_tap_1}, VD2, p. 30]
	Chứng minh $(n + 29)(n + 30)\divby2$, $\forall n\in\mathbb{N}$.
\end{baitoan}

\begin{baitoan}[Tính chia hết cho 2 của 1 tích]
	(a) Với $a,b\in\mathbb{N}$ thỏa điều kiện nào thì $(n + a)(n + b)\divby2$, $\forall n\in\mathbb{N}$? (b) Với $a,b,c\in\mathbb{N}$ thỏa điều kiện nào thì $(n + a)(n + b)(n + c)\divby2$, $\forall n\in\mathbb{N}$? (c) Cho $n\in\mathbb{N}^\star$. Với $a_1,a_2,\ldots,a_n\in\mathbb{N}$ thỏa điều kiện nào thì $\prod_{i=1}^n (m + a_i) = (m + a_1)(m + a_2)\cdots(m + a_n)\divby2$, $\forall n\in\mathbb{N}$?
\end{baitoan}

\begin{baitoan}[Tính chia hết cho 3 của 1 tích]
	(a) Với $a,b\in\mathbb{N}$ thỏa điều kiện nào thì $(n + a)(n + b)\divby3$, $\forall n\in\mathbb{N}$? (b) Với $a,b,c\in\mathbb{N}$ thỏa điều kiện nào thì $(n + a)(n + b)(n + c)\divby3$, $\forall n\in\mathbb{N}$? (c) Cho $n\in\mathbb{N}^\star$. Với $a_1,a_2,\ldots,a_n\in\mathbb{N}$ thỏa điều kiện nào thì $\prod_{i=1}^n (m + a_i) = (m + a_1)(m + a_2)\cdots(m + a_n)\divby3$, $\forall n\in\mathbb{N}$?
\end{baitoan}

\begin{baitoan}[\cite{Binh_boi_duong_Toan_6_tap_1}, VD3, p. 31]
	Chứng minh $39^{2015} + 11^{2016}\divby10$.
\end{baitoan}

\begin{baitoan}
	Với $a,b\in\mathbb{N}$ thỏa điều kiện nào thì: (a) $39^a + 11^b\divby10$? (b) $(\overline{a_ma_{m-1}\ldots a_19})^a + (\overline{b_nb_{n-1}\ldots b_11})^b\divby10$ với $a_i,b_j\in\{0,1,2,\ldots,9\}$, $\forall i = 1,2,\ldots,m$, $\forall j = 1,2,\ldots,n$, $a_nb_m\ne0$?
\end{baitoan}

\begin{baitoan}[\cite{Binh_boi_duong_Toan_6_tap_1}, VD4, p. 31]
	Thay dấu $+$ hoặc $-$ vào các dấu $\star$ trong dãy tính sau để được kết quả là 1 số chia hết cho $2$: $10\star9\star8\star7\star6\star5\star4\star3\star2\star1$.
\end{baitoan}

\begin{baitoan}
	Thay dấu $+$ hoặc $-$ vào các dấu $\star$ trong dãy tính sau để được kết quả là 1 số chia hết cho $2$: $n\star(n - 1)\star(n - 2)\star\cdots3\star2\star1$ với $n\in\mathbb{N}$.
\end{baitoan}

\begin{baitoan}[\cite{Binh_boi_duong_Toan_6_tap_1}, VD5, p. 32]
	Viết các số tự nhiên liên tiếp từ $10$ đến $99$ ta được số $A$. Hỏi $A$ có chia hết cho $9$ không? Vì sao?
\end{baitoan}

\begin{baitoan}
	Cho $n\in\mathbb{N}^\star$. Viết các số tự nhiên liên tiếp từ $10^n$ (số tự nhiên nhỏ nhất có $n + 1$ chữ số) đến $10^{n+1} - 1$ (số tự nhiên lớn nhất có $n + 1$ chữ số) ta được số $A$. Hỏi $A$ có chia hết cho $9$ không? Vì sao?
\end{baitoan}

\begin{baitoan}[\cite{Binh_boi_duong_Toan_6_tap_1}, VD6, p. 32]
	Tìm 2 chữ số $x,y$ biết: (a) $\overline{38x5y}$ chia hết cho $2,5,9$. (b) $\overline{12x3y}\divby45$.
\end{baitoan}

\begin{baitoan}[\cite{Binh_boi_duong_Toan_6_tap_1}, VD7, p. 32]
	Thay $a,b$ bằng các chữ số thích hợp để số $ \overline{2a83b}$ chia hết cho $3$ \& chia cho $5$ dư $1$.
\end{baitoan}

\begin{baitoan}[\cite{Binh_boi_duong_Toan_6_tap_1}, VD8, p. 33]
	Tìm 2 số tự nhiên chia hết cho $9$, biết tổng của chúng bằng $\overline{35\star1}$ \& số lớn gấp đôi số bé.
\end{baitoan}

\begin{baitoan}[\cite{Binh_boi_duong_Toan_6_tap_1}, VD9, p. 33]
	Tìm chữ số $a$ sao cho $\overline{95a14}\divby11$.
\end{baitoan}

\begin{baitoan}[\cite{Binh_boi_duong_Toan_6_tap_1}, 4.1., p. 33]
	Từ 3 trong 5 chữ số $2,5,7,8,0$, ghép thành số có 3 chữ số khác nhau thỏa mãn 1 trong các điều kiện: (a) Là số lớn nhất chia hết cho $2$. (b) Là số nhỏ nhất chia hết cho $2$. (c) Là số lớn nhất chia hết cho $5$. (d) Là số nhỏ nhất chia hết cho $5$. (e) Là số lớn nhất chia hết cho $9$. (f) Là số nhỏ nhất chia hết cho $9$. (g) Là số lớn nhất chia hết cho $3$. (h) Là số nhỏ nhất chia hết cho $3$.
\end{baitoan}

\begin{baitoan}[\cite{Binh_boi_duong_Toan_6_tap_1}, 4.2., p. 33]
	Dùng 3 trong 4 số $,2,4,6,8$, viết tất cả các số tự nhiên có 3 chữ số chia hết cho cả 3 số $2,3,9$.
\end{baitoan}

\begin{baitoan}[\cite{Binh_boi_duong_Toan_6_tap_1}, 4.3., p. 33]
	Có $10$ mẩu que lần lượt dài {\rm1 cm, 2cm, 3cm, $\ldots$, 10 cm}. Hỏi có thể dùng cả $10$ mẫu que đó để xếp thành 1 tam giác có 3 cạnh bằng nhau không?
\end{baitoan}

\begin{baitoan}[\cite{Binh_boi_duong_Toan_6_tap_1}, 4.4., p. 33]
	Chứng minh: (a) $10^{2015} + 8\divby18$. (b) $10^{21} + 20\divby6$.
\end{baitoan}

\begin{baitoan}[\cite{Binh_boi_duong_Toan_6_tap_1}, 4.5., p. 33]
	Chứng minh $(n + 11)(n + 12)\divby2$, $\forall n\in\mathbb{N}$.
\end{baitoan}

\begin{baitoan}[\cite{Binh_boi_duong_Toan_6_tap_1}, 4.6., p. 33]
	Chứng minh tích của 3 số tự nhiên chẵn liên tiếp chia hết cho $48$.
\end{baitoan}

\begin{baitoan}[\cite{Binh_boi_duong_Toan_6_tap_1}, 4.7., p. 33]
	Tìm số tự nhiên có 5 chữ số, các chữ số giống nhau, biết số đó chia cho $5$ dư $4$ \& chia hết cho $2$.
\end{baitoan}

\begin{baitoan}[\cite{Binh_boi_duong_Toan_6_tap_1}, 4.8., p. 34]
	Tìm 2 chữ số $x,y$ biết: (a) $\overline{2x98y}$ chia hết cho $2,3,5$. (b) $\overline{43xy5}\divby45$. (c) $\overline{21x7y}$ chia hết cho $5,18$.
\end{baitoan}

\begin{baitoan}[\cite{Binh_boi_duong_Toan_6_tap_1}, 4.9., p. 34]
	Tìm chữ số $a$ để $\overline{aaaaa96}$ chia hết cho cả $3$ \& $8$.
\end{baitoan}

\begin{baitoan}[\cite{Binh_boi_duong_Toan_6_tap_1}, 4.10., p. 34]
	Tìm chữ số $a$ để $\overline{1aaa1}\divby11$.
\end{baitoan}

\begin{baitoan}[\cite{Binh_boi_duong_Toan_6_tap_1}, 4.11., p. 34]
	Cho $a\in\mathbb{N}$. Đổi chỗ các chữ số của $a$ để được số $b$ gấp $3$ lần số $a$. Chứng minh $a\divby27$.
\end{baitoan}

\begin{baitoan}[\cite{Binh_boi_duong_Toan_6_tap_1}, 4.12., p. 34]
	Cho $n\in\mathbb{N}^\star$. Chứng minh: (a) $6^n - 1\divby5$. (b) $10^n + 18n - 1\divby27$.
\end{baitoan}

\begin{baitoan}[\cite{Binh_boi_duong_Toan_6_tap_1}, 4.13., p. 34]
	Tìm 2 chữ số $a,b$ sao cho: (a) $\overline{71ab}$ chia hết cho $9$, cho $2$, \& chia cho $5$ dư $3$. (b) $\overline{15a3b}$ chia hết cho $2$, chia hết cho $9$, \& chia cho $5$ dư $4$.
\end{baitoan}

\begin{baitoan}[\cite{Binh_boi_duong_Toan_6_tap_1}, 4.14., p. 34]
	Tìm 2 số tự nhiên liên tiếp có 2 chữ số, biết 1 số chia hết cho $4$, số kia chia hết cho $25$.
\end{baitoan}

\begin{baitoan}[\cite{Binh_boi_duong_Toan_6_tap_1}, 4.15., p. 34]
	Tìm số tự nhiên có 4 chữ số sao cho khi nhân số đó với $9$ ta được số mới gồm chính các chữ số của số ấy nhưng viết theo thứ tự ngược lại.
\end{baitoan}

\begin{baitoan}[\cite{Binh_boi_duong_Toan_6_tap_1}, 4.16., p. 34, Thái Lan]
	Nếu đem số $31513$ \& số $34369$ chia cho cùng 1 số có 3 chữ số thì 2 phép chia có số dư bằng nhau. Tìm số dư của 2 phép chia đó.
\end{baitoan}

\begin{baitoan}[\cite{Binh_boi_duong_Toan_6_tap_1}, 4.17., p. 34]
	Chứng minh hiệu của 1 số \& tổng các chữ số của nó chia hết cho $9$.
\end{baitoan}

\begin{dinhnghia}[Hàm tổng các chữ số]
	Ký hiệu $S(n)$ là tổng các chữ số của $n\in\mathbb{N}$.
\end{dinhnghia}

\begin{baitoan}[\cite{Binh_boi_duong_Toan_6_tap_1}, 4.18., p. 34]
	Tìm $n\in\mathbb{N}$ biết $n + S(n) = 88$.
\end{baitoan}

\begin{baitoan}[\cite{Tuyen_Toan_6}, VD23, p. 23]
	Chứng minh: $9^{2n} - 1$ chia hết cho $2$ \& $5,\forall n\in\mathbb{N}$.
\end{baitoan}

\begin{baitoan}[\cite{Tuyen_Toan_6}, VD24, p. 24]
	Cho số $A = \overline{76a23}$. (a) Tìm chữ số $a$ để $A\divby9$. (b) Trong các giá trị vừa tìm được của $a$, có giá trị nào để $A\divby11$ không?
\end{baitoan}

\begin{baitoan}[\cite{Tuyen_Toan_6}, VD25, p. 24]
	Theo dương lịch cứ $4$ năm lại có 1 năm nhuận, i.e., năm chia hết cho $4$. Tuy nhiên các năm có tận cùng bằng 2 chữ số $0$ chỉ được coi là năm nhuận nếu chúng cũng chia hết cho $400$. Tính số năm nhuận từ năm $2000$--$2100$.
\end{baitoan}

\begin{baitoan}[\cite{Tuyen_Toan_6}, 102., p. 24]
	Cho $n\in\mathbb{N}$. Chứng minh $6^n - 1\divby5$.
\end{baitoan}

\begin{baitoan}[\cite{Tuyen_Toan_6}, 103., p. 24]
	Cho $n\in\mathbb{N}$. Chứng minh $5^n - 1\divby4$.
\end{baitoan}

\begin{baitoan}[\cite{Tuyen_Toan_6}, 104., p. 24]
	Chứng minh: (a) $942^{60} - 351^{37}\divby5$. (b) $99^5 - 98^4 + 97^3 - 96^2$ chia hết cho $2$ \& $5$.
\end{baitoan}

\begin{baitoan}[\cite{Tuyen_Toan_6}, 105., p. 24]
	Có 2 số tự nhiên nào mà tổng bằng $3456$ \& số lớn gấp $4$ lần số nhỏ không?
\end{baitoan}

\begin{baitoan}[\cite{Tuyen_Toan_6}, 106., p. 24]
	Cho $a,b\in\mathbb{N}$. Hỏi số $ab(a + b)$ có tận cùng bằng $9$ không?
\end{baitoan}

\begin{baitoan}[\cite{Tuyen_Toan_6}, 107., p. 24]
	Cho $n\in\mathbb{N}$, $A = n^2 + n + 1$. Chứng minh $A\not{\divby}\ 4$, $A\not{\divby}\ 5$.
\end{baitoan}

\begin{baitoan}[\cite{Tuyen_Toan_6}, 108., p. 24]
	Cho số $\overline{abc}\not{\divby}\ 3$. Phải viết số này liên tiếp nhau mấy lần để được 1 số chia hết cho $3$?
\end{baitoan}

\begin{baitoan}[\cite{Tuyen_Toan_6}, 109., p. 24]
	1 số tự nhiên có chữ số đầu tiên lớn hơn chữ số hàng đơn vị. Khi viết số đó theo thứ tự ngược lại thì được 1 số mới kém số cũ là 1 trong 3 số $2020,2021,2022$. Hiệu của chúng là số nào trong 3 số đó?
\end{baitoan}

\begin{baitoan}[\cite{Tuyen_Toan_6}, 110., p. 24]
	Cho biểu thức $A = 1494\cdot1495\cdot1496$. Không thực hiện phép tính, chứng minh: (a) $A\divby180$. (b) $A\divby495$.
\end{baitoan}

\begin{baitoan}[\cite{Tuyen_Toan_6}, 111., p. 24]
	Chứng minh $\forall n\in\mathbb{N}$: (a) $10^n - 1\divby9$. (b) $10^n + 8\divby9$.
\end{baitoan}

\begin{baitoan}[\cite{Tuyen_Toan_6}, 112., p. 25]
	Chứng minh hiệu của 1 số \& tổng các chữ số của nó thì chia hết cho $9$.
\end{baitoan}

\begin{baitoan}[\cite{Tuyen_Toan_6}, 113., p. 25]
	Cho số $A = 8n + \underbrace{1\ldots1}_n$, với $n\in\mathbb{N}^\star$. Chứng minh $A\divby9$.
\end{baitoan}

\begin{baitoan}[\cite{Tuyen_Toan_6}, 114., p. 25]
	Lấy 1 mảnh giấy cắt ra làm $4$ mảnh nhỏ. Lấy 1 mảnh bất kỳ cắt ra  thành $4$ mảnh khác. Cứ thế tiếp tục nhiều lần. (a) Hỏi khi ngừng cắt theo quy luật trên thì có thể được tất cả $60$ mảnh giấy nhỏ không? (b) Phải cắt tất cả bao nhiêu mảnh giấy theo quy luật trên để được tất cả $52$ mảnh giấy nhỏ?
\end{baitoan}

\begin{baitoan}[\cite{Tuyen_Toan_6}, 115., p. 25]
	Chọn bất kỳ $90$ số trong $100$ số tự nhiên từ $1$--$100$ thì có ít nhất bao nhiêu số chia hết cho $9$?
\end{baitoan}

\begin{baitoan}[\cite{Binh_Toan_6_tap_1}, VD25, p. 25]
	Tìm $a\in\mathbb{N}$ có 4 chữ số, $a\divby5,a\divby27$, biết 2 chữ số giữa của $a$ là $97$.
\end{baitoan}

\begin{baitoan}[\cite{Binh_Toan_6_tap_1}, VD26, p. 25]
	2 số tự nhiên $a,2a$ đều có tổng các chữ số bằng $k$. Chứng minh $a\divby9$.
\end{baitoan}

\begin{baitoan}[\cite{Binh_Toan_6_tap_1}, VD27, p. 25]
	Chứng minh số gồm $27$ chữ số $1$ thì chia hết cho $27$.
\end{baitoan}

\begin{baitoan}[\cite{Binh_Toan_6_tap_1}, VD28, p. 26]
	Tìm $a\in\mathbb{N}$ nhỏ nhất sao cho tổng các chữ số của $a$ \& tổng các chữ số của $a + 1$ đều chia hết cho $5$.
\end{baitoan}

\begin{baitoan}[\cite{Binh_Toan_6_tap_1}, VD29, p. 26]
	Cho số tự nhiên $\overline{ab}$ bằng $3$ lần tích các chữ số của nó. (a) Chứng minh $b\divby a$. (b) Giả sử $b = ka$, với $k\in\mathbb{N}$. Chứng minh $k$ là ước của $10$. (c) Tìm các số $\overline{ab}$ thỏa mãn.
\end{baitoan}

\begin{baitoan}[\cite{Binh_Toan_6_tap_1}, VD30, p. 27]
	Tìm $a\in\mathbb{N}$ có 2 chữ số biết $a$ chia hết cho tích các chữ số của nó.
\end{baitoan}

\begin{baitoan}[\cite{Binh_Toan_6_tap_1}, VD31, p. 27]
	Cho $a\in\mathbb{N}$. Gọi $b$ là tổng các chữ số của $a$, $c$ là tổng các chữ số của $b$. Biết $a + 3b + 6c = 1395$. (a) Chứng minh $a\divby9$. (b) Tìm $a$.
\end{baitoan}

\begin{baitoan}[\cite{Binh_Toan_6_tap_1}, 142., p. 27]
	Cho $a = 13! - 11!$. Tìm số dư khi chia $a$ cho $2,5,155$.
\end{baitoan}

\begin{baitoan}[\cite{Binh_Toan_6_tap_1}, 143., p. 27]
	Tổng các số tự nhiên từ $1$ đến $154$ có chia hết cho $2,5$ không?
\end{baitoan}

\begin{baitoan}[\cite{Binh_Toan_6_tap_1}, 144., p. 27]
	Cho $a = \sum_{i=0}^9 11^i = 1 + 11 + \cdots + 11^8 + 11^9$. Chứng minh $a\divby5$.
\end{baitoan}

\begin{baitoan}[\cite{Binh_Toan_6_tap_1}, 145., p. 27]
	Chứng minh $n^2 + n + 6\not{\divby}\ 5$, $\forall n\in\mathbb{N}$.
\end{baitoan}

\begin{baitoan}[\cite{Binh_Toan_6_tap_1}, 146., p. 28]
	Trong các số tự nhiên nhỏ hơn $1000$, có bao nhiêu số chia hết cho $2$ nhưng không chia hết cho $5$?
\end{baitoan}

\begin{baitoan}[\cite{Binh_Toan_6_tap_1}, 147., p. 28]
	Dùng cả $10$ chữ số từ $0$--$9$, viết thành số tự nhiên: (a) nhỏ nhất chia hết cho $4$. (b) lớn nhất chia hết cho $4$.
\end{baitoan}

\begin{baitoan}[\cite{Binh_Toan_6_tap_1}, 148., p. 28]
	Dùng $10$ chữ số khác nhau, viết số chia hết cho $8$ có $10$ chữ số sao cho số đó có giá trị: (a) lớn nhất. (b) nhỏ nhất.
\end{baitoan}

\begin{baitoan}[\cite{Binh_Toan_6_tap_1}, 149., p. 28]
	Dùng $10$ chữ số khác nhau, viết số chia hết cho $25$ có $10$ chữ số sao cho số đó có giá trị: (a) nhỏ nhất. (b) lớn nhất.
\end{baitoan}

\begin{baitoan}[\cite{Binh_Toan_6_tap_1}, 150., p. 28]
	Tìm 2 số tự nhiên liên tiếp nhỏ nhất sao cho tổng các chữ số của mỗi số đều: (a) chia hết cho $8$. (b) chia hết cho $17$.
\end{baitoan}

\begin{baitoan}[\cite{Binh_Toan_6_tap_1}, 151., p. 28]
	Tìm các số tự nhiên chia cho $4$ dư $1$, còn chia cho $25$ thì dư $3$.
\end{baitoan}

\begin{baitoan}[\cite{Binh_Toan_6_tap_1}, 152., p. 28]
	Tìm các số tự nhiên chia cho $8$ dư $3$, còn chia cho $125$ thì dư $12$.
\end{baitoan}

\begin{baitoan}[\cite{Binh_Toan_6_tap_1}, 153., p. 28]
	Có phép trừ 2 số tự nhiên nào mà số trừ gấp 3 lần hiệu \& số bị trừ bằng $1030$ không?
\end{baitoan}

\begin{baitoan}[\cite{Binh_Toan_6_tap_1}, 154., p. 28]
	Điền các chữ số thích hợp vào dấu $\star$ sao cho: (a) $521\star\divby8$. (b) $2\star8\star7\divby9$, biết chữ số hàng chục lớn hơn chữ số hàng nghìn là $2$.
\end{baitoan}

\begin{baitoan}[\cite{Binh_Toan_6_tap_1}, 155., p. 28]
	Tìm 2 chữ số $a,b$ sao cho: (a) $a - b = 4,\overline{7a5b1}\divby3$. (b) $a - b = 6,\overline{4a7} + \overline{1b5}\divby9$.
\end{baitoan}

\begin{baitoan}[\cite{Binh_Toan_6_tap_1}, 156., p. 28]
	Tìm số tự nhiên có 3 chữ số, chia hết cho $5,9$, biết chữ số hàng chục bằng trung bình cộng của 2 chữ số kia.
\end{baitoan}

\begin{baitoan}[\cite{Binh_Toan_6_tap_1}, 157., p. 28]
	Tìm 2 số tự nhiên chia hết cho $9$, biết: (a) Tổng của chúng bằng $\star657$ \& hiệu của chúng bằng $5\star91$. (b) Tổng của chúng bằng $513\star$ \& số lớn gấp đôi số nhỏ.
\end{baitoan}

\begin{baitoan}[\cite{Binh_Toan_6_tap_1}, 158., p. 28]
	An làm phép tính trừ trong đó số bị trừ là số có 3 chữ số, số trừ là số gồm chính 3 chữ số ấy viết theo thứ tự ngược lại. An tính được hiệu bằng $188$. Chứng minh An đã tính sai.
\end{baitoan}

\begin{baitoan}[\cite{Binh_Toan_6_tap_1}, 159., p. 28]
	Tìm $a\in\mathbb{N}$ có 3 chữ số, chia hết cho $45$, biết hiệu giữa số đó \& số gồm chính 3 chữ số ấy viết theo thứ tự ngược lại bằng $297$.
\end{baitoan}

\begin{baitoan}[\cite{Binh_Toan_6_tap_1}, 160., p. 28]
	Tìm $a\in\mathbb{N}$  mà khi xóa chữ số tận cùng của nó thì số ấy giảm đi $1415$ đơn vị.
\end{baitoan}

\begin{baitoan}[\cite{Binh_Toan_6_tap_1}, 161., p. 28]
	1 số tự nhiên gọi là {\rm số hào hiệp} nếu nó chia hết cho mỗi chữ số của nó \& chia hết cho tổng các chữ số của nó, e.g., $12$ là số hào hiệp, $15$ không là số hào hiệp. Tìm số hào hiệp nhỏ nhất chia hết cho $11$.
\end{baitoan}

\begin{baitoan}[\cite{Binh_Toan_6_tap_1}, 162., p. 28]
	Tìm số tự nhiên nhỏ nhất tạo thành bởi 2 chữ số $3,7$ mà chia hết cho $3$ \& chia hết cho $7$.
\end{baitoan}

\begin{baitoan}[\cite{Binh_Toan_6_tap_1}, 163., p. 29]
	Tìm $x\in\mathbb{N}$ biết tổng các chữ số của $x$ bằng $y$, tổng các chữ số của $y$ bằng $z$ \& $x + y + z = 60$.
\end{baitoan}

\begin{baitoan}[\cite{Binh_Toan_6_tap_1}, 164., p. 29]
	Tìm $a\in\mathbb{N}$ biết $a + 6b + 9c = 1551$, trong đó $b$ là tổng các chữ số của $a$, $c$ là tổng các chữ số của $b$.
\end{baitoan}

\begin{baitoan}[\cite{Binh_Toan_6_tap_1}, 165., p. 29]
	Chứng minh: (a) $10^{28} + 8\divby72$. (b) $8^8 + 2^{20}\divby17$.
\end{baitoan}

\begin{baitoan}[\cite{Binh_Toan_6_tap_1}, 166., p. 29]
	(a) Cho $A = \sum_{i=1}^{60} 2^i = 2 + 2^2 + \cdots + 2^{60}$. Chứng minh $A$ chia hết cho $3,7,15$. (b) Cho $B = 3 + 3^3 + 3^5 + \cdots + 3^{1991}$. Chứng minh $B$ chia hết cho $13,41$.
\end{baitoan}

\begin{baitoan}[\cite{Binh_Toan_6_tap_1}, 167., p. 29]
	Chứng minh: (a) $2n + \underbrace{1\ldots1}_{n}\divby3$. (b) $10^n + 18n - 1\divby27$. (c) $10^n + 72n -1\divby81$.
\end{baitoan}

\begin{baitoan}[\cite{Binh_Toan_6_tap_1}, 168., p. 29]
	Chứng minh: (a) Số gồm $81$ chữ số $1$ thì chia hết cho $81$. (b) Số gồm $27$ nhóm chữ số $10$ thì chia hết cho $27$.
\end{baitoan}

\begin{baitoan}[\cite{Binh_Toan_6_tap_1}, 169., p. 29]
	2 số tự nhiên $a,4a$ có tổng các chữ số bằng nhau. Chứng minh $a\divby3$.
\end{baitoan}

\begin{baitoan}[\cite{Binh_Toan_6_tap_1}, 170., p. 29]
	(a) Tổng các chữ số của $3^{100}$ viết trong hệ thập phân có thể bằng $459$ không? (b) Tổng các chữ số của $3^{1000}$ là $a$, tổng các chữ số của $a$ là $b$, tổng các chữ số của $b$ là $c$. Tính $c$.
\end{baitoan}

\begin{baitoan}[\cite{Binh_Toan_6_tap_1}, 171., p. 29]
	Cho $a,b\in\mathbb{N}$ tùy ý có số dư trong phép chia cho $9$ lần lượt là $r_1,r_2$. Chứng minh $r_1r_2$ \& $ab$ có cùng số dư trong phép chia cho $9$.
\end{baitoan}

\begin{baitoan}[\cite{Binh_Toan_6_tap_1}, 172., p. 29]
	1 số tự nhiên chia hết cho $4$ có 3 chữ số đều chẵn, khác nhau, \& khác $0$. Chứng minh tồn tại cách đổi vị trí các chữ số để được 1 số mới chia hết cho $4$.
\end{baitoan}

\begin{baitoan}[\cite{Binh_Toan_6_tap_1}, 173., p. 29]
	Tìm số $\overline{abcd}$ , biết số đó chia hết cho tích 2 số $\overline{ab}\cdot\overline{cd}$.
\end{baitoan}

\begin{baitoan}[\cite{Binh_Toan_6_tap_1}, 174., p. 29]
	Tìm $a\in\mathbb{N}$ có 5 chữ số, biết số đó bằng $45$ lần tích các chữ số của $a$.
\end{baitoan}

\begin{baitoan}[\cite{Binh_Toan_6_tap_1}, 175., p. 29]
	1 cửa hàng có $6$ hòm hàng với khối lượng {\rm316 kg, 327 kg, 336 kg, 338 kg, 349 kg, 351 kg}. Cửa hàng đó đã bán $5$ hòm, trong đó khối lượng hàng bán buổi sáng gấp $4$ lần khối lượng hàng bán buổi chiều. Hỏi hòm còn lại là hòm nào?
\end{baitoan}

\begin{baitoan}[\cite{Binh_Toan_6_tap_1}, 176., p. 29]
	Từ 4 chữ số $1,2,3,4$, lập tất cả các số tự nhiên có 4 chữ số gồm cả 4 chữ số ấy. Trong các số đó, có tồn tại 2 số nào mà 1 số chia hết cho số còn lại không?
\end{baitoan}

\begin{baitoan}[\cite{Binh_Toan_6_tap_1}, 177., p. 29]
	Chứng minh trong tất cả các số tự nhiên khác nhau có 7 chữ số lập bởi cả 7 chữ số $1,2,3,4,5,6,7$, không có 2 số nào mà 1 số chia hết cho số còn lại.
\end{baitoan}

\begin{baitoan}[\cite{Binh_Toan_6_tap_1}, 178., pp. 29--30]
	Cho 4 số tự nhiên không nhất thiết khác nhau. Nếu cộng 3 số trong 4 số đó, ta được 4 tổng, trong đó 3 tổng là $54,55,59$, tổng thứ 4 bằng 1 trong 3 tổng trên. (a) Tính tổng thứ 4. (b) Tính tổng của 4 số đã cho. (c) Tính 4 số đã cho.
\end{baitoan}

\begin{baitoan}
	Tìm \& chứng minh dấu hiệu chia hết cho $11$.
\end{baitoan}

\begin{baitoan}[Tích 2 số tự nhiên liên tiếp]
	Cho $n\in\mathbb{N}$. Xét tích 2 số tự nhiên liên tiếp $A_2(n) = n(n + 1)$. (a) Chứng minh $A_2(n)\divby2$, $\forall n\in\mathbb{N}$. (b) Chứng minh tổng của $n$ số chẵn dương đầu tiên bằng $A_2(n)$. (c) Tìm điều kiện của $n$ để $A_2(n)$ chia hết cho $4,8,16,\ldots,2^m$ với $m\in\mathbb{N}^\star$.
\end{baitoan}

\begin{baitoan}[Tích 2 số tự nhiên chẵn liên tiếp]
	Cho $n\in\mathbb{N}$. Xét tích 2 số tự nhiên chẵn liên tiếp $E_2(n) = 2n(2n + 2)$. (a) Chứng minh $E_2(n)\divby8$, $\forall n\in\mathbb{N}$. (b) Chứng minh tổng của $n$ số chẵn dương đầu tiên bằng $4E_2(n)$. (c) Tìm điều kiện của $n$ để $E_2(n)$ chia hết cho $2^m$ với $m\in\mathbb{N}^\star$.
\end{baitoan}

\begin{baitoan}[Tích 3 số tự nhiên liên tiếp]
	Cho $n\in\mathbb{N}$. Xét tích 3 số tự nhiên liên tiếp $A_3(n) = n(n + 1)(n + 2)$. (a) Chứng minh $A_3(n)\divby6$ với mọi $n$ lẻ. (b) Chứng minh $A_3(n)\divby24$ với mọi $n$ chẵn. (c${}^\star$) Tìm điều kiện của $n$ để $A_3(n)$ chia hết cho $2^a\cdot3^b$ với $a,b\in\mathbb{N}^\star$.
\end{baitoan}

\begin{baitoan}[Tích 4 số tự nhiên liên tiếp]
	Cho $n\in\mathbb{N}$. Xét tích 4 số tự nhiên liên tiếp $A_4(n) = n(n + 1)(n + 2)(n + 3)$. (a) Chứng minh $A_4(n)\divby24$, $\forall n\in\mathbb{N}$. (b${}^\star$) Tìm điều kiện của $n$ để $A_4(n)$ chia hết cho $2^a\cdot3^b$ với $a,b\in\mathbb{N}^\star$.
\end{baitoan}

%------------------------------------------------------------------------------%

\section{Prime. Composite -- Số Nguyên Tố. Hợp Số}
\cite[\S10, pp. 29--30]{SBT_Toan_6_Canh_Dieu_tap_1}: 89. 90. 91. 92. 93. 94. 95. 96. 97. 98. \cite[\S11, pp. 31--32]{SBT_Toan_6_Canh_Dieu_tap_1}: 99. 100. 101. 102. 103. 104. 105. 106. 107. 108.

\begin{baitoan}[\cite{Binh_boi_duong_Toan_6_tap_1}, H1, p. 36]
	Egg có $54$ viên bi \& muốn chia đều số bi đó vào các hộp. Tìm tất cả các cách chia thỏa mãn.
\end{baitoan}

\begin{baitoan}[\cite{Binh_boi_duong_Toan_6_tap_1}, H2, p. 36]
	(a) Số nào có phân tích ra thừa số nguyên tố là $2^3\cdot3^2\cdot7$. (b) Phân tích $2160$ ra thừa số nguyên tố.
\end{baitoan}

\begin{baitoan}[\cite{Binh_boi_duong_Toan_6_tap_1}, H3, p. 36]
	Tìm chữ số $a$ để $\overline{17a}$ là số nguyên tố.
\end{baitoan}

\begin{baitoan}[\cite{Binh_boi_duong_Toan_6_tap_1}, H4, p. 36]
	{\rm Đ{\tt/}S?} Ký hiệu $P$ là tập hợp các số nguyên tố. (a) $19\in P$. (b) $\{3,5,7\}\in P$. (c) $\{71,73\}\in P$. (d) $6\cdot7\cdot8\cdot9 - 5\cdot7\cdot11\in P$. (e) Mọi số nguyên tố đều có tận cùng là số lẻ.
\end{baitoan}

\begin{baitoan}[\cite{Binh_boi_duong_Toan_6_tap_1}, VD1, p. 37]
	Cho 1 phép chia có số bị chia bằng $236$ \& số dư bằng $15$. Tìm số chia \& thương.
\end{baitoan}

\begin{baitoan}[\cite{Binh_boi_duong_Toan_6_tap_1}, VD2, p. 37]
	Có bao nhiêu số là bội của $6$ trong khoảng từ $72$ đến $2016$?
\end{baitoan}

\begin{baitoan}[\cite{Binh_boi_duong_Toan_6_tap_1}, VD3, p. 37]
	Tìm $x\in\mathbb{N}$ sao cho $42\divby(2x + 5)$.
\end{baitoan}

\begin{baitoan}[\cite{Binh_boi_duong_Toan_6_tap_1}, VD4, p. 38]
	Tìm số nguyên tố $p$ sao cho $p + 2$ \& $p + 4$ cũng là 2 số nguyên tố.
\end{baitoan}

\begin{baitoan}[\cite{Binh_boi_duong_Toan_6_tap_1}, VD5, p. 38]
	Cho $p > 3$ \& $2p + 1$ là 2 số nguyên tố. Hỏi $4p + 1$ là số nguyên tố hay hợp số.
\end{baitoan}

\begin{baitoan}[\cite{Binh_boi_duong_Toan_6_tap_1}, VD6, p. 39]
	Tìm số nguyên tố bằng tổng của 2 số nguyên tố \& cũng bằng hiệu của 2 số nguyên tố khác.
\end{baitoan}

\begin{baitoan}[\cite{Binh_boi_duong_Toan_6_tap_1}, VD7, p. 39]
	Phân tích ra thừa số nguyên tố: (a) $2016^7$. (b) $30\cdot4\cdot1975$.
\end{baitoan}

\begin{baitoan}[\cite{Binh_boi_duong_Toan_6_tap_1}, VD8, p. 39]
	Tìm $n\in\mathbb{N}^\star$ thỏa $2 + 4 + 6 + \cdots + 2n = 870$.
\end{baitoan}

\begin{baitoan}[\cite{Binh_boi_duong_Toan_6_tap_1}, VD9, p. 40]
	Tìm $n\in\mathbb{N}^\star$ sao cho $p = (n - 2)(n^2 + n - 5)$ là số nguyên tố.
\end{baitoan}

\begin{baitoan}[\cite{Binh_boi_duong_Toan_6_tap_1}, 5.1., p. 40]
	Tìm tập hợp các số tự nhiên vừa là bội của $9$, vừa là ước của $72$.
\end{baitoan}

\begin{baitoan}[\cite{Binh_boi_duong_Toan_6_tap_1}, 5.2., p. 40]
	Tìm $x\in\mathbb{N}^\star$ thỏa: (a) $x - 1$ là ước của $24$. (b) $36$ là bội của $2x + 1$.
\end{baitoan}

\begin{baitoan}[\cite{Binh_boi_duong_Toan_6_tap_1}, 5.3., p. 40]
	Tìm $x,y\in\mathbb{N}^\star$ thỏa $(2x + 1)(y - 3) = 15$.
\end{baitoan}

\begin{baitoan}[\cite{Binh_boi_duong_Toan_6_tap_1}, 5.4., p. 40]
	Phân tích ra thừa số nguyên tố: (a) $1\cdot12\cdot78$. (b) $1930^8$.
\end{baitoan}

\begin{baitoan}[\cite{Binh_boi_duong_Toan_6_tap_1}, 5.5., p. 40]
	Chứng minh nếu $p$ là 1 số nguyên tố lớn hơn $3$ thì $(p - 1)(p + 1)$ chia hết cho $3$ \& cho $8$.
\end{baitoan}

\begin{baitoan}[\cite{Binh_boi_duong_Toan_6_tap_1}, 5.6., p. 40]
	Tìm chữ số $a$ để $\overline{23a}$ là số nguyên tố.
\end{baitoan}

\begin{baitoan}[\cite{Binh_boi_duong_Toan_6_tap_1}, 5.7., p. 40]
	Tìm số tự nhiên nhỏ nhất có đúng $18$ ước số.
\end{baitoan}

\begin{baitoan}[\cite{Binh_boi_duong_Toan_6_tap_1}, 5.8., p. 40]
	Chứng minh: Nếu 1 số tự nhiên có 3 chữ số tận cùng là $104$ thì số đó có ít nhất $4$ ước số.
\end{baitoan}

\begin{baitoan}[\cite{Binh_boi_duong_Toan_6_tap_1}, 5.9., p. 40]
	Tìm 2 số nguyên tố có tổng bằng $309$.
\end{baitoan}

\begin{baitoan}[\cite{Binh_boi_duong_Toan_6_tap_1}, 5.10., p. 40]
	Tìm số nguyên tố $p$ sao cho $p + 4,p + 8$ cũng là 2 số nguyên tố.
\end{baitoan}

\begin{baitoan}[\cite{Binh_boi_duong_Toan_6_tap_1}, 5.11., p. 40]
	Tìm số nguyên tố $p$ sao cho $p + 6,p + 8,p + 12,p + 14$ cũng là 4 số nguyên tố.
\end{baitoan}

\begin{baitoan}[\cite{Binh_boi_duong_Toan_6_tap_1}, 5.12., p. 40]
	Cho $p > 3$ \& $p + 4$ là 2 số nguyên tố. Chứng minh $p + 8$ là hợp số.
\end{baitoan}

\begin{baitoan}[\cite{Binh_boi_duong_Toan_6_tap_1}, 5.13., p. 40]
	Số $3^2 + 3^4 + 3^6 + \cdots + 3^{2012}$ là số nguyên tố hay hợp số?
\end{baitoan}

\begin{baitoan}[\cite{Binh_boi_duong_Toan_6_tap_1}, 5.14., p. 40]
	2 số nguyên tố được gọi là {\rm sinh đôi} nếu chúng là 2 số nguyên tố \& là 2 số lẻ liên tiếp, e.g., $3$ \& $5$, $11$ \& $13,\ldots$. Chứng minh số tự nhiên lớn hơn $4$ \& nằm giữa 2 số nguyên tố sinh đôi thì chia hết cho $6$.
\end{baitoan}

\begin{baitoan}[\cite{Binh_boi_duong_Toan_6_tap_1}, 5.15., p. 41]
	Tìm 3 số tự nhiên lẻ liên tiếp đều là số nguyên tố.
\end{baitoan}

\begin{baitoan}[\cite{Binh_boi_duong_Toan_6_tap_1}, 5.16., p. 41]
	Tìm $n\in\mathbb{N}^\star$ thỏa $1 + 3 + 5 + \cdots + (2n + 1) = 169$.
\end{baitoan}

\begin{baitoan}[\cite{Binh_boi_duong_Toan_6_tap_1}, 5.17., p. 41]
	Biết số $\overline{abc}$ khi phân tích ra t hừa số nguyên tố có thừa số $3$ \& thừa số $7$. Chứng minh số $a + 19b + 4c$ cũng có tính chất đó.
\end{baitoan}

\begin{baitoan}[\cite{Binh_boi_duong_Toan_6_tap_1}, 5.18., p. 41]
	Tìm chữ số $a$ sao cho số $\overline{aaa}$ là tổng của các số tự nhiên liên tiếp từ $1$ đến số $n$ nào đó.
\end{baitoan}

\begin{baitoan}
	Chứng minh tập hợp các số nguyên tố có vô hạn phần tử \& không có số nguyên tố lớn nhất.
\end{baitoan}
\noindent\textit{Hint.} Giả sử phản chứng: chỉ có hữu hạn số nguyên tố $p_1 < p_2 < \cdots < p_n$. Chứng minh $p\coloneqq\prod_{i=1}^n p_i + 1 = p_1p_2\cdots p_n + 1$ là 1 số nguyên tố lớn hơn mỗi số nguyên tố $p_i$, $\forall i\in\mathbb{N}$.

\begin{baitoan}[\cite{Tuyen_Toan_6}, VD26, p. 25]
	Tìm số nguyên tố $a$ để $4a + 11$ là số nguyên tố nhỏ hơn $30$.
\end{baitoan}

\begin{baitoan}[\cite{Tuyen_Toan_6}, VD27, p. 25]
	Cho $A = \sum_{i=1}^{100} 5^i = 5 + 5^2 + \cdots + 5^{100}$. (a) Hỏi A là số nguyên tố hay hợp số? (b) Số A có phải là số chính phương không?
\end{baitoan}

\begin{baitoan}[\cite{Tuyen_Toan_6}, VD28, p. 2]
	Tính cạnh của 1 hình vuông có diện tích $\rm5929\ m^2$.
\end{baitoan}

\begin{baitoan}[\cite{Tuyen_Toan_6}, 116., p. 26]
	Phân loại số nguyên tố, hợp số: (a) $A = 1\cdot3\cdot5\cdot7\cdots13 + 20$. (b) $B = 147\cdot247\cdot347 - 13$.
\end{baitoan}

\begin{baitoan}[\cite{Tuyen_Toan_6}, 117., p. 26]
	Tìm số bị chia \& thương trong phép chia: $9\star\star:17 = \star\star$. Biết thương là 1 số nguyên tố.
\end{baitoan}

\begin{baitoan}[\cite{Tuyen_Toan_6}, 118., p. 26]
	Cho $a,n\in\mathbb{N}^\star$. Biết $a^n\divby5$. Chứng minh $a^2 + 150\divby25$.
\end{baitoan}

\begin{baitoan}[\cite{Tuyen_Toan_6}, 119., p. 26]
	(a) Cho $n\in\mathbb{N}$, $n\not{\divby}\ 3$. Chứng minh $n^2$ chia cho $3$ dư $1$. (b) Cho $p$ là 1 số nguyên tố lớn hơn $3$. Hỏi $p^2 + 2021$ là số nguyên tố hay hợp số?
\end{baitoan}

\begin{baitoan}[\cite{Tuyen_Toan_6}, 120., p. 26]
	Cho $n\in\mathbb{N}$, $n > 2$, $n\not{\divby}\ 3$. Chứng minh 2 số $n^2\pm1$ không thể đồng thời là 2 số nguyên tố.
\end{baitoan}

\begin{baitoan}[\cite{Tuyen_Toan_6}, 121., p. 26]
	Cho $p > 3,p + 8$ đều là số nguyên tố. Hỏi $p + 100$ là số nguyên tố hay hợp số?
\end{baitoan}

\begin{baitoan}[\cite{Tuyen_Toan_6}, 122., p. 26]
	Phân tích ra thừa số nguyên tố bằng cách hợp lý nhất: (a) $700,9000,210000$. (b) $500,1600,18000$.
\end{baitoan}

\begin{baitoan}[\cite{Tuyen_Toan_6}, 123., p. 26]
	Đếm số ước số của: $90,540,3675$.
\end{baitoan}

\begin{baitoan}[\cite{Tuyen_Toan_6}, 124., p. 26]
	Tìm: (a) 2 số tự nhiên liên tiếp có tích bằng $1260$. (b) 3 số tự nhiên liên tiếp có tích bằng $3360$.
\end{baitoan}

\begin{baitoan}[\cite{Tuyen_Toan_6}, 125., p. 26]
	Tìm: (a) 3 số chẵn liên tiếp có tích bằng $5760$. (b) 3 số lẻ liên tiếp có tích bằng $19575$.
\end{baitoan}

\begin{baitoan}[\cite{Tuyen_Toan_6}, 126., p. 26]
	Tính cạnh của 1 hình lập phương biết thể tích của nó là $\rm1728\ cm^3$.
\end{baitoan}

\begin{baitoan}[\cite{Tuyen_Toan_6}, 127., p. 27]
	Chứng minh 1 số tự nhiên $\ne0$ có số lượng các ước là 1 số lẻ $\Leftrightarrow$ số tự nhiên đó là số chính phương.
\end{baitoan}

\begin{baitoan}[\cite{Tuyen_Toan_6}, 128., p. 27]
	Tìm $n\in\mathbb{N}^\star$ thỏa: (a) $2 + 4 + 6 + \cdots + 2n = 210$. (b) $1 + 3 + 5 + \cdots + (2n - 1) = 225$.
\end{baitoan}

\begin{baitoan}[\cite{Binh_Toan_6_tap_1}, VD32, p. 30]
	Điền các chữ số thích hợp trong phép phân tích ra thừa số nguyên tố: $\overline{abcd} = e\overline{fcga} = en\overline{abc} = enc\overline{ncf} = \ldots$
\end{baitoan}

\begin{baitoan}[\cite{Binh_Toan_6_tap_1}, VD33, p. 30]
	Tìm số nguyên tố $p$ sao cho $p + 2,p + 4$ cũng là 2 số nguyên tố.
\end{baitoan}

\begin{baitoan}[\cite{Binh_Toan_6_tap_1}, VD34, p. 31]
	1 số nguyên tố $p$ chia cho $42$ có số dư $r$ là hợp số. Tìm số dư $r$.
\end{baitoan}

\begin{baitoan}[\cite{Binh_Toan_6_tap_1}, VD35, p. 31]
	Tìm $n\in\mathbb{N}^\star$ nhỏ nhất sao cho $n! + 1$ là hợp số.
\end{baitoan}

\begin{baitoan}[\cite{Binh_Toan_6_tap_1}, 180., p. 31]
	(a) Đếm số số nguyên tố nhỏ hơn $100$. (b) Tính tổng tất cả các số nguyên tố nhỏ hơn $100$.
\end{baitoan}

\begin{baitoan}[\cite{Binh_Toan_6_tap_1}, 181., p. 31]
	Tổng của 3 số nguyên tố bằng $1012$. Tìm số nhỏ nhất trong 3 số nguyên tố đó.
\end{baitoan}

\begin{baitoan}[\cite{Binh_Toan_6_tap_1}, 182., p. 31]
	Tìm 4 số nguyên tố liên tiếp, sao cho tổng của chúng là số nguyên tố.
\end{baitoan}

\begin{baitoan}[\cite{Binh_Toan_6_tap_1}, 183., p. 31]
	Tổng của 2 số nguyên tố có thể bằng $2003$ không?
\end{baitoan}

\begin{baitoan}[\cite{Binh_Toan_6_tap_1}, 184., p. 31]
	Tìm 2 số tự nhiên sao cho tổng \& tích của chúng đều là số nguyên tố.
\end{baitoan}

\begin{baitoan}[\cite{Binh_Toan_6_tap_1}, 185., p. 31]
	Trong 1 cuộc phỏng vấn tuyển nhân viên làm việc ở Tập đoàn Microsoft của Mỹ, 1 ứng viên nhận được câu hỏi: Tìm số tiếp theo trong dãy $4,6,12,18,30,42,60,\ldots$ Nhờ có kiến thức về số nguyên tố, ứng viên đã trả lời đúng. Số tiếp theo của dãy là số nào?
\end{baitoan}

\begin{baitoan}[\cite{Binh_Toan_6_tap_1}, 186., p. 31]
	Phân loại số nguyên tố \& hợp số: (a) $a = \underbrace{1\ldots1}_{2001}, b = \underbrace{1\ldots1}_{2000}, c = 1010101, d = 1112111, e = \sum_{i=1}^{100} i! = 1! + 2! + \cdots + 100!, f = 3\cdot5\cdot7\cdot9 - 28, g = 311141111$.
\end{baitoan}

\begin{baitoan}[\cite{Binh_Toan_6_tap_1}, 187., p. 31]
	Tìm số nguyên tố có 3 chữ số biết nếu viết số đó theo thứ tự ngược lại thì ta được 1 số là lập phương của 1 số tự nhiên.
\end{baitoan}

\begin{baitoan}[\cite{Binh_Toan_6_tap_1}, 188., p. 31]
	Tìm số tự nhiên có 4 chữ số, chữ số hàng nghìn bằng chữ số hàng đơn vị, chữ số hàng trăm bằng chữ số hàng chục, \& số đó viết được dưới dạng tích của 3 số nguyên tố liên tiếp.
\end{baitoan}

\begin{baitoan}[\cite{Binh_Toan_6_tap_1}, 189., p. 32]
	Tìm số nguyên tố $p$ sao cho các số sau cũng là số nguyên tố: (a) $p + 2,p + 10$. (b) $p + 10,p + 20$. (c) $p + 2,p + 6,p + 8,p + 12,p + 14$.
\end{baitoan}

\begin{baitoan}[\cite{Binh_Toan_6_tap_1}, 190., p. 32]
	Tìm số nguyên tố biết số đó bằng tổng của 2 số nguyên tố \& bằng hiệu của 2 số nguyên tố.
\end{baitoan}

\begin{baitoan}[\cite{Binh_Toan_6_tap_1}, 191., p. 32]
	Cho 3 số nguyên tố lớn hơn $3$, trong đó số sau lớn hơn số trước là $d$ đơn vị. Chứng minh $d\divby6$.
\end{baitoan}

\begin{baitoan}[\cite{Binh_Toan_6_tap_1}, 192., p. 32]
	2 số nguyên tố gọi là {\rm sinh đôi} nếu chúng là 2 số nguyên tố lẻ liên tiếp. Chứng minh 1 số tự nhiên lớn hơn $3$ nằm giữa 2 số nguyên tố sinh đôi thì chia hết cho $6$.
\end{baitoan}

\begin{baitoan}[\cite{Binh_Toan_6_tap_1}, 193., p. 32]
	Cho $p > 3$ là số nguyên tố. Biết $p + 2$ cũng là số nguyên tố. Chứng minh $p + 1\divby6$.
\end{baitoan}

\begin{baitoan}[\cite{Binh_Toan_6_tap_1}, 194., p. 32]
	Cho $p > 3,p + 4$ là 2 số nguyên tố. Chứng minh $p + 8$ là hợp số.
\end{baitoan}

\begin{baitoan}[\cite{Binh_Toan_6_tap_1}, 195., p. 32]
	Cho $p,8p - 1$ là 2 số nguyên tố. Chứng minh $8p + 1$ là hợp số.
\end{baitoan}

\begin{baitoan}[\cite{Binh_Toan_6_tap_1}, 196., p. 32]
	1 ngày đầu năm 2002, Huy viết thư hỏi ngày sinh của Long \& nhận được thư trả lời: Mình sinh ngày $a$, tháng $b$, năm $1900 + c$, \& đến nay $d$ tuổi. Biết $abcd = 59007$. Huy đã tính được ngày sinh của anh Long \& kịp viết thư mừng sinh nhật bạn. Tìm ngày sinh của Long.
\end{baitoan}

\begin{baitoan}[\cite{Binh_Toan_6_tap_1}, 197., p. 32]
	1 số nguyên tố chia cho $30$ có số dư là $r$. Tìm $r$ biết $r$ không là số nguyên tố.
\end{baitoan}

\begin{baitoan}[\cite{Binh_Toan_6_tap_1}, 198., p. 32]
	Chứng minh: (a) Số $17$ không viết được dưới dạng tổng của 3 hợp số khác nhau. (b) Mọi số lẻ lớn hơn $17$ đều viết được dưới dạng tổng của 3 hợp số khác nhau.
\end{baitoan}

\begin{baitoan}[\cite{Binh_Toan_6_tap_1}, 199., p. 32]
	Tuổi trung bình của 8 người là $15$, trong đó tuổi mỗi người đều là số nguyên tố. Trong 4 người nhiều tuổi nhất, có 3 người $19$ tuổi. Tuổi trung bình của người nhiều tuổi thứ 4 \& thứ 5 là $11$. Tính tuổi của người nhiều tuổi nhất.
\end{baitoan}

\begin{baitoan}[\cite{TLCT_THCS_Toan_6_so_hoc}, VD4.1, p. 29]
	$30,17$ chia cho $a\in\mathbb{N},a\ne1$ đều dư $r$. Tìm $a,r$.
\end{baitoan}

\begin{baitoan}[\cite{TLCT_THCS_Toan_6_so_hoc}, VD4.2, p. 29]
	Có hơn $20$ học sinh xếp thành 1 vòng tròn. Khi đếm theo chiều kim đồng hồ, bắt đầu từ số $1$, thì 2 số $24,900$ rơi vào cùng 1 học sinh. Có ít nhất bao nhiêu học sinh?
\end{baitoan}

\begin{baitoan}[\cite{TLCT_THCS_Toan_6_so_hoc}, VD4.3, p. 29]
	Tìm $n\in\mathbb{N}^\star$ biết $\sum_{i=1}^n i = 1 + 2 + \cdots + n = 378$.
\end{baitoan}

\begin{baitoan}[\cite{TLCT_THCS_Toan_6_so_hoc}, VD4.4, p. 30]
	Cho tích $800$ số tự nhiên từ $1$ đến $800$: $A = \prod_{i=1}^{800} i = 1\cdot2\cdots800$. (a) Dạng phân tích của A ra thừa số nguyên tố chứa thừa số $5$ với số mũ bao nhiêu? (b) A tận cùng bằng bao nhiêu chữ số $0$?
\end{baitoan}

\begin{baitoan}[\cite{TLCT_THCS_Toan_6_so_hoc}, VD4.5, p. 31]
	Chứng minh chỉ có duy nhất 1 bộ 3 số nguyên tố mà hiệu của 2 số liên tiếp bằng $4$.
\end{baitoan}

\begin{baitoan}[\cite{TLCT_THCS_Toan_6_so_hoc}, VD4.6, p. 31]
	Tìm số tự nhiên nhỏ nhất có $8$ ước số.
\end{baitoan}

\begin{baitoan}[\cite{TLCT_THCS_Toan_6_so_hoc}, VD4.7, p. 31]
	Viết mỗi số sau thành 1 tổng của các hợp số sao cho số số hạng của tổng là nhiều nhất: $100,101,102,103$.
\end{baitoan}

\begin{baitoan}[\cite{TLCT_THCS_Toan_6_so_hoc}, 4.1., p. 32]
	Trong 1 tháng, có 3 ngày chủ nhật là 3 số nguyên tố. Ngày 15 của tháng đó là ngày thứ mấy?
\end{baitoan}

\begin{baitoan}[\cite{TLCT_THCS_Toan_6_so_hoc}, 4.2., p. 32]
	Viết liên tiếp các số tự nhiên từ $1$ đến $99$, ta được 1 số A. A là số nguyên tố hay hợp số?
\end{baitoan}

\begin{baitoan}[\cite{TLCT_THCS_Toan_6_so_hoc}, 4.3., p. 32]
	Xét tính chính phương: (a) $\sum_{i=1}^{20} 2^i = 2 + 2^2 + 2^3 + \cdots + 2^{20}$. (b) $10^{15} + 8$.
\end{baitoan}

\begin{baitoan}[\cite{TLCT_THCS_Toan_6_so_hoc}, 4.4., p. 32]
	Cho $p,p + 14$ là 2 số nguyên tố. Chứng minh $p + 7$ là hợp số.
\end{baitoan}

\begin{baitoan}[\cite{TLCT_THCS_Toan_6_so_hoc}, 4.5., p. 32]
	Cho $p,p + 20,p + 40$ là 3 số nguyên tố. Chứng minh $p + 80$ là số nguyên tố.
\end{baitoan}

\begin{baitoan}[\cite{TLCT_THCS_Toan_6_so_hoc}, 4.6., p. 32]
	Tìm số nguyên tố $p$ sao cho $p + 6,p + 12,p + 18,p + 24$ cũng là 4 số nguyên tố.
\end{baitoan}

\begin{baitoan}[\cite{TLCT_THCS_Toan_6_so_hoc}, 4.7., p. 32]
	Tìm số nguyên tố nhỏ hơn $200$, biết khi chia nó cho $60$ thì số dư là hợp số.
\end{baitoan}

\begin{baitoan}[\cite{TLCT_THCS_Toan_6_so_hoc}, 4.8., p. 32]
	Chứng minh số $\underbrace{1\ldots1}_{10}2\underbrace{1\ldots1}_{10}$ là hợp số.
\end{baitoan}

\begin{baitoan}[\cite{TLCT_THCS_Toan_6_so_hoc}, 4.9., p. 32]
	Tìm 3 số tự nhiên liên tiếp có tích bằng $13800$.
\end{baitoan}

\begin{baitoan}[\cite{TLCT_THCS_Toan_6_so_hoc}, 4.10., p. 32]
	Tìm $n\in\mathbb{N}$ biết: (a) $2n + 1\divby n - 3$. (b) $n^2 + 3\divby n + 1$.
\end{baitoan}

\begin{baitoan}[\cite{TLCT_THCS_Toan_6_so_hoc}, 4.11., p. 33]
	Tìm $n\in\mathbb{N}$ biết: (a) $\sum_{i=1}^n i = 1 + 2 + \cdots + n = 231$. (b) $\sum_{i=1}^n 2i - 1 = 1 + 3 + 5 + \cdots + (2n - 1) = 169$.
\end{baitoan}

\begin{baitoan}[\cite{TLCT_THCS_Toan_6_so_hoc}, 4.12., p. 33]
	Tìm 2 số tự nhiên không chia hết cho $10$ \& có tích bằng $10000$.
\end{baitoan}

\begin{baitoan}[\cite{TLCT_THCS_Toan_6_so_hoc}, 4.13., p. 33]
	Chi tính tổng các số tự nhiên liên tiếp từ $1$ đến $n$ \& nhận thấy tổng đó chia hết cho $29$. Hoàng tính tổng các số tự nhiên từ $1$ đến $m$ \& cũng nhận thấy tổng đó chia hết cho $29$. Tìm $m,n$ biết $m < n < 50$.
\end{baitoan}

\begin{baitoan}[\cite{TLCT_THCS_Toan_6_so_hoc}, 4.14., p. 33]
	Chứng minh tồn tại $99$ số tự nhiên liên tiếp đều là hợp số.
\end{baitoan}

\begin{baitoan}[\cite{TLCT_THCS_Toan_6_so_hoc}, 4.15., p. 33]
	Tìm số tự nhiên lớn nhất có 2 chữ số: (a) Có ít ước số nhất. (b) Có $12$ ước số.
\end{baitoan}

\begin{baitoan}[\cite{TLCT_THCS_Toan_6_so_hoc}, 4.16., p. 33]
	Tìm số tự nhiên nhỏ nhất: (a) Có $7$ ước số. (b) Có $15$ ước số.
\end{baitoan}

\begin{baitoan}[\cite{TLCT_THCS_Toan_6_so_hoc}, 4.17., p. 33]
	Tìm số tự nhiên nhỏ nhất chỉ chứa các thừa số nguyên tố $2,5$, biết khi chia nó cho $2$ thì được 1 số chính phương.
\end{baitoan}

\begin{baitoan}[\cite{TLCT_THCS_Toan_6_so_hoc}, 4.18., p. 33]
	Tìm số tự nhiên nhỏ nhất khác $0$ sao cho khi chia nó cho $2$ thì được 1 số chính phương, khi chia nó cho $3$ thì được lập phương của 1 số tự nhiên.
\end{baitoan}

\begin{baitoan}[\cite{TLCT_THCS_Toan_6_so_hoc}, 4.19., p. 33]
	$n\in\mathbb{N}$ chỉ chứa 2 thừa số nguyên tố. Biết $n^2$ có $21$ ước số. Tính số ước số của $n^3,n^4,n^k$, $\forall k\in\mathbb{N}$.
\end{baitoan}

%------------------------------------------------------------------------------%

\section{Divisor \& Multiple -- Ước \& Bội}

\begin{baitoan}[\cite{Huy_so_hoc}, VD11, p. 11]
	Tìm tất cả các phép chia có số bị chia là $1817$, đồng thời có thương \& dư giống nhau.
\end{baitoan}

\begin{baitoan}[\cite{Huy_so_hoc}, VD21, p. 11]
	Có bao nhiêu phép chia hết có số bị chia là $784$, đồng thời thương \& dư cùng là 1 số tự nhiên 2 chữ số?
\end{baitoan}

\begin{baitoan}[\cite{Binh_Toan_6_tap_1}, VD36, p. 33]
	Tìm số chia \& thương của 1 phép chia có số bị chia bằng $145$, số dư bằng $12$ biết thương khác $1$.
\end{baitoan}

\begin{baitoan}[\cite{Binh_Toan_6_tap_1}, VD37, p. 33]
	Tìm số tự nhiên có 2 chữ số khác nhau sao cho nếu xóa bất kỳ chữ số nào của nó thì số nhận được vẫn là ước của số ban đầu.
\end{baitoan}

\begin{baitoan}[\cite{Binh_Toan_6_tap_1}, VD38, p. 33]
	1 tổ sản xuất được thưởng $840$ nghìn đồng. Số tiền thưởng chia đều cho số người trong tổ. Sau khi chia xong, tổ phát hiện đã bỏ sót không chia cho 1 người vắng mặt, do đó mỗi người được chia đã góp $2$ nghìn đồng \& kết quả là người vắng mặt cũng được nhận số tiền như những người có mặt. Tính số tiền mỗi người đã được thưởng (số tiền đó là 1 số tự nhiên với đơn vị nghìn đồng).
\end{baitoan}

\begin{baitoan}[\cite{Binh_Toan_6_tap_1}, VD39, p. 33]
	Trong 1 buổi họp mặt của 2 câu lạc bộ A \& B, mỗi người bắt tay 1 lần với tất cả những người còn lại. Tính số người của mỗi câu lạc bộ, biết có tất cả $496$ cái bắt tay, trong đó có $241$ cái bắt tay của 2 người trong cùng 1 câu lạc bộ.
\end{baitoan}

\begin{baitoan}[\cite{Binh_Toan_6_tap_1}, VD40, p. 34]
	Tìm 5 số tự nhiên khác nhau, biết khi nhân từng cặp 2 số thì tích nhỏ nhất bằng $28$, tích lớn nhất bằng $240$ \& 1 tích khác bằng $128$.
\end{baitoan}

\begin{baitoan}[\cite{Binh_Toan_6_tap_1}, VD41, p. 34]
	Viết số $108$ dưới dạng tổng các số tự nhiên liên tiếp lớn hơn $0$.
\end{baitoan}

\begin{baitoan}[\cite{Binh_Toan_6_tap_1}, 200., p. 35]
	Tìm $x,y\in\mathbb{N}$ sao cho: (a) $(2x + 1)(y - 3) = 10$. (b) $(3x - 2)(2y - 3) = 1$. (c) $(x + 1)(2y - 1) = 12$. (d) $x + 6 = y(x - 1)$. (e) $x - 3 = y(x + 2)$.
\end{baitoan}

\begin{baitoan}[\cite{Binh_Toan_6_tap_1}, 201., p. 35]
	1 phép chia số tự nhiên có số bị chia bằng $3193$. Tìm số chia \& thương của phép chia đó, biết số chia có 2 chữ số.
\end{baitoan}

\begin{baitoan}[\cite{Binh_Toan_6_tap_1}, 202., p. 35]
	Tìm số chia của 1 phép chia, biết: Số bị chia bằng $236$, số dư bằng $15$, số chia là số tự nhiên có 2 chữ số.
\end{baitoan}	

\begin{baitoan}[\cite{Binh_Toan_6_tap_1}, 203., p. 35]
	Tìm ước của $161$ trong khoảng từ $10$ đến $150$.
\end{baitoan}

\begin{baitoan}[\cite{Binh_Toan_6_tap_1}, 204., p. 35]
	Tìm 2 số tự nhiên liên tiếp có tích bằng $600$.
\end{baitoan}

\begin{baitoan}[\cite{Binh_Toan_6_tap_1}, 205., p. 35]
	Tìm 3 số tự nhiên liên tiếp có tích bằng $2730$.
\end{baitoan}

\begin{baitoan}[\cite{Binh_Toan_6_tap_1}, 206., p. 35]
	Tìm 3 số lẻ liên tiếp có tích bằng $12075$.
\end{baitoan}

\begin{baitoan}[\cite{Binh_Toan_6_tap_1}, 207., p. 35]
	Có 1 số số tự nhiên khác nhau được viết trên bảng. Tích của 2 số nhỏ nhất là $16$, tích của 2 số lớn nhất là $225$. Tính tổng của tất cả các số tự nhiên đó.
\end{baitoan}

\begin{baitoan}[\cite{Binh_Toan_6_tap_1}, 208., p. 35]
	Trên 1 tấm bia có các vòng tròn tính điểm là $18,23,28,33,38$. Muốn trúng thưởng, phải bắn 1 số phát tên để đạt đúng $100$ điểm. Hỏi phải bắn bao nhiêu phát tên \& vào những vòng nào?
\end{baitoan}

\begin{baitoan}[\cite{Binh_Toan_6_tap_1}, 209., p. 35]
	1 tờ hóa đơn bị dây mực, chỗ dây mực biểu thị bởi dấu $\star$. Phục hồi lại các chữ số bị dây mực (dấu $\star$ thay cho 1 hay nhiều chữ số). Giá mua 1 hộp bút: $3200$ đồng. Số hộp bút đã bán: $\star$ chiếc. Giá bán 1 hộp bút: $\star00$ đồng. Thành tiền: $107300$ đồng.
\end{baitoan}

\begin{baitoan}[\cite{Binh_Toan_6_tap_1}, 210., p. 36]
	Tìm $n\in\mathbb{N}$, biết: $\sum_{i=1}^n i = 1 + 2 + 3 + \cdots + n = 820$.
\end{baitoan}

\begin{baitoan}[\cite{Binh_Toan_6_tap_1}, 211., p. 36]
	Viết số $100$ dưới dạng tổng các số lẻ liên tiếp.
\end{baitoan}

\begin{baitoan}[\cite{Binh_Toan_6_tap_1}, 212., p. 36]
	Tân \& Hùng gặp nhau trong hội nghị học sinh giỏi Toán. Tân hỏi số nhà Hùng, Hùng trả lời: - Nhà mình ở chính giữa phố, đoạn phố ấy có tổng các số nhà bằng $161$. Nghĩ 1 chút, Tâm nói: - Bạn ở số nhà $23$ chứ gì! Hỏi Tân đã tìm ra như thế nào?
\end{baitoan}

\begin{baitoan}[\cite{Binh_Toan_6_tap_1}, 213., p. 36]
	Tìm $a,b,c,d\in\{1,2,\ldots,99\}$ sao cho $b\divby a,b\divby c,c\divby a,b\divby d,c\divby d$ \& $a$ có {\rm GTLN} trong các giá trị $a$ có thể nhận được.
\end{baitoan}

\begin{baitoan}[\cite{Binh_Toan_6_tap_1}, 214., p. 36]
	Tìm $n\in\mathbb{N}$, sao cho: (a) $n + 4\divby n + 1$. (b) $n^2 + 4\divby n + 2$. (c) $13n\divby n - 1$.
\end{baitoan}

\begin{baitoan}[\cite{Binh_Toan_6_tap_1}, 215., p. 36]
	Tìm số tự nhiên có 3 chữ số, biết nó tăng gấp $n$ lần nếu cộng mỗi chữ số của nó với $n$ ($n\in\mathbb{N}$, có thể gồm 1 hoặc nhiều chữ số).
\end{baitoan}

\begin{baitoan}[\cite{Binh_Toan_6_tap_1}, 216., p. 36]
	2 công ty A \& B năm trước có số nhân viên bằng nhau. Năm sau, công ty A tuyển thêm số nhân viên mới bằng 4 lần số nhân viên cũ, còn công ty B cho nghỉ việc 5 nhân viên, do đó số nhân viên công ty A là bội của số nhân viên công ty B. Hỏi năm trước mỗi công ty có nhiều nhất bao nhiêu nhân viên?
\end{baitoan}

%------------------------------------------------------------------------------%

\subsection{Divisor. Common Divisor -- Ước. Ước Chung}
\cite[\S12, pp. 33--34]{SBT_Toan_6_Canh_Dieu_tap_1}: 109. 110. 111. 112. 113. 114. 115. 116. 117. 118.

\begin{baitoan}[\cite{Binh_Toan_6_tap_1}, VD42, p. 36]
	Tìm $a\in\mathbb{N}$ biết $264$ chia cho $a$ dư $24$, còn $363$ chia cho $a$ dư $43$.
\end{baitoan}

\begin{baitoan}[\cite{Binh_Toan_6_tap_1}, VD44, p. 37]
	Trên 1 hành tinh, các cư dân chia 1 ngày đêm thành $a$ giờ, chia 1 giờ thành $b$ phút, chia 1 phút thành $c$ giây ($a,b,c\in\mathbb{N}$). Biết 1 ngày đêm có $620$ phút, mỗi giờ có $899$ giây. Hỏi trên hành tinh đó, mỗi ngày đêm gồm bao nhiêu giây?
\end{baitoan}

\begin{baitoan}[\cite{Binh_Toan_6_tap_1}, 217., p. 37]
	Tìm $a\in\mathbb{N}$, biết $398$ chia cho $a$ thì dư $38$, còn $450$ chia cho $a$ thì dư $18$.
\end{baitoan}

\begin{baitoan}[\cite{Binh_Toan_6_tap_1}, 218., p. 37]
	Tìm $a\in\mathbb{N}$, biết $350$ chia cho $a$ thì dư $14$, còn $320$ chia cho $a$ thì dư $26$.
\end{baitoan}

\begin{baitoan}[\cite{Binh_Toan_6_tap_1}, 219., p. 37]
	Có $100$ quyển vở \& $90$ bút chì được thưởng đều cho 1 số học sinh, còn lại $4$ quyển vở \& $18$ bút chì không đủ chia đều. Tính số học sinh được thưởng.
\end{baitoan}

\begin{baitoan}[\cite{Binh_Toan_6_tap_1}, 220., p. 37]
	Phần thưởng cho học sinh của 1 lớp học gồm $128$ vở, $48$ bút chì, $192$ nhãn vở. Có thể chia được nhiều nhất thành bao nhiêu phần thưởng như nhau, mỗi phần thưởng gồm bao nhiêu vở, bút chì, nhãn vở?
\end{baitoan}

\begin{baitoan}[\cite{Binh_Toan_6_tap_1}, 221., p. 37]
	3 khối $6,7,8$ theo thứ tự có $300$ học sinh, $276$ học sinh, $252$ học sinh xếp hàng dọc để diễu hành sao cho số hàng dọc của mỗi khối như nhau. Có thể xếp nhiều nhất thành mấy hàng dọc để mỗi khối đều không có ai lẻ hàng? Khi đó ở mỗi khối có bao nhiêu hàng ngang?
\end{baitoan}

\begin{baitoan}[\cite{Binh_Toan_6_tap_1}, 222., p. 37]
	Người ta muốn chia $200$ bút bi, $240$ bút chì, $320$ tẩy thành 1 số phần thưởng như nhau. Hỏi có thể chia được nhiều nhất thành bao nhiêu phần thưởng, mỗi phần thưởng có bao nhiêu bút bi, bút chì, tẩy?
\end{baitoan}

\begin{baitoan}[\cite{Binh_Toan_6_tap_1}, 223., p. 38]
	Các số $1620$ \& $1410$ chia cho số tự nhiên $a$ có 3 chữ số cùng được số dư là $r$. Tìm $a$ \& $r$.
\end{baitoan}

\begin{baitoan}[\cite{Binh_Toan_6_tap_1}, 224., p. 38]
	Tìm số chia \& thương của 1 phép chia số tự nhiên có số bị chia bằng $9578$ \& các số dư liên tiếp là $5,3,2$.
\end{baitoan}

%------------------------------------------------------------------------------%

\subsection{Multiple. Common Multiple -- Bội. Bội Chung}

\begin{baitoan}[\cite{Binh_Toan_6_tap_1}, VD45, p. 38]
	Tìm số tự nhiên $a$ nhỏ nhất sao cho chia $a$ cho $3$, cho $5$, cho $7$ được số dư theo thứ tự là $2,3,4$.
\end{baitoan}

\begin{baitoan}[\cite{Binh_Toan_6_tap_1}, VD46, p. 38]
	1 số tự nhiên chia cho $3$ thì dư $1$, chia cho $4$ thì dư $2$, chia cho $5$ thì dư $3$, chia cho $6$ thì dư $4$, \& chia hết cho $13$. (a) Tìm số nhỏ nhất có tính chất trên. (b) Tìm dạng chung của tất cả các số có tính chất trên.
\end{baitoan}

\begin{baitoan}[\cite{Binh_Toan_6_tap_1}, VD47, p. 39]
	3 người mua 3 chiếc ô tô cùng loại với cùng 1 giá. Ông A đặt cọc $130$ triệu đồng, mỗi tháng trả $18$ triệu đồng thì trả xong. Ông B đặt cọc $100$ triệu đồng, mỗi tháng trả $24$ triệu đồng thì trả xong. Ông C đặt cọc $60$ triệu đồng, mỗi tháng trả $28$ triệu đồng thì trả xong. Tính giá mỗi chiếc ô tô, biết giá ô tô chưa đến $900$ triệu đồng.
\end{baitoan}

\begin{baitoan}[\cite{Binh_Toan_6_tap_1}, 225., p. 39]
	Tìm các bội chung của $40,60,126$ \& nhỏ hơn $6000$.
\end{baitoan}

\begin{baitoan}[\cite{Binh_Toan_6_tap_1}, 226., p. 39]
	1 cuộc thi chạy tiếp sức theo vòng tròn gồm nhiều chặng. Biết chu vi đường tròn là $330$\emph{m}, mỗi chặng dài $75$\emph{m}, địa điểm xuất phát \& kết thúc cùng 1 chỗ. Hỏi cuộc thi có ít nhất mấy chặng?
\end{baitoan}

\begin{baitoan}[\cite{Binh_Toan_6_tap_1}, 227., p. 39]
	3 ô tô cùng khởi hành 1 lúc từ 1 bến. Thời gian cả đi lẫn về của xe thứ nhất là $40$ phút, của xe thứ 2 là $50$ phút, của xe thứ 3 là $30$ phút. Khi trở về bến, mỗi xe đều nghỉ $10$ phút rồi tiếp tục chạy. Hỏi sau ít nhất bao lâu: (a) Xe thứ nhất \& xe thứ 2 cùng rời bến? (b) Xe thứ 2 \& xe thứ 3 cùng rời bến? (c) Cả 3 xe cùng rời bến?
\end{baitoan}

\begin{baitoan}[\cite{Binh_Toan_6_tap_1}, 228., p. 39]
	1 đơn vị bộ đội khi xếp hàng $20,25,30$ đều dư $15$, nhưng xếp hàng $41$ thì vừa đủ. Tính số người của đơn vị đó biết số người chưa đến $1000$.
\end{baitoan}

\begin{baitoan}[\cite{Binh_Toan_6_tap_1}, 229., p. 39]
	1 chiếc xe đạp xiếc có chu vi bánh xe lớn $21$\emph{dm}, chu vi bánh xe nhỏ $9$\emph{dm}. Hiện nay van của 2 bánh xe đều ở vị trí thấp nhất. Hỏi xe phải lăn bao nhiêu mét nữa thì 2 van của 2 bánh xe lại ở vị trí thấp nhất?
\end{baitoan}

\begin{baitoan}[\cite{Binh_Toan_6_tap_1}, 230., p. 39]
	Tìm $n\in\mathbb{N}$ có 3 chữ số sao cho $n + 6\divby7,n + 7\divby8,n + 8\divby9$.
\end{baitoan}

\begin{baitoan}[\cite{Binh_Toan_6_tap_1}, 231., p. 39]
	Tìm số tự nhiên có 3 chữ số, sao cho chia nó cho $17$, cho $25$ được các số dư theo thứ tự là $8$ \& $16$.
\end{baitoan}

\begin{baitoan}[\cite{Binh_Toan_6_tap_1}, 232., p. 39]
	Tìm $n\in\mathbb{N}$ lớn nhất có 3 chữ số, sao cho $n$ chia cho $8$ thì dư $7$, chia cho $31$ thì dư $28$.
\end{baitoan}

\begin{baitoan}[\cite{Binh_Toan_6_tap_1}, 233., p. 40]
	Tìm số tự nhiên nhỏ hơn $500$, sao cho chia nó cho $15$, cho $35$ được các số dư theo thứ tự là $8$ \& $13$.
\end{baitoan}

\begin{baitoan}[\cite{Binh_Toan_6_tap_1}, 234., p. 40]
	(a) Tìm số tự nhiên lớn nhất có 3 chữ số, sao cho chia nó cho $2$, cho $3$, cho $4$, cho $5$, cho $6$ ta được các số dư theo thứ tự là $1,2,3,4,5$. (b) Tìm dạng chung của các số tự nhiên $a$ chia cho $4$ dư $3$, chia cho $5$ thì dư $4$, chia cho $6$ thì dư $5$, chia hết cho $13$.
\end{baitoan}

\begin{baitoan}[\cite{Binh_Toan_6_tap_1}, 235., p. 40]
	Tìm số tự nhiên nhỏ nhất chia cho $8$ dư $6$, chia cho $12$ dư $10$, chia cho $15$ dư $13$ \& chia hết cho $13$.
\end{baitoan}

\begin{baitoan}[\cite{Binh_Toan_6_tap_1}, 236., p. 40]
	Tìm số tự nhiên nhỏ nhất chia cho $8,10,15,20$ theo thứ tự dư $5,7,12,17$ \& chia hết cho $41$.
\end{baitoan}

\begin{baitoan}[\cite{Binh_Toan_6_tap_1}, 237., p. 40]
	Chị Mai xếp bánh (ít hơn $100$ chiếc) vào các đĩa. Nếu mỗi đĩa xếp $8$ bánh thì có $1$ đĩa chỉ có $3$ chiếc bánh. Nếu mỗi đĩa xếp $7$ chiếc bánh thì có $1$ đĩa chỉ có $5$ chiếc bánh. Nếu mỗi đĩa xếp $3$ chiếc bánh thì có $1$ đĩa chỉ có $1$ chiếc bánh. Tìm số bánh.
\end{baitoan}

\begin{baitoan}[\cite{Binh_Toan_6_tap_1}, 238., p. 40]
	7 người có 7 mảnh đất diện tích bằng nhau. Người thứ nhất trồng 1 cây cam. Người thứ 2 trồng $2$ cây cam. Người thứ 3 trồng $3$ cây cam. $\ldots$ Người thứ 7 trồng $7$ cây cam. Điều đặc biệt là ai cũng thấy các cây cam của mình có số quả bằng nhau. Ngoài ra số cam của mỗi người không chênh lệch nhiều nên sau khi người thứ 7 cho người thứ 2,3,4,5,6 mỗi người $1$ quả cam thì cả 7 người đều có số cam bằng nhau. Tính số cam trên cây của mỗi người lúc đầu, biết không có cây cam nào có hơn $200$ quả.
\end{baitoan}

\begin{baitoan}[\cite{Binh_Toan_6_tap_1}, 239., p. 40]
	Tìm số tự nhiên nhỏ nhất chia cho $5$, cho $7$, cho $9$ có số dư theo thứ tự là $3,4,5$.
\end{baitoan}

\begin{baitoan}[\cite{Binh_Toan_6_tap_1}, 240., p. 40]
	Tìm số tự nhiên nhỏ nhất chia cho $3$, cho $4$, cho $5$ có số dư theo thứ tự là $1,3,1$.
\end{baitoan}

\begin{baitoan}[\cite{Binh_Toan_6_tap_1}, 241., p. 40]
	Trên đoạn đường dài $4800$\emph{m} có các cột điện trồng cách nhau $60$\emph{m}, nay trồn lại cách nhau $80$\emph{m}. Hỏi có bao nhiêu cột không phải trồng lại, biết ở cả 2 đầu đoạn đường đều có cột điện?
\end{baitoan}

\begin{baitoan}[\cite{Binh_Toan_6_tap_1}, 242., p. 40]
	3 con tàu cập bến theo lịch như sau: Tàu I cứ $15$ ngày thì cập bến, tàu II cứ $20$ ngày thì cập bến, tàu III cứ $12$ ngày thì cập bến. Lần đầu cả 3 tàu cùng cập bến vào ngày thứ 6. Hỏi sau đó ít nhất bao lâu, cả 3 tàu lại cùng cập bến vào ngày thứ 6?
\end{baitoan}

\begin{baitoan}[\cite{Binh_Toan_6_tap_1}, 243., p. 40]
	Nếu xếp 1 số sách vào từng túi $10$ cuốn thì vừa hết, vào từng túi $12$ cuốn thì thừa $2$ cuốn, vào từng túi $18$ cuốn thì thừa $8$ cuốn. Biết số sách trong khoảng từ $715$ đến $1000$, tính số sách đó.
\end{baitoan}

\begin{baitoan}[\cite{Binh_Toan_6_tap_1}, 244., p. 40]
	2 lớp 6A, 6B cùng thu nhặt 1 số giấy vụn bằng nhau. Trong lớp 6A, 1 bạn thu được $26$\emph{kg}, còn lại mỗi bạn thu $11$\emph{kg}. Trong lớp 6B, 1 bạn thu được $25$\emph{kg}, còn lại mỗi bạn thu $10$\emph{kg}. Tính số học sinh mỗi lớp, biết số giấy mỗi lớp thu được trong khoảng từ $200$\emph{kg} đến $300$\emph{kg}.
\end{baitoan}

\begin{baitoan}[\cite{Binh_Toan_6_tap_1}, 245., p. 41]
	1 thiết bị điện tử phát ra tiếng kêu ``bíp' sau mỗi $60$ giây, 1 thiết bị điện tử khác phát ra tiếng kêu ``bíp'' sau mỗi $62$ giây. Cả 2 thiết bị này đều phát ra tiếng ``bíp'' lúc $10:00$. Tính thời điểm để cả 2 cùng phát ra tiếng ``bíp'' tiếp theo.
\end{baitoan}

\begin{baitoan}[\cite{Binh_Toan_6_tap_1}, 246., p. 41]
	Có 2 chiếc đồng hồ (có kim giờ \& kim phút). Trong 1 ngày, chiếc thứ nhất chạy nhanh $2$ phút, chiếc thứ 2 chạy chậm $3$ phút. Cả 2 đồng hồ được lấy lại theo giờ chính xác. Hỏi sau ít nhất bao nhiêu lâu, cả 2 đồng hồ lại cùng chỉ giờ chính xác?
\end{baitoan}

%------------------------------------------------------------------------------%

\subsection{Greatest Common Divisor GCD. Least Common Multiple LCM -- Ước Chung Lớn Nhất. Bội Chung Nhỏ Nhất}
\cite[\S13, pp. 36--37]{SBT_Toan_6_Canh_Dieu_tap_1}: 119. 120. 121. 122. 123. 124. 125. 126. 127.

\begin{baitoan}[\cite{Binh_boi_duong_Toan_6_tap_1}, H1, p. 43]
	1 thửa ruộng hình chữ nhật có chiều dài {\rm72 m}, chiều rộng {\rm40 m}. Chicken muốn chia thửa ruộng thành các mảnh đất hình vuông bằng nhau để trồng các loại ngũ cốc. Tính độ dài lớn nhất của hình vuông mà Chicken có thể chia.
\end{baitoan}

\begin{baitoan}[\cite{Binh_boi_duong_Toan_6_tap_1}, H2, p. 43]
	Có 4 thuyền A, B, C, D. Thuyền A cứ $5$ ngày cập bến 1 lần, thuyền B cứ $6$ ngày cập bến  1 lần, thuyền C cứ $8$ ngày cập bến 1 lần \& thuyền D cứ $10$ ngày cập bến 1 lần. Egg nhẩm tính: Nếu ngày hôm nay cả 4 thuyền cùng cập bến thì: (a) Sau ít nhất $a$ ngày nữa, thuyền A cùng cập bến với thuyền D. (b) Sau ít nhất $b$ ngày nữa, thuyền B cùng cập bến với thuyền C. (c) Sau ít nhất $c$ ngày nữa, thuyền B cùng cập bến với thuyền D. (d) Sau ít nhất $d$ ngày nữa, cả 4 thuyền sẽ cùng cập bến lần thứ 2. Tìm $a,b,c,d$.
\end{baitoan}

\begin{baitoan}[\cite{Binh_boi_duong_Toan_6_tap_1}, VD1, p. 43]
	Tìm $\mbox{\rm ƯC}(48,60)$, ${\rm BC}(4,14)$.
\end{baitoan}

\begin{baitoan}[\cite{Binh_boi_duong_Toan_6_tap_1}, VD2, p. 44]
	Tìm $a\in\mathbb{N}$ biết chia $264$ cho $a$ thì dư $24$, còn khi chia $363$ cho $a$ thì được dư là $43$.
\end{baitoan}

\begin{baitoan}[\cite{Binh_boi_duong_Toan_6_tap_1}, VD3, p. 44]
	Tìm số tự nhiên nhỏ nhất có 4 chữ số biết khi chia số đó cho $18,24,30$ thì có số dư lần lượt là $13,19,25$.
\end{baitoan}

\begin{baitoan}[\cite{Binh_boi_duong_Toan_6_tap_1}, VD4, p. 44]
	Tìm $a,b\in\mathbb{N}$ thỏa $a + b = 336$ \& $\mbox{\rm ƯCLN}(a,b) = 24$.
\end{baitoan}

\begin{baitoan}[\cite{Binh_boi_duong_Toan_6_tap_1}, VD5, p. 45]
	Tìm $a,b\in\mathbb{N}$ thỏa $\mbox{\rm ƯCLN}(a,b) = 6$ \& ${\rm BCNN}(a,b) = 36$.
\end{baitoan}

\begin{baitoan}[\cite{Binh_boi_duong_Toan_6_tap_1}, VD6, p. 45]
	Cho $n\in\mathbb{N}^\star$. Chứng minh: $\mbox{\rm ƯCLN}(2n + 5,3n + 7) = 1$.
\end{baitoan}

\begin{baitoan}[\cite{Binh_boi_duong_Toan_6_tap_1}, VD7, p. 46]
	Học sinh khối 6 của 1 trường khi xếp hàng $12$, hàng $15$ hay hàng $18$ thì đều vừa đủ hàng. Tính số học sinh khối 6 của trường đó biết số học sinh này nằm trong khoảng từ $500$ đến $600$ học sinh.
\end{baitoan}

\begin{baitoan}[\cite{Binh_boi_duong_Toan_6_tap_1}, VD8, p. 46]
	1 lớp học có $28$ học sinh nam \& $24$ học sinh nữ. Khi tham gia lao động, {\rm GVCN} muốn chia lớp thành các nhóm sao cho số học sinh nam \& số học sinh nữ được chia đều vào các nhóm. Hỏi {\rm GVCN} có bao nhiêu cách chia nhóm? Cách chia nào có số học sinh trong mỗi nhóm ít nhất?
\end{baitoan}

\begin{baitoan}[\cite{Binh_boi_duong_Toan_6_tap_1}, 6.1., p. 47]
	Tìm $\mbox{\rm ƯC}(54,120,180)$, ${\rm BC}(21,84)$.
\end{baitoan}

\begin{baitoan}[\cite{Binh_boi_duong_Toan_6_tap_1}, 6.2., p. 47]
	1 số chia cho $21$ dư $2$ \& chia cho $12$ dư $5$. Hỏi số đó chia cho $84$ thì dư bao nhiêu?
\end{baitoan}

\begin{baitoan}[\cite{Binh_boi_duong_Toan_6_tap_1}, 6.3., p. 47]
	Tìm $a\in\mathbb{N}$ thỏa mãn: $a\divby7$ \& $a$ chia cho $4$ hoặc $6$ đều dư $3$ biết $a < 350$.
\end{baitoan}

\begin{baitoan}[\cite{Binh_boi_duong_Toan_6_tap_1}, 6.4., p. 47]
	Tìm số tự nhiên lớn nhất có 3 chữ số sao cho chia nó cho $3$, cho $4$, cho $5$ ta được 3 số dư theo thứ tự là $2,3,4$.
\end{baitoan}

\begin{baitoan}[\cite{Binh_boi_duong_Toan_6_tap_1}, 6.5., p. 47]
	Cho $\mbox{\rm ƯCLN}(a,b) = 1$. Chứng minh: (a) $\mbox{\rm ƯCLN}(a,a - b) = 1$ với $a > b$. (b) $\mbox{\rm ƯCLN}(ab,a + b) = 1$.
\end{baitoan}

\begin{baitoan}[\cite{Binh_boi_duong_Toan_6_tap_1}, 6.6., p. 47]
	Cho $n\in\mathbb{N}$. Chứng minh: (a) $\mbox{\rm ƯCLN}(3n + 13,3n + 14) = 1$. (b) $\mbox{\rm ƯCLN}(3n +5,6n + 9) = 1$.
\end{baitoan}

\begin{baitoan}[\cite{Binh_boi_duong_Toan_6_tap_1}, 6.7., p. 47]
	1 lớp học có $27$ học sinh nam \& $18$ học sinh nữ. Có bao nhiêu cách chia lớp đó thành các tổ sao cho số học sinh nam \& số học sinh nữ được chia đều vào các tổ? Biết số tổ lớn hơn $1$.
\end{baitoan}

\begin{baitoan}[\cite{Binh_boi_duong_Toan_6_tap_1}, 6.8., p. 47]
	1 đơn vị bộ đội khi xếp hàng, mỗi hàng có $20$ người, hoặc $25$ người, hoặc $30$ người đều thừa $15$ người. Nếu xếp mỗi hàng $41$ người thì vừa đủ (không có hàng nào thiếu, không có ai ở ngoài hàng). Hỏi đơn vị có bao nhiêu người biết số người của đơn vị chưa đến $1000$?
\end{baitoan}

\begin{baitoan}[\cite{Binh_boi_duong_Toan_6_tap_1}, 6.9., p. 47]
	Tổng số học sinh khối 6 của 1 trường có khoảng từ $235$ đến $250$ em, khi chia cho $3$ thì dư $2$, chia cho $4$ thì dư $3$, chia cho $5$ thì dư $4$, chia cho $6$ thì dư $5$, chia cho $10$ thì dư $9$. Tìm số học sinh của khối 6.
\end{baitoan}

\begin{baitoan}[\cite{Binh_boi_duong_Toan_6_tap_1}, 6.10., p. 47]
	1 trường tổ chức cho học sinh đi tham quan bằng ôtô. Nếu xếp $27$ hay $36$ học sinh lên 1 ôtô thì đều thấy thừa ra $11$ học sinh. Tính số học sinh đi tham quan biết số học sinh đó có khoảng từ $400$ đến $450$ em.
\end{baitoan}

\begin{baitoan}[\cite{Binh_boi_duong_Toan_6_tap_1}, 6.11., p. 47]
	Cho 2 số nguyên tố cùng nhau $a,b$. Chứng minh 2 số $13a + 4b$ \& $15a + 7b$ hoặc nguyên tố cùng nhau hoặc có 1 ước chung là $31$.
\end{baitoan}

\begin{baitoan}[\cite{Binh_boi_duong_Toan_6_tap_1}, 6.12., p. 47]
	Cho $a,b\in\mathbb{N}$ không nguyên tố cùng nhau thỏa $a = 2n + 3$, $b = 3n + 1$ với $n\in\mathbb{N}$. Tìm $\mbox{\rm ƯCLN}(a,b)$.
\end{baitoan}

\begin{baitoan}[\cite{Binh_boi_duong_Toan_6_tap_1}, 6.13., p. 47]
	Tìm $a,b\in\mathbb{N}$ biết: (a) $5a = 13b$ \& $\mbox{\rm ƯCLN}(a,b) = 48$. (b) ${\rm BCNN}(a,b) = 360$ \& $ab = 6480$. (c) $a + b = 40$ \& ${\rm BCNN}(a,b) = 7\mbox{\rm ƯCLN}(a,b)$.
\end{baitoan}

\begin{baitoan}[\cite{Binh_boi_duong_Toan_6_tap_1}, 6.14., p. 47]
	Tìm $a,b\in\mathbb{N}$ biết $a + 2b = 48$ \& $\mbox{\rm ƯCLN}(a,b) + 3{\rm BCNN}(a,b) = 114$.
\end{baitoan}

\begin{baitoan}[\cite{Binh_boi_duong_Toan_6_tap_1}, 6.15., p. 47]
	Tìm $a,b\in\mathbb{N}$ biết $\mbox{\rm ƯCLN}(a,b) + {\rm BCNN}(a,b) = 21$.
\end{baitoan}

\begin{baitoan}[\cite{Binh_boi_duong_Toan_6_tap_1}, 6.16., p. 47]
	Cho $a = 123456789$ \& $b = 987654321$. Tìm $\mbox{\rm ƯCLN}(a,b)$.
\end{baitoan}

\begin{baitoan}[\cite{Binh_boi_duong_Toan_6_tap_1}, 6.17., p. 47, Thừa Thiên--Huế 2007]
	Tìm 4 chữ số $a,b,c,d$ sao cho 4 số $a,\overline{ad},\overline{cd},\overline{abcd}$ là 4 số chính phương.
\end{baitoan}

\begin{baitoan}[\cite{Binh_boi_duong_Toan_6_tap_1}, p. 48, 1 số tính chất mở rộng về ƯCLN \& BCNN]
	Chứng minh: (a) $\mbox{\rm ƯCLN}(a,b,c) = \mbox{\rm ƯCLN}(\mbox{\rm ƯCLN}(a,b),c)$. (b) ${\rm BCNN}(a,b,c) = {\rm BCNN}({\rm BCNN}(a,b),c)$. (c) $\mbox{\rm ƯCLN}(ma,mb) = m\mbox{\rm ƯCLN}(a,b)$. (d) ${\rm BCNN}(ma,mb) = m{\rm BCNN}(a,b)$. (e) $\mbox{\rm ƯCLN}(a + kb,b) = \mbox{\rm ƯCLN}(a,b)$. (f) Nếu $ab\divby m$ \& $\mbox{\rm ƯCLN}(b,m) = 1$ thì $a\divby m$. (g) Nếu $a\divby m$ \& $a\divby n$ thì $a\divby{\rm BCNN}(m,n)$. (h) Nếu $a\divby m$, $a\divby n$, \& $\mbox{\rm ƯCLN}(m,n) = 1$ thì $a\divby mn$.
\end{baitoan}

\begin{baitoan}[\cite{Tuyen_Toan_6}, VD29, p. 27]
	Tìm $b\in\mathbb{N}$ biết khia chia $326$ cho $b$ thì dư $11$ còn chia $533$ cho $b$ thì dư $13$.
\end{baitoan}

\begin{baitoan}[\cite{Tuyen_Toan_6}, VD30, p. 28]
	Chứng minh 2 số tự nhiên liên tiếp là 2 số nguyên tố cùng nhau.
\end{baitoan}

\begin{baitoan}[\cite{Tuyen_Toan_6}, 129., p. 28]
	Tìm {\rm ƯCLN, ƯC} của 3 số $432,504,720$.
\end{baitoan}

\begin{baitoan}[\cite{Tuyen_Toan_6}, 130., p. 28]
	Tìm $x\in\mathbb{N}$ lớn nhất sao cho $x + 495,195 - x$ đều chia hết cho $x$.
\end{baitoan}

\begin{baitoan}[\cite{Tuyen_Toan_6}, 131., p. 28]
	1 căn phòng hình chữ nhật có kích thước $630\times480$ {\rm cm} được lát loại gạch hình vuông. Muốn cho 2 hàng gạch cuối cùng sát với 2 bức tường liên tiếp không bị cắt xén thì kích thước lớn nhất của viên gạch là bao nhiêu? Với loại gạch này thì cần bao nhiêu viên gạch để lát cả căn phòng?
\end{baitoan}

\begin{baitoan}[\cite{Tuyen_Toan_6}, 132., p. 28]
	2 lớp 6A, 6B cùng tham gia góp sách truyện để xây dựng thư viện. Mỗi học sinh góp số quyển sách như nhau. Tổng kết lại, lớp 6A góp được $36$ quyển, lớp 6B góp được $39$ quyển. Hỏi mỗi lớp có bao nhiêu bạn góp sách xây dựng thư viện?
\end{baitoan}

\begin{baitoan}[\cite{Tuyen_Toan_6}, 133., p. 28]
	Chứng minh các số sau nguyên tố cùng nhau: (a) 2 số lẻ liên tiếp. (b) $2n + 5,3n + 7$, với $n\in\mathbb{N}$.
\end{baitoan}

\begin{baitoan}[\cite{Tuyen_Toan_6}, 134., p. 28]
	Cho $(a,b) = 1$. Chứng minh $(a,a - b) = 1$. (b) $(ab,a + b) = 1$.
\end{baitoan}

\begin{baitoan}[\cite{Tuyen_Toan_6}, 135., p. 28]
	Cho $a,b$ là 2 số tự nhiên không nguyên tố cùng nhau, $a = 4n + 3,b = 5n + 1$, $n\in\mathbb{N}$. Tìm $(a,b)$.
\end{baitoan}

\begin{baitoan}[\cite{Tuyen_Toan_6}, 136., p. 28]
	{\rm ƯCLN} của 2 số là $45$. Số lớn là $270$. Tìm số nhỏ.
\end{baitoan}

\begin{baitoan}[\cite{Tuyen_Toan_6}, 137., p. 28]
	Tìm 2 số biết tổng của chúng là $162$ \& {\rm ƯCLN} của chúng là $18$.
\end{baitoan}

\begin{baitoan}[\cite{Tuyen_Toan_6}, 138., p. 28]
	Tìm 2 số tự nhiên nhỏ hơn $200$ biết hiệu của chúng là $90$ \& {\rm ƯCLN} của chúng là $15$.
\end{baitoan}

\begin{baitoan}[\cite{Tuyen_Toan_6}, 139., p. 28]
	Tìm 2 số biết tích của chúng là $8748$ \& {\rm ƯCLN} của chúng là $27$.
\end{baitoan}

\begin{baitoan}[\cite{Tuyen_Toan_6}, 140., p. 28]
	$a\in\mathbb{N}$ \& $5$ lần số $a$ có tổng các chữ số như nhau. Chứng minh $a\divby9$.
\end{baitoan}

\begin{baitoan}[\cite{Tuyen_Toan_6}, 141., p. 28]
	Có $64$ người đi tham quan bằng 2 loại xe: loại $12$ chỗ ngồi \& loại $7$ chỗ ngồi (không kể lái xe). Biết số người đi vừa đủ số ghế ngồi, tính số xe mỗi loại.
\end{baitoan}

\begin{baitoan}[\cite{Tuyen_Toan_6}, VD31, p. 29]
	Tìm số tự nhiên nhỏ nhất có $3$ chữ số chia cho $18,30,45$ có số dư lần lượt là $8,20,35$.
\end{baitoan}

\begin{baitoan}[\cite{Tuyen_Toan_6}, VD32, p. 30]
	Trong tiết học thể dục hôm nay, thầy giáo cho các bạn trong lớp xếp hàng $4$, hàng $6$, hàng $9$, mỗi lần đều thấy thừa $2$ học sinh. Thầy không cần biết mỗi lần có bao nhiêu hàng, thầy nói ngay số học sinh có mặt là $38$. Giải thích cách tính nhẩm của thầy.
\end{baitoan}

\begin{baitoan}[\cite{Tuyen_Toan_6}, 142., p. 30]
	1 xe lăn dành cho người khuyết tật có chu vi bánh trước là {\rm63 cm}, chu vi bánh sau là {\rm186 cm}. Người ta đánh dấu 2 điểm tiếp đất của 2 bánh xe này. Hỏi mỗi bánh xe phải lăn ít nhất bao nhiêu vòng thì 2 điểm đã đánh dấu lại cùng tiếp đất 1 lúc?
\end{baitoan}

\begin{baitoan}[\cite{Tuyen_Toan_6}, 143., p. 30]
	3 học sinh mỗi người mua 1 loại bút. Giá tiền 3 loại lần lượt là $4800$ đồng, $6000$ đồng, $8000$ đồng. Biết số tiền phải trả như nhau, hỏi mỗi học sinh mua ít nhất bao nhiêu bút?
\end{baitoan}

\begin{baitoan}[\cite{Tuyen_Toan_6}, 144., p. 30]
	Tìm các bội chung lớn hơn $5000$ nhưng nhỏ hơn $10000$ của 3 số $126,140,180$.
\end{baitoan}

\begin{baitoan}[\cite{Tuyen_Toan_6}, 145., p. 30]
	1 số tự nhiên chia cho $12,18,21$ đều dư $5$. Tìm số đó biết nó xấp xỉ $1000$.
\end{baitoan}

\begin{baitoan}[\cite{Tuyen_Toan_6}, 146., p. 30]
	Khối 6 của 1 trường có chưa tới $400$ học sinh. Khi xếp hàng $10,12,15$ đều dư $3$ nhưng nếu xếp hàng $11$ thì không dư. Tính số học sinh khối 6.
\end{baitoan}

\begin{baitoan}[\cite{Tuyen_Toan_6}, 147., p. 30]
	Tìm $a,b\in\mathbb{N}$ biết ${\rm BCNN}(a,b) = 300,\mbox{\rm ƯCLN}(a,b) = 15$.
\end{baitoan}

\begin{baitoan}[\cite{Tuyen_Toan_6}, 148., p. 30]
	Tìm $a,b\in\mathbb{N}$ biết tích của chúng là $2940$ \& {\rm BCNN} của chúng là $210$.
\end{baitoan}

\begin{baitoan}[\cite{Tuyen_Toan_6}, 149., p. 30]
	Tìm $a,b\in\mathbb{N}$ biết tổng của {\rm BCNN} với {\rm ƯCLN} của chúng là $15$.
\end{baitoan}

\begin{baitoan}[\cite{Tuyen_Toan_6}, 150., p. 30]
	Tìm $a\in\mathbb{N}$ nhỏ nhất có 3 chữ số sao cho chia cho $11$ thì dư $5$, chia cho $13$ thì dư $8$.
\end{baitoan}

\begin{baitoan}[\cite{Tuyen_Toan_6}, 151., p. 30]
	Tìm $x\in\mathbb{N}$ nhỏ nhất sao cho $x$ chia cho $3$ dư $2$, $x$ chia cho $5$ dư $3$, $x$ chia cho $7$ dư $4$.
\end{baitoan}

\begin{baitoan}[\cite{Tuyen_Toan_6}, 152., p. 30]
	Chứng minh nếu $a$ là 1 số lẻ không chia hết cho $3$ thì $a^2 - 1\divby6$.
\end{baitoan}

\begin{baitoan}[\cite{Tuyen_Toan_6}, 153., p. 30]
	Chứng minh tích của $5$ số tự nhiên liên tiếp chia hết cho $120$.
\end{baitoan}

\begin{baitoan}[\cite{TLCT_THCS_Toan_6_so_hoc}, VD5.1, p. 35]
	Tìm $n\in\mathbb{N}$ lớn nhất sao cho khi chia $364,414,539$ cho $n$, ta được 3 số dư bằng nhau.
\end{baitoan}

\begin{baitoan}[\cite{TLCT_THCS_Toan_6_so_hoc}, VD5.2, p. 36]
	Tìm $n\in\mathbb{N},n < 30$ để 2 số $3n + 4,5n + 1$ có ước chung khác $1$.
\end{baitoan}

\begin{baitoan}[\cite{TLCT_THCS_Toan_6_so_hoc}, VD5.3, p. 36]
	Tổng của 5 số tự nhiên bằng $156$. {\rm ƯCLN} của chúng có thể nhận {\rm GTLN} bằng bao nhiêu?
\end{baitoan}

\begin{baitoan}[\cite{TLCT_THCS_Toan_6_so_hoc}, VD5.4, p. 36]
	Có 3 đèn tín hiệu, chúng phát sáng cùng lúc vào {\rm8:00}. Đèn thứ nhất có $4$ phút phát sáng 1 lần. Thời gian đầu tiên để cả 3 đèn cùng phát sáng sau {\rm12:00} là lúc mấy giờ?
\end{baitoan}

\begin{baitoan}[\cite{TLCT_THCS_Toan_6_so_hoc}, VD5.5, p. 37]
	Điền các chữ số thích hợp vào dấu $\star$ để số $A = \overline{679\star\star\star}$ chia hết cho tất cả 4 số $5,6,7,9$.
\end{baitoan}

\begin{baitoan}[\cite{TLCT_THCS_Toan_6_so_hoc}, VD5.6, p. 37]
	Tìm $a,b\in\mathbb{N}$ thỏa $\mbox{\rm ƯCLN}(a,b) = 12,{\rm BCNN}(a,b) = 240$.
\end{baitoan}

\begin{baitoan}[\cite{TLCT_THCS_Toan_6_so_hoc}, 5.1., p. 38]
	Tìm ước chung của 3 số $1820,3080,4900$ trong khoảng từ $40$ đến $100$.
\end{baitoan}

\begin{baitoan}[\cite{TLCT_THCS_Toan_6_so_hoc}, 5.2., p. 38]
	Tìm {\rm ƯCLN} của $121212,181818$.
\end{baitoan}

\begin{baitoan}[\cite{TLCT_THCS_Toan_6_so_hoc}, 5.3., p. 38]
	Tìm {\rm ƯCLN} bằng cách dùng thuật toán Euclide: (a) $11111,1111$. (b) $342,266$.
\end{baitoan}

\begin{baitoan}[\cite{TLCT_THCS_Toan_6_so_hoc}, 5.4., p. 38]
	Tìm $a\in\mathbb{N}$ biết $296$ chia cho $a$ thì dư $16$, còn $230$ chia cho $a$ thì dư $10$.
\end{baitoan}

\begin{baitoan}[\cite{TLCT_THCS_Toan_6_so_hoc}, 5.5., p. 38]
	Chứng minh cặp số nguyên tố cùng nhau $\forall n\in\mathbb{N}$: (a) $2n + 1,6n + 5$. (b) $3n + 2,5n + 3$.
\end{baitoan}

\begin{baitoan}[\cite{TLCT_THCS_Toan_6_so_hoc}, 5.6., p. 38]
	Tìm $n\in\mathbb{N}$ để $3n + 1\divby7$.
\end{baitoan}

\begin{baitoan}[\cite{TLCT_THCS_Toan_6_so_hoc}, 5.7., p. 38]
	Tìm $n\in\mathbb{N}$ để $2n + 1,7n + 2$ nguyên tố cùng nhau.
\end{baitoan}

\begin{baitoan}[\cite{TLCT_THCS_Toan_6_so_hoc}, 5.8., p. 38]
	Tìm $a,b\in\mathbb{N}$ thỏa: (a) $a + b = 96,\mbox{\rm ƯCLN}(a,b) = 12$. (b) $a + b = 72,\mbox{\rm ƯCLN}(a,b) = 8$.
\end{baitoan}

\begin{baitoan}[\cite{TLCT_THCS_Toan_6_so_hoc}, 5.9., p. 38]
	Tìm $a,b\in\mathbb{N},a,b < 200$ thỏa: (a) $a - b = 96,\mbox{\rm ƯCLN}(a,b) = 16$. (b) $a - b = 90,\mbox{\rm ƯCLN}(a,b) = 15$.
\end{baitoan}

\begin{baitoan}[\cite{TLCT_THCS_Toan_6_so_hoc}, 5.10., p. 38]
	Tìm $a,b\in\mathbb{N}$ thỏa: (a) $ab = 448,\mbox{\rm ƯCLN}(a,b) = 4$. (b) $ab = 294,\mbox{\rm ƯCLN}(a,b) = 7$.
\end{baitoan}

\begin{baitoan}[\cite{TLCT_THCS_Toan_6_so_hoc}, 5.11., p. 38]
	Tìm $a,b\in\mathbb{N}$ thỏa $\mbox{\rm ƯCLN}(a,b) = 10,{\rm BCNN}(a,b) = 120$.
\end{baitoan}

\begin{baitoan}[\cite{TLCT_THCS_Toan_6_so_hoc}, 5.12., p. 38]
	Tìm $n\in\mathbb{N}$ biết trong 3 số $6,16,n$, bất cứ số nào cũng là ước của tích 2 số kia.
\end{baitoan}

\begin{baitoan}[\cite{TLCT_THCS_Toan_6_so_hoc}, 5.13., p. 38]
	Tuấn \& Tú ở cùng 1 nhà \& làm việc tại 2 công ty khác nhau. Tuấn cứ $10$ ngày lại đi trực 1 lần, Tú cứ $15$ ngày lại đi trực 1 lần. 2 người cùng trực vào ngày Chủ nhật 1.1.2012. Hỏi trong năm 2012, 2 người cùng trực vào 1 ngày Chủ nhật mấy lần?
\end{baitoan}

\begin{baitoan}[\cite{TLCT_THCS_Toan_6_so_hoc}, 5.14., p. 39]
	Có 2 chiếc đồng hồ. Trong 1 ngày, chiếc thứ nhất chạy nhanh $10$ phút, chiếc thứ 2 chạy chậm $6$ phút. Cả 2 đồng hồ được lấy lại theo giờ chính xác. Sau ít nhất bao lâu, cả 2 đồng hồ lại cùng chỉ giờ chính xác?
\end{baitoan}

\begin{baitoan}[\cite{TLCT_THCS_Toan_6_so_hoc}, 5.15., p. 39]
	Tìm số tự nhiên lớn nhất có 3 chữ số, biết chia nó cho $10$ thì dư $3$, chia nó cho $12$ thì dư $5$, chia nó cho $15$ thì dư $8$ \& nó chia hết cho $19$.
\end{baitoan}

\begin{baitoan}[\cite{TLCT_THCS_Toan_6_so_hoc}, 5.16., p. 39]
	Tìm $a,b,c\in\mathbb{N}^\star$ nhỏ nhất sao cho $16a = 25b = 30c$.
\end{baitoan}

\begin{baitoan}[\cite{TLCT_THCS_Toan_6_so_hoc}, 5.17., p. 39]
	Tìm số tự nhiên nhỏ nhất để khi chia cho $5,8,12$ thì số dư lần lượt là $2,6,8$.
\end{baitoan}

\begin{baitoan}[\cite{TLCT_THCS_Toan_6_so_hoc}, 5.18., p. 39]
	Điền chữ số thích hợp vào dấu $\star$ để $\overline{456\star\star}$ chia hết cho tất cả 3 số $4,5,6$.
\end{baitoan}

\begin{baitoan}[\cite{TLCT_THCS_Toan_6_so_hoc}, 5.19., p. 39]
	Chứng minh tích của 4 số tự nhiên liên tiếp thì chia hết cho $24$.
\end{baitoan}

\begin{baitoan}[\cite{TLCT_THCS_Toan_6_so_hoc}, 5.20., p. 39]
	Tìm $a,b\in\mathbb{N}$ biết: (a) ${\rm BCNN}(a,b) + \mbox{\rm ƯCLN}(a,b) = 19$. (b) ${\rm BCNN}(a,b) - \mbox{\rm ƯCLN}(a,b) = 3$.
\end{baitoan}
Chứng minh:

\begin{baitoan}[\cite{Giang_sang_tao_so_hoc}, 1, p. 11, USAMO1972 round 1]
	$\dfrac{[a,b,c]^2}{[a,b][b,c][c,a]} = \dfrac{(a,b,c)^2}{(a,b)(b,c)(c,a)}$.
\end{baitoan}

\begin{baitoan}[\cite{Giang_sang_tao_so_hoc}, 2, p. 13]
	$\dfrac{[a_1,a_2,\ldots,a_n]^2}{[a_1,a_2][a_2,a_3]\cdots[a_{n-1},a_n][a_n,a_1]} = \dfrac{(a_1,a_2,\ldots,a_n)^2}{(a_1,a_2)(a_2,a_3)\cdots(a_{n-1},a_n)(a_n,a_1)}$.
\end{baitoan}

\begin{baitoan}[\cite{Giang_sang_tao_so_hoc}, 3, pp. 13--14]
	$[a,b,c] = \dfrac{abc(a,b,c)}{(a,b)(b,c)(c,a)}$.
\end{baitoan}

\begin{baitoan}[\cite{Giang_sang_tao_so_hoc}, 4, p. 14]
	$[a,b,c] = \dfrac{(a,b,c)[a,b][b,c][c,a]}{abc}$.
\end{baitoan}

\begin{baitoan}[\cite{Giang_sang_tao_so_hoc}, 5, p. 15]
	$[a_1,a_2,\ldots,a_n] = \dfrac{a_1a_2\cdots a_n(a_1,a_2,\ldots,a_n)}{(a_1,a_2)(a_2,a_3)\cdots(a_n,a_1)}$.
\end{baitoan}

\begin{baitoan}[\cite{Giang_sang_tao_so_hoc}, 6, p. 16]
	$[a_1,a_2,\ldots,a_n] = \dfrac{(a_1,a_2,\ldots,a_n)[a_1,a_2][a_2,a_3]\cdots[a_n,a_1]}{a_1a_2\cdots a_n}$.
\end{baitoan}

\begin{baitoan}[\cite{Giang_sang_tao_so_hoc}, 7, p. 17]
	$\dfrac{[a,b,c]}{[a,b][b,c][c,a]} = \dfrac{(a,b,c)}{abc}$.
\end{baitoan}

\begin{baitoan}[\cite{Giang_sang_tao_so_hoc}, 8, p. 17]
	$\dfrac{[a,b,c]^3}{[a,b][b,c][c,a]} = \dfrac{abc(a,b,c)^3}{(a,b)^2(b,c)^2(c,a)^2}$.
\end{baitoan}

\begin{baitoan}[\cite{Giang_sang_tao_so_hoc}, 9, p. 18]
	$\dfrac{[a_1,a_2,\ldots,a_n]}{[a_1,a_2][a_2,a_3]\cdots[a_{n-1},a_n][a_n,a_1]} = \dfrac{(a_1,a_2,\ldots,a_n)}{a_1a_2\cdots a_n}$.
\end{baitoan}

\begin{baitoan}[\cite{Giang_sang_tao_so_hoc}, 10, p. 18]
	$\dfrac{[a_1,a_2,\ldots,a_n]^3}{[a_1,a_2][a_2,a_3]\cdots[a_{n-1},a_n][a_n,a_1]} = \dfrac{a_1a_2\cdots a_n(a_1,a_2,\ldots,a_n)^3}{(a_1,a_2)^2(a_2,a_3)^2\cdots(a_{n-1},a_n)^2(a_n,a_1)^2}$.
\end{baitoan}

\begin{baitoan}[\cite{Giang_sang_tao_so_hoc}, 11, p. 19]
	$(a,b,c)[ab,bc,ca] = abc$.
\end{baitoan}

\begin{baitoan}[\cite{Giang_sang_tao_so_hoc}, 12, p. 19]
	$[a,b,c] = [a,[b,c]]$.
\end{baitoan}

\begin{baitoan}[\cite{Giang_sang_tao_so_hoc}, 13, p. 19, Klamkin 1988]
	(a) $[a,(b,c)] = ([a,b],[c,d])$. (b) $(a,[b,c]) = [(a,b),(c,d)]$. (c) $([a,b],[b,c],[c,a]) = [(a,b),(b,c),(c,a)]$. (d) $(ab,cd) = (a,c)(b,c) = \left(\dfrac{a}{(a,c)},\dfrac{d}{(b,d)}\right)\left(\dfrac{c}{(a,c)},\dfrac{b}{(b,d)}\right)$.
\end{baitoan}

\begin{baitoan}[\cite{Giang_sang_tao_so_hoc}, 14, p. 20]
	$(a,b)[a,b] = ab$.
\end{baitoan}

%------------------------------------------------------------------------------%

\section{Nguyên Lý Dirichlet \& Bài Toán Chia Hết}

\begin{baitoan}[\cite{Tuyen_Toan_6}, VD33, p. 31]
	Cho $7$ số tự nhiên bất kỳ. Chứng minh bao giờ cũng có thể chọn ra 2 số mà hiệu của chúng chia hết cho $6$.
\end{baitoan}

\begin{baitoan}[\cite{Tuyen_Toan_6}, VD34, p. 32]
	Cho 3 số lẻ. Chứng minh tồn tại 2 số có tổng hay hiệu chia hết cho $8$.
\end{baitoan}

\begin{baitoan}[\cite{Tuyen_Toan_6}, VD35, p. 32]
	Chứng minh có 1 số tự nhiên gồm toàn chữ số $3$ chia hết cho $41$.
\end{baitoan}

\begin{baitoan}[\cite{Tuyen_Toan_6}, 154., p. 33]
	Chứng minh trong $11$ số tự nhiên bất kỳ bao giờ cũng có ít nhất $2$ số có chữ số tận cùng giống nhau.
\end{baitoan}

\begin{baitoan}[\cite{Tuyen_Toan_6}, 155., p. 33]
	Chứng minh tồn tại 1 bội của $13$ gồm toàn chữ số $2$.
\end{baitoan}

\begin{baitoan}[\cite{Tuyen_Toan_6}, 156., p. 33]
	Chứng minh có thể tìm được 1 số có dạng $987987\cdots987$ chia hết cho $2021$.
\end{baitoan}

\begin{baitoan}[\cite{Tuyen_Toan_6}, 157., p. 33]
	Cho dãy số $10,10^2,10^3,\ldots,10^{20}$. Chứng minh tồn tại 1 số chia cho $19$ dư $1$.
\end{baitoan}

\begin{baitoan}[\cite{Tuyen_Toan_6}, 158., p. 33]
	Chứng minh tồn tại 1 bội số là bội của $19$ có tổng các chữ số bằng $19$.
\end{baitoan}

\begin{baitoan}[\cite{Tuyen_Toan_6}, 159., p. 33]
	Cho 3 số nguyên tố lớn hơn $3$. Chứng minh tồn tại 2 số có tổng hoặc hiệu chia hết cho $12$.
\end{baitoan}

\begin{baitoan}[\cite{Tuyen_Toan_6}, 160., p. 33]
	Chứng minh trong 3 số tự nhiên bất kỳ luôn chọn được 2 số có tổng chia hết cho $2$.
\end{baitoan}

\begin{baitoan}[\cite{Tuyen_Toan_6}, 161., p. 33]
	Chứng minh trong 7 số tự nhiên bất kỳ ta luôn chọn được 4 số có tổng chia hết cho $4$.
\end{baitoan}

\begin{baitoan}[\cite{Tuyen_Toan_6}, 162., p. 33]
	Cho 5 số tự nhiên bất kỳ, chứng minh luôn chọn được 3 số có tổng chia hết cho $3$.
\end{baitoan}

\begin{baitoan}[\cite{Tuyen_Toan_6}, 163., p. 33]
	Cho 5 số tự nhiên lẻ bất kỳ, chứng minh luôn chọn được 4 số có tổng chia hết cho $4$.
\end{baitoan}

\begin{baitoan}[\cite{Tuyen_Toan_6}, 164., p. 33]
	Viết 6 số tự nhiên vào 6 mặt của 1 con xúc xắc. Chứng minh khi ta gieo mặt xúc xắc xuống mặt bàn thì trong 5 mặt có thể nhìn thấy bao giờ cũng tìm được 1 hay nhiều mặt để tổng các số trên mặt đó chia hết cho $5$.
\end{baitoan}

%------------------------------------------------------------------------------%

\section{Miscellaneous}
\cite[BTCCI, pp. 37--38]{SBT_Toan_6_Canh_Dieu_tap_1}: 128. 129. 130. 131. 132. 133. 134. 135. 136. 137. 138. 139. 140.

\begin{baitoan}[\cite{Tuyen_Toan_6}, VD36, p. 33]
	Chứng minh tích các ước của $50$ là $50^3$.
\end{baitoan}

\begin{baitoan}[\cite{Tuyen_Toan_6}, VD3, p. 33]
	1 bệnh nhân được bác sĩ chỉ định nhỏ thuốc đau mắt, $4$ giờ 1 lần \& nhỏ tai $6$ giờ 1 lần. Nếu bệnh nhân này vừa nhỏ thuốc đau mắt vừa nhỏ tai vào lúc {\rm6:00} hằng ngày thì ít nhất đến mấy giờ bệnh nhân này lại vừa nhỏ mắt vừa nhỏ tai?
\end{baitoan}

\begin{baitoan}[\cite{Tuyen_Toan_6}, 165., p. 34]
	Tính hợp lý: (a) $\underbrace{19 + 19 + \cdots + 19}_{23} + \underbrace{77 + 77 + \cdots + 77}_{19}$. (b) $1000!(456\cdot789789 - 789\cdot456456)$.
\end{baitoan}

\begin{baitoan}[\cite{Tuyen_Toan_6}, 166., p. 34]
	Tìm $x\in\mathbb{N}$ thỏa: (a) $x + (x + 1) + (x + 2) + \cdots + (x + 30) = 1240$. (b) $1 + 2 + \cdots + x = 210$.
\end{baitoan}

\begin{baitoan}[\cite{Tuyen_Toan_6}, 167., p. 34]
	Chứng minh: (a) $10^n + 5^3\divby9$. (b) $43^{43} - 17^{17}\divby10$. (c) $A = \underbrace{5\ldots5}_{2n}$, $A\divby11$ nhưng $A\not{\divby}\ 125$.
\end{baitoan}

\begin{baitoan}[\cite{Tuyen_Toan_6}, 168., p. 34]
	Tìm $a\in\mathbb{N}$ nhỏ nhất sao cho $a$ chia cho $17$ dư $5$, $a$ chia cho $19$ dư $12$.
\end{baitoan}

\begin{baitoan}[\cite{Tuyen_Toan_6}, 169., p. 34]
	Tìm $a\in\mathbb{N}$ biết $355$ chia cho $a$ dư $13$ \& số $836$ chia cho $a$ dư $8$.
\end{baitoan}

\begin{baitoan}[\cite{Tuyen_Toan_6}, 170., p. 34]
	$a\in\mathbb{N}$ chia cho $7$ dư $5$, chia cho $13$ dư $4$. Tìm số dư khi $a$ chia cho $91$.
\end{baitoan}

\begin{baitoan}[\cite{Tuyen_Toan_6}, 171., p. 34]
	Biết ngày 1.2.2023 là thứ 4. (a) Hỏi ngày 1.3.2023, 1.4.2023 là thứ mấy? (b) Ngày 1.2.2024 là thứ mấy?
\end{baitoan}

\begin{baitoan}[\cite{Tuyen_Toan_6}, 172., p. 34]
	Tính tổng: (a) $A = \sum_{i=0}^{20} 2^i = 1 + 2 + 2^2 + \cdots + 2^{20}$. (b) $B = 4 + 4^3 + 4^5 + \cdots + 4^{49}$.
\end{baitoan}

\begin{baitoan}[\cite{Tuyen_Toan_6}, 173., p. 34]
	Cho $A = \sum_{i=1}^{24} 4^i = 4 + 4^2 + \cdots + 4^{24}$. Chứng minh $A\divby20,A\divby21,A\divby420$.
\end{baitoan}

\begin{baitoan}[\cite{Tuyen_Toan_6}, 174., p. 34]
	Cho $n = 29k$ với $k\in\mathbb{N}$. Với giá trị nào của $k$ thì $n$ là: (a) Số nguyên tố. (b) Hợp số. (c) Không phải là số nguyên tố cũng không phải là hợp số?
\end{baitoan}

\begin{baitoan}[\cite{Tuyen_Toan_6}, 175., p. 34]
	Cho $a$ là 1 hợp số, khi phân tích ra thừa số nguyên tố chỉ chứa 2 thừa số nguyên tố khác nhau là $p_1,p_2$. Biết $a^3$ có tất cả $40$ ước, tính số ước của $a^2$.
\end{baitoan}

\begin{baitoan}[\cite{Tuyen_Toan_6}, 176., p. 34]
	Cho 3 số $12,18,27$. (a) Tìm số lớn nhất có 3 chữ số chia hết cho 3 số đó. (b) Tìm số nhỏ nhất có 4 chữ số chia cho mỗi số đó đều dư $1$. (c) Tìm số nhỏ nhất có 4 chữ số chia cho $12$ dư $10$, chia cho $18$ dư $16$, chia cho $27$ dư $25$.
\end{baitoan}

%------------------------------------------------------------------------------%

\printbibliography[heading=bibintoc]

\end{document}