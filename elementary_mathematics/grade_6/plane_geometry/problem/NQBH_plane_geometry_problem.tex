\documentclass{article}
\usepackage[backend=biber,natbib=true,style=alphabetic,maxbibnames=50]{biblatex}
\addbibresource{/home/nqbh/reference/bib.bib}
\usepackage[utf8]{vietnam}
\usepackage{tocloft}
\renewcommand{\cftsecleader}{\cftdotfill{\cftdotsep}}
\usepackage[colorlinks=true,linkcolor=blue,urlcolor=red,citecolor=magenta]{hyperref}
\usepackage{amsmath,amssymb,amsthm,float,graphicx,mathtools,tikz}
\usetikzlibrary{angles,calc,intersections,matrix,patterns,quotes,shadings}
\allowdisplaybreaks
\newtheorem{assumption}{Assumption}
\newtheorem{baitoan}{}
\newtheorem{cauhoi}{Câu hỏi}
\newtheorem{conjecture}{Conjecture}
\newtheorem{corollary}{Corollary}
\newtheorem{dangtoan}{Dạng toán}
\newtheorem{definition}{Definition}
\newtheorem{dinhly}{Định lý}
\newtheorem{dinhnghia}{Định nghĩa}
\newtheorem{example}{Example}
\newtheorem{ghichu}{Ghi chú}
\newtheorem{hequa}{Hệ quả}
\newtheorem{hypothesis}{Hypothesis}
\newtheorem{lemma}{Lemma}
\newtheorem{luuy}{Lưu ý}
\newtheorem{nhanxet}{Nhận xét}
\newtheorem{notation}{Notation}
\newtheorem{note}{Note}
\newtheorem{principle}{Principle}
\newtheorem{problem}{Problem}
\newtheorem{proposition}{Proposition}
\newtheorem{question}{Question}
\newtheorem{remark}{Remark}
\newtheorem{theorem}{Theorem}
\newtheorem{vidu}{Ví dụ}
\usepackage[left=1cm,right=1cm,top=5mm,bottom=5mm,footskip=4mm]{geometry}
\def\labelitemii{$\circ$}
\DeclareRobustCommand{\divby}{%
	\mathrel{\vbox{\baselineskip.65ex\lineskiplimit0pt\hbox{.}\hbox{.}\hbox{.}}}%
}

\title{Problem: Plane Geometry -- Bài Tập: Hình Học Phẳng}
\author{Nguyễn Quản Bá Hồng\footnote{e-mail: \texttt{nguyenquanbahong@gmail.com}, website: \url{https://nqbh.github.io}, Ben Tre City, Vietnam.}}
\date{\today}

\begin{document}
\maketitle
\begin{abstract}
	Latest version:
	\begin{itemize}
		\item \textit{Problem: Plane Geometry -- Bài Tập: Hình Học Phẳng}.\\{\sc url}: \url{https://github.com/NQBH/elementary_STEM_beyond/blob/main/elementary_mathematics/grade_6/plane_geometry/problem/NQBH_plane_geometry_problem.pdf}.
		\item \textit{Problem \& Solution: Plane Geometry -- Bài Tập \& Lời Giải: Hình Học Phẳng}.\\{\sc url}: \url{https://github.com/NQBH/elementary_STEM_beyond/blob/main/elementary_mathematics/grade_6/plane_geometry/solution/NQBH_plane_geometry_solution.pdf}.
	\end{itemize}
\end{abstract}
\tableofcontents

%------------------------------------------------------------------------------%

%------------------------------------------------------------------------------%

\section{Plane. Point. Line -- Mặt Phẳng. Điểm. Đường Thẳng}
\fbox{1} 3 hình hình học không định nghĩa: mặt phẳng, điểm, đường thẳng. Điểm được đặt tên bằng chữ cái in hoa, e.g., $A,B,C,D$, $\ldots,M,N,P,Q,\ldots,X,Y,Z$. Đường thẳng được đặt tên bằng chữ cái in thường, e.g., $a,b,c,d,\ldots,m,n,p,q,\ldots,x,y,z$. \fbox{2} {\sf Về vị trí của điểm \& đường thẳng}: Với 1 đường thẳng bất kỳ, có vô số điểm thuộc đường thẳng đó \& có vô số điểm không thuộc đường thẳng đó. Điểm $A$ thuộc đường thẳng $d$ ký hiệu là $A\in d$. Điểm $B$ không thuộc đường thẳng $d$ ký hiệu là $B\notin d$. \fbox{3} {\sf Tiên đề về sự xác định đường thẳng}: Có 1 \& chỉ 1 đường thẳng đi qua 2 điểm phân biệt. Khi 1 đường thẳng đi qua 2 điểm $A,B$, có đường thẳng $AB$ hoặc đường thẳng $BA$. 1 quan hệ hình học được định nghĩa: 3 điểm thẳng hàng. Khi 3 điểm $A,B,C$ cùng thuộc 1 đường thẳng thì chúng \textit{thẳng hàng}. Nếu 3 điểm $A,B,C$ không cùng thuộc bất cứ 1 đường thẳng nào thì chúng \textit{không thẳng hàng}. \fbox{4} 2 đường thẳng phân biệt hoặc có 1 điểm chung, hoặc không có điểm chung nào. \fbox{5} Với 2 đường thẳng bất kỳ thì giữa chúng hoặc có 1 điểm chung (2 đường thẳng cắt nhau), hoặc không có điểm chung nào (2 đường thẳng song song), hoặc có vô số điểm chung (2 đường thẳng trùng nhau). \fbox{6} {\sf Tính chất về thứ tự của 3 điểm trên đường thẳng}: Trong 3 điểm thẳng hàng, có 1 điểm \& chỉ 1 điểm nằm giữa 2 điểm còn lại. 1 quan hệ hình học không định nghĩa: Điểm nằm giữa 2 điểm khác. \fbox{7} Với 3 điểm $A,B,C$ thẳng hàng mà điểm $B$ không nằm giữa 2 điểm $A,C$, điểm $C$ không nằm giữa 2 điểm $A,B$, thì điểm $A$ phải nằm giữa 2 điểm $B,C$. \fbox{8} Với 3 điểm $A,B,C$ thẳng hàng mà 2 điểm $A,B$ nằm cùng phía đối với điểm $C$ \& 2 điểm $A,C$ nằm cùng phía đối với điểm $B$, thì điểm $A$ nằm giữa 2 điểm $B,C$. \fbox{9} Nếu điểm $A$ nằm giữa 2 điểm $B,C$ mà điểm $M$ nằm giữa 2 điểm $A,B$ \& điểm $N$ nằm giữa 2 điểm $A,C$ thì điểm $A$ nằm giữa 2 điểm $M,N$. \fbox{10} {\sf Vị trí tương đối của 2 đường thẳng}: Song song: $a\parallel b\Leftrightarrow a\cap b = \emptyset$, i.e., $a,b$ không có điểm chung. Cắt nhau: $|a\cap b| = 1$, i.e., $a,b$ có đúng 1 điểm chung. Trùng nhau: $a\equiv b\Leftrightarrow|a\cap b|\ge 2\Leftrightarrow|a\cap b| = \infty$, i.e., $a,b$ có (ít nhất) 2 điểm chung.\\

\noindent SGK: \cite[\S1, p. 79]{SGK_Toan_6_Canh_Dieu_tap_1}: 1. 2. 3. 4. 5. 6. 7. SBT: \cite{SBT_Toan_6_Canh_Dieu_tap_2}: 1. 2. 3. 4. 5. 6. 7. 8. 9. 10. 11. 12.

\begin{baitoan}[\cite{Binh_boi_duong_Toan_6_tap_2}, H1--H4, p. 57]
	{\rm Đ{\tt/}S?} Nếu sai, sửa cho đúng. (a) Với mỗi đường thẳng a, có các điểm thuộc a \& các điểm không thuộc a. (b) Nếu 2 đường thẳng có 2 điểm chung thì chúng có vô số điểm chung. (c) Nếu điểm M không nằm giữa A,B thì 3 điểm A,B,M không thẳng hàng. (d) Cho 3 điểm A,B,C. Nếu không có điểm nào nằm giữa 2 điểm còn lại thì 3 điểm đó không thẳng hàng.
\end{baitoan}

\begin{baitoan}[\cite{Binh_boi_duong_Toan_6_tap_2}, VD1, p. 57]
	(a) Vẽ 2 đường thẳng a,b cắt nhau tại O. Lấy $A\in a,B\in b$, $A\ne O,B\ne O$. (b) A,O,B thẳng hàng không?
\end{baitoan}

\begin{baitoan}[\cite{Binh_boi_duong_Toan_6_tap_2}, VD2, p. 57]
	Cho 3 điểm A,B,C thẳng hàng, 3 điểm B,C,D cũng thẳng hàng. A,B,C,D thẳng hàng không?
\end{baitoan}

\begin{baitoan}[\cite{Binh_boi_duong_Toan_6_tap_2}, VD3, p. 58]
	Trên đường thẳng d lấy 4 điểm E,F,G,H theo thứ tự đó. (a) F nằm giữa 2 điểm nào? (b) G nằm giữa 2 điểm nào? (c) Đếm số bộ 3 điểm thẳng hàng.
\end{baitoan}

\begin{baitoan}[\cite{Binh_boi_duong_Toan_6_tap_2}, VD4, p. 58]
	Cho 5 điểm A,B,C,D,E trong đó không có 3 điểm nào thẳng hàng. Vẽ các đường thẳng đi qua các cặp điểm. Tìm số đường thẳng vẽ được bằng 2 phương pháp: (a) Liệt kê. (b) Lập luận.
\end{baitoan}

\begin{baitoan}
	Chứng minh: (a) Có $C_n^2 = \dfrac{n(n - 1)}{2}$ đoạn thẳng tạo bởi $n$ điểm phân biệt. (b) Có $C_n^2 = \dfrac{n(n - 1)}{2}$ đường thẳng tạo bởi $n$ điểm phân biệt trong đó không có 3 điểm nào thẳng hàng.
\end{baitoan}

\begin{baitoan}[\cite{Binh_boi_duong_Toan_6_tap_2}, VD5, p. 59]
	Cho 1 số điểm trong đó không có 3 điểm nào thẳng hàng. Vẽ các đường thẳng đi qua các cặp điểm được tất cả $36$ đường thẳng. Tính số điểm đã cho.
\end{baitoan}

\begin{baitoan}[\cite{Binh_boi_duong_Toan_6_tap_2}, VD6, p. 59]
	Cho 3 đường thẳng phân biệt. Tính số giao điểm của 3 đường thẳng này.
\end{baitoan}

\begin{baitoan}
	(a) Tính số giao điểm của 4 đường thẳng phân biệt. (b) Tính số giao điểm của $n\in\mathbb{N}^\star$ đường thẳng phân biệt.
\end{baitoan}

\begin{baitoan}[\cite{Binh_boi_duong_Toan_6_tap_2}, VD7, p. 60]
	Cho $10$ đường thẳng đôi một cắt nhau, trong đó có đúng 3 đường thẳng cùng đi qua 1 điểm. Đếm số giao điểm.
\end{baitoan}

\begin{baitoan}
	Cho $n\in\mathbb{N}^\star$ đường thẳng đôi một cắt nhau, trong đó có 1 số bộ đường thẳng cùng đi qua 1 điểm. Đếm số giao điểm.
\end{baitoan}

\begin{baitoan}[\cite{Binh_boi_duong_Toan_6_tap_2}, 10.1., p. 60]
	Trên đường thẳng d lấy 5 điểm A,B,C,D,E theo thứ tự đó. C nằm giữa 2 điểm nào?
\end{baitoan}

\begin{baitoan}[\cite{Binh_boi_duong_Toan_6_tap_2}, 10.2., p. 60]
	Cho 3 điểm A,B,C. Biết mỗi điểm A,B đều không nằm giữa 2 điểm còn lại. Tìm điều kiện để: (a) C nằm giữa 2 điểm còn lại. (b) C không nằm giữa 2 điểm còn lại.
\end{baitoan}

\begin{baitoan}[\cite{Binh_boi_duong_Toan_6_tap_2}, 10.3., p. 60]
	Cho $201$ điểm trong đó không có 3 điểm nào thẳng hàng. Vẽ các đường thẳng đi qua các cặp điểm. Đếm số đường thẳng.
\end{baitoan}

\begin{baitoan}[\cite{Binh_boi_duong_Toan_6_tap_2}, 10.4., p. 60]
	Cho $15$ điểm trong đó có đúng 3 điểm thẳng hàng. Vẽ các đường thẳng đi qua các cặp điểm. Đếm số đường thẳng.
\end{baitoan}

\begin{baitoan}[\cite{Binh_boi_duong_Toan_6_tap_2}, 10.5., p. 60]
	Cho $12$ điểm trong đó không có 3 điểm nào thẳng hàng. Vẽ các đường thẳng đi qua các cặp điểm. Nếu bớt đi $4$ điểm thì số đường thẳng vẽ được giảm đi bao nhiêu?
\end{baitoan}

\begin{baitoan}[\cite{Binh_boi_duong_Toan_6_tap_2}, 10.6., p. 60]
	Cho 1 số điểm trong đó không có 3 điểm nào thẳng hàng. Vẽ các đường thẳng đi qua các cặp điểm. Tính số điểm cho trước nếu số đường thẳng vẽ được là: (a) $120$. (b) $300$.
\end{baitoan}

\begin{baitoan}[\cite{Binh_boi_duong_Toan_6_tap_2}, 10.7., p. 60]
	Cho 5 điểm A,B,C,D,E. Vẽ các đường thẳng đi qua các cặp điểm. Đếm số đường thẳng.
\end{baitoan}

\begin{baitoan}[\cite{Binh_boi_duong_Toan_6_tap_2}, 10.8., p. 60]
	Cho 4 điểm A,B,C,D. Vẽ các đường thẳng đi qua các cặp điểm. Tìm điều kiện của 4 điểm để số đường thẳng vẽ được là $4$.
\end{baitoan}

\begin{baitoan}[\cite{Binh_boi_duong_Toan_6_tap_2}, 10.9., p. 60]
	Vẽ $4$ đường thẳng trong đó có đúng 2 đường thẳng song song sao cho số giao điểm: (a) Nhiều nhất. (b) Ít nhất.
\end{baitoan}

\begin{baitoan}[\cite{Binh_boi_duong_Toan_6_tap_2}, 10.10., p. 61]
	Cho 1 số điểm trong đó có đúng 3 điểm thẳng hàng. Vẽ các đường thẳng đi qua các cặp điểm. Biết số đường thẳng vẽ được là $53$. Tính số điểm.
\end{baitoan}

\begin{baitoan}[\cite{Binh_boi_duong_Toan_6_tap_2}, 10.11., p. 61]
	Cho biết 3 đường thẳng a,b,m cùng đi qua 1 điểm. 3 đường thẳng a,b,n cùng đi qua 1 điểm. Chứng minh 4 đường thẳng a,b,m,n đồng quy.
\end{baitoan}

\begin{baitoan}[\cite{Binh_boi_duong_Toan_6_tap_2}, 10.12., p. 61]
	Cho 3 đường thẳng phân biệt cắt nhau từng đôi một tại A,B,C. A,B,C thẳng hàng không?
\end{baitoan}

\begin{baitoan}[\cite{Binh_boi_duong_Toan_6_tap_2}, 10.13., p. 61]
	Vẽ điểm M nằm giữa P,Q, điểm Q nằm giữa 2 điểm P,N. Chứng minh M,N,P,Q thẳng hàng.
\end{baitoan}

\begin{baitoan}[\cite{Binh_boi_duong_Toan_6_tap_2}, 10.14., p. 61]
	Cho $8$ đường thẳng đôi một cắt nhau. Đếm số giao điểm nếu: (a) Trong số các đường thẳng đã cho, không có 3 đường thẳng nào cùng đi qua 1 điểm. (b) Mỗi giao điểm đều là điểm chung của 2 đường thẳng, chỉ trừ 1 giao điểm là điểm chung của đúng 4 đường thẳng.
\end{baitoan}

\begin{baitoan}[\cite{Binh_boi_duong_Toan_6_tap_2}, 10.15., p. 61]
	Cho 1 số điểm trong đó có đúng $4$ điểm thẳng hàng. Vẽ các đường thẳng đi qua các cặp điểm. Biết số đường thẳng vẽ được là $31$. Đếm số điểm.
\end{baitoan}

\begin{baitoan}[\cite{Binh_boi_duong_Toan_6_tap_2}, p. 61]
	Trồng $7$ cây thành $5$ hàng, mỗi hàng $3$ cây.
\end{baitoan}

\begin{baitoan}[\cite{Tuyen_Toan_6}, VD8, p. 87, \cite{Binh_Toan_6_tap_2}, 1., p. 65]
	Cho 4 điểm $A,B,C,D$ sao cho 3 điểm $A,B,C$ thẳng hàng; 3 điểm $B,C,D$ cũng thẳng hàng. Hỏi 4 điểm $A,B,C,D$ có thẳng hàng không? Vì sao?
\end{baitoan}

\begin{baitoan}
	Trên mặt phẳng, cho $n$ điểm $A_i$, $i = 1,2,\ldots,n$, $n\in\mathbb{N}$, $n\ge3$. Giả sử $3$ điểm bất kỳ trong số chúng đều thẳng hàng. Hỏi $n$ điểm đó có thẳng hàng không?
\end{baitoan}

\begin{baitoan}
	Trên mặt phẳng, cho $n$ điểm $A_i$, $i = 1,2,\ldots,n$, $n\in\mathbb{N}$, $n\ge3$. Giả sử 3 điểm $A_i,A_{i+1},A_{i+2}$ thẳng hàng $\forall i = 1,2,\ldots,n-2$. Hỏi $n$ điểm đó có thẳng hàng không?
\end{baitoan}

\begin{baitoan}[\cite{Tuyen_Toan_6}, VD9, p. 88]
	Trên đường thẳng $a$ lấy 4 điểm $M,N,P,Q$ theo thứ tự đó. Hỏi: (a) Điểm $N$ nằm giữa 2 điểm nào? (b) Điểm $P$ không nằm giữa 2 điểm nào?
\end{baitoan}

\begin{baitoan}[\cite{Tuyen_Toan_6}, VD10, p. 88]
	Cho $12$ điểm trong đó không có 3 điểm nào thẳng hàng. Cứ qua 2 điểm vẽ 1 đường thẳng. Hỏi: (a) Vẽ được tất cả bao nhiêu đường thẳng? (b) Nếu thay $12$ điểm bằng $n$ điểm, $n\in\mathbb{N}$, $n\ge2$, thì vẽ được bao nhiêu đường thẳng?
\end{baitoan}

\begin{baitoan}[\cite{Tuyen_Toan_6}, 38., p. 88]
	Vẽ 5 điểm $C,D,E,F,G$ không thẳng hàng nhưng 3 điểm $C,D,E$ thẳng hàng; 3 điểm $E,F,G$ thẳng hàng.
\end{baitoan}

\begin{baitoan}[\cite{Tuyen_Toan_6}, 39., p. 89]
	Trái Đất quay quanh Mặt Trời; Mặt Trăng quay quanh Trái Đất. Mặt Trời chiếu sáng tới Trái Đất \& Mặt Trăng. Khi 3 thiên thể này thẳng hàng thì xảy ra nhật thực hoặc nguyệt thực (là hiện tượng Mặt Trời hoặc Mặt Trăng đang sáng bỗng nhiên bị che lấp \& tối đi). Hỏi: (a) Khi xảy ra nhật thực thì Mặt Trăng ở vị trí nào? (b) Khi xảy ra nguyệt thực thì Trái Đất ở vị trí nào?
\end{baitoan}

\begin{baitoan}[\cite{Tuyen_Toan_6}, 40., p. 89]
	Cho tứ giác $ABCD$, $O$ là giao điểm 2 đường chéo. Qua $O$, vẽ 2 đường thẳng $a,b$ sao cho $a$ cắt cạnh $AB,CD$ lần lượt tại $M,N$, $b$ cắt cạnh $AD,BC$ lần lượt tại $E,F$. Có bao nhiêu trường hợp 1 điểm nằm giữa 2 điểm khác? Kể ra tất cả các trường hợp đó.
\end{baitoan}

\begin{baitoan}[\cite{Tuyen_Toan_6}, 41., p. 89]
	Theo bài toán trước, ta có thể trồng $9$ cây thành $8$ hàng, mỗi hàng $3$ cây. Vẽ sơ đồ trồng $9$ cây thành: (a) $9$ hàng, mỗi hàng $3$ cây; (b) $10$ hàng, mỗi hàng $3$ cây.
\end{baitoan}

\begin{baitoan}[\cite{Tuyen_Toan_6}, 42., p. 89]
	Cho 2 điểm $A,B$. (a) Vẽ đường thẳng $m$ đi qua $A,B$; (b) Vẽ đường thẳng $n$ đi qua $A$ nhưng không đi qua $B$; (c) Vẽ đường thẳng $p$ không có điểm chung nào với đường thẳng $m$.
\end{baitoan}

\begin{baitoan}[\cite{Tuyen_Toan_6}, 43., p. 89]
	Cho 4 điểm $A,B,C,D$ trong đó không có 3 điểm nào thẳng hàng. Xác định điểm $M$ sao cho 3 điểm $M,A,B$ thẳng hàng; 3 điểm $M,C,D$ thẳng hàng.
\end{baitoan}

\begin{baitoan}[\cite{Tuyen_Toan_6}, 44., p. 89]
	Cho 3 điểm $C,O,D$ thẳng hàng. Biết điểm $C$ không nằm giữa 2 điểm $O,D$, điểm $O$ không nằm giữa 2 điểm $C,D$. Hỏi trong 3 điểm đã cho, điểm nào nằm giữa 2 điểm còn lại?
\end{baitoan}

\begin{baitoan}[\cite{Tuyen_Toan_6}, 45., p. 89]
	Cho 3 điểm $A,B,C$ trong đó không có điểm nào nằm giữa 2 điểm còn lại. Hỏi 3 điểm $A,B,C$ có thẳng hàng không?
\end{baitoan}

\begin{baitoan}[\cite{Tuyen_Toan_6}, 46., p. 89]
	Cho 6 điểm. Cứ qua 2 điểm vẽ 1 đường thẳng. Hỏi: (a) Nếu trong 6 điểm đó không có 3 điểm nào thẳng hàng thì sẽ vẽ được bao nhiêu đường thẳng? (b) Nếu trong 6 điểm đó có đúng 3 điểm thẳng hàng thì sẽ vẽ được bao nhiêu đường thẳng?
\end{baitoan}

\begin{baitoan}[\cite{Tuyen_Toan_6}, 47., p. 89]
	Giải bóng đá vô địch quốc gia hạng chuyên nghiệp có 16 đội tham gia đấu vòng tròn 2 lượt đi \& về. Tính tổng số trận đấu.
\end{baitoan}

\begin{baitoan}[\cite{Tuyen_Toan_6}, 48., p. 89]
	Cho $n$ điểm, $n\in\mathbb{N}$, $n\ge2$, trong đó không có 3 điểm nào thẳng hàng. Cứ qua 2 điểm vẽ 1 đường thẳng. Biết số đường thẳng vẽ được là $36$, tính giá trị của $n$.
\end{baitoan}

\begin{baitoan}[\cite{Tuyen_Toan_6}, 49., p. 89]
	Cho 11 đường thẳng đôi một cắt nhau. Hỏi: (a) Nếu trong số đó không có 3 đường thẳng nào cùng đi qua 1 điểm thì có tất cả bao nhiêu giao điểm của chúng? (b) Nếu trong 11 đường thẳng đó có đúng 5 đường thẳng cùng đi qua 1 điểm thì có tất cả bao nhiêu giao điểm của chúng?
\end{baitoan}

\begin{baitoan}[\cite{Tuyen_Toan_6}, 50., p. 90]
	Cho $n$ điểm, $n\in\mathbb{N}$, $n\ge2$, trong đó không có 3 điểm nào thẳng hàng. Cứ qua 2 điểm vẽ 1 đường thẳng. Tìm $n$ biết nếu có thêm 1 điểm (không thẳng hàng với bất kỳ 2 điểm nào trong số $n$ điểm đã cho) thì số đường thẳng vẽ được tăng thêm là $8$.
\end{baitoan}

\begin{baitoan}[\cite{Tuyen_Toan_6}, 51., p. 90]
	Cho $13$ điểm trong đó không có 3 điểm nào thẳng hàng. Cứ qua 2 điểm vẽ 1 đường thẳng. Nếu ta bớt đi $4$ điểm thì số đường thẳng vẽ được giảm đi bao nhiêu?
\end{baitoan}

\begin{baitoan}[\cite{Tuyen_Toan_6}, 52., p. 90]
	Cho $n$ điểm, $n\in\mathbb{N}$, $n\ge2$, trong đó không có 3 điểm nào thẳng hàng. Nếu bớt đi 1 điểm thì số đường thẳng vẽ được qua các cặp điểm giảm đi $10$ đường thẳng, tính $n$.
\end{baitoan}

\begin{baitoan}[\cite{Binh_Toan_6_tap_2}, VD1, p. 64]
	Cho 2 đường thẳng cắt nhau. Nếu vẽ thêm 1 đường thẳng thứ 3 cắt cả 2 đường thẳng trên thì số giao điểm của các đường thẳng thay đổi như thế nào?
\end{baitoan}

\begin{baitoan}[\cite{Binh_Toan_6_tap_2}, VD2, p. 64]
	Giải thích vì sao 2 đường thẳng phân biệt hoặc có 1 điểm chung, hoặc không có điểm chung nào.
\end{baitoan}

\begin{baitoan}[\cite{Binh_Toan_6_tap_2}, 2., p. 65]
	Vẽ 5 điểm $A,B,C,D,O$ sao cho 3 điểm $A,B,C$ thẳng hàng, 3 điểm $B,C,D$ thẳng hàng, 3 điểm $C,D,O$ không thẳng hàng. (a) $A,B,D$ có thẳng hàng không? Vì sao? (b) Kẻ các đường thẳng, mỗi đường thẳng đi qua ít nhất 2 điểm trong 5 điểm nói trên. Kể tên các đường thẳng trong hình vẽ (các đường thẳng trùng nhau chỉ kể là 1 đường thẳng).
\end{baitoan}

\begin{baitoan}[\cite{Binh_Toan_6_tap_2}, 3., p. 65]
	Cho các điểm $A,B,C,D,E$ thuộc cùng 1 đường thẳng theo thứ tự ấy. Điểm $C$ nằm giữa 2 điểm nào? Điểm $C$ không nằm giữa 2 điểm nào?
\end{baitoan}

\begin{baitoan}[\cite{Binh_Toan_6_tap_2}, 4., p. 65]
	Cho $A,B,C$ là 3 điểm thẳng hàng. Điểm nào nằm giữa 2 điểm còn lại nếu $A$ không nằm giữa $B$ \& $C$, $B$ không nằm giữa $A$ \& $C$?
\end{baitoan}

\begin{baitoan}[\cite{Binh_Toan_6_tap_2}, 5., p. 65]
	Cho 4 điểm $A,B,C,D$ trong đó điểm $B$ nằm giữa 2 điểm $A$ \& $C$, điểm $B$ nằm giữa $A$ \& $D$. Có thể khẳng định điểm $D$ nằm giữa $B$ \& $C$ không?
\end{baitoan}

\begin{baitoan}[\cite{Binh_Toan_6_tap_2}, 6., p. 65]
	(a) Xếp $10$ điểm thành $5$ hàng, mỗi hàng có $4$ điểm. (b) Xếp $7$ điểm thành $6$ hàng, mỗi hàng có $3$ điểm. (c) Người ta trồng $12$ cây thành $6$ hàng, mỗi hàng có $4$ cây. Vẽ sơ đồ vị trí của $12$ cây đó.
\end{baitoan}

\begin{baitoan}[\cite{TLCT_THCS_Toan_6_hinh_hoc}, VD1.1, p. 6]
	Vẽ 5 điểm $A,B,C,M,N$ trong đó 3 điểm $A,B,C$ thẳng hàng, 3 điểm $A,B,M$ không thẳng hàng \& 3 điểm $A,B,N$ thẳng hàng. (a) Giải thích vì sao vẽ được như vậy. (b) Chứng minh 4 điểm $A,B,C,N$ cùng thuộc 1 đường thẳng d. (c) {\rm Đ{\tt/}S?} $A\in d,B\notin d,M\in d,N\notin d$. (d) 2 đường thẳng $AN,BC$ có phân biệt không? 2 đường thẳng $AB,MN$ trùng nhau không? (e) Có bao nhiêu đường thẳng đi qua từng cặp 2 điểm trong số 5 điểm đó.
\end{baitoan}

\begin{baitoan}[\cite{TLCT_THCS_Toan_6_hinh_hoc}, VD1.2, p. 7]
	(a) Cho 4 điểm phân biệt. Cứ qua 2 điểm, vẽ được 1 đường thẳng. Đếm số đường thẳng. (b) Qua 5 điểm vẽ được nhiều nhất bao nhiêu đường thẳng?
\end{baitoan}

\begin{baitoan}[\cite{TLCT_THCS_Toan_6_hinh_hoc}, VD1.3, p. 9]
	Cho 5 điểm $A,B,C,D,E,F$ lần lượt cùng thuộc 1 đường thẳng d. (a) C nằm giữa 2 điểm nào? (b) B nằm giữa 2 điểm nào? (c) E không nằm giữa 2 điểm nào?
\end{baitoan}

\begin{baitoan}[\cite{TLCT_THCS_Toan_6_hinh_hoc}, VD1.4, p. 9]
	Cho 5 điểm $A,B,C,D,E,F$. Biết 3 điểm $A,B,C$ thẳng hàng, 3 điểm $B,C,E$ thẳng hàng, 3 điểm $C,E,F$ thẳng hàng. Chứng minh 5 điểm $A,B,C,D,E,F$ thẳng hàng.
\end{baitoan}

%------------------------------------------------------------------------------%

\section{Line segment -- Đoạn Thẳng}
\fbox{1} \textit{Đoạn thẳng} $AB$ là hình gồm 2 điểm $A,B$ \& tất cả các điểm nằm giữa $A,B$. Đoạn thẳng $AB$ còn gọi là đoạn thẳng $BA$. 2 điểm $A,B$ là 2 \textit{mút} (hoặc 2 \textit{đầu}) của đoạn thẳng $AB$. Mỗi đoạn thẳng có 1 độ dài. Độ dài đoạn thẳng là 1 số lớn hơn $0$. \fbox{2} Có thể so sánh 2 đoạn thẳng bằng cách so sánh độ dài của chúng. \fbox{3} 2 đoạn thẳng $AB,CD$ cắt nhau tại điểm $O$. Điểm $O$ gọi là \textit{giao điểm} của 2 đoạn thẳng $AB,CD$, viết $AB\cap CD = \{O\}$. Điểm $O$ nằm giữa 2 điểm $A,B$, điểm $O$ nằm giữa 2 điểm $C,D$. \fbox{4} Đoạn thẳng $AB$ \& tia $Ox$ cắt nhau tại điểm $I$, gọi là \textit{giao điểm} của tia $Ox$ \& đoạn thẳng $AB$, viết $Ox\cap AB = \{I\}$. Điểm $I$ nằm giữa 2 điểm $A,B$, 2 tia $OI,Ox$ trùng nhau. \fbox{5} Đoạn thẳng $AB$ \& đường thẳng $xy$ cắt nhau tại điểm $K$, gọi là \textit{giao điểm} của đường thẳng $xy$ \& đoạn thẳng $AB$, viết $xy\cap AB = \{K\}$. Điểm K nằm giữa 2 điểm $A,B$, 2 tia $Kx,Ky$ đối nhau. \fbox{6} {\sf Tính chất cộng các đoạn thẳng}: Nếu điểm M nằm giữa $A,B$ thì $AM + MB = AB$. Tổng quát: Nếu $A_1A_2\ldots A_n = A_1A_n$, i.e., độ dài đường gấp khúc $A_1A_2\ldots A_n$ bằng độ dài đoạn thẳng $A_1A_n$ thì $A_1,A_2,\ldots,A_n$ thẳng hàng theo thứ tự ấy. Nếu $AM + MB = AB$ thì M nằm giữa 2 điểm $A,B$. Nếu $A,B$ thuộc tia $Ox$ \& $OA < OB$ thì A nằm giữa $O,B$. See \href{https://en.wikipedia.org/wiki/Line_segment}{Wikipedia{\tt/}line segment}.

\begin{baitoan}[\cite{Binh_Toan_6_tap_2}, VD7, p. 68]
	Chứng minh nếu 2 điểm $A,B$ cùng thuộc tia Ox \& $OA < OB$ thì điểm A nằm giữa 2 điểm $O,B$.
\end{baitoan}

\begin{baitoan}[\cite{Binh_Toan_6_tap_2}, VD8, p. 69]
	Cho đoạn thẳng $AB = 3$ {\rm cm}. Điểm C thuộc đường thẳng AB sao cho $BC = 1$ {\rm cm}. Tính  đoạn thẳng AC.
\end{baitoan}

\begin{baitoan}[\cite{Binh_Toan_6_tap_2}, 15., p. 69]
	Cho đoạn thẳng AB. Trên tia đối của tia AB lấy C, trên tia đối của tia BA lấy D sao cho $BD = AC$. Chứng minh $BC = AD$.
\end{baitoan}

\begin{baitoan}[\cite{Binh_Toan_6_tap_2}, 16., p. 69]
	Cho đoạn thẳng AB có độ dài {\rm8 cm}. Trên tia AB lấy C sao cho $AC = 2$ {\rm cm}, trên tia BA lấy D sao cho $BD = 3$ {\rm cm}. Tính  $CB,CD$.
\end{baitoan}

\begin{baitoan}[\cite{Binh_Toan_6_tap_2}, 17., p. 69]
	Cho 3 điểm $A,B,C$ thẳng hàng. Biết $AB = 5$ {\rm cm}, $BC = 2$ {\rm cm}. Tính  AC.
\end{baitoan}

\begin{baitoan}[\cite{Binh_Toan_6_tap_2}, 18., p. 69]
	Trên tia Ox, vẽ 2 điểm $A,B$ sao cho $OA = a,OB = b$. Điểm C thuộc đoạn thẳng AB sao cho $AC = \frac{1}{2}BC$. Tính  OC.
\end{baitoan}

\begin{baitoan}[\cite{Binh_Toan_6_tap_2}, 19., p. 69, triangle number]
	Gọi $T_n$, $n\in\mathbb{N}^\star$, là số điểm trên mặt phẳng sao cho chúng tạo thành 1 tam giác đều có cạnh bằng $n - 1$ đơn vị \& 2 điểm gần nhau (không có điểm nào ở giữa 2 điểm đó trong số $T_n$ điểm đó) thì cách nhau $1$ đơn vị. Tìm công thức các số tam giác $T_n$.
\end{baitoan}
See, e.g., \href{https://vi.wikipedia.org/wiki/S%E1%BB%91_tam_gi%C3%A1c}{Wikipedia{\tt/}số tam giác}, \href{https://en.wikipedia.org/wiki/Triangular_number}{Wikipedia{\tt/}triangle number}.

\cite[20., p. 70]{Binh_Toan_6_tap_2}.

\begin{baitoan}[\cite{Binh_Toan_6_tap_2}, VD9, p. 70]
	Cho điểm M là trung điểm của đoạn thẳng AB. Chứng minh $AM = BM = \frac{1}{2}AB$.
\end{baitoan}

\begin{baitoan}[\cite{Binh_Toan_6_tap_2}, VD10, p. 71]
	Cho đoạn thẳng AB có độ dài a. Trên tia AB lấy M sao cho $AM = \dfrac{a}{2}$. Chứng minh M là trung điểm AB.
\end{baitoan}

\begin{baitoan}[\cite{Binh_Toan_6_tap_2}, VD11, p. 71]
	Cho đoạn thẳng $OA = a$, điểm B nằm trong đoạn thẳng OA sao cho $OB = b$. $M,N,I$ lần lượt là trung điểm $OA,OB,AB$. Tính  $IM,IN$ theo $a,b$.
\end{baitoan}

\begin{baitoan}[\cite{Binh_Toan_6_tap_2}, 21., p. 71]
	Cho $\Delta ABC$, 2 đường trung tuyến $BD,CE$ cắt nhau ở K. Kẻ đoạn thẳng DE. Đo độ dài rồi cho biết mỗi cạnh của $\Delta KDE$ bằng nửa cạnh nào của $\Delta KBC$.
\end{baitoan}

\begin{baitoan}[\cite{Binh_Toan_6_tap_2}, 22., p. 71]
	Cho đoạn thẳng $AB = 5$ {\rm cm}, điểm C nằm giữa $A,B$, 2 điểm $D,E$ lần lượt là trung điểm $AC,CB$. Tính  DE.
\end{baitoan}

\begin{baitoan}[\cite{Binh_Toan_6_tap_2}, 23., p. 71]
	Cho đoạn thẳng $AB = 5$ {\rm cm}, điểm C nằm giữa $A,B$ sao cho $AC = 2$ {\rm cm}, 2 điểm $D,E$ lần lượt là trung điểm $AC,CB$. I là trung điểm DE. Tính  $DE,CI$.
\end{baitoan}

\begin{baitoan}[\cite{Binh_Toan_6_tap_2}, 24., p. 71]
	Cho 4 điểm $A,B,C,D$ thẳng hàng theo thứ tự ấy. $M,N$ lần lượt là trung điểm $AB,CD$. (a) Biết $AC = 4$ {\rm cm}, $BD = 6$ {\rm cm}, tính MN. (b) Biết $MN = 5$ {\rm cm}, tính $AC + BD$.
\end{baitoan}

\begin{baitoan}[\cite{Binh_Toan_6_tap_2}, 25., p. 71]
	Cho đoạn thẳng AB với O là trung điểm. Điểm C thuộc đoạn thẳng OB, $OC = 1$ {\rm cm}. Tính $CA - CB$.
\end{baitoan}

\begin{baitoan}[\cite{Binh_Toan_6_tap_2}, 26., p. 72]
	Cho đoạn thẳng AB, điểm C nằm trong đoạn thẳng AB, O là trung điểm của AC. Biết $OB = 3$ {\rm cm}. Tính $AB + BC$.
\end{baitoan}

\begin{baitoan}[\cite{Binh_Toan_6_tap_2}, 27., p. 72]
	(a) Cho đoạn thẳng $AB = 2a$, điểm C nằm giữa $A,B$, 2 điểm $M,N$ lần lượt là trung điểm $AC,BC$. Chứng minh $MN = a$. (b) Kết quả (a) còn đúng không nếu điểm C thuộc đường thẳng AB?
\end{baitoan}

\begin{baitoan}[\cite{Binh_Toan_6_tap_2}, 28., p. 72]
	Cho điểm C thuộc đoạn thẳng AB có $CA = a,CB = b$. I là trung điểm AB. Tính IC.
\end{baitoan}

\begin{baitoan}[\cite{Binh_Toan_6_tap_2}, 29., p. 72]
	Cho điểm C thuộc đường thẳng AB nhưng không thuộc đoạn thẳng AB. Biết $CA = a,CB = b$. I là trung điểm AB. Tính IC.
\end{baitoan}

\begin{baitoan}[\cite{Binh_Toan_6_tap_2}, 30., p. 72]
	Trên tia Ox có 2 điểm $A,B$, $OA = a,OB = b$. I là trung điểm AB. Tính OI.
\end{baitoan}

\begin{baitoan}[\cite{Binh_Toan_6_tap_2}, 31., p. 72]
	Cho điểm O nằm trong đoạn thẳng AB có $OA = a,Ob = b$. $M,N,I$ lần lượt là trung điểm $OA,OB,AB$. Tính $IM,IN$.
\end{baitoan}

\begin{baitoan}[\cite{TLCT_THCS_Toan_6_hinh_hoc}, VD1.11, p. 13]
	Vẽ 2 đoạn thẳng $AB,CD$ cắt nhau tại điểm I. Kể tên các đoạn thẳng.
\end{baitoan}

\begin{baitoan}[\cite{TLCT_THCS_Toan_6_hinh_hoc}, VD1.12, p. 13]
	Cho 2 đường thẳng phân biệt $AB,CD$. Biết đường thẳng AB cắt đoạn thẳng CD \& đường thẳng CD cắt đoạn thẳng AB. Chứng minh đoạn thẳng AB cắt đoạn thẳng CD.
\end{baitoan}

\begin{baitoan}[\cite{TLCT_THCS_Toan_6_hinh_hoc}, VD1.13, p. 14]
	2 đường thẳng $d,d'$ cắt nhau tại O. Lấy 4 điểm $A,B,M,N$ trên đường thẳng $d'$ sao cho O nằm giữa $A,B$, B nằm giữa $O,M$, N nằm giữa $O,A$. d có cắt 3 đoạn thẳng $AB,AM,AN$ không?
\end{baitoan}

\begin{baitoan}[\cite{TLCT_THCS_Toan_6_hinh_hoc}, VD1.14, p. 15]
	Cho 4 điểm $A,B,C,D$. Qua 2 điểm vẽ 1 đường thẳng. (a) Nếu không có 3 điểm nào thẳng hàng, đếm số đoạn thẳng. (b) Nếu có 3 điểm thẳng hàng, giả sử là $A,B,C$, đếm số đoạn thẳng. (c) Xét trường hợp cả 4 điểm thẳng hàng, đếm số đoạn thẳng. (d) Trong trường hợp 4 điểm thuộc đường thẳng xy, tính số đoạn thẳng, tia.
\end{baitoan}

\begin{baitoan}[\cite{TLCT_THCS_Toan_6_hinh_hoc}, VD1.15, p. 16]
	Qua 2 điểm vẽ được 1 \& chỉ 1 đường thẳng. (a) Cho 3 điểm không thẳng hàng, vẽ được bao nhiêu đường thẳng qua 2 trong 3 điểm đó? (b) Cho 4 điểm, 5 điểm trong đó không có 3 điểm nào thẳng hàng, vẽ được bao nhiêu đường thẳng qua 2 trong các điểm đó? (c) Cho $100$ điểm trong đó không có 3 điểm nào thẳng hàng, vẽ được bao nhiêu đường thẳng qua 2 trong các điểm đó?
\end{baitoan}

\begin{baitoan}[\cite{TLCT_THCS_Toan_6_hinh_hoc}, VD1.16, p. 17]
	Cho $50$ điểm. Vẽ được bao nhiêu đường thẳng qua 2 điểm trong $50$ điểm đó nếu: (a) Không có 3 điểm nào thẳng hàng? (b) Có đúng 3 điểm thẳng hàng? (c) Có đúng $10$ điểm thẳng hàng. (d) Có đúng $n$ điểm thẳng hàng với $n\in\mathbb{N},3\le n\le50$.
\end{baitoan}

\begin{baitoan}[\cite{TLCT_THCS_Toan_6_hinh_hoc}, VD1.17, p. 18]
	Cho n điểm mà không có 3 điểm nào thẳng hàng. Cứ qua 2 điểm vẽ 1 đường thẳng. (a) Biết $n = 123$. Tính số đường thẳng vẽ được. (b) Biết số đường thẳng vẽ được là $378$. Tính số điểm n. (c) Số đường thẳng có thể là $2012$ không?
\end{baitoan}

\begin{baitoan}[\cite{TLCT_THCS_Toan_6_hinh_hoc}, VD1.18, p. 19]
	Trên mặt phẳng cho 4 đường thẳng khác nhau. (a) Có thể vẽ 4 đường thẳng đôi một cắt nhau sao cho số giao điểm của các đường thẳng là $1,2,3$ không? (b) Tính số giao điểm vẽ được nhiều nhất.
\end{baitoan}

\begin{baitoan}[\cite{TLCT_THCS_Toan_6_hinh_hoc}, VD1.19, p. 19]
	Biết bất kỳ 2 đường thẳng nào cũng cắt nhau \& không có 3 đường thẳng nào đồng quy. Tính số giao điểm của các đường thẳng nếu có n đường thẳng: (a) $n\in\{3,4,5\}$. (b) $n = 100$. (c) Xét trường hợp tổng quát $n\in\mathbb{N},n\ge3$.
\end{baitoan}

\begin{baitoan}[\cite{TLCT_THCS_Toan_6_hinh_hoc}, 1.1., pp. 20--21]
	Cho 6 điểm $A,B,C,O,M,N$ sao cho $A,B,C$ không thẳng hàng, $A,B,O$ thẳng hàng, $O,C,M$ thẳng hàng, $C,M,N$ thẳng hàng. (a) Chứng minh $O,C,M,N$ cùng thuộc 1 đường thẳng. (b) 2 đường thẳng $MN,AB$ trùng nhau không? (c) Cứ qua 2 điểm vẽ 1 đường thẳng. Đếm số được thẳng được vẽ \& liệt kê.
\end{baitoan}

\begin{baitoan}[\cite{TLCT_THCS_Toan_6_hinh_hoc}, 1.2., p. 21]
	Chứng minh 5 điểm $A,B,C,M,N$ thẳng hàng biết $A,B,M$ thẳng hàng, $B,C,N$ thẳng hàng, $A,M,N$ thẳng hàng.
\end{baitoan}

\begin{baitoan}[\cite{TLCT_THCS_Toan_6_hinh_hoc}, 1.3., p. 21]
	Cho 4 điểm A,B,C,M trong đó B nằm giữa A,C, M nằm giữa A,B. Trong 3 điểm B,C,M, điểm nào nằm giữa 2 điểm còn lại?
\end{baitoan}

\begin{baitoan}[\cite{TLCT_THCS_Toan_6_hinh_hoc}, 1.4., p. 21]
	Cho 2 tia AM,AN đối nhau. (a) Lấy điểm B sao cho điểm N nằm giữa 2 điểm A,B. A có nằm giữa 2 điểm M,N không? (b) Lấy điểm $C\ne A$ nằm giữa 2 điểm M,N. C có nằm giữa B,M không? (c) Trong 3 điểm A,B,C, điểm nào nằm giữa 2 điểm còn lại?
\end{baitoan}

\begin{baitoan}[\cite{TLCT_THCS_Toan_6_hinh_hoc}, 1.5., p. 21]
	Cho 4 điểm A,B,C,D thẳng hàng theo thứ tự đó. (a) Đếm số đoạn thẳng \& liệt kê. (b) Nếu 4 điểm A,B,C,D thẳng hàng nhưng không theo thứ tự đó, đếm số đoạn thẳng. (c) Lấy điểm O không thuộc đường thẳng AB. Nối điểm O với A,B,C,D. Đếm số đoạn thẳng.
\end{baitoan}

\begin{baitoan}[\cite{TLCT_THCS_Toan_6_hinh_hoc}, 1.6., p. 21]
	Cho n đường thẳng trong đó bất cứ 2 đường thẳng nào cũng cắt nhau, không có 3 đường nào đồng quy. (a) Tính số giao điểm của các đường thẳng khi $n = 124$. (b) Tìm n để số giao điểm bằng $124$.
\end{baitoan}

\begin{baitoan}[\cite{TLCT_THCS_Toan_6_hinh_hoc}, 1.7., p. 21]
	Cho n điểm trong đó không có 3 điểm nào thẳng hàng. Cứ qua 2 điểm vẽ 1 đường thẳng. (a) Tính số đường thẳng vẽ được khi $n = 24$. (b) Tìm n để số đường thẳng bằng $240$.
\end{baitoan}

\begin{baitoan}[\cite{TLCT_THCS_Toan_6_hinh_hoc}, 1.8., p. 21]
	Cho n điểm, nối từng cặp 2 điểm. (a) Tính số đoạn thẳng khi $n = 100$. (b) Tìm n để số đoạn thẳng bằng tổng các số từ $1$ đến $99$.
\end{baitoan}

\begin{baitoan}[\cite{TLCT_THCS_Toan_6_hinh_hoc}, 1.9., p. 21]
	$5$ đường thẳng chia mặt phẳng thành nhiều nhất bao nhiêu miền?
\end{baitoan}

\begin{baitoan}[\cite{TLCT_THCS_Toan_6_hinh_hoc}, VD2.1, p. 22]
	Cho đoạn thẳng $AB = 5$ {\rm cm}. Lấy điểm M thuộc đường thẳng AB mà $BM = 2$ {\rm cm}. Tính độ dài đoạn thẳng AM.
\end{baitoan}

\begin{baitoan}[\cite{TLCT_THCS_Toan_6_hinh_hoc}, VD2.2, p. 22]
	Cho đoạn thẳng $AB = a$. Lấy điểm M thuộc đường thẳng AB mà $BM = b$. Tính độ dài đoạn thẳng AM theo $a,b\in(0,\infty)$.
\end{baitoan}

\begin{baitoan}[\cite{TLCT_THCS_Toan_6_hinh_hoc}, VD2.3, p. 23]
	Cho C là 1 điểm thuộc đoạn thẳng AB \& không trùng với 2 điểm A,B. A có nằm giữa B,C không?
\end{baitoan}

\begin{baitoan}[\cite{TLCT_THCS_Toan_6_hinh_hoc}, VD2.4, p. 24]
	3 điểm A,B,C có thẳng hàng không nếu: (a) $AB = 2,BC = 7,AC = 5$? (b) $AB = 3,BC = 7,AC = 5$? (c) Đặt $BC = a,CA = b,AB = c$. Tìm điều kiện của $a,b,c\in(0,\infty)$ để: (i) A,B,C thẳng hàng. (ii) A,B,C không thẳng hàng.
\end{baitoan}

\begin{baitoan}[\cite{TLCT_THCS_Toan_6_hinh_hoc}, VD2.5, p. 24]
	Cho độ dài 3 đoạn thẳng $BC = a,CA = b,AB = c$. Điểm nào nằm giữa 2 điểm còn lại biết: $0 < \min\{b,c\}\le\max\{b,c\} < a$ nhưng $a < b + c$?
\end{baitoan}

\begin{baitoan}[\cite{TLCT_THCS_Toan_6_hinh_hoc}, 2.1., p. 25]
	Cho 4 điểm A,B,C,D theo thứ tự đó cùng thuộc 1 đường thẳng xy. (a) Đếm số đoạn thẳng trên đường thẳng xy \& liệt kê. (b) Chỉ ra các đoạn thẳng là tổng các đoạn thẳng khác.
\end{baitoan}

\begin{baitoan}[\cite{TLCT_THCS_Toan_6_hinh_hoc}, 2.2., p. 25]
	Cho 3 điểm A,B,C mà độ dài của 3 đoạn thẳng thỏa mãn $AB + BC > AC$. Có thể kết luận A,B,C không thẳng hàng không?
\end{baitoan}

\begin{baitoan}[\cite{TLCT_THCS_Toan_6_hinh_hoc}, 2.3., p. 25]
	A,B,C có thẳng hàng không nếu: (a) $AB = \dfrac{1}{2},BC = \dfrac{1}{3},CA = \dfrac{1}{6}$? (b) $AB = 5,BC = 11,CA = 7$?
\end{baitoan}

\begin{baitoan}[\cite{TLCT_THCS_Toan_6_hinh_hoc}, 2.4., p. 25]
	Cho A,B,C thẳng hàng. Điểm nào nằm giữa 2 điểm còn lại nếu: (a) $AB = 2,BC = 13,CA = 11$? (b) $AC = 7,BC = 11$?
\end{baitoan}

\begin{baitoan}[\cite{TLCT_THCS_Toan_6_hinh_hoc}, 2.5., p. 25]
	Cho A,B,C,D thẳng hàng theo thứ tự đó. (a) So sánh AB,CD biết $AC = BD$. (b) So sánh AC,BD biết $AB = CD$.
\end{baitoan}

\begin{baitoan}[\cite{TLCT_THCS_Toan_6_hinh_hoc}, 2.6., p. 25]
	A,B,O thuộc đường thẳng xy. Tính độ dài đoạn thẳng AB biết $OA + OB = a,OA - OB = b$, $0 < b < a$.
\end{baitoan}

\begin{baitoan}[\cite{TLCT_THCS_Toan_6_hinh_hoc}, 2.7., p. 25]
	Cho đoạn thẳng $AB = 5$. Trên tia BA lấy M sao cho $AM = 2$. (a) Trong 3 điểm A,B,M, điểm nào nằm giữa 2 điểm còn lại? (b) Tính độ dài đoạn thẳng BM. (c) Lấy điểm N thuộc tia đối của tia BA sao cho $BN = 1$. Tính MN.
\end{baitoan}

\begin{baitoan}[\cite{TLCT_THCS_Toan_6_hinh_hoc}, 2.8., p. 25]
	Cho đoạn thẳng $AB = 7$ \& điểm M nằm giữa A,B sao cho $BM = 5$. Trên tia đối của tia MA lấy N sao cho $MN = 7$. Chứng minh $AM = BN$.
\end{baitoan}

\subsection{Midpoint of a segment -- Trung điểm của 1 đoạn thẳng}
\fbox{1} \textit{Trung điểm} $M$ của đoạn thẳng $AB$ là điểm nằm giữa $A,B$ \& cách đều $A,B$. Điểm $M$ cách đều $A,B$ có nghĩa là độ dài 2 đoạn thẳng $MA,MB$ bằng nhau: $MA = MB$. Trung điểm của đoạn thẳng $AB$ còn được gọi là \textit{điểm chính giữa} của đoạn thẳng $AB$. Mỗi đoạn thẳng chỉ có 1 trung điểm duy nhất. \fbox{2} Nếu $M$ là trung điểm của đoạn thẳng $AB$ thì $M$ nằm giữa 2 điểm $A,B$ \& $MA = MB$. \fbox{3} {\sf Tính chất trung điểm}: Nếu $M$ là trung điểm của đoạn thẳng $AB$ thì: (i) $AM = BM = \frac{1}{2}AB$. (ii) \textit{Các đoạn thẳng có chung 1 trung điểm}: Cho 4 điểm $A,B,C,D$ cùng thuộc đường thẳng $xy$ theo thứ tự đó. Nếu biết các đoạn thẳng $AD,BC$ có chung trung điểm thì $AB = CD,AC = BD$. \fbox{4} {\sf Dấu hiệu trung điểm (nhận biết trung điểm của 1 đoạn thẳng)}: (i) Nếu trên đoạn thẳng $AB$ tồn tại 1 điểm $M$ sao cho $AM = \frac{1}{2}AB$ (hoặc $BM = \frac{1}{2}AB$) thì điểm $M$ là trung điểm của đoạn thẳng $AB$. (ii) Cho 4 điểm $A,B,C,D$ thuộc đường thẳng $xy$ theo thứ tự đó. Nếu $AB = CD$ thì $AD,BC$ có chung 1 trung điểm. Nếu $AC = BD$ thì $AD,BC$ có chung 1 trung điểm.

\begin{baitoan}[\cite{TLCT_THCS_Toan_6_hinh_hoc}, VD3.1, p. 28]
	Cho 2 tia đối nhau $Ox,Ox'$. (a) Trên tia Ox lấy A sao cho $OA = 6$. Trên tia $Ox'$ lấy B sao cho $OB = 6$. Chứng minh O là trung điểm đoạn thẳng AB. (b) Lấy C thuộc tia $Ox'$ sao cho $OC = 3$. C là trung điểm của cá đoạn thẳng nào?
\end{baitoan}

\begin{baitoan}[\cite{TLCT_THCS_Toan_6_hinh_hoc}, VD3., p. 28]
	Trên tia Ox lấy 3 điểm A,B,C sao cho $OA = 3,OB = 6,OC = 9$. (a) Trên tia Ox có bao nhiêu đoạn thẳng mà các điểm đầu là 2 trong số 4 điểm A,B,C,O \& liệt kê. (b) Trong 4 điểm A,B,C,O, điểm nào là trung điểm của các đoạn thẳng đã liệt kê. (c) Chứng minh OC,AB có chung 1 trung điểm.
\end{baitoan}

\begin{baitoan}[\cite{TLCT_THCS_Toan_6_hinh_hoc}, VD3.3, p. 29]
	Cho M là trung điểm của đoạn thẳng AB, C,D lần lượt là trung điểm của 2 đoạn thẳng AM,BM. E,G lần lượt là trung điểm của MC,MD. AB,EF có chung 1 trung điểm không?
\end{baitoan}

\begin{baitoan}[\cite{TLCT_THCS_Toan_6_hinh_hoc}, VD3.4, p. 30]
	Ghi 5 điểm O,A,B,C,D tại các điểm biểu diễn số $0,1,-2,-4,4$ trên trục số. Có các điểm nào là trung điểm của các đoạn thẳng có điểm đầu là 2 trong số 5 điểm đã cho?
\end{baitoan}

%------------------------------------------------------------------------------%

\subsection{Compute length of a segment -- Tính độ dài 1 đoạn thẳng}
\fbox{1} Nếu $M$ thuộc đoạn thẳng $AB$ thì $AM + MB = AB$. \fbox{2} Nếu $M$ là trung điểm của đoạn thẳng $AB$ thì $AM = MB = \frac{1}{2}AB$.

\begin{baitoan}[\cite{TLCT_THCS_Toan_6_hinh_hoc}, VD3.5, p. 31]
	Cho đoạn thẳng AB \& 1 điểm C nằm giữa 2 điểm A,B. M,N lần lượt là trung điểm của AC,BC. (a) Biết $AB = 20$. Tính độ dài đoạn thẳng MN. (b) Giả sử $MN = a$. Tính độ dài đoạn thẳng AB.
\end{baitoan}

\begin{baitoan}[\cite{TLCT_THCS_Toan_6_hinh_hoc}, VD3.6, p. 32]
	Trên đường thẳng xy đặt điểm O. Lấy 2 điểm A,B thuộc đường thẳng xy sao cho $OA = a,OB = b$, $0 < b < a$, trong đó O nằm giữa A,B. (a) Tính độ dài đoạn thẳng AB. (b) M,N lần lượt là trung điểm của OA,OB. Tính độ dài đoạn thẳng MN. (c) C là trung điểm của đoạn thẳng AB. Tính độ dài đoạn thẳng OC. (d) 2 đoạn thẳng MC,AN có chung 1 trung điểm không?
\end{baitoan}

\begin{baitoan}[\cite{TLCT_THCS_Toan_6_hinh_hoc}, VD3.7, pp. 32--33]
	Trên đường thẳng xy đặt điểm O. Lấy $A,B\in xy$ sao cho $OA = a,OB = b,0 < b < a$, trong đó B nằm giữa O,A. (a) Tính độ dài đoạn thẳng AB. (b) M,N lần lượt là trung điểm OA,OB. Tính độ dài đoạn thẳng MN. (c) C là trung điểm đoạn thẳng AB. Tính độ dài đoạn thẳng OC. (d) 2 đoạn thẳng MC,AN có chung 1 trung điểm không?
\end{baitoan}

\begin{baitoan}[\cite{TLCT_THCS_Toan_6_hinh_hoc}, VD3.8, p. 34]
	(a) Trên đường thẳng xy đặt 2 điểm A,B. O là trung điểm của AB. Lấy $M\in xy,M\notin\{A,B,O\}$. So sánh 2 đoạn thẳng MA,MB. (b) Trên đường thẳng xy đặt 3 điểm A,B,C theo thứ tự đó. Xác định vị trí điểm M trên đường thẳng xy sao cho $MB < \min\{MA,MC\}$.
\end{baitoan}

\begin{baitoan}[\cite{TLCT_THCS_Toan_6_hinh_hoc}, 3.1., p. 36]
	Lấy 5 điểm A,B,C,D,E trên tia Ox sao cho $OA = 3,OB = 5,OC = 7,OD = 11,OE = 13$. (a) Điểm nào là trung điểm của đoạn thẳng nào? (b) Các đoạn thẳng nào có chung 1 trung điểm?
\end{baitoan}

\begin{baitoan}[\cite{TLCT_THCS_Toan_6_hinh_hoc}, 3.2., p. 36]
	Cho O thuộc đường thẳng xy. Lấy A thuộc tia Ox mà $OA = 5$, B thuộc tia Oy mà $OB = 8$. Giả sử C thuộc tia Oy sao cho O là trung điểm của đoạn thẳng AC. (a) Tính độ dài đoạn thẳng BC. (b) Lấy điểm D thuộc tia Ox sao cho $OD = 8$. Chứng minh 2 đoạn thẳng AC,BD có chung 1 trung điểm.
\end{baitoan}

\begin{baitoan}[\cite{TLCT_THCS_Toan_6_hinh_hoc}, 3.3., p. 36]
	Trên đoạn thẳng $AC = 12$, lấy B sao cho $AB = 5$. (a) Tính độ dài đoạn thẳng MN biết 2 điểm M,N lần lượt là trung điểm của 2 đoạn thẳng AB,BC. (b) Lấy điểm D thuộc tia đối của tia CA sao cho $CD = 7$. Chứng minh C là trung điểm đoạn thẳng BD. (c) N có là trung điểm đoạn thẳng MK nếu K là trung điểm đoạn thẳng CD không?
\end{baitoan}

\begin{baitoan}[\cite{TLCT_THCS_Toan_6_hinh_hoc}, 3.4., p. 37]
	Cho 2 tia đối nhau $Ox,Ox'$. Trên tia Ox lấy 2 điểm A,B sao cho $OA = 1,OB = 7$. Trên tia $Ox'$ lấy C sao cho $OC = 5$. A có là trung điểm của đoạn thẳng BC không?
\end{baitoan}

\begin{baitoan}[\cite{TLCT_THCS_Toan_6_hinh_hoc}, 3.5., p. 37]
	Cho 4 điểm A,C,D,B theo thứ tự thuộc đường thẳng xy. Biết $AB = 6,AC = 2,CD = 1$. Chứng minh D là trung điểm đoạn thẳng AB.
\end{baitoan}

\begin{baitoan}[\cite{TLCT_THCS_Toan_6_hinh_hoc}, 3.6., p. 37]
	Trên đường thẳng xy đặt 3 điểm O,A,B. Giả sử $OA = a,OB = a + b$. Tính khoảng cách giữa trung điểm M của OA \& trung điểm N của OB.
\end{baitoan}

\begin{baitoan}[\cite{TLCT_THCS_Toan_6_hinh_hoc}, 3.7., p. 37]
	Trên đường thẳng xy đặt 4 điểm phân biệt A,B,C,D theo thứ tự đó sao cho $AB = 60,BC = 20,CD = 60$. Các cặp đoạn thẳng nào có chung trung điểm?
\end{baitoan}

\begin{baitoan}[\cite{TLCT_THCS_Toan_6_hinh_hoc}, 3.8., p. 37]
	Cho C thuộc tia đối của tia AB hoặc tia đối của tia BA. Chứng minh $CM = \frac{1}{2}(AC + BC)$ với M là trung điểm đoạn thẳng AB.
\end{baitoan}

\begin{baitoan}[\cite{TLCT_THCS_Toan_6_hinh_hoc}, 3.9., p. 37]
	Cho đoạn thẳng $AA_0 = 1$. (a) Lấy $A_1$ là trung điểm đoạn thẳng $AA_0$. Tính tỷ số $\dfrac{AA_0}{AA_1}$. (b) Tương tự, lấy các điểm $A_2,A_3,\ldots,A_{2012}$ lần lượt là trung điểm của các đoạn thẳng $AA_1,AA_2,\ldots,AA_{2021}$. Đặt $S\coloneqq\sum_{i=1}^{2012} \dfrac{AA_0}{AA_i} = \dfrac{AA_0}{AA_1} + \dfrac{AA_0}{AA_2} + \cdots + \dfrac{AA_0}{AA_{2012}}$. So sánh $S,S^{2013}$.
\end{baitoan}

%------------------------------------------------------------------------------%

\section{Ray -- Tia}
\fbox{1} Hình gồm điểm $O$ \& 1 phần đường thẳng bị chia ra bởi điểm $O$ gọi là 1 \textit{tia} gốc $O$. Tia $Ox$ còn gọi là \textit{1 nửa đường thẳng gốc $O$}. Tia $Ox$ không bị giới hạn về phía $x$. \fbox{2} 2 tia chung gốc $Ox,Oy$ tạo thành đường thẳng $xy$ gọi là \textit{2 tia đối nhau}. Mỗi điểm trên đường thẳng là gốc chung của 2 tia đối nhau. Mỗi tia chỉ có 1 tia đối. \fbox{3} Cho 2 tia chung gốc $Ox,Oy$, có: hoặc đó là 2 tia đối nhau, hoặc là 2 tia trùng nhau, hoặc là 2 tia không đối nhau, không trùng nhau. \fbox{4} \textit{Về thứ tự của 3 điểm trên 1 đường thẳng}: Cho 3 điểm $A,B,C$ thẳng hàng, nếu 2 tia $AB,AC$ đối nhau thì điểm $A$ nằm giữa 2 điểm $B,C$. \fbox{5} \textit{Về sự xác định tia}: Nếu điểm $A$ nằm giữa 2 điểm $B,C$ thì 2 tia $AB,AC$ đối nhau, 2 tia $BA,BC$ trùng nhau, 2 tia $CA,CB$ trùng nhau. \fbox{6} Điểm $M$ thuộc tia $Ox$ thì 2 tia $OM,Ox$ trùng nhau. $M\in Ox\land M\ne O\Leftrightarrow OM\equiv Ox$. \fbox{7} 2 tia chung gốc \& có thêm 1 điểm chung thì trùng nhau.

\begin{baitoan}[\cite{Binh_boi_duong_Toan_6_tap_2}, VD1, p. 63]
	Trên đường thẳng xy lấy điểm O. Trên tia Ox lấy điểm A, trên tia Oy lấy điểm B. (a) Kể tên các tia trùng nhau gốc A (các tia này chỉ coi là 1). (b) Kể tên các tia đối nhau.
\end{baitoan}

\begin{baitoan}[\cite{Binh_boi_duong_Toan_6_tap_2}, VD2, p. 63]
	Lấy 3 điểm C,O,D theo thứ tự đó trên đường thẳng xy. Vẽ tia $Ot\not\subset xy$. Lấy E,F thuộc tia Ot. Đếm số tia \& liệt kê.
\end{baitoan}

\begin{baitoan}[\cite{Binh_boi_duong_Toan_6_tap_2}, VD3, p. 64]
	Cho 2 tia Ox,Oy đối nhau. (a) Nêu cách vẽ 2 điểm E,F sao cho tia OE trùng với tia Ox, tia OF trùng với tia Oy. (b) Điểm nào nằm giữa 2 điểm khác?
\end{baitoan}

\begin{baitoan}[\cite{Binh_boi_duong_Toan_6_tap_2}, VD4, p. 64]
	Cho điểm O nằm giữa 2 điểm A,B. Vẽ điểm C nằm giữa 2 điểm A,O. O nằm giữa 2 điểm nào?
\end{baitoan}

\begin{baitoan}[\cite{Binh_boi_duong_Toan_6_tap_2}, VD5, p. 64]
	Cho điểm O nằm giữa 2 điểm A,B. Điểm M nằm giữa O,A, điểm N nằm giữa O,B. O nằm giữa 2 điểm nào?
\end{baitoan}

\begin{baitoan}[\cite{Binh_boi_duong_Toan_6_tap_2}, 11.1., p. 64]
	Trên đường thẳng xy lấy 2 điểm M,N, N thuộc tia My. Xác định vị trí của điểm O sao cho: (a) 2 tia OM,ON đối nhau. (b) 2 tia OM,ON trùng nhau.
\end{baitoan}

\begin{baitoan}[\cite{Binh_boi_duong_Toan_6_tap_2}, 11.3., p. 64]
	Trên đường thẳng xy lấy 2 điểm A,B, B thuộc tia Ay. Lấy điểm O nằm ngoài xy. 1 điểm C di động trên xy. Vẽ tia OC. Xác định vị trí của C để: (a) Tia OC không cắt tia By. (b) Tia OC không cắt 2 tia Ax,By. (c) Tia OC cắt cả 2 tia Bx,By.
\end{baitoan}

\begin{baitoan}[\cite{Binh_boi_duong_Toan_6_tap_2}, 11.4., p. 64]
	Cho tia Ox \& 3 điểm A,B,C sao cho 2 tia OA,Ox trùng nhau, 2 tia OB,OC đều là tia đối của tia Ox. (a) Chứng minh O,A,B,C thẳng hàng. (b) O nằm giữa 2 điểm nào?
\end{baitoan}

\begin{baitoan}[\cite{Binh_boi_duong_Toan_6_tap_2}, 11.5., p. 65]
	Cho biết 2 tia NM,NP đối nhau, 2 tia PN,PQ đối nhau. Chứng minh: (a) M,N,P,Q thẳng hàng. (b) P nằm giữa 2 điểm M,Q.
\end{baitoan}

\begin{baitoan}[\cite{Binh_boi_duong_Toan_6_tap_2}, 11.6., p. 65]
	Cho đường thẳng xy \& điểm $O\notin xy$. Lấy $n\in\mathbb{N}^\star$ điểm $A_1,A_2,\ldots,A_n$ trên xy. Vẽ các tia gốc O lần lượt đi qua $A_1,A_2,\ldots,A_n$. Có tất cả $40$ tia. Tính n.
\end{baitoan}

\begin{baitoan}[\cite{Binh_Toan_6_tap_2}, VD3, p. 66]
	Cho 3 điểm $A,B,C$ trong đó 2 tia $BA,BC$ đối nhau. Trong 3 điểm $A,B,C$ điểm nào nằm giữa 2 điểm còn lại?
\end{baitoan}

\begin{baitoan}[\cite{Binh_Toan_6_tap_2}, VD4, p. 66]
	Điểm B nằm giữa 2 điểm $A,C$. Tìm các tia đối nhau, trùng nhau.
\end{baitoan}

\begin{baitoan}[\cite{Binh_Toan_6_tap_2}, VD5, p. 66]
	Cho 2 đoạn thẳng $AB,CD$ cắt nhau tại điểm O nằm giữa 2 đầu của mỗi đoạn thẳng. (a) Kể tên các đoạn thẳng. (b) Điểm O là điểm chung của 2 đoạn thẳng nào?
\end{baitoan}
\noindent\cite[VD6, p. 66, 14., p. 68]{Binh_Toan_6_tap_2}.

\begin{baitoan}[\cite{Binh_Toan_6_tap_2}, 7., p. 67]
	O là 1 điểm của đường thẳng xy. Vẽ điểm A thuộc tia Ox, vẽ 2 điểm $B,C$ thuộc tia Oy sao cho C nằm giữa $B,O$. (a) Đếm số tia, số đoạn thẳng. (b) Kể tên các cặp tia đối nhau.
\end{baitoan}

\begin{baitoan}[\cite{Binh_Toan_6_tap_2}, 8., p. 67]
	Cho 5 điểm $A,B,C,M,N$ thỏa điểm C nằm giữa $A,B$, điểm M nằm giữa $A,C$, điểm N nằm giữa $B,C$. (a) Tia $CM,CN$ trùng với tia nào? (b) Vì sao điểm C nằm giữa $M,N$?
\end{baitoan}

\begin{baitoan}[\cite{Binh_Toan_6_tap_2}, 9., p. 67]
	Cho điểm B nằm giữa 2 điểm $A,C$, điểm C nằm giữa 2 điểm $B,D$. Vì sao điểm B nằm giữa $A,D$?
\end{baitoan}

\begin{baitoan}[\cite{Binh_Toan_6_tap_2}, 10., p. 67]
	Cho điểm B nằm giữa 2 điểm $A,C$, điểm D nằm giữa 2 điểm $B,C$. Điểm D có nằm giữa $A,B$ không?
\end{baitoan}

\begin{baitoan}[\cite{Binh_Toan_6_tap_2}, 11., p. 67]
	Cho điểm B nằm giữa 2 điểm $A,C$, điểm D thuộc tia BC \& không trùng B. Điểm B có nằm giữa $A,D$ không?
\end{baitoan}

\begin{baitoan}[\cite{Binh_Toan_6_tap_2}, 12., p. 67]
	Cho 3 điểm $A,B,C$ không thẳng hàng. Vẽ đường thẳng a không đi qua $A,B,C$ sao cho đường thẳng a: (a) Cắt 2 đoạn thẳng $AB,AC$. (b) Không cắt mỗi đoạn thẳng $AB,BC,CA$.
\end{baitoan}

\begin{baitoan}[\cite{Binh_Toan_6_tap_2}, 13., p. 67]
	(a) Vẽ 6 đoạn thẳng sao cho mỗi đoạn thẳng cắt đúng 3 đoạn thẳng khác. (b) Vẽ 8 đoạn thẳng sao cho mỗi đoạn thẳng cắt đúng 3 đoạn thẳng khác.
\end{baitoan}

\begin{baitoan}[\cite{TLCT_THCS_Toan_6_hinh_hoc}, VD1.5, p. 10]
	Cho điểm $O$ thuộc đường thẳng $xx'$. Lấy 2 điểm $A,B$ thuộc tia Ox sao cho A nằm giữa $B,O$. Đếm số tia. Đếm số cặp tia đối nhau.
\end{baitoan}

\begin{baitoan}[\cite{TLCT_THCS_Toan_6_hinh_hoc}, VD1.6, p. 11]
	Cho 3 điểm $A,B,C$ không thẳng hàng. Đặt tên đường thẳng BC là $xx'$, đường thẳng CA là $yy'$ \& đường thẳng AB là $zz'$. Liệt kê các cặp tia đối nhau, trùng nhau.
\end{baitoan}

\begin{baitoan}[\cite{TLCT_THCS_Toan_6_hinh_hoc}, VD1.7, p. 11]
	Cho 3 điểm $A,B,C$. (a) Khi nào 2 tia $BA,BC$ đối nhau? (b) Khi nào 2 tia $CA,CB$ trùng nhau? (c) Khi nào 2 tia $AB,AC$ không là 2 tia đối nhau \& cùng không là 2 tia trùng nhau?
\end{baitoan}

\begin{baitoan}[\cite{TLCT_THCS_Toan_6_hinh_hoc}, VD1.8, p. 12]
	Cho điểm B nằm giữa 2 điểm $A,C$. Điểm C nằm giữa 2 điểm $B,D$. C có nằm giữa $A,D$ không?
\end{baitoan}

\begin{baitoan}[\cite{TLCT_THCS_Toan_6_hinh_hoc}, VD1.9, p. 12]
	Cho điểm B nằm giữa 2 điểm $A,D$ \& điểm C nằm giữa 2 điểm $B,D$. C có nằm giữa $A,B$ không?
\end{baitoan}

\begin{baitoan}[\cite{TLCT_THCS_Toan_6_hinh_hoc}, VD1.10, p. 12]
	Cho điểm A nằm giữa 2 điểm $B,C$. Biết M nằm giữa $A,B$, N nằm giữa $A,C$. A có nằm giữa $M,N$ không?
\end{baitoan}

\subsection{Nửa mặt phẳng. Tia nằm giữa 2 tia}
\fbox{1} Hình gồm đường thẳng $a$ \& 1 phần mặt phẳng bị chia ra bởi $a$ gọi là \textit{1 nửa mặt phẳng bờ $a$}. 2 nửa mặt phẳng (I), (II) có chung bờ $a$ gọi là \textit{2 nửa mặt phẳng đối nhau}. \fbox{2} Mỗi đường thẳng $a$ chia mặt phẳng thành 2 phần: Nếu 2 điểm $A,B$ thuộc 1 phần thì đường thẳng $a$ không cắt đoạn thẳng $AB$. Nếu 2 điểm $A,B$ thuộc 2 phần khác nhau thì đường thẳng $a$ cắt đoạn thẳng $AB$. \fbox{3} {\sf Dấu hiệu nhận biết tia nằm giữa 2 tia}: Cho 3 tia $Ox,Oy,Oz$ chung gốc $O$. Nếu có điểm $A$ thuộc tia $Ox$, điểm $B$ thuộc tia $Oy$, $A\ne O,B\ne O$, mà tia $Oz$ cắt đoạn thẳng $AB$ tại điểm $I$ nằm giữa $A,B$ thì tia $Oz$ nằm giữa 2 tia $Ox,Oy$. \fbox{4} Nếu 2 tia $Ox,Oy$ đối nhau thì mọi tia $Oz$ khác $Ox,Oy$ đều nằm giữa 2 tia $Ox,Oy$. \fbox{5} {\sf Dấu hiệu đường thẳng cắt đoạn thẳng}: {\sc Định lý Pasch về tam giác}: Có 3 điểm $A,B,C$ không thẳng hàng \& không điểm nào thuộc đường thẳng $a$. Nếu đường thẳng $a$ cắt đoạn thẳng $BC$ thì đường thẳng $a$ hoặc cắt đoạn thẳng $AB$ hoặc cắt đoạn thẳng $AC$. \fbox{6} {\sf Dấu hiệu nhận biết tia nằm giữa 2 tia}: Cho 2 tia $Oy,Oz$ cùng thuộc nửa mặt phẳng bờ chứa tia $Ox$. Biết tia $Oy$ nằm giữa 2 tia $Ox,Oz$. Nếu tia $Om$ nằm giữa 2 tia $Oy,Oz$, hoặc tia $Om$ nằm giữa 2 tia $Ox,Oy$, thì tia $Om$ nằm giữa 2 tia $Ox,Oz$.

\begin{baitoan}[\cite{TLCT_THCS_Toan_6_hinh_hoc}, VD4.1, p. 40]
	Cho 4 điểm O,A,B,C trong đó A,B,C thẳng hàng. Biết A không nằm giữa B,C, B không nằm giữa A,C. (a) Trong 3 tia OA,OB,OC, tia nào nằm giữa 2 tia còn lại? (b) Vẽ tia Om là tia đối của tia OC, $M\ne O$. Trong 3 tia OA,OB,OM, tia nào nằm giữa 2 tia còn lại?
\end{baitoan}

%------------------------------------------------------------------------------%

\section{Angle -- Góc}
\fbox{1} \textit{Góc} là 1 hình gồm 2 tia chung gốc. Gốc chung của tia gọi là \textit{đỉnh} của góc, 2 tia gọi là 2 \textit{cạnh} của góc. Có nhiều cách ký hiệu 1 góc, e.g., góc $xOy$, góc $MON$, góc $O$, $\angle xOy,\widehat{xOy},\widehat{MON}$. \fbox{2} \textit{Góc bẹt} là góc có 2 cạnh là 2 tia đối nhau. \fbox{3} Để vẽ góc thì vẽ đỉnh \& vẽ 2 cạnh của góc. Dùng thước đo góc để đo độ lớn của 1 góc.

\begin{baitoan}[\cite{Binh_Toan_6_tap_2}, VD12, p. 72]
	Cho đường thẳng a \& 3 điểm $A,B,C$ sao cho a không cắt 2 đoạn thẳng $AB,AC$. a có cắt đoạn thẳng BC không?
\end{baitoan}

\begin{baitoan}[\cite{Binh_Toan_6_tap_2}, VD1, p. 73]
	Cho 5 tia chung gốc $OA,OB,OC,OD,OE$. Kể tên các góc.
\end{baitoan}

\begin{baitoan}[\cite{Binh_Toan_6_tap_2}, 32., p. 73]
	Cho 3 điểm $A,B,C$ không nằm trên đường thẳng a, trong đó a cắt 2 đoạn thẳng $AB,AC$. a có cắt đoạn thẳng BC không?
\end{baitoan}

\begin{baitoan}[\cite{Binh_Toan_6_tap_2}, 33., p. 73]
	Cho 3 điểm $A,B,C$ không nằm trên đường thẳng a sao cho a cắt đoạn thẳng AB, không cắt đoạn thẳng BC. a có cắt đoạn thẳng AC không?
\end{baitoan}

\begin{baitoan}[\cite{Binh_Toan_6_tap_2}, 34., p. 73]
	3 điểm $A,B,C$ không nằm trên đường thẳng a. Chứng minh hoặc đường thẳng a không cắt đoạn thẳng nào trong 3 đoạn thẳng $AB,BC,CA$, hoặc đường thẳng a chỉ cắt 2 trong 3 đoạn thẳng đó.
\end{baitoan}

\begin{baitoan}[\cite{Binh_Toan_6_tap_2}, 35., p. 73]
	4 điểm $A,B,C,D$ không nằm trên đường thẳng a. Chứng minh a hoặc không cắt, hoặc cắt 3, hoặc cắt 4 đoạn thẳng trong 6 đoạn thẳng $AB,AC,AD,BC,BD,CD$.
\end{baitoan}

\begin{baitoan}[\cite{Binh_Toan_6_tap_2}, 36., p. 73]
	Cho góc bẹt xOy, vẽ 3 tia $Oa,Ob,Oc$ thuộc cùng 1 nửa mặt phẳng có bờ xy. Đếm số góc \& kể tên chúng.
\end{baitoan}

\begin{baitoan}[\cite{TLCT_THCS_Toan_6_hinh_hoc}, VD4.2, p. 41]
	Đếm số góc tạo bởi: (a) $3$ tia chung gốc OA,OB,OC theo thứ tự đó. (b) $4$ tia tia chung gốc OA,OB,OC,OD theo thứ tự đó. (c) $5$ tia tia chung gốc OA,OB,OC,OD,OE theo thứ tự đó. (d) $100$ tia chung gốc. (e) $n\in\mathbb{N}^\star$ tia chung gốc.
\end{baitoan}

\begin{baitoan}[\cite{TLCT_THCS_Toan_6_hinh_hoc}, 4.1., p. 42]
	Lấy 2 điểm M,N trên tia Ox \& P thuộc Oy là tia đối của tia Ox, M,N,P khác O. (a) Liệt kê các tia có trên đường thẳng xy. (b) Đếm số đoạn thẳng trên đường thẳng xy \& liệt kê. (c) O thuộc các đoạn thẳng nào?
\end{baitoan}

\begin{baitoan}[\cite{TLCT_THCS_Toan_6_hinh_hoc}, 4.2., p. 42]
	Cho A,B,C không thuộc đường thẳng a. (a) a có thể cắt chỉ 1 trong 3 đoạn thẳng AB,BC,CA được không? (b) Biết a không cắt 2 đoạn thẳng AB,AC. a có cắt đoạn thẳng BC không?
\end{baitoan}

\begin{baitoan}[\cite{TLCT_THCS_Toan_6_hinh_hoc}, 4.3., pp. 42--43]
	Cho đường thẳng a. (a) Nếu 4 điểm A,B,C,D không nằm trên đường thẳng a, hỏi a có thể cắt bao nhiêu đoạn thẳng trong 6 đoạn thẳng AB,AC,AD,BC,BD,CD? (b) Nếu 4 điểm A,B,C,D không thuộc đường thẳng a mà đường thẳng a không cắt các đoạn thẳng AB,AC,CD thì a có cắt đoạn thẳng AD không?
\end{baitoan}

\begin{baitoan}[\cite{TLCT_THCS_Toan_6_hinh_hoc}, 4.4., p. 43]
	Cho M nằm giữa 2 điểm A,B. Lấy O không thuộc đường thẳng AB. Vẽ 3 tia OA,OB,OM. (a) Trong 3 tia OA,OB,OM thì tia nào không nằm giữa 2 tia còn lại? (b) Lấy N sao cho A nằm giữa O,N. Trong 2 tia OM,ON thì tia nào cắt đoạn thẳng BN?
\end{baitoan}

\begin{baitoan}[\cite{TLCT_THCS_Toan_6_hinh_hoc}, 4.5., p. 43]
	Cho 3 điểm A,B,C nằm ngoài đường thẳng a. (a) Biết a cắt 2 đoạn thẳng AB,AC. a có cắt đoạn thẳng BC không? (b) Chứng minh nếu đường thẳng a cắt đoạn thẳng AB \& không cắt đoạn thẳng AC thì a sẽ cắt đoạn thẳng BC.
\end{baitoan}

%------------------------------------------------------------------------------%

\subsection{Số đo góc}
\fbox{1} Mỗi góc có 1 số đo, số đo của mỗi góc $\le180^\circ$. Đơn vị của góc là độ, phút, giây. Các góc đặc biệt: góc nhọn $0^\circ < \alpha < 90^\circ$, góc vuông $\alpha = 90^\circ$, góc tù $90^\circ < \alpha < 180^\circ$, góc bẹt $\alpha = 180^\circ$. \fbox{2} So sánh 2 góc bằng cách so sánh các số đo của chúng. \fbox{3} Nếu tia $Oy$ nằm giữa 2 tia $Ox,Oz$ thì $\widehat{xOy} + \widehat{yOz} = \widehat{xOz}$. \fbox{4} {\sf Dấu hiệu về thứ tự của tia}: Nếu có đẳng thức về góc $\widehat{xOy} + \widehat{yOz} = \widehat{xOz}$ thì tia $Oy$ nằm giữa 2 tia $Ox,Oz$. \fbox{5} {\sf Dấu hiệu về thứ tự của tia}: Nếu 2 tia $Oy,Oz$ cùng thuộc nửa mặt phẳng bờ chứa tia $Ox$ \& có $\widehat{xOy} < \widehat{xOz}$ thì tia $Oy$ nằm giữa 2 tia $Ox,Oz$.

\begin{baitoan}[\cite{Binh_Toan_6_tap_2}, VD14, p. 74]
	Cho tia Oc nằm giữa 2 tia $Oa,Ob$ không đối nhau, tia Om nằm giữa tia $Oa,Oc$, tia On nằm giữa 2 tia $Ob,Oc$O. Tia Oc có nằm giữa 2 tia $Om,On$ không?
\end{baitoan}

\begin{baitoan}[\cite{Binh_Toan_6_tap_2}, VD15, p. 74]
	Chứng minh nếu 1 đường thẳng không đi qua các đỉnh của 1 tam giác \& cắt 1 cạnh của tam giác ấy thì nó cắt 1 \& chỉ 1 trong 2 cạnh còn lại.
\end{baitoan}

\begin{baitoan}[\cite{Binh_Toan_6_tap_2}, VD16, p. 74]
	Cho góc tù AOB. Vẽ 2 tia $OC,OD$ nằm trong góc AOB sao cho $AOC,BOD$ là 2 góc vuông. Chứng minh: (a) $\widehat{AOD} = \widehat{BOC}$. (b) $\widehat{AOB},\widehat{COD}$ bù nhau.
\end{baitoan}

\begin{baitoan}[\cite{Binh_Toan_6_tap_2}, 37., p. 75]
	Cho điểm B nằm giữa 2 điểm $A,C$, điểm D thuộc tia BC \& không trùng B, điểm O nằm ngoài đường thẳng AC. Trong 3 tia $OA,OB,OD$, tia nào nằm giữa 2 tia còn lại?
\end{baitoan}

\begin{baitoan}[\cite{Binh_Toan_6_tap_2}, 38., p. 75]
	Cho 2 tia $Oa,Ob$ không đối nhau. Trên tia Oa lấy $A\ne O$, trên tia Ob lấy $B\ne O$. 1 điểm C bất kỳ nằm giữa $A,B$. Vẽ điểm M sao cho điểm O nằm giữa $C,M$. (a) Chứng minh tia OC nằm giữa 2 tia $OA,OB$. (b) Trong 3 tia $OA,OB,OM$, có tia nào nằm giữa 2 tia còn lại không? Phát biểu thành 1 tính chất.
\end{baitoan}

\begin{baitoan}[\cite{Binh_Toan_6_tap_2}, 39., p. 75]
	Có thể khẳng định trong 3 tia chung gốc, bao giờ cũng có 1 tia nằm giữa 2 tia còn lại không?
\end{baitoan}

\begin{baitoan}[\cite{Binh_Toan_6_tap_2}, 40., p. 75]
	2 đường thẳng $AB,CD$ cắt nhau ở O. Biết $ \widehat{AOC} - \widehat{BOC} = 5^\circ$. Tính $\widehat{AOC},\widehat{BOC},\widehat{BOD},\widehat{AOD}$.
\end{baitoan}

\begin{baitoan}[\cite{Binh_Toan_6_tap_2}, 41., p. 75]
	Cho điểm B nằm giữa 2 điểm $A,D$, điểm O nằm ngoài đường thẳng AD. Biết $\widehat{AOD} = 80^\circ,\widehat{AOB} = 50^\circ$. Tính $\widehat{BOD}$.
\end{baitoan}

\begin{baitoan}[\cite{Binh_Toan_6_tap_2}, 42., p. 75]
	Cho $\widehat{xOy} = 90^\circ$, vẽ tia Oz thỏa $\widehat{yOz} = 30^\circ$. (a) Tia Oz có xác định duy nhất không? (b) Tính $\widehat{xOz}$ trong từng trường hợp.
\end{baitoan}

\begin{baitoan}[\cite{Binh_Toan_6_tap_2}, 43., p. 75]
	2 đường thẳng $AB,CD$ cắt nhau ở O. Biết $\widehat{AOC} = 70^\circ$. Tính $\widehat{AOD},\widehat{BOC},\widehat{BOD}$.
\end{baitoan}

\begin{baitoan}[\cite{Binh_Toan_6_tap_2}, 44., p. 75]
	Tính góc tạo bởi kim giờ \& kim phút của đồng hồ lúc: (a) {\rm2:10}. (b) {\rm10:42}.
\end{baitoan}

\begin{baitoan}[\cite{Binh_Toan_6_tap_2}, 45., p. 76]
	Cho $\Delta ABC$, D nằm giữa $A,C$, E nằm giữa $A,B$. Chứng minh đường thẳng BD cắt đoạn thẳng CE, đường thẳng CE cắt đoạn thẳng BD.
\end{baitoan}

\begin{baitoan}[\cite{Binh_Toan_6_tap_2}, 46., p. 76]
	Cho $\Delta ABC$. Chứng minh bao giờ cũng vẽ được 1 đường thẳng không đi qua 3 đỉnh của $\Delta ABC$ \& cắt cả 3 tia $AB,BC,CA$.
\end{baitoan}

\begin{baitoan}[\cite{Binh_Toan_6_tap_2}, 47., p. 76]
	Cho điểm O nằm trong $\Delta ABC$. Chứng minh: (a) Tia BO cắt đoạn thẳng AB tại 1 điểm D nằm giữa $A,C$. (b) Điểm O nằm giữa $B,D$. (c) Trong 3 tia $OA,OB,OC$, không có tia nào nằm giữa 2 tia còn lại.
\end{baitoan}

\begin{baitoan}[\cite{TLCT_THCS_Toan_6_hinh_hoc}, VD5.1, p. 44]
	Cho 3 tia Ox,Oy,Oz. Tính $\widehat{yOz}$ nếu: (a) Tia Ox nằm giữa 2 tia Oy,Oz sao cho $\widehat{xOy} = 80^\circ,\widehat{xOz} = 30^\circ$. (b) $\widehat{xOy} = \alpha,\widehat{xOz} = \beta,0^\circ < \alpha + \beta < 180^\circ,\alpha\ne\beta$.
\end{baitoan}

\begin{baitoan}[\cite{TLCT_THCS_Toan_6_hinh_hoc}, VD5.2, p. 45]
	Cho $\widehat{xOy} = 45^\circ,\widehat{yOz} = 80^\circ,\widehat{zOx} = 35^\circ$. Trong 3 tia Ox,Oy,Oz thì tia nào nằm giữa 2 tia còn lại?
\end{baitoan}

\subsection{2 góc bù nhau, phụ nhau}
\fbox{1} \textit{2 góc kề nhau} là 2 góc có 1 cạnh chung \& 2 cạnh còn lại nằm trên 2 nửa mặt phẳng đối nhau có bờ chứa cạnh chung. \fbox{2} \textit{2 góc phụ nhau} là 2 góc có tổng số đo bằng $90^\circ$. 2 góc bù nhau là 2 góc có tổng số đo bằng $180^\circ$. \fbox{3} \textit{2 góc kề bù} là 2 góc vừa kề nhau, vừa bù nhau. \fbox{4} Trên nửa mặt phẳng cho trước có bờ chứa tia $Ox$, bao giờ cũng vẽ được 1 \& chỉ 1 tia $Oy$ sao cho $\widehat{xOy} = \alpha$. \fbox{5} {\sf Dấu hiệu về tia đối nhau}: Nếu $\widehat{xOy},\widehat{yOz}$ kề nhau mà $\widehat{xOy} + \widehat{yOz} = 180^\circ$ thì tia $Ox,Oz$ đối nhau. \fbox{6} {\sf Dấu hiệu về thứ tự của tia}: Cho 2 tia đối nhau $Ox,Oy$ \& 2 điểm $A,B$ thuộc 2 nửa mặt phẳng đối nhau bờ $xy$. Biết $\widehat{AOx} = \alpha,\widehat{BOx} = \beta$. Nếu $\alpha + \beta > 180^\circ$ thì tia $Oy$ nằm giữa 2 tia $OA,OB$. Nếu $\alpha + \beta\le180^\circ$ thì tia $Ox$ nằm giữa 2 tia $OA,OB$.

\begin{baitoan}[\cite{TLCT_THCS_Toan_6_hinh_hoc}, VD5.3, p. 47]
	Cho góc bẹt xOy. Trên cùng nửa mặt phẳng bờ xy, vẽ 3 tia OA,OB,OC sao cho $\widehat{AOx} = 35^\circ,\widehat{BOx} = \frac{1}{2}\widehat{AOx},\widehat{COy} = \frac{1}{2}\widehat{AOy}$. Tính $\widehat{BOC}$.
\end{baitoan}

\begin{baitoan}[\cite{TLCT_THCS_Toan_6_hinh_hoc}, 5.1., p. 48]
	Cho 3 tia chung gốc Ox,Oy,Oz sao cho tia đối của tia Oz nằm giữa 2 tia Ox,Oy. Chứng minh $\widehat{xOy} + \widehat{yOz} + \widehat{zOx} = 360^\circ$.
\end{baitoan}

\begin{baitoan}[\cite{TLCT_THCS_Toan_6_hinh_hoc}, 5.2., p. 48]
	Cho $\widehat{xOy} = 180^\circ$. Vẽ 2 tia OM,ON cùng nằm trong 1 nửa mặt phẳng bờ xy sao cho $\widehat{MOx} = \alpha,\widehat{NOy} = \beta,0^\circ < \alpha + \beta < 180^\circ,\alpha\ne\beta$. Tính $\widehat{MON}$.
\end{baitoan}

\begin{baitoan}[\cite{TLCT_THCS_Toan_6_hinh_hoc}, 5.3., p. 48]
	Cho $\widehat{xOy} = 105^\circ,\widehat{xOz} = 125^\circ$. Tính $\widehat{yOz}$.
\end{baitoan}

\begin{baitoan}[\cite{TLCT_THCS_Toan_6_hinh_hoc}, 5.4., p. 48]
	Cho 3 tia chung gốc OA,OB,OC sao cho $\widehat{AOB} = 40^\circ,\widehat{AOC} = 35^\circ$. (a) Tính $\widehat{BOC}$. (b) Vẽ tia OD là tia đối của tia OA. Tính $\widehat{BOD},\widehat{COD}$.
\end{baitoan}

\begin{baitoan}[\cite{TLCT_THCS_Toan_6_hinh_hoc}, 5.5., p. 48]
	Cho 3 đường thẳng AM,BN,CD đồng quy tại O. (a) Liệt kê các góc kề với $\widehat{AOD}$. (b) Liệt kê các góc kề bù với $\widehat{AOD}$. (c) Tìm các góc bằng nhau.
\end{baitoan}

\begin{baitoan}[\cite{TLCT_THCS_Toan_6_hinh_hoc}, 5.6., p. 48]
	Cho 2 đường thẳng AB,CD cắt nhau tại O. Tính $\widehat{BOC},\widehat{BOD}$ nếu: (a) $\widehat{AOC} = \alpha\in(0^\circ,180^\circ)$. (b) Biết $\widehat{AOC} - \widehat{BOC} = \alpha\in(0^\circ,180^\circ)$.
\end{baitoan}

\begin{baitoan}[\cite{TLCT_THCS_Toan_6_hinh_hoc}, 5.7., p. 48]
	Cho 2 tia đối nhau OA,OB. Chứng minh 2 tia OM,ON đối nhau, biết 2 tia OM,ON nằm trong 2 nửa mặt phẳng đối nhau bờ AB mà $\widehat{AOM} = \widehat{BON} = \alpha\in(0^\circ,180^\circ)$.
\end{baitoan}

%------------------------------------------------------------------------------%

\subsection{2 góc kề nhau}

\begin{baitoan}[\cite{Binh_Toan_6_tap_2}, VD17, p. 76]
	Chứng minh: (a) Nếu 2 góc kề nhau có 2 cạnh ngoài là 2 tia đối nhau thì 2 góc đó bù nhau. (b) Nếu 2 góc kề nhau mà bù nhau thì 2 cạnh ngoài của chúng là 2 tia đối nhau.
\end{baitoan}

\begin{baitoan}[\cite{Binh_Toan_6_tap_2}, VD18, p. 77]
	Cho 3 tia chung gốc $OA,OB,OC$. Tính $\widehat{BOC}$ biết: (a) $\widehat{AOB} = 130^\circ,\widehat{AOC} = 30^\circ$. (b) $\widehat{AOB} = 130^\circ,\widehat{AOC} = 80^\circ$. (c) $\widehat{AOB} = \alpha,\widehat{AOC} = \beta$ với $\alpha,\beta\in(0^\circ,180^\circ)$.
\end{baitoan}

\begin{baitoan}[\cite{Binh_Toan_6_tap_2}, 48., p. 78]
	Cho 3 đường thẳng $AD,BE,CF$ đồng quy ở O, trong đó tia OB nằm giữa 2 tia $OA,OC$. Kể tên các góc kề với $\widehat{AOB}$.
\end{baitoan}

\begin{baitoan}[\cite{Binh_Toan_6_tap_2}, 49., p. 78]
	Cho 2 tia $Ox,Oy$ đối nhau. Trên 2 nửa mặt phẳng đối nhau có bờ chứa tia Ox, vẽ 2 tia $Om,On$ sao cho $\widehat{xOm} = 70^\circ,\widehat{yOn} = 70^\circ$. Chứng minh 2 tia $Om,On$ đối nhau.
\end{baitoan}

\begin{baitoan}[\cite{Binh_Toan_6_tap_2}, 50., p. 78]
	Cho $\widehat{xOy},\widehat{xOz}$ kề nhau. Tính $\widehat{yOz}$ biết: (a) $\widehat{xOy} = 40^\circ,\widehat{xOz} = 140^\circ$. (b) $\widehat{xOy} = 50^\circ,\widehat{xOz} = 70^\circ$. (c) $\widehat{xOy} = 120^\circ,\widehat{xOz} = 130^\circ$. (d) $\widehat{xOy} = \alpha,\widehat{xOz} = \beta$ với $\alpha,\beta\in(0^\circ,180^\circ)$.
\end{baitoan}

\begin{baitoan}[\cite{Binh_Toan_6_tap_2}, 51., p. 78]
	Cho 3 tia $Ox,Oy,Oz$. Tính $\widehat{yOz}$ biết: (a) $\widehat{xOy} = 60^\circ,\widehat{xOz} = 40^\circ$. (b) $\widehat{xOy} = 120^\circ,\widehat{xOz} = 100^\circ$. (c) $\widehat{xOy} = \alpha,\widehat{xOz} = \beta$ với $\alpha,\beta\in(0^\circ,180^\circ)$.
\end{baitoan}

\begin{baitoan}[\cite{Binh_Toan_6_tap_2}, 52., p. 78]
	Cho 4 tia $OA,OB,OC,OD$ tạo thành 4 góc $AOB,BOC,COD,DOA$ không có điểm trong chung. Tính số đo mỗi góc ấy biết: (a) $\widehat{BOC} = 3\widehat{AOB},\widehat{COD} = 5\widehat{AOB},\widehat{DOA} = 6\widehat{AOB}$. (b) $\widehat{BOC} = a\widehat{AOB},\widehat{COD} = b\widehat{AOB},\widehat{DOA} = c\widehat{AOB}$ với $a,b,c > 0$.
\end{baitoan}

\begin{baitoan}[\cite{Binh_Toan_6_tap_2}, 52., p. 78]
	Cho 3 góc $AOB,BOC,COD$ không có điểm trong chung \& đều có số đo bằng $\alpha$. Tính $\widehat{AOD}$.
\end{baitoan}

%------------------------------------------------------------------------------%

\subsection{Tia phân giác của 1 góc}
\fbox{1} $OM$ là \textit{tia phân giác} của $\widehat{AOB}\Leftrightarrow$ Tia $OM$ nằm giữa 2 tia $OA,OB$ \& $\widehat{AOM} = \widehat{BOM}$. Đường thẳng chứa tia phân giác của 1 góc là \textit{đường phân giác} của góc đó. Mỗi góc không là góc bẹt chỉ có 1 tia phân giác. \fbox{2} Nếu tia $OM$ là phân giác của $\widehat{AOB}$ thì: Tia $OM$ nằm giữa 2 tia $OA,OB$, $\widehat{AOM} = \widehat{BOM}$. \fbox{3} {\sf Tính chất phân giác của 1 góc}: (i) Nếu $OM$ là tia phân giác của $\widehat{AOB}$ thì $\widehat{AOM} = \widehat{BOM} = \frac{1}{2}\widehat{AOB}$. (ii) Cho 4 tia chung gốc $OA,OB,OC,OD$ xếp theo thứ tự đó mà $\widehat{AOD} < 180^\circ$. Nếu $\widehat{AOD},\widehat{BOC}$ có chung tia phân giác $OM$ thì $\widehat{AOB} = \widehat{COD},\widehat{AOC} = \widehat{BOD}$. \fbox{4} {\sf Dấu hiệu nhận biết tia phân giác của 1 góc}: (i) Nếu trên nửa mặt phẳng bờ chứa tia $OA$ vẽ 2 tia $OB,OM$ sao cho $\widehat{AOM} = \frac{1}{2}\widehat{AOB}$ thì $OM$ là tia phân giác của $\widehat{AOB}$. (ii) Cho 4 tia chung gốc $OA,OB,OC,OD$ xếp theo thứ tự đó sao cho $\widehat{AOD} < 180^\circ$. Nếu $\widehat{AOB} = \widehat{COD}$ thì $\widehat{AOD},\widehat{BOC}$ có chung tia phân giác.

\begin{baitoan}[\cite{TLCT_THCS_Toan_6_hinh_hoc}, VD6.1, p. 51]
	Cho 2 điểm A,B nằm trên 2 nửa mặt phẳng đối nhau bờ chứa tia Ox. (a) Biết $\widehat{AOx} = \widehat{BOx} = 30^\circ$. Chứng minh tia Ox là tia phân giác của $\widehat{AOB}$. (b) Cho $\widehat{AOx} = \widehat{BOx} = 130^\circ$. Tia Ox có là tia phân giác của $\widehat{AOB}$? (c) Cho $\widehat{AOx} = \widehat{BOx} = \alpha\in(0^\circ,180^\circ)$. Tìm điều kiện của $\alpha$ để tia Ox là tia phân giác của $\widehat{AOB}$.
\end{baitoan}

\begin{baitoan}[\cite{TLCT_THCS_Toan_6_hinh_hoc}, VD6.2, p. 52]
	Cho góc bẹt $\widehat{AOB}$. Trên cùng nửa mặt phẳng bờ AB vẽ 2 tia OM,OC sao cho $\widehat{AOM} = 50^\circ,\widehat{BOC} = 80^\circ$. Chứng minh tia OM là tia phân giác của $\widehat{AOC}$.
\end{baitoan}

\begin{baitoan}[\cite{TLCT_THCS_Toan_6_hinh_hoc}, VD6.3, p. 52]
	Cho điểm O thuộc đường thẳng xy. Trên cùng nửa mặt phẳng bờ xy, vẽ 4 tia OA,OB,OC,OD sao cho $\widehat{AOx} = 30^\circ,\widehat{BOx} = 60^\circ,\widehat{COx} = 90^\circ,\widehat{DOx} = 120^\circ$. Tìm các tia phân giác của các góc.
\end{baitoan}

\begin{baitoan}[\cite{TLCT_THCS_Toan_6_hinh_hoc}, VD6.4, p. 53]
	Cho $\widehat{AOx},\widehat{BOx}$ không kề nhau. (a) Vẽ hình biết $\widehat{AOx} = 38^\circ,\widehat{BOx} = 112^\circ$. Trong 3 tia OA,OB,Ox, tia nào nằm giữa 2 tia còn lại? (b) Tính $\widehat{AOB}$. (c) Vẽ tia phân giác OM của $\widehat{AOB}$. Tính $\widehat{MOx}$. (d) Cho $\widehat{AOx} = \alpha,\widehat{BOx} = \beta,0^\circ < \alpha + \beta < 180^\circ,\alpha\ne\beta$. Tìm điều kiện giữa $\alpha,\beta$ để tia OA nằm giữa 2 tia OB,Ox. Tính $\widehat{MOx}$ theo $\alpha,\beta$.
\end{baitoan}

\begin{baitoan}[\cite{TLCT_THCS_Toan_6_hinh_hoc}, VD6.5, p. 54]
	Cho $\widehat{AOx},\widehat{BOx}$ kề nhau. Biết $\widehat{AOx} = \alpha,\widehat{BOx} = \beta,0^\circ < \alpha + \beta\le180^\circ$. Vẽ tia phân giác OM của $\widehat{AOB}$. Tính $\widehat{MOx}$ theo $\alpha,\beta$.
\end{baitoan}

\begin{baitoan}[\cite{TLCT_THCS_Toan_6_hinh_hoc}, VD6.6, p. 55]
	Cho $\widehat{AOC},\widehat{BOC}$ kề nhau. (a) Vẽ hình biết $\widehat{AOC} = 54^\circ,\widehat{BOC} = 118^\circ$. (b) Vẽ tia phân giác OM của $\widehat{AOC}$ \& tia phân giác ON của $\widehat{BOC}$. Tính $\widehat{MON}$. (c) Giả sử $\widehat{AOC} = \alpha,\widehat{BOC} = \beta$. Tìm điều kiện của $\alpha,\beta$ để $\widehat{MON} = 45^\circ,\widehat{MON} = 90^\circ$, biết $0^\circ < \alpha + \beta\le180^\circ$.
\end{baitoan}

\begin{baitoan}[\cite{TLCT_THCS_Toan_6_hinh_hoc}, VD6.7, p. 56]
	Cho $\widehat{AOC},\widehat{BOC}$. Biết $\widehat{AOC} = \alpha,\widehat{BOC} = \beta,0^\circ < \alpha + \beta\le180^\circ,\alpha\ne\beta$. Vẽ 2 tia phân giác OM,ON của $\widehat{AOC},\widehat{BOC}$. Tính $\widehat{MON}$ theo $\alpha,\beta$.
\end{baitoan}

\begin{baitoan}[\cite{TLCT_THCS_Toan_6_hinh_hoc}, VD6.8, p. 57]
	Vẽ $\widehat{AOA'} = \widehat{BOB'} = 90^\circ$ kề với $\widehat{AOB}$. (a) Tính tổng số đo $\widehat{AOB},\widehat{A'OB'}$. (b) Chứng minh 2 tia phân giác $OM,OM'$ của $\widehat{AOB},\widehat{A'OB'}$ đối nhau.
\end{baitoan}

\begin{baitoan}[\cite{TLCT_THCS_Toan_6_hinh_hoc}, 6.1., p. 58]
	Cho $\widehat{xOy} = 70^\circ$. Vẽ 2 tia phân giác OM,ON lần lượt của $\widehat{xOy},\widehat{MOx}$. Tính $\widehat{MOy},\widehat{MON},\widehat{NOy}$.
\end{baitoan}

\begin{baitoan}[\cite{TLCT_THCS_Toan_6_hinh_hoc}, 6.2., p. 58]
	Cho 2 tia đối nhau OA,OB. Trên cùng nửa mặt phẳng bờ AB vẽ 2 tia OC,OD sao cho $\widehat{AOC} = 140^\circ,\widehat{BOD} = 80^\circ$. Tia OC có là tia phân giác của $\widehat{BOD}$?
\end{baitoan}

\begin{baitoan}[\cite{TLCT_THCS_Toan_6_hinh_hoc}, 6.3., pp. 58--59]
	Cho $\widehat{AOD} = 120^\circ$. Vẽ 2 tia OB,OC cùng thuộc 1 nửa mặt phẳng bờ chứa 2 tia OA,OD sao cho $\widehat{AOB} = \widehat{BOC} = \widehat{COD}$. (a) Tìm các tia phân giác của các góc. (b) Nếu OM là tia phân giác của $\widehat{AOD}$ thì OM có là tia phân giác của $\widehat{BOC}$?
\end{baitoan}

\begin{baitoan}[\cite{TLCT_THCS_Toan_6_hinh_hoc}, 6.4., p. 59]
	Cho $\widehat{AOB} = 80^\circ$. Vẽ tia OC nằm giữa 2 tia OA,OB sao cho $\widehat{AOC} = 30^\circ$. Vẽ tia OD nằm giữa 2 tia OA,OB sao cho $\widehat{COD} = 10^\circ$. OD có là tia phân giác của $\widehat{AOB}$?
\end{baitoan}

\begin{baitoan}[\cite{TLCT_THCS_Toan_6_hinh_hoc}, 6.5., p. 59]
	Trên đường thẳng xy lấy 1 điểm O. Vẽ $\widehat{AOx} = 90^\circ,\widehat{MOy} = 45^\circ$. (a) Nếu 2 tia OA,OM cùng nằm trong 1 nửa mặt phẳng bờ xy thì tia OM có là tia phân giác của $\widehat{AOy}$? (b) Chứng minh nếu 2 tia OA,OM nằm trong 2 nửa mặt phẳng đối nhau bờ xy thì $\widehat{AOM} = \widehat{MOx}$. Trong trường hợp này tia OM có là tia phân giác của $\widehat{AOx}$?
\end{baitoan}

\begin{baitoan}[\cite{TLCT_THCS_Toan_6_hinh_hoc}, 6.6., p. 59]
	Cho 3 góc chung đỉnh $\widehat{AOB} = \widehat{BOC} = \widehat{COA} = 120^\circ$. Chứng minh tia đối của tia OA là tia phân giác của $\widehat{BOC}$.
\end{baitoan}

\begin{baitoan}[\cite{TLCT_THCS_Toan_6_hinh_hoc}, 6.7., p. 59]
	Cho $\widehat{AOB} = 35^\circ$. Vẽ $\widehat{AOA'} = \widehat{BOB'} = 90^\circ$ cùng kề với $\widehat{AOB}$. Vẽ Ox,Oy là 2 tia phân giác của $\widehat{AOA'},\widehat{BOB'}$. (a) Chứng minh 2 tia Ox,Oy không đối nhau. (b) Tính $\widehat{AOB}$ để 2 tia Ox,Oy đối nhau.
\end{baitoan}

\begin{baitoan}[\cite{TLCT_THCS_Toan_6_hinh_hoc}, 6.8., p. 59]
	Cho $\widehat{AOA'} = \widehat{BOB'} = 90^\circ$ cùng không kề với $AOB$. Vẽ 2 tia $OM,OM'$ lần lượt là tia phân giác của $\widehat{AOB},\widehat{A'OB'}$. (a) 2 tia $OM,OM'$ đối nhau không? (b) Tính số đo góc hợp bởi các tia phân giác của $\widehat{AOB'},\widehat{BOA'}$.
\end{baitoan}

\begin{baitoan}[\cite{TLCT_THCS_Toan_6_hinh_hoc}, 6.9., p. 59]
	Cho $\widehat{BAC}$. Vẽ $\widehat{CAM}$ kề với $\widehat{BAC}$ sao cho $\widehat{CAM} = \widehat{BAM} = \alpha$. (a) Chứng minh $\alpha\ge90^\circ$. (b) Chứng minh tia AM là tia đối của tia phân giác của $\widehat{BAC}$.
\end{baitoan}

\begin{baitoan}
	Vẽ tia phân giác $OA_1$ của  $\widehat{AOB} = \alpha\in(0^\circ,180^\circ]$. Vẽ tia phân giác $OA_2$ của $\widehat{AOA_1}$, vẽ tia phân giác $OA_3$ của $\widehat{AOA_2},\ldots$ vẽ tia phân giác $OA_n$ của $\widehat{AOA_{n-1}}$, $\forall n\in\mathbb{N}^\star$. Tính $\widehat{AOA_n},\widehat{A_nOB},\widehat{A_mOA_n}$, $\forall m,n\in\mathbb{N}^\star$.
\end{baitoan}

\begin{baitoan}
	Cho $\widehat{AOC} = \alpha,\widehat{BOC} = \beta$. Vẽ $OA_1,OB_1$ lần lượt là 2 tia phân giác $\widehat{AOC},\widehat{BOC}$, vẽ $OA_n,OB_n$ lần lượt là 2 tia phân giác $\widehat{AOA_{n-1}},\widehat{BOB_{n-1}}$, $\forall n\in\mathbb{N}^\star$. Tính $\widehat{AOA_n},\widehat{A_nOB},\widehat{A_nOC},\widehat{AOB_n},\widehat{BOB_n},\widehat{B_nOC},\widehat{A_mOA_n},\widehat{A_mOB_n},\widehat{B_mOB_n}$, $\forall m,n\in\mathbb{N}^\star$.
\end{baitoan}

\begin{baitoan}
	Cho $\widehat{AOC} = \alpha,\widehat{BOC} = \beta$. Vẽ $OA_1,OB_1$ lần lượt là 2 tia phân giác $\widehat{AOC},\widehat{BOC}$, vẽ $OA_n,OB_n$ lần lượt là 2 tia phân giác $\widehat{COA_{n-1}},\widehat{COB_{n-1}}$, $\forall n\in\mathbb{N}^\star$. Tính $\widehat{AOA_n},\widehat{A_nOB},\widehat{A_nOC},\widehat{AOB_n},\widehat{BOB_n},\widehat{B_nOC},\widehat{A_mOA_n},\widehat{A_mOB_n},\widehat{B_mOB_n}$, $\forall m,n\in\mathbb{N}^\star$.
\end{baitoan}

%------------------------------------------------------------------------------%

\section{Circle. Triangle -- Đường Tròn. Tam Giác}
\fbox{1} \textit{Đường tròn} tâm $O$, bán kính $R$ là hình gồm các điểm cách $O$ 1 khoảng bằng $R$, ký hiệu $(O;R)\coloneqq S_R(O) = \partial B_R(O) = \{M\in\mathbb{R}^2|OM = R\}$. \fbox{2} \textit{Hình tròn} là hình gồm các điểm nằm trên đường tròn \& các điểm nằm bên trong đường tròn: $B_R(O)\coloneqq\{M\in\mathbb{R}^2|OM\le R\}$. \fbox{3} 2 điểm $A,B\in(O;R)$ chia đường tròn thành 2 cung tròn. Đoạn thẳng nối 2 điểm $A,B$ là \textit{dây cung}. Dây đi qua tâm là \textit{đường kính}. \fbox{4} Tam giác $ABC$ là hình gồm 3 đoạn thẳng $AB,BC,CA$ khi 3 điểm $A,B,C$ không thẳng hàng, i.e., $\Delta ABC\coloneqq AB\cup BC\cup CA$, $\forall A,B,C\in\mathbb{R}^2,C\notin AB$.

\begin{baitoan}[\cite{TLCT_THCS_Toan_6_hinh_hoc}, VD7.1, p. 60]
	Cho 5 điểm bất kỳ thuộc đường tròn $(O)$. Đếm số dây cung, số cung tạo bởi 2 trong 5 điểm đó.
\end{baitoan}

\begin{baitoan}[\cite{TLCT_THCS_Toan_6_hinh_hoc}, VD7.2, p. 61]
	Trên cạnh AC của $\Delta ABC$ lấy điểm M. Vẽ đoạn thẳng BM. Tính $\widehat{CBM}$, biết $\widehat{ABC} = 70^\circ,\widehat{ABM} = 30^\circ$.
\end{baitoan}

\begin{baitoan}[\cite{TLCT_THCS_Toan_6_hinh_hoc}, VD7.3, p. 61]
	Cho điểm M không thuộc đường thẳng xy. Lấy $A,B\in xy$ thì tồn tại 1 tam giác có đỉnh là điểm M \& 2 đỉnh còn lại là 2 điểm A,B. (a) Nếu có thêm 1 điểm thứ 3 cũng thuộc đường thẳng xy thì vẽ được bao nhiêu tam giác có đỉnh là M \& 2 đỉnh còn lại là 2 trong 3 điểm thuộc xy? (b) Nếu có $100$ điểm trên xy thì vẽ được bao nhiêu tam giác có đỉnh là M \& 2 đỉnh còn lại là 2 trong số $100$ điểm thuộc xy?
\end{baitoan}

\begin{baitoan}[\cite{TLCT_THCS_Toan_6_hinh_hoc}, VD7.4, pp. 61--62]
	(a) Vẽ $\Delta ABC$ có $\widehat{A} = 60^\circ,AC = 9$. (b) Trên tia AC lấy điểm M sao cho $AM = 2$, trên tia CA lấy điểm D sao cho $CD = 5$. Chứng minh M là trung điểm đoạn thẳng AD. (c) Vẽ 2 đoạn thẳng BM,BD. Đếm số tam giác \& liệt kê. (d) $\widehat{BAC},\widehat{BMC}$ là góc của các tam giác nào? (e) Tìm các góc kề bù với $\widehat{BMC},\widehat{BDC}$.
\end{baitoan}

\begin{baitoan}[\cite{TLCT_THCS_Toan_6_hinh_hoc}, VD7.5, p. 62]
	Cho $6$ điểm trên mặt phẳng sao cho không có $3$ điểm nào thẳng hàng. Cứ qua $2$ điểm vẽ 1 đoạn thẳng \& tô đoạn thẳng đó bằng màu xanh hoặc đỏ. Chứng minh tồn tại 1 tam giác có $3$ đỉnh là $3$ điểm trong số $6$ điểm đã cho \& có các cạnh cùng được tô màu xanh hoặc cùng màu đỏ.
\end{baitoan}

\begin{baitoan}[\cite{TLCT_THCS_Toan_6_hinh_hoc}, 7.1., p. 63]
	Cho $\Delta ABC$. Trên cạnh BC lấy 3 điểm D,I,K. Kẻ 3 đoạn thẳng BD,BI,BK. Liệt kê các tam giác.
\end{baitoan}

\begin{baitoan}[\cite{TLCT_THCS_Toan_6_hinh_hoc}, 7.2., p. 63]
	Cho $\Delta ABC$. Lấy D,E lần lượt thuộc 2 cạnh AC,AB. 2 đoạn thẳng BD,CE giao nhau tại O. Nối AO. Đếm số tam giác.
\end{baitoan}

\begin{baitoan}[\cite{TLCT_THCS_Toan_6_hinh_hoc}, 7.3., p. 63]
	Cho $\Delta ABC$. Đường thẳng a cắt cạnh AB tại D nằm giữa A,B, cắt cạnh AC tại E nằm giữa A,C. a có cắt cạnh BC không?
\end{baitoan}

\begin{baitoan}[\cite{TLCT_THCS_Toan_6_hinh_hoc}, 7.4., p. 63]
	Cho A,B,C,D,E nằm trên 1 đường tròn. Nối từng cặp điểm. Đếm số tam giác \& liệt kê.
\end{baitoan}

%------------------------------------------------------------------------------%

\section{Tính Số Điểm, Số Đường Thẳng, Số Đoạn Thẳng, Số Tam Giác, Số Góc}

\begin{baitoan}[\cite{Binh_Toan_6_tap_2}, VD19, p. 78]
	(a) Cho $100$ điểm trong đó không có 3 điểm nào thẳng hàng. Cứ qua 2 điểm vẽ 1 đường thẳng. Đếm số đoạn thẳng, đường thẳng.
\end{baitoan}

\begin{baitoan}[\cite{Binh_Toan_6_tap_2}, VD20, p. 79]
	Trên mặt phẳng có 4 đường thẳng. Số giao điểm của các đường thẳng có thể bằng bao nhiêu?
\end{baitoan}

\begin{baitoan}[\cite{Binh_Toan_6_tap_2}, VD21, p. 80]
	Cho $n\in\mathbb{N},n\ge2$. Nối từng cặp 2 điểm trong n điểm đó thành các đoạn thẳng. (a) Đếm số đoạn thẳng nếu trong n điểm đó không có 3 điểm nào thẳng hàng. (b) Đếm số đoạn thẳng nếu trong n điểm đó có đúng 3 điểm thẳng hàng. (c) Tính n biết có tất cả $1770$ đoạn thẳng.
\end{baitoan}

\begin{baitoan}[\cite{Binh_Toan_6_tap_2}, VD22, p. 80]
	Cho $\Delta ABC$, $D,E$ lần lượt nằm trong cạnh $AC,AB$, K là giao điểm của $BD,CE$. Kẽ đoạn thẳng DE. Đếm số tam giác. 
\end{baitoan}

\begin{baitoan}[\cite{Binh_Toan_6_tap_2}, VD23, p. 81]
	Cho $n\in\mathbb{N},n\ge2$. Vẽ $n$ tia chung gốc. Đếm số góc.
\end{baitoan}

\begin{baitoan}[\cite{Binh_Toan_6_tap_2}, 54., p. 81]
	Cho n điểm $A_1,A_2,\ldots,A_n$ trong đó không có 3 điểm nào thẳng hàng. Cứ qua 2 điểm, kẻ 1 đường thẳng. (a) Kể tên các đường thẳng nếu $n = 4$. (b) Tính số đường thẳng nếu $n = 20$. (c) Tính số đường thẳng theo n. (d) Tính n biết số đường thẳng kẻ được là $1128$. (e) Số đường thẳng có thể bằng $2004$ không?
\end{baitoan}

\begin{baitoan}[\cite{Binh_Toan_6_tap_2}, 55., p. 81]
	Cho $100$ điểm trong đó có đúng 4 điểm thẳng hàng, ngoài ra không có 3 điểm nào thẳng hàng. Cứ qua 2 điểm, vẽ 1 đường thẳng. Đếm số đường thẳng.
\end{baitoan}

\begin{baitoan}[\cite{Binh_Toan_6_tap_2}, 56., p. 81]
	Cho n điểm trong đó không có 3 điểm nào thẳng hàng. Cứ qua 2 điểm, vẽ 1 đường thẳng. Biết có tất cả $105$ đường thẳng. Tính n.
\end{baitoan}

\begin{baitoan}[\cite{Binh_Toan_6_tap_2}, 57., p. 81]
	Cho 4 điểm, bất cứ 2 điểm nào cũng có ít nhất 1 đường thẳng đi qua. Có thể có bao nhiêu đường thẳng?
\end{baitoan}

\begin{baitoan}[\cite{Binh_Toan_6_tap_2}, 58., p. 81]
	(a) Cho 3 đường thẳng cắt nhau đôi một. Có thể có bao nhiêu giao điểm? (b) Vẽ 3 đường thẳng sao cho số giao điểm (của 2 hoặc 3 đường thẳng) lần lượt là $0,1,2,3$.
\end{baitoan}

\begin{baitoan}[\cite{Binh_Toan_6_tap_2}, 59., p. 81]
	Cho $101$ đường thẳng trong đó bất cứ 2 đường thẳng nào cũng cắt nhau, không có 3 đường thẳng nào đồng quy. Tính số giao điểm của chúng.
\end{baitoan}

\begin{baitoan}[\cite{Binh_Toan_6_tap_2}, 60., p. 81]
	Cho n đường thẳng trong đó bất cứ 2 đường thẳng nào cũng cắt nhau, không có 3 đường thẳng nào đồng quy. Biết số giao điểm của đường thẳng đó là $780$. Tính n.
\end{baitoan}

\begin{baitoan}[\cite{Binh_Toan_6_tap_2}, 61., p. 81]
	Cho $10$ điểm. Nối từng cặp điểm trong $10$ điểm đó thành các đoạn thẳng. Tính số đoạn thẳng mà 2 mút thuộc tập $10$ điểm đã cho, nếu trong các điểm đã cho: (a) Không có 3 điểm nào thẳng hàng. (b) Có đúng 3 điểm thẳng hàng.
\end{baitoan}

\begin{baitoan}[\cite{Binh_Toan_6_tap_2}, 62., p. 82]
	Cho n điểm. Nối từng cặp điểm trong n điểm đó thành các đoạn thẳng. Tính n biết có tất cả $435$ đoạn thẳng.
\end{baitoan}

\begin{baitoan}[\cite{Binh_Toan_6_tap_2}, 63., p. 82]
	1 đường thẳng chia mặt phẳng thành 2 miền. (a) 2 đường thẳng có thể chia mặt phẳng thành mấy miền? (b) 3 đường thẳng có thể chia mặt phẳng thành mấy miền? (c) 4 đường thẳng chia mặt phẳng nhiều nhất thành mấy miền? (d) $n\in\mathbb{N}^\star$ đường thẳng chia mặt phẳng nhiều nhất thành mấy miền?
\end{baitoan}

\begin{baitoan}[\cite{Binh_Toan_6_tap_2}, 64., p. 82]
	Cho $10$ điểm thuộc đường thẳng a \& 1 điểm nằm ngoài a. Đếm số tam giác có 3 đỉnh trong 11 điểm đó.
\end{baitoan}

\begin{baitoan}[\cite{Binh_Toan_6_tap_2}, 65., p. 82]
	Cho $\widehat{xOy}\ne180^\circ$. Trên tia Ox lấy 3 điểm không trùng O là $A,B,C$. Trên tia Oy lấy 4 điểm không trùng O là $D,E,F,G$. Đếm số tam giác mà 3 đỉnh nằm trong 8 điểm $O,A,B,C,D,E,F,G$.
\end{baitoan}

\begin{baitoan}[\cite{Binh_Toan_6_tap_2}, 66., p. 82]
	(a) Cho n tia chung gốc tạo thành tất cả $190$ góc. Tính n. (b) Cho n tia chung gốc tạo thành tất cả m góc. Tính n theo m.
\end{baitoan}

\begin{baitoan}[Đếm số đoạn thẳng, đường thẳng tổng quát]
	Cho $n\in\mathbb{N},n\ge2$. (a) Cho $n$ điểm trong đó không có 3 điểm nào thẳng hàng, đếm số đoạn thẳng, đường thẳng đi qua 2 điểm trong chúng. (b) Cho $n$ điểm trong đó có đúng 1 bộ $m$ điểm thẳng hàng với nhau, đếm số đoạn thẳng, đường thẳng đi qua 2 điểm trong chúng. (c) Cho $n$ điểm trong đó có đúng $m$ bộ điểm thẳng hàng với nhau lần lượt nằm trên các đường thẳng $a_1,a_2,\ldots,a_m$. Biết đường thẳng $a_i$ có đúng $a_i$ điểm trong $n$ điểm đã cho thẳng hàng. Đếm số đoạn thẳng, đường thẳng đi qua 2 điểm trong chúng.
\end{baitoan}

%------------------------------------------------------------------------------%

\section{Đếm Số. Đếm Hình}

\begin{baitoan}[\cite{Binh_Toan_6_tap_2}, VD24, p. 82]
	Đếm số số tự nhiên có 3 chữ số, các chữ số khác nhau, lập từ 3 trong 5 chữ số $1,2,3,4,5$.
\end{baitoan}

\begin{baitoan}[\cite{Binh_Toan_6_tap_2}, VD25, p. 83]
	Đếm số cách sắp xếp nhất, nhì, ba trong: (a) 6 đội bóng thi đấu. (b) $n\in\mathbb{N}$ đội bóng thi đấu.
\end{baitoan}

\begin{baitoan}[\cite{Binh_Toan_6_tap_2}, VD26, p. 83]
	Đếm số cách gọi tên tam giác có 3 đỉnh là $A,B,C$.
\end{baitoan}

\begin{baitoan}[\cite{Binh_Toan_6_tap_2}, VD27, p. 83]
	Đếm số cách giao hoán các thừa số của tích $abcd$.
\end{baitoan}

\begin{baitoan}[\cite{Binh_Toan_6_tap_2}, VD28, p. 84]
	Đếm số cách sắp xếp 5 người ngồi: (a) Trên 1 ghế dài. (b) Xung quanh 1 bàn tròn.
\end{baitoan}

\begin{baitoan}[\cite{Binh_Toan_6_tap_2}, VD29, p. 84]
	Đếm số đoạn thẳng mà 2 đầu mút là $2$ trong $5$ điểm đã cho.
\end{baitoan}

\begin{baitoan}[\cite{Binh_Toan_6_tap_2}, VD30, p. 84]
	Cho $9$ điểm trên mặt phẳng, trong đó không có $3$ điểm nào thẳng hàng. Đếm số tam giác tạo thành.
\end{baitoan}

\begin{baitoan}[\cite{TLCT_THCS_Toan_6_hinh_hoc}, VD9.8, p. 79]
	Đếm số hình chữ nhật tạo bởi m đường thẳng đứng \& n đường nằm ngang đôi một cắt nhau với $m,n\in\mathbb{N}^\star$.
\end{baitoan}

\begin{baitoan}[\cite{Binh_Toan_6_tap_2}, VD31, p. 85]
	Trong số $4$ học sinh giỏi Văn \& $9$ học sinh giỏi Toán, lập ra 1 nhóm gồm $7$ học sinh, trong đó có ít nhất $2$ học sinh giỏi Văn. Đếm số cách lập nhóm.
\end{baitoan}

\begin{baitoan}[\cite{Binh_Toan_6_tap_2}, VD32, p. 85]
	(a) Đếm số cách xếp $2$ bi đen, $4$ bi trắng thành 1 dãy. (b) Đếm số cách xếp $2$ bi đen, $9$ bi trắng thành 1 dãy. (c) Đếm số cách xếp $m$ bi đen, $n$ bi trắng thành 1 dãy với $m,n\in\mathbb{N}$.
\end{baitoan}

\begin{baitoan}[\cite{Binh_Toan_6_tap_2}, VD33, p. 85]
	(a) Đếm số cách xếp $3$ bi đen, $4$ bi trắng thành 1 dãy. (b) Đếm số cách xếp $3$ bi đen, $9$ bi trắng thành 1 dãy. 
\end{baitoan}

\begin{baitoan}[\cite{Binh_Toan_6_tap_2}, VD34, p. 86]
	Đếm số số tự nhiên không quá 3 chữ số mà tổng các chữ số bằng $4$.
\end{baitoan}

\begin{baitoan}[\cite{Binh_Toan_6_tap_2}, VD35, p. 87]
	Đếm số số tự nhiên không quá 4 chữ số mà tổng các chữ số bằng $4$.
\end{baitoan}

\begin{baitoan}[\cite{Binh_Toan_6_tap_2}, VD36, p. 87]
	Đếm số số tự nhiên không quá 4 chữ số mà tổng các chữ số bằng $9$.
\end{baitoan}

\begin{baitoan}[\cite{Binh_Toan_6_tap_2}, 67., p. 87]
	Dùng 5 chữ số $1,2,3,4,5$ để: (a) Lập được bao nhiêu số tự nhiên có 4 chữ số, trong đó các chữ số khác nhau? Tính tổng các số được lập. (b) Lập được bao nhiêu số chẵnn, số lẻ có 5 chữ số khác nhau? (c) Lập được bao nhiêu số có 5 chữ số, trong đó 2 chữ số kề nhau phải khác nhau? (d) Lập được bao nhiêu số tự nhiên có 4 chữ số, các chữ số khác nhau, trong đó có 2 chữ số lẻ, 2 chữ số chẵn?
\end{baitoan}

\begin{baitoan}[\cite{Binh_Toan_6_tap_2}, 68., p. 87]
	Từ 5 chữ số $0,1,2,3,4$, có thể lập được bao nhiêu số tự nhiên: (a) Gồm 5 chữ số khác nhau? (b) Gồm 4 chữ số khác nhau. (c) Gồm 3 chữ số khác nhau. (d) Gồm 3 chữ số có thể giống nhau.
\end{baitoan}

\begin{baitoan}[\cite{Binh_Toan_6_tap_2}, 69., pp. 87--88]
	Từ 5 chữ số $0,1,3,5,6$, có thể lập được bao nhiêu số tự nhiên gồm 5 chữ số khác nhau thỏa 1 trong các điều kiện: (a) $\not{\divby}2$. (b) $\divby2$. (c) $\divby5$.
\end{baitoan}

\begin{baitoan}[\cite{Binh_Toan_6_tap_2}, 70., p. 88]
	(a) Dùng 3 chữ số $1,2,7$ có thể viết được bao nhiêu số tự nhiên có 5 chữ số sao cho 2 chữ số $2,7$ có mặt 1 lần, còn chữ số $1$ có mặt 3 lần? (b) Như (a) nếu thêm điều kiện các số phải đếm lớn hơn $20000$.
\end{baitoan}

\begin{baitoan}[\cite{Binh_Toan_6_tap_2}, 71., p. 88]
	Đếm số số tự nhiên có 4 chữ số lập bởi các số $1,2,3$ \& $\divby9$?
\end{baitoan}

\begin{baitoan}[\cite{Binh_Toan_6_tap_2}, 72., p. 88]
	Đếm số tự nhiên có $11$ chữ số, gồm $5$ chữ số $1$ \& $6$ chữ số $2$ sao cho đọc xuôi \& đọc ngược đều giống nhau.
\end{baitoan}

\begin{baitoan}[\cite{Binh_Toan_6_tap_2}, 73., p. 88]
	Đếm số số tự nhiên có không quá 3 chữ số mà tổng các chữ số bằng $9$.
\end{baitoan}

\begin{baitoan}[\cite{Binh_Toan_6_tap_2}, 74., p. 88]
	Đếm số số tự nhiên có 4 chữ số mà tích các chữ số bằng $24$.
\end{baitoan}

\begin{baitoan}[\cite{Binh_Toan_6_tap_2}, 75., p. 88]
	Đếm số số nguyên dương có 5 chữ số mà tổng các chữ số của nó bằng tích các chữ số đó.
\end{baitoan}

\begin{baitoan}[\cite{Binh_Toan_6_tap_2}, 76., p. 88]
	Cho $10$ điểm trên mặt phẳng, không có 3 điểm nào thẳng hàng. Cứ qua 2 điểm, kẻ 1 đường thẳng. Đếm số đường thẳng.
\end{baitoan}

\begin{baitoan}[\cite{Binh_Toan_6_tap_2}, 77., p. 88]
	Có $n\in\mathbb{N}^\star$ điểm trên mặt phẳng. Có tất cả $91$ đoạn thẳng nối 2 trong n điểm đó. Tính n.
\end{baitoan}

\begin{baitoan}[\cite{Binh_Toan_6_tap_2}, 78., p. 88]
	Cho $n\in\mathbb{N}^\star$ tia chung gốc tạo thành tất cả $153$ góc. Tính n.
\end{baitoan}

\begin{baitoan}[\cite{Binh_Toan_6_tap_2}, 79., p. 88]
	Đếm số cách gọi tên: (a) Hình vuông ABCD. (b) Đa giác lồi $A_1A_2\ldots A_n$ với $n\in\mathbb{N},n\ge3$.
\end{baitoan}

\begin{baitoan}[\cite{Binh_Toan_6_tap_2}, 80., p. 88]
	Cho hình vuông $4\times4$. Đếm số hình chữ nhật, số hình vuông.
\end{baitoan}

\begin{baitoan}[\cite{Binh_Toan_6_tap_2}, 81., p. 88]
	Có $12$ điểm trên mặt phẳng trong đó không có 3 điểm nào thẳng hàng. Đếm số tam giác tạo thành.
\end{baitoan}

\begin{baitoan}[\cite{Binh_Toan_6_tap_2}, 82., p. 88]
	Cho $\widehat{xAy}\ne180^\circ$. Trên tia Ax lấy $6$ điểm khác A, trên tia Ay lấy $5$ điểm khác A. Trong $12$ điểm này, kể cả điểm A, 2 điểm nào cũng được nối với nhau bởi 1 đoạn thẳng. Đếm số tam giác mà các đỉnh là $3$ trong $12$ điểm đó.
\end{baitoan}

\begin{baitoan}[\cite{Binh_Toan_6_tap_2}, 83., p. 89]
	Có $9$ đội bóng tham dự 1 giải bóng đá, mỗi đội phải đấu 2 trận với mỗi đội khác, ở sân nhà \& ở sân khách. Đếm số trận đấu.
\end{baitoan}

\begin{baitoan}[\cite{Binh_Toan_6_tap_2}, 84., p. 89]
	Có $2$ viên bi đỏ giống nhau, $8$ viên bi xanh giống nhau. Đếm số cách xếp thành 1 hàng gồm cả $10$ viên bi.
\end{baitoan}

\begin{baitoan}[\cite{Binh_Toan_6_tap_2}, 85., p. 89]
	1 ôtô có $8$ chỗ, kể cả chỗ của người lái xe. Đếm số cách xếp chỗ $8$ người trên xe, biết trong đó có $2$ người biết lái xe.
\end{baitoan}

\begin{baitoan}[\cite{Binh_Toan_6_tap_2}, 86., p. 89]
	Có 2 cặp bạn ngồi trên 1 ghế băng có 4 chỗ để chụp ảnh. Đếm số cách sắp xếp sao cho 2 người cùng cặp phải ngồi cạnh nhau.
\end{baitoan}

\begin{baitoan}[\cite{Binh_Toan_6_tap_2}, 87., p. 89]
	Đếm số cách sắp xếp 5 bạn $A,B,C,D,E$ ngồi trên 1 ghế dài sao cho $A,B$ ngồi cạnh nhau.
\end{baitoan}

\begin{baitoan}[\cite{Binh_Toan_6_tap_2}, 88., p. 89]
	Đếm số cách sắp xếp 5 bạn $A,B,C,D,E$ ngồi xung quanh 1 bàn tròn sao cho $A,B$ ngồi cạnh nhau.
\end{baitoan}

\begin{baitoan}[\cite{Binh_Toan_6_tap_2}, 89., p. 89]
	1 nhóm $5$ bạn gồm $3$ nam, $2$ nữ xếp thành 1 hàng ngang để chụp ảnh, sao cho $2$ bạn nữ không đứng cạnh nhau. Đếm số cách sắp xếp.
\end{baitoan}

\begin{baitoan}[\cite{Binh_Toan_6_tap_2}, 90., p. 89]
	Đếm số cách chọn $3$ tấm ảnh từ $6$ tấm ảnh khác nhau.
\end{baitoan}

\begin{baitoan}[\cite{Binh_Toan_6_tap_2}, 91., p. 89]
	Đếm số cách lập nhóm $3$ người từ 1 tổ $10$ người để làm nhiệm vụ trực nhật.
\end{baitoan}

\begin{baitoan}[\cite{Binh_Toan_6_tap_2}, 92., p. 89]
	1 tổ học sinh có $5$ nam, $3$ nữ. Đếm số cách lập nhóm $5$ người gồm $3$ nam, $2$ nữ.
\end{baitoan}

\begin{baitoan}[\cite{Binh_Toan_6_tap_2}, 93., p. 89]
	Đếm số cách chia $8$ chiếc kẹo cho $3$ người để ai cũng được nhận kẹo.
\end{baitoan}

%------------------------------------------------------------------------------%

\printbibliography[heading=bibintoc]
	
\end{document}