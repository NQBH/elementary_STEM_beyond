\documentclass{article}
\usepackage[backend=biber,natbib=true,style=alphabetic,maxbibnames=50]{biblatex}
\addbibresource{/home/nqbh/reference/bib.bib}
\usepackage[utf8]{vietnam}
\usepackage{tocloft}
\renewcommand{\cftsecleader}{\cftdotfill{\cftdotsep}}
\usepackage[colorlinks=true,linkcolor=blue,urlcolor=red,citecolor=magenta]{hyperref}
\usepackage{amsmath,amssymb,amsthm,float,graphicx,mathtools,tikz}
\usetikzlibrary{angles,calc,intersections,matrix,patterns,quotes,shadings}
\allowdisplaybreaks
\newtheorem{assumption}{Assumption}
\newtheorem{baitoan}{}
\newtheorem{cauhoi}{Câu hỏi}
\newtheorem{conjecture}{Conjecture}
\newtheorem{corollary}{Corollary}
\newtheorem{dangtoan}{Dạng toán}
\newtheorem{definition}{Definition}
\newtheorem{dinhly}{Định lý}
\newtheorem{dinhnghia}{Định nghĩa}
\newtheorem{example}{Example}
\newtheorem{ghichu}{Ghi chú}
\newtheorem{hequa}{Hệ quả}
\newtheorem{hypothesis}{Hypothesis}
\newtheorem{lemma}{Lemma}
\newtheorem{luuy}{Lưu ý}
\newtheorem{nhanxet}{Nhận xét}
\newtheorem{notation}{Notation}
\newtheorem{note}{Note}
\newtheorem{principle}{Principle}
\newtheorem{problem}{Problem}
\newtheorem{proposition}{Proposition}
\newtheorem{question}{Question}
\newtheorem{remark}{Remark}
\newtheorem{theorem}{Theorem}
\newtheorem{vidu}{Ví dụ}
\usepackage[left=1cm,right=1cm,top=5mm,bottom=5mm,footskip=4mm]{geometry}
\def\labelitemii{$\circ$}
\DeclareRobustCommand{\divby}{%
	\mathrel{\vbox{\baselineskip.65ex\lineskiplimit0pt\hbox{.}\hbox{.}\hbox{.}}}%
}

\title{Problem: Fraction \& Decimal -- Bài Tập: Phân Số \& Số Thập Phân}
\author{Nguyễn Quản Bá Hồng\footnote{Independent Researcher, Ben Tre City, Vietnam\\e-mail: \texttt{nguyenquanbahong@gmail.com}; website: \url{https://nqbh.github.io}.}}
\date{\today}

\begin{document}
\maketitle
\tableofcontents

%------------------------------------------------------------------------------%

\section{Basic Properties of Fraction. Simplify Fraction -- Phân Số. Tính Chất Cơ Bản của Phân Số. Rút Gọn Phân Số}

\begin{baitoan}[\cite{Binh_boi_duong_Toan_6_tap_2}, H1, p. 5]
	Các cách viết nào sau đây là phân số? $\dfrac{6}{7},\dfrac{-1.3}{5},\dfrac{12}{-8},\dfrac{-7}{0},\dfrac{2.34}{-6.5},\dfrac{-0}{3}$.
\end{baitoan}

\begin{baitoan}[\cite{Binh_boi_duong_Toan_6_tap_2}, H2, p. 5]
	Rút gọn $\dfrac{30 + 6}{30 + 12}$ về phân số tối giản.
\end{baitoan}

\begin{baitoan}[\cite{Binh_boi_duong_Toan_6_tap_2}, H3, p. 6]
	(a) Khi nào thì 1 phân số có thể viết được dưới dạng 1 số nguyên? (b) 2 phân số $\dfrac{a}{b}$ \& $\dfrac{-a}{-b}$ có bằng nhau không? (c) Nếu chia cả tử \& mẫu của 1 phân số cho cùng 1 số nguyên $n\ne0$ thì ta có được 1 phân số bằng nó không? (d) Khi chia cả tử \& mẫu của phân số $\dfrac{a}{b}$, $a,b > 0$, cho 1 ước chung của $a,b$ thì ta có thu được 1 phân số tối giản không? (e) Nếu $\dfrac{a}{b}$ là phân số tối giản thì mọi phân số bằng $\dfrac{a}{b}$ có dạng gì?
\end{baitoan}

\begin{baitoan}[\cite{Binh_boi_duong_Toan_6_tap_2}, VD1, p. 6]
	Cho tập hợp $A = \{-2,0,7\}$. Viết tất cả các phân số $\dfrac{a}{b}$ với $a,b\in A$.
\end{baitoan}

\begin{baitoan}[\cite{Binh_boi_duong_Toan_6_tap_2}, VD2, p. 6]
	Tìm $x,y\in\mathbb{Z}$ thỏa $\dfrac{x}{15} = \dfrac{24}{y} = \dfrac{-6}{5}$.
\end{baitoan}

\begin{baitoan}[\cite{Binh_boi_duong_Toan_6_tap_2}, VD3, p. 6]
	Rút gọn: (a) $\dfrac{-64}{96}$. (b) $\dfrac{131313}{252525}$. (c) $\dfrac{3510 - 135}{4680 - 180}$. (d) $\dfrac{2^4\cdot3^2}{6^2\cdot5}$.
\end{baitoan}

\begin{baitoan}[\cite{Binh_boi_duong_Toan_6_tap_2}, VD4, p. 7]
	Viết tất cả các phân số bằng $ \dfrac{30}{102}$ mà có tử \& mẫu là các số tự nhiên có 2 chữ số.
\end{baitoan}

\begin{baitoan}[\cite{Binh_boi_duong_Toan_6_tap_2}, VD5, p. 7]
	Tìm phân số bằng $\dfrac{35}{80}$ biết tổng của mẫu số \& 2 lần tử số bằng $210$.
\end{baitoan}

\begin{baitoan}[\cite{Binh_boi_duong_Toan_6_tap_2}, VD6, p. 8]
	Cho biểu thức $A = \dfrac{2n - 1}{n + 3}$ với $n\in\mathbb{Z}$. Tìm $n$ để $A\in\mathbb{Z}$.
\end{baitoan}

\begin{baitoan}[\cite{Binh_boi_duong_Toan_6_tap_2}, VD7, p. 8]
	Cho phân số $\dfrac{25}{49}$. Hỏi cần bớt ở tử \& mẫu của phân số đã cho cùng 1 số nào để được 1 phân số mới bằng $\dfrac{5}{13}$?
\end{baitoan}

\begin{baitoan}[\cite{Binh_boi_duong_Toan_6_tap_2}, 1.1., p. 8]
	Rút gọn các phân số trong trả lời của các câu hỏi sau thành phân số tối giản: (a) 1 mẫu Bắc Bộ bằng $\rm3600 m^2$. Hỏi 1 mẫu Bắc Bộ bằng mấy phần của 1 hecta? $\rm1\ ha = 10000\ m^2$. (b) Mỗi khoảng thời gian sau bằng bao nhiêu phần của 1 giờ: $15$ phút, $30$ phút, $45$ phút, $90$ phút, $4500$ giây? (c) Pound là đơn vị đo khối lượng được dùng phổ biến ở nước Anh \& 1 số nước khác. Biết $100$ pound $= 45$ {\rm kg}, hỏi 1 pound bằng mấy phần của {\rm1 kg}? (d) Inch là 1 trong các đơn vị đo chiều dài phổ biến trên thế giới. Biết $\rm1\ in = 2.54\ cm$, hỏi {\rm1 cm} bằng mấy phần của {\rm1 inch}?
\end{baitoan}

\begin{baitoan}[\cite{Binh_boi_duong_Toan_6_tap_2}, 1.2., p. 8]
	Tìm $x\in\mathbb{Z}$ thỏa: (a) $\dfrac{x - 2}{15} = \dfrac{9}{5}$. (b) $\dfrac{2 - x}{16} = \dfrac{-4}{x - 2}$.
\end{baitoan}

\begin{baitoan}[\cite{Binh_boi_duong_Toan_6_tap_2}, 1.3., p. 9]
	Tìm $x,y\in\mathbb{Z}$ thỏa: (a) $ \dfrac{x}{7} = \dfrac{5}{y}$ \& $x > y$. (b) $\dfrac{2}{x} = \dfrac{y}{-7}$ \& $x > 0$.
\end{baitoan}

\begin{baitoan}[\cite{Binh_boi_duong_Toan_6_tap_2}, 1.4., p. 9]
	Cho phân số $\dfrac{7}{11}$. Cần cộng vào tử số \& mẫu số của phân số đã cho với cùng 1 số nào để được phân số mới bằng $\dfrac{3}{4}$?
\end{baitoan}

\begin{baitoan}[\cite{Binh_boi_duong_Toan_6_tap_2}, 1.5., p. 9]
	Cho phân số $\dfrac{89}{143}$. Tìm 1 số tự nhiên để khi thêm vào tử số \& bớt đi ở mẫu số của phân số đã cho với cùng số đó thì ta được 1 phân số mới bằng $\dfrac{12}{17}$.
\end{baitoan}

\begin{baitoan}[\cite{Binh_boi_duong_Toan_6_tap_2}, 1.6., p. 9]
	Tìm phân số bằng phân số $\dfrac{147}{252}$ biết phân số đó có: (a) Tổng của tử \& mẫu bằng $228$. (b) Hiệu của tử \& mẫu bằng $40$. (c) Tích của tử \& mẫu bằng $756$.
\end{baitoan}

\begin{baitoan}[\cite{Binh_boi_duong_Toan_6_tap_2}, 1.7., p. 9]
	Tìm phân số có mẫu bằng $7$ biết khi cộng tử với $12$ \& nhân mẫu với $3$ thì ta được 1 phân số mới bằng phân số ban đầu.
\end{baitoan}

\begin{baitoan}[\cite{Binh_boi_duong_Toan_6_tap_2}, 1.8., p. 9]
	Giải thích tại sao $\forall n\in\mathbb{N}^\star$, giá trị của 2 biểu thức sau là phân số tối giản: (a) $\dfrac{2n + 5}{3n + 7}$. (b) $\dfrac{6n - 14}{2n - 5}$.
\end{baitoan}

\begin{baitoan}[\cite{Binh_boi_duong_Toan_6_tap_2}, 1.9., p. 9]
	Tìm phân số $\dfrac{a}{b}$ bằng phân số $\dfrac{54}{126}$ biết: (a) $\mbox{\rm ƯCLN}(a,b) = 12$. (b) ${\rm BCNN}(a,b) = 105$.
\end{baitoan}

\begin{baitoan}[\cite{Binh_boi_duong_Toan_6_tap_2}, 1.10., p. 9]
	Tìm $n\in\mathbb{Z}$ để 3 biểu thức sau đồng thời có giá trị nguyên: $\dfrac{-8}{n},\dfrac{13}{n - 1},\dfrac{4}{n + 2}$.
\end{baitoan}

\begin{baitoan}[\cite{Binh_boi_duong_Toan_6_tap_2}, p. 10]
	Làm sao để cắt ra được 1 đoạn dây dài {\rm10 m} từ 1 sợi dây dài {\rm16 m} mà không dùng thước đo?
\end{baitoan}

\begin{baitoan}[\cite{TLCT_THCS_Toan_6_so_hoc}, VD8.1, p. 51]
	Quan sát dãy phân số $\dfrac{1}{2},\dfrac{5}{12},\dfrac{1}{3},\dfrac{1}{4},\dfrac{1}{6}$ để viết tiếp 1 phân số nữa theo quy luật của dãy.
\end{baitoan}

\begin{baitoan}[\cite{TLCT_THCS_Toan_6_so_hoc}, VD8.2, p. 52]
	Tìm $n\in\mathbb{Z}$ để cả 3 phân số $\dfrac{15}{n},\dfrac{12}{n + 2},\dfrac{6}{2n - 5}\in\mathbb{Z}$.
\end{baitoan}

\begin{baitoan}[\cite{TLCT_THCS_Toan_6_so_hoc}, VD8.3, p. 52]
	Cho phân số $\dfrac{1 + 2 + \cdots + 20}{6 + 7 + \cdots + 36}$. Xóa 1 số hạng ở tử \& 1 số hạng ở mẫu của phân số này để giá trị của phân số đó không đổi.
\end{baitoan}

\begin{baitoan}[\cite{TLCT_THCS_Toan_6_so_hoc}, VD8.4, p. 53]
	Tìm $a,b\in\mathbb{N}$ biết $\dfrac{a}{b} = \dfrac{132}{143},{\rm BCNN}(a,b) = 1092$.
\end{baitoan}

\begin{baitoan}[\cite{TLCT_THCS_Toan_6_so_hoc}, 8.1., p. 53]
	Tìm $x\in\mathbb{Z}$ thỏa: (a) $\dfrac{7}{x} = \dfrac{x}{28}$. (b) $\dfrac{10 + x}{17 + x} = \dfrac{3}{4}$. (c) $\dfrac{40 + x}{77 - x} = \dfrac{6}{7}$.
\end{baitoan}

\begin{baitoan}[\cite{TLCT_THCS_Toan_6_so_hoc}, 8.2., p. 53]
	Tìm $a,b\in\mathbb{Z}$ thỏa $a^3 + b^3 = 1216$ \& phân số $\dfrac{a}{b}$ rút gọn được thành $\dfrac{3}{5}$.
\end{baitoan}

\begin{baitoan}[\cite{TLCT_THCS_Toan_6_so_hoc}, 8.3., p. 53]
	Viết các phân số tối giản $\dfrac{a}{b}$  với $a,b\in\mathbb{Z}$ \& $ab = 100$.
\end{baitoan}

\begin{baitoan}[\cite{TLCT_THCS_Toan_6_so_hoc}, 8.4., p. 53]
	Rút gọn phân số: (a) $\dfrac{10\cdot11 + 50\cdot55 + 70\cdot77}{11\cdot12 + 55\cdot60 + 77\cdot84}$. (b) $\dfrac{1\cdot3\cdot5\cdots49}{26\cdot27\cdot28\cdots50}$.
\end{baitoan}

\begin{baitoan}[\cite{TLCT_THCS_Toan_6_so_hoc}, 8.5., p. 53]
	Tìm $n\in\mathbb{Z}$ thỏa: (a) $\dfrac{n + 3}{n - 2}$ là số nguyên âm. (b) $\dfrac{n + 7}{3n - 1}\in\mathbb{Z}$. (c) $\dfrac{3n + 2}{4n - 5}\in\mathbb{Z}$.
\end{baitoan}

\begin{baitoan}[\cite{TLCT_THCS_Toan_6_so_hoc}, 8.6., p. 54]
	Chứng minh phân số $\dfrac{n - 5}{3n - 14}$ tối giản $\forall n\in\mathbb{Z}$.
\end{baitoan}

\begin{baitoan}[\cite{TLCT_THCS_Toan_6_so_hoc}, 8.7., p. 54]
	Tìm $n\in\mathbb{Z}$ thỏa $\dfrac{2n - 1}{3n + 2}$ rút gọn được.
\end{baitoan}

\begin{baitoan}[\cite{TLCT_THCS_Toan_6_so_hoc}, 8.8., p. 54]
	Tìm $n\in\mathbb{N}$ nhỏ nhất để 5 phân số $\dfrac{n + 7}{3},\dfrac{n + 8}{4},\dfrac{n + 9}{5},\dfrac{n + 10}{6},\dfrac{n + 11}{7}$.
\end{baitoan}

\begin{baitoan}[\cite{TLCT_THCS_Toan_6_so_hoc}, 8.9., p. 54]
	Tìm $a,b\in\mathbb{Z}$ biết $\dfrac{a}{b} = \dfrac{49}{56},\mbox{\rm ƯCLN}(a,b) = 12$.
\end{baitoan}

\begin{baitoan}[\cite{TLCT_THCS_Toan_6_so_hoc}, 8.10., p. 54]
	Tìm $a,b,c,d\in\mathbb{N}$ nhỏ nhất thỏa $\dfrac{a}{b} = \dfrac{5}{14},\dfrac{b}{c} = \dfrac{21}{28},\dfrac{c}{d} = \dfrac{6}{11}$.
\end{baitoan}

%------------------------------------------------------------------------------%

\section{Quy Đồng Mẫu Nhiều Phân Số. So Sánh Phân Số}

\begin{baitoan}[\cite{Binh_boi_duong_Toan_6_tap_2}, H1, p. 12]
	{\rm Đ{\tt/}S?} (a) 1 cách để quy đồng mẫu 2 phân số $\dfrac{a}{b}$ \& $\dfrac{c}{d}$ là $\dfrac{a}{b} = \dfrac{ad}{bd}$, $\frac{c}{d} = \dfrac{cb}{db}$. (b) Phân số dương lớn hơn phân số âm. (c) Trong 2 phân số có cùng mẫu, phân số nào có tử lớn hơn thì phân số đó lớn hơn. (d) Trong 2 phân số có mẫu số dương \& tử bằng nhau, phân số nào có mẫu nhỏ hơn thì phân số đó lớn hơn. (e) $3\dfrac{7}{5}$ là 1 hỗn số. (f) Phân số $\dfrac{22}{5}$ viết dưới dạng hỗn số là $4\dfrac{2}{5}$.
\end{baitoan}

\begin{baitoan}[\cite{Binh_boi_duong_Toan_6_tap_2}, H2, p. 12]
	{\rm Đ{\tt/}S?} Thực hiện quy đồng mẫu 2 phân số $\dfrac{10}{12}$ \& $\dfrac{5}{8}$, Sơn \& Huy đã làm như sau: Sơn: $\dfrac{10}{12} = \dfrac{10\cdot8}{12\cdot8} = \dfrac{80}{96}$, $\dfrac{5}{8} = \dfrac{5\cdot12}{8\cdot12} = \dfrac{60}{96}$. Huy: $\dfrac{10}{12} = \dfrac{5}{6} = \dfrac{5\cdot4}{6\cdot4} = \dfrac{20}{24}$, $\dfrac{5}{8} = \dfrac{5\cdot3}{8\cdot3} = \dfrac{15}{24}$.
\end{baitoan}

\begin{baitoan}[\cite{Binh_boi_duong_Toan_6_tap_2}, H3, p. 12]
	{\rm Đ{\tt/}S?} Để so sánh 2 phân số $\dfrac{23}{-5}$ \&  $\dfrac{24}{-5}$, Hà đã giải thích như sau: $\dfrac{23}{-5}$ \&  $\dfrac{24}{-5}$ là 2 phân số co cùng mẫu \& $23 < 24$ nên $\dfrac{23}{-5} < \dfrac{24}{-5}$.
\end{baitoan}

\begin{baitoan}[\cite{Binh_boi_duong_Toan_6_tap_2}, VD1, p. 12]
	Quy đồng mẫu các phân số: (a) $\dfrac{-3}{16},\dfrac{5}{-24}$. (b) $\dfrac{3}{14},\dfrac{-5}{18},\dfrac{25}{-42}$. (c) $\dfrac{3}{16},\dfrac{5}{48},\dfrac{-7}{4}$. (d) $\dfrac{3}{7},\dfrac{-8}{5},\dfrac{5}{12}$. (e) $\dfrac{-15}{18},\dfrac{42}{-72},\dfrac{32}{120}$.
\end{baitoan}

\begin{baitoan}[\cite{Binh_boi_duong_Toan_6_tap_2}, VD2, p. 13]
	Sắp xếp các phân số sau theo thứ tự tăng dần: (a) $\dfrac{5}{24},\dfrac{11}{36},\dfrac{17}{60}$. (b) $\dfrac{23}{47},\dfrac{69}{85},\dfrac{92}{137}$. (c) $\dfrac{17}{60},\dfrac{16}{73}$.
\end{baitoan}

\begin{baitoan}[\cite{Binh_boi_duong_Toan_6_tap_2}, VD3, p. 14]
	(a) Viết 2 phân số sau dưới dạng hỗn số: $\dfrac{26}{5},\dfrac{153}{25}$. (b) Viết 2 hỗn số sau dưới dạng phân số: $3\dfrac{2}{7},8\dfrac{3}{5}$.
\end{baitoan}

\begin{baitoan}[\cite{Binh_boi_duong_Toan_6_tap_2}, VD4, p. 14]
	So sánh $\dfrac{62}{15},\dfrac{70}{17}$.
\end{baitoan}

\begin{baitoan}[\cite{Binh_boi_duong_Toan_6_tap_2}, VD5, p. 15]
	2 bạn Mai \& Đào đi xe đạp đến trường với cùng tốc độ. Mai đi hết $\dfrac{2}{3}$ giờ, Đào đi hết $\dfrac{3}{4}$ giờ. Hỏi nhà ai cách xa trường hơn?
\end{baitoan}

\begin{baitoan}[\cite{Binh_boi_duong_Toan_6_tap_2}, VD6, p. 15]
	Tìm các phân số có mẫu là $5$, lớn hơn $\dfrac{-2}{3}$ \& nhỏ hơn $\dfrac{1}{-6}$.
\end{baitoan}

\begin{baitoan}[\cite{Binh_boi_duong_Toan_6_tap_2}, 2.1., p. 15]
	Quy đồng mẫu các phân số: (a) $\dfrac{13}{30},\dfrac{-7}{120}$. (b) $\dfrac{36}{75},\dfrac{6}{11}$. (c) $\dfrac{3}{25},\dfrac{4}{5},\dfrac{-8}{75}$. (d) $\dfrac{-5}{18},\dfrac{17}{60},\dfrac{32}{-45}$.
\end{baitoan}

\begin{baitoan}[\cite{Binh_boi_duong_Toan_6_tap_2}, 2.2., p. 15]
	Sắp xếp các phân số sau theo thứ tự từ nhỏ đến lớn: (a) $\dfrac{6}{7},\dfrac{9}{10}$. (b) $\dfrac{10}{-21},\dfrac{-4}{7},\dfrac{7}{9}$. (c) $\dfrac{-5}{-28},\dfrac{6}{35},\dfrac{27}{-180}$. (d) $\dfrac{3}{5},\dfrac{-5}{3},\dfrac{-36}{-60},\dfrac{18}{21}$.
\end{baitoan}

\begin{baitoan}[\cite{Binh_boi_duong_Toan_6_tap_2}, 2.3., p. 15]
	2 bạn mai \& Đào vào hiệu sách chọn được 1 cuốn sách mà cả 2 cùng thích, mỗi bạn mua 1 quyển. Sau ngày nghỉ cuối tuần, Mai đã đọc được $\dfrac{7}{8}$ số trang còn Đào đã đọc được $\dfrac{4}{5}$ số trang của cuốn sách đó. Hỏi ai đã đọc được nhiều hơn?
\end{baitoan}

\begin{baitoan}[\cite{Binh_boi_duong_Toan_6_tap_2}, 2.4., p. 16]
	Sơ kết học kỳ I, lớp 6A có $\dfrac{3}{4}$ số học sinh là học sinh giỏi môn Toán, $\dfrac{3}{5}$ số học sinh là học sinh giỏi môn Ngữ Văn, $\dfrac{2}{3}$ số học sinh là học sinh giỏi môn Anh văn. Sắp xếp theo thứ tự các môn học này theo số lượng học sinh giỏi từ nhiều nhất đến ít nhất.
\end{baitoan}

\begin{baitoan}[\cite{Binh_boi_duong_Toan_6_tap_2}, 2.5., p. 16]
	Tìm 4 phân số lớn hơn $\dfrac{5}{12}$ \& nhỏ hơn $\dfrac{5}{8}$.
\end{baitoan}

\begin{baitoan}[\cite{Binh_boi_duong_Toan_6_tap_2}, 2.6., p. 16]
	Tìm các phân số thỏa mãn: (a) Có mẫu là $20$, lớn hơn $\dfrac{4}{13}$, \& nhỏ hơn $\dfrac{5}{13}$. (b) Có mẫu là $14$, lớn hơn $\dfrac{-2}{21}$, \& nhỏ hơn $\dfrac{2}{9}$.
\end{baitoan}

\begin{baitoan}[\cite{Binh_boi_duong_Toan_6_tap_2}, 2.7., p. 16]
	Tìm các số nguyên $n$ lớn hơn $\dfrac{283}{23}$ \& nhỏ hơn $\dfrac{467}{31}$.
\end{baitoan}

\begin{baitoan}[\cite{Binh_boi_duong_Toan_6_tap_2}, 2.8., p. 16]
	An \& Bình đạp xe với tốc độ không đổi trên cùng 1 quãng đường. An đi hết $36$ phút, Bình đi hết $44$ phút. (a) So sánh quãng đường mà An đi được trong $20$ phút với quãng đường mà Bình đi được trong $26$ phút. (b) Bình phải đi trong bao lâu để được quãng đường bằng quãng đường An đi được trong $18$ phút?
\end{baitoan}

\begin{baitoan}[\cite{Binh_boi_duong_Toan_6_tap_2}, 2.9., p. 16]
	Tìm $x\in\mathbb{Z}$ thỏa $\dfrac{-7}{12} < \dfrac{x - 1}{4} < \dfrac{2}{3}$.
\end{baitoan}

\begin{baitoan}[\cite{Binh_boi_duong_Toan_6_tap_2}, 2.10., p. 16]
	Tìm $x,y\in\mathbb{Z}$ thỏa: (a) $\dfrac{-1}{3} < \dfrac{x}{36} < \dfrac{y}{118} < \dfrac{-1}{4}$. (b) $\dfrac{1}{220} < \dfrac{x}{165} < \dfrac{y}{132} < \frac{1}{60}$.
\end{baitoan}

\begin{baitoan}[\cite{Binh_boi_duong_Toan_6_tap_2}, 2.11., p. 16]
	Cho $a,b,c\in\mathbb{N}^\star$. Chứng minh: (a) Nếu $\dfrac{a}{b} < 1$ thì $\dfrac{a}{b} < \dfrac{a + c}{b + c}$. (b) Nếu $\dfrac{a}{b} > 1$ thì $\dfrac{a}{b} > \dfrac{a + c}{b + c}$. Áp dụng: So sánh $\dfrac{17}{18}$ \& $\dfrac{26}{27}$.
\end{baitoan}

%------------------------------------------------------------------------------%

\section{Calculus of Fraction -- Tính Toán với Phân Số}

\begin{baitoan}[\cite{Binh_boi_duong_Toan_6_tap_2}, 2., p. 16]
	
\end{baitoan}

%------------------------------------------------------------------------------%

\section{Miscellaneous}

%------------------------------------------------------------------------------%

\printbibliography[heading=bibintoc]
	
\end{document}