\documentclass{article}
\usepackage[backend=biber,natbib=true,style=alphabetic,maxbibnames=50]{biblatex}
\addbibresource{/home/nqbh/reference/bib.bib}
\usepackage[utf8]{vietnam}
\usepackage{tocloft}
\renewcommand{\cftsecleader}{\cftdotfill{\cftdotsep}}
\usepackage[colorlinks=true,linkcolor=blue,urlcolor=red,citecolor=magenta]{hyperref}
\usepackage{amsmath,amssymb,amsthm,float,graphicx,mathtools,tikz}
\usetikzlibrary{angles,calc,intersections,matrix,patterns,quotes,shadings}
\allowdisplaybreaks
\newtheorem{assumption}{Assumption}
\newtheorem{baitoan}{}
\newtheorem{cauhoi}{Câu hỏi}
\newtheorem{conjecture}{Conjecture}
\newtheorem{corollary}{Corollary}
\newtheorem{dangtoan}{Dạng toán}
\newtheorem{definition}{Definition}
\newtheorem{dinhly}{Định lý}
\newtheorem{dinhnghia}{Định nghĩa}
\newtheorem{example}{Example}
\newtheorem{ghichu}{Ghi chú}
\newtheorem{hequa}{Hệ quả}
\newtheorem{hypothesis}{Hypothesis}
\newtheorem{lemma}{Lemma}
\newtheorem{luuy}{Lưu ý}
\newtheorem{nhanxet}{Nhận xét}
\newtheorem{notation}{Notation}
\newtheorem{note}{Note}
\newtheorem{principle}{Principle}
\newtheorem{problem}{Problem}
\newtheorem{proposition}{Proposition}
\newtheorem{question}{Question}
\newtheorem{remark}{Remark}
\newtheorem{theorem}{Theorem}
\newtheorem{vidu}{Ví dụ}
\usepackage[left=1cm,right=1cm,top=5mm,bottom=5mm,footskip=4mm]{geometry}
\def\labelitemii{$\circ$}
\DeclareRobustCommand{\divby}{%
	\mathrel{\vbox{\baselineskip.65ex\lineskiplimit0pt\hbox{.}\hbox{.}\hbox{.}}}%
}

\title{Problem: Fraction {\it\&} Decimal -- Bài Tập: Phân Số {\it\&} Số Thập Phân}
\author{Nguyễn Quản Bá Hồng\footnote{e-mail: \texttt{nguyenquanbahong@gmail.com}, website: \url{https://nqbh.github.io}, Ben Tre City, Vietnam.}}
\date{\today}

\begin{document}
\maketitle
\begin{abstract}
	Latest version:
	\begin{itemize}
		\item \textit{Problem: Fraction \& Decimal -- Bài Tập: Phân Số {\it\&} Số Thập Phân}.\\{\sc url}: \url{https://github.com/NQBH/elementary_STEM_beyond/blob/main/elementary_mathematics/grade_6/fraction_decimal/problem/NQBH_fraction_decimal_problem.pdf}.
		\item \textit{Problem \& Solution: Fraction \& Decimal -- Bài Tập \& Lời Giải: Phân Số {\it\&} Số Thập Phân}.\\{\sc url}: \url{https://github.com/NQBH/elementary_STEM_beyond/blob/main/elementary_mathematics/grade_6/fraction_decimal/solution/NQBH_fraction_decimal_solution.pdf}.
	\end{itemize}
\end{abstract}
\tableofcontents

%------------------------------------------------------------------------------%

\section{Basic Properties of Fraction. Simplify Fraction -- Phân Số. Tính Chất Cơ Bản của Phân Số. Rút Gọn Phân Số}
SGK \cite[Chap. V, \S1, pp. 25--30]{SGK_Toan_6_Canh_Dieu_tap_2}: LT1. LT2. LT3. LT4. LT5. 1. 2. 3. 4. 5. 6. 7. SBT \cite[Chap. V, \S1, pp. 29--32]{SBT_Toan_6_Canh_Dieu_tap_2}: 1. 2. 3. 4. 5. 6. 7. 8. 9. 10. 11. 12. 13. 14.

\begin{baitoan}[{\sf Program}: Irreducible fraction]
	Viết chương trình {\sf Pascal, Python, C{\tt/}C++} để kiểm tra 1 phân số có tối giản với mẫu dương hay chưa, nếu chưa thì tối giản phân số đó.
\end{baitoan}

\begin{itemize}
	\item Python script: \url{https://github.com/NQBH/elementary_STEM_beyond/blob/main/elementary_computer_science/Python/irreducible_fraction.py}.
\end{itemize}

\begin{baitoan}[{\sf Program}: Reduce fractions to a common denominator]
	Viết chương trình {\sf Pascal, Python, C{\tt/}C++} để quy đồng các phân số.
\end{baitoan}

\begin{itemize}
	\item Python script: \url{https://github.com/NQBH/elementary_STEM_beyond/blob/main/elementary_computer_science/Python/reduce_common_denominator.py}.
\end{itemize}

\begin{baitoan}[{\sf Program}: Compare fractions]
	Viết chương trình {\sf Pascal, Python, C{\tt/}C++} để so sánh các phân số.
\end{baitoan}

\begin{itemize}
	\item Python script: \url{https://github.com/NQBH/elementary_STEM_beyond/blob/main/elementary_computer_science/Python/compare_fraction.py}.
\end{itemize}

\begin{baitoan}[\cite{Binh_boi_duong_Toan_6_tap_2}, H1, p. 5]
	Cách viết nào là phân số? $\dfrac{6}{7},\dfrac{-1.3}{5},\dfrac{12}{-8},\dfrac{-7}{0},\dfrac{2.34}{-6.5},\dfrac{-0}{3}$.
\end{baitoan}

\begin{baitoan}[\cite{Binh_boi_duong_Toan_6_tap_2}, H2, p. 5]
	Rút gọn $\dfrac{30 + 6}{30 + 12}$ về phân số tối giản.
\end{baitoan}

\begin{baitoan}[\cite{Binh_boi_duong_Toan_6_tap_2}, H3, p. 6]
	(a) Khi nào thì 1 phân số có thể viết được dưới dạng 1 số nguyên? (b) 2 phân số $\dfrac{a}{b}$ \& $\dfrac{-a}{-b}$ có bằng nhau không? (c) Nếu chia cả tử \& mẫu của 1 phân số cho cùng 1 số nguyên $n\ne0$ thì ta có được 1 phân số bằng nó không? (d) Khi chia cả tử \& mẫu của phân số $\dfrac{a}{b}$, $a,b > 0$, cho 1 ước chung của $a,b$ thì ta có thu được 1 phân số tối giản không? (e) Nếu $\dfrac{a}{b}$ là phân số tối giản thì mọi phân số bằng $\dfrac{a}{b}$ có dạng gì?
\end{baitoan}

\begin{baitoan}[\cite{Binh_boi_duong_Toan_6_tap_2}, VD1, p. 6]
	Cho tập hợp $A = \{-2,0,7\}$. Viết tất cả các phân số $\dfrac{a}{b}$ với $a,b\in A$.
\end{baitoan}

\begin{baitoan}[\cite{Binh_boi_duong_Toan_6_tap_2}, VD2, p. 6]
	Tìm $x,y\in\mathbb{Z}$ thỏa $\dfrac{x}{15} = \dfrac{24}{y} = \dfrac{-6}{5}$.
\end{baitoan}

\begin{baitoan}[\cite{Binh_boi_duong_Toan_6_tap_2}, VD3, p. 6]
	Rút gọn: (a) $\dfrac{-64}{96}$. (b) $\dfrac{131313}{252525}$. (c) $\dfrac{3510 - 135}{4680 - 180}$. (d) $\dfrac{2^4\cdot3^2}{6^2\cdot5}$.
\end{baitoan}

\begin{baitoan}[\cite{Binh_boi_duong_Toan_6_tap_2}, VD4, p. 7]
	Viết tất cả các phân số bằng $ \dfrac{30}{102}$ mà có tử \& mẫu là các số tự nhiên có 2 chữ số.
\end{baitoan}

\begin{baitoan}[\cite{Binh_boi_duong_Toan_6_tap_2}, VD5, p. 7]
	Tìm phân số bằng $\dfrac{35}{80}$ biết tổng của mẫu số \& 2 lần tử số bằng $210$.
\end{baitoan}

\begin{baitoan}[\cite{Binh_boi_duong_Toan_6_tap_2}, VD6, p. 8]
	Cho biểu thức $A = \dfrac{2n - 1}{n + 3}$ với $n\in\mathbb{Z}$. Tìm $n$ để $A\in\mathbb{Z}$.
\end{baitoan}

\begin{baitoan}[\cite{Binh_boi_duong_Toan_6_tap_2}, VD7, p. 8]
	Cho $\dfrac{25}{49}$. Cần bớt ở tử \& mẫu của phân số đã cho cùng 1 số nào để được 1 phân số mới bằng $\dfrac{5}{13}$?
\end{baitoan}

\begin{baitoan}[\cite{Binh_boi_duong_Toan_6_tap_2}, 1.1., p. 8]
	Rút gọn các phân số trong trả lời của các câu hỏi sau thành phân số tối giản: (a) 1 mẫu Bắc Bộ bằng $\rm3600 m^2$. 1 mẫu Bắc Bộ bằng mấy phần của 1 hecta? $\rm1\ ha = 10000\ m^2$. (b) Mỗi khoảng thời gian sau bằng bao nhiêu phần của 1 giờ: $15$ phút, $30$ phút, $45$ phút, $90$ phút, $4500$ giây? (c) Pound là đơn vị đo khối lượng được dùng phổ biến ở nước Anh \& 1 số nước khác. Biết $100$ pound $= 45$ {\rm kg}, hỏi 1 pound bằng mấy phần của {\rm1 kg}? (d) Inch là 1 trong các đơn vị đo chiều dài phổ biến trên thế giới. Biết $\rm1\ in = 2.54\ cm$, hỏi {\rm1 cm} bằng mấy phần của {\rm1 inch}?
\end{baitoan}

\begin{baitoan}[\cite{Binh_boi_duong_Toan_6_tap_2}, 1.2., p. 8]
	Tìm $x\in\mathbb{Z}$ thỏa: (a) $\dfrac{x - 2}{15} = \dfrac{9}{5}$. (b) $\dfrac{2 - x}{16} = \dfrac{-4}{x - 2}$.
\end{baitoan}

\begin{baitoan}[\cite{Binh_boi_duong_Toan_6_tap_2}, 1.3., p. 9]
	Tìm $x,y\in\mathbb{Z}$ thỏa: (a) $ \dfrac{x}{7} = \dfrac{5}{y}$ \& $x > y$. (b) $\dfrac{2}{x} = \dfrac{y}{-7}$ \& $x > 0$.
\end{baitoan}

\begin{baitoan}
	Biện luận theo $a,b,c,d,e,f,g,h\in\mathbb{Q}$ để tìm $x,y\in\mathbb{Q}$ thỏa $\dfrac{ax + b}{c} = \dfrac{d}{ey + f} = \dfrac{g}{h}$.
\end{baitoan}

\begin{baitoan}[\cite{Binh_boi_duong_Toan_6_tap_2}, 1.4., p. 9]
	Cho phân số $\dfrac{7}{11}$. Cần cộng vào tử số \& mẫu số của phân số đã cho với cùng 1 số nào để được phân số mới bằng $\dfrac{3}{4}$?
\end{baitoan}

\begin{baitoan}[\cite{Binh_boi_duong_Toan_6_tap_2}, 1.5., p. 9]
	Cho phân số $\dfrac{89}{143}$. Tìm 1 số tự nhiên để khi thêm vào tử số \& bớt đi ở mẫu số của phân số đã cho với cùng số đó thì ta được 1 phân số mới bằng $\dfrac{12}{17}$.
\end{baitoan}

\begin{baitoan}[\cite{Binh_boi_duong_Toan_6_tap_2}, 1.6., p. 9]
	Tìm phân số bằng phân số $\dfrac{147}{252}$ biết phân số đó có: (a) Tổng của tử \& mẫu bằng $228$. (b) Hiệu của tử \& mẫu bằng $40$. (c) Tích của tử \& mẫu bằng $756$.
\end{baitoan}

\begin{baitoan}[\cite{Binh_boi_duong_Toan_6_tap_2}, 1.7., p. 9]
	Tìm phân số có mẫu bằng $7$ biết khi cộng tử với $12$ \& nhân mẫu với $3$ thì ta được 1 phân số mới bằng phân số ban đầu.
\end{baitoan}

\begin{baitoan}[\cite{Binh_boi_duong_Toan_6_tap_2}, 1.8., p. 9]
	Giải thích tại sao $\forall n\in\mathbb{N}^\star$, giá trị của 2 biểu thức sau là phân số tối giản: (a) $\dfrac{2n + 5}{3n + 7}$. (b) $\dfrac{6n - 14}{2n - 5}$.
\end{baitoan}

\begin{baitoan}[\cite{Binh_boi_duong_Toan_6_tap_2}, 1.9., p. 9]
	Tìm phân số $\dfrac{a}{b}$ bằng phân số $\dfrac{54}{126}$ biết: (a) $\mbox{\rm ƯCLN}(a,b) = 12$. (b) ${\rm BCNN}(a,b) = 105$.
\end{baitoan}

\begin{baitoan}[\cite{Binh_boi_duong_Toan_6_tap_2}, 1.10., p. 9]
	Tìm $n\in\mathbb{Z}$ để 3 biểu thức sau đồng thời có giá trị nguyên: $\dfrac{-8}{n},\dfrac{13}{n - 1},\dfrac{4}{n + 2}$.
\end{baitoan}

\begin{baitoan}[\cite{Binh_boi_duong_Toan_6_tap_2}, p. 10]
	Làm sao để cắt ra được 1 đoạn dây dài {\rm10 m} từ 1 sợi dây dài {\rm16 m} mà không dùng thước đo? Mở rộng bài toán.
\end{baitoan}

\begin{baitoan}[\cite{Tuyen_Toan_6}, VD49, p. 45]
	Cho $A = \{-5,0,9\}$. Viết tất cả các phân số $\dfrac{a}{b}$ với $a,b\in A$. Có bao nhiêu phân số thỏa mãn?
\end{baitoan}

\begin{baitoan}[\cite{Tuyen_Toan_6}, VD50, p. 46]
	Viết tập hợp $B$ các phân số bằng phân số $\dfrac{7}{-15}$ với mẫu dương có 2 chữ số.
\end{baitoan}

\begin{baitoan}
	Cho trước $a,b\in\mathbb{Z}$, $b\ne0$, \& $n\in\mathbb{N}^\star$. Viết tập hợp $B$ các phân số bằng phân số $\dfrac{a}{b}$ với mẫu dương có $n$ chữ số.
\end{baitoan}

\begin{baitoan}[\cite{Tuyen_Toan_6}, VD51, p. 46]
	Tìm phân số bằng phân số $\dfrac{32}{60}$, biết tổng của tử \& mẫu là $115$.
\end{baitoan}

\begin{baitoan}
	Cho trước $a,b,n\in\mathbb{Z}$, $b\ne0$. Tìm phân số bằng phân số $\dfrac{a}{b}$, biết tổng của tử \& mẫu là $n$.
\end{baitoan}

\begin{baitoan}[\cite{Tuyen_Toan_6}, 236., p. 47]
	Trong các phân số sau, những phân số nào bằng nhau? $\dfrac{15}{60},\dfrac{-7}{5},\dfrac{6}{15},\dfrac{28}{-20},\dfrac{3}{12}$.
\end{baitoan}

\begin{baitoan}[\cite{Tuyen_Toan_6}, 237., p. 47]
	Cho $A = \dfrac{3n - 5}{n + 4}$. Tìm $n\in\mathbb{Z}$ để $A\in\mathbb{Z}$.
\end{baitoan}

\begin{baitoan}[\cite{Tuyen_Toan_6}, 238., p. 47]
	Tìm $n\in\mathbb{Z}$ để cho các phân số sau đồng thời có giá trị nguyên: $\dfrac{-12}{n},\dfrac{15}{n - 2},\dfrac{8}{n + 1}$.
\end{baitoan}

\begin{baitoan}[\cite{Tuyen_Toan_6}, 239., p. 47]
	Tìm $x\in\mathbb{Z}$ biết: (a) $\dfrac{x - 1}{9} = \dfrac{8}{3}$. (b) $\dfrac{-x}{4} = \dfrac{-9}{x}$. (c) $\dfrac{x}{4} = \dfrac{18}{x + 1}$.
\end{baitoan}

\begin{baitoan}[\cite{Tuyen_Toan_6}, 240., p. 47]
	Tìm $x,y\in\mathbb{Z}$ thỏa $\dfrac{x - 4}{y - 3} = \dfrac{4}{3}$ \& $x - y = 5$.
\end{baitoan}

\begin{baitoan}[\cite{Tuyen_Toan_6}, 241., p. 47]
	Viết dạng tổng quát các phân số bằng phân số $\dfrac{-12}{30}$.
\end{baitoan}

\begin{baitoan}[\cite{Tuyen_Toan_6}, 242., p. 47]
	Rút gọn phân số: (a) $\dfrac{990}{2610}$. (b) $\dfrac{374}{506}$. (c) $\dfrac{3600 - 75}{8400 - 175}$. (d) $\dfrac{9^{14}\cdot25^5\cdot8^7}{18^{12}\cdot625^3\cdot24^3}$.
\end{baitoan}

\begin{baitoan}[\cite{Tuyen_Toan_6}, 243., p. 47]
	Cho phân số $\dfrac{a}{b}$. Chứng minh: Nếu $\dfrac{a - x}{b - y} = \dfrac{a}{b}$ thì $\dfrac{x}{y} = \dfrac{a}{b}$.
\end{baitoan}

\begin{baitoan}[\cite{Tuyen_Toan_6}, 244., p. 47]
	Cho phân số $A = \dfrac{1 + 3 + 5 + \cdots + 19}{21 + 23 + 25 + \cdots + 39}$. (a) Rút gọn $A$. (b) Xóa 1 số hạng ở tử \& xóa 1 số hạng ở mẫu để được 1 phân số mới vẫn bằng $A$.
\end{baitoan}

\begin{baitoan}[\cite{Tuyen_Toan_6}, 245., p. 47]
	Rút gọn phân số $A = \dfrac{71\cdot52 + 53}{530\cdot71 - 180}$ mà không cần thực hiện các phép tính ở tử.
\end{baitoan}

\begin{baitoan}[\cite{Tuyen_Toan_6}, 246., p. 47]
	2 phân số sau có bằng nhau không? $\dfrac{\overline{abab}}{\overline{cdcd}},\dfrac{\overline{ababab}}{\overline{cdcdcd}}$.
\end{baitoan}

\begin{baitoan}[\cite{Tuyen_Toan_6}, 247., p. 47]
	Chứng minh: (a) $\dfrac{1\cdot3\cdot5\cdots39}{21\cdot22\cdot23\cdots40} = \dfrac{1}{2^{20}}$. (b) $\dfrac{1\cdot3\cdot5\cdots(2n - 1)}{(n + 1)(n + 2)(n + 3)\cdots2n} = \dfrac{1}{2^n}$ với $n\in\mathbb{N}^\star$.
\end{baitoan}

\begin{baitoan}[\cite{Tuyen_Toan_6}, 248., p. 47]
	Tìm phân số $\dfrac{a}{b}$ bằng phân số $\dfrac{60}{108}$ biết: (a) $\mbox{\rm ƯCLN}(a,b) = 15$. (b) $\operatorname{BCNN}(a,b) = 180$.
\end{baitoan}

\begin{baitoan}[\cite{Tuyen_Toan_6}, 249., p. 48]
	Tìm phân số bằng phân số $\dfrac{200}{520}$ sao cho: (a) Tổng của tử \& mẫu là $306$. (b) Hiệu của tử \& mẫu là $184$. (c) Tích của tử \& mẫu là $2340$.
\end{baitoan}

\begin{baitoan}[\cite{Tuyen_Toan_6}, 250., p. 48]
	Chứng minh: $\forall n\in\mathbb{Z}$, các phân số sau là các phân số tối giản: (a) $\dfrac{3n - 2}{4n - 3}$. (b) $\dfrac{4n + 1}{6n + 1}$.
\end{baitoan}

\begin{baitoan}[\cite{Tuyen_Toan_6}, 251., p. 48]
	Cho $\dfrac{a}{b}$ là 1 phân số chưa tối giản. Chứng minh các phân số sau chưa tối giản: (a) $\dfrac{a}{a - b}$. (b) $\dfrac{2a}{a - 2b}$.
\end{baitoan}

\begin{baitoan}[\cite{Binh_Toan_6_tap_2}, VD1, p. 4]
	Tìm $n\in\mathbb{N}$ để phân số $A = \dfrac{n + 10}{2n - 8}\in\mathbb{Z}$.
\end{baitoan}

\begin{baitoan}
	Cho $a,b,c,d\in\mathbb{Z}$, $c^2 + d^2\ne0$. Tìm $n\in\mathbb{N}$ để phân số $A = \dfrac{an + b}{cn + d}\in\mathbb{Z}$.
\end{baitoan}

\begin{baitoan}[\cite{Binh_Toan_6_tap_2}, VD2, p. 5]
	Tìm $n\in\mathbb{N}$ để phân số $A = \dfrac{21n + 3}{6n + 4}$ rút gọn được.
\end{baitoan}

\begin{baitoan}
	Cho $a,b,c,d\in\mathbb{Z}$, $c^2 + d^2\ne0$. Tìm $n\in\mathbb{N}$ để phân số $A = \dfrac{an + b}{cn + d}$ rút gọn được.
\end{baitoan}

\begin{baitoan}[\cite{Binh_Toan_6_tap_2}, VD3, p. 5]
	Tìm $a,b,c,d\in\mathbb{N}$ nhỏ nhất sao cho $\dfrac{a}{b} = \dfrac{3}{5}$, $\dfrac{b}{c} = \dfrac{12}{21}$, $\dfrac{c}{d} = \dfrac{6}{11}$.
\end{baitoan}

\begin{baitoan}[\cite{Binh_Toan_6_tap_2}, VD4, p. 5]
	Tìm số tự nhiên lớn nhất có 3 chữ số sao cho số đó bằng mỗi tổng $a + b,c + d,e + f$ \& $\dfrac{a}{b} = \dfrac{35}{49},\dfrac{c}{d} = \dfrac{130}{143},\dfrac{e}{f} = \dfrac{7}{13}$.
\end{baitoan}

\begin{baitoan}[\cite{Binh_Toan_6_tap_2}, 1., p. 6]
	Rút gọn phân số: (a) $\dfrac{199\ldots9}{99\ldots95}$ ($10$ chữ số $9$ ở tử, $10$ chữ số $9$ ở mẫu). (b) $\dfrac{121212}{424242}$. (c) $\dfrac{187187187}{221221221}$. (d) $\dfrac{3\cdot7\cdot13\cdot37\cdot39 - 10101}{505050 + 70707}$.
\end{baitoan}

\begin{baitoan}[\cite{Binh_Toan_6_tap_2}, 2., p. 6]
	Chứng minh các phân số sau có giá trị là số tự nhiên: (a) $\dfrac{10^{2002} + 2}{3}$. (b) $\dfrac{10^{2003} + 8}{9}$.
\end{baitoan}

\begin{baitoan}[\cite{Binh_Toan_6_tap_2}, 3., p. 6]
	Chứng mih các phân số sau bằng nhau: (a) $\dfrac{1717}{2929}$ \& $\dfrac{171717}{292929}$. (b) $\dfrac{3210 - 34}{4170 - 41}$ \& $\dfrac{6420 - 68}{8340 - 82}$. (c) $\dfrac{2106}{7320}$, $\dfrac{4212}{14640}$, \& $\dfrac{6318}{21960}$.
\end{baitoan}

\begin{baitoan}[\cite{Binh_Toan_6_tap_2}, 4., p. 6]
	Tìm $x,y\in\mathbb{Z}$ thỏa: (a) $\dfrac{x}{3} = \dfrac{y}{5}$. (b) $\dfrac{x}{28} = \dfrac{y}{35}$.
\end{baitoan}

\begin{baitoan}[\cite{Binh_Toan_6_tap_2}, 5., p. 6]
	Tìm các phân số $\dfrac{a}{b}$, $a\in\mathbb{N}$, $b\in\mathbb{N}^\star$, có giá trị bằng: (a) $\dfrac{36}{45}$ biết $\mbox{\rm BCNN}(a,b) = 300$. (b) $\dfrac{21}{35}$ biết $\mbox{\rm ƯCLN}(a,b) = 30$. (c) $\dfrac{15}{35}$ biết $\mbox{\rm ƯCLN}(a,b)\cdot\mbox{\rm BCNN}(a,b) = 3549$.
\end{baitoan}

\begin{baitoan}[\cite{Binh_Toan_6_tap_2}, 6., p. 7]
	Chứng minh các phân số sau tối giản với mọi $n\in\mathbb{N}$. (a) $\dfrac{n + 1}{2n + 3}$. (b) $\dfrac{2n + 3}{4n + 8}$. (c) $\dfrac{3n + 2}{5n + 3}$.
\end{baitoan}

\begin{baitoan}[\cite{Binh_Toan_6_tap_2}, 7., p. 7]
	Cho phân số $A = \dfrac{63}{3n + 1}$, $n\in\mathbb{N}$. (a) Với giá trị nào của $n$ thì $A$ rút gọn được? (b) Với giá trị nào của $n$ thì $A\in\mathbb{N}$?
\end{baitoan}

\begin{baitoan}[\cite{Binh_Toan_6_tap_2}, 8., p. 7]
	Tìm các số tự nhiên $n$ để các phân số sau là phân số tối giản: (a) $\dfrac{2n + 3}{4n + 1}$. (b) $\dfrac{3n + 2}{7n + 1}$. (c) $\dfrac{2n + 7}{5n + 2}$.
\end{baitoan}

\begin{baitoan}[\cite{Binh_Toan_6_tap_2}, 9., p. 7]
	Có bao nhiêu số nguyên dương $n$ không vượt quá $1000$ để phân số $\dfrac{n + 12}{n^2 + 9n - 13}$ là phân số tối giản?
\end{baitoan}

\begin{baitoan}[\cite{Binh_Toan_6_tap_2}, 10., p. 7]
	Tìm $n\in\mathbb{N}$ để phân số $\dfrac{n + 3}{2n - 2}\in\mathbb{Z}$.
\end{baitoan}

\begin{baitoan}[\cite{Binh_Toan_6_tap_2}, 11., p. 7]
	Tìm các số nguyên $n$ sao cho các phân số sau có giá trị là số nguyên: (a) $\dfrac{12}{3n - 1}$. (b) $\dfrac{2n + 3}{7}$.
\end{baitoan}

\begin{baitoan}[\cite{Binh_Toan_6_tap_2}, 12., p. 7]
	Tìm $n\in\mathbb{N}$ để phân số $A = \dfrac{8n + 193}{4n + 3}$: (a) Có giá trị là số tự nhiên. (b) Là phân số tối giản. (c) Với giá trị nào của $n$ trong khoảng từ $150$ đến $170$ thì phân số $A$ rút gọn được?
\end{baitoan}

\begin{baitoan}[\cite{Binh_Toan_6_tap_2}, 13., p. 7]
	Tìm các phân số tối giản nhỏ hơn $1$ có tử \& mẫu đều dương, biết tích của tử \& mẫu của phân số bằng $120$.
\end{baitoan}

\begin{baitoan}[\cite{Binh_Toan_6_tap_2}, 14., p. 7]
	Tìm $n\in\mathbb{N}$ nhỏ nhất để các phân số sau đều là phân số tối giản: $\dfrac{5}{n + 8},\dfrac{6}{n + 9},\dfrac{7}{n + 10},\ldots,\dfrac{17}{n + 20}$.
\end{baitoan}

\begin{baitoan}[\cite{Binh_Toan_6_tap_2}, 15., p. 7]
	Cho 3 phân số $\dfrac{15}{42},\dfrac{49}{56},\dfrac{36}{51}$. Biến đổi 3 phân số trên thành các phân số bằng chúng sao cho mẫu của phân số thứ nhất bằng tử của phân số thứ 2, mẫu của phân số thứ 2 bằng tử của phân số thứ 3.
\end{baitoan}

\begin{baitoan}[\cite{Binh_Toan_6_tap_2}, 16., p. 7]
	Cho 3 phân số $\dfrac{5}{8},\dfrac{11}{20},\dfrac{4}{15}$. Tìm 3 phân số (có tử \& mẫu dương) theo thứ tự bằng 3 phân số trên sao cho hiệu của mẫu \& tử của mỗi phân số này đều bằng nhau \& hiệu đó có giá trị nhỏ nhất.
\end{baitoan}

\begin{baitoan}[\cite{Binh_Toan_6_tap_2}, 17., p. 8]
	Tìm các phân số lớn hơn $\dfrac{1}{5}$ \& khác số tự nhiên biết nếu lấy mẫu nhân với 1 số, lấy tử cộng với số đó thì giá trị của phân số không đổi.
\end{baitoan}

\begin{baitoan}[\cite{Binh_Toan_6_tap_2}, 18., p. 8]
	Cho phân số $A = \dfrac{23 + 22 + 21 + \cdots + 13}{11 + 10 + 9 + \cdots + 1}$. Nêu cách xóa 1 số hạng ở tử \& 1 số hạng ở mẫu của $A$ để được 1 phân số mới vẫn bằng phân số $A$.
\end{baitoan}

\begin{baitoan}[\cite{Binh_Toan_6_tap_2}, 19., p. 8, Bộ sử Hume]
	Người Anh có thói quen xếp bộ sử nước Anh của Hume (David Hume, nhà sử học Scotland) gồm 9 tập ở tủ sách đặc biệt gồm 2 ngăn: ngăn trên xếp 5 cuốn, ngăn dưới xếp 4 cuốn, ở gáy các cuốn sách đó ghi các số $1,2,3,\ldots,9$. Nếu chủ nhân xếp $\dfrac{13458}{6729}$ (phân số này có giá trị bằng $2$) thì chứng tỏ chủ nhân đã đọc 2 tập (riêng trường hợp mới đọc 1 tập thì xếp $\dfrac{12345}{6789}$). Nêu cách xếp 9 cuốn sách đó để chứng tỏ chủ nhân của bộ sách đã đọc $3,4,5,6,7,8,9$ tập.
\end{baitoan}

\begin{baitoan}[\cite{TLCT_THCS_Toan_6_so_hoc}, VD8.1, p. 51]
	Quan sát dãy phân số $\dfrac{1}{2},\dfrac{5}{12},\dfrac{1}{3},\dfrac{1}{4},\dfrac{1}{6}$ để viết tiếp 1 phân số nữa theo quy luật của dãy.
\end{baitoan}

\begin{baitoan}[\cite{TLCT_THCS_Toan_6_so_hoc}, VD8.2, p. 52]
	Tìm $n\in\mathbb{Z}$ để cả 3 phân số $\dfrac{15}{n},\dfrac{12}{n + 2},\dfrac{6}{2n - 5}\in\mathbb{Z}$.
\end{baitoan}

\begin{baitoan}[\cite{TLCT_THCS_Toan_6_so_hoc}, VD8.3, p. 52]
	Cho phân số $\dfrac{1 + 2 + \cdots + 20}{6 + 7 + \cdots + 36}$. Xóa 1 số hạng ở tử \& 1 số hạng ở mẫu của phân số này để giá trị của phân số đó không đổi.
\end{baitoan}

\begin{baitoan}[\cite{TLCT_THCS_Toan_6_so_hoc}, VD8.4, p. 53]
	Tìm $a,b\in\mathbb{N}$ biết $\dfrac{a}{b} = \dfrac{132}{143},{\rm BCNN}(a,b) = 1092$.
\end{baitoan}

\begin{baitoan}[\cite{TLCT_THCS_Toan_6_so_hoc}, 8.1., p. 53]
	Tìm $x\in\mathbb{Z}$ thỏa: (a) $\dfrac{7}{x} = \dfrac{x}{28}$. (b) $\dfrac{10 + x}{17 + x} = \dfrac{3}{4}$. (c) $\dfrac{40 + x}{77 - x} = \dfrac{6}{7}$.
\end{baitoan}

\begin{baitoan}[\cite{TLCT_THCS_Toan_6_so_hoc}, 8.2., p. 53]
	Tìm $a,b\in\mathbb{Z}$ thỏa $a^3 + b^3 = 1216$ \& phân số $\dfrac{a}{b}$ rút gọn được thành $\dfrac{3}{5}$.
\end{baitoan}

\begin{baitoan}[\cite{TLCT_THCS_Toan_6_so_hoc}, 8.3., p. 53]
	Viết các phân số tối giản $\dfrac{a}{b}$  với $a,b\in\mathbb{Z}$ \& $ab = 100$.
\end{baitoan}

\begin{baitoan}[\cite{TLCT_THCS_Toan_6_so_hoc}, 8.4., p. 53]
	Rút gọn phân số: (a) $\dfrac{10\cdot11 + 50\cdot55 + 70\cdot77}{11\cdot12 + 55\cdot60 + 77\cdot84}$. (b) $\dfrac{1\cdot3\cdot5\cdots49}{26\cdot27\cdot28\cdots50}$.
\end{baitoan}

\begin{baitoan}[\cite{TLCT_THCS_Toan_6_so_hoc}, 8.5., p. 53]
	Tìm $n\in\mathbb{Z}$ thỏa: (a) $\dfrac{n + 3}{n - 2}$ là số nguyên âm. (b) $\dfrac{n + 7}{3n - 1}\in\mathbb{Z}$. (c) $\dfrac{3n + 2}{4n - 5}\in\mathbb{Z}$.
\end{baitoan}

\begin{baitoan}[\cite{TLCT_THCS_Toan_6_so_hoc}, 8.6., p. 54]
	Chứng minh phân số $\dfrac{n - 5}{3n - 14}$ tối giản $\forall n\in\mathbb{Z}$.
\end{baitoan}

\begin{baitoan}[\cite{TLCT_THCS_Toan_6_so_hoc}, 8.7., p. 54]
	Tìm $n\in\mathbb{Z}$ thỏa $\dfrac{2n - 1}{3n + 2}$ rút gọn được.
\end{baitoan}

\begin{baitoan}[\cite{TLCT_THCS_Toan_6_so_hoc}, 8.8., p. 54]
	Tìm $n\in\mathbb{N}$ nhỏ nhất để 5 phân số $\dfrac{n + 7}{3},\dfrac{n + 8}{4},\dfrac{n + 9}{5},\dfrac{n + 10}{6},\dfrac{n + 11}{7}$ tối giản.
\end{baitoan}

\begin{baitoan}[\cite{TLCT_THCS_Toan_6_so_hoc}, 8.9., p. 54]
	Tìm $a,b\in\mathbb{Z}$ biết $\dfrac{a}{b} = \dfrac{49}{56},\mbox{\rm ƯCLN}(a,b) = 12$.
\end{baitoan}

\begin{baitoan}[\cite{TLCT_THCS_Toan_6_so_hoc}, 8.10., p. 54]
	Tìm $a,b,c,d\in\mathbb{N}$ nhỏ nhất thỏa $\dfrac{a}{b} = \dfrac{5}{14},\dfrac{b}{c} = \dfrac{21}{28},\dfrac{c}{d} = \dfrac{6}{11}$.
\end{baitoan}

\begin{baitoan}
	Biện luận theo $a,b,c,d\in\mathbb{Z}$ để tìm $n\in\mathbb{Z}$ để phân số $\dfrac{an + b}{cn + d}$: (a) Tối giản. (b) Rút gọn được.
\end{baitoan}

\begin{baitoan}
	Cho $a,b,c\in\mathbb{N}^\star$. Tìm $x,y\in\mathbb{N}$ theo $a,b,c$ thỏa: (a) $\dfrac{x}{y} = \dfrac{a}{b},\mbox{\rm ƯCLN}(x,y) = c$. (b) $\dfrac{x}{y} = \dfrac{a}{b},{\rm BCNN}(x,y) = c$.
\end{baitoan}

%------------------------------------------------------------------------------%

\section{So Sánh Phân Số. Hỗn Số Dương}
SGK \cite[Chap. V, \S2, pp. 31--33]{SGK_Toan_6_Canh_Dieu_tap_2}: LT1. LT2. 1. 2. 3. 4. 5. SBT \cite[Chap. V, \S2, pp. 34--35]{SBT_Toan_6_Canh_Dieu_tap_2}: 15. 16. 17. 18. 19. 20. 21. 22. 23. 24. 25. 26.

\begin{baitoan}[{\sf Program}: Interchange between fraction \& mixed number]
	Viết chương trình {\sf Pascal, Python, C{\tt/}C++} để chuyển đổi giữa hỗn số (âm \& dương) \& phân số.
\end{baitoan}

\begin{baitoan}[\cite{Binh_boi_duong_Toan_6_tap_2}, H1, p. 12]
	{\rm Đ{\tt/}S?} (a) 1 cách để quy đồng mẫu 2 phân số $\dfrac{a}{b}$ \& $\dfrac{c}{d}$ là $\dfrac{a}{b} = \dfrac{ad}{bd}$, $\dfrac{c}{d} = \dfrac{cb}{db}$. (b) Phân số dương lớn hơn phân số âm. (c) Trong 2 phân số có cùng mẫu, phân số nào có tử lớn hơn thì phân số đó lớn hơn. (d) Trong 2 phân số có mẫu số dương \& tử bằng nhau, phân số nào có mẫu nhỏ hơn thì phân số đó lớn hơn. (e) $3\dfrac{7}{5}$ là 1 hỗn số. (f) Phân số $\dfrac{22}{5}$ viết dưới dạng hỗn số là $4\dfrac{2}{5}$.
\end{baitoan}

\begin{baitoan}[\cite{Binh_boi_duong_Toan_6_tap_2}, H2, p. 12]
	{\rm Đ{\tt/}S?} Thực hiện quy đồng mẫu 2 phân số $\dfrac{10}{12}$ \& $\dfrac{5}{8}$, Sơn \& Huy đã làm như sau: Sơn: $\dfrac{10}{12} = \dfrac{10\cdot8}{12\cdot8} = \dfrac{80}{96}$, $\dfrac{5}{8} = \dfrac{5\cdot12}{8\cdot12} = \dfrac{60}{96}$. Huy: $\dfrac{10}{12} = \dfrac{5}{6} = \dfrac{5\cdot4}{6\cdot4} = \dfrac{20}{24}$, $\dfrac{5}{8} = \dfrac{5\cdot3}{8\cdot3} = \dfrac{15}{24}$.
\end{baitoan}

\begin{baitoan}[\cite{Binh_boi_duong_Toan_6_tap_2}, H3, p. 12]
	{\rm Đ{\tt/}S?} Để so sánh 2 phân số $\dfrac{23}{-5}$ \&  $\dfrac{24}{-5}$, Hà đã giải thích như sau: $\dfrac{23}{-5}$ \&  $\dfrac{24}{-5}$ là 2 phân số có cùng mẫu \& $23 < 24$ nên $\dfrac{23}{-5} < \dfrac{24}{-5}$.
\end{baitoan}

\begin{baitoan}[\cite{Binh_boi_duong_Toan_6_tap_2}, VD1, p. 12]
	Quy đồng mẫu các phân số: (a) $\dfrac{-3}{16},\dfrac{5}{-24}$. (b) $\dfrac{3}{14},\dfrac{-5}{18},\dfrac{25}{-42}$. (c) $\dfrac{3}{16},\dfrac{5}{48},\dfrac{-7}{4}$. (d) $\dfrac{3}{7},\dfrac{-8}{5},\dfrac{5}{12}$. (e) $\dfrac{-15}{18},\dfrac{42}{-72},\dfrac{32}{120}$.
\end{baitoan}

\begin{baitoan}[\cite{Binh_boi_duong_Toan_6_tap_2}, VD2, p. 13]
	Sắp xếp các phân số sau theo thứ tự tăng dần: (a) $\dfrac{5}{24},\dfrac{11}{36},\dfrac{17}{60}$. (b) $\dfrac{23}{47},\dfrac{69}{85},\dfrac{92}{137}$. (c) $\dfrac{17}{60},\dfrac{16}{73}$.
\end{baitoan}

\begin{baitoan}[\cite{Binh_boi_duong_Toan_6_tap_2}, VD3, p. 14]
	(a) Viết 2 phân số sau dưới dạng hỗn số: $\dfrac{26}{5},\dfrac{153}{25}$. (b) Viết 2 hỗn số sau dưới dạng phân số: $3\dfrac{2}{7},8\dfrac{3}{5}$.
\end{baitoan}

\begin{baitoan}[\cite{Binh_boi_duong_Toan_6_tap_2}, VD4, p. 14]
	So sánh $\dfrac{62}{15},\dfrac{70}{17}$.
\end{baitoan}

\begin{baitoan}[\cite{Binh_boi_duong_Toan_6_tap_2}, VD5, p. 15]
	Mai \& Đào đi xe đạp đến trường với cùng tốc độ. Mai đi hết $\dfrac{2}{3}$ giờ, Đào đi hết $\dfrac{3}{4}$ giờ. Nhà ai cách xa trường hơn?
\end{baitoan}

\begin{baitoan}[\cite{Binh_boi_duong_Toan_6_tap_2}, VD6, p. 15]
	Tìm các phân số có mẫu là $5$, lớn hơn $\dfrac{-2}{3}$ \& nhỏ hơn $\dfrac{1}{-6}$.
\end{baitoan}

\begin{baitoan}[\cite{Binh_boi_duong_Toan_6_tap_2}, 2.1., p. 15]
	Quy đồng mẫu các phân số: (a) $\dfrac{13}{30},\dfrac{-7}{120}$. (b) $\dfrac{36}{75},\dfrac{6}{11}$. (c) $\dfrac{3}{25},\dfrac{4}{5},\dfrac{-8}{75}$. (d) $\dfrac{-5}{18},\dfrac{17}{60},\dfrac{32}{-45}$.
\end{baitoan}

\begin{baitoan}[\cite{Binh_boi_duong_Toan_6_tap_2}, 2.2., p. 15]
	Sắp xếp các phân số sau theo thứ tự từ nhỏ đến lớn: (a) $\dfrac{6}{7},\dfrac{9}{10}$. (b) $\dfrac{10}{-21},\dfrac{-4}{7},\dfrac{7}{9}$. (c) $\dfrac{-5}{-28},\dfrac{6}{35},\dfrac{27}{-180}$. (d) $\dfrac{3}{5},\dfrac{-5}{3},\dfrac{-36}{-60},\dfrac{18}{21}$.
\end{baitoan}

\begin{baitoan}[\cite{Binh_boi_duong_Toan_6_tap_2}, 2.3., p. 15]
	2 bạn mai \& Đào vào hiệu sách chọn được 1 cuốn sách mà cả 2 cùng thích, mỗi bạn mua 1 quyển. Sau ngày nghỉ cuối tuần, Mai đã đọc được $\dfrac{7}{8}$ số trang còn Đào đã đọc được $\dfrac{4}{5}$ số trang của cuốn sách đó. Ai đã đọc được nhiều hơn?
\end{baitoan}

\begin{baitoan}[\cite{Binh_boi_duong_Toan_6_tap_2}, 2.4., p. 16]
	Sơ kết học kỳ I, lớp 6A có $\dfrac{3}{4}$ số học sinh là học sinh giỏi môn Toán, $\dfrac{3}{5}$ số học sinh là học sinh giỏi môn Ngữ Văn, $\dfrac{2}{3}$ số học sinh là học sinh giỏi môn Anh văn. Sắp xếp theo thứ tự các môn học này theo số lượng học sinh giỏi từ nhiều nhất đến ít nhất.
\end{baitoan}

\begin{baitoan}[\cite{Binh_boi_duong_Toan_6_tap_2}, 2.5., p. 16]
	Tìm 4 phân số lớn hơn $\dfrac{5}{12}$ \& nhỏ hơn $\dfrac{5}{8}$.
\end{baitoan}

\begin{baitoan}[\cite{Binh_boi_duong_Toan_6_tap_2}, 2.6., p. 16]
	Tìm các phân số thỏa mãn: (a) Có mẫu là $20$, lớn hơn $\dfrac{4}{13}$, \& nhỏ hơn $\dfrac{5}{13}$. (b) Có mẫu là $14$, lớn hơn $\dfrac{-2}{21}$, \& nhỏ hơn $\dfrac{2}{9}$.
\end{baitoan}

\begin{baitoan}[\cite{Binh_boi_duong_Toan_6_tap_2}, 2.7., p. 16]
	Tìm các số nguyên $n$ lớn hơn $\dfrac{283}{23}$ \& nhỏ hơn $\dfrac{467}{31}$.
\end{baitoan}

\begin{baitoan}[\cite{Binh_boi_duong_Toan_6_tap_2}, 2.8., p. 16]
	An \& Bình đạp xe với tốc độ không đổi trên cùng 1 quãng đường. An đi hết $36$ phút, Bình đi hết $44$ phút. (a) So sánh quãng đường mà An đi được trong $20$ phút với quãng đường mà Bình đi được trong $26$ phút. (b) Bình phải đi trong bao lâu để được quãng đường bằng quãng đường An đi được trong $18$ phút?
\end{baitoan}

\begin{baitoan}[\cite{Binh_boi_duong_Toan_6_tap_2}, 2.9., p. 16]
	Tìm $x\in\mathbb{Z}$ thỏa $\dfrac{-7}{12} < \dfrac{x - 1}{4} < \dfrac{2}{3}$.
\end{baitoan}

\begin{baitoan}[\cite{Binh_boi_duong_Toan_6_tap_2}, 2.10., p. 16]
	Tìm $x,y\in\mathbb{Z}$ thỏa: (a) $\dfrac{-1}{3} < \dfrac{x}{36} < \dfrac{y}{118} < \dfrac{-1}{4}$. (b) $\dfrac{1}{220} < \dfrac{x}{165} < \dfrac{y}{132} < \dfrac{1}{60}$.
\end{baitoan}

\begin{baitoan}[\cite{Binh_boi_duong_Toan_6_tap_2}, 2.11., p. 16]
	Cho $a,b,c\in\mathbb{N}^\star$. Chứng minh: (a) Nếu $\dfrac{a}{b} < 1$ thì $\dfrac{a}{b} < \dfrac{a + c}{b + c}$. (b) Nếu $\dfrac{a}{b} > 1$ thì $\dfrac{a}{b} > \dfrac{a + c}{b + c}$. Áp dụng: So sánh $\dfrac{17}{18}$ \& $\dfrac{26}{27}$.
\end{baitoan}

\begin{baitoan}[\cite{Tuyen_Toan_6}, VD52, p. 48]
	So sánh 2 phân số $\dfrac{-101}{-100}$ \& $\dfrac{200}{201}$.
\end{baitoan}

\begin{baitoan}
	Cho $a,b,c,d\in\mathbb{N}$, $a > b > 0$, $d > c > 0$. So sánh các số $\dfrac{\pm a}{\pm b},\dfrac{\pm c}{\pm d}$.
\end{baitoan}

\begin{baitoan}[\cite{Tuyen_Toan_6}, VD53, p. 49]
	Sắp xếp 4 phân số $\dfrac{5}{8},\dfrac{9}{16},\dfrac{2}{-3},\dfrac{-7}{12}$ theo thứ tự tăng dần.
\end{baitoan}

\begin{baitoan}[\cite{Tuyen_Toan_6}, 253., p. 49]
	Quy đồng mẫu rồi so sánh phân số: (a) $\dfrac{-8}{21},\dfrac{-789}{3131}$. (b) $\dfrac{11}{2^3\cdot3^4\cdot5^2},\dfrac{29}{2^2\cdot3^4\cdot5^3}$. (c) $\dfrac{1}{n},\dfrac{1}{n + 1}$, $\forall n\in\mathbb{N}$.
\end{baitoan}

\begin{baitoan}[\cite{Tuyen_Toan_6}, 254., p. 49]
	Quy đồng mẫu rồi sắp xếp phân số theo thứ tự tăng dần: (a) $\dfrac{7}{39},\dfrac{11}{65},\dfrac{9}{52}$. (b) $\dfrac{17}{30},\dfrac{-19}{30},\dfrac{38}{45},\dfrac{-13}{18}$.
\end{baitoan}

\begin{baitoan}[\cite{Tuyen_Toan_6}, 255., p. 49]
	Quy đồng tử rồi sắp xếp phân số theo thứ tự tăng dần: $\dfrac{9}{382},\dfrac{6}{257},\dfrac{15}{643}$.
\end{baitoan}

\begin{baitoan}[\cite{Tuyen_Toan_6}, 256., p. 49]
	Sắp xếp phân số theo thứ tự tăng dần: (a) $\dfrac{29}{40},\dfrac{28}{41},\dfrac{29}{41}$. (b) $\dfrac{307}{587},\dfrac{317}{587},\dfrac{307}{593}$.
\end{baitoan}

\begin{baitoan}[\cite{Tuyen_Toan_6}, 257., p. 49]
	So sánh: (a) $\dfrac{179}{197},\dfrac{971}{917}$. (b) $\dfrac{183}{184},\dfrac{-184}{-183}$.
\end{baitoan}

\begin{baitoan}[\cite{Tuyen_Toan_6}, 258., p. 49]
	So sánh: (a) $\dfrac{83}{13},\dfrac{23}{3},\dfrac{123}{23}$. (b) $\dfrac{53}{17},\dfrac{71}{23}$.
\end{baitoan}

\begin{baitoan}[\cite{Tuyen_Toan_6}, 259., p. 49]
	Cho $a\in\{-5,7,9\},b\in\{0,16,17,18,19\}$. Tìm {\rm GTLN, GTNN} của phân số $\dfrac{a}{b}$.
\end{baitoan}

\begin{baitoan}[\cite{Tuyen_Toan_6}, 260., p. 50]
	Đếm số phân số lớn hơn $\dfrac{7}{8}$ \& nhỏ hơn $\dfrac{9}{10}$ mà: (a) Mẫu là $40$. (b) Mẫu là $80$. (c) Mẫu là $400$.
\end{baitoan}

\begin{baitoan}[\cite{Tuyen_Toan_6}, 261., p. 50]
	Đếm số phân số lớn hơn $\dfrac{1}{6}$ \& nhỏ hơn $\dfrac{1}{4}$ mà: (a) Tử là $1$. (b) Tử là $5$.
\end{baitoan}

\begin{baitoan}[\cite{Tuyen_Toan_6}, 262., p. 50]
	Tìm 9 phân số lớn hơn $\dfrac{7}{12}$ \& nhỏ hơng $\dfrac{5}{8}$.
\end{baitoan}

\begin{baitoan}[\cite{Tuyen_Toan_6}, 263., p. 50]
	2 người cùng đi quãng đường AB. Người thứ nhất đi hết {\rm32 ph}, người thứ 2 đi hết {\rm48 ph}. (a) So sánh quãng đường người thứ nhất đi trong {\rm20 ph} với quãng đường người thứ 2 đi trong {\rm28 ph}. (b) Người thứ 2 phải đi trong bao lâu để được quãng đường bằng quãng đường người thứ nhất đi trong {\rm24 ph}?
\end{baitoan}

\begin{baitoan}[\cite{Tuyen_Toan_6}, VD54, p. 51]
	So sánh $\dfrac{18}{31},\dfrac{15}{37}$.
\end{baitoan}

\begin{baitoan}[\cite{Tuyen_Toan_6}, VD55, p. 52]
	So sánh $\dfrac{12}{47},\dfrac{19}{77}$.
\end{baitoan}

\begin{baitoan}[\cite{Tuyen_Toan_6}, VD56, p. 52]
	Chuyển phân số về hỗn số rồi sắp xếp tăng dần: $\dfrac{155}{9},\dfrac{87}{5}\dfrac{123}{8}$.
\end{baitoan}

\begin{baitoan}[\cite{Tuyen_Toan_6}, VD57, p. 52]
	So sánh $A = \dfrac{99^{15} + 2}{99^{16} + 2},B = \dfrac{99^{14} + 1}{99^{15} + 1}$.
\end{baitoan}

\begin{baitoan}[\cite{Tuyen_Toan_6}, 264., p. 53]
	So sánh: (a) $\dfrac{64}{85},\dfrac{73}{81}$. (b) $\dfrac{n + 1}{n + 2},\dfrac{n}{n + 3}$, $\forall n\in\mathbb{Z}$.
\end{baitoan}

\begin{baitoan}[\cite{Tuyen_Toan_6}, 265., p. 53]
	Sắp xếp tăng dần: $\dfrac{77}{95},\dfrac{76}{99},\dfrac{77}{99},\dfrac{75}{101}$.
\end{baitoan}

\begin{baitoan}[\cite{Tuyen_Toan_6}, 266., p. 53]
	So sánh: (a) $\dfrac{67}{77},\dfrac{73}{83}$. (b) $\dfrac{456}{461},\dfrac{123}{128}$. (c) $\dfrac{2003\cdot2004 - 1}{2003\cdot2004},\dfrac{2004\cdot2005 - 1}{2004\cdot2005}$.
\end{baitoan}

\begin{baitoan}[\cite{Tuyen_Toan_6}, 267., p. 53]
	So sánh: (a) $\dfrac{11}{32},\dfrac{16}{49}$. (b) $\dfrac{58}{89},\dfrac{36}{53}$.
\end{baitoan}

\begin{baitoan}[\cite{Tuyen_Toan_6}, 268., p. 53]
	So sánh $A = \dfrac{3535\cdot232323}{353535\cdot2323},B = \dfrac{3535}{3534},C = \dfrac{2323}{2322}$.
\end{baitoan}

\begin{baitoan}[\cite{Tuyen_Toan_6}, 269., p. 53]
	So sánh $A = \dfrac{5(11\cdot13 - 22\cdot26)}{22\cdot26 - 44\cdot52},B = \dfrac{138^2 - 690}{137^2 - 548}$.
\end{baitoan}

\begin{baitoan}[\cite{Tuyen_Toan_6}, 270., p. 53]
	So sánh: (a) $\dfrac{53}{57},\dfrac{531}{571}$. (b) $\dfrac{25}{26},\dfrac{25251}{26261}$.
\end{baitoan}

\begin{baitoan}[\cite{Tuyen_Toan_6}, 271., p. 53]
	Sắp xếp tăng dần: $\dfrac{731}{217},\dfrac{1711}{341},\dfrac{721}{143},\dfrac{221}{71}$.
\end{baitoan}

\begin{baitoan}[\cite{Tuyen_Toan_6}, 272., p. 53]
	So sánh $A = \dfrac{10^{11} - 1}{10^{12} - 1},B = \dfrac{10^{10} + 1}{10^{11} + 1}$.
\end{baitoan}

\begin{baitoan}[\cite{Tuyen_Toan_6}, 273., p. 53]
	So sánh phân số mà không được thực hiện các phép tính ở mẫu: $A = \dfrac{54\cdot107 - 53}{53\cdot107 + 54},B = \dfrac{135\cdot269 - 133}{134\cdot269 + 135}$.
\end{baitoan}

\begin{baitoan}[\cite{Tuyen_Toan_6}, 274., p. 53]
	So sánh: (a) $\left(\dfrac{1}{80}\right)^7,\left(\dfrac{1}{243}\right)^6$. (b) $\left(\dfrac{3}{8}\right)^5,\left(\dfrac{5}{243}\right)^3$.
\end{baitoan}

\begin{baitoan}[\cite{Binh_Toan_6_tap_2}, VD5, p. 8]
	So sánh $A = \dfrac{10^{15} + 1}{10^{16} + 1}$ \& $B = \dfrac{10^{16} + 1}{10^{17} + 1}$.
\end{baitoan}

\begin{baitoan}[\cite{Binh_Toan_6_tap_2}, VD6, p. 9]
	1 phân số có tử \& mẫu đều là số nguyên dương. Nếu cộng tử \& mẫu của phân số đó với cùng $n\in\mathbb{N}^\star$ thì phân số thay đổi thế nào?
\end{baitoan}

\begin{baitoan}[\cite{Binh_Toan_6_tap_2}, VD7, p. 9]
	So sánh $\left(\dfrac{1}{32}\right)^7$ \& $\left(\dfrac{1}{16}\right)^9$.
\end{baitoan}

\begin{baitoan}[\cite{Binh_Toan_6_tap_2}, VD8, p. 9]
	Chứng minh $95^8$ là 1 số có $16$ chữ số khi viết kết quả của nó trong hệ thập phân.
\end{baitoan}

\begin{baitoan}[\cite{Binh_Toan_6_tap_2}, VD9, p. 10]
	Cho $a,b\in\mathbb{N}^\star$ thỏa $\dfrac{5}{7} < \dfrac{a}{b} < \dfrac{9}{11}$. Tìm $a + b$ khi $b$ nhỏ nhất.
\end{baitoan}

\begin{baitoan}[\cite{Binh_Toan_6_tap_2}, 20., p. 10]
	So sánh $a,b\in\mathbb{N}$ biết $\dfrac{1 + 2 + 3 + \cdots + a}{a} < \dfrac{1 + 2 + 3 + \cdots + b}{b}$.
\end{baitoan}

\begin{baitoan}[\cite{Binh_Toan_6_tap_2}, 21., p. 10]
	So sánh: (a) $\dfrac{18}{91}$ \& $\dfrac{23}{114}$. (b) $\dfrac{21}{52}$ \& $\dfrac{213}{523}$. (c) $\dfrac{1313}{9191}$ \& $\dfrac{1111}{7373}$.
\end{baitoan}

\begin{baitoan}[\cite{Binh_Toan_6_tap_2}, 22., p. 10]
	So sánh các phân số sau, với $n\in\mathbb{N}$: (a) $\dfrac{n}{n + 1}$ \& $\dfrac{n + 2}{n + 3}$. (b) $\dfrac{n + 1}{n + 4}$ \& $\dfrac{n}{n + 5}$. (c) $\dfrac{n}{2n + 1}$ \& $\dfrac{3n + 1}{6n + 3}$.
\end{baitoan}

\begin{baitoan}[\cite{Binh_Toan_6_tap_2}, 23., p. 11]
	So sánh $A$ \& $B$: (a) $A = \dfrac{20}{39} + \dfrac{22}{27} + \dfrac{18}{43}$, $B = \dfrac{14}{39} + \dfrac{22}{29} + \dfrac{18}{41}$. (b) $A = \dfrac{3}{8^3} + \dfrac{7}{8^4}$, $B = \dfrac{7}{8^3} + \dfrac{3}{8^4}$. (c) $A = \dfrac{10^7 + 5}{10^7 - 8}$, $B = \dfrac{10^8 + 6}{10^8 - 7}$. (d) $A = \dfrac{10^{1992} + 1}{10^{1991} + 1}$, $B = \dfrac{10^{1993} + 1}{10^{1992} + 1}$.
\end{baitoan}

\begin{baitoan}[\cite{Binh_Toan_6_tap_2}, 24., p. 11]
	Tìm $x\in\mathbb{N}$ sao cho $\dfrac{4}{11} < \dfrac{x}{20} < \dfrac{5}{11}$.
\end{baitoan}

\begin{baitoan}[\cite{Binh_Toan_6_tap_2}, 25., p. 11]
	Tìm 2 phân số có các mẫu bằng $9$, các tử là 2 số tự nhiên liên tiếp sao cho phân số $\dfrac{4}{7}$ nằm giữa 2 phân số đó.
\end{baitoan}

\begin{baitoan}[\cite{Binh_Toan_6_tap_2}, 26., p. 11]
	Tìm 2 phân số có các tử bằng $1$, các mẫu là 2 số tự nhiên liên tiếp sao cho phân số $\dfrac{13}{84}$ nằm giữa 2 phân số đó.
\end{baitoan}

\begin{baitoan}[\cite{Binh_Toan_6_tap_2}, 27., p. 11]
	Tìm 2 phân số có mẫu bằng $21$ biết nó lớn hơn $\dfrac{5}{7}$ \& nhỏ hơn $\dfrac{5}{6}$.
\end{baitoan}

\begin{baitoan}[\cite{Binh_Toan_6_tap_2}, 28., p. 11]
	Tìm phân số $\dfrac{a}{b}$ sao cho $a$ là số tự nhiên nhỏ nhất thỏa mãn $\dfrac{4}{15} < \dfrac{a}{b} < \dfrac{1}{3}$.
\end{baitoan}

\begin{baitoan}[\cite{Binh_Toan_6_tap_2}, 29., p. 11]
	Tìm phân số $\dfrac{a}{b}$ lớn nhất nhỏ hơn $1$ với $a,b$ là các số nguyên dương có 1 chữ số.
\end{baitoan}

\begin{baitoan}[\cite{Binh_Toan_6_tap_2}, 30., p. 11]
	So sánh 2 phân số $\left(\dfrac{1}{243}\right)^9$ \& $\left(\dfrac{1}{83}\right)^{13}$.
\end{baitoan}

\begin{baitoan}[\cite{TLCT_THCS_Toan_6_so_hoc}, VD9.1, p. 55]
	(a) Chứng minh trong 2 phân số có cùng 1 tử, tử \& mẫu đều dương, phân số nào có mẫu nhỏ hơn thì phân số đó lớn hơn. (b) Áp dụng tính chất này để so sánh phân số: (i) $\dfrac{10}{11},\dfrac{12}{13},\dfrac{15}{16}$. (ii) $\dfrac{n + 1}{n + 5},\dfrac{n + 2}{n + 3}$, $n\in\mathbb{N}$.
\end{baitoan}

\begin{baitoan}[\cite{TLCT_THCS_Toan_6_so_hoc}, VD9.2, p. 56]
	Chứng minh nếu cộng cả tử \& mẫu của 1 phân số nhỏ hơn $1$, tử \& mẫu đều dương, với cùng 1 số nguyên dương thì giá trị của phân số đó tăng thêm.
\end{baitoan}

\begin{baitoan}[\cite{TLCT_THCS_Toan_6_so_hoc}, VD9.3, p. 56]
	So sánh $A = \dfrac{13579}{34567},B = \dfrac{13580}{34569}$.
\end{baitoan}

\begin{baitoan}[\cite{TLCT_THCS_Toan_6_so_hoc}, VD9.4, p. 57]
	So sánh $A = \dfrac{10^8 + 1}{10^9 + 1},B = \dfrac{10^9 + 1}{10^{10} + 1}$.
\end{baitoan}

\begin{baitoan}[\cite{TLCT_THCS_Toan_6_so_hoc}, 9.1., p. 57]
	Xếp các phân số $\dfrac{10}{19},\dfrac{12}{23},\dfrac{15}{17},\dfrac{20}{29},\dfrac{60}{71}$ theo thứ tự tăng dần.
\end{baitoan}

\begin{baitoan}[\cite{TLCT_THCS_Toan_6_so_hoc}, 9.2., p. 57]
	Tìm phân số tối giản $\dfrac{a}{b} < 1$ có $ab = 80$.
\end{baitoan}

\begin{baitoan}[\cite{TLCT_THCS_Toan_6_so_hoc}, 9.3., p. 58]
	Tìm $x\in\mathbb{Z}$ thỏa: (a) $\dfrac{1}{5} < \dfrac{x}{30} < \dfrac{1}{4}$. (b) $\dfrac{5}{8} < \dfrac{4}{x} < \dfrac{5}{7}$.
\end{baitoan}

\begin{baitoan}[\cite{TLCT_THCS_Toan_6_so_hoc}, 9.4., p. 58]
	So sánh phân số mà không quy đồng mẫu hoặc tử: (a) $\dfrac{7}{15},\dfrac{20}{39}$. (b) $\dfrac{14}{41},\dfrac{17}{54}$.
\end{baitoan}

\begin{baitoan}[\cite{TLCT_THCS_Toan_6_so_hoc}, 9.5., p. 58]
	So sánh phân số $\forall n\in\mathbb{N}$: (a) $\dfrac{n}{2n + 3}$. (b) $\dfrac{n}{3n + 1},\dfrac{2n}{6n + 1}$.
\end{baitoan}

\begin{baitoan}[\cite{TLCT_THCS_Toan_6_so_hoc}, 9.6., p. 58]
	So sánh phân số: (a) $A = \dfrac{35420}{35423},B = \dfrac{25343}{25345}$. (b) $C = \dfrac{5^{12} + 1}{5^{13} + 1},D = \dfrac{5^{11} + 1}{5^{12} + 1}$.
\end{baitoan}

\begin{baitoan}[\cite{TLCT_THCS_Toan_6_so_hoc}, 9.7., p. 58]
	Cho $x,y\in\mathbb{N},1\le y < x\le30$. (a) Tìm {\rm GTLN} của phân số $\dfrac{x + y}{x - y}$. (b) Tìm {\rm GTLN} của phân số $\dfrac{xy}{x - y}$.
\end{baitoan}

Hỗn số dương $=$ hỗn hợp của số nguyên dương \& phân số dương.

\begin{baitoan}[\cite{Binh_Toan_6_tap_2}, VD2, p. 32]
	(a) Viết phân số $\dfrac{7}{4}$ dưới dạng tổng của 1 số nguyên dương \& 1 phân số bé hơn $1$. (b) Viết phân số $\dfrac{21}{5}$ dưới dạng hỗn số.
\end{baitoan}

\begin{baitoan}[\cite{Binh_Toan_6_tap_2}, VD3, 2, p. 33]
	(a) Viết hỗn số $2\dfrac{3}{5}$ thành phân số. (b) Viết mỗi phân số sau thành hỗn số: $\dfrac{14}{3},\dfrac{22}{7}$. (c) Viết mỗi hỗn số sau thành phân số: $2\dfrac{3}{4},5\dfrac{1}{6}$.
\end{baitoan}

\begin{baitoan}[\cite{SGK_Toan_6_Canh_Dieu_tap_2}, 4., p. 33]
	(a) Viết các số đo thời gian dưới dạng hỗn số với đơn vị là giờ: $2$ giờ $15$ phút, $10$ giờ $20$ phút. (b) Viết các số đo diện tích sau dưới dạng hỗn số với đơn vị là hecta biết $1$\emph{ha} $= 100$\emph{a}: \emph{1ha7a, 3ha50a},
\end{baitoan}

\begin{baitoan}[Chuyển đổi phân số \& hỗn số dương]
	(a) Khi nào 1 phân số có thể chuyển thành 1 hỗn số dương? (b) Ngược lại, khi nào 1 hỗn số dương có thể chuyển thành 1 phân số? (c) Nếu định nghĩa 1 \emph{hỗn số âm} là 1 số đối của hỗn số dương tương ứng thì khi nào 1 hỗn số âm có thể chuyển thành 1 phân số?
\end{baitoan}

\begin{baitoan}[Công thức hỗn số dương]
	Chứng minh:
	\begin{align*}
		\dfrac{ac + b}{c} &= a + \dfrac{b}{c} = a\dfrac{b}{c},\ \forall a,b,c\in\mathbb{N}^\star,\\
		\dfrac{a}{b} &= \dfrac{\left\lfloor\dfrac{a}{b}\right\rfloor b + \left\{\dfrac{a}{b}\right\}b}{b} = \left\lfloor\dfrac{a}{b}\right\rfloor + \left\{\dfrac{a}{b}\right\} = \left\lfloor\dfrac{a}{b}\right\rfloor\left\{\dfrac{a}{b}\right\},\ \forall a,b\in\mathbb{N}^\star,\, a > b.
	\end{align*}
\end{baitoan}

%------------------------------------------------------------------------------%

\section{Calculus of Fraction -- Tính Toán với Phân Số}

\subsection{Operator $\pm$ of Fractions -- Phép $\pm$ Phân Số}
\fbox{1} Phân số Ai Cập là phân số có dạng $\dfrac{1}{n}$ với $n\in\mathbb{N}^\star$.

\noindent SGK \cite[Chap. V, \S3, pp. 34--38]{SGK_Toan_6_Canh_Dieu_tap_2}: LT1. LT2. LT3. LT4. LT5. 1. 2. 3. 4. 5. 6. 7. 8. SBT \cite[Chap. V, \S3, pp. 37--38]{SBT_Toan_6_Canh_Dieu_tap_2}: 27. 28. 29. 30. 31. 32. 33. 34. 35. 36. 37.

\begin{baitoan}[\cite{Binh_boi_duong_Toan_6_tap_2}, H1--H4, p. 19]
	(a) Cách để cộng, nhân 2 phân số? (b) Tính: $x = \dfrac{1}{2} + \dfrac{-2}{5},y = 2 - \dfrac{1}{2} - \dfrac{1}{4},z = \dfrac{-8}{3}\cdot\dfrac{9}{-14},t = \dfrac{6}{5}:\dfrac{-9}{10}$.
\end{baitoan}

\begin{baitoan}[\cite{Binh_boi_duong_Toan_6_tap_2}, VD1, p. 19]
	Tính: (a) $\dfrac{4}{5} + \dfrac{-6}{5}$. (b) $\dfrac{9}{10} - \dfrac{3}{10}$. (c) $\dfrac{-5}{9} - \dfrac{2}{9}$. (d) $\dfrac{37}{19} + \dfrac{12}{-19}$. (e) $\dfrac{2}{3} + \dfrac{-1}{4}$. (f) $\dfrac{1}{-3} + \dfrac{8}{9}$. (g) $-\dfrac{1}{6} + \dfrac{3}{-8}$. (h) $2 - \dfrac{1}{15} + \dfrac{2}{3}$.
\end{baitoan}

\begin{baitoan}[\cite{Binh_boi_duong_Toan_6_tap_2}, VD2, p. 20]
	Tính: (a) $\dfrac{1}{2}\cdot\dfrac{3}{-4}$. (b) $\dfrac{-2}{3}\cdot\dfrac{9}{-4}$. (c) $3\cdot\dfrac{11}{6}$. (d) $\dfrac{-8}{5}\cdot\dfrac{-25}{24}$. (e) $\dfrac{3}{10}:\dfrac{-9}{15}$. (f) $-12:\dfrac{4}{5}$. (g) $\dfrac{2}{-5}:\dfrac{-8}{25}$. (h) $\dfrac{3}{5}:(-9)$.
\end{baitoan}

\begin{baitoan}[\cite{Binh_boi_duong_Toan_6_tap_2}, VD3, p. 20]
	Tìm $x\in\mathbb{R}$ thỏa: (a) $x - \dfrac{1}{2} = \dfrac{3}{8}$. (b) $\dfrac{6}{7} - 2x = \dfrac{2}{3}$. (c) $\dfrac{-2}{3}x = \dfrac{8}{9}$. (d) $x:\dfrac{24}{5} = \dfrac{-5}{4}$.
\end{baitoan}

\begin{baitoan}[\cite{Binh_boi_duong_Toan_6_tap_2}, VD4, p. 21]
	Tính hợp lý: (a) $A = \dfrac{3}{-7} + \dfrac{6}{11} + \dfrac{17}{7} - \dfrac{9}{11}$. (b) $B = \dfrac{3}{5}\cdot\dfrac{-6}{13} - \dfrac{3}{5}\cdot\dfrac{7}{13}$. (c) $C = \dfrac{6}{5}\cdot\dfrac{-8}{7}\cdot\dfrac{-10}{3}$. (d) $D = \dfrac{3 - \dfrac{5}{12} + \dfrac{1}{6}}{\dfrac{1}{3} + \dfrac{9}{4} - \dfrac{7}{6}}$.
\end{baitoan}

\begin{baitoan}[\cite{Binh_boi_duong_Toan_6_tap_2}, VD5, p. 21]
	(a) Chứng minh $\dfrac{1}{n(n + 1)} = \dfrac{1}{n} - \dfrac{1}{n + 1}$, $\forall n\in\mathbb{N}^\star$. (b) Tính $A(2021) = \sum_{i=1}^{2020} \dfrac{1}{i(i + 1)} = \dfrac{1}{1\cdot2} + \dfrac{1}{2\cdot3} + \dfrac{1}{3\cdot4} + \cdots + \dfrac{1}{2020\cdot2021}$, $A(n) = \sum_{i=1}^n \dfrac{1}{i(i + 1)} = \dfrac{1}{1\cdot2} + \dfrac{1}{2\cdot3} + \dfrac{1}{3\cdot4} + \cdots + \dfrac{1}{n(n + 1)}$.
\end{baitoan}

\begin{baitoan}[\cite{Binh_boi_duong_Toan_6_tap_2}, VD6, p. 21]
	$\dfrac{2}{3}$ số công nhân trong 1 nhà máy là nữ \& $\dfrac{3}{4}$ trong số họ đang nuôi con nhỏ. Số công nhân nữ đang nuôi con nhỏ bằng mấy phần tổng số công nhân của nhà máy?
\end{baitoan}

\begin{baitoan}[\cite{Binh_boi_duong_Toan_6_tap_2}, 3.1., p. 22]
	Tính: (a) $\dfrac{5}{2} + \dfrac{-4}{3}$. (b) $\dfrac{7}{9} - \dfrac{1}{6}$. (c) $2 - \dfrac{5}{4}$. (d) $\dfrac{2}{3} - \dfrac{5}{2} + \dfrac{3}{4}$. (e) $\dfrac{7}{2}\cdot\dfrac{-6}{5}$. (f) $-16\cdot\dfrac{3}{8}$. (g) $\dfrac{-8}{5}:\dfrac{2}{-3}$. (h) $\dfrac{-6}{5}:12$.
\end{baitoan}

\begin{baitoan}[\cite{Binh_boi_duong_Toan_6_tap_2}, 3.2., p. 22]
	Tính: (a) $\dfrac{3}{4} - \dfrac{5}{2} + \dfrac{-3}{16}$. (b) $\dfrac{1}{3} - \dfrac{5}{7}\cdot\dfrac{14}{25}$. (c) $\left(\dfrac{7}{10} - \dfrac{5}{2}\right)\cdot\dfrac{2}{3} - 2$. (d) $\left(1 + \dfrac{7}{3}\right):\left(2 - \dfrac{5}{12} - \dfrac{3}{2}\right)$.
\end{baitoan}

\begin{baitoan}[\cite{Binh_boi_duong_Toan_6_tap_2}, 3.3., p. 22]
	Tính hợp lý: (a) $\dfrac{3}{17} + \dfrac{1}{22} + \dfrac{5}{3} - \dfrac{23}{22} + \dfrac{14}{17}$. (b) $\dfrac{-3}{5}\cdot\dfrac{5}{8} + \dfrac{5}{8}\cdot\dfrac{4}{5}$. (c) $\dfrac{8}{7}\cdot\dfrac{9}{19} - \dfrac{8}{19}\cdot\dfrac{5}{7} + \dfrac{3}{19}\cdot\dfrac{8}{7}$. (d) $\dfrac{\dfrac{6}{5} - \dfrac{4}{7} + \dfrac{8}{19}}{\dfrac{9}{5} - \dfrac{6}{7} + \dfrac{12}{19}}$.
\end{baitoan}

\begin{baitoan}[\cite{Binh_boi_duong_Toan_6_tap_2}, 3.4., p. 22]
	Tìm $x\in\mathbb{R}$ thỏa: (a) $x - \dfrac{1}{2} = \dfrac{3}{4}$. (b) $x + \dfrac{2}{5} = \dfrac{-7}{10}$. (c) $-\dfrac{2}{3}x + \dfrac{1}{5} = \dfrac{4}{3}$. (d) $\dfrac{1}{3}:x - \dfrac{3}{2} = 5$. (e) $x\in\mathbb{Z}$ thỏa $\dfrac{-2}{3} - 1 + \dfrac{1}{4}\le x\le\dfrac{3}{10} + \dfrac{27}{5} - \dfrac{3}{2}$.
\end{baitoan}

\begin{baitoan}[\cite{Binh_boi_duong_Toan_6_tap_2}, 3.5., p. 22]
	Tính giá trị biểu thức $A = 2(a + b) - ab$ với $a = \dfrac{1}{3},b = \dfrac{2}{5}$.
\end{baitoan}

\begin{baitoan}[\cite{Binh_boi_duong_Toan_6_tap_2}, 3.6., p. 22]
	Huy viết ra 2 phân số, biết mẫu số của phân số thứ nhất bằng $3$ lần tử số của phân số thứ 2, mẫu số của phân số thứ 2 bằng $2$ lần tử số của phân số thứ nhất. Tính tích 2 phân số đó.
\end{baitoan}

\begin{baitoan}[\cite{Binh_boi_duong_Toan_6_tap_2}, 3.7., p. 22]
	1 tổ sản xuất trong tuần thứ nhất làm được $\dfrac{1}{4}$ kế hoạch công việc của tháng, tuần thứ 2 làm được $\dfrac{2}{5}$ kế hoạch, tuần thứ 3 làm được $\dfrac{4}{10}$ kế hoạch. Sau 3 tuần làm việc, tổ sản xuất đã thực hiện vượt kế hoạch hay còn phải làm tiếp bao nhiêu phần kế hoạch công việc của tháng đó?
\end{baitoan}

\begin{baitoan}[\cite{Binh_boi_duong_Toan_6_tap_2}, 3.8., p. 22]
	Minh đã ăn $\dfrac{3}{4}$ của 1 nửa chiếc pizza. Minh đã ăn mấy phần của chiếc pizza?
\end{baitoan}

\begin{baitoan}[\cite{Binh_boi_duong_Toan_6_tap_2}, 3.9., p. 22]
	So sánh: (a) $A = \dfrac{17^{13} + 1}{17^{14} + 1},B = \dfrac{17^{14} + 1}{17^{15} + 1}$. (b) $A = \dfrac{x^a + d}{x^b + d},B = \dfrac{x^b + d}{x^c + d}$ với $x\in\mathbb{Q},a,b,c,d\in\mathbb{N}^\star$ thỏa $a + c = 2b$.
\end{baitoan}

\begin{baitoan}[\cite{Binh_boi_duong_Toan_6_tap_2}, 3.10., p. 22]
	Tính: (a) $A = \dfrac{3}{1\cdot4} + \dfrac{3}{4\cdot7} + \dfrac{3}{7\cdot10} + \dfrac{3}{10\cdot13} + \dfrac{3}{13\cdot16} + \dfrac{3}{16\cdot19}$. (b) $B = \dfrac{1}{1\cdot3} + \dfrac{1}{3\cdot5} + \dfrac{1}{5\cdot7} + \cdots + \dfrac{1}{2021\cdot2023}$. (c) $C = 5 + \dfrac{5}{3} + \dfrac{5}{9} + \dfrac{5}{27} + \cdots + \dfrac{5}{729} + \dfrac{5}{2187}$. (d) $D = \prod_{i=2}^{1000} 1 - \dfrac{1}{i} = \left(1 - \dfrac{1}{2}\right)\left(1 - \dfrac{1}{3}\right)\cdots\left(1 - \dfrac{1}{1000}\right)$. (e) $E = \sum_{i=1}^{98} \dfrac{1}{i(i + 1)(i + 2)} = \dfrac{1}{1\cdot2\cdot3} + \dfrac{1}{2\cdot3\cdot4} + \dfrac{1}{3\cdot4\cdot5} + \cdots + \dfrac{1}{98\cdot99\cdot100}$.
\end{baitoan}

\begin{baitoan}
	Tính: (a) Cho $A_n = \sum_{i=0}^n \dfrac{1}{(a + ib)[a + (i + 1)b]}$, $B_n = \sum_{i=0}^n \dfrac{1}{(a + ib)[a + (i + 1)b][a + (i + 2)b]}$ hay $A_n = \sum_{i=0}^n \dfrac{1}{a_ia_{i+1}}$, $B_n = \sum_{i=0}^n \dfrac{1}{a_ia_{i+1}a_{i+2}}$ với $(a_n)$ là cấp số cộng cho bởi $a_0 = a,a_n = a_{n-1} + b$, $\forall n\in\mathbb{N}^\star$ ($a,b\in\mathbb{R}$ sao cho cấp số cộng này không chứa số $0$). (b) $C_n = \prod_{i=2}^n 1 - \dfrac{1}{i},D_n = \prod_{i=1}^n 1 + \dfrac{1}{i}$. (c) $E_n = \sum_{i=1}^n \dfrac{1}{i(i + 1)(i + 2)}$.	
\end{baitoan}

\begin{baitoan}[\cite{Binh_boi_duong_Toan_6_tap_2}, 3.11., p. 22]
	Chứng minh: (a) $\sum_{i=101}^{200} \dfrac{1}{i} = \dfrac{1}{101} + \dfrac{1}{102} + \cdots + \dfrac{1}{200} > \dfrac{1}{2}$. (b) $\sum_{i=2}^{2050} \dfrac{1}{i^2} = \dfrac{1}{2^2} + \dfrac{1}{3^2} + \cdots + \dfrac{1}{2050^2} < 1$. (c) $\dfrac{2}{10\cdot12} + \dfrac{2}{12\cdot14} + \dfrac{2}{14\cdot16} + \cdots + \dfrac{2}{98\cdot100} < \dfrac{1}{10}$.
\end{baitoan}

\begin{baitoan}
	Cho $n\in\mathbb{N}^\star,a_i > 0,\forall i = 1,2,\ldots,n$. Chứng minh $\dfrac{n}{\max_{1\le i\le n} a_i}\le\sum_{i=1}^n \dfrac{1}{a_i}\le\dfrac{n}{\min_{1\le i\le n} a_i}$.
\end{baitoan}

\begin{baitoan}[\cite{Tuyen_Toan_6}, VD58, p. 54]
	Viết $\dfrac{3}{4}$ thành tổng của của $2,3,4$ phân số Ai Cập khác nhau.
\end{baitoan}

\begin{baitoan}[\cite{Tuyen_Toan_6}, VD, p. 54]
	Cho $b\in\mathbb{N},b > 1$. Chứng minh $\dfrac{1}{b} - \dfrac{1}{b + 1} < \dfrac{1}{b^2} < \dfrac{1}{b - 1} - \dfrac{1}{b}$.
\end{baitoan}

\begin{baitoan}[\cite{Tuyen_Toan_6}, 275., p. 55]
	Tính: (a) $\dfrac{4}{6} + \dfrac{27}{81}$. (b) $\dfrac{48}{96} + \dfrac{-135}{270}$. (c) $\dfrac{30303}{80808} + \dfrac{303030}{484848}$.
\end{baitoan}

\begin{baitoan}[\cite{Tuyen_Toan_6}, 276., p. 55]
	Tính hợp lý: (a) $\left(\dfrac{1}{4} + \dfrac{-5}{13}\right) + \left(\dfrac{2}{11} + \dfrac{-8}{13} + \dfrac{3}{4}\right)$. (b) $\left(\dfrac{21}{31} + \dfrac{-16}{7}\right) + \left(\dfrac{44}{53} + \dfrac{10}{31}\right) + \dfrac{9}{53}$.
\end{baitoan}

\begin{baitoan}[\cite{Tuyen_Toan_6}, 277., p. 55]
	Chứng minh $A = \dfrac{3}{8} + \dfrac{3}{15} + \dfrac{3}{7} > 1,B = \dfrac{19}{60} + \dfrac{29}{100} + \dfrac{39}{150} + \dfrac{49}{300} > 1$.
\end{baitoan}

\begin{baitoan}[\cite{Tuyen_Toan_6}, 278., p. 55]
	Cho $A = 9\dfrac{2}{3} + 6\dfrac{3}{4} + \dfrac{7}{12}$. Tính A bằng 2 cách: (a) Viết các hỗn số dưới dạng phân số. (b) Viết các hỗn số dưới dạng tổng của 1 số nguyên với 1 phân số.
\end{baitoan}

\begin{baitoan}[\cite{Tuyen_Toan_6}, 279., p. 55]
	1 người đi xe đạp từ A đến B hết {\rm5 h}. Người thứ 2 đi xe máy từ B đến A hết {\rm2 h}. Người đi xe máy khởi hành sau người đi xe đạp {\rm2 h}. Sau khi người đi xe máy đi được {\rm1 h} thì 2 người đã gặp nhau chưa?
\end{baitoan}

\begin{baitoan}[\cite{Tuyen_Toan_6}, 280., p. 56]
	Cho 3 vòi nước cùng chảy vào 1 bể cạn. Vòi A chảy 1 mình thì sau {\rm6 h} bể sẽ đầy. Vòi B chảy 1 mình thì mất {\rm3 h} còn vòi C chảy 1 mình thì mất {\rm2 h} mới đầy bể. Mất bao lâu bể sẽ đầy nếu mở cả 3 vòi cùng 1 lúc?
\end{baitoan}

\begin{baitoan}[\cite{Tuyen_Toan_6}, 281., p. 56]
	Viết $\dfrac{1}{3},\dfrac{7}{8}$ thành tổng của 3 phân số Ai Cập. Áp dụng: Tìm cách chia đều $7$ quả táo cho $8$ bạn.
\end{baitoan}

\begin{baitoan}[\cite{Tuyen_Toan_6}, 282., p. 56]
	Tính tổng các phân số lớn hơn $\dfrac{1}{5}$ \& nhỏ hơn $\dfrac{1}{4}$ \& có tử là $4$.
\end{baitoan}

\begin{baitoan}[\cite{Tuyen_Toan_6}, 283., p. 56]
	Cho phân số $A = \dfrac{n + 1}{n - 2}$. Tìm $n\in\mathbb{Z}$ để: (a) $A\in\mathbb{Z}$. (b) $A\ \max$.
\end{baitoan}

\begin{baitoan}
	Cho $a,b,c,d\in\mathbb{Z}$, $A = \dfrac{an + b}{cn + d}$. Tìm $n\in\mathbb{Z}$ để: (a) $A\in\mathbb{Z}$. (b) A $\max$. (c) A $\min$.
\end{baitoan}

\begin{baitoan}[\cite{Tuyen_Toan_6}, 284., p. 56]
	Cho phân số $B = \dfrac{10n}{5n - 3}$. Tìm $n\in\mathbb{Z}$ để: (a) $B\in\mathbb{Z}$. (b) $B\ \max$.
\end{baitoan}

\begin{baitoan}[\cite{Tuyen_Toan_6}, 285., p. 56]
	Cho $A = \dfrac{3}{10} + \dfrac{3}{11} + \dfrac{3}{12} + \dfrac{3}{13} + \dfrac{3}{14}$. Chứng minh $1 < A < 2$, từ đó suy ra $A\notin\mathbb{Z}$.
\end{baitoan}

\begin{baitoan}[\cite{Tuyen_Toan_6}, 286., p. 56]
	Cho $A = \sum_{i=31}^{60} \dfrac{1}{i} = \dfrac{1}{31} + \dfrac{1}{32} + \cdots + \dfrac{1}{60}$. Chứng minh $\dfrac{3}{5} < A < \dfrac{4}{5}$.
\end{baitoan}

\begin{baitoan}[\cite{Tuyen_Toan_6}, 287., p. 56]
	Chứng minh $A = \dfrac{1}{2} + \dfrac{1}{3} + \dfrac{1}{4}\notin\mathbb{Z}$, $B = \sum_{i=2}^{16} \dfrac{1}{i} = \dfrac{1}{2} + \dfrac{1}{3} + \cdots + \dfrac{1}{16}\notin\mathbb{Z}$, $C_n = \sum_{i=2}^{2^n} \dfrac{1}{i}\notin\mathbb{Z}$, $D_{m,n} = \sum_{i=2^m}^{2^n} \dfrac{1}{i}\notin\mathbb{Z}$.
\end{baitoan}

\begin{baitoan}[\cite{Tuyen_Toan_6}, 288., p. 56]
	Tính: (a) $\dfrac{25}{42} - \dfrac{20}{63}$. (b) $\dfrac{9}{50} - \dfrac{13}{75} - \dfrac{1}{6}$. (c) $\dfrac{2}{15} - \dfrac{2}{65} - \dfrac{4}{39}$.
\end{baitoan}

\begin{baitoan}[\cite{Tuyen_Toan_6}, 289., p. 56]
	Tìm $x\in\mathbb{Q}$ thỏa: (a) $x + \dfrac{7}{12} = \dfrac{17}{18} - \dfrac{1}{9}$. (b) $\dfrac{29}{30} - \left(\dfrac{13}{23} + x\right) = \dfrac{7}{69}$.
\end{baitoan}

\begin{baitoan}[\cite{Tuyen_Toan_6}, 290., p. 56]
	Tính hợp lý: (a) $\dfrac{31}{23} - \left(\dfrac{7}{32} + \dfrac{8}{23}\right)$. (b) $\left(\dfrac{1}{3} + \dfrac{12}{67} + \dfrac{13}{41}\right) - \left(\dfrac{79}{67} - \dfrac{28}{41}\right)$. (c) $\dfrac{38}{45} - \left(\dfrac{8}{45} - \dfrac{17}{51} - \dfrac{3}{11}\right)$.
\end{baitoan}

\begin{baitoan}[\cite{Tuyen_Toan_6}, 291., p. 57]
	1 người đi quãng đường AB trong {\rm4 h}. Giờ đầu đi được $\dfrac{1}{3}$ quãng đường AB. Giờ thứ 2 đi kém giờ đầu là $\dfrac{1}{12}$ quãng đường AB. Giờ thứ 3 đi kém giờ thứ 2 là $\dfrac{1}{12}$ quãng đường AB. Giờ thứ 4 đi được mấy phần quãng đường AB?
\end{baitoan}

\begin{baitoan}[\cite{Tuyen_Toan_6}, 292., p. 57]
	Cho phân số $A = \dfrac{6n - 1}{3n + 2}$. Tìm $n\in\mathbb{Z}$ để: (a) $A\in\mathbb{Z}$. (b) $A\ \min$.
\end{baitoan}

\begin{baitoan}[\cite{Tuyen_Toan_6}, 293., p. 57]
	Chứng minh $\dfrac{2}{5} < A = \dfrac{1}{2^2} + \dfrac{1}{3^2} + \cdots + \dfrac{1}{9^2}$.
\end{baitoan}

\subsection{Operator $\cdot,:$ of Fractions -- Phép $\cdot,:$ Phân Số}
SGK \cite[Chap. V, \S4, pp. 40--43]{SGK_Toan_6_Canh_Dieu_tap_2}: LT1. LT2. LT3. LT4. LT5. 1. 2. 3. 4. 5. 6. 7. 8. SBT \cite[Chap. V, \S4, pp. 40--42]{SBT_Toan_6_Canh_Dieu_tap_2}: 38. 39. 40. 41. 42. 43. 44. 45. 46. 47. 48. 49. 50. 51. 52.

\begin{baitoan}[\cite{Tuyen_Toan_6}, VD60, p. 57]
	Cho 2 phân số $\dfrac{a}{b},\dfrac{a}{c}$. Tìm hệ thức giữa $a,b,c$ để tổng của 2 phân số bằng tích của chúng. Áp dụng: Tìm vài cặp phân số sao cho tổng của chúng bằng tích của chúng.
\end{baitoan}

\begin{baitoan}[\cite{Tuyen_Toan_6}, VD61, p. 58]
	Chứng minh tổng của 2 số hữu tỷ dương nghịch đảo nhau thì lớn hơn hoặc bằng $2$.
\end{baitoan}

\begin{baitoan}[\cite{Tuyen_Toan_6}, 294., p. 58]
	Tìm $x\in\mathbb{Z}$ thỏa: (a) $\dfrac{-5}{6}\cdot\dfrac{120}{25} < x < \dfrac{-7}{15}\cdot\dfrac{9}{14}$. (b) $\left(\dfrac{-5}{3}\right)^3 < x < \dfrac{-24}{35}\cdot\dfrac{-5}{6}$.
\end{baitoan}

\begin{baitoan}[\cite{Tuyen_Toan_6}, 295., p. 58]
	Tính: (a) $\left(\dfrac{9}{10} - \dfrac{15}{16}\right)\left(\dfrac{5}{12} - \dfrac{11}{15} - \dfrac{7}{20}\right)$. (b) $\dfrac{-3}{5} + \dfrac{28}{5}\left(\dfrac{43}{56} + \dfrac{5}{24} - \dfrac{21}{63}\right)$.
\end{baitoan}

\begin{baitoan}[\cite{Tuyen_Toan_6}, 296., p. 59]
	Tính hợp lý: (a) $\dfrac{17}{5}\cdot\dfrac{-31}{125}\cdot\dfrac{1}{2}\cdot\dfrac{10}{17}\cdot\dfrac{-1}{2^3}$. (b) $\left(\dfrac{11}{4}\cdot\dfrac{-5}{9} - \dfrac{4}{9}\cdot\dfrac{11}{4}\right)\cdot\dfrac{8}{33}$. (c) $\left(\dfrac{17}{28} + \dfrac{18}{29} - \dfrac{19}{30} - \dfrac{20}{31}\right)\cdot\left(\dfrac{-5}{12} + \dfrac{1}{4} + \dfrac{1}{6}\right)$.
\end{baitoan}

\begin{baitoan}[\cite{Tuyen_Toan_6}, 297., p. 59]
	Tính: (a) $\prod_{i=2}^{99} 1 + \dfrac{1}{i} = \left(1 + \dfrac{1}{2}\right)\left(1 + \dfrac{1}{3}\right)\cdots\left(1 + \dfrac{1}{99}\right)$. (b) $\dfrac{3}{2^2}\cdot\dfrac{8}{3^2}\cdot\dfrac{15}{4^2}\cdots\dfrac{899}{30^2}$. (c) $\prod_{i=2}^{100} \dfrac{1}{i} - 1 = \left(\dfrac{1}{2} - 1\right)\left(\dfrac{1}{3} - 1\right)\cdots\left(\dfrac{1}{100} - 1\right)$.
\end{baitoan}

\begin{baitoan}
	Cho $m,n\in\mathbb{N}^\star,m\le n$. Tính: (a) $A_{m,n} = \prod_{i=m}^n 1 + \dfrac{1}{i}$, $B_{m,n} = \prod_{i=m}^n 1 + \dfrac{2}{i}$. (b) $C_{m,n} = \prod_{i=m}^n \dfrac{1}{i} - 1$, $D_{m,n} = \prod_{i=m}^n 1 - \dfrac{2}{i}$. (c) $E_{m,n} = \prod_{i=m}^n \dfrac{i^2 - 1}{i}$, $F_{m,n} = \prod_{i=m}^n \dfrac{i^2 - 4}{i}$.
\end{baitoan}

\begin{baitoan}[\cite{Tuyen_Toan_6}, 298., p. 59]
	2 người đi bộ cùng khởi hành từ 2 địa điểm $A,B$, đi ngược chiều để gặp nhau. Người thứ nhất đi trong {\rm36 ph} với vận tốc $\dfrac{7}{2}$ {\rm km{\tt/}h} rồi tạm nghỉ. Người thứ 2 đi trong {\rm45 ph} với vận tốc $\dfrac{10}{3}$ {\rm km{\tt/}h} rồi tạm nghỉ. Biết cho đến lúc tạm nghỉ thì họ chưa gặp nhau, còn cách nhau $\dfrac{2}{5}$ {\rm km}. Tính khoảng cách AB.
\end{baitoan}

\begin{baitoan}[\cite{Tuyen_Toan_6}, 299., p. 59]
	Tìm $n\in\mathbb{Z}$ để $\dfrac{19}{n - 1}\cdot\dfrac{n}{9}\in\mathbb{Z}$.
\end{baitoan}

\begin{baitoan}[\cite{Tuyen_Toan_6}, 300., p. 59]
	Tìm số nguyên âm lớn nhất để khi nhân nó với mỗi một trong 3 phân số tối giản $\dfrac{5}{6},\dfrac{-7}{15},\dfrac{11}{21}$ đều được tích là 3 số nguyên.
\end{baitoan}

\begin{baitoan}[\cite{Tuyen_Toan_6}, 301., p. 59]
	Cho $A = \dfrac{1}{2}\cdot\dfrac{3}{4}\cdot\dfrac{5}{6}\cdots\dfrac{99}{100},B = \dfrac{2}{3}\cdot\dfrac{4}{5}\cdot\dfrac{6}{7}\cdots\dfrac{100}{101}$. (a) Chứng minh $A < B$. (b) Tính AB. (c) Chứng minh $A < \dfrac{1}{10}$.
\end{baitoan}

\begin{baitoan}[\cite{Tuyen_Toan_6}, 302., p. 59]
	Tính: (a) $\left(\dfrac{7}{20} + \dfrac{11}{15} - \dfrac{15}{12}\right):\left(\dfrac{11}{20} - \dfrac{26}{45}\right)$. (b) $\dfrac{5 - \dfrac{5}{3} + \dfrac{5}{9} - \dfrac{5}{27}}{8 - \dfrac{8}{3} + \dfrac{8}{9} - \dfrac{8}{27}}:\dfrac{15 - \dfrac{15}{11} + \dfrac{15}{121}}{16 - \dfrac{16}{11} + \dfrac{16}{121}}$. (c) $\dfrac{\dfrac{1}{9} - \dfrac{5}{6} - 4}{\dfrac{7}{12} - \dfrac{1}{36} - 10}$.
\end{baitoan}

\begin{baitoan}[\cite{Tuyen_Toan_6}, 303., p. 60]
	Tìm $x\in\mathbb{Z}$ thỏa: (a) $\left(x + \dfrac{1}{4} - \dfrac{1}{3}\right):\left(2 + \dfrac{1}{6} - \dfrac{1}{4}\right) = \dfrac{7}{46}$. (b) $\dfrac{13}{15} - \left(\dfrac{13}{21} + x\right)\cdot\dfrac{7}{12} = \dfrac{7}{10}$.
\end{baitoan}

\begin{baitoan}[\cite{Tuyen_Toan_6}, 304., p. 60]
	Cho $A = \dfrac{n^9 + 1}{n^{10} + 1},B = \dfrac{n^8 + 1}{n^9 + 1}$ với $n\in\mathbb{N},n > 1$. So sánh nghịch đảo của $A,B$ rồi so sánh $A,B$.
\end{baitoan}

\begin{baitoan}[\cite{Tuyen_Toan_6}, 305., p. 60]
	Cho 2 phân số có tổng bằng $-3$, tích bằng $\dfrac{12}{5}$. Tính tổng các số nghịch đảo của 2 phân số đó.
\end{baitoan}

\begin{baitoan}[\cite{Tuyen_Toan_6}, 306., p. 60]
	Tích của 2 phân số là $\dfrac{2}{5}$. Nếu thêm vào thừa số thứ 2 $3$ đơn vị thì tích là $\dfrac{28}{15}$. Tìm 2 phân số đó.
\end{baitoan}

\begin{baitoan}[\cite{Tuyen_Toan_6}, 307., p. 60]
	1 canô xuôi 1 khúc sông từ A đến B mất {\rm6 h}, ngược dòng từ B đến A mất {\rm7 h 30 ph}. Tính thời gian để 1 khúc gỗ trôi từ A đến B.
\end{baitoan}

\begin{baitoan}[\cite{Tuyen_Toan_6}, 308., p. 60]
	Lúc hơn {\rm3:00}, kim giờ ở trước kim phút đúng $4$ khoảng chia phút. Lúc đó mấy giờ?
\end{baitoan}

\begin{baitoan}[\cite{Tuyen_Toan_6}, 309., p. 60]
	2 vòi nước $A,B$ cùng chảy vào 1 bể. Sau {\rm10 ph}, đóng vòi B, hỏi vòi A phải chảy thêm trong bao lâu nữa thì bể đầy nước, biết 1 mình vòi A chảy đầy bể trong {\rm45 ph}, 1 mình vòi B chảy đầy bể trong {\rm30 ph}.
\end{baitoan}

\begin{baitoan}[\cite{Tuyen_Toan_6}, 310., p. 60]
	Cho $a,b,c\in\mathbb{N}^\star$, $A = \dfrac{a + b}{c} + \dfrac{b + c}{a} + \dfrac{c + a}{b}$. (a) Chứng minh $A\ge6$. (b) Tìm $\min A$.
\end{baitoan}

\begin{baitoan}[\cite{Tuyen_Toan_6}, 311., p. 60]
	Cho $a,b,c\in\mathbb{N}^\star,x,y,z\in\mathbb{Q},x + y + z = 5$. Biết $S_1 = \dfrac{b}{a}x + \dfrac{c}{a}z,S_2 = \dfrac{a}{b}x + \dfrac{c}{b}y,S_3 = \dfrac{a}{c}z + \dfrac{b}{c}y$. Chứng minh $S_1 + S_2 + S_3\ge10$.
\end{baitoan}

\begin{baitoan}[\cite{Binh_Toan_6_tap_2}, VD10, p. 11]
	Rút gọn $A = \dfrac{2 - \dfrac{2}{19} + \dfrac{2}{43} - \dfrac{2}{1943}}{3 - \dfrac{3}{19} + \dfrac{3}{43} - \dfrac{3}{1943}}:\dfrac{4 - \dfrac{4}{29} + \dfrac{4}{41} - \dfrac{4}{2941}}{5 - \dfrac{5}{29} + \dfrac{5}{41} - \dfrac{5}{2941}}$.
\end{baitoan}

\begin{baitoan}
	Cho $m,n\in\mathbb{N}^\star,a,b,c,d,a_i,b_j\in\mathbb{Z}^\star$, $\forall i = 1,2,\ldots,m, j = 1,2,\ldots,n$. Rút gọn:
	\begin{align*}
		\dfrac{\sum_{i=1}^m \dfrac{aa_i}{b_i}}{\sum_{i=1}^m \dfrac{ba_i}{b_i}}\cdot\dfrac{\sum_{i=1}^n \dfrac{cc_i}{d_i}}{\sum_{i=1}^n \dfrac{dc_i}{d_i}}.
	\end{align*}
\end{baitoan}

\begin{baitoan}[\cite{Binh_Toan_6_tap_2}, VD11, p. 12]
	Từ các kết quả: $\dfrac{1}{2}\cdot\dfrac{1}{3} = \dfrac{1}{2} - \dfrac{1}{3},\dfrac{3}{1} - \dfrac{2}{1} = 1,\dfrac{7}{16}\cdot\dfrac{7}{23} = \dfrac{7}{16} - \dfrac{7}{23},\dfrac{23}{7} - \dfrac{16}{7} = 1$, rút ra nhận xét \& chứng minh nhận xét đó. (b) Tính $A = \dfrac{4}{3}\cdot\dfrac{4}{7} + \dfrac{4}{7}\cdot\dfrac{4}{11} +  \dfrac{4}{11}\cdot\dfrac{4}{15} + \cdots + \dfrac{4}{95}\cdot\dfrac{4}{99}$.
\end{baitoan}

\begin{baitoan}[\cite{Binh_Toan_6_tap_2}, VD12, p. 12]
	Tìm $x,y\in\mathbb{Z}$ thỏa $\dfrac{3}{x} + \dfrac{y}{3} = \dfrac{5}{6}$.
\end{baitoan}

\begin{baitoan}[\cite{Binh_Toan_6_tap_2}, VD13, p. 13]
	Khi chia $135$ cho 1 số tự nhiên, được thương bằng $6$ \& còn dư. Tìm số chia \& số dư.
\end{baitoan}

\begin{baitoan}[\cite{Binh_Toan_6_tap_2}, VD14, p. 13]
	Viết phân số $\dfrac{1}{4}$ thành tổng của 2 phân số có tử bằng $1$, mẫu dương \& khác nhau.
\end{baitoan}

\begin{baitoan}[\cite{Binh_Toan_6_tap_2}, VD15, p. 13]
	Tìm 2 chữ số $a,b$ thỏa $\overline{0.ab}\cdot(a + b) = 0.36$.
\end{baitoan}

\begin{baitoan}[\cite{Binh_Toan_6_tap_2}, VD16, p. 14]
	(a) Cho $A = \dfrac{1}{2\dfrac{1}{\square}} + \dfrac{1}{3\dfrac{1}{\square}}$. Điền 2 số $4,5$ vào ô vuông để A có {\rm GTLN}. (b) Cho $B = \dfrac{1}{a + \dfrac{1}{b}} + \dfrac{1}{c + \dfrac{1}{d}} + \dfrac{1}{e + \dfrac{1}{f}}$ với $a,b,c,d,e,f\in\{1,2,3,4,5,6\}$. Dựa vào a), tìm $a,b,c,d,e,f$ để B có {\rm GTLN}.
\end{baitoan}

\begin{baitoan}[\cite{Binh_Toan_6_tap_2}, VD17, p. 15]
	Tìm các giá trị nguyên của biểu thức $a - b + \dfrac{c}{d} + \dfrac{e}{f}$ với $a,b,c,d,e,f$ là các số tự nhiên khác nhau từ $1$ đến $6$, $a > b,d < f$, 2 phân số $\dfrac{c}{d},\dfrac{e}{f}$ không là 2 số tự nhiên.
\end{baitoan}

\begin{baitoan}[\cite{Binh_Toan_6_tap_2}, VD18, p. 15]
	Cho đẳng thức $a + \dfrac{17b}{c} + \dfrac{de}{f} = 47$. Thay $a,b,c,d,e,f$ bởi các số tự nhiên khác nhau từ $1$ đến $6$, trong đó $d < e$, 2 phân số $\dfrac{17b}{c},\dfrac{de}{f}$ đều là 2 số tự nhiên.
\end{baitoan}

\begin{baitoan}[\cite{Binh_Toan_6_tap_2}, 31., p. 17]
	Tính: (a) $\dfrac{2}{5} + \dfrac{-1}{6} - \dfrac{3}{4} - \dfrac{-2}{3}$. (b) $\dfrac{7}{10} - \dfrac{-3}{4} + \dfrac{-5}{6} - \dfrac{1}{5} + \dfrac{-2}{3}$. (c) $\dfrac{\left(\dfrac{1}{2} - 0.75\right)\left(0.2 - \dfrac{2}{5}\right)}{\dfrac{5}{9} - 1\dfrac{1}{12}}$. (d) $\dfrac{\dfrac{2}{3} + \dfrac{2}{7} - \dfrac{1}{14}}{-1 - \dfrac{3}{7} + \dfrac{3}{28}}$.
\end{baitoan}

\begin{baitoan}[\cite{Binh_Toan_6_tap_2}, 32., p. 17]
	Tính nhanh: (a) $\left(\dfrac{84}{13}\right)^2 + \left(\dfrac{35}{13}\right)^2$. (b) $\dfrac{12 - \dfrac{12}{7} - \dfrac{12}{289} - \dfrac{12}{85}}{4 - \dfrac{4}{7} - \dfrac{4}{289} - \dfrac{4}{85}}:\dfrac{3 + \dfrac{3}{13} + \dfrac{3}{169} + \dfrac{3}{91}}{7 + \dfrac{7}{13} + \dfrac{7}{169} + \dfrac{7}{91}}$.\\(c) $\dfrac{1\cdot2 + 2\cdot4 + 3\cdot6 + 4\cdot8 + 5\cdot10}{3\cdot4 + 6\cdot8 + 9\cdot12 + 12\cdot16 + 15\cdot20}$.
\end{baitoan}

\begin{baitoan}[\cite{Binh_Toan_6_tap_2}, 33., p. 18]
	Tìm $x\in\mathbb{R}$ thỏa: (a) $\dfrac{3}{4}x + 1 = -2$. (b) $\dfrac{2}{3} + \dfrac{1}{3}:x = -1$. (c) $|2x - 1| = 5$.
\end{baitoan}

\begin{baitoan}[\cite{Binh_Toan_6_tap_2}, 34., p. 18]
	Tìm $n\in\mathbb{Z}$ để: (a) $A = \dfrac{3n + 4}{n - 1}\in\mathbb{Z}$. (b) $B = \dfrac{6n - 3}{3n + 1}\in\mathbb{Z}$.
\end{baitoan}

\begin{baitoan}[\cite{Binh_Toan_6_tap_2}, 35., p. 18]
	Tính: (a) $A = \dfrac{1.6:\left(1\dfrac{3}{5}\cdot1.25\right)}{0.64 - \dfrac{1}{25}} + \dfrac{\left(1.08 - \dfrac{2}{25}\right):\dfrac{4}{7}}{\left(5\dfrac{5}{9} - 2\dfrac{1}{4}\right)\cdot2\dfrac{2}{17}} + 0.6\cdot0.5:\dfrac{2}{5}$. (b) $8\dfrac{1}{5}\cdot\left(11\dfrac{94}{1591} - 6\dfrac{38}{1517}\right):8\dfrac{11}{43}$. (c) $C = 10101\cdot\left(\dfrac{5}{111111} + \dfrac{5}{222222} - \dfrac{4}{3\cdot7\cdot11\cdot13\cdot37}\right)$.
\end{baitoan}

\begin{baitoan}[\cite{Binh_Toan_6_tap_2}, 36., p. 18]
	(a) Tìm $a,b\in\mathbb{Z}$ thỏa $\dfrac{a}{2} + \dfrac{b}{3} = \dfrac{a + b}{2 + 3}$. (b) Tìm $a,b,c\in\mathbb{Z}$ thỏa $\dfrac{52}{9} = 5 + \dfrac{1}{a + \dfrac{1}{b + \dfrac{1}{c}}}$.
\end{baitoan}

\begin{baitoan}[\cite{Binh_Toan_6_tap_2}, 37., p. 18]
	Tìm 3 chữ số $a,b,c$ khác nhau thỏa $\overline{a.bc}:(a + b + c) = 0.25$.
\end{baitoan}

\begin{baitoan}[\cite{Binh_Toan_6_tap_2}, 38., p. 18]
	Tính tổng các phân số tối giản lớn hơn $15$ nhưng nhỏ hơn $30$ \& có mẫu bằng $17$.
\end{baitoan}

\begin{baitoan}[\cite{Binh_Toan_6_tap_2}, 39., p. 18]
	Tìm $x,y\in\mathbb{Z}$ thỏa: (a) $\dfrac{x}{3} - \dfrac{4}{y} = \dfrac{1}{5}$. (b) $\dfrac{4}{x} + \dfrac{y}{3} = \dfrac{5}{6}$.
\end{baitoan}

\begin{baitoan}[\cite{Binh_Toan_6_tap_2}, 40., p. 18]
	Tìm $x,y\in\mathbb{Z}$ thỏa: (a) $\dfrac{5}{x} - \dfrac{y}{3} = \dfrac{1}{6}$. (b) $\dfrac{x}{6} - \dfrac{2}{y} = \dfrac{1}{30}$.
\end{baitoan}

\begin{baitoan}[\cite{Binh_Toan_6_tap_2}, 41., p. 18]
	Thay $a,b,c,d,e$ bằng các chữ số khác nhau, khác $0$ \& khác 4 chữ số đã có $(3,4,6,8)$ để $\dfrac{a}{34} + \dfrac{b}{68} + \dfrac{c}{\overline{mn}} = 1$.
\end{baitoan}

\begin{baitoan}[\cite{Binh_Toan_6_tap_2}, 42., p. 19]
	Chứng minh phân số có thể viết được dưới dạng tổng các phân số có tử bằng $1$, mẫu dương \& khác nhau: (a) $\dfrac{1}{6}$. (b) $\dfrac{15}{22}$. (c) $\dfrac{5}{11}$.
\end{baitoan}

\begin{baitoan}[\cite{Binh_Toan_6_tap_2}, 43., p. 19]
	Tìm số bị chia, số chia biết khi cộng số bị chia với $10$ \& nhân số chia với $10$ thì thương không đổi.
\end{baitoan}

\begin{baitoan}[\cite{Binh_Toan_6_tap_2}, 44., p. 19]
	Cho A là số tự nhiên được viết bởi $2004$ chữ số $4$. Tính phần thập phân của phép chia A cho $15$.
\end{baitoan}

\begin{baitoan}[\cite{Binh_Toan_6_tap_2}, 45., p. 19]
	Chứng minh $2^{100}$ là số có $31$ chữ số khi viết kết quả của nó trong hệ thập phân.
\end{baitoan}

\begin{baitoan}[\cite{Binh_Toan_6_tap_2}, 46., p. 19]
	Chứng minh $\dfrac{1}{3} + \dfrac{1}{31} + \dfrac{1}{35} + \dfrac{1}{37} + \dfrac{1}{47} + \dfrac{1}{53} + \dfrac{1}{61} < \dfrac{1}{2}$.
\end{baitoan}

\begin{baitoan}[\cite{Binh_Toan_6_tap_2}, 47., p. 19]
	Cho $a,b,c$ là 3 số nguyên tố khác nhau đôi một. Chứng minh $\dfrac{1}{[a,b]} + \dfrac{1}{[b,c]} + \dfrac{1}{[c,a]}\le\dfrac{1}{3}$.
\end{baitoan}

\begin{baitoan}[\cite{Binh_Toan_6_tap_2}, 48., p. 19]
	(a) Tìm $a\in\mathbb{N}^\star$ nhỏ nhất sao cho khi nhân a với $\dfrac{5}{12}$, với $\dfrac{10}{21}$, đều được tích là 2 số tự nhiên. (b) Tìm phân số nhỏ nhất khác $0$ sao cho khi chia nó cho $\dfrac{14}{9}$, cho $\dfrac{45}{27}$, đều được thương là 2 số tự nhiên.
\end{baitoan}

\begin{baitoan}[\cite{Binh_Toan_6_tap_2}, 49., p. 19]
	Tìm phân số lớn nhất sao cho khi chia 3 phân số $\dfrac{28}{15},\dfrac{21}{10},\dfrac{49}{84}$ cho nó, đều được thương là 3 số tự nhiên.
\end{baitoan}

\begin{baitoan}[\cite{Binh_Toan_6_tap_2}, 50., p. 19]
	Tìm các phân số có tử \& mẫu đều dương sao cho tổng của phân số đó với nghịch đảo của nó có {\rm GTNN}.
\end{baitoan}

\begin{baitoan}[\cite{Binh_Toan_6_tap_2}, 51., p. 19]
	Tìm {\rm GTLN, GTNN} của thương trong phép chia 1 số tự nhiên có 3 chữ số cho tổng 3 chữ số của nó.
\end{baitoan}

\begin{baitoan}[\cite{Binh_Toan_6_tap_2}, 52., p. 19]
	Tìm 2 số tự nhiên sao cho tích của 2 số đó gấp 4 lần tổng của chúng.
\end{baitoan}

\begin{baitoan}[\cite{Binh_Toan_6_tap_2}, 53., p. 19]
	Viết $\dfrac{1}{6}$ thành tổng của 2 phân số có tử bằng $1$, mẫu dương \& khác nhau.
\end{baitoan}

\begin{baitoan}[\cite{Binh_Toan_6_tap_2}, 54., p. 19]
	Tìm 2 phân số có tử bằng $1$, 2 mẫu dương, biết tổng của 2 phân số ấy cộng với tích của chúng bằng $\dfrac{1}{2}$.
\end{baitoan}

\begin{baitoan}[\cite{Binh_Toan_6_tap_2}, 55., p. 19]
	Tìm 3 số tự nhiên khác nhau có tổng các nghịch đảo của chúng bằng 1 số tự nhiên.
\end{baitoan}

\begin{baitoan}[\cite{Binh_Toan_6_tap_2}, 56., p. 20]
	Tìm 4 số tự nhiên sao cho tổng nghịch đảo các bình phương của chúng bằng $1$.
\end{baitoan}

\begin{baitoan}[\cite{Binh_Toan_6_tap_2}, 57., p. 20]
	Tìm 3 số nguyên tố $a,b,c$ khác nhau sao cho $abc < ab + bc + ca$.
\end{baitoan}

\begin{baitoan}[\cite{Binh_Toan_6_tap_2}, 58., p. 20]
	An rót 1 cốc đầy nước chè, uống $\dfrac{1}{6}$ cốc, đổ thêm nước lọc cho đầy, uống $\dfrac{1}{3}$ cốc rồi đổ thêm đầy nước lọc, lại uống $\dfrac{1}{2}$ cốc, đổ đầy nước lọc rồi uống hết. An uống nước lọc hay uống nước chè nhiều hơn?
\end{baitoan}

\begin{baitoan}[\cite{Binh_Toan_6_tap_2}, 59., p. 20]
	Bác Tâm ra chợ bán 2 loại ổi có số lượng bằng nhau: loại to bán $1000$ đồng 2 quả, loại nhỏ bán $1000$ đồng 3 quả. Vì có việc bận nên bác giao cho con gái bán số ổi đó. Cô con gái bán tất cả số ỏi cho 1 người nên cô tính như sau cho gọn: cứ $5$ quả giá $2000$ đồng. Đến khi kiểm lại tiền thì cô thấy bị hụt đi $5000$ đồng so với số tiền phải bán. Giải thích vì sao bị hụt đi $5000$ đồng? Tính số ổi mang đi bán.
\end{baitoan}

\begin{baitoan}[\cite{Binh_Toan_6_tap_2}, 60., p. 20]
	2 người lái buôn Ả Rập lúc nghỉ góp bánh ăn chung, Ali góp $3$ cái, Xaba góp $5$ cái. Vừa lúc đó, 1 người khách đến \& họ mời người đó ăn cùng. Ăn xong, người khách trả cho 2 người kia $8$ đina. Ali nói với Xaba: Anh góp $5$ cái thì nhận $5$ đina, tôi góp $3$ cái thì nhận $3$ đina. Xaba không đồng ý. 1 người qua đường biết chuyện đã giải quyết cho Ali nhận $1$ đina, Xaba nhận $7$ đina \& giải thích rõ để 2 người vui vẻ nhận. Người qua đường đã giải thích như thế nào?
\end{baitoan}

\begin{baitoan}[\cite{Binh_Toan_6_tap_2}, 61., p. 20]
	Ông Lưu có 1 bình chứa đầy {\rm10 l} rượu. Lần thứ nhất, ông dùng {\rm1 l}, rồi đổ nước lọc vào cho đầy bình. Lần thứ 2, ông dùng {\rm1 l}, rồi đổ nước lọc vào cho đầy bình. Cứ như vậy đến lần thứ 6. Ông Ly nói với ông: Thế thì bây giờ bình rượu của ông chỉ còn {\rm4 l} rượu nguyên chất. Ông Lưu vẫn còn tỉnh táo nói: Bình rượu của tôi phải có hơn {\rm5 l} rượu nguyên chất. Ai đúng ai sai?
\end{baitoan}

\begin{baitoan}[\cite{TLCT_THCS_Toan_6_so_hoc}, VD10.1, p. 59]
	Tìm $a,b\in\mathbb{N}$ thỏa $\dfrac{a}{5} + \dfrac{b}{3} = \dfrac{13}{15}$.
\end{baitoan}

\begin{baitoan}[\cite{TLCT_THCS_Toan_6_so_hoc}, VD10.2, p. 59]
	(a) Cho trước phân số $\dfrac{a}{b}\ne-1$. Tìm phân số $\dfrac{c}{d}$ thỏa $\dfrac{a}{b} - \dfrac{c}{d} = \dfrac{a}{b}\cdot\dfrac{c}{d}$. (b) Tìm phân số $\dfrac{c}{d}$ có tính chất trên, nếu phân số $\dfrac{a}{b}$ bằng $\dfrac{1}{3},\dfrac{3}{5}$.
\end{baitoan}

\begin{baitoan}[\cite{TLCT_THCS_Toan_6_so_hoc}, VD10.3, p. 60]
	Tính $A = \dfrac{1}{2} + \dfrac{5}{6} + \dfrac{11}{12} + \dfrac{19}{20} + \dfrac{29}{30} + \dfrac{41}{42} + \dfrac{55}{56} + \dfrac{71}{72} + \dfrac{89}{90}$.
\end{baitoan}

\begin{baitoan}[\cite{TLCT_THCS_Toan_6_so_hoc}, VD10.4, p. 60]
	Cho $A = \dfrac{1}{4}\cdot\dfrac{3}{6}\cdot\dfrac{5}{8}\cdots\dfrac{43}{46}\cdot\dfrac{45}{48},B = \dfrac{2}{5}\cdot\dfrac{4}{7}\cdot\dfrac{6}{9}\cdots\dfrac{44}{47}\cdot\dfrac{46}{49}$. (a) So sánh $A,B$. (b) Chứng minh $A < \dfrac{1}{133}$.
\end{baitoan}

\begin{baitoan}[\cite{TLCT_THCS_Toan_6_so_hoc}, VD10.5, p. 61]
	Tìm phân số $\dfrac{a}{b}$ lớn nhất sao cho khi chia mỗi phân số $\dfrac{12}{35},\dfrac{8}{21},\dfrac{52}{91}$ cho $\dfrac{a}{b}$, ta đều được các số tự nhiên.
\end{baitoan}

\begin{baitoan}[\cite{TLCT_THCS_Toan_6_so_hoc}, VD10.6, p. 62]
	Tìm $n\in\mathbb{Z}$ để phân số $\dfrac{20n + 13}{4n + 3}$ có {\rm GTNN}.
\end{baitoan}

\begin{baitoan}[\cite{TLCT_THCS_Toan_6_so_hoc}, VD10.7, p. 62]
	Tìm $x,y\in\mathbb{N}$ thỏa $\dfrac{1}{x} + \dfrac{1}{y} = \dfrac{1}{8}$.
\end{baitoan}

\begin{baitoan}[\cite{TLCT_THCS_Toan_6_so_hoc}, VD10.8, p. 63]
	Viết tiếp các phân số vào dãy các phân số có quy luật: $3,4\dfrac{1}{2},6\dfrac{3}{4},10\dfrac{1}{8},15\dfrac{3}{16}$.
\end{baitoan}

\begin{baitoan}[\cite{TLCT_THCS_Toan_6_so_hoc}, VD10.9, p. 63]
	Thay $a,b$ bởi 2 chữ số thích hợp: $\overline{0.ab}\cdot(a + b) = 0.36$.
\end{baitoan}

\begin{baitoan}[\cite{TLCT_THCS_Toan_6_so_hoc}, 10.1., p. 64]
	Tìm phân số $\dfrac{a}{b}$, biết nó bằng trung bình cộng của 3 phân số $\dfrac{7}{18},\dfrac{11}{18},dfrac{a}{b}$.
\end{baitoan}

\begin{baitoan}[\cite{TLCT_THCS_Toan_6_so_hoc}, 10.2., p. 64]
	Chứng minh phân số có thể viết được dưới dạng tổng của 2 phân số có tử bằng $1$, mẫu khác nhau: (a) $\dfrac{7}{10}$. (b) $\dfrac{2}{3}$.
\end{baitoan}

\begin{baitoan}[\cite{TLCT_THCS_Toan_6_so_hoc}, 10.3., p. 64]
	Chứng minh phân số có thể viết được dưới dạng tổng của 3 phân số có tử bằng $1$, mẫu khác nhau: (a) $\dfrac{17}{18}$. (b) $\dfrac{5}{8}$.
\end{baitoan}

\begin{baitoan}[\cite{TLCT_THCS_Toan_6_so_hoc}, 10.4., p. 64]
	Tìm $x,y\in\mathbb{Z}$ thỏa: (a) $\dfrac{x}{10} - \dfrac{1}{y} = \dfrac{3}{10}$. (b) $\dfrac{1}{x} + \dfrac{y}{2} = \dfrac{5}{8}$. 
\end{baitoan}

\begin{baitoan}[\cite{TLCT_THCS_Toan_6_so_hoc}, 10.5., p. 64]
	Cho phân số $\dfrac{a}{b}\ne1$. Tìm phân số $\dfrac{c}{d}$ thỏa $\dfrac{a}{b} + \dfrac{c}{d} = \dfrac{a}{b}\cdot\dfrac{c}{d}$.
\end{baitoan}

\begin{baitoan}[\cite{TLCT_THCS_Toan_6_so_hoc}, 10.6., p. 64]
	Tính $\dfrac{3 - \dfrac{3}{20} + \dfrac{3}{13} - \dfrac{3}{2013}}{7 - \dfrac{7}{20} + \dfrac{7}{13} - \dfrac{7}{2013}}$.
\end{baitoan}

\begin{baitoan}[\cite{TLCT_THCS_Toan_6_so_hoc}, 10.7., p. 64]
	Tính $\dfrac{1}{1\cdot5} + \dfrac{1}{5\cdot9} + \dfrac{1}{9\cdot13} + \dfrac{1}{13\cdot17} + \cdots + \dfrac{1}{41\cdot45}$.
\end{baitoan}

\begin{baitoan}[\cite{TLCT_THCS_Toan_6_so_hoc}, 10.8., p. 65]
	Cho $A = \dfrac{1}{31} + \dfrac{1}{32} + \dfrac{1}{33} + \cdots + \dfrac{1}{60}$. Chứng minh $A > \dfrac{7}{12}$.
\end{baitoan}

\begin{baitoan}[\cite{TLCT_THCS_Toan_6_so_hoc}, 10.9., p. 65]
	Cho $A = \sum_{i=3}^{50} \dfrac{1}{i^2} = \dfrac{1}{3^2} + \dfrac{1}{4^2} + \cdots + \dfrac{1}{50^2}$. Chứng minh: (a) $A > \dfrac{1}{4}$. (b) $A < \dfrac{4}{9}$.
\end{baitoan}

\begin{baitoan}[\cite{TLCT_THCS_Toan_6_so_hoc}, 10.10., p. 65]
	Tính $\dfrac{3}{4}\cdot\dfrac{8}{9}\cdot\dfrac{15}{16}\cdot\dfrac{24}{25}\cdot\dfrac{35}{36}\cdot\dfrac{48}{49}\cdot\dfrac{63}{64}$.
\end{baitoan}

\begin{baitoan}[\cite{TLCT_THCS_Toan_6_so_hoc}, 10.11., p. 65]
	Cho $A = \dfrac{1}{2}\cdot\dfrac{3}{4}\cdot\dfrac{5}{6}\cdot\dfrac{7}{8}\cdots\dfrac{79}{80}$. Chứng minh $A < \dfrac{1}{9}$.
\end{baitoan}

\begin{baitoan}[\cite{TLCT_THCS_Toan_6_so_hoc}, 10.12., p. 65]
	Chứng minh $1\cdot3\cdot5\cdots19 = \dfrac{11}{2}\cdot\dfrac{12}{2}\cdot\dfrac{13}{2}\cdots\dfrac{20}{2}\cdot$.
\end{baitoan}

\begin{baitoan}[\cite{TLCT_THCS_Toan_6_so_hoc}, 10.13., p. 65]
	Chứng minh $1 - \dfrac{1}{2} + \dfrac{1}{3} - \dfrac{1}{4} + \dfrac{1}{5} - \dfrac{1}{6} + \cdots + \dfrac{1}{20} = \dfrac{1}{11} + \dfrac{1}{12} + \dfrac{1}{13} + \cdots + \dfrac{1}{20}$.
\end{baitoan}

\begin{baitoan}[\cite{TLCT_THCS_Toan_6_so_hoc}, 10.14., p. 65]
	Tính $\dfrac{\dfrac{1}{19} + \dfrac{2}{18} + \dfrac{3}{17} + \cdots + \dfrac{18}{2} + \dfrac{19}{1}}{\dfrac{1}{2} + \dfrac{1}{3} + \dfrac{1}{4} + \cdots + \dfrac{1}{19} + \dfrac{1}{20}}$.
\end{baitoan}

\begin{baitoan}[\cite{TLCT_THCS_Toan_6_so_hoc}, 10.15., p. 65]
	Tính: (a) $A = \sum_{i=1}^9 \dfrac{1}{2^i} = \dfrac{1}{2} + \dfrac{1}{2^2} + \dfrac{1}{2^3} + \cdots + \dfrac{1}{2^9}$. (b) $B = \dfrac{1}{4} + \dfrac{1}{12} + \dfrac{1}{36} + \dfrac{1}{108} + \dfrac{1}{324} + \dfrac{1}{972}$.
\end{baitoan}

\begin{baitoan}[\cite{TLCT_THCS_Toan_6_so_hoc}, 10.16., p. 65]
	Tìm $a,b\in\mathbb{Q}$ thỏa $a + b = 3(a - b) = 2\cdot\dfrac{a}{b}$.
\end{baitoan}

\begin{baitoan}[\cite{TLCT_THCS_Toan_6_so_hoc}, 10.17., p. 65]
	Cho $\dfrac{a}{b} = \sum_{i=2}^9 \dfrac{1}{i} = \dfrac{1}{2} + \dfrac{1}{3} + \cdots + \dfrac{1}{9}$. Chứng minh $a\divby11$.
\end{baitoan}

\begin{baitoan}[\cite{TLCT_THCS_Toan_6_so_hoc}, 10.18., p. 66]
	Chứng minh $\sum_{i=2}^{50} \dfrac{1}{i} = \dfrac{1}{2} + \dfrac{1}{3} + \cdots + \dfrac{1}{50}\notin\mathbb{N}$.
\end{baitoan}

\begin{baitoan}[\cite{TLCT_THCS_Toan_6_so_hoc}, 10.19., p. 66]
	Tìm $a\in\mathbb{N}$ nhỏ nhất, biết nhân $a$ với $\dfrac{8}{15}$ hoặc $\dfrac{21}{36}$ thì 2 kết quả đều là số tự nhiên.
\end{baitoan}

\begin{baitoan}[\cite{TLCT_THCS_Toan_6_so_hoc}, 10.20., p. 66]
	So sánh $A = \dfrac{8^9 + 12}{8^9 + 7},B = \dfrac{8^{10} + 4}{8^{10} - 1}$.
\end{baitoan}

\begin{baitoan}[\cite{TLCT_THCS_Toan_6_so_hoc}, 10.21., p. 66]
	Tìm $n\in\mathbb{Z}$ để phân số $\dfrac{4n + 9}{2n + 3}$ có {\rm GTLN}.
\end{baitoan}

\begin{baitoan}[\cite{TLCT_THCS_Toan_6_so_hoc}, 10.22., p. 66]
	Tìm số tự nhiên có 2 chữ số sao cho tỷ số của số đó \& tổng các chữ số của nó có {\rm GTNN}.
\end{baitoan}

\begin{baitoan}[\cite{TLCT_THCS_Toan_6_so_hoc}, 10.23., p. 66]
	Tìm tỷ số lớn nhất của số tự nhiên có 3 chữ số \& tổng các chữ số của nó.
\end{baitoan}

\begin{baitoan}[\cite{TLCT_THCS_Toan_6_so_hoc}, 10.24., p. 66]
	So sánh $A = \dfrac{7^{10}}{1 + 7 + 7^2 + \cdots + 7^9},B = \dfrac{5^{10}}{1 + 5 + 5^2 + \cdots + 5^9}$.
\end{baitoan}

\begin{baitoan}[\cite{TLCT_THCS_Toan_6_so_hoc}, 10.25., p. 66]
	Tìm 3 số nguyên dương khác nhau sao cho tổng các nghịch đảo của chúng bằng $1$.
\end{baitoan}

%------------------------------------------------------------------------------%

\section{Decimal. Round Up Decimal -- Số Thập phân. Làm Tròn Số Thập Phân}
\fbox{1} Mẫu của \textit{phân số thập phân} là lũy thừa của 10, i.e., $\dfrac{a}{10^n}$, $\forall a\in\mathbb{Z},n\in\mathbb{N}^\star$. \fbox{2} Số chữ số của phần thập phân đúng bằng số chữ số 0 ở mẫu của phân số thập phân. \fbox{3} Số thập phân gồm \textit{phần số nguyên} bên trái dấu thập phân \& \textit{phần thập phân} bên phải dấu thập phân. \fbox{4} Số thập phân âm $< 0 <$ số thập phân dương. \fbox{5} Với 2 số thập phân $a,b$: $a > b\Leftrightarrow -a < -b$.\\

\noindent SGK \cite[Chap. V, \S5, pp. 44--47]{SGK_Toan_6_Canh_Dieu_tap_2}: LT1. LT2. LT3. 1. 2. 3. 4. 5. SBT \cite[Chap. V, \S5, pp. 44--45]{SBT_Toan_6_Canh_Dieu_tap_2}: 53. 54. 55. 56. 57. 58. 59. 60. 61. 62. 63. 64. 65. SGK \cite[Chap. V, \S6, pp. 48--51]{SGK_Toan_6_Canh_Dieu_tap_2}: LT1. LT2. LT3. LT4. LT5. 1. 2. 3. 4. 5. 6. SBT \cite[Chap. V, \S5, pp. 46--48]{SBT_Toan_6_Canh_Dieu_tap_2}: 66. 67. 68. 69. 70. 71. 72. SGK \cite[Chap. V, \S7, pp. 52--56]{SGK_Toan_6_Canh_Dieu_tap_2}: LT1. LT2. LT3. 1. 2. 3. 4. 5. 4. 5. 6. 7. 8. 9. SBT \cite[Chap. V, \S7, pp. 49--50]{SBT_Toan_6_Canh_Dieu_tap_2}: 73. 74. 75. 76. 77. 78. 79. 80. 81. 82. SGK \cite[Chap. V, \S8, pp. 57--60]{SGK_Toan_6_Canh_Dieu_tap_2}: LT1. LT2. 1. 2. 3. 4. SBT \cite[Chap. V, \S8, pp. 51--52]{SBT_Toan_6_Canh_Dieu_tap_2}: 83. 84. 85. 86. 87. 88. 89. 90. 91. 92.

\begin{baitoan}[{\sf Program}: Round decimal]
	Viết chương trình {\sf Pascal, Python, C{\tt/}C++} để làm tròn số thập phân với độ chính xác cho trước.
\end{baitoan}

\begin{baitoan}[{\sf Program}: Interchange between fraction \& decimal]
	Viết chương trình {\sf Pascal, Python, C{\tt/}C++} để chuyển đổi giữa số thập phân \& phân số.
\end{baitoan}

\begin{baitoan}[\cite{Binh_boi_duong_Toan_6_tap_2}, H1, p. 29]
	Tìm phân số thập phân: $\dfrac{5}{3},\dfrac{6}{100},\dfrac{-2}{11},\dfrac{5}{20}$.
\end{baitoan}

\begin{baitoan}[\cite{Binh_boi_duong_Toan_6_tap_2}, H2, p. 29]
	Viết phân số $\dfrac{-276}{10^4}$ dưới dạng số thập phân.
\end{baitoan}

\begin{baitoan}[\cite{Binh_boi_duong_Toan_6_tap_2}, H3, p. 29]
	{\rm Đ{\tt/}S?} Nếu sai, sửa cho đúng. (a) Mọi phân số đều viết được dưới dạng số thập phân. (b) Mọi số thập phân đều viết được dưới dạng phân số thập phân. (c) $0$ là số thập phân âm. (d) $-12.1$ có phần số nguyên là $12$.
\end{baitoan}

\begin{baitoan}[\cite{Binh_boi_duong_Toan_6_tap_2}, H4, p. 29]
	Làm tròn số $0.0276$ đến hàng phần trăm.
\end{baitoan}

\begin{baitoan}[\cite{Binh_boi_duong_Toan_6_tap_2}, VD1, p. 29]
	Trong 5 phân số $\dfrac{-2}{5},\dfrac{4}{3},\dfrac{9}{40},\dfrac{-7}{11},\dfrac{7}{8}$. Phân số nào viết được dưới dạng phân số thập phân? Viết chúng dưới dạng phân số thập phân \& số thập phân.
\end{baitoan}

\begin{baitoan}
	Tìm dạng tổng quát của các phân số có thể viết được thành phân số thập phân.
\end{baitoan}

\begin{baitoan}[\cite{Binh_boi_duong_Toan_6_tap_2}, VD2, p. 30]
	Sắp giảm dần: $25.56,-6.072,-6.7,2.84,235.6$. (b) Sắp tăng dần: $53\%,\dfrac{5}{11},-6.7,-6\dfrac{3}{5},\dfrac{1}{2},-7.635$.
\end{baitoan}

\begin{baitoan}[\cite{Binh_boi_duong_Toan_6_tap_2}, VD3, p. 30]
	Tìm 5 số thập phân nằm giữa 2 số thập phân $0.1,0.2$.
\end{baitoan}

\begin{baitoan}[\cite{Binh_boi_duong_Toan_6_tap_2}, 5.1., p. 31]
	Viết phân số dưới dạng phân số thập phân \& số thập phân: $\dfrac{-4}{5},\dfrac{29}{50},\dfrac{34}{16},\dfrac{-101}{25}$.
\end{baitoan}

\begin{baitoan}[\cite{Binh_boi_duong_Toan_6_tap_2}, 5.3., p. 31]
	Tìm chữ số $x,y,z$ thỏa: (a) $1.254\ge\overline{1.x54}$. (b) $\overline{85.03y4} < 85.04$. (c) $4.3531 < \overline{4.3z6} < 4.37$.
\end{baitoan}

\begin{baitoan}[\cite{Binh_boi_duong_Toan_6_tap_2}, 5.4., p. 31]
	Làm tròn số: (a) $7.235$ đến hàng phần trăm. (b) $0.19962$ đến chữ số thập phân thứ 3. (c) $345769.198$ đến chữ số hàng chục.
\end{baitoan}

\begin{baitoan}[\cite{Binh_boi_duong_Toan_6_tap_2}, 5.5., p. 31]
	Gia tốc trọng trường đo được tại 1 số hành tinh: Mộc tinh $\rm23.12\ m{\tt/}s^2$, Thổ tinh $\rm8.96\ m{\tt/}s^2$, Thiên Vương tinh $\rm8.69\ m{\tt/}s^2$, Hỏa tinh $\rm3.69\ m{\tt/}s^2$. Làm tròn chúng đến hàng phần mười.
\end{baitoan}

\begin{baitoan}[\cite{Binh_boi_duong_Toan_6_tap_2}, 5.6., p. 31]
	Sắp tăng dần: $\dfrac{10}{3},-0.2,-\dfrac{1}{4},3.33,0.5,\dfrac{3}{2},50.7\%$.
\end{baitoan}

\begin{baitoan}[\cite{Binh_boi_duong_Toan_6_tap_2}, 5.7., p. 32]
	Viết 2 số thập phân lớn hơn $0.11$ đồng thời bé hơn $\dfrac{1}{9}$.
\end{baitoan}

\begin{baitoan}[\cite{Binh_boi_duong_Toan_6_tap_2}, 5.8., p. 32]
	Tìm 1 số thập phân có $4$ chữ số, biết khi viết các chữ số theo thứ tự ngược lại \& giữ nguyên vị dấu thập phân thì được số thập phân mới bằng số thập phân ban đầu \& kết quả làm tròn số thập phân đã cho đến hàng đơn vị là $38$.
\end{baitoan}
Bằng các chữ số từ $0$ đến $9$, có thể viết được các số vô cùng lớn \& cả các số vô cùng bé.

\begin{baitoan}[\cite{Binh_boi_duong_Toan_6_tap_2}, p. 32, speed of light -- tốc độ ánh sáng]
	Tốc độ ánh sáng, 1 cách tổng quát hơn, tốc độ lan truyền của bức xạ điện tử trong chân không là 1 hằng số vật lý cơ bản quan trọng trong nhiều lĩnh vực Vật lý, có giá trị chính xác bằng {\rm299792458 m{\tt/}s}. Làm tròn số này đến các hàng có thể.
\end{baitoan}

\begin{baitoan}[\cite{Binh_boi_duong_Toan_6_tap_2}, p. 32, diameter of corona virus cell -- đường kính tế bào virus corona]
	Virus corona gây bệnh Covid-19 có dạng hình cầu với đường kính khoảng {\rm0.000125 mm}. Đổi đơn vị sang m, $\mu$m.
\end{baitoan}

%------------------------------------------------------------------------------%

\section{Calculus on Decimal -- Các Phép Tính Với Số Thập Phân}
\fbox{1} Nhân nhẩm với $10^n,n\in\mathbb{N}$, dịch dấu thập phân sang phải $n$ chữ số. \fbox{2} Chia nhẩm với $10^n,n\in\mathbb{N}$, i.e., nhân nhẩm với $10^{-n} = \underbrace{0.0\ldots0}_n1$, dịch dấu thập phân sang trái $n$ chữ số.

\begin{baitoan}[\cite{Binh_boi_duong_Toan_6_tap_2}, H1, p. 33]
	Tính tổng cân nặng 2 bao gạo có cân nặng lần lượt là {\rm5.03 kg, 4.9 kg}.
\end{baitoan}

\begin{baitoan}[\cite{Binh_boi_duong_Toan_6_tap_2}, H2, p. 33]
	Do dịch Covid-19, 1 công ty đạt lợi nhuận $-0.75$ tỷ USD trong quý I \& $0.34$ tỷ USD trong quý II. (a) Cả 2 quý công ty đạt lợi nhuận bao nhiêu? (b) Lợi nhuận của quý II nhiều hơn quý I bao nhiêu?
\end{baitoan}

\begin{baitoan}[\cite{Binh_boi_duong_Toan_6_tap_2}, H3--H4, p. 33]
	Tính: (a) $-3.15\cdot(-0.1)$. (b) $7.68:(-0.24)$.
\end{baitoan}

\begin{baitoan}[\cite{Binh_boi_duong_Toan_6_tap_2}, VD1, p. 34]
	Tính: (a) $-32.721 + 27.04$. (b) $-2.49 -(-3.156)$. (c) $-2.15\cdot2.03$. (d) $10.165:(-2.14)$.
\end{baitoan}

\begin{baitoan}[\cite{Binh_boi_duong_Toan_6_tap_2}, VD2, p. 34]
	Tính: (a) $A = -13.8 -(-47.53) - \dfrac{316}{5} + 52.47$. (b) $B = 0.25\cdot1.25\cdot40\cdot8\cdot0.002\cdot5$. (c) $C = 3.15\cdot(-2.6) + 3.15\cdot\dfrac{3}{5} - 315\cdot0.08$.
\end{baitoan}

\begin{baitoan}[\cite{Binh_boi_duong_Toan_6_tap_2}, VD3, p. 35]
	Doanh thu 2 tháng đầu năm của 1 cửa hàng thực phẩm đạt $107.1$ triệu đồng, trong đó doanh thu tháng 1 bằng $0.7$ lần doanh thu tháng 2. Tính doanh thu mỗi tháng trong 2 tháng đầu năm của cửa hàng.
\end{baitoan}

\begin{baitoan}[\cite{Binh_boi_duong_Toan_6_tap_2}, VD4, p. 35]
	Khi thực hiện phép cộng 2 số thập phân, An đã viết nhầm dấu phẩy của 1 số hạng sang bên phải 1 chữ số, do đó dẫn đến kết quả sai là $692.22$. An đã cộng 2 số nào? Biết tổng đúng là $100.56$.
\end{baitoan}

\begin{baitoan}[\cite{Binh_boi_duong_Toan_6_tap_2}, VD5, p. 35]
	Thay các chữ số $a,b,c,d$ bởi các chữ số thích hợp để được phép tính đúng: $\overline{7ab.a} + \overline{c14.d} = \overline{d51.c}$.
\end{baitoan}

\begin{baitoan}[\cite{Binh_boi_duong_Toan_6_tap_2}, 6.1., p. 36]
	Vì phép đo thường có sai số nên để chính xác, người ta thường thực hiện phép đo nhiều lần rồi lấy kết quả trung bình cộng của các lần đo. Nam đo khối lượng riêng của 1 viên sỏi $\rm g{\tt/}m^3$ 3 lần được kết quả: $5,4.769,5.167$. Tìm khối lượng riêng của viên sỏi dựa vào kết quả đo của Nam.
\end{baitoan}

\begin{baitoan}[\cite{Binh_boi_duong_Toan_6_tap_2}, 6.2., p. 36]
	Hưởng ứng Tết trồng cây, người ta trồng 1 cây bạch đàn với chiều cao ban đầu {\rm40 cm}. Sau 4 năm, cây bạch đàn có chiều cao {\rm6.73 m}. Cây bạch đàn đã cao gấp cây con ban đầu mấy lần? Làm tròn kết quả đến hàng phần mười.
\end{baitoan}

\begin{baitoan}[\cite{Binh_boi_duong_Toan_6_tap_2}, 6.3., p. 36]
	1 phòng học có các kích thước dài, rộng, cao lần lượt là {\rm12 m, 8.2 m, 3.8 m}. Biết thể tích oxygen chiếm khoảng $21\%$ thể tích không khí. Ước lượng thể tích khí oxygen trong phòng học này, làm tròn kết quả đến hàng phần mười.
\end{baitoan}

\begin{baitoan}[\cite{Binh_boi_duong_Toan_6_tap_2}, 6.4., p. 36]
	Tính hợp lý: (a) $-62.87 + 35.14 + (-4.13) + 4\dfrac{43}{50}$. (b) $12.5\%\cdot0.5\cdot(-80)\cdot12\cdot0.25$. (c) $\dfrac{-2}{5}\cdot3.54 - 0.04\cdot137.5 + (-0.4)\cdot2.71$.
\end{baitoan}

\begin{baitoan}[\cite{Binh_boi_duong_Toan_6_tap_2}, 6.5., p. 36]
	Tìm $x\in\mathbb{Q}$ thỏa: (a) $(x + 3.6):0.3 = 9.6$. (b) $x:14.12 - 33.2 = 66.8$.
\end{baitoan}

\begin{baitoan}[\cite{Binh_boi_duong_Toan_6_tap_2}, 6.6., p. 36]
	Tính: (a) $A = a\cdot1.26 + 3.24a + 4.5\cdot(-0.73)$ với $a = -999.27$. (b) $B = \dfrac{-4}{25}b - b\cdot0.34 + 0.371:(-0.01)$ với $b = -37.1$.
\end{baitoan}

\begin{baitoan}[\cite{Binh_boi_duong_Toan_6_tap_2}, 6.7., p. 36]
	Nhiệt độ trung bình các tháng trong năm tại Bắc Cực được ghi lại: {\rm$-19.1^\circ$ C, $-14.4^\circ$ C, $-10.4^\circ$ C, $0.6^\circ$ C, $10.2^\circ$ C, $16.6^\circ$ C, $17.5^\circ$ C, $14.4^\circ$ C, $8^\circ$ C, $-1.3^\circ$ C, $-12.7^\circ$ C, $-17.2^\circ$ C}. Tính nhiệt độ trung bình năm tại Bắc Cực dựa vào các số liệu này.
\end{baitoan}

\begin{baitoan}[\cite{Binh_boi_duong_Toan_6_tap_2}, 6.8., p. 37]
	Tìm 1 số thập phân, biết nếu dịch dấu phẩy của số thập phân đó sang bên phải 2 chữ số thì ta được 1 số mới lớn hơn số phải tìm là $324.225$.
\end{baitoan}

\begin{baitoan}[\cite{Binh_boi_duong_Toan_6_tap_2}, 6.9., p. 37]
	Tổng của 1 số tự nhiên \& 1 số thập phân là $2077.15$. Nếu bỏ dấu phẩy của số thập phân trong phép tính đi thì được tổng mới bằng $8824$. Tìm 2 số đó.
\end{baitoan}

\begin{baitoan}[\cite{Binh_boi_duong_Toan_6_tap_2}, 6.10., p. 37]
	1 kho lương thực nhập vào kho 3 đợt gạo được $12.68$ tấn. Đợt thứ nhất nhập số gạo bằng $\dfrac{2}{5}$ đợt thứ 2, đợt thứ 3 nhập số gạo nhiều hơn tổng số gạo 2 đợt đầu là $2.67$ tấn. Tính số gạo mỗi đợt nhập.
\end{baitoan}

\begin{baitoan}[\cite{Binh_boi_duong_Toan_6_tap_2}, 6.11., p. 37]
	Thay 3 chữ số $a,b,c$ bởi các chữ số thích hợp để $\overline{ab.c} - \overline{a.bc} = 32.13$.
\end{baitoan}

\begin{baitoan}[\cite{Binh_boi_duong_Toan_6_tap_2}, 6.12., p. 37]
	Tính hợp lý: $A = \dfrac{72 + 36\cdot2 + 24\cdot3 + 18\cdot4 + 12\cdot6 + 144}{9.8 + 8.7 + \cdots + 3.2 + 2.1 - 1.2 - 2.3 - \cdots - 7.8 - 8.9}$.
\end{baitoan}

\begin{baitoan}[\cite{Tuyen_Toan_6}, VD65, p. 64]
	Tính $A = \dfrac{\left(6\dfrac{1}{2} - 8\right):0.05}{\left(7\dfrac{1}{20} - 5.65\right)\cdot6 + 1.6}$.
\end{baitoan}

\begin{baitoan}[\cite{Tuyen_Toan_6}, 326., p. 65]
	Viết phân số thập phân dưới dạng số thập phân: $\dfrac{69}{1000},8\dfrac{77}{100},\dfrac{34567}{10^4},\dfrac{abc}{10^n}$, $n\in\mathbb{N}$.
\end{baitoan}

\begin{baitoan}[\cite{Tuyen_Toan_6}, 327., p. 65]
	Đổi ra {\rm m} \& viết kết quả dưới dạng số thập phân: {\rm34 cm, 524 mm, 70 mm, 93 dm}.
\end{baitoan}

\begin{baitoan}[\cite{Tuyen_Toan_6}, 328., p. 65]
	Số thập phân $0.0295$ thay đổi như thế nào nếu: (a) Chuyển dấu phẩy sang trái 1 hàng. (b) Bỏ chữ số $0$ sau dấu phẩy. (c) Thêm $n$ chữ số $0$ vào bên phải chữ số $5$.
\end{baitoan}

\begin{baitoan}[\cite{Tuyen_Toan_6}, 329., p. 65]
	Viết phân số dưới dạng phân số thập phân rồi viết thành số thập phân: $\dfrac{19}{20},\dfrac{310}{125},\dfrac{102}{15},\dfrac{84}{105}$.
\end{baitoan}

\begin{baitoan}[\cite{Tuyen_Toan_6}, 330., p. 65]
	Viết số thập phân dưới dạng phân số tối giản: $0.675,-2.65,4.375$.
\end{baitoan}

\begin{baitoan}[\cite{Tuyen_Toan_6}, 331., p. 65]
	Dùng cả 3 chữ số $1,3,7$ \& dấu thập phân, có thể lập được bao nhiêu số thập phân có 3 chữ số? Sắp xếp chúng tăng dần.
\end{baitoan}

\begin{baitoan}[\cite{Tuyen_Toan_6}, 332., p. 65]
	Làm tròn số $8072.654$ đến: (a) Hàng trăm. (b) Hàng đơn vị. (c) Hàng phần trăm.
\end{baitoan}

\begin{baitoan}[\cite{Tuyen_Toan_6}, 333., p. 65]
	Đếm số số nguyên sau khi làm tròn trăm cho kết quả là $6700$.
\end{baitoan}

\begin{baitoan}[\cite{Tuyen_Toan_6}, 334., p. 65]
	Tính: (a) $\left(2\dfrac{5}{6} + 1\dfrac{4}{9}\right):\left(10\dfrac{1}{12} - 9.5\right)$. (b) $-\dfrac{1}{7}(9.5 - 8.75):\dfrac{2}{7} + 0.625:1\dfrac{2}{3}$.
\end{baitoan}

\begin{baitoan}[\cite{Tuyen_Toan_6}, 335., pp. 65--66]
	Hà dự trù $250000$ đồng để mua 1 số thực phẩm với khối lượng \& giá mỗi loại được kê:
	\begin{table}[H]
		\centering
		\begin{tabular}{|c|r|c|}
			\hline
			Loại hàng & Giá (đồng{\tt/}kg) & Dự định mua (kg) \\
			\hline
			Cải bắp & 17000 & 1.8 \\
			\hline
			Khoai tây & 19000 & 1.9 \\
			\hline
			Cà chua & 8000 & 0.8 \\
			\hline
			Thịt lợn & 128000 & 0.5 \\
			\hline
			Thịt bò & 146000 & 0.5 \\
			\hline
		\end{tabular}
	\end{table}
	Ước tính xem Hà có dự trù đủ tiền không?
\end{baitoan}

%------------------------------------------------------------------------------%

\section{Ratio. Percentage -- Tỷ Số. Tỷ Số Phần Trăm $\%$}
\fbox{1} \textit{Tỷ số} của 2 số $a,b\in\mathbb{R},b\ne0$ là $a:b$ hoặc $\dfrac{a}{b}$. \fbox{2} Tỷ số thường được dùng để nói về thương của 2 đại lượng cùng loại \& cùng đơn vị đo. \fbox{3} Tỷ số $\%$ của 2 số $a,b\in\mathbb{R},b\ne0$ là  $\dfrac{a\cdot100}{b}\%$.

\noindent SGK \cite[Chap. V, \S9, pp. 61--66]{SGK_Toan_6_Canh_Dieu_tap_2}: LT1. LT2. LT3. LT4. LT5. 1. 2. 3. 4. 5. SBT \cite[Chap. V, \S9, pp. 54--56]{SBT_Toan_6_Canh_Dieu_tap_2}: 93. 94. 95. 96. 97. 98. 99. 100. 101. 102. 103. 104.

\begin{baitoan}[{\sf Program}: Interchange between ratio \& percentage]
	Viết chương trình {\sf Pascal, Python, C{\tt/}C++} để chuyển đổi giữa tỷ số \& tỷ số $\%$.
\end{baitoan}

\begin{baitoan}[\cite{Binh_boi_duong_Toan_6_tap_2}, H1, p. 39]
	Tại 1 bệnh viện địa phương, cứ $5$ trẻ sinh ra thì có $2$ trẻ em trai. Tính tỷ số giữa số trẻ em trai, trẻ em gái \& tổng số trẻ được sinh ra.
\end{baitoan}

\begin{baitoan}[\cite{Binh_boi_duong_Toan_6_tap_2}, H2, p. 39]
	Mua 1 cái chảo giá $250000$ đồng \& được giảm giá $30000$ đồng. Tính tỷ số $\%$ số tiền được giảm.
\end{baitoan}

\begin{baitoan}[\cite{Binh_boi_duong_Toan_6_tap_2}, H3--H4, p. 39]
	Trung bình, nước chiếm khoảng $70\%$ trọng lượng cơ thể con người. (a) Tính khối lượng nước trong cơ thể của 1 người nặng $m = 50$ {\rm kg}. (b) Tính khối lượng cơ thể có khối lượng nước cơ thể khoảng $n = 49$ {\rm kg}.
\end{baitoan}

\begin{baitoan}[\cite{Binh_boi_duong_Toan_6_tap_2}, VD1, p. 39]
	Tìm 2 số biết: (a) Tổng của 2 số bằng $77$, tỷ số của 2 số là $\dfrac{3}{4}$. (b) Hiệu của chúng bằng $15.4$ \& tỷ số của chúng bằng $3.2$
\end{baitoan}

\begin{baitoan}
	Tìm 2 số khi biết 2 trong các yếu tố: tổng, hiệu, tích, thương (tỷ số) của chúng.
\end{baitoan}

\begin{baitoan}[\cite{Binh_boi_duong_Toan_6_tap_2}, VD2, p. 39]
	2 chuồng gà có tất cả $350$ gà con. Nếu chuyển $50$ gà con từ chuồng thứ nhất sang chuồng thứ 2 thì số gà trong chuồng thứ 2 gấp rưỡi số gà trong chuồng thứ nhất. Tính số gà có trong mỗi chuồng ban đầu.
\end{baitoan}

\begin{baitoan}[\cite{Binh_boi_duong_Toan_6_tap_2}, VD3, p. 40]
	Giá xem phim được niêm yết tại 1 rạp chiếu phim là $80000$ đồng{\tt/}vé. Sau khi giảm giá vé nhân ngày Quốc tế thiếu nhi 1.6, số khán giả đến rạp đã tăng thêm $25\%$ so với suất chiếu trước đó, làm cho doanh thu cũng tăng $12.5\%$. Tính giá vé sau khi giảm.
\end{baitoan}

\begin{baitoan}[\cite{Binh_boi_duong_Toan_6_tap_2}, VD4, p. 40]
	1 cửa hàng bán giảm giá $15\%$ để thanh lý 1 chiếc máy giặt nhưng vẫn không bán được nên tiếp tục giảm thêm $10\%$ so với giá đã giảm. Biết giá niêm yết ban đầu của chiếc máy giặt là $7$ triệu đồng, tính giá của máy giặt sau 2 lần giảm.
\end{baitoan}

\begin{baitoan}[\cite{Binh_boi_duong_Toan_6_tap_2}, 7.1., p. 40]
	Khoảng cách giữa Hà Nội \& Lạng Sơn (đường chim bay) trên bản đồ tỷ lệ $1:1000000$ là {\rm13 cm}. Tính khoảng cách thực tế giữa 2 thành phố đó.
\end{baitoan}

\begin{baitoan}[\cite{Binh_boi_duong_Toan_6_tap_2}, 7.2., p. 40]
	Trong 1 đội tình nguyện dọn rác tại bãi biễn, vào mùa hè, số thành viên là học sinh phổ thông chiếm $\dfrac{3}{7}$ tổng số thành viên còn lại. Vào năm học, số thành viên là học sinh phổ thông giảm đi $5$ người nên chỉ chiếm $\dfrac{1}{4}$ tổng số thành viên còn lại. Tính số thành viên của đội tình nguyện vào mùa hè.
\end{baitoan}

\begin{baitoan}[\cite{Binh_boi_duong_Toan_6_tap_2}, 7.3., p. 41]
	2 rổ cà chua có tất cả $76$ quả. Minh nhặt $5$ quả cà chua ở rổ thứ nhất chuyển sang rổ thứ 2 thì tỷ số của số cà chua ở rổ thứ nhất \& rổ thứ 2 là $\dfrac{8}{11}$. Tính số cà chua ở mỗi rổ lúc đầu.
\end{baitoan}

\begin{baitoan}[\cite{Binh_boi_duong_Toan_6_tap_2}, 7.4., p. 41]
	Hóa đơn tiền điện tháng này là $253174$ đồng (chưa thuế giá trị gia tăng VAT -- thuế mà người tiêu dùng phải trả khi mua 1 mặt hàng). Tính tổng số tiền điện phải trả, biết thuế VAT cho mặt hàng điện tiêu dùng là $10\%$, làm tròn kết quả đến hàng trăm.
\end{baitoan}

\begin{baitoan}[\cite{Binh_boi_duong_Toan_6_tap_2}, 7.5., p. 41]
	Trong 1 đợt khám sức khỏe, lớp 6A có $30\%$ các học sinh bị mắc các tật về mắt (viễn, cận, loạn thị,$\ldots$). Biết số học sinh không mắc các tật về mắt trong lớp 6A là $28$. Tính số học sinh lớp 6A.
\end{baitoan}

\begin{baitoan}[\cite{Binh_boi_duong_Toan_6_tap_2}, 7.6., p. 41]
	Cuối học kỳ I, số học sinh giỏi khối 6 của 1 trường bằng $\dfrac{2}{7}$ số học sinh còn lại. Cuối năm, số học sinh giỏi tăng thêm $5$ bạn nên số học sinh giỏi bằng $\dfrac{3}{8}$ số học sinh còn lại. Tính số học sinh khối 6.
\end{baitoan}

\begin{baitoan}[\cite{Binh_boi_duong_Toan_6_tap_2}, 7.7., p. 41]
	1 chiếc điện thoại có giá niêm yết tại cửa hàng là $4300000$ (chưa thuế). Hôm nay cửa hàng chạy chương trình khuyến mãi giảm $15\%$ tất cả sản phẩm thì khi mua chiếc điện thoại này người mua phải trả bao nhiêu tiền cả thuế VAT?
\end{baitoan}

\begin{baitoan}[\cite{Binh_boi_duong_Toan_6_tap_2}, 7.8., p. 41]
	Nếu giảm chiều dài của hình chữ nhật đi $20\%$ thì phải tăng hay giảm chiều rộng bao nhiêu $\%$ để diện tích hình chữ nhật giảm đi $10\%$?
\end{baitoan}

\begin{baitoan}[\cite{Binh_boi_duong_Toan_6_tap_2}, 7.9., p. 41]
	Nhờ ứng dụng công nghệ, thời gian để chế tạo 1 sản phẩm giảm $25\%$ so với trước đây. Năng suất lao động đã tăng bao nhiêu $\%$?
\end{baitoan}

\begin{baitoan}[\cite{Binh_boi_duong_Toan_6_tap_2}, 7.10., p. 41]
	1 người gửi tiết kiệm ngân hàng số tiền $10$ triệu đồng với lãi suất $7\%${\tt/}năm, sau mỗi năm tiền lãi được nhập vào gốc. Tính số tiền cả gốc lẫn lãi người đó nhận được sau 2 năm.
\end{baitoan}

\begin{baitoan}[\cite{Binh_boi_duong_Toan_6_tap_2}, 7.11., p. 41]
	1 cây gỗ tươi có khối lượng {\rm400 kg}, trong đó chứa $85\%$ nước. Cây gỗ này bốc hơi bao nhiêu {\rm kg} nước thì gỗ khô chứa $60\%$ nước?
\end{baitoan}

\begin{baitoan}[\cite{Binh_boi_duong_Toan_6_tap_2}, 7.12., p. 41]
	Số bò sữa của nông trường Ban Mai nhiều hơn $12.5\%$ so với số bò sửa của nông trường Thống Nhất. Tuy nhiên số lít sữa trung bình của mỗi con bò sữa ở nông trường Ban Mai lại ít hơn so với số lít sữa trung bình của mỗi con bò ở nông trường Thống Nhất là $8\%$. Tổng số lít sữa thu được của nông trường nào nhiều hơn \& nhiều hơn bao nhiêu $\%$?
\end{baitoan}

\begin{baitoan}[\cite{Tuyen_Toan_6}, VD68, p. 68]
	Tìm tỷ số \& tỷ số $\%$ của {\rm320 m, 0.8 km}.
\end{baitoan}

\begin{baitoan}[\cite{Tuyen_Toan_6}, VD69, p. 69]
	Ở 1 trại thí nghiệm lúa, giống A đạt năng suất $15$ tấn{\tt/}ha, giống B đạt $12$ tấn{\tt/}ha. (a) Năng suất giống A cao hơn năng suất giống B bao nhiêu tấn{\tt/}ha? Năng suất giống B thấp hơn năng suất giống A bao nhiêu tấn{\tt/}ha? (b) Năng suất giống A cao hơn năng suất giống B bao nhiêu $\%$? Năng suất giống B thấp hơn năng suất giống A bao nhiêu $\%$? 
\end{baitoan}

\begin{baitoan}[\cite{Tuyen_Toan_6}, 346., p. 69]
	Tìm tỷ số \& tỷ số $\%$ của: (a) {\rm2700 m, 6 km}. (b) {\rm $\dfrac{3}{10}$ h, 30 ph}. (c) $8.7,7\dfrac{1}{4}$.
\end{baitoan}

\begin{baitoan}[\cite{Tuyen_Toan_6}, 347., p. 69]
	Rút gọn tỷ số: (a) $\dfrac{2\dfrac{11}{12}}{6\dfrac{1}{8}}$. (b) $66\dfrac{2}{3}\%$. (c) $0.72:2.7$.
\end{baitoan}

\begin{baitoan}[\cite{Tuyen_Toan_6}, 348., p. 69]
	1 mảnh vải có diện tích $\rm\dfrac{4}{3}\ m^2$. Tìm cách để cắt ra đúng $\rm1\ m^2$ mà không dùng đến thước đo.
\end{baitoan}

\begin{baitoan}[\cite{Tuyen_Toan_6}, 349., p. 69]
	Giá cà phê giảm đi $20\%$. Tăng thêm bao nhiêu $\%$ để trở lại giá cũ?
\end{baitoan}

\begin{baitoan}[\cite{Tuyen_Toan_6}, 350., p. 70]
	1 cửa hàng thông báo: (a) Mặt hàng 1: Mua 4 tặng 1. (b) Mặt hàng 2: Mua 5 tặng 1. Vỡi mỗi mặt hàng, cửa hàng đã hạ giá bao nhiêu $\%$?
\end{baitoan}

\begin{baitoan}[\cite{Tuyen_Toan_6}, 351., p. 70]
	Vàng 4 số 9 là loại vàng có tỷ lệ nguyên chất là $\dfrac{9999}{10000}$ hay $99.99\%$. Càng nhiều số $9$ sau dấu phẩy thì tỷ lệ vàng nguyên chất càng cao, giá trị càng lớn. Tính xem trong {\rm10 kg} vàng này thì có bao nhiêu {\rm g} không phải là vàng lẫn trong đó.
\end{baitoan}

\begin{baitoan}[\cite{Tuyen_Toan_6}, 352., p. 70]
	Xếp loại văn hóa của lớp 6A chỉ có 2 loại giỏi \& khá. Cuối học kỳ I, tỷ số giữa học sinh giỏi \& khá là $\dfrac{3}{2}$. Cuối học kỳ II có thêm $1$ học sinh khá trở thành giỏi nên tỷ số giữa học sinh giỏi \& khá là $\dfrac{5}{3}$. Tính số học sinh của lớp.
\end{baitoan}

\begin{baitoan}[\cite{Tuyen_Toan_6}, 353., p. 70]
	1 lớp học có số nam chiếm $40\%$ số học sinh cả lớp. Sau khi có $4$ học sinh nam chuyển đi thì số nam bằng $\dfrac{1}{3}$ số học sinh cả lớp. Tính số học sinh nam lúc đầu.
\end{baitoan}

\begin{baitoan}[\cite{Tuyen_Toan_6}, 354., p. 70]
	Đầu năm học, số học sinh nữ của lớp 6A bằng $90\%$ số nam. Giữa năm học có thêm $4$ học sinh nam chuyển vào lớp nên số nữ bằng $75\%$ số nam. Tính xem đầu năm lớp 6A có bao nhiêu học sinh?
\end{baitoan}

\begin{baitoan}[\cite{Tuyen_Toan_6}, 355., p. 70]
	Lúc đầu, số trứng gà bằng số trứng vịt. Sau khi bán $80$ quả trứng gà \& $70$ quả trứng vịt thì số trứng gà còn lại bằng $48\%$ tổng số trứng còn lại. Mỗi loại còn bao nhiêu quả?
\end{baitoan}

\begin{baitoan}[\cite{Tuyen_Toan_6}, 356., p. 70]
	2 người đi mua gạo. Người thứ nhất mua gạo nếp, người thứ 2 mua gạo tẻ. Giá gạo tẻ rẻ hơn giá gạo nếp là $20\%$. Biết khối lượng gạo tẻ mà người thứ 2 mua nhiều hơn khối lượng gạo nếp mà người thứ nhất mua là $20\%$. Người nào trả tiền ít hơn? Ít hơn mấy $\%$ so với người kia?
\end{baitoan}

\begin{baitoan}[\cite{Tuyen_Toan_6}, 357., p. 70]
	Khoảng cách giữa Hà Nội \& Hải Phòng là {\rm102 km} nhưng trên 1 bản đồ, khoảng cách ấy chỉ là {\rm10.2 cm}. (a) Tính tỷ xích số của bản đồ. (b) Cũng trên bản đồ này, khoảng cách từ điểm cực Bắc nước ta ở Hà Giang đến điểm cực Nam ở mũi Cà Mau dài {\rm162 cm}. Tính khoảng cách đó trên thực tế.
\end{baitoan}

\begin{baitoan}[\cite{Binh_Toan_6_tap_2}, VD25, p. 28]
	1 tủ sách gồm 2 ngăn. Tỷ số giữa số sách của ngăn trên so với ngăn dưới là $4:3$. Sau khi thêm $30$ cuốn sách vào ngăn dưới thì tỷ số giữa số sách của ngăn trên so với ngăn dưới là $10:9$. Tính số sách ở mỗi ngăn lúc đầu.
\end{baitoan}

\begin{baitoan}[\cite{Binh_Toan_6_tap_2}, VD26, p. 28]
	Trong 1 buổi tham quan, số học sinh khối 6 tham gia bằng $\dfrac{5}{6}$ số học sinh khối 7. Tỷ số học sinh nam \& nữ trong khối 6 là $2:3$, trong khối 7 là $3:4$. Biết số nam khối 6 ít hơn số nam khối 7 là $8$ học sinh. Tính số học sinh mỗi khối 6, 7 đi tham quan.
\end{baitoan}

\begin{baitoan}[\cite{Binh_Toan_6_tap_2}, VD27, p. 29]
	Có 2 cố đựng 2 loại dầu ăn với khối lượng bằng nhau. Tỷ lệ khối lượng dầu \& nước trong cốc A là $2:1$, trong cốc B là $3:1$. Đổ 2 cốc A \& B vào 1 hộp C rỗng. Tính tỷ lệ khối lượng dầu \& nước trong hộp C.
\end{baitoan}

\begin{baitoan}[\cite{Binh_Toan_6_tap_2}, VD28, p. 30]
	Tìm số tự nhiên có 2 chữ số sao cho tỷ số của số đó \& tổng các chữ số của nó: (a) Lớn nhất. (b) Nhỏ nhất.
\end{baitoan}

\begin{baitoan}[\cite{Binh_Toan_6_tap_2}, 71., p. 30]
	1 sợi dây dài $1\dfrac{1}{3}$ {\rm m}. Tìm cách để cắt ra đoạn dây dài $\dfrac{1}{2}$ {\rm m} mà không có thước đo trong tay.
\end{baitoan}

\begin{baitoan}[\cite{Binh_Toan_6_tap_2}, 72., p. 31]
	Cho phân số $\dfrac{9}{43}$. Tìm $x\in\mathbb{N}$ để: (a) Cộng cả tử \& mẫu của phân số đã cho với $x$, được 1 phân số mới có giá trị bàng $\dfrac{1}{3}$. (b) Cộng tử của phân số đã cho với $x$, lấy mẫu trừ $x$, được 1 phân số mới có giá trị bằng $\dfrac{5}{8}$.
\end{baitoan}

\begin{baitoan}[\cite{Binh_Toan_6_tap_2}, 73., p. 31]
	Tìm 1 phân số có mẫu bằng $12$, biết nếu cộng tử với $15$ \& nhân mẫu với $4$ thì giá trị của phân số không đổi.
\end{baitoan}

\begin{baitoan}[\cite{Binh_Toan_6_tap_2}, 74., p. 31]
	Ở lớp 6A, số học sinh giỏi học kỳ 1 bằng $\dfrac{3}{7}$ số học sinh còn lại. Cuối năm có thêm $4$ học sinh đạt loại giỏi nên số học sinh giỏi bằng $\dfrac{2}{3}$ số học sinh còn lại. Tính số học sinh của lớp 6A.
\end{baitoan}

\begin{baitoan}[\cite{Binh_Toan_6_tap_2}, 75., p. 31]
	Tìm 3 số có tổng bằng $210$, biết $\dfrac{6}{7}$ số thứ nhất bằng $\dfrac{9}{11}$ số thứ 2 \& bằng $\dfrac{2}{3}$ số thứ 3.
\end{baitoan}

\begin{baitoan}[\cite{Binh_Toan_6_tap_2}, 76., p. 31]
	Tìm số tự nhiên có 2 chữ số, biết nếu chia số đó cho tích các chữ số của nó thì được $\dfrac{8}{3}$ \& hiệu giữa số phải tìm với số gồm các chữ số của số đó viết theo thứ tự ngược lại bằng $18$.
\end{baitoan}

\begin{baitoan}[\cite{Binh_Toan_6_tap_2}, 77., p. 31]
	Số dân của 2 phường $A,B$ trước kia tỷ lệ với $2,3$. Hiện nay số dân phường A tăng thêm $8000$ người, phường B tăng thêm $4000$ người, do đó số dân phường A bằng $\dfrac{3}{4}$ số dân phường B. Tính số dân của mỗi phường hiện nay.
\end{baitoan}

\begin{baitoan}[\cite{Binh_Toan_6_tap_2}, 78., p. 31]
	Tuổi mẹ hiện nay gấp $2.3$ lần tuổi con. $16$ năm trước, tuổi mẹ gấp $7.5$ lần tuổi con. Mấy năm sau thì tuổi mẹ gấp đôi tuổi con?
\end{baitoan}

\begin{baitoan}[\cite{Binh_Toan_6_tap_2}, 79., p. 31]
	Tuấn đến nhà Tú chơi, thấy 1 chú chó \& 1 chú mèo, cả 2 đều rất xinh xắn. Tuấn ôm chó \& vuốt đuôi nó. Em của Tú láu lỉnh bảo Tuấn: Đừng đoán tuổi chó dựa vào số khoanh đuôi. 2 năm trước tuổi chó gấp 3 tuổi mèo, còn bây giờ chỉ gấp đôi. Tính tuổi chó, tuổi mèo.
\end{baitoan}

\begin{baitoan}[\cite{Binh_Toan_6_tap_2}, 80., p. 31]
	An, Bảo là 2 người bạn. An nói với Bảo: Khi tôi bằng tuổi anh hiện nay thì tuổi tôi bằng $\dfrac{4}{3}$ tuổi anh. Bảo nói: Khi tôi gấp đôi tuổi anh hiện nay thì tuổi 2 chúng ta cộng lại bằng $75$. Tính tuổi mỗi người hiện nay.
\end{baitoan}

\begin{baitoan}[\cite{Binh_Toan_6_tap_2}, 81., p. 32]
	2 ôtô đều đã đi được quãng đường dài {\rm3500 km}. Ngoài các lốp đang dùng, mỗi ôtô còn có thêm 1 chiếc lốp dự trữ. Tất cả các lốp đang dùng \& dự trữ được sử dụng tương đương như nhau. Tính số {\rm km} đã chạy của mỗi lốp xe ở mỗi ôtô, biết ôtô thứ nhất là xe 4 bánh, ôtô thứ 2 là xe 6 bánh.
\end{baitoan}

\begin{baitoan}[\cite{Binh_Toan_6_tap_2}, 82., p. 32]
	An có 1 cốc kem. Lần 1, An ăn $\dfrac{1}{2}$ cốc kem. Lần 2, An ăn $\dfrac{1}{3}$ cốc kem. Lần 3, An ăn $\dfrac{1}{4}$ cốc kem. Cứ như vậy cho đến khi số kem còn lại nhỏ hơn $\dfrac{1}{8}$  số kem ban đầu thi An ăn nốt. An ăn hết cốc kem sau mấy lần?
\end{baitoan}

\begin{baitoan}[\cite{Binh_Toan_6_tap_2}, 83., p. 32]
	Có 1 chai chứa {\rm1 l} rượu \& 1 bình chứa {\rm1 l} nước lọc. Rót từ chai ra 1 cốc rượu đem đổ vào bình. Sau khi rượu tan đều trong bình, rót từ bình ra 1 cốc nước có lẫn rượu, đem đổ vào chai. Sau 2 lần rót này, so sánh lượng rượu lấy ra từ chai với lượng nước lọc lấy ra từ bình.
\end{baitoan}

\begin{baitoan}[\cite{Binh_Toan_6_tap_2}, 84., p. 32]
	Có 2 chai, chai I chứa {\rm1 l} nước, chai II là chai không. Lần 1, rót tất cả nước ở chai I sang chai II. Lần 2, rót $\dfrac{1}{2}$ số nước ở chai II sang chai I. Lần 3, rót $\dfrac{1}{3}$ số nước ở chai I sang chai II. Lần 4, rót $\dfrac{1}{4}$ số nước ở chai II sang chai I. Cứ tiếp tục như vậy. (a) Tính số nước ở chai I sau lần rót thứ 2, thứ 4. (b) Chứng tỏ cứ sau 1 số chẵn lần rót thì số nước ở 2 chai lại bằng nhau.
\end{baitoan}

\begin{baitoan}[\cite{Binh_Toan_6_tap_2}, VD29, p. 33]
	Tính diện tích 1 hình chữ nhật biết nếu chiều dài tăng $20\%$, chiều rộng giảm $20\%$ thì diện tích giảm $\rm30\ m^2$.
\end{baitoan}

\begin{baitoan}[\cite{Binh_Toan_6_tap_2}, VD30, p. 33]
	Bác Tuân có {\rm475 kg} hạt cà phê tươi với tỷ lệ nước trong hạt là $20\%$. Sau khi phơi khô thì tỷ lệ nước trong hạt cà phê khô là $5\%$. Tính lượng cà phê sau khi phơi khô.
\end{baitoan}

\begin{baitoan}[\cite{Binh_Toan_6_tap_2}, VD31, p. 33]
	Có 1 số con bò. Khi cân, 2 con nhẹ nhất có cân nặng bằng $25\%$ tổng cân nặng của cả nhóm, 3 con nặng nhất có cân nặng bằng $60\%$ tổng cân nặng của cả nhóm. Tính số con bò của nhóm đó.
\end{baitoan}

\begin{baitoan}[\cite{Binh_Toan_6_tap_2}, 85., p. 34]
	Khối 6 của 1 trường đầu năm học có số học sinh nữ bằng số học sinh nam. Đầu học kỳ 2, khối 6 nhận thêm 8 học sinh nữ \& 2 học sinh nam nên số học sinh nữ bằng $51\%$ tổng số học sinh. Tính tổng số học sinh đầu năm học.
\end{baitoan}

\begin{baitoan}[\cite{Binh_Toan_6_tap_2}, 86., p. 34]
	Khi đi cùng 1 quãng đường, nếu vận tốc giảm $20\%$ thì thời gian đi quãng đường ấy tăng bao nhiêu $\%$?
\end{baitoan}

\begin{baitoan}[\cite{Binh_Toan_6_tap_2}, 87., p. 34]
	(a) Chiều dài hình chữ nhật tăng $10\%$, chiều rộng giảm $10\%$. Diện tích hình chữ nhật đó thay đổi như thế nào? (b) Giá 1 sản phẩm tăng $30\%$, sau 1 thời gian lại giảm $30\%$. Giá sản phẩm đó tăng hay giảm so với lúc ban đầu? (c) Khi làm 1 công việc, nếu năng suất lao động tăng $25\%$ thì thời gian hoàn thành công việc ấy giảm bao nhiêu $\%$?
\end{baitoan}

\begin{baitoan}[\cite{Binh_Toan_6_tap_2}, 88., p. 34]
	(a) Cạnh của 1 hình vuông tăng $20\%$. Diện tích của nó tăng bao nhiêu $\%$? (b) Cạnh của 1 hình lập phương tăng $50\%$. Thể tích của nó tăng bao nhiêu $\%$? (c) Đáy của 1 tam giác tăng $20\%$, chiều cao tương ứng giảm $20\%$. Diện tích của tam giác đó tăng hay giảm bao nhiêu $\%$? (d) 1 chiếc tủ nếu bán với giá $x$ đồng thì lỗ $10\%$ so với giá vốn, nếu bán với giá $y$ đồng thì lãi $10\%$ so với giá vốn. Tính tỷ số của $y$ \& $x$.
\end{baitoan}

\begin{baitoan}[\cite{Binh_Toan_6_tap_2}, 89., p. 34]
	Giá rau tháng 7 thấp hơn giá rau tháng 6 là $10\%$, giá rau tháng 8 cao hơn giá rau tháng 7 là $10\%$. Giá rau tháng 8 so với tháng 6 cao hơn hay thấp hơn bao nhiêu $\%$?
\end{baitoan}

\begin{baitoan}[\cite{Binh_Toan_6_tap_2}, 90., p. 34]
	Chiều dài 1 hình chữ nhật tăng {\rm36 m}, chiều rộng giảm $16\%$. Tìm chiều dài mới biết diện tích mới lớn hơn diện tích cũ là $5\%$.
\end{baitoan}

\begin{baitoan}[\cite{Binh_Toan_6_tap_2}, 91., p. 34]
	(a) Thu hoạch lúa của nông trường A nhiều hơn so với nông trường B là $26\%$, diện tích trồng lúa của nông trường A nhiều hơn so với nông trường B là $5\%$. Năng suất của nông trường A nhiều hơn nông trường B là bao nhiêu $\%$? (b) Số hộp sữa loại I ít hơn so với loại II là $12.5\%$ nhưng lượng sữa trong mỗi hộp lại nhiều hơn $8\%$. Lượng sữa tổng cộng của loại nào ít hơn?
\end{baitoan}

\begin{baitoan}[\cite{Binh_Toan_6_tap_2}, 92., p. 35]
	Năm 2002, 1 xí nghiệp làm được $159135$ sản phẩm, tính ra so với 2 năm trước số sản phẩm tăng hàng năm là $3\%$. Tính số sản phẩm của xí nghiệp đó năm 2000.
\end{baitoan}

\begin{baitoan}[\cite{Binh_Toan_6_tap_2}, 93., p. 35]
	Phải thêm bao nhiêu {\rm kg} dung dịch acid có nồng độ $5\%$ để được dung dịch acid có nồng độ $3\%$? (Dung dịch acid có nồng độ $5\%$ tức là trong {\rm100 kg} dung dịch có {\rm5 kg} acid nguyên chất \& {\rm95 kg} nước).
\end{baitoan}

\begin{baitoan}[\cite{Binh_Toan_6_tap_2}, 94., p. 35]
	1 cây gỗ tươi có khối lượng {\rm400 kg}, trong đó chứa $85\%$ nước. Cây gỗ này bốc hơi bao nhiêu {\rm kg} nước thì được gỗ khô chứa $60\%$ nước?
\end{baitoan}

\begin{baitoan}[\cite{Binh_Toan_6_tap_2}, 95., p. 35]
	Năm trước 2 công trường có $500$ con bò. Năm sau, số bò của nông trường I tăng $25\%$, số bò của nông trường II tăng $12.5\%$, nên số bò của cả 2 nông trường tăng $20\%$. Tính số bò năm trước của mỗi nông trường.
\end{baitoan}

\begin{baitoan}[\cite{Binh_Toan_6_tap_2}, 96., p. 35]
	Hoa được mẹ cho tiền mua $30$ quyển vở. Cửa hàng đang có chương trình khuyến mãi: Cứ mua $20$ quyển vở thì được giảm giá $25\%$, cứ mua $5$ quyển vở thì được giảm giá $10\%$. Với số tiền mang đi để mua $30$ quyển vở thì Hoa mua được nhiều nhất mấy quyển vở?
\end{baitoan}

\begin{baitoan}[\cite{Binh_Toan_6_tap_2}, 97., p. 35]
	1 công ty có 1 số quần áo đưa ra bán ở hội chợ. Công ty đã bán theo từng bộ (1 áo \& 1 quần) với giá áo \& giá quần bằng nhau. Tính ra so với bán riêng áo \& bán riêng quần thì giá áo bán theo cả bộ lãi $30\%$, còn giá quần bán theo cả bộ lỗ $30\%$. Giá bán theo cả bộ (1 áo \& 1 quần) bằng bao nhiêu $\%$ so với bán riêng 1 áo \& 1 quần?
\end{baitoan}

\begin{baitoan}[\cite{Binh_Toan_6_tap_2}, 98., p. 35]
	Trong Hội chợ Xuân, 1 công ty có $\dfrac{1}{3}$ tổng số sản phẩm đạt huy chương vàng. Biết trong tổng số sản phẩm của công ty thì số sản phẩm may chiếm $\dfrac{3}{4}$. Trong số sản phẩm may, số sản phẩm đạt huy chương vàng chiếm ít nhất, nhiều nhất là bao nhiêu $\%$?
\end{baitoan}

\begin{baitoan}[\cite{TLCT_THCS_Toan_6_so_hoc}, VD12.1, p. 71]
	1 cửa hàng có 2 loại quạt, giá tiền như nhau. Quạt màu vàng được giảm giá 2 lần, mỗi lần giảm giá $10\%$. Quạt màu xanh được giảm giá 1 lần $20\%$. Sau khi giảm giá như trên thì loại quạt nào rẻ hơn?
\end{baitoan}

\begin{baitoan}[\cite{TLCT_THCS_Toan_6_so_hoc}, VD12.2, p. 71]
	(a) Chiều dài 1 hình chữ nhật tăng $25\%$. Chiều rộng hình chữ nhật phải giảm bao nhiêu $\%$ để chu vi hình chữ nhật không đổi, biết chiều dài gấp đôi chiều rộng? (b) Chiều dài 1 hình chữ nhật tăng $25\%$. Chiều rộng hình chữ nhật phải giảm bao nhiêu $\%$ để diện tích hình chữ nhật không đổi?
\end{baitoan}

\begin{baitoan}[\cite{TLCT_THCS_Toan_6_so_hoc}, VD12.3, p. 72]
	1 quả dưa hấu có khối lượng {\rm1000 g} chứa $93\%$ nước. 1 tuần sau, lượng nước chỉ còn $90\%$. Khi đó, khối lượng quả dưa hấu còn bao nhiêu {\rm g}?
\end{baitoan}

\begin{baitoan}[\cite{TLCT_THCS_Toan_6_so_hoc}, VD12.4, p. 73]
	1 cửa hàng trong ngày khai trương hạ giá hàng $12\%$ so với giá bán trong ngày thường. Tuy vậy, cửa hàng vẫn lãi $10\%$ so với giá gốc. Nếu không hạ giá thì cửa hàng lãi bao nhiêu $\%$ so với giá gốc?
\end{baitoan}

\begin{baitoan}[\cite{TLCT_THCS_Toan_6_so_hoc}, 12.1., p. 73]
	Phân số $\dfrac{1}{2}$ tăng thành $\dfrac{7}{8}$ thì giá trị của phân số đó tăng thêm bao nhiêu $\%$?
\end{baitoan}

\begin{baitoan}[\cite{TLCT_THCS_Toan_6_so_hoc}, 12.2., p. 73]
	1 xí nghiệp có khối lượng công việc tăng thêm $40\%$, còn năng suất lao động của công nhân tăng thêm $25\%$. Số công nhân cần tăng thêm bao nhiêu $\%$?
\end{baitoan}

\begin{baitoan}[\cite{TLCT_THCS_Toan_6_so_hoc}, 12.3., p. 73]
	Giá lúa tăng $25\%$. Giá lúa phải giảm bao nhiêu $\%$ để trở lại giá cũ?
\end{baitoan}

\begin{baitoan}[\cite{TLCT_THCS_Toan_6_so_hoc}, 12.4., p. 73]
	Giá rau tháng 4 cao hơn so với tháng 3 là $10\%$. Giá rau tháng 5 thấp hơn so với tháng 4 là $10\%$. Giá rau tháng 5 so với tháng 3 bằng bao nhiêu $\%$?
\end{baitoan}

\begin{baitoan}[\cite{TLCT_THCS_Toan_6_so_hoc}, 12.5., p. 73]
	1 cửa hàng nhập 1 loại đồ chơi, rồi định giá bán là {\rm50000 đ{\tt/}chiếc}. Trong ngày Tết thiếu nhi 1.6, cửa hàng hạ giá $12\%$, tính ra so với giá nhập vào vẫn lãi $10\%$. (a) Tính giá nhập của đồ chơi ấy. (b) So với giá nhập, thì giá bán trong ngày thường lãi bao nhiêu $\%$?
\end{baitoan}

\begin{baitoan}[\cite{TLCT_THCS_Toan_6_so_hoc}, 12.6., p. 74]
	1 cửa hàng bán quần áo thanh lý hàng nên đã giảm $10\%$ so với giá bình thường, nhưng không bán được nên giảm tiếp $10\%$ nữa (so với giá đã giảm) \& đã bán hết hàng. Tính ra cửa hàng vẫn lãi $5.4\%$ so với giá gốc. Giá bình thường bằng bao nhiêu $\%$ giá gốc?
\end{baitoan}

\begin{baitoan}[\cite{TLCT_THCS_Toan_6_so_hoc}, 12.7., p. 74]
	1 hãng điện thoại có 3 phương án trả tiền cước điện thoại: Phương án I: Trả $99$ xu cho $20$ phút đầu, sau đó từ phút thứ $21$ thì mỗi phút trả thêm $5$ xu. Phương án II: Kể từ lúc đầu tiên, mỗi phút trả $10$ xu. Phương án III: Trả $25$ xu, sau đó kể từ phút đầu tiên mỗi phút trả $8$ xu. 1 khách hàng trong tháng có $10\%$ cuộc gọi $1$ phút, $10\%$ cuộc gọi $5$ phút, $30\%$ cuộc gọi $10$ phút, $30\%$ cuộc gọi $20$ phút, $20\%$ cuộc gọi $30$ phút. Người đó nên chọn phương án nào để tiền cước ít nhất?
\end{baitoan}

\begin{baitoan}[\cite{TLCT_THCS_Toan_6_so_hoc}, 12.8., p. 74]
	1 cửa hàng sách hạ giá $10\%$ trong ngày lễ, tuy vậy cửa hàng vẫn còn lãi $8\%$. Trong ngày thường cửa hàng lãi bao nhiêu $\%$?
\end{baitoan}

\begin{baitoan}[\cite{TLCT_THCS_Toan_6_so_hoc}, 12.9., p. 74]
	Nước biển chứa $5\%$ muối. Cần thêm bao nhiêu {\rm kg} nước lã vào {\rm20 kg} nước biển để tỷ lệ muối trong dung dịch là $2\%$?
\end{baitoan}

\begin{baitoan}[\cite{TLCT_THCS_Toan_6_so_hoc}, 12.10., p. 74]
	Ông Ngọc có {\rm500 kg} hạt cà phê tươi, đem phơi khô để tỷ lệ nước trong hạt cà phê còn $5\%$. Biết tỷ lệ nước trong hạt cà phê tươi là $24\%$. Tính khối lượng nước cần bay hơi.
\end{baitoan}

\begin{baitoan}[\cite{TLCT_THCS_Toan_6_so_hoc}, 12.11., p. 74]
	Phơi {\rm450 kg} hạt tươi thì được hạt khô. Biết tỷ lệ nước trong hạt tươi là $20\%$, tỷ lệ nước trong hạt khô là $10\%$. Tính khối lượng hạt khô.
\end{baitoan}

\begin{baitoan}[\cite{TLCT_THCS_Toan_6_so_hoc}, 12.12., p. 74]
	Chị Mai ngâm {\rm15 kg} hạt giống có tỷ lệ nước là $4\%$ vào 1 thùng nước. Chị muốn tỷ lệ nước trong hạt giống sau khi ngâm là $10\%$ để cho khả năng nảy mầm cao hơn. Tính khối lượng hạt giống sau khi ngâm.
\end{baitoan}

\begin{baitoan}[\cite{TLCT_THCS_Toan_6_so_hoc}, 12.13., p. 74]
	Phơi {\rm60 kg} cỏ tươi, sau 1 tuần thì còn {\rm30 kg} cỏ khô. Biết tỷ lệ nước trong cỏ tươi là $70\%$. Tỷ lệ nước trong cỏ khô là bao nhiêu $\%$?
\end{baitoan}

%------------------------------------------------------------------------------%

\section{Các Bài Toán Về Phân Số \& Tỷ Số}
2 bài toán cơ bản về phân số: \fbox{1} \textit{Tìm giá trị phân số của 1 số}: Giá trị $\dfrac{m}{n}$ của 1 số $a\in\mathbb{R}$ là $b = a\cdot\dfrac{m}{n} = \dfrac{am}{n}$, $\forall m,n\in\mathbb{N},n\ne0$. \fbox{2} \textit{Tìm 1 số biết giá trị 1 phân số của nó}: Biết $\dfrac{m}{n}$ của $a\in\mathbb{R}$ bằng $b\in\mathbb{R}$ thì $a = b:\dfrac{m}{n}$, $\forall m,n\in\mathbb{N},n\ne0$. \fbox{3} 2 bài toán cơ bản về phân số cũng áp dụng được đối với số thập phân (i.e., số thực).\\

\noindent SGK \cite[Chap. V, \S10, pp. 67--70]{SGK_Toan_6_Canh_Dieu_tap_2}: LT1. LT2. 1. 2. 3. 4. 5. 6. 7. 8. SBT \cite[Chap. V, \S10, pp. 57--59]{SBT_Toan_6_Canh_Dieu_tap_2}: 105. 106. 107. 108. 109. 110. 111. 112. 113. 114. 115. 116. 117. 118. 119. 120.

\begin{baitoan}[\cite{Binh_boi_duong_Toan_6_tap_2}, H1--H4, p. 21]
	Tính: (a) $\dfrac{5}{2}$ của $50$. (b) $\dfrac{2}{3}$ {\rm h}. (c) 1 số biết $\dfrac{3}{4}$ của số đó bằng $15000$. (d) $\dfrac{1}{4}$ số quả táo nặng {\rm3 kg}. Tính cân nặng tất cả số quả táo đó.
\end{baitoan}

\begin{baitoan}[\cite{Binh_boi_duong_Toan_6_tap_2}, VD1, p. 26]
	$\dfrac{3}{4}$ trong số $240$ vật nuôi ở 1 trang trại là gà, số còn lại là lợn. Tính số gà, số lợn của trang trại.
\end{baitoan}

\begin{baitoan}[\cite{Binh_boi_duong_Toan_6_tap_2}, VD2, p. 26]
	Tại 1 điểm trong giữ xe, người trông xe đã phát ra $120$ vé gửi xe các loại. Có $\dfrac{3}{4}$ số xe gửi là xe máy \& $\dfrac{2}{3}$ số xe máy là xe tay ga. Tính số xe máy tay ga ở điểm trong giữ xe lúc đó.
\end{baitoan}

\begin{baitoan}[\cite{Binh_boi_duong_Toan_6_tap_2}, VD3, p. 26]
	An thu hoạch $\dfrac{2}{5}$ diện tích trồng lúa thì được khoảng {\rm2000 kg} thóc. Tính số tấn thóc nếu An thu hoạch hết diện tích trồng lúa của mình.
\end{baitoan}

\begin{baitoan}[\cite{Binh_boi_duong_Toan_6_tap_2}, VD4, p. 26]
	Kết thúc học kỳ I, lớp 6A có $27$ học sinh có điểm trung bình môn Toán đạt từ $8.0$ trở lên \& chiếm $\dfrac{3}{4}$ tổng số học sinh của lớp. Tính số học sinh lớp 6A.
\end{baitoan}

\begin{baitoan}[\cite{Binh_boi_duong_Toan_6_tap_2}, 4.1., p. 26]
	Bác Minh gửi tiết kiệm $20$ triệu đồng vào 1 ngân hàng với lãi suất $7\%$ 1 năm. Tính số tiền lãi bác Minh nhận được sau 1 năm.
\end{baitoan}

\begin{baitoan}[\cite{Binh_boi_duong_Toan_6_tap_2}, 4.2., p. 26]
	Có $84$ quả, $\dfrac{4}{7}$ số quả là táo, còn lại là lê. Tính số táo, số lê.
\end{baitoan}

\begin{baitoan}[\cite{Binh_boi_duong_Toan_6_tap_2}, 4.3., p. 26]
	Hoa đã đọc được $\dfrac{3}{5}$ số trang của 1 cuốn truyện \& còn $64$ trang nữa thì hết cuốn truyện đó. Tính số trang cuốn truyện của Hoa.
\end{baitoan}

\begin{baitoan}[\cite{Binh_boi_duong_Toan_6_tap_2}, 4.4., p. 27]
	1 mảnh vườn hình chữ nhật có chiều dai {\rm20 m}, chiều dài bằng $\dfrac{5}{3}$ chiều rộng. Tính diện tích mảnh vườn đó.
\end{baitoan}

\begin{baitoan}[\cite{Binh_boi_duong_Toan_6_tap_2}, 4.5., p. 27]
	An đếm $\dfrac{2}{3}$ kho hàng thì thấy có $1600$ sản phẩm. Hôm nay, kho hàng xuất bán $\dfrac{1}{4}$ số hàng trong kho, tức là bao nhiêu sản phẩm?
\end{baitoan}

\begin{baitoan}[\cite{Binh_boi_duong_Toan_6_tap_2}, 4.6., p. 27]
	Bộ sưu tập tem của Tuấn có $\dfrac{3}{4}$ số tem về chủ đề động vật, $\dfrac{1}{12}$ số tem về chủ đề các loài hoa, số còn lại là tem về các sự kiện lịch sử. Biết Tuấn có tất cả $16$ con tem thuộc chủ đề các sự kiện lịch sử, tính tổng số tem \& số tem thuộc các chủ đề còn lại trong bộ sưu tập của Tuấn.
\end{baitoan}

\begin{baitoan}[\cite{Binh_boi_duong_Toan_6_tap_2}, 4.7., p. 27]
	1 bác nông dân mang trứng gà ra chợ bán. Bác bán cho người thứ nhất 1 nửa số trứng bớt lại $6$ quả, bán cho người thứ 2 $\dfrac{1}{3}$ số trứng còn lại bớt đi $6$ quả, bán cho người thứ 3 $\dfrac{1}{4}$ số trứng còn lại bớt đi $6$ quả. Kiểm tra lại, bác nông dân thấy mình vẫn còn đúng 1 nửa số trứng lúc đem ra chợ. Tính số trứng bác nông dân đã mang ra chợ.
\end{baitoan}

\begin{baitoan}[\cite{Tuyen_Toan_6}, VD66, p. 66]
	$\dfrac{2}{5}$ của 1 số a là $120$. Tính $\dfrac{7}{10}$ của a.
\end{baitoan}

\begin{baitoan}[\cite{Tuyen_Toan_6}, VD67., p. 67]
	Quốc kỳ Việt Nam là hình chữ nhật, chiều rộng bằng $\dfrac{2}{3}$ chiều dài. Biết 1 lá cờ đỏ sao vàng có chiều rộng {\rm80 cm}, tính diện tích lá cờ.
\end{baitoan}

\begin{baitoan}[\cite{Tuyen_Toan_6}, 336., p. 67]
	Viết tập hợp A các số nguyên x lớn hơn $\dfrac{9}{10}$ của $-1\dfrac{7}{18}$ nhưng nhỏ hơn $\dfrac{8}{35}$ của $31.5$.
\end{baitoan}

\begin{baitoan}[\cite{Tuyen_Toan_6}, 337., p. 67]
	1 khu vườn hình chữ nhật có diện tích $\rm500\ m^2$. Nếu giảm chiều dài đi $\dfrac{1}{5}$ của nó \& tăng chiều rộng thêm $\dfrac{1}{5}$ của nó thì diện tích khu vườn tăng thêm hay giảm đi bao nhiêu $\rm m^2$?
\end{baitoan}

\begin{baitoan}[\cite{Tuyen_Toan_6}, 338., p. 67]
	1 lớp học có chưa đến $50$ học sinh. Cuối năm có $\dfrac{3}{10}$ số học sinh xếp loại giỏi, $\dfrac{3}{8}$ số học sinh xếp loại khá, còn lại là học sinh trung bình. Tính số học sinh mỗi loại.
\end{baitoan}

\begin{baitoan}[\cite{Tuyen_Toan_6}, 339., p. 67]
	1 quầy hàng trong {\rm3 h} bán được $44$ quả dưa hấu. Giờ đầu bán $\dfrac{1}{3}$ số dưa đó \& $\dfrac{1}{3}$ quả. Giờ thứ 2 bán $\dfrac{1}{3}$ số dưa đó \& $\dfrac{1}{3}$ quả. Tính số quả giờ thứ 3 bán.
\end{baitoan}

\begin{baitoan}[\cite{Tuyen_Toan_6}, 340., p. 67]
	3 người chung nhau mua hết 1 rổ trứng. Người thứ nhất mua $\dfrac{1}{2}$ số trứng mà 2 người kia mua. Số trứng người thứ 2 mau bằng $\dfrac{3}{5}$ số trứng người thứ nhất mua. Người thứ 3 mua $14$ quả. Tính số trứng lúc đầu trong rổ.
\end{baitoan}

\begin{baitoan}[\cite{Tuyen_Toan_6}, 341., p. 67]
	1 khu vườn trồng hoa hồng, hoa cúc, \& hoa huệ. Phần trồng hoa hồng chiếm $\dfrac{3}{7}$ diện tích vườn \& bằng $\dfrac{6}{5}$ diện tích trồng hoa cúc. Còn lại $\rm90\ m^2$ trồng hoa huệ. Tính diện tích khu vườn.
\end{baitoan}

\begin{baitoan}[\cite{Tuyen_Toan_6}, 342., p. 67]
	1 ôtô chạy hết quãng đường AB trong {\rm3 h}. Giờ đầu chạy được $\dfrac{2}{5}$ quãng đường AB. Giờ thứ 2 chạy được $\dfrac{2}{5}$ quãng đường còn lại \& thêm {\rm4 km}. Giờ thứ 3 chạy nốt {\rm50 km} cuối. Tính vận tốc trung bình của ôtô trên quãng đường AB.
\end{baitoan}

\begin{baitoan}[\cite{Tuyen_Toan_6}, 343., p. 67]
	1 bà bán trứng cho 3 người: bán cho người thứ nhất $\dfrac{1}{4}$ số trứng \& $3$ quả, bán cho người thứ 2 $\dfrac{1}{3}$ số trứng còn lại \& $4$ quả, bán cho người thứ 3 $\dfrac{1}{2}$ số trứng còn lại \& $5$ quả. Cuối cùng còn lại $6$ quả. Tính số trứng bà đã bán cho 3 người.
\end{baitoan}

\begin{baitoan}[\cite{Tuyen_Toan_6}, 344., p. 68]
	4 người ngồi chung nhau 1 giỏ xoài. Người thứ nhất mua $\dfrac{1}{5}$ số xoài \& $1$ quả, người thứ 2 mua $\dfrac{2}{5}$ số xoài còn lại \& bớt $1$ quả, người thứ 3 mua $\dfrac{3}{5}$ số xoài còn lại \& cũng bớt $1$ quả. Người thứ 4 mua nốt $5$ quả cuối cùng. Tính số xoài trong giỏ.
\end{baitoan}

\begin{baitoan}[\cite{Tuyen_Toan_6}, 345., p. 68]
	1 người đăng ký mua 1 căn hộ chung cư, trả tiền làm 4 đợt. Đợt đầu ngay khi ký hợp đồng mua bán nhà, phải trả $\dfrac{1}{3}$ số tiền mua căn hộ. Đợt 2, 6 tháng sau trả $\dfrac{1}{4}$ số tiền mua căn hộ. Đợt 3, 6 tháng sau đợt 2, trả $\dfrac{1}{5}$ số tiền mua căn hộ. Đợt cuối, 6 tháng sau đợt 3, trả nốt số tiền $390$ triệu đồng \& nhận chìa khóa căn hộ. Tính giá tiền mua căn hộ đó.
\end{baitoan}

\begin{baitoan}[\cite{Binh_Toan_6_tap_2}, VD19, p. 21]
	Trong 1 hội nghị học sinh giỏi, số học sinh nữ chiếm $\dfrac{2}{5}$, trong đó $\dfrac{3}{8}$ số nữ là học sinh lớp 6. Trong số học sinh nam dự hội nghị, $\dfrac{2}{9}$ là học sinh lớp 6. Biết số học sinh dự hội nghị trong khoảng từ $100$ đến $170$ bạn. Tính số học sinh giỏi nam \& nữ lớp 6.
\end{baitoan}

\begin{baitoan}[\cite{Binh_Toan_6_tap_2}, VD20, p. 21]
	Trên quãng đường AC dài {\rm200 km} có 1 địa điểm B cách A là {\rm10 km}. Lúc {\rm7:00}, 1 ôtô đi từ A, 1 ôtô khác đi từ B, cả 2 cùng đi tới C với vận tốc thứ tự bằng {\rm50 km{\tt/}h, 40 km{\tt/}h}. Lúc mấy giờ thì khoảng cách đến C của xe thứ 2 gấp đôi khoảng cách đến C của xe thứ nhất?
\end{baitoan}

\begin{baitoan}[\cite{Binh_Toan_6_tap_2}, VD21, p. 23]
	2 xe khởi hành cùng 1 lúc, xe thứ nhất đi từ A đến B, xe thứ 2 đi từ B đến A. Sau {\rm1 h 30 ph}, 2 xe còn cách nhau {\rm108 km}. Tính quãng đường AB, biết xe thứ nhất đi cả quãng đường AB hết {\rm6 h}, xe thứ 2 đi cả quãng đường BA hết {\rm5 h}.
\end{baitoan}

\begin{baitoan}[\cite{Binh_Toan_6_tap_2}, VD22, p. 24, bài toán của Newton]
	3 công nhân cùng làm 1 công việc. Người thứ nhất có thể hoàn thành công việc đó trong $3$ tuần, người thứ 2 có thể hoàn thành 1 công việc nhiều gấp 3 công việc đó trong $8$ tuần, người thứ 3 có thể hoàn thành 1 công việc nhiều gấp $5$ công việc đó trong $12$ tuần. Nếu 3 người cùng làm công việc ban đầu thì họ kết thúc công việc trong bao lâu?
\end{baitoan}

\begin{baitoan}[\cite{Binh_Toan_6_tap_2}, VD23, p. 24]
	1 nông dân ra chợ bán hết số cam của mình cho $5$ người: Người thứ nhất mua $\dfrac{1}{2}$ số cam rồi mua thêm $\dfrac{1}{2}$ quả, người thứ 2 mua $\dfrac{1}{2}$ số cam còn lại rồi mua thêm $\dfrac{1}{2}$ quả, người thứ 3 mua $\dfrac{1}{2}$ số còn lại rồi rồi mua thêm $\dfrac{1}{2}$ quả, người thứ 4 mua $\dfrac{1}{2}$ số cam còn lại rồi mua thêm $\dfrac{1}{2}$ quả, người thứ 5 mua $\dfrac{1}{2}$ số cam còn lại rồi mua thêm $\dfrac{1}{2}$ quả thì vừa hết. Tính số cam người nông dân đem bán \& số cam mỗi người khách đã mua.
\end{baitoan}

\begin{baitoan}[\cite{Binh_Toan_6_tap_2}, VD24, p. 26, bài toán Euler]
	1 người cha khi qua đời để lại di chúc chia gia tài cho các con: Người con thứ nhất được chia $100$ cuaron \& $\dfrac{1}{10}$ số còn lại, người con thứ 2  được chia $200$ cuaron \& $\dfrac{1}{10}$ số còn lại, người con thứ 3 được chia $300$ cuaron $\dfrac{1}{10}$ số còn lại, $\ldots$ Cứ tiếp tục như vậy thì toàn bộ gia sản được chia đều cho các con. Tính số tiền của gia tài đó \& số t iền mỗi người con được chia.
\end{baitoan}

\begin{baitoan}[\cite{Binh_Toan_6_tap_2}, 62., pp. 26--27]
	2 công nhân phải làm 1 số dụng cụ, tổng số lượng chưa đến $1000$ chiếc. Trong 3 ngày, người thứ nhất lần lượt làm được $\dfrac{1}{7},\dfrac{1}{6},\dfrac{9}{20}$ kế hoạch của mình, người thứ 2 lần lượt làm được $\dfrac{1}{4},\dfrac{3}{11},\dfrac{3}{7}$ kế hoạch của mình. Tính số dụng cụ mỗi người phải làm, biết số dụng cụ mỗi người đã làm mỗi ngày là các số tự nhiên.
\end{baitoan}

\begin{baitoan}[\cite{Binh_Toan_6_tap_2}, 63., p. 27]
	2 vòi nước I, II cùng chảy vào 1 bể rỗng thì {\rm12 ph} đầy bể. Nếu vòi I chảy trong {\rm18 ph} rồi khóa lại, vòi II chảy tiếp trong {\rm8 ph} thì cũng đầy bể. Vòi I chảy 1 mình bao lâu đầy bể?
\end{baitoan}

\begin{baitoan}[\cite{Binh_Toan_6_tap_2}, 64., p. 27]
	1 công nhân có thể hoàn thành 1 công việc trong {\rm6 h}, người thứ 2 hoàn thành công việc đó trong {\rm15 h}. Đầu tiên người thứ nhất làm, sau đó chỉ có người thứ 2 làm, tổng cộng {\rm9h} thì xong công việc. Tính số giờ mỗi người làm \& tổng số dụng cụ làm được, biết người thứ nhất làm nhiều hơn người thứ 2 là $150$ dụng cụ.
\end{baitoan}

\begin{baitoan}[\cite{Binh_Toan_6_tap_2}, 65., p. 27]
	1 công nhân làm 1 mình xong 1 công việc trong $10$ ngày, người thứ 2 làm xong công việc đó trong $15$ ngày, còn người thứ 3 muốn hoàn thành công việc này cần 1 số ngày gấp $5$ lần số ngày 2 người trên cùng làm để hoàn thành công việc. Tính số ngày để hoàn thành công việc nếu cả 3 người cùng làm công việc đó.
\end{baitoan}

\begin{baitoan}[\cite{Binh_Toan_6_tap_2}, 66., p. 27]
	1 người cần $15$ ngày để làm xong 1 công việc, trong khi đó người thứ 2 làm xong công việc đó cần $18$ ngày. Cả 2 cùng làm $3$ ngày, sau đó chỉ còn người thứ nhất làm thêm $3$ ngày nữa thì có người thứ 3 đến giúp \& 2 người này làm $4$ ngày thì xong. Nếu người thứ 3 làm 1 mình thì sau bao lâu sẽ xong công việc này?
\end{baitoan}

\begin{baitoan}[\cite{Binh_Toan_6_tap_2}, 67., p. 27]
	3 máy cày cùng cày 1 cánh đồng. Lúc đầu chỉ có máy thứ nhất \& máy thứ 2 cày trong {\rm3 h}, sau đó máy thứ 2 nghỉ, máy thứ 3 vào làm thay với năng suất gấp đôi máy thứ 2 \& trong {\rm5 h} thì 2 máy này xong cánh đồng. Mỗi máy cày 1 mình xong cánh đồng đó trong bao lâu, biết nếu máy thứ nhất \& máy thứ 2 cùng làm thì sau {\rm12 h} xong công việc?
\end{baitoan}

\begin{baitoan}[\cite{Binh_Toan_6_tap_2}, 68., p. 27]
	1 người ra chợ bán trứng. Người khách thứ nhất mua $\dfrac{1}{2}$ số trứng rồi mua thêm $2$ quả, người thứ 2 mua $\dfrac{1}{2}$ số trứng còn lại rồi mua thêm $2$ quả, người thứ 3 mua $\dfrac{1}{2}$ số trứng còn lại rồi mua thêm $2$ quả, người thứ 4 mua $\dfrac{1}{2}$ số trứng còn lại rồi mua thêm $2$ quả thì hết. Tính số tiền trứng người bán trứng bán được, biết giá 1 chục trứng là $30000$ đồng.
\end{baitoan}

\begin{baitoan}[\cite{Binh_Toan_6_tap_2}, 69., p. 27]
	Trong dịp Tết trồng cây, khối 6 phân chia số cây cho các lớp để trồng: Lớp 6A trồng $10$ cây \& $\dfrac{1}{8}$ số còn lại, lớp 6B trồng $15$ cây \& $\dfrac{1}{8}$ số còn lại, lớp 6C trồng $20$ cây \& $\dfrac{1}{8}$ số còn lại, $\ldots$ Cứ chia như vây cho đến lớp cuối cùng thì vừa hết số cây \& số cây các lớp được chia để trồng đều bằng nhau. Tính số lớp 6 \& số cây mỗi lớp trồng.
\end{baitoan}

\begin{baitoan}[\cite{Binh_Toan_6_tap_2}, 70., p. 27]
	3 bạn $A,B,C$ chơi bài tú lơ khơ. Lần đầu, $24$ lá bài được chia cho 3 người, nhưng do chia nhầm nên số lá bài của 3 người không bằng nhau. Họ điều chỉnh lại: Lần chia thứ 2, A lấy 1 nửa số lá bài của mình chia đều cho 2 bạn. Lần chia thứ 3, B lấy 1 nửa số lá bài của mình chia đều cho 2 bạn. Lần chia thứ 4, C lấy 1 nửa số lá bài của mình chia đều cho 2 bạn. Sau lần chia thứ 4 thì mỗi người đều có $8$ lá bài. Tính số lá bài mỗi người được chia lần đầu.
\end{baitoan}

\begin{baitoan}[\cite{TLCT_THCS_Toan_6_so_hoc}, VD11.1, pp. 66--67]
	Tâm đã có 1 số điểm kiểm tra Toán \& còn 1 bài kiểm tra nữa, các bài kiểm tra đều tính hệ số 1 ngang nhau. Nếu bài kiểm tra này Tâm được $10$ điểm thì Tâm đạt điểm trung bình là $9$. Nhưng vì trong bài kiểm tra cuối, Tâm chỉ được $7.5$ điểm (điểm không làm tròn thành 1 số nguyên) nên điểm trung bình của Tâm chỉ là $8.5$. Tâm có tất cả bao nhiêu bài kiểm tra?
\end{baitoan}

\begin{baitoan}[\cite{TLCT_THCS_Toan_6_so_hoc}, VD11.2, p. 67]
	Có $20$ viên bi đỏ, $30$ viên bi trắng, \& 1 số viên bi xanh, tất cả để trong hộp. Nếu lấy ra trong hộp 1 viên bi thì cơ hội có thể lấy được 1 viên bi xanh là $\dfrac{9}{11}$. Tính số bi xanh.
\end{baitoan}

\begin{baitoan}[\cite{TLCT_THCS_Toan_6_so_hoc}, VD11.3, p. 67]
	1 lớp học mua 1 số vở về chia đều cho các học sinh. Nếu chỉ chia cho các bạn nữ thì mỗi bạn nhận $15$ quyển. Nếu chỉ chia cho các bạn nam thì mỗi bạn nhận $10$ quyển. Nếu chia đều cho tất cả các bạn trong lớp thì mỗi bạn nhận được bao nhiêu quyển vở?
\end{baitoan}

\begin{baitoan}[\cite{TLCT_THCS_Toan_6_so_hoc}, VD11.4, p. 68]
	4 bạn An, Bách, Cảnh, Dũng đi chơi, nhưng Dũng không mang tiền. An cho Dũng $\dfrac{1}{5}$ số tiền của mình. Bách cho Dũng $\dfrac{1}{4}$ số tiền của mình. Cảnh cho Dũng $\dfrac{1}{3}$ số tiền của mình. Kết quả số tiền Dũng nhận được từ 3 bạn đều bằng nhau. Cuối cùng Dũng có số tiền bằng mấy phần tổng số tiền của cả nhóm?
\end{baitoan}

\begin{baitoan}[\cite{TLCT_THCS_Toan_6_so_hoc}, 11.1., p. 69]
	Tổng của 3 số bằng $148$. Nếu nhân số thứ nhất với $4$, nhân số thứ 2 với $5$, nhân số thứ 3 với $6$ thì được 3 tích bằng nhau. Tính mỗi số.
\end{baitoan}

\begin{baitoan}[\cite{TLCT_THCS_Toan_6_so_hoc}, 11.2., p. 69]
	Tổng của 3 số bằng $147$. Biết $\dfrac{2}{3}$ số thứ nhất bằng $\dfrac{3}{4}$ số thứ 2 \& bằng $\dfrac{4}{5}$ số thứ 3. Tính mỗi số.
\end{baitoan}

\begin{baitoan}[\cite{TLCT_THCS_Toan_6_so_hoc}, 11.3., p. 69]
	Trong 1 buổi đi tham quan, số nữ đăng ký tham gia bằng $\dfrac{1}{4}$ số nam. Nhưng sau đó 1 bạn nữ xin nghỉ, 1 bạn nam xin đi thêm nên số nữ đi tham quan bằng $\dfrac{1}{5}$ số nam. Tính số học sinh nữ \& nam đã đi tham quan.
\end{baitoan}

\begin{baitoan}[\cite{TLCT_THCS_Toan_6_so_hoc}, 11.4., p. 69]
	Tú có 2 ngăn sách. Số sách ở ngăn I bằng $\dfrac{2}{5}$ tổng số sách ở 2 ngăn. Tú cho bạn mượn $4$ quyển sách ở ngăn I nên số sách ở ngăn I bằng $\dfrac{1}{3}$ tổng số sách ở 2 ngăn. Tính tổng số sách ở 2 ngăn lúc đầu.
\end{baitoan}

\begin{baitoan}[\cite{TLCT_THCS_Toan_6_so_hoc}, 11.5., p. 70]
	Hiện nay, tuổi mẹ gấp 3 tuổi con. Cách đây 4 năm, tuổi mẹ gấp 4 tuổi con. Tính tuổi mỗi người hiện nay.
\end{baitoan}

\begin{baitoan}[\cite{TLCT_THCS_Toan_6_so_hoc}, 11.6., p. 70]
	2 máy cày làm việc trên 1 cánh đồng. Nếu cả 2 máy cùng cày thì {\rm10 h} xong công việc. Nhưng thực tế 2 máy chỉ cùng làm việc {\rm7 h} đầu, sau đó máy thứ nhất đi cày nơi khác, máy thứu 2 làm tiếp {\rm9 h} nữa mới xong. Nếu máy thứ 2 làm việc 1 mình thì trong bao lâu cày xong cánh đồng?
\end{baitoan}

\begin{baitoan}[\cite{TLCT_THCS_Toan_6_so_hoc}, 11.7., p. 70]
	3 vòi nước I, II, III nếu chảy 1 mình vào 1 bể cạn thì chảy đầy bể lần lượt trong {\rm4 h, 6 h, 9 h}. Lúc đầu, mở 2 vòi I \& II trong {\rm1 h 30 ph}, sau đó đóng vòi I rồi mở tiếp vòi III cùng chảy với vòi II cho đến khi đầy bể. Vòi III chảy trong bao lâu?
\end{baitoan}

\begin{baitoan}[\cite{TLCT_THCS_Toan_6_so_hoc}, 11.8., p. 70]
	Có 3 vòi nước chảy vào 1 bể cạn. Nếu 2 vòi I \& II cùng chảy thì bể đầy sau {\rm45 ph}. Nếu 2 vòi II \& III cùng chảy thì bể đầy sau {\rm1 h}. Nếu 2 vòi I \& III cùng chảy thì bể đầy sau {\rm36 ph}. (a) Nếu cả 3 vòi cùng chảy thì bể đầy trong bao lâu? (b) Riêng mỗi vòi chảy 1 mình thì bể đầy trong bao lâu?
\end{baitoan}

\begin{baitoan}[\cite{TLCT_THCS_Toan_6_so_hoc}, 11.9., p. 70]
	3 người đến cửa hàng mua 1 số táo. Người I mua $\dfrac{1}{2}$ số táo rồi mua thêm $\dfrac{1}{2}$ quả. Người II mua $\dfrac{2}{3}$ số còn lại rồi mua thêm $\dfrac{2}{3}$ quả. Người III mua $\dfrac{3}{4}$ số còn lại rồi mua thêm $\dfrac{3}{4}$ quả thì vừa hết số táo của cửa hàng. Tính số táo của cửa hàng có lúc đầu.
\end{baitoan}

\begin{baitoan}[\cite{TLCT_THCS_Toan_6_so_hoc}, 11.10., p. 70]
	1 số học sinh được thưởng 1 số vở. Bạn I được thưởng $2$ quyển vở \& $\dfrac{1}{5}$ số còn lại. Bạn II được thưởng $4$ quyển vở \& $\dfrac{1}{5}$ số còn lại. Bạn III được thưởng $6$ quyển vở \& $\dfrac{1}{5}$ số còn lại. $\ldots$ Cứ như vậy thì số vở được chia đều cho các bạn \& không còn thừa quyển nào. Tính số học sinh được thưởng \& số vở.
\end{baitoan}

%------------------------------------------------------------------------------%

\section{Movement Problem -- Toán Chuyển Động}
\fbox{1} \fbox{$s = vt,v = \dfrac{s}{t},t = \dfrac{s}{v}$}. \fbox{2} Thời gian để 2 chất điểm gặp nhau trong chuyển động cùng chiều: \fbox{$t = \dfrac{s}{v_1 - v_2}$}, trong chuyển động ngược chiều: \fbox{$t = \dfrac{s}{v_1 + v_2}$}, với $t$: thời gian để 2 chất điểm gặp nhau, $s$: khoảng cách lúc đầu giữa 2 chất điểm, $v_1,v_2$: 2 vận tốc của 2 chất điểm. \fbox{3} \textit{Chuyển động có dòng nước}: Vận tốc xuôi $=$ vận tốc thuyền $+$ vận tốc dòng nước. Vận tốc ngược $=$ vận tốc thuyền $-$ vận tốc dòng nước.

\begin{baitoan}[\cite{Binh_Toan_6_tap_2}, VD48, p. 48]
	1 người đi từ A đến B với vận tốc {\rm15 km{\tt/}h}. Sau đó {\rm1 h 30 ph}, người thứ 2 cũng rời A đi về B với vận tốc {\rm20 km{\tt/}h} \& đến B trước người thứ nhất {\rm30 ph}. Tính quãng đường AB.
\end{baitoan}

\begin{baitoan}[\cite{Binh_Toan_6_tap_2}, VD49, p. 49]
	Đồng hồ đang chỉ {\rm4:10}. Sau ít nhất bao lâu thì 2 kim đồng hồ nằm đối diện nhau trên 1 đường thẳng?
\end{baitoan}

\begin{baitoan}[\cite{Binh_Toan_6_tap_2}, VD50, p. 49]
	2 xe ôtô đi từ 2 điểm $A,B$ về phía nhau, xe thứ nhất khởi hành từ A lúc {\rm7:00}, xe thứ 2 khởi hành từ B lúc {\rm7:10}. Biết để đi hết quãng đường AB, xe thứ nhất cần {\rm2 h}, xe thứ 2 cần {\rm3 h}. 2 xe gặp nhau lúc mấy giờ?
\end{baitoan}

\begin{baitoan}[\cite{Binh_Toan_6_tap_2}, VD51, p. 50]
	Trên quãng đường AB, 2 xe ôtô đi từ A \& từ B ngược chiều nhau. Nếu 2 xe khởi hành cùng 1 lúc thì chúng gặp nhau tại 1 điểm cách A {\rm12 km}, cách B {\rm18 km}. Nếu muốn gặp nhau ở chính giữa đường thì xe thứ nhất (đi từ A) phải khởi hành trước xe kia {\rm10 ph}. Tính vận tốc mỗi xe.
\end{baitoan}

\begin{baitoan}[\cite{Binh_Toan_6_tap_2}, VD52, p. 50]
	1 xe lửa đi hết 1 cái cầu dài {\rm12 m} hết {\rm12 s} \& đi hết 1 cái cầu dài {\rm148 m} hết {\rm20 s}. Tính chiều dài \& vận tốc của xe lửa.
\end{baitoan}

\begin{baitoan}[\cite{Binh_Toan_6_tap_2}, VD53, p. 51]
	1 canô chạy xuôi khúc sông AB hết {\rm6 h} \& chạy ngược khúc sông ấy hết {\rm9 h}. Tính thời gian để 1 phao trôi theo dòng nước từ A đến B.
\end{baitoan}

\begin{baitoan}[\cite{Binh_Toan_6_tap_2}, VD54, p. 51]
	1 người đi xe đạp từ A đến B gồm 1 đoạn lên dốc AC \& 1 đoạn xuống dốc CB. Thời gian đi AB là {\rm2 h}, thời gian về BA là {\rm1 h 45 ph}. Tính chiều dài quãng đường AB biết cứ lúc lên dốc thì người đó đi với vận tốc {\rm10 km{\tt/}h}, cứ lúc xuống dốc thì người đó đi với vận tốc {\rm15 km{\tt/}h}.
\end{baitoan}

\begin{baitoan}[\cite{Binh_Toan_6_tap_2}, VD55, p. 52]
	1 xe tải đi từ A đến B, vận tốc {\rm40 km{\tt/}h}. Sau đó 1 thời gian, 1 xe du lịch rời A, vận tốc {\rm60 km{\tt/}h}, \& như vậy sẽ đến B cùng lúc với xe tải. Nhưng đi đến C, được $\dfrac{1}{5}$ quãng đường AB, xe tải giảm vận tốc xuống còn {\rm35 km{\tt/}h}, do đó xe du lịch gặp xe tải ở D, cách B {\rm30 km}. Tính quãng đường AB.
\end{baitoan}

\begin{baitoan}[\cite{Binh_Toan_6_tap_2}, VD56, p. 53]
	1 ôtô đi quãng đường AB với vận tốc \& thời gian dự định. Nếu ôtô tăng vận tốc thêm $20\%$ thì đến B trước dự định {\rm1 h}. Nếu ôtô đi {\rm120 km} rồi tăng vận tốc thêm $25\%$ thì đến B trước dự định {\rm48 ph}. Tính: (a) Thời gian dự định đi hết quãng đường AB. (b) Chiều dài quãng đường AB.
\end{baitoan}

\begin{baitoan}[\cite{Binh_Toan_6_tap_2}, VD57, p. 54]
	An đi từ A \& đã đến B gặp bạn đúng giờ hẹn. An nói với bạn rằng nếu An đi với vận tốc ít hơn vận tốc đã đi {\rm6 km{\tt/}h} thì đến B sau giờ hẹn {\rm2 h}. Còn nếu An đi với vận tốc nhiều hơn vận tốc đã đi {\rm10 km{\tt/}h} thì đến B trước giừo hẹn {\rm2 h}. Tính thời gian An đã đi quãng đường AB \& chiều dài quãng đường AB.
\end{baitoan}

\begin{baitoan}[\cite{Binh_Toan_6_tap_2}, VD58, p. 54]
	An đi từ A \& đã đến B gặp bạn đúng giờ hẹn. An nói với bạn rằng nếu An đi với vận tốc ít hơn vận tốc đã đi $v_1 = 16$ {\rm km{\tt/}h} thì đến B sau giờ hẹn $t_1 = 2$ {\rm h}. Còn nếu An đi với vận tốc nhiều hơn vận tốc đã đi $v_2 = 20$ {\rm km{\tt/}h} thì đến B trước giừo hẹn $t_2 = 1$ {\rm h}. Tính thời gian An đã đi quãng đường AB \& chiều dài quãng đường AB cụ thể \& theo $v_1,v_2,t_1,t_2\in(0,\infty)$.
\end{baitoan}

\begin{baitoan}[\cite{Binh_Toan_6_tap_2}, VD59, p. 56]
	1 người đi xe đạp từ A đến B, đi từ A với vận tốc {\rm10 km{\tt/}h}, nhưng đi từ chính giữa đường đến B với vận tốc {\rm15 km{\tt/}h}. Tính vận tốc trung bình của người đó đi trên cả quãng đường.
\end{baitoan}

\begin{baitoan}[Công thức tính vận tốc trung bình]
	(a) 1 người đi xe đạp từ A đến B, đi từ A với vận tốc $v_1$ {\rm km{\tt/}h}, nhưng đi từ 1 điểm C nằm giữa $A,B$ với vận tốc $v_2$ {\rm km{\tt/}h}. Tính vận tốc trung bình của người đó đi trên cả quãng đường. (b) 1 người đi đoạn đường gấp khúc $A_1A_2\ldots A_nA_{n+1}$, $n\in\mathbb{N},n\ge2$, với vận tốc trên các đoạn đường $A_1A_2,A_2A_3,\ldots,A_{n-1}A_n,A_nA_{n+1}$ lần lượt là $v_1,v_2,\ldots,v_n$. Tính vận tốc trung bình của người đó đi trên cả quãng đường.
\end{baitoan}

\begin{baitoan}[\cite{Binh_Toan_6_tap_2}, 123., p. 56]
	(a) 1 người đi từ A đến B với vận tốc {\rm40 km{\tt/}h}. Đi được nửa đường, người đó dừng lại chữa xe trong {\rm30 ph}, nên để đến B đúng hạn người đó đi tiếp với vận tốc {\rm50 km{\tt/}h}. Tính quãng đường AB. (b) 1 ôtô đi từ A đến B với vận tốc {\rm30 km{\tt/}h}. Lúc về sau khi đi được $\dfrac{1}{3}$ quãng đường với vận tốc cũ, xe dừng lại chữa mất {\rm40 ph}, nên muốn thời gian về bằng thời gian đi, ôtô đã đi với vận tốc {\rm36 km{\tt/}h}. Tính quãng đường AB.
\end{baitoan}

\begin{baitoan}[\cite{Binh_Toan_6_tap_2}, 124., p. 56]
	1 ôtô đi từ A đến B. Người lái xe thấy nếu xe đi với vận tốc {\rm45 km{\tt/}h} thì đến B sau giờ hẹn {\rm10 ph}, còn nếu xe đi với vận tốc {\rm55 km{\tt/}h} thì đến B trước giờ hẹn {\rm6 ph}. Tính quãng đường AB.
\end{baitoan}

\begin{baitoan}[\cite{Binh_Toan_6_tap_2}, 125., p. 56]
	3 ôtô cùng khởi hành 1 lúc từ A để đến B. Vận tốc xe I là {\rm40 km{\tt/}h}, vận tốc xe II là {\rm50 km{\tt/}h}. Tính quãng đường AB \& vận tốc xe III biết xe III đến B trước xe I {\rm20 ph} \& sau xe II {\rm16 ph}.
\end{baitoan}

\begin{baitoan}[\cite{Binh_Toan_6_tap_2}, 126., p. 56]
	Long đi từ A đến B với vận tốc không đổi. Lúc {\rm9:00} Long đã đi được $\dfrac{1}{6}$ quãng đường, đến {\rm11:00} Long đi được $\dfrac{3}{4}$ quãng đường. Long đến B lúc mấy giờ?
\end{baitoan}

\begin{baitoan}[\cite{Binh_Toan_6_tap_2}, 127., p. 56]
	1 người đi xe máy từ A đến B với vận tốc {\rm25 km{\tt/}h} rồi đi tiếp từ B đến C với vận tốc {\rm20 km{\tt/}h}. Quãng đường BC dài hơn AB {\rm30 km}. Thời gian đi trên quãng đường BC nhiều hơn thời gian đi trên quãng đường AB là {\rm2 h}. Tính quãng đường $AB,BC$.
\end{baitoan}

\begin{baitoan}[\cite{Binh_Toan_6_tap_2}, 128., p. 57]
	Phúc, Du cùng khởi hành 1 lúc từ nhà mình \& đi về phía nhau. Phúc đi nhanh gấp $\dfrac{4}{3}$ Du \& họ gặp nhau sau {\rm72 ph}. Phúc phải khởi hành sau Du bao lâu để họ gặp nhau ở chính giữa quãng đường?
\end{baitoan}

\begin{baitoan}[\cite{Binh_Toan_6_tap_2}, 129., p. 57]
	(a) 1 xe lửa đi vượt qua 1 cột điện trong {\rm16 s} \& đi hết 1 cái cầu dài {\rm210 m} trong {\rm30 s}. Tính vận tốc xe lửa \& chiều dài xe lửa. (b) 1 xe lửa đi vượt qua 1 người đi xe đạp cùng chiều trong {\rm16 s}. Tính chiều dãie lửa biết vận tốc người đi xe đạp, xe lửa lần lượt là {\rm18 km{\tt/}h, 54 km{\tt/}h}.
\end{baitoan}

\begin{baitoan}[\cite{Binh_Toan_6_tap_2}, 130., p. 57]
	(a) 1 người đi bộ với vận tốc {\rm1 m{\tt/}s} đi ngược chiều với xe lửa \& gặp xe lửa trong {\rm10 s}. Tính vận tốc xe lửa biết nó dài {\rm170 m}. (b) 1 xe lửa dài {\rm110 m} đi qua 1 cầu dài {\rm160 m} hết {\rm18 s} \& đi vượt 1 người đi xe đạp cùng chiều hết {\rm10 s}. Tính vận tốc của người đi xe đạp.
\end{baitoan}

\begin{baitoan}[\cite{Binh_Toan_6_tap_2}, 131., p. 57]
	Trên chiếc cầu AB, 1 con chó đang ở vị trí C với $CA = \frac{1}{3}CB$ thì nhìn thấy 1 ôtô sắp lên cầu từ phía A. Con chó chạy về phía A \& gặp lại ôtô tại A. Nếu con chó chạy về phía B thì nó gặp ôtô tại B. Tính vận tốc của con chó, biết vận tốc ôtô là {\rm60 km{\tt/}h}.
\end{baitoan}

\begin{baitoan}[\cite{Binh_Toan_6_tap_2}, 132., p. 57]
	Trong 1 cuộc thi chạy {\rm2000 m}, các vận động viên chạy với vận tốc không đổi trên suốt quãng đường. Người thứ nhất về đích khi người thứ 2 còn cách đích {\rm200 m} \& người thứ 3 còn cách đích {\rm290 m}. Khi người thứ 2 đến đích thì người thứ 3 còn cách đích bao nhiêu {\rm m}?
\end{baitoan}

\begin{baitoan}[\cite{Binh_Toan_6_tap_2}, 133., p. 57]
	Đạt chạy nhanh gấp 2 lần so với đi bộ. Từ nhà đến trường, có lúc Đạt chạy, có lúc đi bộ. Hôm trước, thời gian Đạt chạy gấp đôi thời gian đi bộ. Hôm sau thời gian Đạt đi bộ gấp đôi thời gian Đạt chạy. Từ nhà đến trường, hôm trước Đạt cần {\rm12 ph}, hôm sau cần bao lâu?
\end{baitoan}

\begin{baitoan}[\cite{Binh_Toan_6_tap_2}, 134., p. 57]
	2 người đi xe đạp cùng đi từ A về phía B. Người thứ nhất khởi hành lúc {\rm8:00}, người thứ 2 khởi hành lúc {\rm8:30}. 2 người gặp nhau lúc mấy giờ biết quãng đường người thứ nhất đi trong {\rm45 ph} bằng quãng đường người thứ 2 đi trong {\rm40 ph}?
\end{baitoan}

\begin{baitoan}[\cite{Binh_Toan_6_tap_2}, 135., p. 57]
	2 người đi xe đạp từ 2 địa điểm cách nhau {\rm12 km} \& đi cùng chiều. Nếu họ không khởi hành cùng 1 lúc thì sau {\rm3 h}, người đi nhanh đuổi kịp người đi chậm. Nếu người đi nhanh đi sau người đi chậm {\rm1 h} thì sau {\rm5 h 30 ph} người đi nhanh mới đuổi kịp người đi chậm. Tính vận tốc mỗi người.
\end{baitoan}

\begin{baitoan}[\cite{Binh_Toan_6_tap_2}, 136., p. 57]
	Lúc {\rm7:00}, 1 người đi từ A để đến B. Lúc {\rm10:00}, người thứ 2 đi từ B để đến A, với vận tốc lớn hơn vận tốc người thứ nhất là {\rm3 km{\tt/}h}. 2 người gặp nhau lúc {\rm14:00}. Tính vận tốc mỗi người, biết quãng đường AB dài {\rm177 km}.
\end{baitoan}

\begin{baitoan}[\cite{Binh_Toan_6_tap_2}, 137., p. 57]
	Trên quãng đường AB dài {\rm235 km}, 1 ôtô đi từ A đến C trong {\rm3 h}, sau đó đi tiếp từ C đến B trong {\rm2 h} với vận tốc lớn hơn vận tốc trên quãng đường AC là {\rm5 km{\tt/}h}. Tính vận tốc ôtô đi trên quãng đường AC.
\end{baitoan}

\begin{baitoan}[\cite{Binh_Toan_6_tap_2}, 138., p. 57]
	1 ôtô đi từ A qua C để đến B, quãng đường CB ngắn hơn quãng đường AC {\rm55 km}. Sau khi đi từ A đến C trong {\rm4 h}, người lái xe tính nếu từ C ôtô giảm vận tốc đi {\rm5 km{\tt/}h} thì sau {\rm3 h} nữa sẽ tới B. Tính quãng đường AB.
\end{baitoan}

\begin{baitoan}[\cite{Binh_Toan_6_tap_2}, 139., p. 58]
	(a) Sau lúc {\rm12:00} thì sớm nhất lúc mấy giờ 2 kim đồng hồ lại trùng nhau? (b) Trong 1 ngày đêm, 2 kim đồng hồ tạo với nhau góc vuông bao nhiêu lần?
\end{baitoan}

\begin{baitoan}[\cite{Binh_Toan_6_tap_2}, 140., p. 58]
	Hiện tại là {\rm3:00}, kim phút tạo với kim giờ 1 góc vuông. Tính thời điểm 2 kim tạo với nhau góc vuông: (a) Lần thứ nhất tiếp theo. (b) Lần thứ 2 tiếp theo. (c) Lần thứ $n\in\mathbb{N}^\star$ tiếp theo.
\end{baitoan}

\begin{baitoan}[\cite{Binh_Toan_6_tap_2}, 141., p. 58]
	Hiện tại là {\rm12:00}. Tính thời điểm kim phút \& kim giờ của đồng hồ tạo với nhau góc $110^\circ$ lần tiếp theo.
\end{baitoan}

\begin{baitoan}
	Hiện tại là $x:y$ với $x\in\{0,1,2,\ldots,23\},y\in\{00,01,02,\ldots,59\}$. Tính thời điểm kim phút \& kim giờ của đồng hồ tạo với nhau góc $\alpha$ lần tiếp theo.
\end{baitoan}

\begin{baitoan}[\cite{Binh_Toan_6_tap_2}, 142., p. 58]
	Quãng đường AB dài {\rm43.8 km}. Lúc {\rm6:00} xe thứ nhất đi từ A để đến B, lúc {\rm6:25} xe thứ 2 đi từ B để đến A. 2 xe gặp nhau trên đường đi lúc {\rm6:45} \& đến khi đó thì xe thứ nhất đã đi nhiều hơn xe thứ 2 {\rm11.4 km}. Tính vận tốc mỗi xe.
\end{baitoan}

\begin{baitoan}[\cite{Binh_Toan_6_tap_2}, 143., p. 58]
	Nhà Tùng ở cạnh đường xe lửa từ ga A đến ga B. 1 hôm xe lửa đi qua nhà Tùng \& đến ga B lúc {\rm16:00}. Hôm sau, xe lửa lại đi từ B lúc {\rm5:00} để về A với vận tốc như ngày hôm trước. Tùng ngạc nhiên thấy giờ xe lửa đi qua nhà mình trong cả 2 ngày đều như nhau. Xe lửa đã đi qua nhà Tùng lúc mấy giờ?
\end{baitoan}

\begin{baitoan}[\cite{Binh_Toan_6_tap_2}, 144., p. 58]
	Quãng đường AB dài {\rm20 km}. Lúc {\rm6:00} ngày 1.6, Minh đi từ A \& đến B lúc {\rm8:00}. Hôm sau, Minh đi từ B lúc {\rm7:00} \& đến A lúc {\rm8:20}. Như vậy trên quãng đường AB có 1 địa điểm C mà Minh có mặt cùng 1 giờ trong cả 2 ngày. Tính xem địa điểm C cách A bao nhiêu {\rm km} \& Minh có mặt ở C lúc mấy giờ?
\end{baitoan}

\begin{baitoan}[\cite{Binh_Toan_6_tap_2}, 145., p. 58]
	Để đi từ A đến B, xe thứ nhất cần {\rm1 h}, xe thứ 2 cần {\rm40 ph}. Nếu xe thứ 2 đi sau xe thứ nhất {\rm10 ph} thì sau bao lâu xe thứ 2 đuổi kịp xe thứ nhất?
\end{baitoan}

\begin{baitoan}[\cite{Binh_Toan_6_tap_2}, 146., p. 58]
	Quãng đường AB dài {\rm44 km}. Lúc {\rm6:00}, người thứ nhất đi từ A để đến B. Lúc {\rm7:08}, người thứ 2 đi từ B để đến A. Họ gặp nhau lúc {\rm9:00}. Tính vận tốc mỗi người, biết vận tốc người thứ nhất lớn hơn vận tốc người thứ 2 là {\rm2.5 km{\tt/}h}.
\end{baitoan}

\begin{baitoan}[\cite{Binh_Toan_6_tap_2}, 147., p. 58]
	An đi xe đạp từ A đến B, vận tốc $8\frac{1}{3}$ {\rm km{\tt/}h}. Trên đường đi An gặp Bắc đi từ B lại, An đi ngược lại nói chuyện với bạn với vận tốc bằng vận tốc của bạn. Sau khi cùng đi với bạn {\rm3.75 km}, An quay lại đi với vận tốc như lúc đầu \& đến B chậm hơn dự định {\rm57 ph}. Tính vận tốc của Bắc.
\end{baitoan}

\begin{baitoan}[\cite{Binh_Toan_6_tap_2}, 148., p. 58]
	1 canô xuôi khúc sông từ A đến B hết {\rm2 h} \& ngược khúc sông đó hết {\rm3 h}. Biết vận tốc dòng nước là {\rm3 km{\tt/}h}, tính quãng sông AB.
\end{baitoan}

\begin{baitoan}[\cite{Binh_Toan_6_tap_2}, 149., pp. 58--59]
	Khi đi gặp chuyến xuôi dòng. Nhẹ nhàng đến bến chỉ trong $4$ giờ. Khi về từ lúc xuống đò. Đến khi cập bến $8$ giờ hết veo. Hỏi rằng riêng 1 khóm bèo. Trôi theo dòng nước hết bao nhiêu giờ?
\end{baitoan}

\begin{baitoan}[\cite{Binh_Toan_6_tap_2}, 150., p. 59]
	Thành tập bơi ngược dòng sông Hồng. Đến cầu Long Biên, Thành đánh rơi bình nước. Sau {\rm20 ph}, anh mới biết \& quay lại tìm thì thấy bình nước ở cạnh bến Phà Đen cách cầu Long Biên {\rm2 km}. Tính vận tốc dòng nước.
\end{baitoan}

\begin{baitoan}[\cite{Binh_Toan_6_tap_2}, 151., p. 59]
	1 người đi từ A đến B. Lúc đầu người đó đi với vận tốc {\rm15 km{\tt/}h}. Khi còn cách B là {\rm21 km}, người đó tăng vận tốc thành {\rm18 km{\tt/}h}. Lúc về người đó đi với vận tốc không đổi là {\rm18 km{\tt/}h}. Thời gian cả đi lẫn về là {\rm4 h 10 ph}. Tính quãng đường AB.
\end{baitoan}

\begin{baitoan}[\cite{Binh_Toan_6_tap_2}, 152., p. 59]
	1 người đi từ A đến B gồm 1 đoạn lên dốc, 1 đoạn xuống dốc, vận tốc lên dốc là {\rm12 km{\tt/}h}, vận tốc xuống dốc là {\rm20 km{\tt/}h}, tổng cộng hết {\rm1 h 35 ph}. Lúc về, người đó đi từ B về A, vận tốc lên dốc cũng là {\rm12 km{\tt/}h}, vận tốc xuống dốc cũng là {\rm20 km{\tt/}h}, tổng cộng hết {\rm1 h 45 ph}. Tính quãng đường AB.
\end{baitoan}

\begin{baitoan}[\cite{Binh_Toan_6_tap_2}, 153., p. 59]
	1 người đi quãng AB gồm 1 đoạn lên dốc, 1 đoạn xuống dốc. Thời gian tổng cộng cả đi lẫn về là {\rm7 h}. Biết cứ lên dốc thì người đó đi với vận tốc {\rm18 km{\tt/}h}, cứ xuống dốc thì người đó đi với vận tốc {\rm24 km{\tt/}h}. Tính quãng đường AB.
\end{baitoan}

\begin{baitoan}[\cite{Binh_Toan_6_tap_2}, 154., p. 59]
	1 người đi xe đạp từ A đến B gồm 1 đoạn lên dốc \& 1 đoạn nằm ngang hết tổng cộng {\rm2 h}. Lúc về người đó đi từ B đến A hết {\rm1 h 10 ph}. Biết vận tốc trên đoạn lên dốc là {\rm8 km{\tt/}h}, vận tốc trên đoạn xuống dốc là {\rm18 km{\tt/}h}, vận tốc trên đoạn nằm ngang là {\rm12 km{\tt/}h}. Tính quãng đường AB.
\end{baitoan}

\begin{baitoan}[\cite{Binh_Toan_6_tap_2}, 155., p. 59]
	1 người đi quãng đường AB gồm đoạn AC \& CB. Lúc đi, vận tốc trên AC là {\rm12 km{\tt/}h}, vận tốc trên CB là {\rm8 km{\tt/}h}, hết {\rm3 h 30 ph}. Lúc về, vận tốc trên BC là {\rm30 km{\tt/}h}, vận tốc trên CA là {\rm20 km{\tt/}h} hết {\rm1 h 36 ph}. Tính quãng đường AB.
\end{baitoan}

\begin{baitoan}[\cite{Binh_Toan_6_tap_2}, 156., p. 59]
	1 ôtô đi từ A đến B, vận tốc {\rm50 km{\tt/}h}. Sau đó 1 thời gian, xe thứ 2 rời A, vận tốc {\rm60 km{\tt/}h} \& như vậy sẽ gặp xe thứ nhất tại 1 điểm còn cách B {\rm10 km}. Nhưng đi được $\dfrac{1}{5}$ quãng đường, xe thứ nhất giảm vận tốc còn {\rm45 km{\tt/}h} nên xe thứ 2 đã gặp xe thứ nhất tại chỗ còn cách B {\rm30 km}. Tính quãng đường AB.
\end{baitoan}

\begin{baitoan}[\cite{Binh_Toan_6_tap_2}, 157., p. 59]
	1 ôtô lên dốc \& xuống dốc qua 1 quả đồi, quãng đường lên dốc \& xuống dốc bằng nhau. Vì xe cũ nên vận tốc lên dốc không quá {\rm15 km{\tt/}h}. Vận tốc trung bình của ôtô khi qua cả quả đồi có thể bằng {\rm30 km{\tt/}h} không?
\end{baitoan}

\begin{baitoan}[\cite{Binh_Toan_6_tap_2}, 158., p. 59]
	Lúc {\rm7:00}, 1 người đi xe đạp \& 1 người đi xe máy cùng rời A để đến B, vận tốc lần lượt là {\rm12 km{\tt/}h, 28 km{\tt/}h}. Lúc {\rm7:30}, 1 ôtô với vận tốc {\rm35 km{\tt/}h} cũng rời A để đến B. Ôtô ở vị trí chính giữa 2 người kia vào lúc mấy giờ?
\end{baitoan}

\begin{baitoan}[\cite{Binh_Toan_6_tap_2}, 159., p. 59]
	3 người cùng khởi hành 1 lúc từ $A,B,C$ về phía D với vận tốc lần lượt là {\rm30 m{\tt/}ph, 50 m{\tt/}ph, 20 m{\tt/}ph}. Sau bao lâu thì người thứ 3 ở vị trí chính giữa 2 người kia biết $AB = 400$ {\rm m}, $BC = 300$ {\rm m}, 4 địa điểm $A,B,C,D$ theo thứ tự như trên?
\end{baitoan}

\begin{baitoan}[\cite{Binh_Toan_6_tap_2}, 160., p. 60]
	Trên quãng đường AB có địa điểm C cách A {\rm10 km}. Lúc {\rm8:00}, người thứ nhất \& người thứ 2 rời A, người thứ 3 rời C, tất cả đi về phía B với vận tốc lần lượt là {\rm30 km{\tt/}h, 40 km{\tt/}h, 20 km{\tt/}h}. Lúc mấy giờ thì người thứ 3 có khoảng cách đến 2 người kia bằng nhau?
\end{baitoan}

\begin{baitoan}[\cite{Binh_Toan_6_tap_2}, 161., p. 60]
	3 ôtô cùng khởi hành 1 lúc từ A để đến B. Vận tốc xe thứ 2 bằng {\rm40 km{\tt/}h} \& bằng trung bình cộng của vận tốc 2 xe kia. 1 người đứng ở chính giữa quãng đường AB thấy xe thứ nhất đi qua lúc {\rm7:06}, xe thứ 2 đi qua lúc {\rm7:15}, xe thứ 3 đi qua lúc {\rm7:30}. 3 xe khởi hành từ A lúc mấy giờ?
\end{baitoan}

\begin{baitoan}[\cite{Binh_Toan_6_tap_2}, 162., p. 60]
	2 người cùng đi 1 lúc từ A đến B, quãng đường dài {\rm40 km}, vận tốc lần lượt bằng {\rm10 km{\tt/}h, 14 km{\tt/}h}. Sau bao lâu quãng đường còn lại đến B của người thứ nhất gấp 3 quãng đường còn lại đến B của người thứ 2?
\end{baitoan}

\begin{baitoan}[\cite{Binh_Toan_6_tap_2}, 163., p. 60]
	2 người cùng khởi hành lúc {\rm8:00} từ A để đến B với vận tốc lần lượt là {\rm12 km{\tt/}h, 30 km{\tt/}h}. Người thứ 2 đến B thì quay lại ngay \& gặp người thứ nhất lúc {\rm8:40}. (a) Tính quãng đường AB. (b) Người thứ 2 đến B lúc mấy giờ?
\end{baitoan}

\begin{baitoan}[\cite{Binh_Toan_6_tap_2}, 164., p. 60]
	2 người cùng khởi hành 1 lúc, người thứ nhất đi từ A đến B, người thứ 2 đi từ B đến A. Người thứ nhất đến B thì quay lại ngay, người thứ 2 đến A thì quay lại ngay. Họ gặp nhau lần thứ nhất tại C cách A {\rm10 km} \& gặp nhau lần thứ 2 tại D cách B {\rm8 km}. Tính quãng đường AB.
\end{baitoan}

\begin{baitoan}[\cite{Binh_Toan_6_tap_2}, 165., p. 60]
	2 bạn An, Bích cùng rời nhà lúc {\rm7:00}. An đến nhà Bích, Bích đến nhà An. Lần thứ nhất, 2 người đã đi qua nhau ở chỗ cách nhà Bích {\rm300 m} mà không biết. Anđi tiếp đến nhà Bích rồi quay lại ngay, Bích cũng đi tiếp đến nhà An rồi quay lại ngay. 2 người gặp nhau lần thứ 2 cách nhà An {\rm200 m}, lúc đó là {\rm7:20}. Tính khoảng cách từ nhà An đến nhà Bích \& vận tốc của mỗi người.
\end{baitoan}

\begin{baitoan}[\cite{Binh_Toan_6_tap_2}, 166., p. 60]
	2 người đi xe đạp cũng khởi hành lúc {\rm7:00}: người thứ nhất đi từ A đến B, người thứ 2 đi từ B đến A. Lúc {\rm7:30} họ gặp nhau lần thứ nhất ở M cách A {\rm6 km}, người thứ nhất tiếp tục đi đến B thì quay lại, người thứ 2 tiếp tục đi đến A thì quay lại. Họ gặp nhau lần thứ 2 ở N cách B {\rm4 km}. Tính: (a) Lúc 2 người gặp nhau lần thứ 2. (b) Chiều dài quãng đường AB. (c) Vận tốc mỗi người. (d) Khoảng cách đến A của người thứ nhất lúc người thứ 2 về đến B.
\end{baitoan}

\begin{baitoan}[\cite{Binh_Toan_6_tap_2}, 167., p. 60]
	2 người cùng khởi hành 1 lúc từ 2 địa điểm $A,B$. Người thứ nhất đi từ A đến B rồi quay lại ngay. Người thứ 2 đi từ B đến A rồi quay lại ngay. 2 người gặp nhau lần thứ 2 tại địa điểm C cách A {\rm6 km}. Tính quãng đường AB, biết vận tốc người thứ 2 bằng $\dfrac{2}{3}$ vận tốc người thứ nhất.
\end{baitoan}

\begin{baitoan}[\cite{Binh_Toan_6_tap_2}, 168., p. 61]
	An \& Bích đi từ A \& B ngược chiều nhau trên đường tròn đường kính AB. Họ gặp nhau lần thứ nhất ở C cách A {\rm60 m}, gặp nhau lần thứ 2 ở D cách B {\rm80 m}. Tính chu vi đường tròn.
\end{baitoan}

\begin{baitoan}[\cite{Binh_Toan_6_tap_2}, 169., p. 61]
	Trên 1 sân vận động, có nhiều làn chạy, làn trong cùng ngắn nhất có chiều dài {\rm400 m}, làn ngoài cùng dài nhất nhưng chưa quá {\rm500 m}. A chạy ở làn trong cùng, B chạy ở làn ngoài cùng, có cùng vạch xuất phát \& cùng chạy về 1 phía, vận tốc của B gấp 3 lần vận tốc của A. Lúc 2 người lần đầu tiên trở về vạch xuất phát thì A chạy được 3 vòng. Tính: (a) Số vòng B chạy được. (b) Độ dài đường chạy ngoài cùng, trong cùng.
\end{baitoan}

\begin{baitoan}[\cite{Binh_Toan_6_tap_2}, 170., p. 61]
	Quãng đường AB từ chân đồi A đến đỉnh đồi B dài {\rm700 m}. Đức \& Lộc cùng chạy từ A lên B rồi từ B xuống A, vận tốc xuống dốc gấp đôi vận tốc lên dốc. Họ xuất phát cùng lúc ở chân đồi, Đức đến đỉnh đồi trước, quay xuống ngay \& gặp Lộc cách đỉnh đồi {\rm70 m}. (a) Tính tỷ số vận tốc của Lộc \& Đức khi lên dốc. (b) Lúc Đức chạy xuống dốc đến chân đồi thì Lộc còn cách chân đồi bao xa?
\end{baitoan}

\begin{baitoan}[\cite{Binh_Toan_6_tap_2}, 171., p. 61]
	Hàng ngày, Phong được bố từ nhà đến trường đón \& bao giờ cũng về nhà lúc {\rm12:00}. Hôm ấy tan học sớm {\rm45 ph}, chờ bố thì lâu, Phong đi bộ về luôn \& gặp bố trên đường đi. Về nhà lúc {\rm11:40}, Phong nói với bố: Thế là con đi bộ trong {\rm25 ph}. Con tính sai rồi. -- Bố Phong cười nói. Tính giúp Phong.
\end{baitoan}

\begin{baitoan}[\cite{Binh_Toan_6_tap_2}, 172., p. 61]
	Hòa được bố từ nhà đến trường đón mỗi khi tan học. Hôm ấy, bố Hòa đến đón Hòa với giờ như thường lệ, nhưng do Hòa tan học sớm {\rm45 ph} \& đi bộ về nhà ngay nên Hòa gặp bố trên đường về \& 2 bố con về nhà sớm hơn thường lệ {\rm10 ph}. Tính quãng đường Hòa đã đi bộ, biết Hòa đi với vận tốc {\rm6 km{\tt/}h}.
\end{baitoan}

\begin{baitoan}[\cite{Binh_Toan_6_tap_2}, 173., pp. 61--62]
	3 người khởi hành cùng 1 lúc đi từ A đến B. Người thứ 2 đi chậm hơn người thứ nhất {\rm4 km{\tt/}h} nên đến B sau người thứ nhất {\rm1 h}. Người thứ 3 đi nhanh hơn người thứ nhất {\rm6 km{\tt/}h} nên đến B trước người thứ nhất {\rm1 h}. Tính thời gian người thứ nhất đi quãng đường AB \& chiều dài quãng đường đó.
\end{baitoan}

\begin{baitoan}[\cite{Binh_Toan_6_tap_2}, 174., p. 62]
	1 người đi từ A dến B với thời gian dự định. Người đó thấy nếu tăng thêm vận tốc {\rm9 km{\tt/}h} thì đến B sớm {\rm1 h}, còn nếu giảm vận tốc {\rm3 km{\tt/}h} thì đến B chậm {\rm30 ph}. Tính vận tốc \& thời gian dự định đi quãng đường AB của người đó.
\end{baitoan}

\begin{baitoan}[\cite{Binh_Toan_6_tap_2}, 175., p. 62]
	Trung đi từ A đến B với vận tốc {\rm24 km{\tt/}h}. Lúc về, trên nửa đầu của quãng đường (là BM), Trung đi với vận tốc {\rm20 km{\tt/}h}. Muốn thời gian về BA bằng thời gian đi AB thì trên quãng đường còn lại (là MA) Trung phải đi với vận tốc bao nhiêu?
\end{baitoan}

\begin{baitoan}[\cite{Binh_Toan_6_tap_2}, 176., p. 62]
	Kiên đi từ A đến B rồi trở về A, thời gian đi ít hơn thời gian về {\rm2 h}, vận tốc lúc về là {\rm10 km{\tt/}h}. Tính quãng đường AB, biết vận tốc trung bình cả đi \& về là {\rm12 km{\tt/}h}.
\end{baitoan}

\begin{baitoan}[\cite{Binh_Toan_6_tap_2}, 177., p. 62]
	(a) 1 máy bay có vận tốc trung bình trong cả chuyến bay là {\rm700 km{\tt/}h}. Trên quãng đường đầu, vận tốc của máy bay là {\rm800 km{\tt/}h}. Tính vận tốc của máy bay trên quãng đường sau, biết thời gian bay quãng đường đầu bằng $\frac{1}{3}$ thời gian cả chuyến bay. (b) 1 ôtô đi $\frac{2}{3}$ quãng đường AB với vận tốc {\rm40 km{\tt/}h} rồi đi phần còn lại với vận tốc {\rm60 km{\tt/}h}. Lúc về ôtô đi với vận tốc không đổi \& thời gian về bằng thời gian đi. Tính vận tốc trung bình của ôtô lúc về.
\end{baitoan}

\begin{baitoan}[\cite{Binh_Toan_6_tap_2}, 178., p. 62]
	1 người đi quãng đường từ A đến B gồm 3 đoạn $AC,CD,DB$ với vận tốc lần lượt là {\rm10 km{\tt/}h, 12 km{\tt/}h, 15 km{\tt/}h}. Lúc về, người đó đi $BD,DC,CA$ với vận tốc lần lượt là {\rm10 km{\tt/}h, 12 km{\tt/}h, 15 km{\tt/}h}. Tính quãng đường AB biết thời gian cả đi lẫn về hết {\rm3 h}.
\end{baitoan}

\begin{baitoan}[\cite{Binh_Toan_6_tap_2}, 179., p. 62]
	Trên quãng đường AC có địa điểm B. Lúc {\rm7:00}, người thứ nhất đi từ A, người thứ 2 đi từ B, cả 2 người cùng đến C lúc {\rm10:00}. Trên đường đi, người thứ 2 gặp 1 chiếc xe lửa đi từ C về A lúc {\rm8:30}, người thứ nhất gặp chiếc xe lửa đó lúc {\rm8:40}. Biết quãng đường AB dài {\rm30 km} \& vận tốc xe lửa gấp đôi vận tốc người thứ 2. Xe lửa đi từ C lúc mấy giờ \& quãng đường AC dài bao nhiêu?
\end{baitoan}

\begin{baitoan}[\cite{Binh_Toan_6_tap_2}, 180., p. 62]
	3 bạn cần đi từ A đến B dài {\rm22.5 km} trong {\rm2 h 30 ph} mà chỉ có 1 chiếc xe đạp, xe chỉ đèo được 1 người. Biết vận tốc đi bộ là {\rm5 km{\tt/}h}, vận tốc đi xe là {\rm15 km{\tt/}h} (cả lúc đi 1 mình lẫn lúc đèo bạn). Họ có thể đến B kịp giờ?
\end{baitoan}

\begin{baitoan}[\cite{Binh_Toan_6_tap_2}, 181., p. 62]
	1 thành phố có các đoàn tàu khởi hành sau các khoảng thời gian bằng nhau, chuyển động với vận tốc không đổi \& không dừng lại trên đường đi. 1 khách du lịch đi dọc đường tàu từ A đến B nhận thấy cứ {\rm7 ph} lại gặp 1 đoàn tàu đi cùng chiều với anh, cứ {\rm5 ph} lại gặp 1 đoàn tàu đi ngược lại. Nếu anh đứng lại thì sau bao lâu lại gặp 2 tàu cùng chiều đi qua?
\end{baitoan}

\begin{baitoan}[\cite{TLCT_THCS_Toan_6_so_hoc}, VD13.1, p. 75]
	Lúc {\rm8:00}, người thứ nhất đi từ A \& đến B lúc {\rm12:00}. Lúc {\rm8:30}, người thứ 2 đi từ A \& đến B lúc {\rm11:30}. Người thứ 2 đuổi kịp người thứ nhất lúc mấy giờ?
\end{baitoan}

\begin{baitoan}[\cite{TLCT_THCS_Toan_6_so_hoc}, VD13.2, p. 77]
	Trên quãng đường AB, 2 xe cùng khởi hành 1 lúc, xe tải đi từ A đến B hết {\rm6 h}, xe con đi từ B đến A hết {\rm4 h}. 2 xe gặp nhau sau bao lâu?
\end{baitoan}

\begin{baitoan}[\cite{TLCT_THCS_Toan_6_so_hoc}, VD13.3, p. 77]
	1 người phải đi từ A đến B trong {\rm5 h}. Lúc đầu, người đó đi với vận tốc {\rm30 km{\tt/}h}. Khi còn {\rm75 km} nữa thì được nửa đường, người đó đi với vận tốc {\rm45 km{\tt/}h} để kịp B đúng dự định. Tính quãng đường AB.
\end{baitoan}

\begin{baitoan}[\cite{TLCT_THCS_Toan_6_so_hoc}, VD13.4, p. 78]
	Hòa bơi xuôi dòng nước từ A đến B hết {\rm6 ph}, còn bơi ngược từ B về A hết {\rm10 ph}. 1 cụm bèo trôi theo dòng nước từ A đến B trong bao lâu?
\end{baitoan}

\begin{baitoan}[\cite{TLCT_THCS_Toan_6_so_hoc}, VD13.5, p. 79]
	1 xe lửa chạy với vận tốc {\rm45 km{\tt/}h}. Xe lửa chui vào 1 đường hầm có chiều dài gấp $9$ lần chiều dài của xe lửa \& cần $2$ phút để xe lửa vào \& ra khỏi đường hầm. Tính chiều dài của xe lửa.
\end{baitoan}

\begin{baitoan}[\cite{TLCT_THCS_Toan_6_so_hoc}, VD13.6, p. 80]
	1 ôtô đi nửa đầu của quãng đường AB với vận tốc {\rm30 km{\tt/}h} \& đi nửa sau với vận tốc {\rm60 km{\tt/}h}. Tính vận tốc trung bình của ôtô trên cả quãng đường AB.
\end{baitoan}

\begin{baitoan}[\cite{TLCT_THCS_Toan_6_so_hoc}, 13.1., p. 80]
	1 chiếc tàu chạy trên sông, khi còn {\rm90 km} nữa mới cập bến thì tàu bị thủng, cứ {\rm5 ph} có $2$ tấn nước tràn vào tàu. Nếu có $105$ tấn nước tràn vào tàu thì tàu sẽ bị chìm. Trên tàu có 1 máy bơm, mỗi giờ bơm ra được $10$ tấn nước. Tàu phải chạy ít nhất với vận tốc nào để khi cập bến, tàu vẫn chưa bị chìm?
\end{baitoan}

\begin{baitoan}[\cite{TLCT_THCS_Toan_6_so_hoc}, 13.2., p. 80]
	Hiện nay là {\rm4:00}. Sau ít nhất bao lâu thì kim phút chập với kim giờ?
\end{baitoan}

\begin{baitoan}[\cite{TLCT_THCS_Toan_6_so_hoc}, 13.3., p. 80]
	Trên quãng đường AB, 2 ôtô khởi hành cùng 1 lúc, xe thứ nhất đi từ A đến B hết {\rm2 h}, xe thứ 2 đi từ B đến A hết {\rm3 h}. Đến chỗ gặp nhau, quãng đường xe thứ nhất đã đi nhiều hơn quãng đường xe thứ 2 đã đi là {\rm30 km}. Tính quãng đường AB.
\end{baitoan}

\begin{baitoan}[\cite{TLCT_THCS_Toan_6_so_hoc}, 13.4., p. 81]
	Trên quãng đường AB dài {\rm300 km}, người thứ nhất đi từ A đến B, người thứ 2 đi từ B đến A. Họ khởi hành cùng 1 lúc thì sẽ gặp nhau sau {\rm5 h}. Nhưng người thứ 2 có việc bận, đã khởi hành sau người thứ nhất {\rm40 ph}, do đó sau {\rm4 h 42 ph} mới gặp người thứ nhất. Tính vận tốc mỗi người.
\end{baitoan}

\begin{baitoan}[\cite{TLCT_THCS_Toan_6_so_hoc}, 13.5., p. 81]
	2 xe khởi hành từ A \& từ B cùng 1 lúc, đi ngược chiều \& sẽ gặp nhau sau {\rm6 h}. Vận tốc của xe con bằng $\dfrac{4}{3}$ vận tốc của xe tải. Muốn gặp nhau ở chính giữa quãng đường AB thì xe con phải đi sau xe tải bao lâu?
\end{baitoan}

\begin{baitoan}[\cite{TLCT_THCS_Toan_6_so_hoc}, 13.6., p. 81]
	1 người đi từ A đến B. Người đó tính: nếu đi với vận tốc {\rm40 km{\tt/}h} thì đến B sau giờ hẹn là {\rm20 ph}, còn nếu đi với vận tốc {\rm60 km{\tt/}h} thì đến B trước giờ hẹn là {\rm10 ph}. Tính quãng đường AB.
\end{baitoan}

\begin{baitoan}[\cite{TLCT_THCS_Toan_6_so_hoc}, 13.7., p. 81]
	3 người cùng khởi hành 1 lúc từ A để đến B, vận tốc người I, người II lần lượt là {\rm40 km{\tt/}h, 60 km{\tt/}h}. Người III đến B trước người I là {\rm18 ph} \& sau người II là {\rm12 ph}. Tính quãng đường AB \& vận tốc người III. 
\end{baitoan}

\begin{baitoan}[\cite{TLCT_THCS_Toan_6_so_hoc}, 13.8., p. 81]
	2 người cùng đi từ A về 1 phía, người I khởi hành lúc {\rm7:00}, người II khởi hành lúc {\rm7:30}. 2 người gặp nhau lúc mấy giờ, biết quãng đường người I đi trong {\rm30 ph} bằng quãng đường người II đi trong {\rm20 ph}?
\end{baitoan}

\begin{baitoan}[\cite{TLCT_THCS_Toan_6_so_hoc}, 13.9., p. 81]
	1 ôtô đi từ A đến B với vận tốc {\rm40 km{\tt/}h}. Trên đường đi từ B về A, sau khi đi $\dfrac{1}{3}$ quãng đường với vận tốc cũ, ôtô dừng lại chữa trong {\rm30 ph}. Muốn thời gian từ B về A vẫn bằng thời gian từ A đến B, ôtô phải đi tiếp với vận tốc {\rm60 km{\tt/}h}. Tính quãng đường AB.
\end{baitoan}

\begin{baitoan}[\cite{TLCT_THCS_Toan_6_so_hoc}, 13.10., p. 81]
	Thành đi xe đạp từ A đến B. Sau khi đi {\rm10 km} trong {\rm40 ph}, Thành tính: nếu tiếp tục đi với vận tốc như vậy thì đến B trước giờ hẹn {\rm24 ph}. Anh đã giảm vận tốc đi {\rm3 km{\tt/}h} mà vẫn đến B trước giờ hẹn {\rm10 ph}. Tính quãng đường AB.
\end{baitoan}

\begin{baitoan}[\cite{TLCT_THCS_Toan_6_so_hoc}, 13.11., p. 81]
	1 người đi từ A đến B với vận tốc {\rm40 km{\tt/}h}, rồi đi tiếp từ B đến D với vận tốc {\rm60 km{\tt/}h}. Quãng đường BD dài hơn AB là {\rm10 km}. Thời gian đi BD ít hơn đi AB là {\rm20 ph}. Tính 2 quãng đường $AB,BD$.
\end{baitoan}

\begin{baitoan}[\cite{TLCT_THCS_Toan_6_so_hoc}, 13.12., p. 82]
	1 canô xuôi khúc sông từ A đến B hết {\rm3 h} \& ngược khúc sông đó hết {\rm4.5 h}. Biết vận tốc dòng nước là {\rm3 km{\tt/}h}. Tính vận tốc xuôi, vận tốc ngược, \& chiều dài khúc sông AB.
\end{baitoan}

\begin{baitoan}[\cite{TLCT_THCS_Toan_6_so_hoc}, 13.13., p. 82]
	Thắng đi xe máy với vận tốc {\rm36 km{\tt/}h}. Anh gặp 1 xe lửa dài {\rm75 m} đi cùng chiều chạy song song bên cạnh mình trong {\rm15 s}. Tính vận tốc xe lửa với đơn vị {\rm m{\tt/}s}.
\end{baitoan}

\begin{baitoan}[\cite{TLCT_THCS_Toan_6_so_hoc}, 13.14., p. 82]
	1 người đi từ A đến B. Người đó đi $\dfrac{1}{3}$ quãng đường với vận tốc {\rm20 km{\tt/}h} rồi đi phần còn lại với vận tốc {\rm10 km{\tt/}h}. HTính vận tốc trung bình của người đó trên cả quãng đường AB.
\end{baitoan}

\begin{baitoan}[\cite{TLCT_THCS_Toan_6_so_hoc}, 13.15., p. 82]
	Lúc {\rm6:00}, 1 xe tải \& 1 xe máy cùng xuất phát từ A để đến B. Vận tốc xe tải là {\rm50 km{\tt/}h}, vận tốc xe máy là {\rm30 km{\tt/}h}. Sau đó {\rm2 h}, 1 xe con cũng đi từ A để đến B, vận tốc {\rm60 km{\tt/}h}. Đến mấy giờ thì xe con ở chính giữa xe máy \& xe tải?
\end{baitoan}

\begin{baitoan}[\cite{TLCT_THCS_Toan_6_so_hoc}, 13.16., p. 82]
	2 xe bus cùng khởi hành 1 lúc với vận tốc không đổi, xe thứ nhất đi từ A đến B, xe thứ 2 đi từ B đến A. Xe thứ nhất đến B thì quay lại ngay, xe thứ 2 đến A thì quay lại ngay. 2 xe gặp nhau lần thứ nhất tại C cách A là {\rm5 km} \& gặp nhau lần thứ 2 tại D cách B là {\rm4 km}. Tính quãng đường AB.
\end{baitoan}

%------------------------------------------------------------------------------%

\section{Productivity -- Toán Năng Suất Về Công Việc Làm Đồng Thời}

\begin{baitoan}[\cite{Tuyen_Toan_6}, VD70, p. 71]
	1 cơ sở sản xuất cần đóng gói 1 số thùng hàng. Nếu 2 người cùng làm thì sau {\rm8 h} sẽ xong. Nếu người thứ nhất làm 1 mình thì sau {\rm12 h} sẽ xong. (a) Nếu người thứ 2 làm 1 mình thì sau bao lâu sẽ xong? (b) Nếu cả 2 cùng đóng gói được tổng cộng là $216$ thùng hàng thì mỗi người đã đóng được bao nhiêu thùng hàng?
\end{baitoan}

\begin{baitoan}[\cite{Tuyen_Toan_6}, 358., p. 71]
	2 người cùng làm 1 công việc sau {\rm2 h 24 ph} sẽ xong. Biết người thứ nhất làm 1 mình công việc đó thì mất {\rm4 h}, hỏi người thứ 2 làm $\frac{1}{2}$ công việc đó thì bao lâu mới xong?
\end{baitoan}

\begin{baitoan}[\cite{Tuyen_Toan_6}, 359., p. 71]
	2 vòi nước cùng chảy vào 1 bể sau {\rm40 ph} sẽ đầy. Biết năng suất chảy của vòi I gấp đôi vòi II. (a) Nếu mỗi chảy 1 mình thì bao lâu mới đầy bể? (b) Tính dung tích của bể biết mỗi phút vòi I chảy nhanh hơn vòi II {\rm20 l}.
\end{baitoan}

\begin{baitoan}[\cite{Tuyen_Toan_6}, 360., p. 71]
	2 tốp thợ cùng làm 1 công việc thì sau {\rm12 h} sẽ xong. Họ cùng làm được {\rm4 h} thì tốp thứ nhất nghỉ, tốp thứ 2 làm nốt trong {\rm20 h} nữa mới xong. Nếu làm riêng thì mỗi tốp phải mất bao nhiêu giờ mới làm xong công việc đó?
\end{baitoan}

\begin{baitoan}[\cite{Tuyen_Toan_6}, 361., p. 71]
	Để bơm cạn nước trong 1 hồ, người ta chuẩn bị 3 máy bơm. Máy bơm A có thể bơm cạn nước hồ trong {\rm8 h}. Máy bơm B chỉ cần {\rm10 h} là có thể bơm hết nước trong 2 hồ như thế. Máy bơm C cũng chỉ cần {\rm16 h} là có thể bơm hết nước trong 3 hồ như vậy. Nếu cả 3 máy bơm cùng bơm nước trong {\rm2 h} thì hồ có cạn hết nước không?
\end{baitoan}

\begin{baitoan}[\cite{Tuyen_Toan_6}, 362., p. 72]
	Có 3 máy đào đất công suất khác nhau. Máy A có thể hoàn thành công việc trong {\rm12 h}, máy B sau {\rm18 h}. Công suất máy C bằng trung bình cộng công suất của 2 máy $A,B$. (a) Chứng minh thời gian để máy C làm 1 mình xong công việc nhỏ hơn trung bình cộng 2 thời gian của máy A \& máy B. (b) Cả 3 máy cùng làm trong {\rm5 h} thì có xong công việc đó không?
\end{baitoan}

\begin{baitoan}[\cite{Tuyen_Toan_6}, 363., p. 72]
	3 máy bơm cùng bơm nước vào 1 bể bơi. Nếu máy A \& máy B cùng chạy thì sau {\rm1 h 20 ph} sẽ đầy. Nếu máy bơm B \& máy bơm C cùng chạy thì sau {\rm1 h 30 ph} sẽ đầy. Nếu máy bơm C \& máy bơm A cùng chạy thì sau {\rm2 h 24 ph} sẽ đầy. Mỗi máy bơm chạy 1 mình thì bao lâu mới đầy bể?
\end{baitoan}

\begin{baitoan}[\cite{Tuyen_Toan_6}, 364., p. 72]
	1 vòi nước chảy vào 1 bể trong {\rm60 ph} thì đầy. 1 vòi thứ 2 lấy nước ra dùng trong {\rm90 ph} thì cạn hết 1 bể nước đầy. Sau khi rửa bể \& tháo hết nứo ra, mở cả 2 vòi cùng 1 lúc, sau {\rm45 ph} thì trong bể có {\rm1000 l} nước. Tính dung tích của bể.
\end{baitoan}

\begin{baitoan}[\cite{Tuyen_Toan_6}, 365., p. 72]
	1 xe tải khởi hành từ A lúc {\rm7:00} \& đến B lúc {\rm12:00}. 1 xe con khởi hành từ B lúc {\rm7:30} \& đến A lúc {\rm11:30}. (a) 2 xe gặp nhau lúc mấy giờ? (b) Biết vận tốc xe con hơn vận tốc xe tải là {\rm10 km{\tt/}h}, tính quãng đường AB.
\end{baitoan}

%------------------------------------------------------------------------------%

\section{Base-$d$ System -- Hệ Ghi Số Với Cơ Số Bất Kỳ}

\begin{baitoan}[{\sf Program}: Interchange between base-$d$ systems]
	Viết chương trình {\sf Pascal, Python, C{\tt/}C++} để chuyển đổi giữa 2 hệ cơ số khác nhau.
\end{baitoan}

\begin{baitoan}[\cite{TLCT_THCS_Toan_6_so_hoc}, VD14.1, p. 83]
	Đổi số $1304_{(5)}$ thành số viết trong hệ thập phân.
\end{baitoan}

\begin{baitoan}[\cite{TLCT_THCS_Toan_6_so_hoc}, VD14.2, p. 83]
	Đổi số $204$ trong hệ thập phân thành số viết trong hệ cơ số $5$.
\end{baitoan}

\begin{baitoan}[\cite{TLCT_THCS_Toan_6_so_hoc}, VD14.3, p. 84]
	1 đội quân được tổ chức theo nguyên tắc ``tam tam chế'', i.e., cứ 3 lính thì lập thành 1 tổ, cứ 3 tổ thì lập thành 1 tiểu đội, cứ 3 tiểu đội thì lập thành 1 trung đội, $\ldots$ Các cấp từ nhỏ đến lớn là tổ, tiểu đội, trung đội, đại đội, tiểu đoàn, $\ldots$ Có $422$ lính thì lập được thành các cấp nào?
\end{baitoan}

\begin{baitoan}[\cite{TLCT_THCS_Toan_6_so_hoc}, VD14.4, p. 85]
	Tính trong hệ cơ số $4$: (a) $321 + 103$. (b) $123 - 31$.
\end{baitoan}

\begin{baitoan}[\cite{TLCT_THCS_Toan_6_so_hoc}, VD14.5, p. 85]
	Đổi số $1101_{(2)}$ thành số viết trong hệ thập phân.
\end{baitoan}

\begin{baitoan}[\cite{TLCT_THCS_Toan_6_so_hoc}, VD14.6, p. 85]
	Đổi số $43$ thành số viết trong hệ nhị phân.
\end{baitoan}

\begin{baitoan}[\cite{TLCT_THCS_Toan_6_so_hoc}, VD14.7, p. 86]
	Mai có $31$ chiếc tem đựng trong $5$ phong bì. Muốn lấy ra 1 số tem bất kỳ từ $1$ đến $30$, Mai chỉ cần lấy ra 1 số phong bì là có đúng số tem định lấy. Tính số chiếc tem trong mỗi phong bì.
\end{baitoan}

\begin{baitoan}[\cite{TLCT_THCS_Toan_6_so_hoc}, VD14.8, p. 86]
	Tính trong hệ nhị phân: (a) $1101 + 110$. (b) $1101 - 110$. (c) $1101\cdot11$.
\end{baitoan}

\begin{baitoan}[\cite{TLCT_THCS_Toan_6_so_hoc}, 14.1., p. 87]
	Có bao nhiêu đơn vị trong số lớn nhất có 1 chữ số được viết trong: (a) Hệ thập phân? (b) Hệ cơ số $7$?
\end{baitoan}

\begin{baitoan}[\cite{TLCT_THCS_Toan_6_so_hoc}, 14.2., p. 87]
	Có bao nhiêu đơn vị trong số lớn nhất có 2 chữ số được viết trong: (a) Hệ thập phân? (b) Hệ cơ số $6$?
\end{baitoan}

\begin{baitoan}[\cite{TLCT_THCS_Toan_6_so_hoc}, 14.3., p. 87]
	Nếu viết thêm vào bên phải số sau 1 chữ số $0$ thì nó tăng gấp mấy lần? (a) $41$. (b) $35_{(8)}$.
\end{baitoan}

\begin{baitoan}[\cite{TLCT_THCS_Toan_6_so_hoc}, 14.4., p. 87]
	Tính trong hệ cơ số $6$: (a) $132 + 241$. (b) $553 - 315$.
\end{baitoan}

\begin{baitoan}[\cite{TLCT_THCS_Toan_6_so_hoc}, 14.5., p. 87]
	Tính tỏng hệ nhị phân: (a) $1001 + 11$. (b) $10010 - 1011$. (c) $101\cdot101$.
\end{baitoan}

\begin{baitoan}[\cite{TLCT_THCS_Toan_6_so_hoc}, 14.6., p. 88]
	Trong hệ cơ số nào, có: (a) $3 + 4 = 10$? (b) $3 + 2 = 11$? (c) $2\cdot3 = 10$?
\end{baitoan}

\begin{baitoan}[\cite{TLCT_THCS_Toan_6_so_hoc}, 14.7., p. 88]
	Tìm $x$ thỏa: (a) $43_{(x)} = 31$. (b) $23_(4) = x_{(3)}$. (c) $145_{(x)} = 145\cdot2$.
\end{baitoan}

\begin{baitoan}[\cite{TLCT_THCS_Toan_6_so_hoc}, 14.8., p. 88]
	Để cân các vật có khối lượng là 1 số tự nhiên từ {\rm1 kg--29 kg}, có thể dùng 1 cân đĩa có 2 đĩa cân với $5$ quả cân, các quả cân chỉ để ở 1 đĩa cân. $5$ quả cân đó có khối lượng bao nhiêu?
\end{baitoan}

\begin{baitoan}[\cite{TLCT_THCS_Toan_6_so_hoc}, 14.9., p. 88]
	Cho $A = \sum_{i=0}^{20} 2^i = 1 + 2 + 2^2 + 2^3 + \cdots + 2^{20}$. Viết $A + 1$ dưới dạng 1 lũy thừa.
\end{baitoan}

\begin{baitoan}[\cite{TLCT_THCS_Toan_6_so_hoc}, 14.11., p. 88]
	Chứng minh trong mọi hệ cơ số: (a) Số $100$ là số chính phương. (b) Số $110$ là hợp số.
\end{baitoan}

\begin{baitoan}[\cite{TLCT_THCS_Toan_6_so_hoc}, 14.12., p. 88]
	Chứng minh trong hệ cơ số $4$: $\overline{abc}_{(4)}\divby3\Leftrightarrow a + b + c\divby3$.
\end{baitoan}

\begin{baitoan}[\cite{TLCT_THCS_Toan_6_so_hoc}, 14.13., p. 88]
	Tìm số $\overline{abc}$ viết trong hệ thập phân bằng số $\overline{cba}$ viết trong hệ cơ số $9$.
\end{baitoan}

%------------------------------------------------------------------------------%

\section{Special Sequence -- Dãy Số Viết Theo Quy Luật}

\begin{baitoan}[{\sf Program}: Arithmetic sequence]
	Viết chương trình {\sf Pascal, Python, C{\tt/}C++} để in ra cấp số cộng \& tính tổng của cấp số cộng $\{a + bn\}_{n=0}^{+\infty}$ với số hạng đầu $a\in\mathbb{R}$ \& công sai $b\in\mathbb{R}$ nhập từ bàn phím.
\end{baitoan}

\begin{baitoan}[{\sf Program}: Geometric sequence]
	Viết chương trình {\sf Pascal, Python, C{\tt/}C++} để in ra cấp số nhân \& tính tổng của cấp số nhân $\{aq^n\}_{n=0}^{+\infty}$ với số hạng đầu $a\in\mathbb{R}$ \& công bội $q\in\mathbb{R}$ nhập từ bàn phím.
\end{baitoan}

\begin{baitoan}[\cite{Tuyen_Toan_6}, VD62, p. 60]
	(a) Chứng minh $\dfrac{m}{a(a + m)} = \dfrac{1}{a} - \dfrac{1}{a + m}$, $\forall a,m\in\mathbb{R},a(a + m)\ne0$. (b) Tính $A = \dfrac{3}{4\cdot7} + \dfrac{3}{7\cdot10} + \dfrac{3}{10\cdot13} + \cdots + \dfrac{3}{73\cdot76}$. (c) Mở rộng.
\end{baitoan}

\begin{baitoan}[\cite{Tuyen_Toan_6}, VD63, p. 61]
	Tính $A = \dfrac{1}{10} + \dfrac{1}{15} + \dfrac{1}{21} + \cdots + \dfrac{1}{120}$.
\end{baitoan}

\begin{baitoan}[\cite{Tuyen_Toan_6}, VD64, p. 61]
	Chứng minh $A = \sum_{i=1}^{20} \dfrac{1}{2^i} = \dfrac{1}{2} + \dfrac{1}{2^2} + \cdots + \dfrac{1}{2^{20}} < 1$.
\end{baitoan}

\begin{baitoan}[\cite{Tuyen_Toan_6}, 312., p. 62]
	Tính: (a) $A = \dfrac{2}{3\cdot5} + \dfrac{2}{5\cdot7} + \dfrac{2}{7\cdot9} + \cdots + \dfrac{2}{37\cdot39}$. (b) $B = \dfrac{4}{3\cdot7} + \dfrac{4}{7\cdot11} + \dfrac{4}{11\cdot15} + \cdots + \dfrac{4}{107\cdot111}$. (c) $C = \dfrac{2}{15} + \dfrac{2}{35} + \dfrac{2}{63} + \cdots + \dfrac{2}{399}$.
\end{baitoan}

\begin{baitoan}[\cite{Tuyen_Toan_6}, 313., p. 62]
	Tính: (a) $A = \dfrac{7}{10\cdot11} + \dfrac{7}{11\cdot12} + \dfrac{7}{12\cdot13} + \cdots + \dfrac{7}{69\cdot70}$. (b) $B = \dfrac{6}{15\cdot18} + \dfrac{6}{18\cdot21} + \dfrac{6}{21\cdot24} + \cdots + \dfrac{6}{87\cdot90}$. (c) $C = \dfrac{3^2}{8\cdot11} + \dfrac{3^2}{11\cdot14} + \dfrac{3^2}{4\cdot17} + \cdots + \dfrac{3^2}{197\cdot200}$.
\end{baitoan}

\begin{baitoan}[\cite{Tuyen_Toan_6}, 314., p. 62]
	Tính: (a) $A = \dfrac{1}{25\cdot27} + \dfrac{1}{27\cdot29} + \dfrac{1}{29\cdot31} + \cdots + \dfrac{1}{73\cdot75}$. (b) $B = \dfrac{15}{90\cdot94} + \dfrac{15}{94\cdot98} + \dfrac{15}{98\cdot102} + \cdots + \dfrac{15}{146\cdot150}$. (c) $C = \dfrac{10}{56} + \dfrac{10}{140} + \dfrac{10}{260} + \cdots + \dfrac{10}{1400}$.
\end{baitoan}

\begin{baitoan}[\cite{Tuyen_Toan_6}, 315., p. 62]
	Tính $A = \sum_{i=1}^{10} \dfrac{i}{2^i} = \dfrac{1}{2} + \dfrac{2}{4} + \dfrac{3}{8} + \dfrac{4}{16} + \cdots + \dfrac{10}{2^{10}}$.
\end{baitoan}

\begin{baitoan}[\cite{Tuyen_Toan_6}, 316., p. 62]
	Chứng minh $A = \dfrac{1}{5} + \dfrac{1}{13} + \dfrac{1}{25} + \cdots + \dfrac{1}{19^2 + 20^2} < \dfrac{17}{40}$.
\end{baitoan}

\begin{baitoan}[\cite{Tuyen_Toan_6}, 317., p. 62]
	Chứng minh $A = \sum_{i=4}^{2030} \dfrac{1}{i\cdot i!} = \dfrac{1}{4\cdot4!} + \dfrac{1}{5\cdot5!} + \cdots + \dfrac{1}{2030\cdot2030!} < \dfrac{1}{8}$.
\end{baitoan}

\begin{baitoan}[\cite{Tuyen_Toan_6}, 318., p. 63]
	Chứng minh $A = \sum_{i=3}^n \dfrac{1}{(i - 2)\cdot i!} = \dfrac{1}{1!3} + \dfrac{1}{2!4} + \dfrac{1}{3!5} + \cdots + \dfrac{1}{(n - 2)!n} < \dfrac{1}{2}$.
\end{baitoan}

\begin{baitoan}[\cite{Tuyen_Toan_6}, 319., p. 63]
	Chứng minh $\dfrac{1}{1\cdot6} + \dfrac{1}{6\cdot11} + \dfrac{1}{11\cdot16} + \cdots + \dfrac{1}{(5n + 1)(5n + 6)} = \dfrac{n + 1}{5n + 6}$, $\forall n\in\mathbb{N}$.
\end{baitoan}

\begin{baitoan}[\cite{Tuyen_Toan_6}, 320., p. 63]
	Tìm $x\in\mathbb{N}$ thỏa $x - \dfrac{20}{11\cdot13} - \dfrac{20}{13\cdot15} - \dfrac{20}{15\cdot17} - \dfrac{20}{53\cdot55} = \dfrac{3}{11}$.
\end{baitoan}

\begin{baitoan}[\cite{Tuyen_Toan_6}, 321., p. 63]
	Tìm $x\in\mathbb{N}$ thỏa $\dfrac{1}{21} + \dfrac{1}{28} + \dfrac{1}{36} + \cdots + \dfrac{2}{x(x + 1)} = \dfrac{2}{9}$.
\end{baitoan}

\begin{baitoan}[\cite{Tuyen_Toan_6}, 322., p. 63]
	Chứng minh: (a) $\dfrac{2m}{a(a + m)(a + 2m)} = \dfrac{1}{a(a + m)} - \dfrac{1}{{a + m}(a + 2m)}$. (b) $A = \sum_{i=1}^{18} \dfrac{1}{i(i + 1){i + 2}} = \dfrac{1}{1\cdot2\cdot3} + \dfrac{1}{2\cdot3\cdot4} + \dfrac{1}{3\cdot4\cdot5} + \cdots + \dfrac{1}{18\cdot19\cdot20} < \dfrac{1}{4}$. (c) $B = \dfrac{36}{1\cdot3\cdot5} + \dfrac{36}{3\cdot5\cdot7} + \dfrac{36}{5\cdot7\cdot9} + \cdots + \dfrac{36}{25\cdot27\cdot29} < 3$.
\end{baitoan}

\begin{baitoan}[\cite{Tuyen_Toan_6}, 323., p. 63]
	Chứng minh: (a) $A = \sum_{i=2}^n \dfrac{1}{i^2} = \dfrac{1}{2^2} + \dfrac{1}{3^2} + \dfrac{1}{4^2} + \cdots + \dfrac{1}{n^2} < 1$, $\forall n\in\mathbb{N},n\ge2$. (b) $B = \sum_{i=2}^n \dfrac{1}{(2i)^2} = \dfrac{1}{4^2} + \dfrac{1}{6^2} + \dfrac{1}{8^2} + \cdots + \dfrac{1}{(2n)^2} < \dfrac{1}{4}$, $\forall n\in\mathbb{N},n\ge2$. (c) $C = \sum_{i=3}^n \dfrac{2!}{i!} = \dfrac{2!}{3!} + \dfrac{2!}{4!} + \dfrac{2!}{5!} + \cdots + \dfrac{2!}{n!} < 1$, $\forall n\in\mathbb{N},n\ge3$.
\end{baitoan}

\begin{baitoan}[\cite{Tuyen_Toan_6}, 324., p. 63]
	Chứng minh $A = \sum_{i=1}^{99} \dfrac{i}{5^{i+1}} = \dfrac{1}{5^2} + \dfrac{2}{5^3} + \dfrac{3}{5^4} + \cdots + \dfrac{99}{5^{100}} < \dfrac{1}{16}$.
\end{baitoan}

\begin{baitoan}[\cite{Tuyen_Toan_6}, 325., p. 63]
	Chứng minh $\dfrac{1}{26} + \dfrac{1}{27} + \dfrac{1}{28} + \cdots + \dfrac{1}{50} = 1 - \dfrac{1}{2} + \dfrac{1}{3} - \dfrac{1}{4} + \cdots + \dfrac{1}{49} - \dfrac{1}{50}$.
\end{baitoan}

\begin{baitoan}[\cite{Binh_Toan_6_tap_2}, VD32, p. 36]
	Tính nhanh: $A = \sum_{i=1}^8 \dfrac{1}{3^i} = \dfrac{1}{3} + \dfrac{1}{3^2} + \cdots + \dfrac{1}{3^8}$.
\end{baitoan}

\begin{baitoan}[\cite{Binh_Toan_6_tap_2}, VD33, p. 36]
	Tính tổng $100$ số hạng đầu tiên của dãy: (a) $\dfrac{1}{1\cdot2},\dfrac{1}{2\cdot3},\dfrac{1}{3\cdot4},\dfrac{1}{4\cdot5},\ldots$ (b) $\dfrac{1}{6},\dfrac{1}{66},\dfrac{1}{176},\dfrac{1}{336},\ldots$
\end{baitoan}

\begin{baitoan}[\cite{Binh_Toan_6_tap_2}, VD34, p. 37]
	Tìm tích $xy$ biết $\dfrac{1}{4\cdot15} + \dfrac{1}{5\cdot18} + \dfrac{1}{6\cdot21} + \cdots + \dfrac{1}{xy} = \dfrac{3}{40}$.
\end{baitoan}

\begin{baitoan}[\cite{Binh_Toan_6_tap_2}, VD35, p. 37]
	Tính tổng $A = \sum_{i=1}^{37} \dfrac{1}{i(i + 1)(i + 2)} = \dfrac{1}{1\cdot2\cdot3} + \dfrac{1}{2\cdot3\cdot4} + \dfrac{1}{3\cdot4\cdot5} + \cdots + \dfrac{1}{37\cdot38\cdot39}$.
\end{baitoan}

\begin{baitoan}[\cite{Binh_Toan_6_tap_2}, VD36, p. 38]
	Tính: (a) $A = \dfrac{1 + \dfrac{1}{3} + \dfrac{1}{5} + \cdots + \dfrac{1}{97} + \dfrac{1}{99}}{\dfrac{1}{1\cdot99} + \dfrac{1}{3\cdot97} + \dfrac{1}{5\cdot9} + \cdots + \dfrac{1}{97\cdot3} + \dfrac{1}{99\cdot1}}$. (b) $B = \dfrac{\dfrac{1}{2} + \dfrac{1}{3} + \dfrac{1}{4} + \cdots + \dfrac{1}{100}}{\dfrac{99}{1} + \dfrac{98}{2} + \dfrac{97}{3} + \cdots + \frac{1}{99}}$.
\end{baitoan}

\begin{baitoan}[\cite{Binh_Toan_6_tap_2}, VD37, p. 38]
	Tìm tích của $98$ số đầu tiên của dãy $1\dfrac{1}{3},1\dfrac{1}{8},1\dfrac{1}{15},1\dfrac{1}{24},1\dfrac{1}{35},\ldots$
\end{baitoan}

\begin{baitoan}[\cite{Binh_Toan_6_tap_2}, VD38, p. 39]
	(a) Tính $A = \dfrac{1 + (1 + 2) + (1 + 2 + 3) + \cdots + (1 + 2 + \cdots + 98)}{1\cdot98 + 2\cdot97 + 3\cdot96 + \cdots + 98\cdot1}$.\\(b) Chứng minh $B = \dfrac{1\cdot98 + 2\cdot97 + 3\cdot96 + \cdots + 98\cdot1}{1\cdot2 + 2\dot3 + 3\cdot4 + \cdots + 98\cdot99} = \dfrac{1}{2}$.
\end{baitoan}

\begin{baitoan}[\cite{Binh_Toan_6_tap_2}, VD39, p. 39]
	Cho $A = \sum_{i=1}^{100} \dfrac{1}{i} = 1 + \dfrac{1}{2} + \dfrac{1}{3} + \cdots + \dfrac{1}{100}$. Chứng minh $A\notin\mathbb{Z}$.
\end{baitoan}

\begin{baitoan}[\cite{Binh_Toan_6_tap_2}, VD40, p. 40]
	Tổng $\dfrac{1}{50} + \dfrac{1}{51} + \cdots + \dfrac{1}{99} = \dfrac{a}{b}$. Chứng minh $a\divby149$.
\end{baitoan}

\begin{baitoan}[\cite{Binh_Toan_6_tap_2}, VD41, p. 41]
	(a) Lập 1 dãy số gồm 9 số hạng, số hạng thứ nhất bằng $0$, số hạng thứ 2 bằng $3$, từ số hạng thứ 3 trở đi thì số hạng sau gấp đôi số hạng liền trước. (b) Cộng $4$ vào các số hạng của dãy trên. (c) Chia các số hạng của dãy số ở (b) cho $10$.
\end{baitoan}

\begin{baitoan}[\cite{Binh_Toan_6_tap_2}, VD42, p. 41]
	Cjp $A = \sum_{i=101}^{200} = \dfrac{1}{i} = \dfrac{1}{101} + \dfrac{1}{102} + \cdots + \dfrac{1}{200}$. Chứng minh: (a) $A > \dfrac{7}{12}$. (b) $A > \dfrac{5}{8}$.
\end{baitoan}

\begin{baitoan}[\cite{Binh_Toan_6_tap_2}, VD43, p. 42]
	(a) Cho dãy số tự nhiên liên tiếp $a,a + 1,\ldots,b - 1,b$, trong đó $b > a + 1$, dãy này gồm 1 số chẵn số hạng. Ghép các số này thành từng cặp 2 số ở 2 đầu \& 2 số cách đều 2 đầu. Chứng minh 2 số thuộc cặp ngoai fcungf có tích nhỏ nhất, 2 số thuộc cặp trong cùng có tích lớn nhất. (b) Áp dụng nhận xét này để chứng minh $\dfrac{5}{8} < \sum_{i=101}^{200} = \dfrac{1}{i} = \dfrac{1}{101} + \dfrac{1}{102} + \cdots + \dfrac{1}{200} < \dfrac{3}{4}$.
\end{baitoan}

\begin{baitoan}[\cite{Binh_Toan_6_tap_2}, VD44, p. 43]
	Chứng minh $A = \sum_{i=2}^{100} \dfrac{1}{i^2} = \dfrac{1}{2^2} + \dfrac{1}{3^2} + \dfrac{1}{4^2} + \cdots + \dfrac{1}{100^2} < \dfrac{3}{4}$.
\end{baitoan}

\begin{baitoan}[\cite{Binh_Toan_6_tap_2}, VD45, p. 43]
	Cho $A = \dfrac{199!!}{200!!} = \dfrac{1}{2}\cdot\dfrac{3}{4}\cdot\dfrac{5}{6}\cdots\dfrac{199}{200}$. Chứng minh $A^2 < \dfrac{1}{201}$.
\end{baitoan}

\begin{baitoan}[\cite{Binh_Toan_6_tap_2}, VD46, pp. 43--44]
	Cho $A = \sum_{i=1}^{2^{100} - 1} \dfrac{1}{i} = 1 + \dfrac{1}{2} + \dfrac{1}{3} + \cdots + \dfrac{1}{2^{100} - 1}$.Chứng minh $50 < A < 100$.
\end{baitoan}

\begin{baitoan}[\cite{Binh_Toan_6_tap_2}, VD47, p. 44]
	Chứng minh luôn tồn tại $n\in\mathbb{N}$ để $\sum_{i=1}^n \dfrac{1}{i} = 1 + \dfrac{1}{2} + \dfrac{1}{3} + \cdots + \dfrac{1}{n} > 1000$.
\end{baitoan}

\begin{baitoan}[\cite{Binh_Toan_6_tap_2}, 99., p. 44]
	Tính nhanh $A = \sum_{i=1}^{10} \dfrac{1}{2^i} = \dfrac{1}{2} + \dfrac{1}{2^2} + \dfrac{1}{2^3} + \cdots + \dfrac{1}{2^10}$.
\end{baitoan}

\begin{baitoan}[\cite{Binh_Toan_6_tap_2}, 100., pp. 44--45]
	Viết tất cả các phân số dương thành dãy $\dfrac{1}{1},\dfrac{2}{1},\dfrac{1}{2},\dfrac{3}{1},\dfrac{2}{2},\dfrac{1}{3},\dfrac{4}{1},\dfrac{3}{2},\dfrac{2}{3},\dfrac{1}{4},\ldots$ (a) Nêu quy luật của dãy \& viết tiếp $5$ phân số nữa theo quy luật ấy. (b) Phân số $\dfrac{50}{31}$ là số hạng thứ mấy của dãy?
\end{baitoan}

\begin{baitoan}[\cite{Binh_Toan_6_tap_2}, 101., p. 45]
	Tính: (a) $A = \dfrac{1}{1\cdot3} + \dfrac{1}{3\cdot5} + \dfrac{1}{5\cdot7} + \cdots + \dfrac{1}{17\cdot19}$. (b) $B = \dfrac{1}{1\cdot4} + \dfrac{1}{4\cdot7} + \dfrac{1}{7\cdot10} + \cdots + \dfrac{1}{16\cdot19}$.
\end{baitoan}

\begin{baitoan}[\cite{Binh_Toan_6_tap_2}, 102., p. 45]
	Tìm $x\in\mathbb{N}$ thỏa: (a) $\dfrac{1}{5\cdot8} + \dfrac{1}{8\cdot11} + \dfrac{1}{11\cdot14} + \cdots + \dfrac{1}{x(x + 3)} = \dfrac{101}{1540}$. (b) $1 + \dfrac{1}{3} + \dfrac{1}{6} + \dfrac{1}{10} + \cdots + \dfrac{1}{x(x + 1):2} = 1\dfrac{1991}{1993}$.
\end{baitoan}

\begin{baitoan}[\cite{Binh_Toan_6_tap_2}, 103., p. 45]
	Tính $A = \dfrac{1}{1} + \dfrac{1}{1 + 2} + \dfrac{1}{1 + 2 + 3} + \cdots + \dfrac{1}{1 + 2 + \cdots + 20}$.
\end{baitoan}

\begin{baitoan}[\cite{Binh_Toan_6_tap_2}, 104., p. 45]
	Tính $A = \dfrac{1}{4\cdot15} + \dfrac{1}{5\cdot18} + \dfrac{1}{6\cdot21} + \cdots + \dfrac{1}{39\cdot120}$.
\end{baitoan}

\begin{baitoan}[\cite{Binh_Toan_6_tap_2}, 105., p. 45]
	Tìm số hạng cuối thứ $30$ của $A = \dfrac{4}{6\cdot10} + \dfrac{5}{10\cdot15} + \dfrac{6}{15\cdot21} + \cdots$ rồi tính A.
\end{baitoan}

\begin{baitoan}[\cite{Binh_Toan_6_tap_2}, 106., p. 45]
	Chứng minh tổng của $100$ số hạng đầu của dãy $\dfrac{1}{5},\dfrac{1}{45},\dfrac{1}{117},\dfrac{1}{221},\dfrac{1}{357},\cdots$ nhỏ hơn $\dfrac{1}{4}$.
\end{baitoan}

\begin{baitoan}[\cite{Binh_Toan_6_tap_2}, 107., p. 45]
	Tính $A = \left(\dfrac{1}{2} + \dfrac{1}{3} + \dfrac{1}{4} + \cdots + \dfrac{1}{10}\right) + \left(\dfrac{2}{3} + \dfrac{2}{4} + \cdots + \dfrac{2}{10}\right) + \left(\dfrac{3}{4} + \dfrac{3}{5} + \cdots + \dfrac{3}{10}\right) + \cdots + \left(\dfrac{8}{9} + \dfrac{8}{10}\right) + \dfrac{9}{10}$.
\end{baitoan}

\begin{baitoan}[\cite{Binh_Toan_6_tap_2}, 108., p. 46]
	Tính $\dfrac{A}{B}$ với $A = \dfrac{1}{1\cdot300} + \dfrac{1}{2\cdot301} + \dfrac{1}{3\cdot302} + \cdots + \dfrac{1}{101\cdot400},B = \dfrac{1}{1\cdot102} + \dfrac{1}{2\cdot103} + \dfrac{1}{3\cdot104} + \cdots + \dfrac{1}{299\cdot400}$.
\end{baitoan}

\begin{baitoan}[\cite{Binh_Toan_6_tap_2}, 109., p. 46]
	Chứng minh $100 - \left(1 + \dfrac{1}{2} + \cdots + \dfrac{1}{100}\right) = \dfrac{1}{2} + \dfrac{2}{3} + \dfrac{3}{4} + \cdots + \dfrac{99}{100}$.
\end{baitoan}

\begin{baitoan}[\cite{Binh_Toan_6_tap_2}, 110., p. 46]
	Tính $\dfrac{A}{B}$ với $A = \dfrac{1}{2} + \dfrac{1}{3} + \dfrac{1}{4} + \cdots + \dfrac{1}{200},B = \dfrac{1}{199} + \dfrac{2}{198} + \dfrac{3}{197} + \cdots + \dfrac{198}{2} + \dfrac{199}{1}$.
\end{baitoan}

\begin{baitoan}[\cite{Binh_Toan_6_tap_2}, 111., p. 46]
	Chứng minh $\left(1 + \dfrac{1}{3} + \dfrac{1}{5} + \cdots + \dfrac{1}{99}\right) - \left(\dfrac{1}{2} + \dfrac{1}{4} + \dfrac{1}{6} + \cdots + \dfrac{1}{100}\right) = \dfrac{1}{51} + \dfrac{1}{52} + \dfrac{1}{53} + \cdots + \dfrac{1}{100}$. (b) Tính $A = \left(\dfrac{1}{51} + \dfrac{1}{52} + \dfrac{1}{53} + \cdots + \dfrac{1}{100}\right):\left(\dfrac{1}{1\cdot2} + \dfrac{1}{3\cdot4} + \dfrac{1}{5\cdot6} + \cdots + \dfrac{1}{99\cdot100}\right)$.
\end{baitoan}

\begin{baitoan}[\cite{Binh_Toan_6_tap_2}, 112b., p. 46]
	Tính: (a) $A_{27} = \sum_{i=1}^{27} \dfrac{1}{i(i + 1)(i + 2)(i + 3)} = \dfrac{1}{1\cdot2\cdot3\cdot4} + \dfrac{1}{1\cdot2\cdot3\cdot4\cdot5} + \dfrac{1}{3\cdot4\cdot5\cdot6} + \cdots + \dfrac{1}{27\cdot28\cdot29\cdot30}$. (b) $A_n = \sum_{i=1}^n \dfrac{1}{i(i + 1)(i + 2)(i + 3)} = \dfrac{1}{1\cdot2\cdot3\cdot4} + \dfrac{1}{1\cdot2\cdot3\cdot4\cdot5} + \dfrac{1}{3\cdot4\cdot5\cdot6} + \cdots + \dfrac{1}{n(n + 1)(n + 2)(n + 3)}$.
\end{baitoan}

\begin{baitoan}[\cite{Binh_Toan_6_tap_2}, 113., p. 46]
	Tính: (a) $A = \dfrac{8}{9}\cdot\dfrac{15}{16}\cdot\dfrac{24}{25}\cdots\dfrac{2499}{2500}$. (b) $B = \left(1 - \dfrac{1}{3}\right)\left(1 - \dfrac{1}{6}\right)\left(1 - \dfrac{1}{10}\right)\left(1 - \dfrac{1}{15}\right)\cdots\left(1 - \dfrac{1}{780}\right)$.
\end{baitoan}

\begin{baitoan}[\cite{Binh_Toan_6_tap_2}, 114., p. 46]
	Chứng minh $A = \dfrac{1}{5} + \dfrac{1}{7} + \dfrac{1}{9} + \cdots + \dfrac{1}{101}\notin\mathbb{N}$.
\end{baitoan}

\begin{baitoan}[\cite{Binh_Toan_6_tap_2}, 115., pp. 46--47]
	(a) Chứng minh $A = \left(\dfrac{1}{1} + \dfrac{1}{2} + \dfrac{1}{3} + \cdots + \dfrac{1}{98}\right)\cdot2\cdot3\cdots98\divby99$. (b) Cho $B = \dfrac{1}{1} + \dfrac{1}{2} + \dfrac{1}{3} + \cdots + \dfrac{1}{96} = \dfrac{a}{b}$. Chứng minh $a\divby97$.
\end{baitoan}

\begin{baitoan}[\cite{Binh_Toan_6_tap_2}, 117., p. 47]
	Chứng minh $\dfrac{1}{6} < \sum_{i=5}^{100} \dfrac{1}{i^2} = \dfrac{1}{5^2} + \dfrac{1}{6^2} + \cdots + \dfrac{1}{100^2} < \dfrac{1}{4}$.
\end{baitoan}

\begin{baitoan}[\cite{Binh_Toan_6_tap_2}, 119., p. 47]
	Chứng minh $\dfrac{4}{3} < \sum_{i=11}^{70} \dfrac{1}{i} = \dfrac{1}{11} + \dfrac{1}{12} + \dfrac{1}{13} + \cdots + \dfrac{1}{70} < 2.5$.
\end{baitoan}

\begin{baitoan}[\cite{Binh_Toan_6_tap_2}, 120., p. 47]
	Chứng minh $0.2 < \dfrac{1}{2} - \dfrac{1}{3} + \dfrac{1}{4} - \dfrac{1}{5} + \cdots + \dfrac{1}{98} - \dfrac{1}{99} < 0.4$.
\end{baitoan}

\begin{baitoan}[\cite{Binh_Toan_6_tap_2}, 121., p. 47]
	Chứng minh $A > B$ với $A = \dfrac{1 + 5 + 5^2 + \cdots + 5^9}{1 + 5 + 5^2 + \cdots + 5^8},B = \dfrac{1 + 3 + 3^2 + \cdots + 3^9}{1 + 3 + 3^2 + \cdots + 3^8}$. 
\end{baitoan}

\begin{baitoan}[\cite{Binh_Toan_6_tap_2}, 122., p. 47]
	Cho $A = \dfrac{1}{2}\cdot\dfrac{3}{4}\cdot\dfrac{5}{6}\cdots\dfrac{99}{100}$. Chứng minh $\dfrac{1}{15} < A < \dfrac{1}{10}$.
\end{baitoan}

\begin{baitoan}[\cite{TLCT_THCS_Toan_6_so_hoc}, VD15.1, p. 90]
	Viết thêm 3 số hạng rồi tìm công thức của số hạng tổng quát của dãy số: (a) $1,4,7,10,\ldots$ (b) $1,2,3,5,8,\ldots$ (c) $2,12,30,56,90,\ldots$ (d) $-1,-2,-6,-24,-120,\ldots$ (e) $1,6,15,28,45,\ldots$ (f) $4,18,40,70,108,\ldots$
\end{baitoan}

\begin{baitoan}[\cite{TLCT_THCS_Toan_6_so_hoc}, VD15.2, p. 91]
	Tính $A = \sum_{i=1}^n i = 1 + 2 + \cdots + n$.
\end{baitoan}

\begin{baitoan}[\cite{TLCT_THCS_Toan_6_so_hoc}, VD15.3, p. 91]
	(a) Tính bằng nhiều cách $A = 2 + 4 + 6 + \cdots + 96 + 98$. (b) Tính $B = 1 - 2 + 3 - 4 + \cdots + 2011 - 2012 + 2013$. (c) Tìm $n\in\mathbb{N}^\star$ thỏa $3 + 4 + 5 + \cdots + n = 525$. (d) Tìm $n\in\mathbb{N}^\star$ \& chữ số $a$ biết $\sum_{i=1}^n i = 1 + 2 + \cdots + n = \overline{aaa}$.
\end{baitoan}

\begin{baitoan}[\cite{TLCT_THCS_Toan_6_so_hoc}, VD15.4, p. 92]
	Cho dãy số $5,9,13,17,21,\ldots$ (a) Nhận xét dãy số trên \& tìm số hạng thứ $10$, số hạng thứ $n$. (b) 2 số $12345,2011$ có mặt trong dãy số đó không? là số thứ bao nhiêu của dãy? (c) Tìm tổng $100$ số đầu tiên của dãy số đó.
\end{baitoan}

\begin{baitoan}[\cite{TLCT_THCS_Toan_6_so_hoc}, VD15.5, p. 93]
	Cho dãy số tự nhiên chẵn liên tiếp $2,4,6,\ldots,2010,2012$. (a) Nếu viết liên tiếp các số của dãy thành số $a = 2468101214\ldots20102012$ thì a có bao nhiêu chữ số? (b) Tìm chữ số thứ $2012$ của a.
\end{baitoan}

\begin{baitoan}[\cite{TLCT_THCS_Toan_6_so_hoc}, VD15.6, p. 93]
	(a) Rút gọn tổng $A = \sum_{i=0}^{50} 2^i = 2^0 + 2^1 + 2^2 + \cdots + 2^{50}$. (b) Rút gọn tổng $B = \sum_{i=1}^{100} 5^i = 5 + 5^2 + 5^3 + \cdots + 5^{100}$. (c) Rút gọn tổng $C = \sum_{i=1}^{2010} (-1)^{i+1}3^i = 3 - 3^2 + 3^3 - 3^4 + \cdots + 3^{2007} - 3^{2008} + 3^{2009} - 3^{2010}$. (d) Chứng minh $S_n =  a + aq+ aq^2 + \cdots + aq^{n-1} = \dfrac{q^n - 1}{q - 1}a$. Áp dụng rút gọn tổng $S_{100} = 5 + 5\cdot9 + 5\cdot9^2 + 5\cdot9^3 + \cdots + 5\cdot9^{99}$.
\end{baitoan}

\begin{baitoan}[\cite{TLCT_THCS_Toan_6_so_hoc}, VD15.7, p. 94]
	Tìm $x\in\mathbb{Q}$ thỏa $(x + 2) + (4x + 4) + (7x + 6) + \cdots + (25x + 18) + (28x + 20) = 1560$.
\end{baitoan}

\begin{baitoan}[\cite{TLCT_THCS_Toan_6_so_hoc}, VD15.8, p. 94]
	(a) Tính tổng $A = 1\cdot2 + 2\cdot3 + 3\cdot4 + \cdots + 99\cdot100$. (b) Chứng minh $A_n = 1\cdot2 + 2\cdot3 + 3\cdot4 + \cdots + n(n + 1) = \dfrac{1}{3}n(n + 1)(n + 2)$, $\forall n\in\mathbb{N}^\star$. (c) Sử dụng a), tính nhanh $B = \sum_{i=1}^{100} i^2 = 1^2 + 2^2 + \cdots + 100^2$. (d) Tính nhanh $C = 1\cdot100 + 2\cdot99 + 3\cdot98 + \cdots + 98\cdot3 + 99\cdot2 + 100\cdot1$.
\end{baitoan}

\begin{baitoan}[\cite{TLCT_THCS_Toan_6_so_hoc}, VD15.9, p. 95]
	Cho $A = \sum_{i=0}^{11} 4^i = 1 + 4 + 4^2 + \cdots + 4^{11}$. Chứng minh: (a) $A\divby21$. (b) $A\divby105$. (c) $A\divby4097$.
\end{baitoan}

\begin{baitoan}[\cite{TLCT_THCS_Toan_6_so_hoc}, VD15.10, p. 95]
	Cho $a_n = \underbrace{1\ldots1}_{2n}$, $\forall n\in\mathbb{N}^\star$. Xét xem dãy $a_1,a_2,\ldots,a_{2013}$ có bao nhiêu số chia hết cho $13$?
\end{baitoan}

\begin{baitoan}[\cite{TLCT_THCS_Toan_6_so_hoc}, VD15.11, p. 96]
	Tính $A = 1 + \sum_{i=1}^{100} ii!  = 1 + 1\cdot1! + 2\cdot2! + \cdots + 100\cdot100!$.
\end{baitoan}

\begin{baitoan}[\cite{TLCT_THCS_Toan_6_so_hoc}, VD15.12, p. 96]
	(a) Tính $A = \sum_{i=1}^{50} \dfrac{1}{i(i + 1)} = \dfrac{1}{1\cdot2} + \dfrac{1}{2\cdot3} + \cdots + \dfrac{1}{49\cdot50} + \dfrac{1}{50\cdot51}$. (b) Chứng minh $A_n = \sum_{i=1}^n \dfrac{1}{i(i + 1)} = \dfrac{1}{1\cdot2} + \dfrac{1}{2\cdot3} + \cdots + \dfrac{1}{n\cdot(n + 1)} = \dfrac{n}{n + 1}$, $\forall n\in\mathbb{N}^\star$.
\end{baitoan}

\begin{baitoan}[\cite{TLCT_THCS_Toan_6_so_hoc}, VD15.13, p. 97]
	Tính tổng $50$ số hạng đầu tiên của dãy $\dfrac{1}{2\cdot4},\dfrac{1}{4\cdot6},\dfrac{1}{6\cdot8},\ldots$
\end{baitoan}

\begin{baitoan}[\cite{TLCT_THCS_Toan_6_so_hoc}, VD15.14, p. 97, (b) \cite{Binh_boi_duong_Toan_6_tap_2}, 112a., p. 46]
	Tính: (a) $A_{48} = \sum_{i=1}^{48} \dfrac{1}{i(i + 1)(i + 2)} = \dfrac{1}{1\cdot2\cdot3} + \dfrac{1}{2\cdot3\cdot4} + \cdots + \dfrac{1}{48\cdot49\cdot50}$. (b) $A_{98} = \sum_{i=1}^{98} \dfrac{1}{i(i + 1)(i + 2)} = \dfrac{1}{1\cdot2\cdot3} + \dfrac{1}{2\cdot3\cdot4} + \cdots + \dfrac{1}{98\cdot99\cdot100}$. (c) $A_n = \sum_{i=1}^n \dfrac{1}{i(i + 1)(i + 2)} = \dfrac{1}{1\cdot2\cdot3} + \dfrac{1}{2\cdot3\cdot4} + \cdots + \dfrac{1}{n(n + 1)(n + 2)}$, $\forall n\in\mathbb{N}^\star$.
\end{baitoan}

\begin{baitoan}[\cite{TLCT_THCS_Toan_6_so_hoc}, VD15.15, p. 98]
	(a) Cho $A = \sum_{i=2}^{200} \dfrac{1}{i!} = \dfrac{1}{2!} + \dfrac{1}{3!} + \cdots + \dfrac{1}{200!}$. Chứng minh $A < 1$. (b) Chứng minh $1 - \dfrac{1}{2} + \dfrac{1}{3} - \dfrac{1}{4} + \cdots + \dfrac{1}{99} - \dfrac{1}{100} = \dfrac{1}{51} + \dfrac{1}{52} + \cdots + \dfrac{1}{100}$.
\end{baitoan}

\begin{baitoan}[\cite{TLCT_THCS_Toan_6_so_hoc}, VD15.16, p. 98, \cite{Binh_Toan_6_tap_2}, 118., p. 47]
	Cho $A = \sum_{i=1}^{100} \dfrac{1}{i!} = \dfrac{1}{1!} + \dfrac{1}{2!} + \cdots + \dfrac{1}{100!}$. Chứng minh $3! - A > 4$.
\end{baitoan}

\begin{baitoan}[\cite{TLCT_THCS_Toan_6_so_hoc}, VD15.17, p. 98]
	(a) Tính $A = 1 + 2.25 + 3.5 + 4.75 + \cdots + 26$. (b) Tính $B = 1.2 + 2.3 + 3.4 + \cdots + 8.9 + 9.10 + 10.11 + 11.12 + \cdots + 18.19$.
\end{baitoan}

\begin{baitoan}[\cite{TLCT_THCS_Toan_6_so_hoc}, VD15.18, p. 99]
	Tính nhanh $A = \dfrac{18\cdot275 + 3\cdot666 + 9\cdot614\cdot2}{1 + 4 + 7 + \cdots + 58 + 61 + 62 + 62.2 + 62.3 + 62.4 - 271}$.
\end{baitoan}

\begin{baitoan}[\cite{TLCT_THCS_Toan_6_so_hoc}, VD15.19, p. 99]
	Rút gọn $A = \dfrac{1 + 15^4 + 15^8 + \cdots + 15^{96} + 15^{100}}{1 + 15^2 + 15^4 + \cdots + 15^{98} + 15^{100} + 15^{102}}$.
\end{baitoan}

\begin{baitoan}[\cite{TLCT_THCS_Toan_6_so_hoc}, VD15.20, pp. 99--100]
	Tính nhanh: (a) $A = \prod_{i=2}^{2012} 1  - \dfrac{1}{i} = \left(1 - \dfrac{1}{2}\right)\left(1 - \dfrac{1}{3}\right)\left(1 - \dfrac{1}{4}\right)\cdots\left(1 - \dfrac{1}{2012}\right)$. (b) $B = \prod_{i=1}^{99} \dfrac{i^2}{i(i + 1)} = \dfrac{1^2}{1\cdot2}\cdot\dfrac{2^2}{2\cdot3}\cdot\dfrac{3^2}{3\cdot4}\cdots\dfrac{99^2}{99\cdot100}$. (c) $C = \left(\dfrac{1}{3} - 1\right)\left(\dfrac{1}{6} - 1\right)\left(\dfrac{1}{10} - 1\right)\left(\dfrac{1}{15} - 1\right)\left(\dfrac{1}{21} - 1\right)\left(\dfrac{1}{28} - 1\right)\left(\dfrac{1}{36} - 1\right)$. (d) $D = \left(\dfrac{6}{8} + 1\right)\left(\dfrac{6}{18} + 1\right)\left(\dfrac{6}{30} + 1\right)\cdots\left(\dfrac{6}{10700} + 1\right)$.
\end{baitoan}

\begin{baitoan}[\cite{TLCT_THCS_Toan_6_so_hoc}, VD15.21, p. 100]
	Cho $A = \dfrac{100^2 + 1^2}{100\cdot1} + \dfrac{99^2 + 2^2}{99\cdot2} + \dfrac{98^2 + 3^2}{98\cdot3} + \cdots + \dfrac{52^2 + 49^2}{52\cdot49} + \dfrac{51^2 + 50^2}{51\cdot50}$, $B = \sum_{i=2}^{101} \dfrac{1}{i} = \dfrac{1}{2} + \dfrac{1}{3} + \dfrac{1}{4} + \cdots + \dfrac{1}{101}$, $C = \dfrac{1}{100\cdot1} + \dfrac{1}{99\cdot2} + \dfrac{1}{98\cdot3} + \cdots + \dfrac{1}{52\cdot49} + \dfrac{1}{51\cdot50}$. (b) Tính $\dfrac{A}{B}$. (b) Tính $B - 101C$.
\end{baitoan}

\begin{baitoan}[\cite{TLCT_THCS_Toan_6_so_hoc}, 15.1., p. 101]
	Viết thêm 2 số hạng tiếp theo của dãy số \& tìm số hạng tổng quát của dãy: (a) $2,6,12,20,\ldots$ (b) $4,10,18,28,40,\ldots$ (c) $3,6,11,18,27,\ldots$
\end{baitoan}

\begin{baitoan}[\cite{TLCT_THCS_Toan_6_so_hoc}, 15.2., p. 101]
	Cho dãy số $1,5,9,13,\ldots,37,\ldots$ (a) $37$ là số hạng thứ mấy của dãy? (b) Tìm số hạng thứ $100$. Tìm số hạng tổng quát của dãy. (c) Số $2000$, số $2013$ có thuộc dãy này không?
\end{baitoan}

\begin{baitoan}[\cite{TLCT_THCS_Toan_6_so_hoc}, 15.3., p. 101]
	Cho dãy số $2,11,29,56,92,\ldots$ (a) Tìm số hạng thứ $100$ của dãy. (b) Số $407$ là số hạng thứ bao nhiêu của dãy?
\end{baitoan}

\begin{baitoan}[\cite{TLCT_THCS_Toan_6_so_hoc}, 15.4., p. 101]
	(a) Tính tổng $A = 1 + 3 + 5 + 7 + \cdots + 2011 + 2013$. (b) Tính tổng của $100$ số tự nhiên chẵn liên tiếp bắt đầu từ số $100$.
\end{baitoan}

\begin{baitoan}[\cite{TLCT_THCS_Toan_6_so_hoc}, 15.5., p. 102]
	Tính hợp lý: (a) $A = -1 + 3 - 5 = 7 - 9 + \cdots - 2009 + 2011 - 2013$. (b) $B = 2 - 4 + 6 - 8 + \cdots + 2006 - 2008 + 2010 - 2012$. (c) $C = 1 + 2 - 3 - 4 + 5 + 6 - 7 - 8 + \cdots - 111 - 112 + 113 + 114 + 115$.
\end{baitoan}

\begin{baitoan}[\cite{TLCT_THCS_Toan_6_so_hoc}, 15.6., p. 102]
	Cho dãy số $7,10,13,16,19,\ldots$ (a) Tìm số thứ $n$. (b) Tính tổng $100$ số hạng đầu tiên của dãy.
\end{baitoan}

\begin{baitoan}[\cite{TLCT_THCS_Toan_6_so_hoc}, 15.7., p. 102]
	(a) Cho $a$ là tổng các số tự nhiên chẵn không vượt quá $200$, $b$ là tổng các số tự nhiên lẻ nhỏ hơn $200$. Tính $a - b$. (b) Tìm $n\in\mathbb{N}^\star$ thỏa $1 + 3 + 5 + \cdots + (2n - 1) = 1225$.
\end{baitoan}

\begin{baitoan}[\cite{TLCT_THCS_Toan_6_so_hoc}, 15.8., p. 102]
	Tìm $x\in\mathbb{Q}$ thỏa $(x + 1) + (x + 2) + (x + 3) + \cdots + (x + 100) = 5750$. (b) $(2x - 1) + (4x - 2) + \cdots + (400x - 200) = 5 + 10 + \cdots + 1000$.
\end{baitoan}

\begin{baitoan}[\cite{TLCT_THCS_Toan_6_so_hoc}, 15.9., p. 102]
	Viết liên tiếp các số tự nhiên thành dãy $123456789101112\ldots$ (a) Tìm tổng các chữ số của $a = 12345\ldots9899100$. (b) Chữ số $2$ ở hàng nghìn của số $2012$ là chữ số thứ bao nhiêu của dãy?
\end{baitoan}

\begin{baitoan}[\cite{TLCT_THCS_Toan_6_so_hoc}, 15.10., p. 102]
	Viết liên tiếp các số tự nhiên lẻ thành dãy $13579111315\ldots$ (a) Tìm chữ số thứ $100$ của dãy. (b) Chữ số $1$ của số $2013$ là chữ số thứ bao nhiêu của dãy?
\end{baitoan}

\begin{baitoan}[\cite{TLCT_THCS_Toan_6_so_hoc}, 15.11., p. 102]
	Cho $a = 46\cdot47\cdot48\cdots89\cdot90$. (a) Có bao nhiêu thừa số $2$ khi phân tích $a$ ra thừa số nguyên tố? (b) $a$ có tận cùng là bao nhiêu chữ số $0$ sau khi thực hiện phép nhân?
\end{baitoan}

\begin{baitoan}[\cite{TLCT_THCS_Toan_6_so_hoc}, 15.12., p. 102]
	(a) Cho $A = \sum_{i=0}^{2010} 2013^i = 2013^0 + 2013^1 + 2013^2 + \cdots + 2013^{2009} + 2013^{2010}$. Tính $2012A + 1$. (b) Cho $a,n\in\mathbb{N}^\star,a\ne0,a\ne1$. Rút gọn tổng $S = \sum_{i=0}^n a^i = a^0 + a^1 + a^2 + \cdots + a^n$.
\end{baitoan}

\begin{baitoan}[\cite{TLCT_THCS_Toan_6_so_hoc}, 15.13., p. 102]
	(a) Cho $A = \sum_{i=1}^{99} 3^i = 3 + 3^2 + 3^3 + 3^4 + \cdots + 3^{99}$. Tìm $n\in\mathbb{N}$ biết $2A + 3 = 3^{2n}$. (b) Chứng minh $4B + 25$ là 1 lũy thừa của $5$ với $B = \sum_{i=2}^{2012} 5^i = 5^2 + 5^3 + \cdots + 5^{2012}$. (c) Cho $C = \sum_{i=0}^{400} 4^i = 1 + 4 + 4^2 + \cdots + 4^{100},D = 4^{101}$. Chứng minh $3C < D$.
\end{baitoan}

\begin{baitoan}[\cite{TLCT_THCS_Toan_6_so_hoc}, 15.14., p. 102]
	Tính $A = 10\cdot11 + 11\cdot12 + 12\cdot13 + \cdots + 28\cdot29 + 29\cdot30$.
\end{baitoan}

\begin{baitoan}[\cite{TLCT_THCS_Toan_6_so_hoc}, 15.15., p. 103]
	Tính: (a) $A = \sum_{i=1}^{100} i^2 = 1^2 + 2^2 + 3^3 + \cdots + 10)^2$. (b) $B = \sum_{i=101}^{200} 101^2 + 102^2 + \cdots + 199^2 + 200^2$. (c) $C = 1\cdot3 + 2\cdot4 + 3\cdot5 + 4\cdot6 + \cdots + 99\cdot101 + 100\cdot102$. (d) $D = 1\cdot100 + 2\cdot99 + 3\cdot98 + \cdots + 99\cdot2 + 100\cdot1$. (e) $E = \sum_{i=1}^{98} i(i + 1)(i + 2) = 1\cdot2\cdot3 + 2\cdot3\cdot4 + 3\cdot4\cdot5 + \cdots + 98\cdot99\cdot100$.
\end{baitoan}

\begin{baitoan}[\cite{TLCT_THCS_Toan_6_so_hoc}, 15.16., p. 103]
	Cho $S_1 = 3,S_2 = 9,S_3 = 18,S_4 = 30,S_5 = 45,\ldots$ Tính $S_{100}$.
\end{baitoan}

\begin{baitoan}[\cite{TLCT_THCS_Toan_6_so_hoc}, 15.17., p. 103]
	(a) Khi phân tích ra thừa số nguyên tố, $100!$ chứa thừa số nguyên tố $3$ với số mũ bằng bao nhiêu? (b) Tích $30\cdot31\cdot32\cdots89\cdot90$ có bao nhiêu thừa số $3$ khi phân tích ra thừa số nguyên tố?
\end{baitoan}

\begin{baitoan}[\cite{TLCT_THCS_Toan_6_so_hoc}, 15.18., p. 103]
	Trong dãy số tự nhiên $1,2,3,\ldots,n$, lập các tổng: $A_1 = 1,A_2 = 2 + 3,A_3 = 4 + 5 + 6,A_4 = 7 + 8 + 9 +10,A_5 = 11 + 12 + 13 + 14 + 15,\ldots$ Tính $A_{101} - A_{100}$.
\end{baitoan}

\begin{baitoan}[\cite{TLCT_THCS_Toan_6_so_hoc}, 15.19., p. 103]
	Tính $A = \sum_{i=1}^{100} \underbrace{3\ldots3}_i = 3 + 33 + 333 + \cdots + \underbrace{3\ldots3}_{100}$.
\end{baitoan}

\begin{baitoan}[\cite{TLCT_THCS_Toan_6_so_hoc}, 15.20., p. 103]
	Tính tổng $S$ các số hạng đứng trước số $1$ cuối cùng trong tổng: $1 + 5 + 1 + 5 + 5 + 1 + 5 + 5 + 5 + 1 + 5 + 5 + 5 + 5 + 1 + \cdots + 1 + \underbrace{5 + 5 + \cdots + 5}_{2012} + 1$.
\end{baitoan}

\begin{baitoan}[\cite{TLCT_THCS_Toan_6_so_hoc}, 15.21., p. 103]
	Gọi $S_n$ là tổng các chữ số của $n\in\mathbb{N}$. Tính $\sum_{i=1}^{2013} S_i = S_1 + S_2 + \cdots + S_{2013}$.
\end{baitoan}

\begin{baitoan}[\cite{TLCT_THCS_Toan_6_so_hoc}, 15.22., p. 103]
	Tính: (a) $A = \sum_{i=15}^{2012} \dfrac{1}{i(i + 1)} = \dfrac{1}{15\cdot16} + \dfrac{1}{16\cdot17} + \dfrac{1}{17\cdot18} + \cdots + \dfrac{1}{2011\cdot2012} + \dfrac{1}{2012\cdot2013}$. (b) $B = \left(1 - \dfrac{1}{2}\right) + \left(1 - \dfrac{1}{4}\right) + \left(1 - \dfrac{1}{8}\right) + \cdots + \left(1 - \dfrac{1}{512}\right) + \left(1 - \dfrac{1}{1024}\right)$. (c) $C = 4\cdot5^{100}\left(\dfrac{1}{5} + \dfrac{1}{5^2} + \dfrac{1}{5^3} + \cdots + \dfrac{1}{5^{100}}\right) + 1$.
\end{baitoan}

\begin{baitoan}[\cite{TLCT_THCS_Toan_6_so_hoc}, 15.23., p. 104]
	Tính $A = 50 + \dfrac{50}{3} + \dfrac{25}{3} + \dfrac{20}{4} + \dfrac{10}{3} + \dfrac{10}{6\cdot7} + \cdots + \dfrac{100}{98\cdot99} + \dfrac{1}{99}$.
\end{baitoan}

\begin{baitoan}[\cite{TLCT_THCS_Toan_6_so_hoc}, 15.24., p. 104]
	Tính tổng $100$ \& $n\in\mathbb{N}$ số hạng đầu tiên của dãy: (a) $\dfrac{1}{3},\dfrac{1}{15},\dfrac{1}{35},\ldots$ (b) $\dfrac{1}{5},\dfrac{1}{45},\dfrac{1}{117},\dfrac{1}{221},\ldots$
\end{baitoan}

\begin{baitoan}[\cite{TLCT_THCS_Toan_6_so_hoc}, 15.25., p. 104]
	Tính: (a) $A = \dfrac{1}{1\cdot2} - \dfrac{1}{1\cdot2\cdot3} + \dfrac{1}{2\cdot3} - \dfrac{1}{2\cdot3\cdot4} + \dfrac{1}{3\cdot4} - \dfrac{1}{3\cdot4\cdot5} + \cdots + \dfrac{1}{99\cdot100} - \dfrac{1}{99\cdot100\cdot101}$. (b) $B = \dfrac{1}{1\cdot2\cdot3\cdot4} + \dfrac{1}{2\cdot3\cdot4\cdot5} + \dfrac{1}{3\cdot4\cdot5\cdot6} + \cdots + \dfrac{1}{47\cdot48\cdot49\cdot50}$.
\end{baitoan}

\begin{baitoan}[\cite{TLCT_THCS_Toan_6_so_hoc}, 15.26., p. 104]
	Tìm $x$ thỏa: (a) $\dfrac{1}{2013}x + 1 + \dfrac{1}{2} + \dfrac{1}{6} + \dfrac{1}{12} + \cdots + \dfrac{1}{2012\cdot2013} = 2$. (b) $2x + \dfrac{7}{6} + \dfrac{13}{12} + \dfrac{21}{20} + \dfrac{31}{30} + \dfrac{43}{42} + \dfrac{57}{56} + \dfrac{73}{72} + \dfrac{91}{90} = 10$.
\end{baitoan}

\begin{baitoan}[\cite{TLCT_THCS_Toan_6_so_hoc}, 15.27., p. 104]
	Tìm $x$ thỏa: (a) $\left(\dfrac{1}{1\cdot2\cdot3} + \dfrac{1}{2\cdot3\cdot4} + \dfrac{1}{3\cdot4\cdot5} + \cdots + \dfrac{1}{98\cdot99\cdot100}\right)x = \dfrac{49}{200}$. (b) $\dfrac{1}{5\cdot8} + \dfrac{1}{8\cdot11} + \dfrac{1}{11\cdot14} + \cdots + \dfrac{1}{x(x + 3)} = \dfrac{98}{1545}$.
\end{baitoan}

\begin{baitoan}[\cite{TLCT_THCS_Toan_6_so_hoc}, 15.28., p. 104]
	Cho 9 số $36,60,90,126,168,216,270,330,396$. Tính tỷ số giữa tổng nghịch đảo của 9 số này \& tổng của chúng.
\end{baitoan}

\begin{baitoan}[\cite{TLCT_THCS_Toan_6_so_hoc}, 15.29., p. 104]
	Rút gọn: (a) $A = \dfrac{1 + (1 + 2) + (1 + 2 + 3) + \cdots + (1 + 2 + 3 + \cdots + 100)}{(1\cdot100 + 2\cdot99 + \cdots 99\cdot2 + 100\cdot1)\cdot2013}$.\\(b) $B = \left(\dfrac{5^3}{6} + \dfrac{5^3}{12} + \dfrac{5^3}{20} + \dfrac{5^3}{30} + \dfrac{5^3}{42} + \dfrac{5^3}{56} + \dfrac{5^3}{72} + \dfrac{5^3}{90}\right):\dfrac{1124\cdot2247 - 1123}{1124 + 1123\cdot2247}$.
\end{baitoan}

\begin{baitoan}[\cite{TLCT_THCS_Toan_6_so_hoc}, 15.30., p. 104]
	Tính $A = \dfrac{4 + \dfrac{3}{5} + \cdots + \dfrac{3}{95} + \dfrac{3}{97} + \dfrac{3}{99}}{\dfrac{1}{1\cdot99} + \dfrac{1}{3\cdot97} + \dfrac{1}{5\cdot95} + \cdots + \dfrac{1}{95\cdot5} + \dfrac{1}{97\cdot3} + \dfrac{1}{99\cdot1}}$.
\end{baitoan}

\begin{baitoan}[\cite{TLCT_THCS_Toan_6_so_hoc}, 15.31., p. 105, \cite{Binh_Toan_6_tap_2}, 116., p. 47]
	(a) Chứng minh $A = \sum_{i=2}^{100} \dfrac{1}{i^2} = \dfrac{1}{2^2} + \dfrac{1}{3^2} + \cdots + \dfrac{1}{100^2} < 1$. (b) Chứng minh $B = \sum_{i=1}^{100} \dfrac{1}{i^2} = \dfrac{1}{1^2} + \dfrac{1}{2^2} + \dfrac{1}{3^2} + \cdots + \dfrac{1}{100^2} < 1\dfrac{3}{4}$. (c) Chứng minh $C = \dfrac{1}{2^2} + \dfrac{1}{4^2} + \dfrac{1}{6^2} + \cdots + \dfrac{1}{100^2} < \dfrac{1}{2}$. (d) Chứng minh $D = 2 + \dfrac{3}{4} + \dfrac{8}{9} + \dfrac{15}{16} + \cdots + \dfrac{2499}{2500} > 50$.
\end{baitoan}

\begin{baitoan}[\cite{TLCT_THCS_Toan_6_so_hoc}, 15.32., p. 105, \cite{Binh_Toan_6_tap_2}, 118., p. 47]
	(a) Cho $A = \sum_{i=4}^{100} \dfrac{1}{i^2} = \dfrac{1}{4^2} + \dfrac{1}{5^2} + \cdots + \dfrac{1}{100^2}$. Chứng minh $\dfrac{1}{5} < A < \dfrac{1}{3}$. (b) Chứng minh $\dfrac{1}{6} < \dfrac{1}{5^2} + \dfrac{1}{6^2} + \cdots + \dfrac{1}{100^2} < \dfrac{1}{4}$. (c) Cho $B = \sum_{i=100}^{199} \dfrac{1}{i^2} = \dfrac{1}{100^2} + \dfrac{1}{5^2} + \cdots + \dfrac{1}{199^2}$. Chứng minh $\dfrac{1}{200} < B < \dfrac{1}{99}$.
\end{baitoan}

\begin{baitoan}[\cite{TLCT_THCS_Toan_6_so_hoc}, 15.33., p. 105]
	(a) Chứng minh $A = \sum_{i=1}^{2012} \dfrac{1}{i!} = \dfrac{1}{1!} + \dfrac{1}{2!} + \dfrac{1}{3!} + \cdots + \dfrac{1}{2012!} < 2$. (b) Chứng minh $\dfrac{9}{10!} + \dfrac{10}{11!} + \dfrac{11}{12!} + \cdots + \dfrac{99}{100!} < \dfrac{1}{9!}$.
\end{baitoan}

\begin{baitoan}[\cite{TLCT_THCS_Toan_6_so_hoc}, 15.34., p. 105]
	(a) Cho $A = 1 - \dfrac{1}{2} + \dfrac{1}{3} - \dfrac{1}{4} + \cdots + \dfrac{1}{2011} - \dfrac{1}{2012},B = \dfrac{1}{1007} + \dfrac{1}{1008} + \cdots + \dfrac{1}{2011} + \dfrac{1}{2012}$. Tính $A:B$. (b) Cho $C = \dfrac{1}{2} - \dfrac{3}{4} + \dfrac{5}{6} - \dfrac{7}{8} + \cdots + \dfrac{197}{198} - \dfrac{199}{200},D = \dfrac{1}{51} + \dfrac{1}{52} + \cdots + \dfrac{1}{100}$. Tính $D:C$.
\end{baitoan}

\begin{baitoan}[\cite{TLCT_THCS_Toan_6_so_hoc}, 15.35., p. 105]
	Tính $A = \sum_{i=2}^{2012} \dfrac{1}{i} = \dfrac{1}{2} + \dfrac{1}{3} + \cdots + \dfrac{1}{2012},B = \dfrac{1}{2011} + \dfrac{2}{2010} + \dfrac{3}{2009} + \cdots + \dfrac{2009}{3} + \dfrac{2010}{2} + \dfrac{2011}{1}$.
\end{baitoan}

\begin{baitoan}[\cite{TLCT_THCS_Toan_6_so_hoc}, 15.36., pp. 105--106]
	Tính: (a) $A = 1.1 + 2.6 + 4.1 + 5.6 + \cdots + 148.1 + 149.6$. (b) $B = 1.2 + 2.3 + \cdots + 8.9 + 9.10 + 10.11 + \cdots 98.99 + 99.100 + 100.101 + \cdots + 998.999$. (c) $C = 1.2 + 2.4 + 3.6 + 4.8 + 5.10 + 6.12 + \cdots + 45.90$. (d) $D = 1.2 + 2.4 + 3.6 + 4.8 + 6 + 7.2 + \cdots + 120$.
\end{baitoan}

\begin{baitoan}[\cite{TLCT_THCS_Toan_6_so_hoc}, 15.37., p. 106]
	Tính nhanh: (a) $A = \prod_{i=10}^{100} \left(\dfrac{1}{10} - 1\right)\left(\dfrac{1}{11} - 1\right)\cdots\left(\dfrac{1}{100} - 1\right)$.\\(b) $B = \prod_{i=2}^{10} 1 - \dfrac{1}{i^2} = \left(1 - \dfrac{1}{2^2}\right)\left(1 - \dfrac{1}{3^2}\right)\cdots\left(1 - \dfrac{1}{10^2}\right)$. (c) $C = \left(\dfrac{7}{9} + 1\right)\left(\dfrac{7}{20} + 1\right)\left(\dfrac{7}{33} + 1\right)\cdots\left(\dfrac{7}{10800} + 1\right)$. (d) $D = \left(1 - \dfrac{28}{10}\right)\left(1 - \dfrac{52}{22}\right)\left(1 - \dfrac{80}{36}\right)\cdots\left(1 - \dfrac{21808}{10900}\right)$.
\end{baitoan}

\begin{baitoan}[\cite{TLCT_THCS_Toan_6_so_hoc}, 15.38., p. 106]
	Rút gọn: (a) $A = \dfrac{1 + \dfrac{1}{2} + \dfrac{1}{3} + \cdots + \dfrac{1}{2011} + \dfrac{1}{2012}}{\dfrac{2013}{1} + \dfrac{2014}{2} + \dfrac{2015}{3} + \cdots + \dfrac{4023}{2011} + \dfrac{4024}{2012} - 2012}$.\\(b) $B = \dfrac{\prod_{i=1}^{1000} 1 + \dfrac{2012}{i}}{\prod_{i=1}^{1000} 1 + \dfrac{1000}{i}} = \dfrac{\left(1 + \dfrac{2012}{1}\right)\left(1 + \dfrac{2012}{2}\right)\cdots\left(1 + \dfrac{2012}{1000}\right)}{\left(1 + \dfrac{1000}{1}\right)\left(1 + \dfrac{1000}{2}\right)\cdots\left(1 + \dfrac{1000}{2012}\right)}$.
\end{baitoan}

\begin{baitoan}[\cite{TLCT_THCS_Toan_6_so_hoc}, 15.39., p. 106]
	Tìm $x\in\mathbb{Q}$ thỏa: (a) $\dfrac{2}{1^2}\cdot\dfrac{6}{2^2}\cdot\dfrac{12}{3^2}\cdot\dfrac{20}{4^2}\cdots\dfrac{110}{10^2}x = -20$. (b) $\left(1 + \dfrac{1}{2} + \dfrac{1}{3} + \cdots + \dfrac{1}{2013}\right)x + 2013 = \dfrac{2014}{1} + \dfrac{2015}{2} + \cdots + \dfrac{4025}{2012} + \dfrac{4026}{2013}$. (c) $\left(\dfrac{1}{1\cdot2} + \dfrac{1}{3\cdot4} + \cdots + \dfrac{1}{99\cdot100}\right)x = \dfrac{2012}{51} + \dfrac{2012}{52} + \cdots + \dfrac{2012}{99} + \dfrac{2012}{100}$. (d) $\left(\dfrac{1}{1\cdot3} + \dfrac{1}{3\cdot5} + \cdots + \dfrac{1}{19\cdot21}\right)\cdot420 = [0.4\cdot(7.5 - 2.5x)]:0.25 = 212$.
\end{baitoan}

\begin{baitoan}[\cite{TLCT_THCS_Toan_6_so_hoc}, 15.40., p. 107]
	Cho $A = \dfrac{333}{444}\cdot\dfrac{888}{999}\cdot\dfrac{1515}{1616}\cdot\dfrac{2424}{2525}\cdot\dfrac{3535}{3636}\cdot\dfrac{4848}{4949}\cdot\dfrac{6363}{6464}\cdot\dfrac{8080}{8181}$,\\$B = \left(1 - \dfrac{1}{3}\right)\left(1 - \dfrac{1}{6}\right)\left(1 - \dfrac{1}{10}\right)\left(1 - \dfrac{1}{15}\right)\left(1 - \dfrac{1}{21}\right)\left(1 - \dfrac{1}{28}\right)\left(1 - \dfrac{1}{36}\right)$. (a) Tính $A:B$. (b) Tính tổng các nghịch đảo của $A,B$. (c) Tìm $x,y\in\mathbb{Z}$ sao cho $B < \dfrac{x}{36} < A,-B\ge\dfrac{-10}{y}\ge-A$.
\end{baitoan}

\begin{baitoan}[\cite{TLCT_THCS_Toan_6_so_hoc}, 15.41., p. 107]
	(a) Chứng minh $A = 1 + \dfrac{1}{2} + \dfrac{1}{3} + \cdots + \dfrac{1}{20}\notin\mathbb{Z}$. (b) Cho $\dfrac{a}{b} = \dfrac{1}{1\cdot2} + \dfrac{1}{3\cdot4} + \dfrac{1}{5\cdot6} + \cdots + \dfrac{1}{97\cdot98} + \dfrac{1}{99\cdot100}$. Chứng minh $a\divby151$.
\end{baitoan}

\begin{baitoan}[\cite{TLCT_THCS_Toan_6_so_hoc}, 15.42., p. 107]
	Tính $1.1 + 1.11 + 1.111 + \cdots + 1.\underbrace{1\ldots1}_9 + 1.\underbrace{1\ldots1}_{10}$.
\end{baitoan}

\begin{baitoan}[\cite{TLCT_THCS_Toan_6_so_hoc}, 15.43., p. 107]
	Tìm số hạng thứ $100$ trong dãy: $\dfrac{1}{1},\dfrac{2}{1},\dfrac{1}{2},\dfrac{3}{1},\dfrac{2}{2},\dfrac{1}{3},\dfrac{4}{1},\dfrac{3}{2},\dfrac{2}{3},\dfrac{1}{4},\dfrac{5}{1},\dfrac{4}{2},\dfrac{3}{3},\dfrac{2}{4},\dfrac{1}{5},\ldots$
\end{baitoan}

\begin{baitoan}[\cite{TLCT_THCS_Toan_6_so_hoc}, 15.44., p. 107]
	Cho $A = \prod_{i=1}^{99} 1 + \dfrac{2}{i} = \left(1 + \dfrac{2}{1}\right)\left(1 + \dfrac{2}{2}\right)\left(1 + \dfrac{2}{3}\right)\cdots\left(1 + \dfrac{2}{99}\right),B = (-1 - 2 - 3 - \cdots - 99 - 100)\left(\dfrac{1}{2} + \dfrac{1}{2^2} + \dfrac{1}{2^3} + \cdots + \dfrac{1}{2^{10}}\right)$. Tính $\dfrac{A}{B}$.
\end{baitoan}

\begin{baitoan}[\cite{TLCT_THCS_Toan_6_so_hoc}, 15.45., p. 107]
	Tính nhanh $A = \dfrac{1}{3} + \dfrac{1}{9} + \dfrac{1}{18} + \dfrac{1}{30} + \dfrac{1}{45} + \dfrac{1}{63} + \cdots + \dfrac{1}{14850}$.
\end{baitoan}

\begin{baitoan}[\cite{TLCT_THCS_Toan_6_so_hoc}, 15.46., p. 107]
	Cho $A = 1.01 + 1.02 + \cdots + 9.98 + 9.99 + 10,B = 2 - \dfrac{5}{3} + \dfrac{7}{6} - \dfrac{9}{10} + \dfrac{11}{15} - \dfrac{13}{21} + \dfrac{15}{28} - \dfrac{17}{36} + \dfrac{19}{45}$. Tính $2A + \dfrac{455}{3}B$.
\end{baitoan}

%------------------------------------------------------------------------------%

\section{Miscellaneous}
SGK \cite[BTCCCV, pp. 71--72]{SGK_Toan_6_Canh_Dieu_tap_2}: 1. 2. 3. 4. 5. 6. 7. 8. 9. SBT \cite[BTCCV, pp. 59--63]{SBT_Toan_6_Canh_Dieu_tap_2}: 121. 122. 123. 124. 125. 126. 127. 128. 129. 130. 131. 132. 133. 134. 135. 136. 137. 138.

\begin{baitoan}[\cite{Tuyen_Toan_6}, VD71, p. 72]
	Cho $a,m,n\in\mathbb{N}^\star$, so sánh $A = \dfrac{10}{a^m} + \dfrac{10}{a^n},B = \dfrac{11}{a^m} + \dfrac{9}{a^n}$.
\end{baitoan}

\begin{baitoan}[\cite{Tuyen_Toan_6}, VD72, p. 73]
	Cho 2 phân số có tổng bằng $5$ lần tích của chúng. Tính tổng các số nghịch đảo của 2 phân số đó.
\end{baitoan}

\begin{baitoan}[\cite{Tuyen_Toan_6}, 366., p. 73]
	Sắp xếp tăng dần: (a) $\dfrac{7}{48},\dfrac{11}{72},\dfrac{17}{120}$. (b) $\dfrac{31}{49},\dfrac{62}{97},\dfrac{93}{140}$.
\end{baitoan}

\begin{baitoan}[\cite{Tuyen_Toan_6}, 367., p. 73]
	Rút gọn $A = \dfrac{7\cdot9 + 14\cdot27 + 21\cdot36}{21\cdot27 + 42\cdot81 + 63\cdot108}$.
\end{baitoan}

\begin{baitoan}[\cite{Tuyen_Toan_6}, 368., p. 73]
	Cho $A = \{0,7,14,21,28,35,42\}$. Tìm $a,b\in A$ để: (a) $\dfrac{a}{b}\ \max$. (b) $\dfrac{a - b}{a + b}$ là phân số dương nhỏ nhất có thể.
\end{baitoan}

\begin{baitoan}[\cite{Tuyen_Toan_6}, 369., p. 73]
	Chứng minh $\dfrac{31}{2}\cdot\dfrac{32}{2}\cdot\dfrac{33}{2}\cdots\dfrac{60}{2} = 1\cdot3\cdot5\cdots59$.
\end{baitoan}

\begin{baitoan}[\cite{Tuyen_Toan_6}, 370., p. 73]
	Chứng minh $\dfrac{\left(3\dfrac{2}{15} + \dfrac{1}{5}\right):2\dfrac{1}{2}}{\left(5\dfrac{3}{7} - 2\dfrac{1}{4}\right):4\dfrac{43}{56}} = \dfrac{1.2:\left(1\dfrac{1}{5}\cdot1\dfrac{1}{4}\right)}{0.32 + \dfrac{2}{25}}$.
\end{baitoan}

\begin{baitoan}[\cite{Tuyen_Toan_6}, 371., p. 74]
	Cho $A = \dfrac{\dfrac{1}{99} + \dfrac{2}{98} + \dfrac{3}{97} + \cdots + \dfrac{99}{11}}{\dfrac{1}{2} + \dfrac{1}{3} + \dfrac{1}{4} + \cdots + \dfrac{1}{100}},B = \dfrac{92 - \dfrac{1}{9} - \dfrac{2}{10} - \dfrac{3}{11} - \cdots - \dfrac{92}{100}}{\dfrac{1}{45} + \dfrac{1}{50} + \dfrac{1}{55} + \cdots + \dfrac{1}{500}}$. Tính $A,B$ rồi tính tỷ số $\%$ của $A,B$.
\end{baitoan}

\begin{baitoan}[\cite{Tuyen_Toan_6}, 372., p. 74]
	Tìm $x\in\mathbb{Q}$ thỏa $\dfrac{2}{3}x - 70\dfrac{10}{11}:\left(\dfrac{131313}{151515} + \dfrac{131313}{353535} + \dfrac{131313}{636363} + \dfrac{131313}{999999}\right) = -5$.
\end{baitoan}

\begin{baitoan}[\cite{Tuyen_Toan_6}, 373., p. 74]
	Lúc gần {\rm8:00}, kim phút ở trước kim giờ $9$ khoảng chia phút. Lúc đó là lúc mấy giờ?
\end{baitoan}

\begin{baitoan}[\cite{Tuyen_Toan_6}, 374., p. 74]
	1 nhà máy có 3 phân xưởng. Số công nhân của phân xưởng I bằng $36\%$ tổng số công nhân của nhà máy. Số công nhân của phân xưởng II bằng $\dfrac{3}{5}$ số công nhân của phân xưởng III. Biết số công nhân của phân xưởng III hơn số công nhân của phân xưởng I là $18$ người. Tính số công nhân của mỗi phân xưởng.
\end{baitoan}

\begin{baitoan}[\cite{Tuyen_Toan_6}, 375., p. 74]
	1 xe tải chạy từ A \& có thể đến B sau {\rm6 h}. Sau khi xe tải chạy được {\rm2 h} thì 1 xe con khởi hành từ B chạy về A \& gặp lại xe tải sau {\rm1 h 36 ph}. Tính thời gian xe con chạy từ B tới A.
\end{baitoan}

\begin{baitoan}[\cite{Tuyen_Toan_6}, 376., p. 74]
	1 công nhân có kế hoạch hoàn thành công việc được giao trong {\rm3 h 20 ph}. 1 công nhân khác có thể hoàn thành công việc đó trong {\rm4 h 10 ph}. 2 người cùng làm được $72$ sản phẩm. Tính số sản phẩm mỗi người làm được.
\end{baitoan}

\begin{baitoan}[\cite{Tuyen_Toan_6}, 377., p. 74]
	Có 1 bình đựng đầy 1 chất lỏng được chia làm 2 phần: Phần I còn thiếu $\dfrac{2}{3}$ {\rm l} thì được $\dfrac{2}{3}$ bình. Phần II gồm $\dfrac{2}{3}$ chỗ còn lại \& $\dfrac{2}{3}$ {\rm l}. Tính dung tích của bình \& tỷ số $\%$ giữa 2 phần.
\end{baitoan}

\begin{baitoan}[\cite{Tuyen_Toan_6}, 378., p. 74]
	Hiện nay mẹ $40$ tuổi, con $12$ tuổi. Sau bao nhiêu năm nữa thì tuổi con bằng $\dfrac{3}{7}$ tuổi mẹ?
\end{baitoan}

\begin{baitoan}[\cite{Tuyen_Toan_6}, 379., p. 74]
	Lá cờ Tổ quốc ở cột cờ Lũng Cú, Hà Giang, có diện tích lên tới $\rm54\ m^2$ tượng trưng cho $54$ dân tộc anh em sinh sống trên đất nước Việt Nam. Lá cờ hình chữ nhật, có chiều rộng bằng $\dfrac{2}{3}$ chiều dài. Tính kích thước lá cờ.
\end{baitoan}

%------------------------------------------------------------------------------%

\printbibliography[heading=bibintoc]
	
\end{document}