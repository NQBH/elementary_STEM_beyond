\documentclass{article}
\usepackage[backend=biber,natbib=true,style=alphabetic,maxbibnames=50]{biblatex}
\addbibresource{/home/nqbh/reference/bib.bib}
\usepackage[utf8]{vietnam}
\usepackage{tocloft}
\renewcommand{\cftsecleader}{\cftdotfill{\cftdotsep}}
\usepackage[colorlinks=true,linkcolor=blue,urlcolor=red,citecolor=magenta]{hyperref}
\usepackage{amsmath,amssymb,amsthm,float,graphicx,mathtools,tikz}
\usetikzlibrary{angles,calc,intersections,matrix,patterns,quotes,shadings}
\allowdisplaybreaks
\newtheorem{assumption}{Assumption}
\newtheorem{baitoan}{}
\newtheorem{cauhoi}{Câu hỏi}
\newtheorem{conjecture}{Conjecture}
\newtheorem{corollary}{Corollary}
\newtheorem{dangtoan}{Dạng toán}
\newtheorem{definition}{Definition}
\newtheorem{dinhly}{Định lý}
\newtheorem{dinhnghia}{Định nghĩa}
\newtheorem{example}{Example}
\newtheorem{ghichu}{Ghi chú}
\newtheorem{hequa}{Hệ quả}
\newtheorem{hypothesis}{Hypothesis}
\newtheorem{lemma}{Lemma}
\newtheorem{luuy}{Lưu ý}
\newtheorem{nhanxet}{Nhận xét}
\newtheorem{notation}{Notation}
\newtheorem{note}{Note}
\newtheorem{principle}{Principle}
\newtheorem{problem}{Problem}
\newtheorem{proposition}{Proposition}
\newtheorem{question}{Question}
\newtheorem{remark}{Remark}
\newtheorem{theorem}{Theorem}
\newtheorem{vidu}{Ví dụ}
\usepackage[left=1cm,right=1cm,top=5mm,bottom=5mm,footskip=4mm]{geometry}
\def\labelitemii{$\circ$}
\DeclareRobustCommand{\divby}{%
	\mathrel{\vbox{\baselineskip.65ex\lineskiplimit0pt\hbox{.}\hbox{.}\hbox{.}}}%
}

\title{Problem: Fraction \& Decimal -- Bài Tập: Phân Số \& Số Thập Phân}
\author{Nguyễn Quản Bá Hồng\footnote{Independent Researcher, Ben Tre City, Vietnam\\e-mail: \texttt{nguyenquanbahong@gmail.com}; website: \url{https://nqbh.github.io}.}}
\date{\today}

\begin{document}
\maketitle
\tableofcontents

%------------------------------------------------------------------------------%

\section{Basic Properties of Fraction. Simplify Fraction -- Phân Số. Tính Chất Cơ Bản của Phân Số. Rút Gọn Phân Số}

\begin{baitoan}[\cite{Binh_boi_duong_Toan_6_tap_2}, H1, p. 5]
	Các cách viết nào sau đây là phân số? $\dfrac{6}{7},\dfrac{-1.3}{5},\dfrac{12}{-8},\dfrac{-7}{0},\dfrac{2.34}{-6.5},\dfrac{-0}{3}$.
\end{baitoan}

\begin{baitoan}[\cite{Binh_boi_duong_Toan_6_tap_2}, H2, p. 5]
	Rút gọn $\dfrac{30 + 6}{30 + 12}$ về phân số tối giản.
\end{baitoan}

\begin{baitoan}[\cite{Binh_boi_duong_Toan_6_tap_2}, H3, p. 6]
	(a) Khi nào thì 1 phân số có thể viết được dưới dạng 1 số nguyên? (b) 2 phân số $\dfrac{a}{b}$ \& $\dfrac{-a}{-b}$ có bằng nhau không? (c) Nếu chia cả tử \& mẫu của 1 phân số cho cùng 1 số nguyên $n\ne0$ thì ta có được 1 phân số bằng nó không? (d) Khi chia cả tử \& mẫu của phân số $\dfrac{a}{b}$, $a,b > 0$, cho 1 ước chung của $a,b$ thì ta có thu được 1 phân số tối giản không? (e) Nếu $\dfrac{a}{b}$ là phân số tối giản thì mọi phân số bằng $\dfrac{a}{b}$ có dạng gì?
\end{baitoan}

\begin{baitoan}[\cite{Binh_boi_duong_Toan_6_tap_2}, VD1, p. 6]
	Cho tập hợp $A = \{-2,0,7\}$. Viết tất cả các phân số $\dfrac{a}{b}$ với $a,b\in A$.
\end{baitoan}

\begin{baitoan}[\cite{Binh_boi_duong_Toan_6_tap_2}, VD2, p. 6]
	Tìm $x,y\in\mathbb{Z}$ thỏa $\dfrac{x}{15} = \dfrac{24}{y} = \dfrac{-6}{5}$.
\end{baitoan}

\begin{baitoan}[\cite{Binh_boi_duong_Toan_6_tap_2}, VD3, p. 6]
	Rút gọn: (a) $\dfrac{-64}{96}$. (b) $\dfrac{131313}{252525}$. (c) $\dfrac{3510 - 135}{4680 - 180}$. (d) $\dfrac{2^4\cdot3^2}{6^2\cdot5}$.
\end{baitoan}

\begin{baitoan}[\cite{Binh_boi_duong_Toan_6_tap_2}, VD4, p. 7]
	Viết tất cả các phân số bằng $ \dfrac{30}{102}$ mà có tử \& mẫu là các số tự nhiên có 2 chữ số.
\end{baitoan}

\begin{baitoan}[\cite{Binh_boi_duong_Toan_6_tap_2}, VD5, p. 7]
	Tìm phân số bằng $\dfrac{35}{80}$ biết tổng của mẫu số \& 2 lần tử số bằng $210$.
\end{baitoan}

\begin{baitoan}[\cite{Binh_boi_duong_Toan_6_tap_2}, VD6, p. 8]
	Cho biểu thức $A = \dfrac{2n - 1}{n + 3}$ với $n\in\mathbb{Z}$. Tìm $n$ để $A\in\mathbb{Z}$.
\end{baitoan}

\begin{baitoan}[\cite{Binh_boi_duong_Toan_6_tap_2}, VD7, p. 8]
	Cho phân số $\dfrac{25}{49}$. Hỏi cần bớt ở tử \& mẫu của phân số đã cho cùng 1 số nào để được 1 phân số mới bằng $\dfrac{5}{13}$?
\end{baitoan}

\begin{baitoan}[\cite{Binh_boi_duong_Toan_6_tap_2}, 1.1., p. 8]
	Rút gọn các phân số trong trả lời của các câu hỏi sau thành phân số tối giản: (a) 1 mẫu Bắc Bộ bằng $\rm3600 m^2$. Hỏi 1 mẫu Bắc Bộ bằng mấy phần của 1 hecta? $\rm1\ ha = 10000\ m^2$. (b) Mỗi khoảng thời gian sau bằng bao nhiêu phần của 1 giờ: $15$ phút, $30$ phút, $45$ phút, $90$ phút, $4500$ giây? (c) Pound là đơn vị đo khối lượng được dùng phổ biến ở nước Anh \& 1 số nước khác. Biết $100$ pound $= 45$ {\rm kg}, hỏi 1 pound bằng mấy phần của {\rm1 kg}? (d) Inch là 1 trong các đơn vị đo chiều dài phổ biến trên thế giới. Biết $\rm1\ in = 2.54\ cm$, hỏi {\rm1 cm} bằng mấy phần của {\rm1 inch}?
\end{baitoan}

\begin{baitoan}[\cite{Binh_boi_duong_Toan_6_tap_2}, 1.2., p. 8]
	Tìm $x\in\mathbb{Z}$ thỏa: (a) $\dfrac{x - 2}{15} = \dfrac{9}{5}$. (b) $\dfrac{2 - x}{16} = \dfrac{-4}{x - 2}$.
\end{baitoan}

\begin{baitoan}[\cite{Binh_boi_duong_Toan_6_tap_2}, 1.3., p. 9]
	Tìm $x,y\in\mathbb{Z}$ thỏa: (a) $ \dfrac{x}{7} = \dfrac{5}{y}$ \& $x > y$. (b) $\dfrac{2}{x} = \dfrac{y}{-7}$ \& $x > 0$.
\end{baitoan}

\begin{baitoan}[\cite{Binh_boi_duong_Toan_6_tap_2}, 1.4., p. 9]
	Cho phân số $\dfrac{7}{11}$. Cần cộng vào tử số \& mẫu số của phân số đã cho với cùng 1 số nào để được phân số mới bằng $\dfrac{3}{4}$?
\end{baitoan}

\begin{baitoan}[\cite{Binh_boi_duong_Toan_6_tap_2}, 1.5., p. 9]
	Cho phân số $\dfrac{89}{143}$. Tìm 1 số tự nhiên để khi thêm vào tử số \& bớt đi ở mẫu số của phân số đã cho với cùng số đó thì ta được 1 phân số mới bằng $\dfrac{12}{17}$.
\end{baitoan}

\begin{baitoan}[\cite{Binh_boi_duong_Toan_6_tap_2}, 1.6., p. 9]
	Tìm phân số bằng phân số $\dfrac{147}{252}$ biết phân số đó có: (a) Tổng của tử \& mẫu bằng $228$. (b) Hiệu của tử \& mẫu bằng $40$. (c) Tích của tử \& mẫu bằng $756$.
\end{baitoan}

\begin{baitoan}[\cite{Binh_boi_duong_Toan_6_tap_2}, 1.7., p. 9]
	Tìm phân số có mẫu bằng $7$ biết khi cộng tử với $12$ \& nhân mẫu với $3$ thì ta được 1 phân số mới bằng phân số ban đầu.
\end{baitoan}

\begin{baitoan}[\cite{Binh_boi_duong_Toan_6_tap_2}, 1.8., p. 9]
	Giải thích tại sao $\forall n\in\mathbb{N}^\star$, giá trị của 2 biểu thức sau là phân số tối giản: (a) $\dfrac{2n + 5}{3n + 7}$. (b) $\dfrac{6n - 14}{2n - 5}$.
\end{baitoan}

\begin{baitoan}[\cite{Binh_boi_duong_Toan_6_tap_2}, 1.9., p. 9]
	Tìm phân số $\dfrac{a}{b}$ bằng phân số $\dfrac{54}{126}$ biết: (a) $\mbox{\rm ƯCLN}(a,b) = 12$. (b) ${\rm BCNN}(a,b) = 105$.
\end{baitoan}

\begin{baitoan}[\cite{Binh_boi_duong_Toan_6_tap_2}, 1.10., p. 9]
	Tìm $n\in\mathbb{Z}$ để 3 biểu thức sau đồng thời có giá trị nguyên: $\dfrac{-8}{n},\dfrac{13}{n - 1},\dfrac{4}{n + 2}$.
\end{baitoan}

\begin{baitoan}[\cite{Binh_boi_duong_Toan_6_tap_2}, p. 10]
	Làm sao để cắt ra được 1 đoạn dây dài {\rm10 m} từ 1 sợi dây dài {\rm16 m} mà không dùng thước đo?
\end{baitoan}

\begin{baitoan}[\cite{TLCT_THCS_Toan_6_so_hoc}, VD8.1, p. 51]
	Quan sát dãy phân số $\dfrac{1}{2},\dfrac{5}{12},\dfrac{1}{3},\dfrac{1}{4},\dfrac{1}{6}$ để viết tiếp 1 phân số nữa theo quy luật của dãy.
\end{baitoan}

\begin{baitoan}[\cite{TLCT_THCS_Toan_6_so_hoc}, VD8.2, p. 52]
	Tìm $n\in\mathbb{Z}$ để cả 3 phân số $\dfrac{15}{n},\dfrac{12}{n + 2},\dfrac{6}{2n - 5}\in\mathbb{Z}$.
\end{baitoan}

\begin{baitoan}[\cite{TLCT_THCS_Toan_6_so_hoc}, VD8.3, p. 52]
	Cho phân số $\dfrac{1 + 2 + \cdots + 20}{6 + 7 + \cdots + 36}$. Xóa 1 số hạng ở tử \& 1 số hạng ở mẫu của phân số này để giá trị của phân số đó không đổi.
\end{baitoan}

\begin{baitoan}[\cite{TLCT_THCS_Toan_6_so_hoc}, VD8.4, p. 53]
	Tìm $a,b\in\mathbb{N}$ biết $\dfrac{a}{b} = \dfrac{132}{143},{\rm BCNN}(a,b) = 1092$.
\end{baitoan}

\begin{baitoan}[\cite{TLCT_THCS_Toan_6_so_hoc}, 8.1., p. 53]
	Tìm $x\in\mathbb{Z}$ thỏa: (a) $\dfrac{7}{x} = \dfrac{x}{28}$. (b) $\dfrac{10 + x}{17 + x} = \dfrac{3}{4}$. (c) $\dfrac{40 + x}{77 - x} = \dfrac{6}{7}$.
\end{baitoan}

\begin{baitoan}[\cite{TLCT_THCS_Toan_6_so_hoc}, 8.2., p. 53]
	Tìm $a,b\in\mathbb{Z}$ thỏa $a^3 + b^3 = 1216$ \& phân số $\dfrac{a}{b}$ rút gọn được thành $\dfrac{3}{5}$.
\end{baitoan}

\begin{baitoan}[\cite{TLCT_THCS_Toan_6_so_hoc}, 8.3., p. 53]
	Viết các phân số tối giản $\dfrac{a}{b}$  với $a,b\in\mathbb{Z}$ \& $ab = 100$.
\end{baitoan}

\begin{baitoan}[\cite{TLCT_THCS_Toan_6_so_hoc}, 8.4., p. 53]
	Rút gọn phân số: (a) $\dfrac{10\cdot11 + 50\cdot55 + 70\cdot77}{11\cdot12 + 55\cdot60 + 77\cdot84}$. (b) $\dfrac{1\cdot3\cdot5\cdots49}{26\cdot27\cdot28\cdots50}$.
\end{baitoan}

\begin{baitoan}[\cite{TLCT_THCS_Toan_6_so_hoc}, 8.5., p. 53]
	Tìm $n\in\mathbb{Z}$ thỏa: (a) $\dfrac{n + 3}{n - 2}$ là số nguyên âm. (b) $\dfrac{n + 7}{3n - 1}\in\mathbb{Z}$. (c) $\dfrac{3n + 2}{4n - 5}\in\mathbb{Z}$.
\end{baitoan}

\begin{baitoan}[\cite{TLCT_THCS_Toan_6_so_hoc}, 8.6., p. 54]
	Chứng minh phân số $\dfrac{n - 5}{3n - 14}$ tối giản $\forall n\in\mathbb{Z}$.
\end{baitoan}

\begin{baitoan}[\cite{TLCT_THCS_Toan_6_so_hoc}, 8.7., p. 54]
	Tìm $n\in\mathbb{Z}$ thỏa $\dfrac{2n - 1}{3n + 2}$ rút gọn được.
\end{baitoan}

\begin{baitoan}[\cite{TLCT_THCS_Toan_6_so_hoc}, 8.8., p. 54]
	Tìm $n\in\mathbb{N}$ nhỏ nhất để 5 phân số $\dfrac{n + 7}{3},\dfrac{n + 8}{4},\dfrac{n + 9}{5},\dfrac{n + 10}{6},\dfrac{n + 11}{7}$.
\end{baitoan}

\begin{baitoan}[\cite{TLCT_THCS_Toan_6_so_hoc}, 8.9., p. 54]
	Tìm $a,b\in\mathbb{Z}$ biết $\dfrac{a}{b} = \dfrac{49}{56},\mbox{\rm ƯCLN}(a,b) = 12$.
\end{baitoan}

\begin{baitoan}[\cite{TLCT_THCS_Toan_6_so_hoc}, 8.10., p. 54]
	Tìm $a,b,c,d\in\mathbb{N}$ nhỏ nhất thỏa $\dfrac{a}{b} = \dfrac{5}{14},\dfrac{b}{c} = \dfrac{21}{28},\dfrac{c}{d} = \dfrac{6}{11}$.
\end{baitoan}

%------------------------------------------------------------------------------%

\section{Quy Đồng Mẫu Nhiều Phân Số. So Sánh Phân Số}

\begin{baitoan}[\cite{Binh_boi_duong_Toan_6_tap_2}, H1, p. 12]
	{\rm Đ{\tt/}S?} (a) 1 cách để quy đồng mẫu 2 phân số $\dfrac{a}{b}$ \& $\dfrac{c}{d}$ là $\dfrac{a}{b} = \dfrac{ad}{bd}$, $\frac{c}{d} = \dfrac{cb}{db}$. (b) Phân số dương lớn hơn phân số âm. (c) Trong 2 phân số có cùng mẫu, phân số nào có tử lớn hơn thì phân số đó lớn hơn. (d) Trong 2 phân số có mẫu số dương \& tử bằng nhau, phân số nào có mẫu nhỏ hơn thì phân số đó lớn hơn. (e) $3\dfrac{7}{5}$ là 1 hỗn số. (f) Phân số $\dfrac{22}{5}$ viết dưới dạng hỗn số là $4\dfrac{2}{5}$.
\end{baitoan}

\begin{baitoan}[\cite{Binh_boi_duong_Toan_6_tap_2}, H2, p. 12]
	{\rm Đ{\tt/}S?} Thực hiện quy đồng mẫu 2 phân số $\dfrac{10}{12}$ \& $\dfrac{5}{8}$, Sơn \& Huy đã làm như sau: Sơn: $\dfrac{10}{12} = \dfrac{10\cdot8}{12\cdot8} = \dfrac{80}{96}$, $\dfrac{5}{8} = \dfrac{5\cdot12}{8\cdot12} = \dfrac{60}{96}$. Huy: $\dfrac{10}{12} = \dfrac{5}{6} = \dfrac{5\cdot4}{6\cdot4} = \dfrac{20}{24}$, $\dfrac{5}{8} = \dfrac{5\cdot3}{8\cdot3} = \dfrac{15}{24}$.
\end{baitoan}

\begin{baitoan}[\cite{Binh_boi_duong_Toan_6_tap_2}, H3, p. 12]
	{\rm Đ{\tt/}S?} Để so sánh 2 phân số $\dfrac{23}{-5}$ \&  $\dfrac{24}{-5}$, Hà đã giải thích như sau: $\dfrac{23}{-5}$ \&  $\dfrac{24}{-5}$ là 2 phân số co cùng mẫu \& $23 < 24$ nên $\dfrac{23}{-5} < \dfrac{24}{-5}$.
\end{baitoan}

\begin{baitoan}[\cite{Binh_boi_duong_Toan_6_tap_2}, VD1, p. 12]
	Quy đồng mẫu các phân số: (a) $\dfrac{-3}{16},\dfrac{5}{-24}$. (b) $\dfrac{3}{14},\dfrac{-5}{18},\dfrac{25}{-42}$. (c) $\dfrac{3}{16},\dfrac{5}{48},\dfrac{-7}{4}$. (d) $\dfrac{3}{7},\dfrac{-8}{5},\dfrac{5}{12}$. (e) $\dfrac{-15}{18},\dfrac{42}{-72},\dfrac{32}{120}$.
\end{baitoan}

\begin{baitoan}[\cite{Binh_boi_duong_Toan_6_tap_2}, VD2, p. 13]
	Sắp xếp các phân số sau theo thứ tự tăng dần: (a) $\dfrac{5}{24},\dfrac{11}{36},\dfrac{17}{60}$. (b) $\dfrac{23}{47},\dfrac{69}{85},\dfrac{92}{137}$. (c) $\dfrac{17}{60},\dfrac{16}{73}$.
\end{baitoan}

\begin{baitoan}[\cite{Binh_boi_duong_Toan_6_tap_2}, VD3, p. 14]
	(a) Viết 2 phân số sau dưới dạng hỗn số: $\dfrac{26}{5},\dfrac{153}{25}$. (b) Viết 2 hỗn số sau dưới dạng phân số: $3\dfrac{2}{7},8\dfrac{3}{5}$.
\end{baitoan}

\begin{baitoan}[\cite{Binh_boi_duong_Toan_6_tap_2}, VD4, p. 14]
	So sánh $\dfrac{62}{15},\dfrac{70}{17}$.
\end{baitoan}

\begin{baitoan}[\cite{Binh_boi_duong_Toan_6_tap_2}, VD5, p. 15]
	2 bạn Mai \& Đào đi xe đạp đến trường với cùng tốc độ. Mai đi hết $\dfrac{2}{3}$ giờ, Đào đi hết $\dfrac{3}{4}$ giờ. Hỏi nhà ai cách xa trường hơn?
\end{baitoan}

\begin{baitoan}[\cite{Binh_boi_duong_Toan_6_tap_2}, VD6, p. 15]
	Tìm các phân số có mẫu là $5$, lớn hơn $\dfrac{-2}{3}$ \& nhỏ hơn $\dfrac{1}{-6}$.
\end{baitoan}

\begin{baitoan}[\cite{Binh_boi_duong_Toan_6_tap_2}, 2.1., p. 15]
	Quy đồng mẫu các phân số: (a) $\dfrac{13}{30},\dfrac{-7}{120}$. (b) $\dfrac{36}{75},\dfrac{6}{11}$. (c) $\dfrac{3}{25},\dfrac{4}{5},\dfrac{-8}{75}$. (d) $\dfrac{-5}{18},\dfrac{17}{60},\dfrac{32}{-45}$.
\end{baitoan}

\begin{baitoan}[\cite{Binh_boi_duong_Toan_6_tap_2}, 2.2., p. 15]
	Sắp xếp các phân số sau theo thứ tự từ nhỏ đến lớn: (a) $\dfrac{6}{7},\dfrac{9}{10}$. (b) $\dfrac{10}{-21},\dfrac{-4}{7},\dfrac{7}{9}$. (c) $\dfrac{-5}{-28},\dfrac{6}{35},\dfrac{27}{-180}$. (d) $\dfrac{3}{5},\dfrac{-5}{3},\dfrac{-36}{-60},\dfrac{18}{21}$.
\end{baitoan}

\begin{baitoan}[\cite{Binh_boi_duong_Toan_6_tap_2}, 2.3., p. 15]
	2 bạn mai \& Đào vào hiệu sách chọn được 1 cuốn sách mà cả 2 cùng thích, mỗi bạn mua 1 quyển. Sau ngày nghỉ cuối tuần, Mai đã đọc được $\dfrac{7}{8}$ số trang còn Đào đã đọc được $\dfrac{4}{5}$ số trang của cuốn sách đó. Hỏi ai đã đọc được nhiều hơn?
\end{baitoan}

\begin{baitoan}[\cite{Binh_boi_duong_Toan_6_tap_2}, 2.4., p. 16]
	Sơ kết học kỳ I, lớp 6A có $\dfrac{3}{4}$ số học sinh là học sinh giỏi môn Toán, $\dfrac{3}{5}$ số học sinh là học sinh giỏi môn Ngữ Văn, $\dfrac{2}{3}$ số học sinh là học sinh giỏi môn Anh văn. Sắp xếp theo thứ tự các môn học này theo số lượng học sinh giỏi từ nhiều nhất đến ít nhất.
\end{baitoan}

\begin{baitoan}[\cite{Binh_boi_duong_Toan_6_tap_2}, 2.5., p. 16]
	Tìm 4 phân số lớn hơn $\dfrac{5}{12}$ \& nhỏ hơn $\dfrac{5}{8}$.
\end{baitoan}

\begin{baitoan}[\cite{Binh_boi_duong_Toan_6_tap_2}, 2.6., p. 16]
	Tìm các phân số thỏa mãn: (a) Có mẫu là $20$, lớn hơn $\dfrac{4}{13}$, \& nhỏ hơn $\dfrac{5}{13}$. (b) Có mẫu là $14$, lớn hơn $\dfrac{-2}{21}$, \& nhỏ hơn $\dfrac{2}{9}$.
\end{baitoan}

\begin{baitoan}[\cite{Binh_boi_duong_Toan_6_tap_2}, 2.7., p. 16]
	Tìm các số nguyên $n$ lớn hơn $\dfrac{283}{23}$ \& nhỏ hơn $\dfrac{467}{31}$.
\end{baitoan}

\begin{baitoan}[\cite{Binh_boi_duong_Toan_6_tap_2}, 2.8., p. 16]
	An \& Bình đạp xe với tốc độ không đổi trên cùng 1 quãng đường. An đi hết $36$ phút, Bình đi hết $44$ phút. (a) So sánh quãng đường mà An đi được trong $20$ phút với quãng đường mà Bình đi được trong $26$ phút. (b) Bình phải đi trong bao lâu để được quãng đường bằng quãng đường An đi được trong $18$ phút?
\end{baitoan}

\begin{baitoan}[\cite{Binh_boi_duong_Toan_6_tap_2}, 2.9., p. 16]
	Tìm $x\in\mathbb{Z}$ thỏa $\dfrac{-7}{12} < \dfrac{x - 1}{4} < \dfrac{2}{3}$.
\end{baitoan}

\begin{baitoan}[\cite{Binh_boi_duong_Toan_6_tap_2}, 2.10., p. 16]
	Tìm $x,y\in\mathbb{Z}$ thỏa: (a) $\dfrac{-1}{3} < \dfrac{x}{36} < \dfrac{y}{118} < \dfrac{-1}{4}$. (b) $\dfrac{1}{220} < \dfrac{x}{165} < \dfrac{y}{132} < \frac{1}{60}$.
\end{baitoan}

\begin{baitoan}[\cite{Binh_boi_duong_Toan_6_tap_2}, 2.11., p. 16]
	Cho $a,b,c\in\mathbb{N}^\star$. Chứng minh: (a) Nếu $\dfrac{a}{b} < 1$ thì $\dfrac{a}{b} < \dfrac{a + c}{b + c}$. (b) Nếu $\dfrac{a}{b} > 1$ thì $\dfrac{a}{b} > \dfrac{a + c}{b + c}$. Áp dụng: So sánh $\dfrac{17}{18}$ \& $\dfrac{26}{27}$.
\end{baitoan}

\begin{baitoan}[\cite{TLCT_THCS_Toan_6_so_hoc}, VD9.1, p. 55]
	(a) Chứng minh trong 2 phân số có cùng 1 tử, tử \& mẫu đều dương, phân số nào có mẫu nhỏ hơn thì phân số đó lớn hơn. (b) Áp dụng tính chất này để so sánh phân số: (i) $\frac{10}{11},\dfrac{12}{13},\dfrac{15}{16}$. (ii) $\dfrac{n + 1}{n + 5},\dfrac{n + 2}{n + 3}$, $n\in\mathbb{N}$.
\end{baitoan}

\begin{baitoan}[\cite{TLCT_THCS_Toan_6_so_hoc}, VD9.2, p. 56]
	Chứng minh nếu cộng cả tử \& mẫu của 1 phân số nhỏ hơn $1$, tử \& mẫu đều dương, với cùng 1 số nguyên dương thì giá trị của phân số đó tăng thêm.
\end{baitoan}

\begin{baitoan}[\cite{TLCT_THCS_Toan_6_so_hoc}, VD9.3, p. 56]
	So sánh $A = \dfrac{13579}{34567},B = \dfrac{13580}{34569}$.
\end{baitoan}

\begin{baitoan}[\cite{TLCT_THCS_Toan_6_so_hoc}, VD9.4, p. 57]
	So sánh $A = \dfrac{10^8 + 1}{10^9 + 1},B = \dfrac{10^9 + 1}{10^{10} + 1}$.
\end{baitoan}

\begin{baitoan}[\cite{TLCT_THCS_Toan_6_so_hoc}, 9.1., p. 57]
	Xếp các phân số $\dfrac{10}{19},\dfrac{12}{23},\dfrac{15}{17},\dfrac{20}{29},\dfrac{60}{71}$ theo thứ tự tăng dần.
\end{baitoan}

\begin{baitoan}[\cite{TLCT_THCS_Toan_6_so_hoc}, 9.2., p. 57]
	Tìm phân số tối giản $\dfrac{a}{b} < 1$ có $ab = 80$.
\end{baitoan}

\begin{baitoan}[\cite{TLCT_THCS_Toan_6_so_hoc}, 9.3., p. 58]
	Tìm $x\in\mathbb{Z}$ thỏa: (a) $\dfrac{1}{5} < \dfrac{x}{30} < \dfrac{1}{4}$. (b) $\dfrac{5}{8} < \dfrac{4}{x} < \dfrac{5}{7}$.
\end{baitoan}

\begin{baitoan}[\cite{TLCT_THCS_Toan_6_so_hoc}, 9.4., p. 58]
	So sánh phân số mà không quy đồng mẫu hoặc tử: (a) $\dfrac{7}{15},\dfrac{20}{39}$. (b) $\dfrac{14}{41},\dfrac{17}{54}$.
\end{baitoan}

\begin{baitoan}[\cite{TLCT_THCS_Toan_6_so_hoc}, 9.5., p. 58]
	So sánh phân số $\forall n\in\mathbb{N}$: (a) $\dfrac{n}{2n + 3}$. (b) $\dfrac{n}{3n + 1},\dfrac{2n}{6n + 1}$.
\end{baitoan}

\begin{baitoan}[\cite{TLCT_THCS_Toan_6_so_hoc}, 9.6., p. 58]
	So sánh phân số: (a) $A = \dfrac{35420}{35423},B = \dfrac{25343}{25345}$. (b) $C = \dfrac{5^{12} + 1}{5^{13} + 1},D = \dfrac{5^{11} + 1}{5^{12} + 1}$.
\end{baitoan}

\begin{baitoan}[\cite{TLCT_THCS_Toan_6_so_hoc}, 9.7., p. 58]
	Cho $x,y\in\mathbb{N},1\le y < x\le30$. (a) Tìm {\rm GTLN} của phân số $\dfrac{x + y}{x - y}$. (b) Tìm {\rm GTLN} của phân số $\dfrac{xy}{x - y}$.
\end{baitoan}

%------------------------------------------------------------------------------%

\section{Calculus of Fraction -- Tính Toán với Phân Số}

\begin{baitoan}[\cite{TLCT_THCS_Toan_6_so_hoc}, VD10.1, p. 59]
	Tìm $a,b\in\mathbb{N}$ thỏa $\dfrac{a}{5} + \dfrac{b}{3} = \dfrac{13}{15}$.
\end{baitoan}

\begin{baitoan}[\cite{TLCT_THCS_Toan_6_so_hoc}, VD10.2, p. 59]
	(a) Cho trước phân số $\dfrac{a}{b}\ne-1$. Tìm phân số $\dfrac{c}{d}$ thỏa $\dfrac{a}{b} - \dfrac{c}{d} = \dfrac{a}{b}\cdot\dfrac{c}{d}$. (b) Tìm phân số $\dfrac{c}{d}$ có tính chất trên, nếu phân số $\dfrac{a}{b}$ bằng $\dfrac{1}{3},\dfrac{3}{5}$.
\end{baitoan}

\begin{baitoan}[\cite{TLCT_THCS_Toan_6_so_hoc}, VD10.3, p. 60]
	Tính $A = \dfrac{1}{2} + \dfrac{5}{6} + \dfrac{11}{12} + \dfrac{19}{20} + \dfrac{29}{30} + \dfrac{41}{42} + \dfrac{55}{56} + \dfrac{71}{72} + \dfrac{89}{90}$.
\end{baitoan}

\begin{baitoan}[\cite{TLCT_THCS_Toan_6_so_hoc}, VD10.4, p. 60]
	Cho $A = \dfrac{1}{4}\cdot\dfrac{3}{6}\cdot\dfrac{5}{8}\cdots\dfrac{43}{46}\cdot\dfrac{45}{48},B = \dfrac{2}{5}\cdot\dfrac{4}{7}\cdot\dfrac{6}{9}\cdots\dfrac{44}{47}\cdot\dfrac{46}{49}$. (a) So sánh $A,B$. (b) Chứng minh $A < \dfrac{1}{133}$.
\end{baitoan}

\begin{baitoan}[\cite{TLCT_THCS_Toan_6_so_hoc}, VD10.5, p. 61]
	Tìm phân số $\dfrac{a}{b}$ lớn nhất sao cho khi chia mỗi phân số $\dfrac{12}{35},\dfrac{8}{21},\dfrac{52}{91}$ cho $\dfrac{a}{b}$, ta đều được các số tự nhiên.
\end{baitoan}

\begin{baitoan}[\cite{TLCT_THCS_Toan_6_so_hoc}, VD10.6, p. 62]
	Tìm $n\in\mathbb{Z}$ để phân số $\dfrac{20n + 13}{4n + 3}$ có {\rm GTNN}.
\end{baitoan}

\begin{baitoan}[\cite{TLCT_THCS_Toan_6_so_hoc}, VD10.7, p. 62]
	Tìm $x,y\in\mathbb{N}$ thỏa $\dfrac{1}{x} + \dfrac{1}{y} = \dfrac{1}{8}$.
\end{baitoan}

\begin{baitoan}[\cite{TLCT_THCS_Toan_6_so_hoc}, VD10.8, p. 63]
	Viết tiếp các phân số vào dãy các phân số có quy luật: $3,4\dfrac{1}{2},6\dfrac{3}{4},10\dfrac{1}{8},15\dfrac{3}{16}$.
\end{baitoan}

\begin{baitoan}[\cite{TLCT_THCS_Toan_6_so_hoc}, VD10.9, p. 63]
	Thay $a,b$ bởi 2 chữ số thích hợp: $\overline{0.ab}\cdot(a + b) = 0.36$.
\end{baitoan}

\begin{baitoan}[\cite{TLCT_THCS_Toan_6_so_hoc}, 10.1., p. 64]
	Tìm phân số $\dfrac{a}{b}$, biết nó bằng trung bình cộng của 3 phân số $\dfrac{7}{18},\dfrac{11}{18},dfrac{a}{b}$.
\end{baitoan}

\begin{baitoan}[\cite{TLCT_THCS_Toan_6_so_hoc}, 10.2., p. 64]
	Chứng minh phân số có thể viết được dưới dạng tổng của 2 phân số có tử bằng $1$, mẫu khác nhau: (a) $\dfrac{7}{10}$. (b) $\dfrac{2}{3}$.
\end{baitoan}

\begin{baitoan}[\cite{TLCT_THCS_Toan_6_so_hoc}, 10.3., p. 64]
	Chứng minh phân số có thể viết được dưới dạng tổng của 3 phân số có tử bằng $1$, mẫu khác nhau: (a) $\dfrac{17}{18}$. (b) $\dfrac{5}{8}$.
\end{baitoan}

\begin{baitoan}[\cite{TLCT_THCS_Toan_6_so_hoc}, 10.4., p. 64]
	Tìm $x,y\in\mathbb{Z}$ thỏa: (a) $\dfrac{x}{10} - \dfrac{1}{y} = \dfrac{3}{10}$. (b) $\dfrac{1}{x} + \dfrac{y}{2} = \dfrac{5}{8}$. 
\end{baitoan}

\begin{baitoan}[\cite{TLCT_THCS_Toan_6_so_hoc}, 10.5., p. 64]
	Cho phân số $\dfrac{a}{b}\ne1$. Tìm phân số $\dfrac{c}{d}$ thỏa $\dfrac{a}{b} + \dfrac{c}{d} = \dfrac{a}{b}\cdot\dfrac{c}{d}$.
\end{baitoan}

\begin{baitoan}[\cite{TLCT_THCS_Toan_6_so_hoc}, 10.6., p. 64]
	Tính $\dfrac{3 - \dfrac{3}{20} + \dfrac{3}{13} - \dfrac{3}{2013}}{7 - \dfrac{7}{20} + \dfrac{7}{13} - \dfrac{7}{2013}}$.
\end{baitoan}

\begin{baitoan}[\cite{TLCT_THCS_Toan_6_so_hoc}, 10.7., p. 64]
	Tính $\dfrac{1}{1\cdot5} + \dfrac{1}{5\cdot9} + \dfrac{1}{9\cdot13} + \dfrac{1}{13\cdot17} + \cdots + \dfrac{1}{41\cdot45}$.
\end{baitoan}

\begin{baitoan}[\cite{TLCT_THCS_Toan_6_so_hoc}, 10.8., p. 65]
	Cho $A = \dfrac{1}{31} + \dfrac{1}{32} + \dfrac{1}{33} + \cdots + \dfrac{1}{60}$. Chứng minh $A > \dfrac{7}{12}$.
\end{baitoan}

\begin{baitoan}[\cite{TLCT_THCS_Toan_6_so_hoc}, 10.9., p. 65]
	Cho $A = \sum_{i=3}^{50} \dfrac{1}{i^2} = \dfrac{1}{3^2} + \dfrac{1}{4^2} + \cdots + \dfrac{1}{50^2}$. Chứng minh: (a) $A > \dfrac{1}{4}$. (b) $A < \dfrac{4}{9}$.
\end{baitoan}

\begin{baitoan}[\cite{TLCT_THCS_Toan_6_so_hoc}, 10.10., p. 65]
	Tính $\dfrac{3}{4}\cdot\dfrac{8}{9}\cdot\dfrac{15}{16}\cdot\dfrac{24}{25}\cdot\dfrac{35}{36}\cdot\dfrac{48}{49}\cdot\dfrac{63}{64}$.
\end{baitoan}

\begin{baitoan}[\cite{TLCT_THCS_Toan_6_so_hoc}, 10.11., p. 65]
	Cho $A = \dfrac{1}{2}\cdot\dfrac{3}{4}\cdot\dfrac{5}{6}\cdot\dfrac{7}{8}\cdots\dfrac{79}{80}$. Chứng minh $A < \dfrac{1}{9}$.
\end{baitoan}

\begin{baitoan}[\cite{TLCT_THCS_Toan_6_so_hoc}, 10.12., p. 65]
	Chứng minh $1\cdot3\cdot5\cdots19 = \dfrac{11}{2}\cdot\dfrac{12}{2}\cdot\dfrac{13}{2}\cdots\dfrac{20}{2}\cdot$.
\end{baitoan}

\begin{baitoan}[\cite{TLCT_THCS_Toan_6_so_hoc}, 10.13., p. 65]
	Chứng minh $1 - \dfrac{1}{2} + \dfrac{1}{3} - \dfrac{1}{4} + \dfrac{1}{5} - \dfrac{1}{6} + \cdots + \dfrac{1}{20} = \frac{1}{11} + \dfrac{1}{12} + \dfrac{1}{13} + \cdots + \dfrac{1}{20}$.
\end{baitoan}

\begin{baitoan}[\cite{TLCT_THCS_Toan_6_so_hoc}, 10.14., p. 65]
	Tính $\dfrac{\dfrac{1}{19} + \dfrac{2}{18} + \dfrac{3}{17} + \cdots + \dfrac{18}{2} + \dfrac{19}{1}}{\dfrac{1}{2} + \dfrac{1}{3} + \dfrac{1}{4} + \cdots + \dfrac{1}{19} + \dfrac{1}{20}}$.
\end{baitoan}

\begin{baitoan}[\cite{TLCT_THCS_Toan_6_so_hoc}, 10.15., p. 65]
	Tính: (a) $A = \sum_{i=1}^9 \dfrac{1}{2^i} = \dfrac{1}{2} + \dfrac{1}{2^2} + \dfrac{1}{2^3} + \cdots + \dfrac{1}{2^9}$. (b) $B = \dfrac{1}{4} + \dfrac{1}{12} + \dfrac{1}{36} + \dfrac{1}{108} + \dfrac{1}{324} + \dfrac{1}{972}$.
\end{baitoan}

\begin{baitoan}[\cite{TLCT_THCS_Toan_6_so_hoc}, 10.16., p. 65]
	Tìm $a,b\in\mathbb{Q}$ thỏa $a + b = 3(a - b) = 2\cdot\dfrac{a}{b}$.
\end{baitoan}

\begin{baitoan}[\cite{TLCT_THCS_Toan_6_so_hoc}, 10.17., p. 65]
	Cho $\dfrac{a}{b} = \sum_{i=2}^9 \dfrac{1}{i} = \dfrac{1}{2} + \dfrac{1}{3} + \cdots + \dfrac{1}{9}$. Chứng minh $a\divby11$.
\end{baitoan}

\begin{baitoan}[\cite{TLCT_THCS_Toan_6_so_hoc}, 10.18., p. 66]
	Chứng minh $\sum_{i=2}^{50} \dfrac{1}{i} = \dfrac{1}{2} + \dfrac{1}{3} + \cdots + \dfrac{1}{50}\notin\mathbb{N}$.
\end{baitoan}

\begin{baitoan}[\cite{TLCT_THCS_Toan_6_so_hoc}, 10.19., p. 66]
	Tìm $a\in\mathbb{N}$ nhỏ nhất, biết nhân $a$ với $\dfrac{8}{15}$ hoặc $\dfrac{21}{36}$ thì 2 kết quả đều là số tự nhiên.
\end{baitoan}

\begin{baitoan}[\cite{TLCT_THCS_Toan_6_so_hoc}, 10.20., p. 66]
	So sánh $A = \dfrac{8^9 + 12}{8^9 + 7},B = \dfrac{8^{10} + 4}{8^{10} - 1}$.
\end{baitoan}

\begin{baitoan}[\cite{TLCT_THCS_Toan_6_so_hoc}, 10.21., p. 66]
	Tìm $n\in\mathbb{Z}$ để phân số $\dfrac{4n + 9}{2n + 3}$ có {\rm GTLN}.
\end{baitoan}

\begin{baitoan}[\cite{TLCT_THCS_Toan_6_so_hoc}, 10.22., p. 66]
	Tìm số tự nhiên có 2 chữ số sao cho tỷ số của số đó \& tổng các chữ số của nó có {\rm GTNN}.
\end{baitoan}

\begin{baitoan}[\cite{TLCT_THCS_Toan_6_so_hoc}, 10.23., p. 66]
	Tìm tỷ số lớn nhất của số tự nhiên có 3 chữ số \& tổng các chữ số của nó.
\end{baitoan}

\begin{baitoan}[\cite{TLCT_THCS_Toan_6_so_hoc}, 10.24., p. 66]
	So sánh $A = \dfrac{7^{10}}{1 + 7 + 7^2 + \cdots + 7^9},B = \dfrac{5^{10}}{1 + 5 + 5^2 + \cdots + 5^9}$.
\end{baitoan}

\begin{baitoan}[\cite{TLCT_THCS_Toan_6_so_hoc}, 10.25., p. 66]
	Tìm 3 số nguyên dương khác nhau sao cho tổng các nghịch đảo của chúng bằng $1$.
\end{baitoan}

%------------------------------------------------------------------------------%

\section{Các Bài Toán Về Phân Số \& Tỷ Số}

\begin{baitoan}[\cite{TLCT_THCS_Toan_6_so_hoc}, VD11.1, pp. 66--67]
	Tâm đã có 1 số điểm kiểm tra Toán \& còn 1 bài kiểm tra nữa, các bài kiểm tra đều tính hệ số 1 ngang nhau. Nếu bài kiểm tra này Tâm được $10$ điểm thì Tâm đạt điểm trung bình là $9$. Nhưng vì trong bài kiểm tra cuối, Tâm chỉ được $7.5$ điểm (điểm không làm tròn thành 1 số nguyên) nên điểm trung bình của Tâm chỉ là $8.5$. Hỏi Tâm có tất cả bao nhiêu bài kiểm tra?
\end{baitoan}

\begin{baitoan}[\cite{TLCT_THCS_Toan_6_so_hoc}, VD11.2, p. 67]
	Có $20$ viên bi đỏ, $30$ viên bi trắng, \& 1 số viên bi xanh, tất cả để trong hộp. Nếu lấy ra trong hộp 1 viên bi thì cơ hội có thể lấy được 1 viên bi xanh là $\dfrac{9}{11}$. Tính số bi xanh.
\end{baitoan}

\begin{baitoan}[\cite{TLCT_THCS_Toan_6_so_hoc}, VD11.3, p. 67]
	1 lớp học mua 1 số vở về chia đều cho các học sinh. Nếu chỉ chia cho các bạn nữ thì mỗi bạn nhận $15$ quyển. Nếu chỉ chia cho các bạn nam thì mỗi bạn nhận $10$ quyển. Hỏi nếu chia đều cho tất cả các bạn trong lớp thì mỗi bạn nhận được bao nhiêu quyển vở?
\end{baitoan}

\begin{baitoan}[\cite{TLCT_THCS_Toan_6_so_hoc}, VD11.4, p. 68]
	4 bạn An, Bách, Cảnh, Dũng đi chơi, nhưng Dũng không mang tiền. An cho Dũng $\dfrac{1}{5}$ số tiền của mình. Bách cho Dũng $\dfrac{1}{4}$ số tiền của mình. Cảnh cho Dũng $\dfrac{1}{3}$ số tiền của mình. Kết quả số tiền Dũng nhận được từ 3 bạn đều bằng nhau. Hỏi cuối cùng Dũng có số tiền bằng mấy phần tổng số tiền của cả nhóm?
\end{baitoan}

\begin{baitoan}[\cite{TLCT_THCS_Toan_6_so_hoc}, 11.1., p. 69]
	Tổng của 3 số bằng $148$. Nếu nhân số thứ nhất với $4$, nhân số thứ 2 với $5$, nhân số thứ 3 với $6$ thì được 3 tích bằng nhau. Tính mỗi số.
\end{baitoan}

\begin{baitoan}[\cite{TLCT_THCS_Toan_6_so_hoc}, 11.2., p. 69]
	Tổng của 3 số bằng $147$. Biết $\dfrac{2}{3}$ số thứ nhất bằng $\dfrac{3}{4}$ số thứ 2 \& bằng $\dfrac{4}{5}$ số thứ 3. Tính mỗi số.
\end{baitoan}

\begin{baitoan}[\cite{TLCT_THCS_Toan_6_so_hoc}, 11.3., p. 69]
	Trong 1 buổi đi tham quan, số nữ đăng ký tham gia bằng $\dfrac{1}{4}$ số nam. Nhưng sau đó 1 bạn nữ xin nghỉ, 1 bạn nam xin đi thêm nên số nữ đi tham quan bằng $\dfrac{1}{5}$ số nam. Tính số học sinh nữ \& nam đã đi tham quan.
\end{baitoan}

\begin{baitoan}[\cite{TLCT_THCS_Toan_6_so_hoc}, 11.4., p. 69]
	Tú có 2 ngăn sách. Số sách ở ngăn I bằng $\dfrac{2}{5}$ tổng số sách ở 2 ngăn. Tú cho bạn mượn $4$ quyển sách ở ngăn I nên số sách ở ngăn I bằng $\dfrac{1}{3}$ tổng số sách ở 2 ngăn. Tính tổng số sách ở 2 ngăn lúc đầu.
\end{baitoan}

\begin{baitoan}[\cite{TLCT_THCS_Toan_6_so_hoc}, 11.5., p. 70]
	Hiện nay, tuổi mẹ gấp 3 tuổi con. Cách đây 4 năm, tuổi mẹ gấp 4 tuổi con. Tính tuổi mỗi người hiện nay.
\end{baitoan}

\begin{baitoan}[\cite{TLCT_THCS_Toan_6_so_hoc}, 11.6., p. 70]
	2 máy cày làm việc trên 1 cánh đồng. Nếu cả 2 máy cùng cày thì {\rm10 h} xong công việc. Nhưng thực tế 2 máy chỉ cùng làm việc {\rm7 h} đầu, sau đó máy thứ nhất đi cày nơi khác, máy thứu 2 làm tiếp {\rm9 h} nữa mới xong. Hỏi nếu máy thứ 2 làm việc 1 mình thì trong bao lâu cày xong cánh đồng?
\end{baitoan}

\begin{baitoan}[\cite{TLCT_THCS_Toan_6_so_hoc}, 11.7., p. 70]
	3 vòi nước I, II, III nếu chảy 1 mình vào 1 bể cạn thì chảy đầy bể lần lượt trong {\rm4 h, 6 h, 9 h}. Lúc đầu, mở 2 vòi I \& II trong {\rm1 h 30 ph}, sau đó đóng vòi I rồi mở tiếp vòi III cùng chảy với vòi II cho đến khi đầy bể. Hỏi vòi III chảy trong bao lâu?
\end{baitoan}

\begin{baitoan}[\cite{TLCT_THCS_Toan_6_so_hoc}, 11.8., p. 70]
	Có 3 vòi nước chảy vào 1 bể cạn. Nếu 2 vòi I \& II cùng chảy thì bể đầy sau {\rm45 ph}. Nếu 2 vòi II \& III cùng chảy thì bể đầy sau {\rm1 h}. Nếu 2 vòi I \& III cùng chảy thì bể đầy sau {\rm36 ph}. (a) Nếu cả 3 vòi cùng chảy thì bể đầy trong bao lâu? (b) Riêng mỗi vòi chảy 1 mình thì bể đầy trong bao lâu?
\end{baitoan}

\begin{baitoan}[\cite{TLCT_THCS_Toan_6_so_hoc}, 11.9., p. 70]
	3 người đến cửa hàng mua 1 số táo. Người I mua $\dfrac{1}{2}$ số táo rồi mua thêm $\dfrac{1}{2}$ quả. Người II mua $\dfrac{2}{3}$ số còn lại rồi mua thêm $\dfrac{2}{3}$ quả. Người III mua $\dfrac{3}{4}$ số còn lại rồi mua thêm $\dfrac{3}{4}$ quả thì vừa hết số táo của cửa hàng. Tính số táo của cửa hàng có lúc đầu.
\end{baitoan}

\begin{baitoan}[\cite{TLCT_THCS_Toan_6_so_hoc}, 11.10., p. 70]
	1 số học sinh được thưởng 1 số vở. Bạn I được thưởng $2$ quyển vở \& $\dfrac{1}{5}$ số còn lại. Bạn II được thưởng $4$ quyển vở \& $\dfrac{1}{5}$ số còn lại. Bạn III được thưởng $6$ quyển vở \& $\dfrac{1}{5}$ số còn lại. $\ldots$ Cứ như vậy thì số vở được chia đều cho các bạn \& không còn thừa quyển nào. Tính số học sinh được thưởng \& số vở.
\end{baitoan}

%------------------------------------------------------------------------------%

\section{Percentage -- Phần Trăm $\%$}

\begin{baitoan}[\cite{TLCT_THCS_Toan_6_so_hoc}, VD12.1, p. 71]
	1 cửa hàng có 2 loại quạt, giá tiền như nhau. Quạt màu vàng được giảm giá 2 lần, mỗi lần giảm giá $10\%$. Quạt màu xanh được giảm giá 1 lần $20\%$. Hỏi sau khi giảm giá như trên thì loại quạt nào rẻ hơn?
\end{baitoan}

\begin{baitoan}[\cite{TLCT_THCS_Toan_6_so_hoc}, VD12.2, p. 71]
	(a) Chiều dài 1 hình chữ nhật tăng $25\%$. Chiều rộng hình chữ nhật phải giảm bao nhiêu $\%$ để chu vi hình chữ nhật không đổi, biết chiều dài gấp đôi chiều rộng? (b) Chiều dài 1 hình chữ nhật tăng $25\%$. Chiều rộng hình chữ nhật phải giảm bao nhiêu $\%$ để diện tích hình chữ nhật không đổi?
\end{baitoan}

\begin{baitoan}[\cite{TLCT_THCS_Toan_6_so_hoc}, VD12.3, p. 72]
	1 quả dưa hấu có khối lượng {\rm1000 g} chứa $93\%$ nước. 1 tuần sau, lượng nước chỉ còn $90\%$. Khi đó, khối lượng quả dưa hấu còn bao nhiêu {\rm g}?
\end{baitoan}

\begin{baitoan}[\cite{TLCT_THCS_Toan_6_so_hoc}, VD12.4, p. 73]
	1 cửa hàng trong ngày khai trương hạ giá hàng $12\%$ so với giá bán trong ngày thường. Tuy vậy, cửa hàng vẫn lãi $10\%$ so với giá gốc. Hỏi nếu không hạ giá thì cửa hàng lãi bao nhiêu $\%$ so với giá gốc?
\end{baitoan}

\begin{baitoan}[\cite{TLCT_THCS_Toan_6_so_hoc}, 12.1., p. 73]
	Phân số $\dfrac{1}{2}$ tăng thành $\dfrac{7}{8}$ thì giá trị của phân số đó tăng thêm bao nhiêu $\%$?
\end{baitoan}

\begin{baitoan}[\cite{TLCT_THCS_Toan_6_so_hoc}, 12.2., p. 73]
	1 xí nghiệp có khối lượng công việc tăng thêm $40\%$, còn năng suất lao động của công nhân tăng thêm $25\%$. Hỏi số công nhân cần tăng thêm bao nhiêu $\%$?
\end{baitoan}

\begin{baitoan}[\cite{TLCT_THCS_Toan_6_so_hoc}, 12.3., p. 73]
	Giá lúa tăng $25\%$. Hỏi giá lúa phải giảm bao nhiêu $\%$ để trở lại giá cũ?
\end{baitoan}

\begin{baitoan}[\cite{TLCT_THCS_Toan_6_so_hoc}, 12.4., p. 73]
	Giá rau tháng 4 cao hơn so với tháng 3 là $10\%$. Giá rau tháng 5 thấp hơn so với tháng 4 là $10\%$. Hỏi giá rau tháng 5 so với tháng 3 bằng bao nhiêu $\%$?
\end{baitoan}

\begin{baitoan}[\cite{TLCT_THCS_Toan_6_so_hoc}, 12.5., p. 73]
	1 cửa hàng nhập 1 loại đồ chơi, rồi định giá bán là {\rm50000 đ{\tt/}chiếc}. Trong ngày Tết thiếu nhi 1.6, cửa hàng hạ giá $12\%$, tính ra so với giá nhập vào vẫn lãi $10\%$. (a) Tính giá nhập của đồ chơi ấy. (b) So với giá nhập, thì giá bán trong ngày thường lãi bao nhiêu $\%$?
\end{baitoan}

\begin{baitoan}[\cite{TLCT_THCS_Toan_6_so_hoc}, 12.6., p. 74]
	1 cửa hàng bán quần áo thanh lý hàng nên đã giảm $10\%$ so với giá bình thường, nhưng không bán được nên giảm tiếp $10\%$ nữa (so với giá đã giảm) \& đã bán hết hàng. Tính ra cửa hàng vẫn lãi $5.4\%$ so với giá gốc. Hỏi giá bình thường bằng bao nhiêu $\%$ giá gốc?
\end{baitoan}

\begin{baitoan}[\cite{TLCT_THCS_Toan_6_so_hoc}, 12.7., p. 74]
	1 hãng điện thoại có 3 phương án trả tiền cước điện thoại: Phương án I: Trả $99$ xu cho $20$ phút đầu, sau đó từ phút thứ $21$ thì mỗi phút trả thêm $5$ xu. Phương án II: Kể từ lúc đầu tiên, mỗi phút trả $10$ xu. Phương án III: Trả $25$ xu, sau đó kể từ phút đầu tiên mỗi phút trả $8$ xu. 1 khách hàng trong tháng có $10\%$ cuộc gọi $1$ phút, $10\%$ cuộc gọi $5$ phút, $30\%$ cuộc gọi $10$ phút, $30\%$ cuộc gọi $20$ phút, $20\%$ cuộc gọi $30$ phút. Người đó nên chọn phương án nào để tiền cước ít nhất?
\end{baitoan}

\begin{baitoan}[\cite{TLCT_THCS_Toan_6_so_hoc}, 12.8., p. 74]
	1 cửa hàng sách hạ giá $10\%$ trong ngày lễ, tuy vậy cửa hàng vẫn còn lãi $8\%$. Hỏi trong ngày thường cửa hàng lãi bao nhiêu $\%$?
\end{baitoan}

\begin{baitoan}[\cite{TLCT_THCS_Toan_6_so_hoc}, 12.9., p. 74]
	Nước biển chứa $5\%$ muối. Cần thêm bao nhiêu {\rm kg} nước lã vào {\rm20 kg} nước biển để tỷ lệ muối trong dung dịch là $2\%$?
\end{baitoan}

\begin{baitoan}[\cite{TLCT_THCS_Toan_6_so_hoc}, 12.10., p. 74]
	Ông Ngọc có {\rm500 kg} hạt cà phê tươi, đem phơi khô để tỷ lệ nước trong hạt cà phê còn $5\%$. Biết tỷ lệ nước trong hạt cà phê tươi là $24\%$. Tính khối lượng nước cần bay hơi.
\end{baitoan}

\begin{baitoan}[\cite{TLCT_THCS_Toan_6_so_hoc}, 12.11., p. 74]
	Phơi {\rm450 kg} hạt tươi thì được hạt khô. Biết tỷ lệ nước trong hạt tươi là $20\%$, tỷ lệ nước trong hạt khô là $10\%$. Tính khối lượng hạt khô.
\end{baitoan}

\begin{baitoan}[\cite{TLCT_THCS_Toan_6_so_hoc}, 12.12., p. 74]
	Chị Mai ngâm {\rm15 kg} hạt giống có tỷ lệ nước là $4\%$ vào 1 thùng nước. Chị muốn tỷ lệ nước trong hạt giống sau khi ngâm là $10\%$ để cho khả năng nảy mầm cao hơn. Tính khối lượng hạt giống sau khi ngâm.
\end{baitoan}

\begin{baitoan}[\cite{TLCT_THCS_Toan_6_so_hoc}, 12.13., p. 74]
	Phơi {\rm60 kg} cỏ tươi, sau 1 tuần thì còn {\rm30 kg} cỏ khô. Biết tỷ lệ nước trong cỏ tươi là $70\%$. Hỏi tỷ lệ nước trong cỏ khô là bao nhiêu $\%$?
\end{baitoan}

%------------------------------------------------------------------------------%

\section{Movement Problem -- Toán Chuyển Động}

\begin{baitoan}[\cite{TLCT_THCS_Toan_6_so_hoc}, VD13.1, p. 75]
	Lúc {\rm8:00}, người thứ nhất đi từ A \& đến B lúc {\rm12:00}. Lúc {\rm8:30}, người thứ 2 đi từ A \& đến B lúc {\rm11:30}. Hỏi người thứ 2 đuổi kịp người thứ nhất lúc mấy giờ?
\end{baitoan}

\begin{baitoan}[\cite{TLCT_THCS_Toan_6_so_hoc}, VD13.2, p. 77]
	Trên quãng đường AB, 2 xe cùng khởi hành 1 lúc, xe tải đi từ A đến B hết {\rm6 h}, xe con đi từ B đến A hết {\rm4 h}. 2 xe gặp nhau sau bao lâu?
\end{baitoan}

\begin{baitoan}[\cite{TLCT_THCS_Toan_6_so_hoc}, VD13.3, p. 77]
	1 người phải đi từ A đến B trong {\rm5 h}. Lúc đầu, người đó đi với vận tốc {\rm30 km{\tt/}h}. Khi còn {\rm75 km} nữa thì được nửa đường, người đó đi với vận tốc {\rm45 km{\tt/}h} để kịp B đúng dự định. Tính quãng đường AB.
\end{baitoan}

\begin{baitoan}[\cite{TLCT_THCS_Toan_6_so_hoc}, VD13.4, p. 78]
	Hòa bơi xuôi dòng nước từ A đến B hết {\rm6 ph}, còn bơi ngược từ B về A hết {\rm10 ph}. 1 cụm bèo trôi theo dòng nước từ A đến B trong bao lâu?
\end{baitoan}

\begin{baitoan}[\cite{TLCT_THCS_Toan_6_so_hoc}, VD13.5, p. 79]
	1 xe lửa chạy với vận tốc {\rm45 km{\tt/}h}. Xe lửa chui vào 1 đường hầm có chiều dài gấp $9$ lần chiều dài của xe lửa \& cần $2$ phút để xe lửa vào \& ra khỏi đường hầm. Tính chiều dài của xe lửa.
\end{baitoan}

\begin{baitoan}[\cite{TLCT_THCS_Toan_6_so_hoc}, VD13.6, p. 80]
	1 ôtô đi nửa đầu của quãng đường AB với vận tốc {\rm30 km{\tt/}h} \& đi nửa sau với vận tốc {\rm60 km{\tt/}h}. Tính vận tốc trung bình của ôtô trên cả quãng đường AB.
\end{baitoan}

\begin{baitoan}[\cite{TLCT_THCS_Toan_6_so_hoc}, 13.1., p. 80]
	1 chiếc tàu chạy trên sông, khi còn {\rm90 km} nữa mới cập bến thì tàu bị thủng, cứ {\rm5 ph} có $2$ tấn nước tràn vào tàu. Nếu có $105$ tấn nước tràn vào tàu thì tàu sẽ bị chìm. Trên tàu có 1 máy bơm, mỗi giờ bơm ra được $10$ tấn nước. Tàu phải chạy ít nhất với vận tốc nào để khi cập bến, tàu vẫn chưa bị chìm?
\end{baitoan}

\begin{baitoan}[\cite{TLCT_THCS_Toan_6_so_hoc}, 13.2., p. 80]
	Hiện nay là {\rm4:00}. Sau ít nhất bao lâu thì kim phút chập với kim giờ?
\end{baitoan}

\begin{baitoan}[\cite{TLCT_THCS_Toan_6_so_hoc}, 13.3., p. 80]
	Trên quãng đường AB, 2 ôtô khởi hành cùng 1 lúc, xe thứ nhất đi từ A đến B hết {\rm2 h}, xe thứ 2 đi từ B đến A hết {\rm3 h}. Đến chỗ gặp nhau, quãng đường xe thứ nhất đã đi nhiều hơn quãng đường xe thứ 2 đã đi là {\rm30 km}. Tính quãng đường AB.
\end{baitoan}

\begin{baitoan}[\cite{TLCT_THCS_Toan_6_so_hoc}, 13.4., p. 81]
	Trên quãng đường AB dài {\rm300 km}, người thứ nhất đi từ A đến B, người thứ 2 đi từ B đến A. Họ khởi hành cùng 1 lúc thì sẽ gặp nhau sau {\rm5 h}. Nhưng người thứ 2 có việc bận, đã khởi hành sau người thứ nhất {\rm40 ph}, do đó sau {\rm4 h 42 ph} mới gặp người thứ nhất. Tính vận tốc mỗi người.
\end{baitoan}

\begin{baitoan}[\cite{TLCT_THCS_Toan_6_so_hoc}, 13.5., p. 81]
	2 xe khởi hành từ A \& từ B cùng 1 lúc, đi ngược chiều \& sẽ gặp nhau sau {\rm6 h}. Vận tốc của xe con bằng $\dfrac{4}{3}$ vận tốc của xe tải. Muốn gặp nhau ở chính giữa quãng đường AB thì xe con phải đi sau xe tải bao lâu?
\end{baitoan}

\begin{baitoan}[\cite{TLCT_THCS_Toan_6_so_hoc}, 13.6., p. 81]
	1 người đi từ A đến B. Người đó tính: nếu đi với vận tốc {\rm40 km{\tt/}h} thì đến B sau giờ hẹn là {\rm20 ph}, còn nếu đi với vận tốc {\rm60 km{\tt/}h} thì đến B trước giờ hẹn là {\rm10 ph}. Tính quãng đường AB.
\end{baitoan}

\begin{baitoan}[\cite{TLCT_THCS_Toan_6_so_hoc}, 13.7., p. 81]
	3 người cùng khởi hành 1 lúc từ A để đến B, vận tốc người I, người II lần lượt là {\rm40 km{\tt/}h, 60 km{\tt/}h}. Người III đến B trước người I là {\rm18 ph} \& sau người II là {\rm12 ph}. Tính quãng đường AB \& vận tốc người III. 
\end{baitoan}

\begin{baitoan}[\cite{TLCT_THCS_Toan_6_so_hoc}, 13.8., p. 81]
	2 người cùng đi từ A về 1 phía, người I khởi hành lúc {\rm7:00}, người II khởi hành lúc {\rm7:30}. 2 người gặp nhau lúc mấy giờ, biết quãng đường người I đi trong {\rm30 ph} bằng quãng đường người II đi trong {\rm20 ph}?
\end{baitoan}

\begin{baitoan}[\cite{TLCT_THCS_Toan_6_so_hoc}, 13.9., p. 81]
	1 ôtô đi từ A đến B với vận tốc {\rm40 km{\tt/}h}. Trên đường đi từ B về A, sau khi đi $\dfrac{1}{3}$ quãng đường với vận tốc cũ, ôtô dừng lại chữa trong {\rm30 ph}. Muốn thời gian từ B về A vẫn bằng thời gian từ A đến B, ôtô phải đi tiếp với vận tốc {\rm60 km{\tt/}h}. Tính quãng đường AB.
\end{baitoan}

\begin{baitoan}[\cite{TLCT_THCS_Toan_6_so_hoc}, 13.10., p. 81]
	Thành đi xe đạp từ A đến B. Sau khi đi {\rm10 km} trong {\rm40 ph}, Thành tính: nếu tiếp tục đi với vận tốc như vậy thì đến B trước giờ hẹn {\rm24 ph}. Anh đã giảm vận tốc đi {\rm3 km{\tt/}h} mà vẫn đến B trước giờ hẹn {\rm10 ph}. Tính quãng đường AB.
\end{baitoan}

\begin{baitoan}[\cite{TLCT_THCS_Toan_6_so_hoc}, 13.11., p. 81]
	1 người đi từ A đến B với vận tốc {\rm40 km{\tt/}h}, rồi đi tiếp từ B đến D với vận tốc {\rm60 km{\tt/}h}. Quãng đường BD dài hơn AB là {\rm10 km}. Thời gian đi BD ít hơn đi AB là {\rm20 ph}. Tính 2 quãng đường $AB,BD$.
\end{baitoan}

\begin{baitoan}[\cite{TLCT_THCS_Toan_6_so_hoc}, 13.12., p. 82]
	1 canô xuôi khúc sông từ A đến B hết {\rm3 h} \& ngược khúc sông đó hết {\rm4.5 h}. Biết vận tốc dòng nước là {\rm3 km{\tt/}h}. Tính vận tốc xuôi, vận tốc ngược, \& chiều dài khúc sông AB.
\end{baitoan}

\begin{baitoan}[\cite{TLCT_THCS_Toan_6_so_hoc}, 13.13., p. 82]
	Thắng đi xe máy với vận tốc {\rm36 km{\tt/}h}. Anh gặp 1 xe lửa dài {\rm75 m} đi cùng chiều chạy song song bên cạnh mình trong {\rm15 s}. Tính vận tốc xe lửa với đơn vị {\rm m{\tt/}s}.
\end{baitoan}

\begin{baitoan}[\cite{TLCT_THCS_Toan_6_so_hoc}, 13.14., p. 82]
	1 người đi từ A đến B. Người đó đi $\dfrac{1}{3}$ quãng đường với vận tốc {\rm20 km{\tt/}h} rồi đi phần còn lại với vận tốc {\rm10 km{\tt/}h}. HTính vận tốc trung bình của người đó trên cả quãng đường AB.
\end{baitoan}

\begin{baitoan}[\cite{TLCT_THCS_Toan_6_so_hoc}, 13.15., p. 82]
	Lúc {\rm6:00}, 1 xe tải \& 1 xe máy cùng xuất phát từ A để đến B. Vận tốc xe tải là {\rm50 km{\tt/}h}, vận tốc xe máy là {\rm30 km{\tt/}h}. Sau đó {\rm2 h}, 1 xe con cũng đi từ A để đến B, vận tốc {\rm60 km{\tt/}h}. Đến mấy giờ thì xe con ở chính giữa xe máy \& xe tải?
\end{baitoan}

\begin{baitoan}[\cite{TLCT_THCS_Toan_6_so_hoc}, 13.16., p. 82]
	2 xe bus cùng khởi hành 1 lúc với vận tốc không đổi, xe thứ nhất đi từ A đến B, xe thứ 2 đi từ B đến A. Xe thứ nhất đến B thì quay lại ngay, xe thứ 2 đến A thì quay lại ngay. 2 xe gặp nhau lần thứ nhất tại C cách A là {\rm5 km} \& gặp nhau lần thứ 2 tại D cách B là {\rm4 km}. Tính quãng đường AB.
\end{baitoan}

%------------------------------------------------------------------------------%

\section{Hệ Ghi Số Với Cơ Số Bất Kỳ}

\begin{baitoan}[\cite{TLCT_THCS_Toan_6_so_hoc}, VD14.1, p. 83]
	Đổi số $1304_{(5)}$ thành số viết trong hệ thập phân.
\end{baitoan}

\begin{baitoan}[\cite{TLCT_THCS_Toan_6_so_hoc}, VD14.2, p. 83]
	Đổi số $204$ trong hệ thập phân thành số viết trong hệ cơ số $5$.
\end{baitoan}

\begin{baitoan}[\cite{TLCT_THCS_Toan_6_so_hoc}, VD14.3, p. 84]
	1 đội quân được tổ chức theo nguyên tắc ``tam tam chế'', i.e., cứ 3 lính thì lập thành 1 tổ, cứ 3 tổ thì lập thành 1 tiểu đội, cứ 3 tiểu đội thì lập thành 1 trung đội, $\ldots$ Các cấp từ nhỏ đến lớn là tổ, tiểu đội, trung đội, đại đội, tiểu đoàn, $\ldots$ Có $422$ lính thì lập được thành các cấp nào?
\end{baitoan}

\begin{baitoan}[\cite{TLCT_THCS_Toan_6_so_hoc}, VD14.4, p. 85]
	Tính trong hệ cơ số $4$: (a) $321 + 103$. (b) $123 - 31$.
\end{baitoan}

\begin{baitoan}[\cite{TLCT_THCS_Toan_6_so_hoc}, VD14.5, p. 85]
	Đổi số $1101_{(2)}$ thành số viết trong hệ thập phân.
\end{baitoan}

\begin{baitoan}[\cite{TLCT_THCS_Toan_6_so_hoc}, VD14.6, p. 85]
	Đổi số $43$ thành số viết trong hệ nhị phân.
\end{baitoan}

\begin{baitoan}[\cite{TLCT_THCS_Toan_6_so_hoc}, VD14.7, p. 86]
	Mai có $31$ chiếc tem đựng trong $5$ phong bì. Muốn lấy ra 1 số tem bất kỳ từ $1$ đến $30$, Mai chỉ cần lấy ra 1 số phong bì là có đúng số tem định lấy. Tính số chiếc tem trong mỗi phong bì.
\end{baitoan}

\begin{baitoan}[\cite{TLCT_THCS_Toan_6_so_hoc}, VD14.8, p. 86]
	Tính trong hệ nhị phân: (a) $1101 + 110$. (b) $1101 - 110$. (c) $1101\cdot11$.
\end{baitoan}

\begin{baitoan}[\cite{TLCT_THCS_Toan_6_so_hoc}, 14.1., p. 87]
	Có bao nhiêu đơn vị trong số lớn nhất có 1 chữ số được viết trong: (a) Hệ thập phân? (b) Hệ cơ số $7$?
\end{baitoan}

\begin{baitoan}[\cite{TLCT_THCS_Toan_6_so_hoc}, 14.2., p. 87]
	Có bao nhiêu đơn vị trong số lớn nhất có 2 chữ số được viết trong: (a) Hệ thập phân? (b) Hệ cơ số $6$?
\end{baitoan}

\begin{baitoan}[\cite{TLCT_THCS_Toan_6_so_hoc}, 14.3., p. 87]
	Nếu viết thêm vào bên phải số sau 1 chữ số $0$ thì nó tăng gấp mấy lần? (a) $41$. (b) $35_{(8)}$.
\end{baitoan}

\begin{baitoan}[\cite{TLCT_THCS_Toan_6_so_hoc}, 14.4., p. 87]
	Tính trong hệ cơ số $6$: (a) $132 + 241$. (b) $553 - 315$.
\end{baitoan}

\begin{baitoan}[\cite{TLCT_THCS_Toan_6_so_hoc}, 14.5., p. 87]
	Tính tỏng hệ nhị phân: (a) $1001 + 11$. (b) $10010 - 1011$. (c) $101\cdot101$.
\end{baitoan}

\begin{baitoan}[\cite{TLCT_THCS_Toan_6_so_hoc}, 14.6., p. 88]
	Trong hệ cơ số nào, có: (a) $3 + 4 = 10$? (b) $3 + 2 = 11$? (c) $2\cdot3 = 10$?
\end{baitoan}

\begin{baitoan}[\cite{TLCT_THCS_Toan_6_so_hoc}, 14.7., p. 88]
	Tìm $x$ thỏa: (a) $43_{(x)} = 31$. (b) $23_(4) = x_{(3)}$. (c) $145_{(x)} = 145\cdot2$.
\end{baitoan}

\begin{baitoan}[\cite{TLCT_THCS_Toan_6_so_hoc}, 14.8., p. 88]
	Để cân các vật có khối lượng là 1 số tự nhiên từ {\rm1 kg--29 kg}, có thể dùng 1 cân đĩa có 2 đĩa cân với $5$ quả cân, các quả cân chỉ để ở 1 đĩa cân. $5$ quả cân đó có khối lượng bao nhiêu?
\end{baitoan}

\begin{baitoan}[\cite{TLCT_THCS_Toan_6_so_hoc}, 14.9., p. 88]
	Cho $A = \sum_{i=0}^{20} 2^i = 1 + 2 + 2^2 + 2^3 + \cdots + 2^{20}$. Viết $A + 1$ dưới dạng 1 lũy thừa.
\end{baitoan}

\begin{baitoan}[\cite{TLCT_THCS_Toan_6_so_hoc}, 14.11., p. 88]
	Chứng minh trong mọi hệ cơ số: (a) Số $100$ là số chính phương. (b) Số $110$ là hợp số.
\end{baitoan}

\begin{baitoan}[\cite{TLCT_THCS_Toan_6_so_hoc}, 14.12., p. 88]
	Chứng minh trong hệ cơ số $4$: $\overline{abc}_{(4)}\divby3\Leftrightarrow a + b + c\divby3$.
\end{baitoan}

\begin{baitoan}[\cite{TLCT_THCS_Toan_6_so_hoc}, 14.13., p. 88]
	Tìm số $\overline{abc}$ viết trong hệ thập phân bằng số $\overline{cba}$ viết trong hệ cơ số $9$.
\end{baitoan}

%------------------------------------------------------------------------------%

\section{Dãy Số Viết Theo Quy Luật}

\begin{baitoan}[\cite{TLCT_THCS_Toan_6_so_hoc}, VD15.1, p. 90]
	Viết thêm 3 số hạng rồi tìm công thức của số hạng tổng quát của dãy số: (a) $1,4,7,10,\ldots$ (b) $1,2,3,5,8,\ldots$ (c) $2,12,30,56,90,\ldots$ (d) $-1,-2,-6,-24,-120,\ldots$ (e) $1,6,15,28,45,\ldots$ (f) $4,18,40,70,108,\ldots$
\end{baitoan}

\begin{baitoan}[\cite{TLCT_THCS_Toan_6_so_hoc}, VD15.2, p. 91]
	Tính $A = \sum_{i=1}^n i = 1 + 2 + \cdot + n$.
\end{baitoan}

\begin{baitoan}[\cite{TLCT_THCS_Toan_6_so_hoc}, VD15.3, p. 91]
	(a) Tính bằng nhiều cách $A = 2 + 4 + 6 + \cdots + 96 + 98$. (b) Tính $B = 1 - 2 + 3 - 4 + \cdots + 2011 - 2012 + 2013$. (c) Tìm $n\in\mathbb{N}^\star$ thỏa $3 + 4 + 5 + \cdots + n = 525$. (d) Tìm $n\in\mathbb{N}^\star$ \& chữ số $a$ biết $\sum_{i=1}^n i = 1 + 2 + \cdot + n = \overline{aaa}$.
\end{baitoan}

\begin{baitoan}[\cite{TLCT_THCS_Toan_6_so_hoc}, VD15.4, p. 92]
	Cho dãy số $5,9,13,17,21,\ldots$ (a) Nhận xét dãy số trên \& tìm số hạng thứ $10$, số hạng thứ $n$. (b) 2 số $12345,2011$ có mặt trong dãy số đó không? là số thứ bao nhiêu của dãy? (c) Tìm tổng $100$ số đầu tiên của dãy số đó.
\end{baitoan}

\begin{baitoan}[\cite{TLCT_THCS_Toan_6_so_hoc}, VD15.5, p. 93]
	Cho dãy số tự nhiên chẵn liên tiếp $2,4,6,\ldots,2010,2012$. (a) Nếu viết liên tiếp các số của dãy thành số $a = 2468101214\ldots20102012$ thì a có bao nhiêu chữ số? (b) Tìm chữ số thứ $2012$ của a.
\end{baitoan}

\begin{baitoan}[\cite{TLCT_THCS_Toan_6_so_hoc}, VD15.6, p. 93]
	(a) Rút gọn tổng $A = \sum_{i=0}^{50} 2^i = 2^0 + 2^1 + 2^2 + \cdots + 2^{50}$. (b) Rút gọn tổng $B = \sum_{i=1}^{100} 5^i = 5 + 5^2 + 5^3 + \cdots + 5^{100}$. (c) Rút gọn tổng $C = \sum_{i=1}^{2010} (-1)^{i+1}3^i = 3 - 3^2 + 3^3 - 3^4 + \cdots + 3^{2007} - 3^{2008} + 3^{2009} - 3^{2010}$. (d) Chứng minh $S_n =  a + aq+ aq^2 + \cdots + aq^{n-1} = \dfrac{q^n - 1}{q - 1}a$. Áp dụng rút gọn tổng $S_{100} = 5 + 5\cdot9 + 5\cdot9^2 + 5\cdot9^3 + \cdots + 5\cdot9^{99}$.
\end{baitoan}

\begin{baitoan}[\cite{TLCT_THCS_Toan_6_so_hoc}, VD15.7, p. 94]
	Tìm $x\in\mathbb{Q}$ thỏa $(x + 2) + (4x + 4) + (7x + 6) + \cdots + (25x + 18) + (28x + 20) = 1560$.
\end{baitoan}

\begin{baitoan}[\cite{TLCT_THCS_Toan_6_so_hoc}, VD15.8, p. 94]
	(a) Tính tổng $A = 1\cdot2 + 2\cdot3 + 3\cdot4 + \cdots + 99\cdot100$. (b) Chứng minh $A_n = 1\cdot2 + 2\cdot3 + 3\cdot4 + \cdots + n(n + 1) = \dfrac{1}{3}n(n + 1)(n + 2)$, $\forall n\in\mathbb{N}^\star$. (c) Sử dụng a), tính nhanh $B = \sum_{i=1}^{100} i^2 = 1^2 + 2^2 + \cdots + 100^2$. (d) Tính nhanh $C = 1\cdot100 + 2\cdot99 + 3\cdot98 + \cdots + 98\cdot3 + 99\cdot2 + 100\cdot1$.
\end{baitoan}

\begin{baitoan}[\cite{TLCT_THCS_Toan_6_so_hoc}, VD15.9, p. 95]
	Cho $A = \sum_{i=0}^{11} 4^i = 1 + 4 + 4^2 + \cdots + 4^{11}$. Chứng minh: (a) $A\divby21$. (b) $A\divby105$. (c) $A\divby4097$.
\end{baitoan}

\begin{baitoan}[\cite{TLCT_THCS_Toan_6_so_hoc}, VD15.10, p. 95]
	Cho $a_n = \underbrace{1\ldots1}_{2n}$, $\forall n\in\mathbb{N}^\star$. Xét xem dãy $a_1,a_2,\ldots,a_{2013}$ có bao nhiêu số chia hết cho $13$?
\end{baitoan}

\begin{baitoan}[\cite{TLCT_THCS_Toan_6_so_hoc}, VD15.11, p. 96]
	Tính $A = 1 + \sum_{i=1}^{100} ii!  = 1 + 1\cdot1! + 2\cdot2! + \cdots + 100\cdot100!$.
\end{baitoan}

\begin{baitoan}[\cite{TLCT_THCS_Toan_6_so_hoc}, VD15.12, p. 96]
	(a) Tính $A = \sum_{i=1}^{50} \dfrac{1}{i(i + 1)} = \dfrac{1}{1\cdot2} + \dfrac{1}{2\cdot3} + \cdots + \dfrac{1}{49\cdot50} + \dfrac{1}{50\cdot51}$. (b) Chứng minh $A_n = \sum_{i=1}^n \dfrac{1}{i(i + 1)} = \dfrac{1}{1\cdot2} + \dfrac{1}{2\cdot3} + \cdots + \dfrac{1}{n\cdot(n + 1)} = \dfrac{n}{n + 1}$, $\forall n\in\mathbb{N}^\star$.
\end{baitoan}

\begin{baitoan}[\cite{TLCT_THCS_Toan_6_so_hoc}, VD15.13, p. 97]
	Tính tổng $50$ số hạng đầu tiên của dãy $\dfrac{1}{2\cdot4},\dfrac{1}{4\cdot6},\dfrac{1}{6\cdot8},\ldots$
\end{baitoan}

\begin{baitoan}[\cite{TLCT_THCS_Toan_6_so_hoc}, VD15.14, p. 97]
	Tính: (a) $A = \sum_{i=1}^{48} \dfrac{1}{i(i + 1)(i + 2)} = \dfrac{1}{1\cdot2\cdot3} + \dfrac{1}{2\cdot3\cdot4} + \cdots + \dfrac{1}{48\cdot49\cdot50}$.\\(b) $A_n = \sum_{i=1}^n \dfrac{1}{i(i + 1)(i + 2)} = \dfrac{1}{1\cdot2\cdot3} + \dfrac{1}{2\cdot3\cdot4} + \cdots + \dfrac{1}{n(n + 1)(n + 2)}$, $\forall n\in\mathbb{N}^\star$.
\end{baitoan}

\begin{baitoan}[\cite{TLCT_THCS_Toan_6_so_hoc}, VD15.15, p. 98]
	(a) Cho $A = \sum_{i=2}^{200} \dfrac{1}{i!} = \dfrac{1}{2!} + \dfrac{1}{3!} + \cdots + \dfrac{1}{200!}$. Chứng minh $A < 1$. (b) Chứng minh $1 - \dfrac{1}{2} + \dfrac{1}{3} - \dfrac{1}{4} + \cdots + \dfrac{1}{99} - \dfrac{1}{100} = \dfrac{1}{51} + \dfrac{1}{52} + \cdots + \dfrac{1}{100}$.
\end{baitoan}

\begin{baitoan}[\cite{TLCT_THCS_Toan_6_so_hoc}, VD15.16, p. 98]
	Cho $A = \sum_{i=1}^{100} \dfrac{1}{i!} = \dfrac{1}{1!} + \dfrac{1}{2!} + \cdots + \dfrac{1}{100!}$. Chứng minh $3! - A > 4$.
\end{baitoan}

\begin{baitoan}[\cite{TLCT_THCS_Toan_6_so_hoc}, VD15.17, p. 98]
	(a) Tính $A = 1 + 2.25 + 3.5 + 4.75 + \cdots + 26$. (b) Tính $B = 1.2 + 2.3 + 3.4 + \cdots + 8.9 + 9.10 + 10.11 + 11.12 + \cdots + 18.19$.
\end{baitoan}

\begin{baitoan}[\cite{TLCT_THCS_Toan_6_so_hoc}, VD15.18, p. 99]
	Tính nhanh $A = \dfrac{18\cdot275 + 3\cdot666 + 9\cdot614\cdot2}{1 + 4 + 7 + \cdots + 58 + 61 + 62 + 62.2 + 62.3 + 62.4 - 271}$.
\end{baitoan}

\begin{baitoan}[\cite{TLCT_THCS_Toan_6_so_hoc}, VD15.19, p. 99]
	Rút gọn $A = \dfrac{1 + 15^4 + 15^8 + \cdots + 15^{96} + 15^{100}}{1 + 15^2 + 15^4 + \cdots + 15^{98} + 15^{100} + 15^{102}}$.
\end{baitoan}

\begin{baitoan}[\cite{TLCT_THCS_Toan_6_so_hoc}, VD15.20, pp. 99--100]
	Tính nhanh: (a) $A = \prod_{i=2}^{2012} 1  - \dfrac{1}{i} = \left(1 - \dfrac{1}{2}\right)\left(1 - \dfrac{1}{3}\right)\left(1 - \dfrac{1}{4}\right)\cdots\left(1 - \dfrac{1}{2012}\right)$. (b) $B = \prod_{i=1}^{99} \dfrac{i^2}{i(i + 1)} = \dfrac{1^2}{1\cdot2}\cdot\dfrac{2^2}{2\cdot3}\cdot\dfrac{3^2}{3\cdot4}\cdots\dfrac{99^2}{99\cdot100}$. (c) $C = \left(\dfrac{1}{3} - 1\right)\left(\dfrac{1}{6} - 1\right)\left(\dfrac{1}{10} - 1\right)\left(\dfrac{1}{15} - 1\right)\left(\dfrac{1}{21} - 1\right)\left(\dfrac{1}{28} - 1\right)\left(\dfrac{1}{36} - 1\right)$. (d) $D = \left(\dfrac{6}{8} + 1\right)\left(\dfrac{6}{18} + 1\right)\left(\dfrac{6}{30} + 1\right)\cdots\left(\dfrac{6}{10700} + 1\right)$.
\end{baitoan}

\begin{baitoan}[\cite{TLCT_THCS_Toan_6_so_hoc}, VD15.21, p. 100]
	Cho $A = \dfrac{100^2 + 1^2}{100\cdot1} + \dfrac{99^2 + 2^2}{99\cdot2} + \dfrac{98^2 + 3^2}{98\cdot3} + \cdots + \dfrac{52^2 + 49^2}{52\cdot49} + \dfrac{51^2 + 50^2}{51\cdot50}$, $B = \sum_{i=2}^{101} \dfrac{1}{i} = \dfrac{1}{2} + \dfrac{1}{3} + \dfrac{1}{4} + \cdots + \dfrac{1}{101}$, $C = \dfrac{1}{100\cdot1} + \dfrac{1}{99\cdot2} + \dfrac{1}{98\cdot3} + \cdots + \dfrac{1}{52\cdot49} + \dfrac{1}{51\cdot50}$. (b) Tính $\dfrac{A}{B}$. (b) Tính $B - 101C$.
\end{baitoan}

\begin{baitoan}[\cite{TLCT_THCS_Toan_6_so_hoc}, 15.1., p. 101]
	Viết thêm 2 số hạng tiếp theo của dãy số \& tìm số hạng tổng quát của dãy: (a) $2,6,12,20,\ldots$ (b) $4,10,18,28,40,\ldots$ (c) $3,6,11,18,27,\ldots$
\end{baitoan}

\begin{baitoan}[\cite{TLCT_THCS_Toan_6_so_hoc}, 15.2., p. 101]
	Cho dãy số $1,5,9,13,\ldots,37,\ldots$ (a) $37$ là số hạng thứ mấy của dãy? (b) Tìm số hạng thứ $100$. Tìm số hạng tổng quát của dãy. (c) Số $2000$, số $2013$ có thuộc dãy này không?
\end{baitoan}

\begin{baitoan}[\cite{TLCT_THCS_Toan_6_so_hoc}, 15.3., p. 101]
	Cho dãy số $2,11,29,56,92,\ldots$ (a) Tìm số hạng thứ $100$ của dãy. (b) Số $407$ là số hạng thứ bao nhiêu của dãy?
\end{baitoan}

\begin{baitoan}[\cite{TLCT_THCS_Toan_6_so_hoc}, 15.4., p. 101]
	(a) Tính tổng $A = 1 + 3 + 5 + 7 + \cdots + 2011 + 2013$. (b) Tính tổng của $100$ số tự nhiên chẵn liên tiếp bắt đầu từ số $100$.
\end{baitoan}

\begin{baitoan}[\cite{TLCT_THCS_Toan_6_so_hoc}, 15.5., p. 102]
	Tính hợp lý: (a) $A = -1 + 3 - 5 = 7 - 9 + \cdots - 2009 + 2011 - 2013$. (b) $B = 2 - 4 + 6 - 8 + \cdots + 2006 - 2008 + 2010 - 2012$. (c) $C = 1 + 2 - 3 - 4 + 5 + 6 - 7 - 8 + \cdots - 111 - 112 + 113 + 114 + 115$.
\end{baitoan}

\begin{baitoan}[\cite{TLCT_THCS_Toan_6_so_hoc}, 15.6., p. 102]
	Cho dãy số $7,10,13,16,19,\ldots$ (a) Tìm số thứ $n$. (b) Tính tổng $100$ số hạng đầu tiên của dãy.
\end{baitoan}

\begin{baitoan}[\cite{TLCT_THCS_Toan_6_so_hoc}, 15.7., p. 102]
	(a) Cho $a$ là tổng các số tự nhiên chẵn không vượt quá $200$, $b$ là tổng các số tự nhiên lẻ nhỏ hơn $200$. Tính $a - b$. (b) Tìm $n\in\mathbb{N}^\star$ thỏa $1 + 3 + 5 + \cdots + (2n - 1) = 1225$.
\end{baitoan}

\begin{baitoan}[\cite{TLCT_THCS_Toan_6_so_hoc}, 15.8., p. 102]
	Tìm $x\in\mathbb{Q}$ thỏa $(x + 1) + (x + 2) + (x + 3) + \cdots + (x + 100) = 5750$. (b) $(2x - 1) + (4x - 2) + \cdots + (400x - 200) = 5 + 10 + \cdots + 1000$.
\end{baitoan}

\begin{baitoan}[\cite{TLCT_THCS_Toan_6_so_hoc}, 15.9., p. 102]
	Viết liên tiếp các số tự nhiên thành dãy $123456789101112\ldots$ (a) Tìm tổng các chữ số của $a = 12345\ldots9899100$. (b) Chữ số $2$ ở hàng nghìn của số $2012$ là chữ số thứ bao nhiêu của dãy?
\end{baitoan}

\begin{baitoan}[\cite{TLCT_THCS_Toan_6_so_hoc}, 15.10., p. 102]
	Viết liên tiếp các số tự nhiên lẻ thành dãy $13579111315\ldots$ (a) Tìm chữ số thứ $100$ của dãy. (b) Chữ số $1$ của số $2013$ là chữ số thứ bao nhiêu của dãy?
\end{baitoan}

\begin{baitoan}[\cite{TLCT_THCS_Toan_6_so_hoc}, 15.11., p. 102]
	Cho $a = 46\cdot47\cdot48\cdots89\cdot90$. (a) Có bao nhiêu thừa số $2$ khi phân tích $a$ ra thừa số nguyên tố? (b) $a$ có tận cùng là bao nhiêu chữ số $0$ sau khi thực hiện phép nhân?
\end{baitoan}

\begin{baitoan}[\cite{TLCT_THCS_Toan_6_so_hoc}, 15.12., p. 102]
	(a) Cho $A = \sum_{i=0}^{2010} 2013^i = 2013^0 + 2013^1 + 2013^2 + \cdots + 2013^{2009} + 2013^{2010}$. Tính $2012A + 1$. (b) Cho $a,n\in\mathbb{N}^\star,a\ne0,a\ne1$. Rút gọn tổng $S = \sum_{i=0}^n a^i = a^0 + a^1 + a^2 + \cdots + a^n$.
\end{baitoan}

\begin{baitoan}[\cite{TLCT_THCS_Toan_6_so_hoc}, 15.13., p. 102]
	(a) Cho $A = \sum_{i=1}^{99} 3^i = 3 + 3^2 + 3^3 + 3^4 + \cdots + 3^{99}$. Tìm $n\in\mathbb{N}$ biết $2A + 3 = 3^{2n}$. (b) Chứng minh $4B + 25$ là 1 lũy thừa của $5$ với $B = \sum_{i=2}^{2012} 5^i = 5^2 + 5^3 + \cdots + 5^{2012}$. (c) Cho $C = \sum_{i=0}^{400} 4^i = 1 + 4 + 4^2 + \cdots + 4^{100},D = 4^{101}$. Chứng minh $3C < D$.
\end{baitoan}

\begin{baitoan}[\cite{TLCT_THCS_Toan_6_so_hoc}, 15.14., p. 102]
	Tính $A = 10\cdot11 + 11\cdot12 + 12\cdot13 + \cdots + 28\cdot29 + 29\cdot30$.
\end{baitoan}

\begin{baitoan}[\cite{TLCT_THCS_Toan_6_so_hoc}, 15.15., p. 103]
	Tính: (a) $A = \sum_{i=1}^{100} i^2 = 1^2 + 2^2 + 3^3 + \cdots + 10)^2$. (b) $B = \sum_{i=101}^{200} 101^2 + 102^2 + \cdots + 199^2 + 200^2$. (c) $C = 1\cdot3 + 2\cdot4 + 3\cdot5 + 4\cdot6 + \cdots + 99\cdot101 + 100\cdot102$. (d) $D = 1\cdot100 + 2\cdot99 + 3\cdot98 + \cdots + 99\cdot2 + 100\cdot1$. (e) $E = \sum_{i=1}^{98} i(i + 1)(i + 2) = 1\cdot2\cdot3 + 2\cdot3\cdot4 + 3\cdot4\cdot5 + \cdots + 98\cdot99\cdot100$.
\end{baitoan}

\begin{baitoan}[\cite{TLCT_THCS_Toan_6_so_hoc}, 15.16., p. 103]
	Cho $S_1 = 3,S_2 = 9,S_3 = 18,S_4 = 30,S_5 = 45,\ldots$ Tính $S_{100}$.
\end{baitoan}

\begin{baitoan}[\cite{TLCT_THCS_Toan_6_so_hoc}, 15.17., p. 103]
	(a) Khi phân tích ra thừa số nguyên tố, $100!$ chứa thừa số nguyên tố $3$ với số mũ bằng bao nhiêu? (b) Tích $30\cdot31\cdot32\cdots89\cdot90$ có bao nhiêu thừa số $3$ khi phân tích ra thừa số nguyên tố?
\end{baitoan}

\begin{baitoan}[\cite{TLCT_THCS_Toan_6_so_hoc}, 15.18., p. 103]
	Trong dãy số tự nhiên $1,2,3,\ldots,n$, lập các tổng: $A_1 = 1,A_2 = 2 + 3,A_3 = 4 + 5 + 6,A_4 = 7 + 8 + 9 +10,A_5 = 11 + 12 + 13 + 14 + 15,\ldots$ Tính $A_{101} - A_{100}$.
\end{baitoan}

\begin{baitoan}[\cite{TLCT_THCS_Toan_6_so_hoc}, 15.19., p. 103]
	Tính $A = \sum_{i=1}^{100} \underbrace{3\ldots3}_i = 3 + 33 + 333 + \cdots + \underbrace{3\ldots3}_{100}$.
\end{baitoan}

\begin{baitoan}[\cite{TLCT_THCS_Toan_6_so_hoc}, 15.20., p. 103]
	Tính tổng $S$ các số hạng đứng trước số $1$ cuối cùng trong tổng: $1 + 5 + 1 + 5 + 5 + 1 + 5 + 5 + 5 + 1 + 5 + 5 + 5 + 5 + 1 + \cdots + 1 + \underbrace{5 + 5 + \cdots + 5}_{2012} + 1$.
\end{baitoan}

\begin{baitoan}[\cite{TLCT_THCS_Toan_6_so_hoc}, 15.21., p. 103]
	Gọi $S_n$ là tổng các chữ số của $n\in\mathbb{N}$. Tính $\sum_{i=1}^{2013} S_i = S_1 + S_2 + \cdots + S_{2013}$.
\end{baitoan}

\begin{baitoan}[\cite{TLCT_THCS_Toan_6_so_hoc}, 15.22., p. 103]
	Tính: (a) $A = \sum_{i=15}^{2012} \dfrac{1}{i(i + 1)} = \dfrac{1}{15\cdot16} + \dfrac{1}{16\cdot17} + \dfrac{1}{17\cdot18} + \cdots + \dfrac{1}{2011\cdot2012} + \dfrac{1}{2012\cdot2013}$. (b) $B = \left(1 - \dfrac{1}{2}\right) + \left(1 - \dfrac{1}{4}\right) + \left(1 - \dfrac{1}{8}\right) + \cdots + \left(1 - \dfrac{1}{512}\right) + \left(1 - \dfrac{1}{1024}\right)$. (c) $C = 4\cdot5^{100}\left(\dfrac{1}{5} + \dfrac{1}{5^2} + dfrac{1}{5^3} + \cdots + dfrac{1}{5^{100}}\right) + 1$.
\end{baitoan}

\begin{baitoan}[\cite{TLCT_THCS_Toan_6_so_hoc}, 15.23., p. 104]
	Tính $A = 50 + \dfrac{50}{3} + \dfrac{25}{3} + \dfrac{20}{4} + \dfrac{10}{3} + \dfrac{10}{6\cdot7} + \cdots + \dfrac{100}{98\cdot99} + \dfrac{1}{99}$.
\end{baitoan}

\begin{baitoan}[\cite{TLCT_THCS_Toan_6_so_hoc}, 15.24., p. 104]
	Tính tổng $100$ \& $n\in\mathbb{N}$ số hạng đầu tiên của dãy: (a) $\dfrac{1}{3},\dfrac{1}{15},\dfrac{1}{35},\ldots$ (b) $\dfrac{1}{5},\dfrac{1}{45},\dfrac{1}{117},\dfrac{1}{221},\ldots$
\end{baitoan}

\begin{baitoan}[\cite{TLCT_THCS_Toan_6_so_hoc}, 15.25., p. 104]
	Tính: (a) $A = \dfrac{1}{1\cdot2} - \dfrac{1}{1\cdot2\cdot3} + \dfrac{1}{2\cdot3} - \dfrac{1}{2\cdot3\cdot4} + \dfrac{1}{3\cdot4} - \dfrac{1}{3\cdot4\cdot5} + \cdots + \dfrac{1}{99\cdot100} - \dfrac{1}{99\cdot100\cdot101}$. (b) $B = \dfrac{1}{1\cdot2\cdot3\cdot4} + \dfrac{1}{2\cdot3\cdot4\cdot5} + \dfrac{1}{3\cdot4\cdot5\cdot6} + \cdots + \dfrac{1}{47\cdot48\cdot49\cdot50}$.
\end{baitoan}

\begin{baitoan}[\cite{TLCT_THCS_Toan_6_so_hoc}, 15.26., p. 104]
	Tìm $x$ thỏa: (a) $\dfrac{1}{2013}x + 1 + \dfrac{1}{2} + \dfrac{1}{6} + \dfrac{1}{12} + \cdots + \dfrac{1}{2012\cdot2013} = 2$. (b) $2x + \dfrac{7}{6} + \dfrac{13}{12} + \dfrac{21}{20} + \dfrac{31}{30} + \dfrac{43}{42} + \dfrac{57}{56} + \dfrac{73}{72} + \dfrac{91}{90} = 10$.
\end{baitoan}

\begin{baitoan}[\cite{TLCT_THCS_Toan_6_so_hoc}, 15.27., p. 104]
	Tìm $x$ thỏa: (a) $\left(\dfrac{1}{1\cdot2\cdot3} + \dfrac{1}{2\cdot3\cdot4} + \dfrac{1}{3\cdot4\cdot5} + \cdots + \dfrac{1}{98\cdot99\cdot100}\right)x = \dfrac{49}{200}$. (b) $\dfrac{1}{5\cdot8} + \dfrac{1}{8\cdot11} + \dfrac{1}{11\cdot14} + \cdots + \dfrac{1}{x(x + 3)} = \dfrac{98}{1545}$.
\end{baitoan}

\begin{baitoan}[\cite{TLCT_THCS_Toan_6_so_hoc}, 15.28., p. 104]
	Cho 9 số $36,60,90,126,168,216,270,330,396$. Tính tỷ số giữa tổng nghịch đảo của 9 số này \& tổng của chúng.
\end{baitoan}

\begin{baitoan}[\cite{TLCT_THCS_Toan_6_so_hoc}, 15.29., p. 104]
	Rút gọn: (a) $A = \dfrac{1 + (1 + 2) + (1 + 2 + 3) + \cdots + (1 + 2 + 3 + \cdots + 100)}{(1\cdot100 + 2\cdot99 + \cdots 99\cdot2 + 100\cdot1)\cdot2013}$.\\(b) $B = \left(\dfrac{5^3}{6} + \dfrac{5^3}{12} + \dfrac{5^3}{20} + \dfrac{5^3}{30} + \dfrac{5^3}{42} + \dfrac{5^3}{56} + \dfrac{5^3}{72} + \dfrac{5^3}{90}\right):\dfrac{1124\cdot2247 - 1123}{1124 + 1123\cdot2247}$.
\end{baitoan}

\begin{baitoan}[\cite{TLCT_THCS_Toan_6_so_hoc}, 15.30., p. 104]
	Tính $A = \dfrac{4 + \dfrac{3}{5} + \cdots + \dfrac{3}{95} + \dfrac{3}{97} + \dfrac{3}{99}}{\dfrac{1}{1\cdot99} + \dfrac{1}{3\cdot97} + \dfrac{1}{5\cdot95} + \cdots + \dfrac{1}{95\cdot5} + \dfrac{1}{97\cdot3} + \dfrac{1}{99\cdot1}}$.
\end{baitoan}

\begin{baitoan}[\cite{TLCT_THCS_Toan_6_so_hoc}, 15.31., p. 105]
	(a) Chứng minh $A = \sum_{i=2}^{100} \dfrac{1}{i^2} = \dfrac{1}{2^2} + \dfrac{1}{3^2} + \cdots + \dfrac{1}{100^2} < 1$. (b) Chứng minh $B = \sum_{i=1}^{100} \dfrac{1}{i^2} = \dfrac{1}{1^2} + \dfrac{1}{2^2} + \dfrac{1}{3^2} + \cdots + \dfrac{1}{100^2} < 1\dfrac{3}{4}$. (c) Chứng minh $C = \dfrac{1}{2^2} + \dfrac{1}{4^2} + \dfrac{1}{6^2} + \cdots + \dfrac{1}{100^2} < \dfrac{1}{2}$. (d) Chứng minh $D = 2 + \dfrac{3}{4} + \dfrac{8}{9} + \dfrac{15}{16} + \cdots + \dfrac{2499}{2500} > 50$.
\end{baitoan}

\begin{baitoan}[\cite{TLCT_THCS_Toan_6_so_hoc}, 15.32., p. 105]
	(a) Cho $A = \sum_{i=4}^{100} \dfrac{1}{i^2} = \dfrac{1}{4^2} + \dfrac{1}{5^2} + \cdots + \dfrac{1}{100^2}$. Chứng minh $\dfrac{1}{5} < A < \dfrac{1}{3}$. (b) Cho $B = \sum_{i=100}^{199} \dfrac{1}{i^2} = \dfrac{1}{100^2} + \dfrac{1}{5^2} + \cdots + \dfrac{1}{199^2}$. Chứng minh $\dfrac{1}{200} < B < \dfrac{1}{99}$.
\end{baitoan}

\begin{baitoan}[\cite{TLCT_THCS_Toan_6_so_hoc}, 15.33., p. 105]
	(a) Chứng minh $A = \sum_{i=1}^{2012} \dfrac{1}{i!} = \dfrac{1}{1!} + \dfrac{1}{2!} + \dfrac{1}{3!} + \cdots + \dfrac{1}{2012!} < 2$. (b) Chứng minh $\dfrac{9}{10!} + \dfrac{10}{11!} + \dfrac{11}{12!} + \cdots + \dfrac{99}{100!} < \dfrac{1}{9!}$.
\end{baitoan}

\begin{baitoan}[\cite{TLCT_THCS_Toan_6_so_hoc}, 15.34., p. 105]
	(a) Cho $A = 1 - \dfrac{1}{2} + \dfrac{1}{3} - \dfrac{1}{4} + \cdots + \dfrac{1}{2011} - \dfrac{1}{2012},B = \dfrac{1}{1007} + \dfrac{1}{1008} + \cdots + \dfrac{1}{2011} + \dfrac{1}{2012}$. Tính $A:B$. (b) Cho $C = \dfrac{1}{2} - \dfrac{3}{4} + \dfrac{5}{6} - \dfrac{7}{8} + \cdots + \dfrac{197}{198} - \dfrac{199}{200},D = \dfrac{1}{51} + \dfrac{1}{52} + \cdots + \dfrac{1}{100}$. Tính $D:C$.
\end{baitoan}

\begin{baitoan}[\cite{TLCT_THCS_Toan_6_so_hoc}, 15.35., p. 105]
	Tính $A = \sum_{i=2}^{2012} \dfrac{1}{i} = \dfrac{1}{2} + \dfrac{1}{3} + \cdots + \dfrac{1}{2012},B = \dfrac{1}{2011} + \dfrac{2}{2010} + \dfrac{3}{2009} + \cdots + \dfrac{2009}{3} + \dfrac{2010}{2} + \dfrac{2011}{1}$.
\end{baitoan}

\begin{baitoan}[\cite{TLCT_THCS_Toan_6_so_hoc}, 15.36., pp. 105--106]
	Tính: (a) $A = 1.1 + 2.6 + 4.1 + 5.6 + \cdots + 148.1 + 149.6$. (b) $B = 1.2 + 2.3 + \cdots + 8.9 + 9.10 + 10.11 + \cdots 98.99 + 99.100 + 100.101 + \cdots + 998.999$. (c) $C = 1.2 + 2.4 + 3.6 + 4.8 + 5.10 + 6.12 + \cdots + 45.90$. (d) $D = 1.2 + 2.4 + 3.6 + 4.8 + 6 + 7.2 + \cdots + 120$.
\end{baitoan}

\begin{baitoan}[\cite{TLCT_THCS_Toan_6_so_hoc}, 15.37., p. 106]
	Tính nhanh: (a) $A = \prod_{i=10}^{100} \left(\dfrac{1}{10} - 1\right)\left(\dfrac{1}{11} - 1\right)\cdots\left(\dfrac{1}{100} - 1\right)$.\\(b) $B = \prod_{i=2}^{10} 1 - \dfrac{1}{i^2} = \left(1 - \dfrac{1}{2^2}\right)\left(1 - \dfrac{1}{3^2}\right)\cdots\left(1 - \dfrac{1}{10^2}\right)$. (c) $C = \left(\dfrac{7}{9} + 1\right)\left(\dfrac{7}{20} + 1\right)\left(\dfrac{7}{33} + 1\right)\cdots\left(\dfrac{7}{10800} + 1\right)$. (d) $D = \left(1 - \dfrac{28}{10}\right)\left(1 - \dfrac{52}{22}\right)\left(1 - \dfrac{80}{36}\right)\cdots\left(1 - \dfrac{21808}{10900}\right)$.
\end{baitoan}

\begin{baitoan}[\cite{TLCT_THCS_Toan_6_so_hoc}, 15.38., p. 106]
	Rút gọn: (a) $A = \dfrac{1 + \dfrac{1}{2} + \dfrac{1}{3} + \cdots + \dfrac{1}{2011} + \dfrac{1}{2012}}{\dfrac{2013}{1} + \dfrac{2014}{2} + \dfrac{2015}{3} + \cdots + \dfrac{4023}{2011} + \dfrac{4024}{2012} - 2012}$.\\(b) $B = \dfrac{\prod_{i=1}^{1000} 1 + \dfrac{2012}{i}}{\prod_{i=1}^{1000} 1 + \dfrac{1000}{i}} = \dfrac{\left(1 + \dfrac{2012}{1}\right)\left(1 + \dfrac{2012}{2}\right)\cdots\left(1 + \dfrac{2012}{1000}\right)}{\left(1 + \dfrac{1000}{1}\right)\left(1 + \dfrac{1000}{2}\right)\cdots\left(1 + \dfrac{1000}{2012}\right)}$.
\end{baitoan}

\begin{baitoan}[\cite{TLCT_THCS_Toan_6_so_hoc}, 15.39., p. 106]
	Tìm $x\in\mathbb{Q}$ thỏa: (a) $\dfrac{2}{1^2}\cdot\dfrac{6}{2^2}\cdot\dfrac{12}{3^2}\cdot\dfrac{20}{4^2}\cdots\dfrac{110}{10^2}x = -20$. (b) $\left(1 + \dfrac{1}{2} + \dfrac{1}{3} + \cdots + \dfrac{1}{2013}\right)x + 2013 = \dfrac{2014}{1} + \dfrac{2015}{2} + \cdots + \dfrac{4025}{2012} + \dfrac{4026}{2013}$. (c) $\left(\dfrac{1}{1\cdot2} + \dfrac{1}{3\cdot4} + \cdots + \dfrac{1}{99\cdot100}\right)x = \dfrac{2012}{51} + \dfrac{2012}{52} + \cdots + \dfrac{2012}{99} + \dfrac{2012}{100}$. (d) $\left(\dfrac{1}{1\cdot3} + \dfrac{1}{3\cdot5} + \cdots + \dfrac{1}{19\cdot21}\right)\cdot420 = [0.4\cdot(7.5 - 2.5x)]:0.25 = 212$.
\end{baitoan}

\begin{baitoan}[\cite{TLCT_THCS_Toan_6_so_hoc}, 15.40., p. 107]
	Cho $A = \dfrac{333}{444}\cdot\dfrac{888}{999}\cdot\dfrac{1515}{1616}\cdot\dfrac{2424}{2525}\cdot\dfrac{3535}{3636}\cdot\dfrac{4848}{4949}\cdot\dfrac{6363}{6464}\cdot\dfrac{8080}{8181}$,\\$B = \left(1 - \dfrac{1}{3}\right)\left(1 - \dfrac{1}{6}\right)\left(1 - \dfrac{1}{10}\right)\left(1 - \dfrac{1}{15}\right)\left(1 - \dfrac{1}{21}\right)\left(1 - \dfrac{1}{28}\right)\left(1 - \dfrac{1}{36}\right)$. (a) Tính $A:B$. (b) Tính tổng các nghịch đảo của $A,B$. (c) Tìm $x,y\in\mathbb{Z}$ sao cho $B < \dfrac{x}{36} < A,-B\ge\dfrac{-10}{y}\ge-A$.
\end{baitoan}

\begin{baitoan}[\cite{TLCT_THCS_Toan_6_so_hoc}, 15.41., p. 107]
	(a) Chứng minh $A = 1 + \dfrac{1}{2} + \dfrac{1}{3} + \cdots + \dfrac{1}{20}\notin\mathbb{Z}$. (b) Cho $\dfrac{a}{b} = \dfrac{1}{1\cdot2} + \dfrac{1}{3\cdot4} + \dfrac{1}{5\cdot6} + \cdots + \dfrac{1}{97\cdot98} + \dfrac{1}{99\cdot100}$. Chứng minh $a\divby151$.
\end{baitoan}

\begin{baitoan}[\cite{TLCT_THCS_Toan_6_so_hoc}, 15.42., p. 107]
	Tính $1.1 + 1.11 + 1.111 + \cdots + 1.\underbrace{1\ldots1}_9 + 1.\underbrace{1\ldots1}_{10}$.
\end{baitoan}

\begin{baitoan}[\cite{TLCT_THCS_Toan_6_so_hoc}, 15.43., p. 107]
	Tìm số hạng thứ $100$ trong dãy: $\dfrac{1}{1},\dfrac{2}{1},\dfrac{1}{2},\dfrac{3}{1},\dfrac{2}{2},\dfrac{1}{3},\dfrac{4}{1},\dfrac{3}{2},\dfrac{2}{3},\dfrac{1}{4},\dfrac{5}{1},\dfrac{4}{2},\dfrac{3}{3},\dfrac{2}{4},\dfrac{1}{5},\ldots$
\end{baitoan}

\begin{baitoan}[\cite{TLCT_THCS_Toan_6_so_hoc}, 15.44., p. 107]
	Cho $A = \prod_{i=1}^{99} 1 + \dfrac{2}{i} = \left(1 + \dfrac{2}{1}\right)\left(1 + \dfrac{2}{2}\right)\left(1 + \dfrac{2}{3}\right)\cdots\left(1 + \dfrac{2}{99}\right),B = (-1 - 2 - 3 - \cdots - 99 - 100)\left(\dfrac{1}{2} + \dfrac{1}{2^2} + \dfrac{1}{2^3} + \cdots + \dfrac{1}{2^{10}}\right)$. Tính $\dfrac{A}{B}$.
\end{baitoan}

\begin{baitoan}[\cite{TLCT_THCS_Toan_6_so_hoc}, 15.45., p. 107]
	Tính nhanh $A = \dfrac{1}{3} + \dfrac{1}{9} + \dfrac{1}{18} + \dfrac{1}{30} + \dfrac{1}{45} + \dfrac{1}{63} + \cdots + \dfrac{1}{14850}$.
\end{baitoan}

\begin{baitoan}[\cite{TLCT_THCS_Toan_6_so_hoc}, 15.46., p. 107]
	Cho $A = 1.01 + 1.02 + \cdots + 9.98 + 9.99 + 10,B = 2 - \dfrac{5}{3} + \dfrac{7}{6} - \dfrac{9}{10} + \dfrac{11}{15} - \dfrac{13}{21} + \dfrac{15}{28} - \dfrac{17}{36} + \dfrac{19}{45}$. Tính $2A + \dfrac{455}{3}B$.
\end{baitoan}

%------------------------------------------------------------------------------%

\section{Miscellaneous}

%------------------------------------------------------------------------------%

\printbibliography[heading=bibintoc]
	
\end{document}