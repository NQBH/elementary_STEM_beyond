\documentclass{article}
\usepackage[backend=biber,natbib=true,style=alphabetic,maxbibnames=50]{biblatex}
\addbibresource{/home/nqbh/reference/bib.bib}
\usepackage[utf8]{vietnam}
\usepackage{tocloft}
\renewcommand{\cftsecleader}{\cftdotfill{\cftdotsep}}
\usepackage[colorlinks=true,linkcolor=blue,urlcolor=red,citecolor=magenta]{hyperref}
\usepackage{amsmath,amssymb,amsthm,float,graphicx,mathtools,tikz}
\usetikzlibrary{angles,calc,intersections,matrix,patterns,quotes,shadings}
\allowdisplaybreaks
\newtheorem{assumption}{Assumption}
\newtheorem{baitoan}{}
\newtheorem{cauhoi}{Câu hỏi}
\newtheorem{conjecture}{Conjecture}
\newtheorem{corollary}{Corollary}
\newtheorem{dangtoan}{Dạng toán}
\newtheorem{definition}{Definition}
\newtheorem{dinhly}{Định lý}
\newtheorem{dinhnghia}{Định nghĩa}
\newtheorem{example}{Example}
\newtheorem{ghichu}{Ghi chú}
\newtheorem{hequa}{Hệ quả}
\newtheorem{hypothesis}{Hypothesis}
\newtheorem{lemma}{Lemma}
\newtheorem{luuy}{Lưu ý}
\newtheorem{nhanxet}{Nhận xét}
\newtheorem{notation}{Notation}
\newtheorem{note}{Note}
\newtheorem{principle}{Principle}
\newtheorem{problem}{Problem}
\newtheorem{proposition}{Proposition}
\newtheorem{question}{Question}
\newtheorem{remark}{Remark}
\newtheorem{theorem}{Theorem}
\newtheorem{vidu}{Ví dụ}
\usepackage[left=1cm,right=1cm,top=5mm,bottom=5mm,footskip=4mm]{geometry}
\def\labelitemii{$\circ$}
\DeclareRobustCommand{\divby}{%
	\mathrel{\vbox{\baselineskip.65ex\lineskiplimit0pt\hbox{.}\hbox{.}\hbox{.}}}%
}

\title{Problem: 3D Geometry -- Bài Tập: Hình Học Không Gian}
\author{Nguyễn Quản Bá Hồng\footnote{e-mail: \texttt{nguyenquanbahong@gmail.com}, website: \url{https://nqbh.github.io}, Ben Tre City, Vietnam..}}
\date{\today}

\begin{document}
\maketitle
\tableofcontents

%------------------------------------------------------------------------------%

\section{Line \& Plane in 3D Space -- Đường Thẳng \& Mặt Phẳng Trong Không Gian}

\begin{baitoan}[\cite{BTNC_Toan_11_HH}, VD1.1, p. 5]
	Cho hình chóp $S.ABC$, mặt phẳng $(\alpha)$ cắt $SA,SB,SC$ lần lượt tại $A',B',C'$. Giả sử $B'C'$ cắt $BC$ tại M, $C'A'$ cắt CA tại N, $A'B'$ cắt AB tại P. Chứng minh $M,N,P$ thẳng hàng.
\end{baitoan}

\begin{baitoan}[\cite{BTNC_Toan_11_HH}, VD1.2, p. 5]
	Cho hình chóp $S.ABCD$. 1 mặt phẳng $(\alpha)$ cắt 4 cạnh $SA,SB,SC,SD$ lần lượt tại 4 điểm $A',B',C',D'$. $O = AC\cap BC,O' = A'C'\cap B'D',M = AD\cap BC,M' = A'D'\cap B'C'$. Chứng minh: (a) $S,O,O'$ thẳng hàng. (b) $S,M,M'$ thẳng hàng.
\end{baitoan}

\begin{baitoan}[\cite{BTNC_Toan_11_HH}, VD1.3, p. 5]
	Cho tứ diện ABCD. $M,N$ lần lượt là trung điểm của $AB,CD$. G là trung điểm của MN, $A'$ là trọng tâm $\Delta BCD$. Chứng minh $A,A',G$ thẳng hàng.
\end{baitoan}

\begin{baitoan}[\cite{BTNC_Toan_11_HH}, VD1.4, p. 6]
	Cho tứ diện ABCD. 1 mặt phẳng $(\alpha)$ cắt $AB,BC,CD,DA$ lần lượt tại 4 điểm $M,N,P,Q$. Giả sử $MN\nparallel PQ$. Chứng minh 3 đường thẳng $MN,AC,PQ$ đồng quy tại 1 điểm.
\end{baitoan}

\begin{baitoan}[\cite{BTNC_Toan_11_HH}, VD1.5, p. 6]
	Trong không gian cho $n\in\mathbb{N},n\ge3$ đường thẳng sao cho 2 đường thẳng bất kỳ trong chúng đều cắt nhau \& không có 3 đường thẳng nào trong chúng đồng phẳng. Chứng minh $n$ đường thẳng này đồng quy tại 1 điểm.
\end{baitoan}

%------------------------------------------------------------------------------%

\section{Quan Hệ Vuông Góc Trong Không Gian, Khoảng Cách, Góc, \& Thể Tích}

%------------------------------------------------------------------------------%

\section{Miscellaneous}

%------------------------------------------------------------------------------%

\printbibliography[heading=bibintoc]
	
\end{document}