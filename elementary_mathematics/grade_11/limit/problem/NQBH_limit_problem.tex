\documentclass{article}
\usepackage[backend=biber,natbib=true,style=alphabetic,maxbibnames=50]{biblatex}
\addbibresource{/home/nqbh/reference/bib.bib}
\usepackage[utf8]{vietnam}
\usepackage{tocloft}
\renewcommand{\cftsecleader}{\cftdotfill{\cftdotsep}}
\usepackage[colorlinks=true,linkcolor=blue,urlcolor=red,citecolor=magenta]{hyperref}
\usepackage{amsmath,amssymb,amsthm,float,graphicx,mathtools,tikz}
\usetikzlibrary{angles,calc,intersections,matrix,patterns,quotes,shadings}
\allowdisplaybreaks
\newtheorem{assumption}{Assumption}
\newtheorem{baitoan}{}
\newtheorem{cauhoi}{Câu hỏi}
\newtheorem{conjecture}{Conjecture}
\newtheorem{corollary}{Corollary}
\newtheorem{dangtoan}{Dạng toán}
\newtheorem{definition}{Definition}
\newtheorem{dinhluat}{Định luật}
\newtheorem{dinhly}{Định lý}
\newtheorem{dinhnghia}{Định nghĩa}
\newtheorem{example}{Example}
\newtheorem{ghichu}{Ghi chú}
\newtheorem{hequa}{Hệ quả}
\newtheorem{hypothesis}{Hypothesis}
\newtheorem{lemma}{Lemma}
\newtheorem{luuy}{Lưu ý}
\newtheorem{nhanxet}{Nhận xét}
\newtheorem{notation}{Notation}
\newtheorem{note}{Note}
\newtheorem{principle}{Principle}
\newtheorem{problem}{Problem}
\newtheorem{proposition}{Proposition}
\newtheorem{question}{Question}
\newtheorem{remark}{Remark}
\newtheorem{theorem}{Theorem}
\newtheorem{vidu}{Ví dụ}
\usepackage[left=1cm,right=1cm,top=5mm,bottom=5mm,footskip=4mm]{geometry}
\def\labelitemii{$\circ$}
\DeclareRobustCommand{\divby}{%
	\mathrel{\vbox{\baselineskip.65ex\lineskiplimit0pt\hbox{.}\hbox{.}\hbox{.}}}%
}
\def\labelitemii{$\circ$}

\title{Problem: Limit -- Bài Tập: Giới Hạn}
\author{Nguyễn Quản Bá Hồng\footnote{A Scientist {\it\&} Creative Artist Wannabe. E-mail: {\tt nguyenquanbahong@gmail.com}. Bến Tre City, Việt Nam.}}
\date{\today}

\begin{document}
\maketitle
\begin{abstract}
	This text is a part of the series {\it Some Topics in Elementary STEM \& Beyond}:
	
	{\sc url}: \url{https://nqbh.github.io/elementary_STEM}.
	
	Latest version:
	\begin{itemize}
		\item {\it Problem: Limit -- Bài Tập: Giới Hạn}.
		
		PDF: {\sc url}: \url{https://github.com/NQBH/elementary_STEM_beyond/blob/main/elementary_mathematics/grade_11/limit/problem/NQBH_limit_problem.pdf}.
		
		\TeX: {\sc url}: \url{https://github.com/NQBH/elementary_STEM_beyond/blob/main/elementary_mathematics/grade_11/limit/problem/NQBH_limit_problem.tex}.
		\item {\it Problem \& Solution: Limit -- Bài Tập \& Lời Giải: Giới Hạn}.
		
		PDF: {\sc url}: \url{https://github.com/NQBH/elementary_STEM_beyond/blob/main/elementary_mathematics/grade_11/limit/solution/NQBH_limit_solution.pdf}.
		
		\TeX: {\sc url}: \url{https://github.com/NQBH/elementary_STEM_beyond/blob/main/elementary_mathematics/grade_11/limit/solution/NQBH_limit_solution.tex}.
	\end{itemize}
\end{abstract}
\tableofcontents

%------------------------------------------------------------------------------%

\section{Limit of Sequence -- Giới Hạn của Dãy Số}

\begin{baitoan}[\cite{Hung_nang_cao_phat_trien_Toan_11_tap_1}, VD1, p. 86]
	Cho dãy số $a_n = \dfrac{n}{n + 1}$, $n = 1,2,\ldots$ Chứng minh dãy $(a_n)$ có giới hạn là $1$.
\end{baitoan}

\begin{baitoan}[\cite{Hung_nang_cao_phat_trien_Toan_11_tap_1}, VD2, p. 87]
	Chứng minh $\lim_{n\to+\infty} \dfrac{1}{n} = 0$.
\end{baitoan}

\begin{baitoan}[\cite{Hung_nang_cao_phat_trien_Toan_11_tap_1}, VD3, p. 87]
	Chứng minh $\lim_{n\to+\infty} q^n = 0$ nếu $0 < |q| < 1$.
\end{baitoan}

\begin{baitoan}[\cite{Hung_nang_cao_phat_trien_Toan_11_tap_1}, VD4, p. 87]
	Chứng minh dãy $u_n = (-1)^n$ phân kỳ.
\end{baitoan}

\begin{baitoan}[\cite{Hung_nang_cao_phat_trien_Toan_11_tap_1}, VD5, p. 88]
	Tìm $\lim_{n\to+\infty} \dfrac{n^3 + 3n + 1}{2n^3 - 1}$.
\end{baitoan}

\begin{baitoan}[\cite{Hung_nang_cao_phat_trien_Toan_11_tap_1}, VD6, p. 88]
	Tìm $\lim_{n\to+\infty} \dfrac{n^4 + 2n^3 + 7n^2 + 8n + 9}{2n^4 + 3n^3 + n + 10}$.
\end{baitoan}

\begin{baitoan}[\cite{Hung_nang_cao_phat_trien_Toan_11_tap_1}, VD7, p. 88]
	Tìm $\lim_{n\to+\infty} (n - \sqrt[3]{n} - \sqrt{n})$.
\end{baitoan}

\begin{baitoan}[\cite{Hung_nang_cao_phat_trien_Toan_11_tap_1}, VD1, p. 89]
	Tìm $\lim_{n\to+\infty} \dfrac{\sin n}{n}$.
\end{baitoan}

\begin{baitoan}[\cite{Hung_nang_cao_phat_trien_Toan_11_tap_1}, VD2, p. 89]
	Chứng minh nếu $\lim_{n\to+\infty} |a_n| = 0$ thì $\lim_{n\to+\infty} a_n = 0$.
\end{baitoan}

\begin{baitoan}[\cite{Hung_nang_cao_phat_trien_Toan_11_tap_1}, VD3, p. 89]
	Chứng minh $\lim_{n\to+\infty} \sqrt[n]{n} = 1$.
\end{baitoan}

\begin{baitoan}[\cite{Hung_nang_cao_phat_trien_Toan_11_tap_1}, VD4, p. 89]
	Cho dãy số nguyên dương $(u_n)$ thỏa mãn $u_n > u_{n-1}u_{n+1}$, $\forall n\in\mathbb{N}$, $n\ge2$. Tính giới hạn $\lim_{n\to+\infty} \dfrac{1}{n^2}\sum_{i=1}^n \dfrac{i}{u_i} = \lim_{n\to+\infty} \dfrac{1}{n^2}\left(\dfrac{1}{u_1} + \dfrac{2}{u_2} + \cdots + \dfrac{n}{u_n}\right)$.
\end{baitoan}

\begin{baitoan}[\cite{Hung_nang_cao_phat_trien_Toan_11_tap_1}, VD5, p. 90]
	Tính $\lim_{n\to+\infty} \dfrac{1}{n^2}\sum_{i=2}^n i\cos\dfrac{\pi}{i}$.
\end{baitoan}

\begin{baitoan}[\cite{Hung_nang_cao_phat_trien_Toan_11_tap_1}, VD1, p. 90]
	Cho dãy số $(u_n)$ được xác định theo công thức $u_n = f(u_{n-1})$. Giả sử $u_n\in[a,b]$ với mọi chỉ số $n$ \& $f$ là hàm tăng trên $[a,b]$. Chứng minh: (a) Nếu $u_1\le u_2$ thì $(u_n)$ là dãy tăng. (b) Nếu $u_1\ge u_2$ thì $(u_n)$ là dãy giảm. (c) Nếu hàm $f$ bị chặn thì $(u_n)$ hội tụ.
\end{baitoan}

\begin{baitoan}[\cite{Hung_nang_cao_phat_trien_Toan_11_tap_1}, VD2, p. 90]
	Cho dãy $(u_n)$ được xác định bởi $u_n = \dfrac{1}{3}\left(2u_{n-1} + \dfrac{1}{u_{n-1}^2}\right)$, $\forall n\in\mathbb{N}$, $n\ge2$, $u_1 > 0$. Chứng minh dãy $(u_n)$ hội tụ \& tìm giới hạn của dãy.
\end{baitoan}

\begin{baitoan}[\cite{Hung_nang_cao_phat_trien_Toan_11_tap_1}, VD3, p. 91]
	Tìm $u_1$ để dãy $u_n = u_{n-1}^2 + 3u_{n-1} + 1$ hội tụ.
\end{baitoan}

\begin{baitoan}[\cite{Hung_nang_cao_phat_trien_Toan_11_tap_1}, VD4, p. 92]
	Chứng minh tồn tại $\lim_{n\to+\infty} \left(1 + \dfrac{1}{n}\right)^n$.
\end{baitoan}

\begin{baitoan}[Số Napier $e$]
	Đặt $e\coloneqq\lim_{n\to+\infty} \left(1 + \dfrac{1}{n}\right)^n$. Chứng minh: (a) $ \left(1 + \dfrac{1}{n}\right)^n < e < \left(1 + \dfrac{1}{n}\right)^{n+1}$, $\forall n\in\mathbb{N}^\star$. (b) $\dfrac{1}{n + 1} < \ln\left(1 + \dfrac{1}{n}\right) < \dfrac{1}{n}$, trong đó $\ln x$ là logarith cơ số $e$ của $x$.
\end{baitoan}

\begin{baitoan}[\cite{Hung_nang_cao_phat_trien_Toan_11_tap_1}, VD5, p. 91]
	Chứng minh dãy $u_n = \sum_{i=1}^n \dfrac{1}{i} - \ln n = 1 + \dfrac{1}{2} + \dfrac{1}{3} + \cdots + \dfrac{1}{n} - \ln n$ có giới hạn hữu hạn.
\end{baitoan}

\begin{luuy}
	$C = \lim_{n\to+\infty} \sum_{i=1}^n \dfrac{1}{i} - \ln n = \lim_{n\to+\infty}  1 + \dfrac{1}{2} + \dfrac{1}{3} + \cdots + \dfrac{1}{n} - \ln n$ được gọi là {\rm hằng số Euler}.
\end{luuy}

\begin{baitoan}[\cite{Hung_nang_cao_phat_trien_Toan_11_tap_1}, VD1, p. 92]
	Chứng minh không tồn tại $\lim_{n\to+\infty} \cos\dfrac{n\pi}{2}$.
\end{baitoan}

\begin{baitoan}[\cite{Hung_nang_cao_phat_trien_Toan_11_tap_1}, VD2, p. 92]
	Cho hàm $f:[0,+\infty)\to(0,b)$ liên tục \& nghịch biến. Giả sử hệ phương trình
	\begin{equation*}
		\left\{\begin{split}
			y &= f(x),\\
			x &= f(y),
		\end{split}\right.
	\end{equation*}
	có nghiệm duy nhất $x = y = q$. Chứng minh dãy $u_n = f(u_{n-1})$ hội tụ tới $q$ với $u_1 > 0$.
\end{baitoan}

\begin{baitoan}[\cite{Hung_nang_cao_phat_trien_Toan_11_tap_1}, VD3, p. 93]
	Cho dãy số $u_n = 1 + \dfrac{2}{1 + u_{n-1}}$, $u_1 > 0$. Chứng minh dãy hội tụ \& tìm giới hạn.
\end{baitoan}

\begin{baitoan}[\cite{Hung_nang_cao_phat_trien_Toan_11_tap_1}, VD1, p. 93]
	Cho dãy $a_n = \sum_{i=1}^n \dfrac{1}{i^2} = 1 + \dfrac{1}{2^2} + \cdots + \dfrac{1}{n^2}$, $\forall n\in\mathbb{N}^\star$. Chứng minh dãy này hội tụ.
\end{baitoan}

\begin{baitoan}[\cite{Hung_nang_cao_phat_trien_Toan_11_tap_1}, VD2, p. 93]
	Cho dãy $a_n = \sum_{i=1}^n \dfrac{1}{i} = 1 + \dfrac{1}{2} + \cdots + \dfrac{1}{n}$, $\forall n\in\mathbb{N}^\star$. Chứng minh dãy này phân kỳ.
\end{baitoan}

\begin{baitoan}[\cite{Hung_nang_cao_phat_trien_Toan_11_tap_1}, VD3, p. 94]
	Chứng minh $\lim_{n\to+\infty} \dfrac{1^p + 2^p + \cdots + n^p}{n^{p + 1}} = \dfrac{1}{p + 1}$, $\forall p\in\mathbb{N}$.
\end{baitoan}

\begin{baitoan}[\cite{Hung_nang_cao_phat_trien_Toan_11_tap_1}, VD1, p. 94]
	Khảo sát sự hội tụ của {\rm dãy H\'eron} $(u_n)$ được xác định bởi $u_1 = 1$, $u_n = \dfrac{1}{2}\left(u_{n-1} + \dfrac{2}{u_{n-1}}\right)$, $\forall n\in\mathbb{N}$, $n\ge2$.
\end{baitoan}

\begin{baitoan}[\cite{Hung_nang_cao_phat_trien_Toan_11_tap_1}, VD2, p. 95]
	Cho dãy số $(x_n)$ thỏa mãn $|x_{n+1} - a|\le\alpha|x_n - a|$, $\forall n\in\mathbb{N}$, trong đó $a\in\mathbb{R}$ \& $0 < \alpha < 1$. Chứng minh dãy số $(x_n)$ hội tụ về $a$.
\end{baitoan}

\begin{baitoan}[\cite{Hung_nang_cao_phat_trien_Toan_11_tap_1}, VD3, p. 95]
	Cho dãy số $(x_n)$ xác định bởi $x_1 = a\in\mathbb{R}$, $x_{n+1} = \cos x_n$, $\forall n\in\mathbb{N}^\star$. Chứng minh $(x_n)$ hội tụ.
\end{baitoan}

\begin{baitoan}[\cite{Hung_nang_cao_phat_trien_Toan_11_tap_1}, VD4, p. 95, Canada 1985]
	Dãy số $(x_n)$ thỏa mãn $1 < x_1 < 2$ \& $x_{n+1} = 1 + x_n - \dfrac{1}{2}x_n^2$, $\forall n\in\mathbb{N}^\star$. Chứng minh $(x_n)$ hội tụ. Tìm $\lim_{n\to+\infty} x_n$.
\end{baitoan}

\begin{baitoan}[\cite{Hung_nang_cao_phat_trien_Toan_11_tap_1}, VD5, p. 95, VMO2023]
	Xét dãy số $(a_n)$ thỏa mãn $a_1 = \dfrac{1}{2}$, $a_{n+1} = \sqrt[3]{3a_{n+1} - a_n}$ \& $0\le a_n\le1$, $\forall n\in\mathbb{N}^\star$. Chứng minh dãy $(a_n)$ có giới hạn hữu hạn.
\end{baitoan}

\begin{baitoan}[\cite{Hung_nang_cao_phat_trien_Toan_11_tap_1}, VD6, p. 96, VMO2022]
	Cho dãy số $(u_n)$ xác định bởi $u_1 = 6$, $u_{n+1} = 2 + \sqrt{u_n + 4}$, $\forall n\in\mathbb{N}^\star$.  Chứng minh dãy $(u_n)$ có giới hạn hữu hạn.
\end{baitoan}

\begin{baitoan}[\cite{Hung_nang_cao_phat_trien_Toan_11_tap_1}, VD7, p. 96, VMO2019]
	Cho dãy số $(x_n)$ xác định bởi $x_1 = 1$ \& $x_{n+1} = x_n + 3\sqrt{x_n} + \dfrac{n}{\sqrt{x_n}}$, $\forall n\in\mathbb{N}^\star$. (a) Chứng minh $\lim_{n\to+\infty} \dfrac{n}{x_n} = 0$. (b) Tính giới hạn $\lim_{n\to+\infty} \dfrac{n^2}{x_n}$.
\end{baitoan}

\begin{baitoan}[\cite{Hung_nang_cao_phat_trien_Toan_11_tap_1}, VD1, p. 97, VMO1984]
	Dãy số $(u_n)$ được xác định như sau: $u_1 = 1$, $u_2 = 2$, $u_{n+1} = 3u_n - u_{n-1}$. Dãy số $(v_n)$ được xác định như sau: $v_n = \sum_{i=1}^n {\rm arccot}u_i$. Tìm giới hạn $\lim_{n\to+\infty} v_n$.
\end{baitoan}

\begin{baitoan}[\cite{Hung_nang_cao_phat_trien_Toan_11_tap_1}, VD2, p. 97, VMO1988]
	Dãy số $(u_n)$ bị chặn thỏa mãn điều kiện $u_n + u_{n+1}\ge2u_{n+2}$, $\forall n\in\mathbb{N}^\star$ có nhất thiết hội tụ không?
\end{baitoan}

\begin{baitoan}[\cite{Hung_nang_cao_phat_trien_Toan_11_tap_1}, VD3, p. 98, Olympic 30.4 lần V]
	Cho $x_k = \sum_{i=1}^k \dfrac{i}{(i + 1)!} = \dfrac{1}{2!} + \dfrac{2}{3!} + \cdots + \dfrac{k}{(k + 1)!}$. Tính $\lim_{n\to+\infty} \sqrt[n]{\sum_{i=1}^{1999} x_i^n} = \lim_{n\to+\infty} \sqrt[n]{x_1^n + x_2^n + \cdots + x_{1999}^n}$.
\end{baitoan}

\begin{baitoan}[\cite{Hung_nang_cao_phat_trien_Toan_11_tap_1}, VD4, p. 98, VMO2013A]
	Gọi $F$ là tập hợp tất cả các hàm số $f:(0,+\infty)\to(0,+\infty)$ thỏa mãn $f(3x)\ge f(f(2x)) + x$, $\forall x > 0$. Tìm hằng số $A$ lớn nhất để $f(x)\ge Ax$, $\forall f\in F$, $\forall x > 0$.
\end{baitoan}

\begin{baitoan}[\cite{Hung_nang_cao_phat_trien_Toan_11_tap_1}, VD5, p. 98, Hải Dương 2019--2020]
	Cho dãy số thực $(x_n)$ thỏa mãn $x_1 = \dfrac{1}{6}$, $x_{n+1} = \dfrac{3x_n}{2x_n + 1}$, $\forall n\in\mathbb{N}^\star$. Tìm số hạng tổng quát của dãy số \& tính giới hạn của dãy số đó.
\end{baitoan}

\begin{baitoan}[\cite{Hung_nang_cao_phat_trien_Toan_11_tap_1}, VD6, p. 99, Hải Dương 2015--2016]
	Cho dãy số $(u_n)$ thỏa mãn $u_1 = -1$, $u_{n+1} = \dfrac{u_n}{2} + \dfrac{2}{u_n}$, $\forall n\in\mathbb{N}^\star$ \& dãy số $(v_n)$ thỏa mãn $u_nv_n - u_n + 2v_n + 2 = 0$, $\forall n\in\mathbb{N}^\star$. Tính $v_{2015}$ \& $\lim_{n\to+\infty} u_n$.
\end{baitoan}

\begin{baitoan}[\cite{Hung_nang_cao_phat_trien_Toan_11_tap_1}, VD7, p. 99, Hải Dương 2013--2014]
	Cho dãy số $(u_n)$ thỏa mãn $u_1 = \dfrac{5}{2}$, $u_{n+1} = \dfrac{1}{2}u_n^2 - u_n + 2$. Tính $\lim_{n\to+\infty} \sum_{i=1}^n \dfrac{1}{u_i}$.
\end{baitoan}

\begin{baitoan}[\cite{Hung_nang_cao_phat_trien_Toan_11_tap_1}, VD1, p. 99]
	Cho dãy số $(u_n)$ được xác định: $u_1$, $u_n = \alpha u_{n-1} + \beta$. Biện luận theo tham số $\alpha,\beta$ giá trị giới hạn của dãy số.
\end{baitoan}

\begin{baitoan}[\cite{Hung_nang_cao_phat_trien_Toan_11_tap_1}, VD1, p. 100]
	Cho $(u_n)$ là dãy số hội tụ \& $\lim_{n\to+\infty} u_n = u$. Khi đó, dãy trung bình cộng $v_n =  \dfrac{1}{n}\sum_{i=1}^n u_i$ cũng hội tụ \& $\lim_{n\to+\infty} v_n = u$.
\end{baitoan}

\begin{baitoan}[\cite{Hung_nang_cao_phat_trien_Toan_11_tap_1}, VD2, p. 100]
	Giả sử $\lim_{n\to+\infty} a_n = a$, $\lim_{n\to+\infty} b_n = b$. Chứng minh $\lim_{n\to+\infty} \dfrac{1}{n}\sum_{i=1}^n a_ib_{n+1-i} = \lim_{n\to+\infty} \dfrac{a_1b_n + a_2b_{n-1} + \cdots + a_nb_1}{n} = ab$.  Từ đó, suy ra $\lim_{n\to+\infty} \dfrac{1}{n}\sum_{i=1}^n a_i = \lim_{n\to+\infty} \dfrac{a_1 + a_2 + \cdots + a_n}{n}= a$.
\end{baitoan}

\begin{baitoan}[\cite{Hung_nang_cao_phat_trien_Toan_11_tap_1}, VD3, p. 101]
	Giả sử $a_n > 0$, $\forall n\in\mathbb{N}^\star$. Chứng minh nếu $\lim_{n\to+\infty} a_n = a > 0$ thì $\lim_{n\to+\infty} \sqrt[n]{\prod_{i=1}^n a_i} = \lim_{n\to+\infty} \sqrt[n]{a_1a_2\cdots a_n} = a$.
\end{baitoan}

\begin{baitoan}[\cite{Hung_nang_cao_phat_trien_Toan_11_tap_1}, VD4, p. 101]
	
\end{baitoan}

\begin{baitoan}[\cite{Hung_nang_cao_phat_trien_Toan_11_tap_1}, VD1, p. 100]
	
\end{baitoan}

\begin{baitoan}[\cite{Hung_nang_cao_phat_trien_Toan_11_tap_1}, VD1, p. 100]
	
\end{baitoan}

\begin{baitoan}[\cite{Hung_nang_cao_phat_trien_Toan_11_tap_1}, VD1, p. 100]
	
\end{baitoan}

\begin{baitoan}[\cite{Hung_nang_cao_phat_trien_Toan_11_tap_1}, VD1, p. 100]
	
\end{baitoan}

%------------------------------------------------------------------------------%

\section{Giới Hạn của Hàm Số}

%------------------------------------------------------------------------------%

\printbibliography[heading=bibintoc]
	
\end{document}