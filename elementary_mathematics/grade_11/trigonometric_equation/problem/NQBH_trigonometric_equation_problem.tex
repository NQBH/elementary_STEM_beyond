\documentclass{article}
\usepackage[backend=biber,natbib=true,style=alphabetic,maxbibnames=50]{biblatex}
\addbibresource{/home/nqbh/reference/bib.bib}
\usepackage[utf8]{vietnam}
\usepackage{tocloft}
\renewcommand{\cftsecleader}{\cftdotfill{\cftdotsep}}
\usepackage[colorlinks=true,linkcolor=blue,urlcolor=red,citecolor=magenta]{hyperref}
\usepackage{amsmath,amssymb,amsthm,float,graphicx,mathtools}
\allowdisplaybreaks
\newtheorem{assumption}{Assumption}
\newtheorem{baitoan}{Bài toán}
\newtheorem{cauhoi}{Câu hỏi}
\newtheorem{conjecture}{Conjecture}
\newtheorem{corollary}{Corollary}
\newtheorem{dangtoan}{Dạng toán}
\newtheorem{definition}{Definition}
\newtheorem{dinhly}{Định lý}
\newtheorem{dinhnghia}{Định nghĩa}
\newtheorem{example}{Example}
\newtheorem{ghichu}{Ghi chú}
\newtheorem{hequa}{Hệ quả}
\newtheorem{hypothesis}{Hypothesis}
\newtheorem{lemma}{Lemma}
\newtheorem{luuy}{Lưu ý}
\newtheorem{nhanxet}{Nhận xét}
\newtheorem{notation}{Notation}
\newtheorem{note}{Note}
\newtheorem{principle}{Principle}
\newtheorem{problem}{Problem}
\newtheorem{proposition}{Proposition}
\newtheorem{question}{Question}
\newtheorem{remark}{Remark}
\newtheorem{theorem}{Theorem}
\newtheorem{vidu}{Ví dụ}
\usepackage[left=1cm,right=1cm,top=5mm,bottom=5mm,footskip=4mm]{geometry}
\def\labelitemii{$\circ$}
\DeclareRobustCommand{\divby}{%
	\mathrel{\vbox{\baselineskip.65ex\lineskiplimit0pt\hbox{.}\hbox{.}\hbox{.}}}%
}

\title{Problem: Trigonometric Equation -- Bài Tập: Phương Trình Lượng Giác}
\author{Nguyễn Quản Bá Hồng\footnote{Independent Researcher, Ben Tre City, Vietnam\\e-mail: \texttt{nguyenquanbahong@gmail.com}; website: \url{https://nqbh.github.io}.}}
\date{\today}

\begin{document}
\maketitle
\tableofcontents

%------------------------------------------------------------------------------%

\section{Giá Trị Lượng Giác của Góc Lượng Giác}

\begin{baitoan}[\cite{Hung_nang_cao_phat_trien_Toan_11_tap_1}, Ví dụ 1, p. 8]
	Cho hình vuông $A_0A_1A_2A_3$ nội tiếp đường  tròn tâm O (4 đỉnh được sắp xếp theo chiều ngược chiều quay của kim đồng hồ). Tính số đo của các góc lượng giác $(OA_0,OA_i)$, $(OA_i,OA_j)$, $i,j = 0,1,2,3$, $i\ne j$.
\end{baitoan}

\begin{baitoan}[\cite{Hung_nang_cao_phat_trien_Toan_11_tap_1}, Ví dụ 2, p. 9]
	Tính giá trị biểu thức: (a) $A = \sin\dfrac{7\pi}{6} + \cos9\pi + \tan\left(-\dfrac{5\pi}{4}\right) + \cot\dfrac{7\pi}{2}$. (b) $B = \dfrac{1}{\tan368^\circ} + \dfrac{2\sin2550^\circ\cos(-188^\circ)}{2\cos638^\circ + \cos98^\circ}$. (c) $C = \sin^2 25^\circ + \sin^2 45^\circ + \sin^2 60^\circ + \sin^2 65^\circ$. (d) $D = \tan^2\dfrac{\pi}{8}\tan\dfrac{3\pi}{8}\tan\dfrac{5\pi}{8}$.
\end{baitoan}

\begin{baitoan}[\cite{Hung_nang_cao_phat_trien_Toan_11_tap_1}, Ví dụ 3, p. 9]
	Chứng minh đẳng thức (giả sử các đẳng thức sau đều có nghĩa): (a) $\cos^4 + 2\sin^2x = 1 + \sin^4x$. (b) $\dfrac{\sin x + \cos x}{\sin^3x} = \cot^3x + \cot^2x + \cot x + 1$. (c) $\dfrac{\cot^2x - \cot^2y}{\cot^2x\cot^2y} = \dfrac{\cos^2x - \cos^2y}{\cos^2x\cos^2y}$. (d) $\sqrt{\sin^4x + 4\cos^2x} + \sqrt{\cos^4x + 4\sin^2x} = 3\tan\left(x + \dfrac{\pi}{3}\right)\tan\left(\dfrac{\pi}{6} - x\right)$.
\end{baitoan}

\begin{baitoan}[\cite{Hung_nang_cao_phat_trien_Toan_11_tap_1}, Ví dụ 4, p. 10]
	Đơn giản biểu thức (giả sử các đẳng thức sau đều có nghĩa): (a) $A = \cos(5\pi - x) - \sin\left(\dfrac{3\pi}{2} + x\right) + \tan\left(\dfrac{3\pi}{2} - x\right) + \cot(3\pi - x)$. (b) $B = \dfrac{\sin(900^\circ + x) - \cos(450^\circ - x) + \cot(1080^\circ - x) + \tan(630^\circ - x)}{\cos(450^\circ - x) + \sin(x - 630^\circ) - \tan(810^\circ + x) - \tan(810^\circ - x)}$. (c) $C = \sqrt{2} - \dfrac{1}{\sin(x + 2013\pi)}\sqrt{\dfrac{1}{1 + \cos x} + \dfrac{1}{1 - \cos x}}$ với $\pi < x < 2\pi$.
\end{baitoan}

\begin{baitoan}[\cite{Hung_nang_cao_phat_trien_Toan_11_tap_1}, Ví dụ 5, p. 11]
	Chứng minh biểu thức không phụ thuộc vào $x$ (i.e., độc lập với biến $x$) (giả sử các biểu thức đều có nghĩa): (a) $A = \dfrac{\sin^6x + \cos^6x + 2}{\sin^4x + \cos^4x + 1}$. (b) $B = \dfrac{1 + \cot x}{1 - \cot x} - \dfrac{2 + 2\cot^2x}{(\tan x - 1)(\tan^2x + 1)}$. (c) $C = \sqrt{\sin^4x + 6\cos^2x + 3\cos^4x} + \sqrt{\cos^4x + 6\sin^2x + 3\sin^4x}$.
\end{baitoan}

\begin{baitoan}[\cite{Hung_nang_cao_phat_trien_Toan_11_tap_1}, 1.1., p. 12]
	Tìm số đo $a^\circ$ của góc lượng giác $(Ou,Ov)$ với $0\le a\le360$, biết 1 góc lượng giác cùng tia đầu, tia cuối với góc đó có số đo là: (a) $395^\circ$. (b) $-1052^\circ$. (c) $(20\pi)^\circ$.
\end{baitoan}

\begin{baitoan}[\cite{Hung_nang_cao_phat_trien_Toan_11_tap_1}, 1.2., p. 12]
	Không dùng máy tính bỏ túi, tính giá trị biểu thức: (a) $A = 5\sin^2\dfrac{151\pi}{6} + 3\cos^2\dfrac{85\pi}{3} - 4\tan^2\dfrac{193\pi}{6} + 7\cot^2\dfrac{37\pi}{3}$. (b) $B = \cos^2\dfrac{\pi}{5} + \cos^2\dfrac{2\pi}{5} + \cos^2\dfrac{\pi}{10} + \cos^2\dfrac{3\pi}{10}$. (c) $C = \tan\dfrac{\pi}{9}\tan\dfrac{2\pi}{9}\tan\dfrac{5\pi}{18}\tan\dfrac{7\pi}{18}$.
\end{baitoan}

\begin{baitoan}[\cite{Hung_nang_cao_phat_trien_Toan_11_tap_1}, 1.3., p. 12]
	Rút gọn biểu thức: (a) $A = \cos\left(\dfrac{\pi}{2} + x\right) + \cos(2\pi - x) + \cos(3\pi + x)$. (b) $B = 2\cos x - 3\cos(\pi - x) + 5\sin\left(\dfrac{7x}{2} - x\right) + \cot\left(\dfrac{3\pi}{2} - x\right)$. (c) $C = 2\sin(90^\circ + x) + \sin(900^\circ - x) + \sin(270^\circ + x) - \cos(90^\circ - x)$. (d) $D = \dfrac{\sin(5\pi + x)\cos\left(x - \dfrac{9\pi}{2}\right)\tan(10\pi + x)}{\cos(5\pi - x)\sin\left(\dfrac{11\pi}{2} + x\right)\tan(7\pi - x)}$.
\end{baitoan}

\begin{baitoan}[\cite{Hung_nang_cao_phat_trien_Toan_11_tap_1}, 1.4., p. 12]
	Chứng minh đẳng thức (giả sử các biểu thức đều có nghĩa): (a) $\tan^2x - \sin^2x = \tan^2x\sin^2x$. (b) $\dfrac{\tan^3x}{\sin^2x} - \dfrac{1}{\sin x\cos x} + \dfrac{\cot^3x}{\cos^2x} = \tan^3x + \cot^3x$. (c) $\sin^2x - \tan^2x = \tan^6x(\cos^2x - \cot^2x)$. (d) $\dfrac{\tan^2a - \tan^2b}{\tan^2a\tan^2b} = \dfrac{\sin^2a - \sin^2b}{\sin^2a\sin^2b}$.
\end{baitoan}

\begin{baitoan}[\cite{Hung_nang_cao_phat_trien_Toan_11_tap_1}, 1.5., p. 12]
	Chứng minh biểu thức không phụ thuộc vào $\alpha$: (a) $(\tan\alpha + \cot\alpha)^2 - (\tan\alpha - \cot\alpha)^2$. (b) $2(\sin^6\alpha + \cos^6\alpha) - 3(\sin^4\alpha + \cos^4\alpha)$. (c) $\cot^230^\circ(\sin^8\alpha - \cos^8\alpha) + 4\cos60^\circ(\cos^6\alpha - \sin^6\alpha) - \sin^6(90^\circ - \alpha)(\tan^2\alpha - 1)^3$. (d) $(\sin^4\alpha + \cos^4\alpha - 1)(\tan^2\alpha + \cot^2\alpha + 2)$.
\end{baitoan}

\begin{baitoan}[\cite{Hung_nang_cao_phat_trien_Toan_11_tap_1}, 1.6., p. 13]
	Biết $\tan x + \cot x = m$. Tính: (a) $\tan^2x + \cot^2x$. (b) $\dfrac{\tan^6x + \cot^6x}{\tan^4x + \cot^4x}$. (c) Chứng minh $|m|\ge2$. (d) Biện luận theo tham số $m$ để tìm $x$ thỏa mãn phương trình $\tan x + \cot x = m$.
\end{baitoan}

\begin{baitoan}[\cite{Hung_nang_cao_phat_trien_Toan_11_tap_1}, 1.7., p. 13]
	(a) Cho $\cos a = \dfrac{2}{3}$. Tính $A = \dfrac{\cot a + 3\tan a}{2\cot a + \tan a}$. (b) Cho $\sin a = \dfrac{1}{3}$. Tính $B = \dfrac{3\cot a + 2\tan a + 1}{\cot a + \tan a}$. (c) Cho $\tan a = 2$. Tính $C = \dfrac{2\sin a + 3\cos a}{\sin a + \cos a}$. (d) Cho $\cot a = 5$. Tính $D = 2\cos^2a + 5\sin a\cos a + 1$.
\end{baitoan}

%------------------------------------------------------------------------------%

\section{Trigonometrical Formulas -- Công Thức Lượng Giác}

\begin{baitoan}[\cite{Hung_nang_cao_phat_trien_Toan_11_tap_1}, Ví dụ 1, p. 14]
	Tính giá trị biểu thức lượng giác: (a) $A = \sin22^\circ33'\cos202^\circ30'$. (b) $B = 4\sin^4\dfrac{\pi}{16} + 2\cos\dfrac{\pi}{8}$. (c) $C = \dfrac{\sin\dfrac{\pi}{5} - \sin\dfrac{2\pi}{15}}{\cos\dfrac{\pi}{5} - \cos\dfrac{2\pi}{15}}$. (d) $D = \sin\dfrac{\pi}{9} - \sin\dfrac{5\pi}{9} + \sin\dfrac{7\pi}{9}$.
\end{baitoan}

\begin{baitoan}[\cite{Hung_nang_cao_phat_trien_Toan_11_tap_1},  Ví dụ 2, p. 14]
	Tính giá trị biểu thức lượng giác: (a) $A = \dfrac{1}{\cos290^\circ} + \dfrac{1}{\sqrt{3}\sin250^\circ}$. (b) $B = (1 + \tan20^\circ)(1 + \tan25^\circ)$. (c) $C = \tan9^\circ - \tan27^\circ - \tan63^\circ + \tan81^\circ$. (d) $D = \sin^2\dfrac{\pi}{9} + \sin^2\dfrac{2\pi}{9} + \sin\dfrac{\pi}{9}\sin\dfrac{2\pi}{9}$. 
\end{baitoan}

\begin{baitoan}[\cite{Hung_nang_cao_phat_trien_Toan_11_tap_1}, Ví dụ 3, p. 15]
	Tính giá trị biểu thức lượng giác: (a) $A = \sin\dfrac{\pi}{32}\cos\dfrac{\pi}{32}\cos\dfrac{\pi}{16}\cos\dfrac{\pi}{8}$. (b) $B = \sin10^\circ\sin30^\circ\sin50^\circ\sin70^\circ$. (c) $C = \cos\dfrac{\pi}{5} + \cos\dfrac{3\pi}{5}$. (d) $D = \cos^2\dfrac{\pi}{7} + \cos^2\dfrac{2\pi}{7} + \cos^2\dfrac{3\pi}{7}$.
\end{baitoan}

\begin{baitoan}[\cite{Hung_nang_cao_phat_trien_Toan_11_tap_1}, Ví dụ 4, p. 16]
	Cho $\alpha,\beta$ thỏa mãn $\sin\alpha + \sin\beta = \dfrac{\sqrt{2}}{2}$ \& $\cos\alpha + \cos\beta = \dfrac{\sqrt{6}}{2}$. Tính $\cos(\alpha - \beta),\sin(\alpha + \beta)$.
\end{baitoan}

\begin{baitoan}[\cite{Hung_nang_cao_phat_trien_Toan_11_tap_1}, Ví dụ 5, p. 17]
	Cho $\dfrac{1}{\tan^2\alpha} + \dfrac{1}{\cot^2\alpha} + \dfrac{1}{\sin^2\alpha} + \dfrac{1}{\cos^2\alpha} = 7$. Tính $\cos4\alpha$.
\end{baitoan}

\begin{baitoan}[\cite{Hung_nang_cao_phat_trien_Toan_11_tap_1}, Ví dụ 6, p. 17]
	Chứng minh: (a) $\sin3\alpha = 3\sin\alpha - 4\sin^3\alpha = 4\sin\alpha\sin\left(\dfrac{\pi}{3} - \alpha\right)\sin\left(\dfrac{\pi}{3} + \alpha\right)$. (b) $\sum_{i=1}^n 3^{i-1}\sin^3\dfrac{\alpha}{3^i} = \sin^3\dfrac{\alpha}{3} + 3\sin^3\dfrac{\alpha}{3^2} + \cdots + 3^{n-1}\sin^3\dfrac{\alpha}{3^n} = \dfrac{1}{4}\left(3^n\sin\dfrac{\alpha}{3^n} - \sin\alpha\right)$.
\end{baitoan}

\begin{baitoan}[Công thức nhân 3]
	Chứng minh: (a) $\cos3\alpha = 4\cos^3\alpha - 3\cos\alpha$. (b) $\sin3\alpha = 3\sin\alpha - 4\sin^3\alpha$.
\end{baitoan}

\begin{baitoan}[\cite{Hung_nang_cao_phat_trien_Toan_11_tap_1}, 2.1., p. 18]
	Tính giá trị của biểu thức: (a) $A = \cos^273^\circ + \cos^247^\circ + \cos74^\circ\cos47^\circ$. (b) $B = \sin6^\circ\sin42^\circ\sin66^\circ\sin78^\circ$.
\end{baitoan}

\begin{baitoan}[\cite{Hung_nang_cao_phat_trien_Toan_11_tap_1}, 2.2., p. 18]
	Tính giá trị của biểu thức: (a) $A = \sin^250^\circ + \sin^270^\circ - \cos50^\circ\cos70^\circ$.
\end{baitoan}

\begin{baitoan}[\cite{Hung_nang_cao_phat_trien_Toan_11_tap_1}, 2.3., p. 18]
	Tính giá trị của biểu thức: (a) $A = \cos\dfrac{\pi}{7}\cos\dfrac{4\pi}{7}\cos\dfrac{5\pi}{7}$. (b) $B = \cos\dfrac{2\pi}{7} + \cos\dfrac{4\pi}{7} + \cos\dfrac{6\pi}{7}$.
\end{baitoan}

\begin{baitoan}[\cite{Hung_nang_cao_phat_trien_Toan_11_tap_1}, 2.4., p. 18]
	Tính giá trị của biểu thức: (a) $A = \cos\dfrac{\pi}{7} + \cos\dfrac{3\pi}{7} + \cos\dfrac{5\pi}{7}$. (b) $B = \cos\dfrac{\pi}{7}\cos\dfrac{3\pi}{7} + \cos\dfrac{3\pi}{7}\cos\dfrac{5\pi}{7} + \cos\dfrac{5\pi}{7}\cos\dfrac{\pi}{7}$. (c) $C = \cos\dfrac{\pi}{7}\cos\dfrac{3\pi}{7}\cos\dfrac{5\pi}{7}$.
\end{baitoan}

\begin{baitoan}[\cite{Hung_nang_cao_phat_trien_Toan_11_tap_1}, 2.5., p. 18, Đề nghị Olympic 30.4 2006]
	Chứng minh $\sqrt[3]{\cos\dfrac{2\pi}{7}} + \sqrt[3]{\cos\dfrac{4\pi}{7}} + \sqrt[3]{\cos\dfrac{8\pi}{7}} = \sqrt[3]{\dfrac{5 - 3\sqrt[3]{7}}{2}}$.
\end{baitoan}

\begin{baitoan}[\cite{Hung_nang_cao_phat_trien_Toan_11_tap_1}, 2.6., p. 18]
	Cho $\alpha,\beta$ thỏa mãn $\sin\alpha + \sin\beta= m$ \& $\cos\alpha + \cos\beta = n$, $mn\ne0$. Tính $\cos(\alpha - \beta),\cos(\alpha + \beta),\sin(\alpha + \beta)$.
\end{baitoan}

\begin{baitoan}[\cite{Hung_nang_cao_phat_trien_Toan_11_tap_1}, 2.7., p. 18]
	Tính $A = \prod_{i=1}^{45} (1 + \tan i^\circ) = (1 + \tan1^\circ)(1 + \tan2^\circ)\cdots(1 + \tan45^\circ)$.
\end{baitoan}

\begin{baitoan}[\cite{Hung_nang_cao_phat_trien_Toan_11_tap_1}, 2.8., p. 18]
	Tính $A = \prod_{i=1}^{999} \cos i\alpha = \cos\alpha\cos2\alpha\cos3\alpha\cdots\cos999\alpha$ với $\alpha = \dfrac{2\pi}{999}$. 
\end{baitoan}

\begin{baitoan}[\cite{Hung_nang_cao_phat_trien_Toan_11_tap_1}, 2.9., p. 18]
	Chứng minh $\sin\dfrac{\pi}{9}\sin\dfrac{2\pi}{9}\sin\dfrac{4\pi}{9} = \cos\dfrac{\pi}{18}\cos\dfrac{5\pi}{18}\cos\dfrac{7\pi}{18} = \dfrac{\sqrt{3}}{8}$.
\end{baitoan}

\begin{baitoan}[\cite{Hung_nang_cao_phat_trien_Toan_11_tap_1}, 2.10., p. 18]
	Chứng minh: (a) $\cos x = \dfrac{\sin2x}{2\sin x}$. (b) $\prod_{i=1}^n \cos\dfrac{x}{2^i} = \cos\dfrac{x}{2}\cos\dfrac{x}{2^2}\cdots\cos\dfrac{x}{2^n} = \dfrac{\sin x}{2^n\sin\dfrac{x}{2^n}}$.
\end{baitoan}

\begin{baitoan}[\cite{Hung_nang_cao_phat_trien_Toan_11_tap_1}, 2.11., p. 18]
	Chứng minh: (a) $\dfrac{1}{\sin x} = \cot\dfrac{x}{2} - \cot x$. (b) $\sum_{i=1}^n \dfrac{1}{\sin2^{i-1}\alpha} = \dfrac{1}{\sin\alpha} + \dfrac{1}{\sin2\alpha} + \cdots + \dfrac{1}{\sin2^{n-1}\alpha} = \cot\dfrac{\alpha}{2} - \cot2^{n-1}\alpha$ với $2^{n-1}\alpha\ne k\pi$, $\forall k\in\mathbb{Z}$.
\end{baitoan}

\begin{baitoan}[\cite{Hung_nang_cao_phat_trien_Toan_11_tap_1}, 2.12., p. 18]
	Chứng minh: (a) $\tan x = \cot x - 2\cot2x$. (b) $\sum_{i=1}^n \dfrac{1}{2^i}\tan\dfrac{a}{2^i} = \dfrac{1}{2}\tan\dfrac{a}{2} + \dfrac{1}{2^2}\tan\dfrac{a}{2^2} + \cdots + \dfrac{1}{2^n}\tan\dfrac{a}{2^n} = \dfrac{1}{2^n}\cot\dfrac{a}{2^n} - \cot a$.
\end{baitoan}

\begin{baitoan}[\cite{Hung_nang_cao_phat_trien_Toan_11_tap_1}, 2.13., p. 18]
	Cho $n\in\mathbb{N}^\star$. Chứng minh: $\sum_{i=1}^{n-1} \dfrac{1}{\sin i^\circ\sin(i + 1)^\circ} = \dfrac{1}{\sin1^\circ\sin2^\circ} + \dfrac{1}{\sin2^\circ\sin3^\circ} + \cdots + \dfrac{1}{\sin(n - 1)^\circ\sin n^\circ} = \cot1^\circ - \cot n^\circ$.
\end{baitoan}

\begin{baitoan}[\cite{Hung_nang_cao_phat_trien_Toan_11_tap_1}, 2.14., p. 18]
	Chứng minh $\sum_{i=1}^{89} 2i\sin2i^\circ = 2\sin2^\circ + 4\sin4^\circ + \cdots + 178\sin178^\circ = 90\cot1^\circ$.
\end{baitoan}

%------------------------------------------------------------------------------%

\section{Trigonometrical Function -- Hàm Số Lượng Giác}

\begin{baitoan}[\cite{Hung_nang_cao_phat_trien_Toan_11_tap_1}, Ví dụ 1, p. 21]
	Vẽ đồ thị mỗi hàm số sau trong 1 chu kỳ: (a) $y = 2\cos2\theta$. (b) $y = \dfrac{1}{2}\sin\dfrac{x}{2}$.
\end{baitoan}

\begin{baitoan}[\cite{Hung_nang_cao_phat_trien_Toan_11_tap_1}, Ví dụ 2, p. 21]
	1 bánh xe được gắn cố định trên tường sao cho 1 điểm A trên bánh xe cách mặt đất 1 khoảng cách $d$ {\rm cm} theo công thức $d = 100 - 60\cos\dfrac{4\pi t}{3}$ với $t$ là thời gian được tính bằng giây. (a) Tính khoảng cách từ điểm A so với mặt đất khi $t = 0$. (b) Tính thời gian để bánh xe quay 1 vòng. (c) Tìm khoảng cách lớn nhất \& nhỏ nhất của A so với mặt đất. (d) Vẽ đồ thị hàm số $d$ theo $t$. (e) Trong vòng quay đầu tiên, tìm khoảng thời gian mà điểm A cách mặt đất 1 khoảng nhỏ hơn {\rm70 cm}.
\end{baitoan}

\begin{baitoan}[\cite{Hung_nang_cao_phat_trien_Toan_11_tap_1}, Ví dụ 3, p. 22]
	Tìm {\rm TXĐ} của hàm số $y = \sqrt{\dfrac{\cos x - 1}{4 + \cos x}}$.
\end{baitoan}

\begin{baitoan}[\cite{Hung_nang_cao_phat_trien_Toan_11_tap_1}, Ví dụ 4, p. 22]
	Xét tính chẵn lẻ của hàm số $y = \dfrac{\sin2x}{2\cos x - 3}$.
\end{baitoan}

\begin{baitoan}[\cite{Hung_nang_cao_phat_trien_Toan_11_tap_1}, Ví dụ 5, p. 22]
	Tìm tập giá trị của hàm số $y = \sqrt{3}\sin x - \cos x - 2$.
\end{baitoan}

\begin{baitoan}[\cite{Hung_nang_cao_phat_trien_Toan_11_tap_1}, Ví dụ 6, p. 22]
	{\rm GTLN, GTNN} của hàm số $y = (3 - 5\sin x)^{2018}$ là $M,m$. Tính $M + m$.
\end{baitoan}

\begin{baitoan}[\cite{Hung_nang_cao_phat_trien_Toan_11_tap_1}, Ví dụ 7, p. 22]
	Trong tập giá trị của hàm số $y = \dfrac{2\sin2x + \cos2x}{\sin2x - \cos2x + 3}$ có tất cả bao nhiêu giá trị nguyên?
\end{baitoan}

\begin{baitoan}[\cite{Hung_nang_cao_phat_trien_Toan_11_tap_1}, 3.1., p. 23]
	Cho hàm số $h(x) = \sqrt{\sin^4x + \cos^4x - 2m\sin x\cos x}$. Tìm tất cả các giá trị của tham số $m$ để hàm số xác định $\forall x\in\mathbb{R}$.
\end{baitoan}

\begin{baitoan}[\cite{Hung_nang_cao_phat_trien_Toan_11_tap_1}, 3.2., p. 23]
	Tìm $m$ để hàm số $y = \dfrac{3x}{\sqrt{2\sin^2x - m\sin x + 1}}$ xác định trên $\mathbb{R}$.
\end{baitoan}

\begin{baitoan}[\cite{Hung_nang_cao_phat_trien_Toan_11_tap_1}, 3.3., p. 23]
	Xét tính chẵn lẻ của hàm số $f(x) = \sin^{2007}x + \cos nx$, với $n\in\mathbb{Z}$.
\end{baitoan}

\begin{baitoan}[\cite{Hung_nang_cao_phat_trien_Toan_11_tap_1}, 3.4., p. 23]
	Tìm {\rm GTLN} của hàm số $y = 3\sin^2\left(x + \dfrac{\pi}{12}\right) + 4$.
\end{baitoan}

\begin{baitoan}[\cite{Hung_nang_cao_phat_trien_Toan_11_tap_1}, 3.5., p. 23]
	Tập giá trị của hàm số $y = \sin2x + \sqrt{3}\cos2x + 1$ là đoạn $[a,b]$. Tính tổng $S = a + b$.
\end{baitoan}

\begin{baitoan}[\cite{Hung_nang_cao_phat_trien_Toan_11_tap_1}, 3.6., p. 23]
	Tìm {\rm GTLN} của hàm số $y = \cos^2x + \sin x + 1$.
\end{baitoan}

\begin{baitoan}[\cite{Hung_nang_cao_phat_trien_Toan_11_tap_1}, 3.7., p. 23]
	Gọi $M,m$ lần lượt là {\rm GTLN, GTNN} của hàm số $y = \cos2x + \cos x$. Tính $M + m$.
\end{baitoan}

\begin{baitoan}[\cite{Hung_nang_cao_phat_trien_Toan_11_tap_1}, 3.8., p. 23]
	Tìm {\rm GTLN, GTNN} của hàm số $y = \sin^2x - \sin x + 2$.
\end{baitoan}

\begin{baitoan}[\cite{Hung_nang_cao_phat_trien_Toan_11_tap_1}, 3.9., p. 23]
	Tìm {\rm GTLN} của hàm số $y = 2\cos x + \sin\left(x + \dfrac{\pi}{4}\right)$.
\end{baitoan}

\begin{baitoan}[\cite{Hung_nang_cao_phat_trien_Toan_11_tap_1}, 3.10., p. 23]
	Tìm {\rm GTLN} của hàm số $y = \sqrt{1 + \dfrac{1}{2}\cos^2x} + \dfrac{1}{2}\sqrt{5 + 2\sin^2x}$.
\end{baitoan}

\begin{baitoan}[\cite{Hung_nang_cao_phat_trien_Toan_11_tap_1}, 3.11., p. 23]
	Cho hàm số $y = \dfrac{1}{2 - \cos x} + \dfrac{1}{1 + \cos x}$ với $x\in\left(0,\dfrac{\pi}{2}\right)$. Tìm {\rm GTNN} của hàm số.
\end{baitoan}

\begin{baitoan}[\cite{Hung_nang_cao_phat_trien_Toan_11_tap_1}, 3.12., p. 23]
	Vẽ đồ thị hàm số $y = \sin|x|$.
\end{baitoan}

\begin{baitoan}[\cite{Hung_nang_cao_phat_trien_Toan_11_tap_1}, 3.13., p. 23]
	Chiều cao của thủy triều tại Warnung vào ngày {\rm1.1} so với mực nước biển trung bình là $h(t)$ {\rm m} được đưa ra gần đúng theo quy tắc $h(t) = 4\sin\dfrac{\pi t}{6}$ với $t$ là thời gian (tính bằng giờ) sau nửa đêm. (a) Vẽ đồ thị hàm số $y = h(t)$ với $0\le t\le24$. (b) Thủy triều dâng cao khi nào? (c) Tính độ cao cao nhất của thủy triều. (d) Tính độ cao của thủy triều lúc {\rm8:00}. (e) Tàu thuyền chỉ được rời bến cảng khi thủy triều cao hơn mực nước biển trung bình ít nhất {\rm1 m}. Khi nào tàu thuyền có thể rời bến cảng vào ngày {\rm1.1}?
\end{baitoan}

%------------------------------------------------------------------------------%

\section{Trigonometrical Equation -- Phương Trình Lượng Giác}

\begin{baitoan}[\cite{Hung_nang_cao_phat_trien_Toan_11_tap_1}, Ví dụ 1, p. 25]
	Người ta quan sát thấy Mặt Trời mọc đầu tiên là tại vùng núi đảo ở Maine, Mỹ. Thời điểm Mặt Trời mọc được biểu diễn theo công thức $t(m) = 1.665\sin\dfrac{\pi}{6}(m + 3) + 5.485$ với $t$ là thời điểm (được tính từ nửa đêm) \& $m$ là tháng (tính từ tháng $1$). Khi nào Mặt Trời mọc lúc {\rm7:00}?
\end{baitoan}

\begin{baitoan}[\cite{Hung_nang_cao_phat_trien_Toan_11_tap_1}, Ví dụ 2, p. 25]
	Tìm góc $\alpha\in\left\{\dfrac{\pi}{6},\dfrac{\pi}{4},\dfrac{\pi}{3},\dfrac{\pi}{2}\right\}$ để phương trình $\cos2x + \sqrt{3}\sin2x - 2\cos x = 0\Leftrightarrow\cos(2x - \alpha) = \cos x$.
\end{baitoan}

\begin{baitoan}[\cite{Hung_nang_cao_phat_trien_Toan_11_tap_1}, Ví dụ 3, p. 25]
	Cho phương trình $\sin^4x + \cos^4x + \cos^24x = m$ với $m$ là tham số. (a) Giải phương trình khi $m = \dfrac{3}{2}$. (b) Tìm $m$ để phương trình trên có 4 nghiệm phân biệt thuộc đoạn $\left[-\dfrac{\pi}{4},\dfrac{\pi}{4}\right]$.
\end{baitoan}

\begin{baitoan}[\cite{Hung_nang_cao_phat_trien_Toan_11_tap_1}, Ví dụ 4, p. 26, IMO1963]
	Giải phương trình $\cos^2x + \cos^22x + \cos^23x = 1$.
\end{baitoan}

\begin{baitoan}[\cite{Hung_nang_cao_phat_trien_Toan_11_tap_1}, Ví dụ 5, p. 27]
	Giải phương trình $\sin x\left(1 + \tan x\tan\dfrac{x}{2}\right) + \tan x + 2\sqrt{3} = \dfrac{\sqrt{3}}{\cos^2x}$.
\end{baitoan}

\begin{baitoan}[\cite{Hung_nang_cao_phat_trien_Toan_11_tap_1}, 4.1., p. 27]
	Phương trình $\sqrt{3}\cos x + \sin x = -2$ có bao nhiêu nghiệm trên đoạn $[0,4035\pi]$.
\end{baitoan}

\begin{baitoan}[\cite{Hung_nang_cao_phat_trien_Toan_11_tap_1}, 4.2., p. 27]
	Giải phương trình $(\sin x + \cos x)^2 + 2\sin^2\dfrac{x}{2} = \sin x(2\sqrt{3}\sin x + 4 - \sqrt{3})$.
\end{baitoan}

\begin{baitoan}[\cite{Hung_nang_cao_phat_trien_Toan_11_tap_1}, 4.3., p. 27]
	Giải phương trình $(\sqrt{3} + 1)\cos^2x + (\sqrt{3} - 1)\sin x\cos x + \sin x - \cos x = \sqrt{3}$.
\end{baitoan}

\begin{baitoan}[\cite{Hung_nang_cao_phat_trien_Toan_11_tap_1}, 4.4., p. 27]
	Giải phương trình $2\sin^3x - \cos2x + \sin2x - 2\sin x + 2\cos x - 1 = 0$.
\end{baitoan}

\begin{baitoan}[\cite{Hung_nang_cao_phat_trien_Toan_11_tap_1}, 4.5., p. 27]
	Giải phương trình $\dfrac{\sin^{10}2x + \cos^{10}2x}{\sin^22x - \cos^22x} = -\dfrac{29\cos^34x}{16}$.
\end{baitoan}

\begin{baitoan}[\cite{Hung_nang_cao_phat_trien_Toan_11_tap_1}, 4.6., p. 27]
	Giải phương trình $\dfrac{8}{\sin^32x} + \tan x = \cot^3x$.
\end{baitoan}

\begin{baitoan}[\cite{Hung_nang_cao_phat_trien_Toan_11_tap_1}, 4.7., p. 27]
	Giải phương trình $\sin3x + \cos3x - 2\sqrt{2}\cos\left(x + \dfrac{\pi}{4}\right) + 1 = 0$.
\end{baitoan}

\begin{baitoan}[\cite{Hung_nang_cao_phat_trien_Toan_11_tap_1}, 4.9., p. 27]
	Giải phương trình $3\tan2x - \dfrac{3}{\cos2x} - 2\dfrac{1 - \cot x}{1 + \cot x} + 2\cos2x = 0$.
\end{baitoan}

\begin{baitoan}[\cite{Hung_nang_cao_phat_trien_Toan_11_tap_1}, 4.10., p. 27]
	Giải phương trình $3\tan2x - 2\sin\left(2x - \dfrac{3\pi}{2}\right) + 2\dfrac{\cos x - \sin x}{\cos x + \sin x} = \dfrac{1}{\cos2x}$.
\end{baitoan}

\begin{baitoan}[\cite{Hung_nang_cao_phat_trien_Toan_11_tap_1}, 4.11., p. 27]
	Giải phương trình $\dfrac{\sin^42x + \cos^42x}{\tan\left(\dfrac{\pi}{4} - x\right)\tan\left(\dfrac{\pi}{4} + x\right)} = \cos^44x$.
\end{baitoan}

\begin{baitoan}[\cite{Hung_nang_cao_phat_trien_Toan_11_tap_1}, 4.12., p. 27]
	Giải phương trình $3 + \cot^2x = 3\left(\dfrac{\cos2x}{\sin x} + \dfrac{\sin2x}{\cos x}\right)$.
\end{baitoan}

\begin{baitoan}[\cite{Hung_nang_cao_phat_trien_Toan_11_tap_1}, 4.13., p. 27]
	Giải phương trình $\dfrac{(\cos x - 1)(2\cos x - 1)}{\sin x} = 1 - \sin2x + 2\cos^2x$.
\end{baitoan}

\begin{baitoan}[\cite{Hung_nang_cao_phat_trien_Toan_11_tap_1}, 4.14., p. 27]
	Giải phương trình $4\sin^2\dfrac{x}{2} - \sqrt{3}\cos2x = 1 + 2\cos^2\left(x - \dfrac{3\pi}{4}\right)$.
\end{baitoan}

\begin{baitoan}[\cite{Hung_nang_cao_phat_trien_Toan_11_tap_1}, 4.15., p. 27]
	Giải phương trình $\dfrac{4\sin^2\dfrac{x}{2} - \sqrt{3}\cos2x - 1 - 2\cos^2\left(x - \dfrac{3\pi}{4}\right)}{\sqrt{2\cos3x + 1}} = 0$.
\end{baitoan}

\begin{baitoan}[\cite{Hung_nang_cao_phat_trien_Toan_11_tap_1}, 4.16., p. 28]
	Cho phương trình $\dfrac{(\sin x - \cos x)(\sin2x - 3) - \sin2x - \cos2x + 1}{2\sin x - \sqrt{2}} = 0$. Hỏi phương trình có bao nhiêu nghiệm thuộc khoảng $(2018\pi,2019\pi)$?
\end{baitoan}

\begin{baitoan}[\cite{Hung_nang_cao_phat_trien_Toan_11_tap_1}, 4.17., p. 28]
	Giải phương trình $\sin^23x\cos2x + \sin^2x = 0$.
\end{baitoan}

\begin{baitoan}[\cite{Hung_nang_cao_phat_trien_Toan_11_tap_1}, 4.18., p. 28]
	Giải phương trình $2\cos^3x - \sin2x\sin x = -2\sqrt{2}\cos\left(x + \dfrac{2019\pi}{4}\right)$.
\end{baitoan}

\begin{baitoan}[\cite{Hung_nang_cao_phat_trien_Toan_11_tap_1}, 4.19., p. 28]
	Phương trình $\sin5x + \sqrt{3}\cos5x = 2\sin7x$ có bao nhiêu nghiệm trên khoảng $\left(0,\dfrac{\pi}{2}\right)$?
\end{baitoan}

%------------------------------------------------------------------------------%

\section{Trigonometrical Identity \& Inequality -- Đẳng Thức \& Bất Đẳng Thức Lượng Giác}
Cho $\Delta ABC$, đặt $a\coloneqq BC, b\coloneqq CA, c\coloneqq AB$, $p,R,r$ lần lượt là nửa chu vi, bán kính đường tròn ngoại tiếp, bán kính đường tròn nội tiếp $\Delta ABC$.

\begin{baitoan}[\cite{Hung_nang_cao_phat_trien_Toan_11_tap_1}, p. 28, 1 số đẳng thức lượng giác cơ bản trong tam giác]
	Chứng minh: (a) $\sin A + \sin B + \sin C = 4\cos\dfrac{A}{2}\cos\dfrac{B}{2}\cos\dfrac{C}{2}$. (b) $\cos A + \cos B + \cos C = 1 + 4\sin\dfrac{A}{2}\sin\dfrac{B}{2}\sin\dfrac{C}{2}$. (c) $\sin^2A + \sin^2B + \sin^2C = 2 + 2\cos A\cos B\cos C$. (d) $\tan A + \tan B + \tan C = \tan A\tan B\tan C$. (c) $\tan\dfrac{A}{2}\tan\dfrac{B}{2} + \tan\dfrac{B}{2}\tan\dfrac{C}{2} + \tan\dfrac{C}{2}\tan\dfrac{A}{2} = 1$. (d) $\cot\dfrac{A}{2} + \cot\dfrac{B}{2} + \cot\dfrac{C}{2} = \cot\dfrac{A}{2}\cot\dfrac{B}{2}\cot\dfrac{C}{2}$. (e) $\cot A\cot B + \cot B\cot C + \cot C\cot A = 1$.
\end{baitoan}

\begin{baitoan}[\cite{Hung_nang_cao_phat_trien_Toan_11_tap_1}, Ví dụ 1, p. 29]
	Cho $\Delta ABC$. Chứng minh: (a) $ab + bc + ca = p^2 + r^2 + 4Rr$. (b) $a^2 + b^2 + c^2 = 2(p^2 - r^2 - 4Rr)$. (c) $(a - b)^2 + (b - c)^2 + (c - a)^2 = 2p^2 - 24Rr - 6r^2$.
\end{baitoan}

\begin{baitoan}[\cite{Hung_nang_cao_phat_trien_Toan_11_tap_1}, Ví dụ 2, p. 29]
	Cho $\Delta ABC$. Chứng minh: $\cos A + \cos B + \cos C = \dfrac{R + r}{R}$.
\end{baitoan}

\begin{baitoan}[\cite{Hung_nang_cao_phat_trien_Toan_11_tap_1}, Ví dụ 3, p. 30]
	Cho $\Delta ABC$. Chứng minh: $\cos A + \cos B + \cos C\le\dfrac{3}{2} - \dfrac{(\sin B - \sin C)^2 + (\sin C - \sin A)^2 + (\sin A - \sin B)^2}{4}$.
\end{baitoan}

\begin{baitoan}[\cite{Hung_nang_cao_phat_trien_Toan_11_tap_1}, Ví dụ 4, p. 30]
	Cho $\Delta ABC$. Chứng minh: $\dfrac{r}{R} + \dfrac{(a - b)^2 + (b - c)^2 + (c - a)^2}{16R^2}\le\dfrac{1}{2}$.
\end{baitoan}

\begin{baitoan}[\cite{Hung_nang_cao_phat_trien_Toan_11_tap_1}, Ví dụ 5, p. 30, Bất đẳng thức Gerretsen trong tam giác]
	Cho $\Delta ABC$. Chứng minh: (a) $p^2\le4R^2 + 4Rr + 3r^2$. (b) $a^2 + b^2 + c^2\le8R^2 + 4r^2$.
\end{baitoan}

\begin{baitoan}[\cite{Hung_nang_cao_phat_trien_Toan_11_tap_1}, 5.1., p. 31]
	Cho đa giác đều $31$-cạnh $A_0A_1\ldots A_{30}$. Chứng minh $\dfrac{1}{A_0A_1} = \dfrac{1}{A_0A_2} + \dfrac{1}{A_0A_4} + \dfrac{1}{A_0A_8} + \dfrac{1}{A_0A_{15}}$.
\end{baitoan}

\begin{baitoan}[\cite{Hung_nang_cao_phat_trien_Toan_11_tap_1}, 5.2., p. 31]
	Cho $\Delta ABC$ nội tiếp đường tròn $(O)$. Đường tròn $(I)$ là 1 đường tròn bất kỳ. Từ 3 điểm $A,B,C$ theo thứ tự kẻ 3 tiếp tuyến $Â',BB',CC'$ tới $(I)$. Chứng minh: (a) Nếu $(I)\cap(O) = \emptyset$ thì $aAA',bBB',cCC'$ là 3 cạnh của 1 tam giác. (b) Nếu $(I)\cap(O)\ne\emptyset$ \& cụ thể là: $(I)$ giao cung $BC$ không chứa A thì $aAA'\ge bBB' + cCC'$, $(I)$ giao cung $CA$ không chứa B thì $bBB'\ge cCC' + aAA'$, $(I)$ giao cung $AB$ không chứa C thì $cCC'\ge aAA' + bBB'$. Dấu bằng ở 3 bất đẳng thức xảy ra khi \& chỉ khi đường tròn $(I)$ tiếp xúc (trong hoặc ngoài) với đường tròn $(O)$ tại các điểm thuộc các cung tương ứng.
\end{baitoan}

\begin{baitoan}[\cite{Hung_nang_cao_phat_trien_Toan_11_tap_1}, 5.3., p. 31]
	Cho đa giác $A_1A_2\ldots A_n$ vừa nội tiếp vừa ngoại tiếp \& có tâm ngoại tiếp là O ta ký hiệu các góc $\widehat{A_iOA_{i+1}} = \theta_i$, $i = \overline{1,n}$, $n + 1\equiv1$, \& các góc đa giác lần lượt là $A_1,A_2,\ldots,A_n$. Chứng minh $\sum_{i=1}^n \cos\dfrac{A_i}{2}\ge\sum_{i=1}^n \sin\dfrac{\theta_i}{2}$, i.e., $\cos\dfrac{A_1}{2} + \cos\dfrac{A_2}{2} + \cdots + \cos\dfrac{A_n}{2}\ge\sin\dfrac{\theta_1}{2} + \sin\dfrac{\theta_2}{2} + \cdots + \sin\dfrac{\theta_n}{2}$.
\end{baitoan}

%------------------------------------------------------------------------------%

\section{Miscellaneous}

%------------------------------------------------------------------------------%

\printbibliography[heading=bibintoc]
	
\end{document}