\documentclass{article}
\usepackage[backend=biber,natbib=true,style=alphabetic,maxbibnames=50]{biblatex}
\addbibresource{/home/nqbh/reference/bib.bib}
\usepackage[utf8]{vietnam}
\usepackage{tocloft}
\renewcommand{\cftsecleader}{\cftdotfill{\cftdotsep}}
\usepackage[colorlinks=true,linkcolor=blue,urlcolor=red,citecolor=magenta]{hyperref}
\usepackage{amsmath,amssymb,amsthm,float,graphicx,mathtools,tikz}
\usetikzlibrary{angles,calc,intersections,matrix,patterns,quotes,shadings}
\allowdisplaybreaks
\newtheorem{assumption}{Assumption}
\newtheorem{baitoan}{}
\newtheorem{cauhoi}{Câu hỏi}
\newtheorem{conjecture}{Conjecture}
\newtheorem{corollary}{Corollary}
\newtheorem{dangtoan}{Dạng toán}
\newtheorem{definition}{Definition}
\newtheorem{dinhly}{Định lý}
\newtheorem{dinhnghia}{Định nghĩa}
\newtheorem{example}{Example}
\newtheorem{ghichu}{Ghi chú}
\newtheorem{hequa}{Hệ quả}
\newtheorem{hypothesis}{Hypothesis}
\newtheorem{lemma}{Lemma}
\newtheorem{luuy}{Lưu ý}
\newtheorem{nhanxet}{Nhận xét}
\newtheorem{notation}{Notation}
\newtheorem{note}{Note}
\newtheorem{principle}{Principle}
\newtheorem{problem}{Problem}
\newtheorem{proposition}{Proposition}
\newtheorem{question}{Question}
\newtheorem{remark}{Remark}
\newtheorem{theorem}{Theorem}
\newtheorem{vidu}{Ví dụ}
\usepackage[left=1cm,right=1cm,top=5mm,bottom=5mm,footskip=4mm]{geometry}
\def\labelitemii{$\circ$}
\DeclareRobustCommand{\divby}{%
	\mathrel{\vbox{\baselineskip.65ex\lineskiplimit0pt\hbox{.}\hbox{.}\hbox{.}}}%
}

\title{Problem: Exponentiation {\it\&} Logarithm -- Bài Tập: Hàm Số Mũ {\it\&} Hàm Số Logarith}
\author{Nguyễn Quản Bá Hồng\footnote{Ben Tre City, Vietnam. e-mail: \texttt{nguyenquanbahong@gmail.com}; website: \url{https://nqbh.github.io}.}}
\date{\today}

\begin{document}
\maketitle
\tableofcontents

%------------------------------------------------------------------------------%

\section{Exponentiation with Real Exponent -- Lũy Thừa Với Số Mũ Thực}

\begin{baitoan}[\cite{BTNC_Toan_11_DSGTXSTK}, p. 143]
	Cho $a,b\in(0,\infty)$. Rút gọn biểu thức $A = \dfrac{a^{\frac{11}{5}}b^2 + a^2b^{\frac{11}{5}}}{\sqrt[5]{a} + \sqrt[5]{b}}$.
\end{baitoan}

\begin{baitoan}[\cite{BTNC_Toan_11_DSGTXSTK}, VD1, p. 144]
	Cho $x > 0$. Viết biểu thức $f(x) = \sqrt[7]{x^3\sqrt[5]{x^2}}$ dưới dạng lũy thừa của 1 số với số mũ hữu tỷ.
\end{baitoan}

\begin{baitoan}[\cite{BTNC_Toan_11_DSGTXSTK}, VD2, p. 144]
	Cho $a > 0$. Chứng minh $\dfrac{1}{a^{\frac{1}{4}} + a^{\frac{1}{8}} + 1} + \dfrac{1}{a^{\frac{1}{4}} - a^{\frac{1}{8}} + 1} - \dfrac{2\left(a^{\frac{1}{4}} - 1\right)}{a^{\frac{1}{2}} - a^{\frac{1}{4}} + 1} = \dfrac{4}{a + \sqrt{a} + 1}$.
\end{baitoan}

\begin{baitoan}[\cite{BTNC_Toan_11_DSGTXSTK}, VD1, p. 144, IsraelNO2015]
	Chứng minh $\left(\dfrac{76}{\dfrac{1}{\sqrt[3]{77} - \sqrt[3]{75}} - \sqrt[3]{5775}} + \dfrac{1}{\dfrac{76}{\sqrt[3]{77} + \sqrt[3]{75}} + \sqrt[3]{5775}}\right)^3\in\mathbb{N}$.
\end{baitoan}

\begin{baitoan}[\cite{BTNC_Toan_11_DSGTXSTK}, 1., p. 144]
	Chứng minh $\sqrt[3]{7 + 5\sqrt{2}} + \sqrt[3]{7 - 5\sqrt{2}} = 2$.
\end{baitoan}

\begin{baitoan}[\cite{BTNC_Toan_11_DSGTXSTK}, 2., p. 144]
	So sánh $\sqrt[3]{7} + \sqrt{15},\sqrt{10} + \sqrt[3]{28}$.
\end{baitoan}

\begin{baitoan}[\cite{BTNC_Toan_11_DSGTXSTK}, 3., p. 144]
	Cho $a\in(0,\infty)$. Viết biểu thức $f(a) = \sqrt{a\sqrt{a\sqrt[3]{a^3\sqrt[5]{a^{12}}}}}:a^{-\frac{4}{5}}$ dưới dạng lũy thừa của 1 số với số mũ hữu tỷ.
\end{baitoan}

\begin{baitoan}[\cite{BTNC_Toan_11_DSGTXSTK}, 4., p. 144]
	Cho $x,y\in(0,\infty)$. Viết biểu thức $A = \sqrt[5]{\dfrac{y^7}{x^3}\sqrt[3]{\dfrac{x^2}{y^2}\sqrt{\dfrac{x^5}{y^7}}}}$ dưới dạng lũy thừa của 1 số với số mũ hữu tỷ.
\end{baitoan}

%------------------------------------------------------------------------------%

\section{Logarithm}

%------------------------------------------------------------------------------%

\section{Exponentiation. Logarithm -- Hàm Số Mũ. Hàm Số Logarith}

%------------------------------------------------------------------------------%

\section{Exponential \& Logarithmic Equation, Inequation -- Phương Trình, Bất Phương Trình Mũ \& Logarith}

%------------------------------------------------------------------------------%

\section{Miscellaneous}

%------------------------------------------------------------------------------%

\printbibliography[heading=bibintoc]
	
\end{document}