\documentclass{article}
\usepackage[backend=biber,natbib=true,style=alphabetic,maxbibnames=50]{biblatex}
\addbibresource{/home/nqbh/reference/bib.bib}
\usepackage[utf8]{vietnam}
\usepackage{tocloft}
\renewcommand{\cftsecleader}{\cftdotfill{\cftdotsep}}
\usepackage[colorlinks=true,linkcolor=blue,urlcolor=red,citecolor=magenta]{hyperref}
\usepackage{amsmath,amssymb,amsthm,enumitem,float,graphicx,mathtools,tikz}
\usetikzlibrary{angles,calc,intersections,matrix,patterns,quotes,shadings}
\allowdisplaybreaks
\newtheorem{assumption}{Assumption}
\newtheorem{baitoan}{}
\newtheorem{cauhoi}{Câu hỏi}
\newtheorem{conjecture}{Conjecture}
\newtheorem{corollary}{Corollary}
\newtheorem{dangtoan}{Dạng toán}
\newtheorem{definition}{Definition}
\newtheorem{dinhly}{Định lý}
\newtheorem{dinhnghia}{Định nghĩa}
\newtheorem{example}{Example}
\newtheorem{ghichu}{Ghi chú}
\newtheorem{hequa}{Hệ quả}
\newtheorem{hypothesis}{Hypothesis}
\newtheorem{lemma}{Lemma}
\newtheorem{luuy}{Lưu ý}
\newtheorem{nhanxet}{Nhận xét}
\newtheorem{notation}{Notation}
\newtheorem{note}{Note}
\newtheorem{principle}{Principle}
\newtheorem{problem}{Problem}
\newtheorem{proposition}{Proposition}
\newtheorem{question}{Question}
\newtheorem{remark}{Remark}
\newtheorem{theorem}{Theorem}
\newtheorem{vidu}{Ví dụ}
\usepackage[left=1cm,right=1cm,top=5mm,bottom=5mm,footskip=4mm]{geometry}
\def\labelitemii{$\circ$}
\DeclareRobustCommand{\divby}{%
	\mathrel{\vbox{\baselineskip.65ex\lineskiplimit0pt\hbox{.}\hbox{.}\hbox{.}}}%
}
\def\labelitemii{$\circ$}
\setlist[itemize]{leftmargin=*}
\setlist[enumerate]{leftmargin=*}

\title{Problem: Derivative -- Bài Tập: Đạo Hàm}
\author{Nguyễn Quản Bá Hồng\footnote{A Scientist {\it\&} Creative Artist Wannabe. E-mail: {\tt nguyenquanbahong@gmail.com}. Bến Tre City, Việt Nam.}}
\date{\today}

\begin{document}
	\maketitle
\begin{abstract}
	This text is a part of the series {\it Some Topics in Elementary STEM \& Beyond}:
	
	{\sc url}: \url{https://nqbh.github.io/elementary_STEM}.
	
	Latest version:
	\begin{itemize}
		\item {\it Problem: Derivative -- Bài Tập: Đạo Hàm}.
		
		PDF: {\sc url}: \url{https://github.com/NQBH/elementary_STEM_beyond/blob/main/elementary_mathematics/grade_11/derivative/problem/NQBH_derivative_problem.pdf}.
		
		\TeX: {\sc url}: \url{https://github.com/NQBH/elementary_STEM_beyond/blob/main/elementary_mathematics/grade_11/derivative/problem/NQBH_derivative_problem.tex}.
		\item {\it Problem \& Solution: Derivative -- Bài Tập \& Lời Giải: Đạo Hàm}.
		
		PDF: {\sc url}: \url{https://github.com/NQBH/elementary_STEM_beyond/blob/main/elementary_mathematics/grade_11/derivative/solution/NQBH_derivative_solution.pdf}.
		
		\TeX: {\sc url}: \url{https://github.com/NQBH/elementary_STEM_beyond/blob/main/elementary_mathematics/grade_11/derivative/solution/NQBH_derivative_solution.tex}.
	\end{itemize}
\end{abstract}
\tableofcontents

%------------------------------------------------------------------------------%

\section{Basic}
\textbf{\textsf{Resources -- Tài nguyên.}}
\begin{enumerate}
	\item \cite{SGK_Toan_11_CD_tap_2}. {\sc Đỗ Đức Thái, Phạm Xuân Chung, Nguyễn Sơn Hà, Nguyễn Thị Phương Loan, Phạm Sỹ Nam, Phạm Minh
	Phương}. {\it Toán 11 Tập 1. Cánh Diều}.
\end{enumerate}

%------------------------------------------------------------------------------%

\section{Định Nghĩa Đạo Hàm. Ý Nghĩa Hình Học Của Đạo Hàm}
Nếu quỹ đạo chuyển động của 1 vật hay 1 chất điểm được miêu tả bằng hàm số ${\bf x}(t)$ theo thời gian thì vận tốc ${\bf v}(t) = {\bf x}'(t)$ biểu thị độ nhanh chậm của chuyển động tại 1 thời điểm $t$.

\begin{baitoan}[Derivative of polynomials -- Đạo hàm của các đa thức]
	Tính đạo hàm của hàm số đa thức
	\begin{equation}
		\label{polynomial}
		\tag{P}
		P(x;n,{\bf a})\coloneqq\sum_{i=0}^n a_ix^i = a_nx^n + a_{n-1}x^{n-1} + \cdots + a_1x + a_0,
	\end{equation}
	tại $x = x_0$ bằng định nghĩa, với $\deg P(x;n,{\bf a}) = n\in\mathbb{N}$ \& vector chứa các hệ số của đa thức $P(x;n,{\bf a})$ là ${\bf a}\coloneqq(a_0,a_1,\ldots,a_n)\in\mathbb{R}^n\times\mathbb{R}^\star$.
\end{baitoan}

\begin{baitoan}[Derivative of rational function -- Đạo hàm của phân thức]
	Tính đạo hàm của hàm số phân thức
	\begin{equation}
		\label{rational function}
		\tag{Q}
		Q(x;m,n,{\bf a},{\bf b})\coloneqq\frac{\sum_{i=0}^m a_ix^i}{\sum_{i=0}^n b_ix^i} = \frac{a_mx^m + a_{m-1}x^{m-1} + \cdots + a_1x + a_0}{b_nx^n + b_{n-1}x^{n-1} + \cdots + b_1x + b_0},
	\end{equation}
	tại $x = x_0$ bằng định nghĩa.
\end{baitoan}

\begin{baitoan}[Đạo hàm của căn thức]
	Tính đạo hàm của hàm số căn thức $f(x) = \sqrt[n]{x} = x^{\frac{1}{n}}$, với $n\in\mathbb{N}^\star$, tại $x = x_0$ bằng định nghĩa.
\end{baitoan}
Ta có 3 dạng hàm số sơ cấp thường gặp: hàm đa thức $P(x;n,{\bf a})\coloneqq\sum_{i=0}^n a_ix^i$, hàm phân thức $Q(x;m,n,{\bf a},{\bf b})\coloneqq\frac{\sum_{i=0}^m a_ix^i}{\sum_{i=0}^n b_ix^i}$, hàm căn thức $R_n(x)\coloneqq\sqrt[n]{x}$.

%------------------------------------------------------------------------------%

\section{Differentiation Rules -- Các Quy Tắc Tính Đạo Hàm}

%------------------------------------------------------------------------------%

\section{2nd-Order Derivative -- Đạo Hàm Cấp 2}

%------------------------------------------------------------------------------%

\section{Miscellaneous}

%------------------------------------------------------------------------------%

\printbibliography[heading=bibintoc]
	
\end{document}