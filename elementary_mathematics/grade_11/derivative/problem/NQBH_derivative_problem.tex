\documentclass{article}
\usepackage[backend=biber,natbib=true,style=alphabetic,maxbibnames=50]{biblatex}
\addbibresource{/home/nqbh/reference/bib.bib}
\usepackage[utf8]{vietnam}
\usepackage{tocloft}
\renewcommand{\cftsecleader}{\cftdotfill{\cftdotsep}}
\usepackage[colorlinks=true,linkcolor=blue,urlcolor=red,citecolor=magenta]{hyperref}
\usepackage{amsmath,amssymb,amsthm,enumitem,float,graphicx,mathtools,tikz}
\usetikzlibrary{angles,calc,intersections,matrix,patterns,quotes,shadings}
\allowdisplaybreaks
\newtheorem{assumption}{Assumption}
\newtheorem{baitoan}{}
\newtheorem{cauhoi}{Câu hỏi}
\newtheorem{conjecture}{Conjecture}
\newtheorem{corollary}{Corollary}
\newtheorem{dangtoan}{Dạng toán}
\newtheorem{definition}{Definition}
\newtheorem{dinhly}{Định lý}
\newtheorem{dinhnghia}{Định nghĩa}
\newtheorem{example}{Example}
\newtheorem{ghichu}{Ghi chú}
\newtheorem{hequa}{Hệ quả}
\newtheorem{hypothesis}{Hypothesis}
\newtheorem{lemma}{Lemma}
\newtheorem{luuy}{Lưu ý}
\newtheorem{nhanxet}{Nhận xét}
\newtheorem{notation}{Notation}
\newtheorem{note}{Note}
\newtheorem{principle}{Principle}
\newtheorem{problem}{Problem}
\newtheorem{proposition}{Proposition}
\newtheorem{question}{Question}
\newtheorem{remark}{Remark}
\newtheorem{theorem}{Theorem}
\newtheorem{vidu}{Ví dụ}
\usepackage[left=1cm,right=1cm,top=5mm,bottom=5mm,footskip=4mm]{geometry}
\def\labelitemii{$\circ$}
\DeclareRobustCommand{\divby}{%
	\mathrel{\vbox{\baselineskip.65ex\lineskiplimit0pt\hbox{.}\hbox{.}\hbox{.}}}%
}
\def\labelitemii{$\circ$}
\setlist[itemize]{leftmargin=*}
\setlist[enumerate]{leftmargin=*}

\title{Problem: Derivative -- Bài Tập: Đạo Hàm}
\author{Nguyễn Quản Bá Hồng\footnote{A Scientist {\it\&} Creative Artist Wannabe. E-mail: {\tt nguyenquanbahong@gmail.com}. Bến Tre City, Việt Nam.}}
\date{\today}

\begin{document}
	\maketitle
\begin{abstract}
	This text is a part of the series {\it Some Topics in Elementary STEM \& Beyond}:
	
	{\sc url}: \url{https://nqbh.github.io/elementary_STEM}.
	
	Latest version:
	\begin{itemize}
		\item {\it Problem: Derivative -- Bài Tập: Đạo Hàm}.
		
		PDF: {\sc url}: \url{https://github.com/NQBH/elementary_STEM_beyond/blob/main/elementary_mathematics/grade_11/derivative/problem/NQBH_derivative_problem.pdf}.
		
		\TeX: {\sc url}: \url{https://github.com/NQBH/elementary_STEM_beyond/blob/main/elementary_mathematics/grade_11/derivative/problem/NQBH_derivative_problem.tex}.
		\item {\it Problem \& Solution: Derivative -- Bài Tập \& Lời Giải: Đạo Hàm}.
		
		PDF: {\sc url}: \url{https://github.com/NQBH/elementary_STEM_beyond/blob/main/elementary_mathematics/grade_11/derivative/solution/NQBH_derivative_solution.pdf}.
		
		\TeX: {\sc url}: \url{https://github.com/NQBH/elementary_STEM_beyond/blob/main/elementary_mathematics/grade_11/derivative/solution/NQBH_derivative_solution.tex}.
	\end{itemize}
\end{abstract}
\tableofcontents

%------------------------------------------------------------------------------%

\section{Basic}
\textbf{\textsf{Resources -- Tài nguyên.}}
\begin{enumerate}
	\item \cite{SGK_Toan_11_dai_so_giai_tich_co_ban}. {\sc Trần Văn Hạo, Vũ Tuấn, Đào Ngọc Nam, Lê Văn Tiến, Vũ Viết Yên}. {\it Đại Số \& Giải Tích 11}.
	
	\item \cite{SBT_Toan_11_dai_so_giai_tich_co_ban}. {\sc Vũ Tuấn, Trần Văn Hạo, Đào Ngọc Nam, Lê Văn Tiến, Vũ Viết Yên}. {\it Bài Tập Đại Số \& Giải Tích 11}.
	
	\item \cite{SGK_Toan_11_dai_so_giai_tich_nang_cao}. {\sc Đoàn Quỳnh, Nguyễn Huy Đoan, Nguyễn Xuân Liêm, Nguyễn Khắc Minh, Đặng Hùng Thắng}. {\it Đại Số \& Giải Tích 11 nâng cao}.
	
	\item \cite{SGK_Toan_11_CD_tap_2}. {\sc Đỗ Đức Thái, Phạm Xuân Chung, Nguyễn Sơn Hà, Nguyễn Thị Phương Loan, Phạm Sỹ Nam, Phạm Minh Phương}. {\it Toán 11 Tập 1. Cánh Diều}.
	
	\item \cite{SBT_Toan_11_CD_tap_2}. {\sc Đỗ Đức Thái, Phạm Xuân Chung, Nguyễn Sơn Hà, Nguyễn Thị Phương Loan, Phạm Sỹ Nam, Phạm Minh Phương}. {\it Bài Tập Toán 11 Tập 1. Cánh Diều}.
	
	\item \cite{TLCT_dai_so_giai_tich_11}. {\sc  Đoàn Quỳnh, Trần Nam Dũng, Nguyễn Vũ Lương, Đặng Hùng Thắng}. {\it Tài Liệu Chuyên Toán Đại Số \& Giải Tích 11}.
	
	\item \cite{TLCT_BT_dai_so_giai_tich_11}. {\sc  Đoàn Quỳnh, Trần Nam Dũng, Nguyễn Vũ Lương, Đặng Hùng Thắng}. {\it Tài Liệu Chuyên Toán Bài Tập Đại Số \& Giải Tích 11}.
\end{enumerate}

%------------------------------------------------------------------------------%

\section{Định Nghĩa Đạo Hàm. Ý Nghĩa Hình Học Của Đạo Hàm}
Nếu quỹ đạo chuyển động của 1 vật hay 1 chất điểm được miêu tả bằng hàm số ${\bf x}(t)$ theo thời gian thì vận tốc ${\bf v}(t) = {\bf x}'(t)$ biểu thị độ nhanh chậm của chuyển động tại 1 thời điểm $t$.

\begin{baitoan}[Derivative of polynomials -- Đạo hàm của các đa thức]
	Tính đạo hàm của hàm số đa thức
	\begin{equation}
		\label{polynomial}
		\tag{P}
		P(x;n,{\bf a})\coloneqq\sum_{i=0}^n a_ix^i = a_nx^n + a_{n-1}x^{n-1} + \cdots + a_1x + a_0,
	\end{equation}
	tại $x = x_0$ bằng định nghĩa, với $\deg P(x;n,{\bf a}) = n\in\mathbb{N}$ \& vector chứa các hệ số của đa thức $P(x;n,{\bf a})$ là ${\bf a}\coloneqq(a_0,a_1,\ldots,a_n)\in\mathbb{R}^n\times\mathbb{R}^\star$.
\end{baitoan}

\begin{baitoan}[Derivative of rational function -- Đạo hàm của phân thức]
	Tính đạo hàm của hàm số phân thức
	\begin{equation}
		\label{rational function}
		\tag{Q}
		Q(x;m,n,{\bf a},{\bf b})\coloneqq\frac{\sum_{i=0}^m a_ix^i}{\sum_{i=0}^n b_ix^i} = \frac{a_mx^m + a_{m-1}x^{m-1} + \cdots + a_1x + a_0}{b_nx^n + b_{n-1}x^{n-1} + \cdots + b_1x + b_0},
	\end{equation}
	tại $x = x_0$ bằng định nghĩa.
\end{baitoan}

\begin{baitoan}[Đạo hàm của căn thức]
	Tính đạo hàm của hàm số căn thức $f(x) = \sqrt[n]{x} = x^{\frac{1}{n}}$, với $n\in\mathbb{N}^\star$, tại $x = x_0$ bằng định nghĩa.
\end{baitoan}
Ta có 3 dạng hàm số sơ cấp thường gặp: hàm đa thức $P(x;n,{\bf a})\coloneqq\sum_{i=0}^n a_ix^i$, hàm phân thức $Q(x;m,n,{\bf a},{\bf b})\coloneqq\dfrac{\sum_{i=0}^m a_ix^i}{\sum_{i=0}^n b_ix^i}$, hàm căn thức $R_n(x)\coloneqq\sqrt[n]{x}$.

\begin{baitoan}[\cite{TLCT_BT_dai_so_giai_tich_11}, 1., p. 49]
	Dùng định nghĩa, tính đạo hàm của hàm số tại điểm $x_0$: (a) $y = 2x + 1,x_0 = 2$. (b) $y = x^2 + 3x,x_0 = 1$. (c) $y = ax + b$ tại $x = x_0$. (d) $y = ax^2 + bx + c$ tại $x = x_0$.
\end{baitoan}

\begin{baitoan}[\cite{TLCT_BT_dai_so_giai_tich_11}, 2., p. 49]
	Cho parabol $y = x^2$ \& 2 điểm $A(2,4),B(2 + \Delta x,4 + \Delta y)$ trên parabol đó. (a) Tính hệ số góc của cát tuyến $AB$ biết $\Delta x\in\{1,0.1,0.01\}$. (b) Tính hệ số góc của tiếp tuyến của parabol đã cho tại điểm $A$. (c) Mở rộng cho parabol $y = ax^2 + bx + c$ \& 2 điểm $A(x_0,y_0),B(x_0 + \Delta x,y_0 + \Delta y)$.
\end{baitoan}

\begin{baitoan}[\cite{TLCT_BT_dai_so_giai_tich_11}, 3., p. 49]
	Viết phương trình tiếp tuyến của đồ thị hàm số $y = x^3$ biết: (a) Tiếp tuyến có hoành độ bằng $1$. (b) Tiếp điểm của tung độ bằng $8$. (c) Hệ số góc của tiếp tuyến bằng $3$.
\end{baitoan}

\begin{baitoan}[\cite{TLCT_BT_dai_so_giai_tich_11}, 4., p. 49]
	1 vật rơi tự do có phương trình chuyển động $S = \frac{gt^2}{2}$ với $g\approx9.8{\rm m{\tt/}s^2}$ \& $t$ (s). Tính: (a) Vận tốc trung bình trong khoảng thời gian từ $t$ đến $t + \Delta t$ với độ chính xác $0.001$, biết $t = 5$ \& $\Delta t\in\{0.1,0.001,0.001\}$. (b) Vận tốc tại thời điểm $t = 5$.
\end{baitoan}

\begin{baitoan}[\cite{TLCT_BT_dai_so_giai_tich_11}, 5., p. 49]
	Tính đạo hàm của hàm số $y = \sqrt[3]{x}$ trên $(0,\infty)$.
\end{baitoan}

\begin{baitoan}[\cite{TLCT_BT_dai_so_giai_tich_11}, 6., p. 49]
	Tính đạo hàm của hàm số $y = x|x|$ tại điểm $x_0 = 0$ (nếu có).
\end{baitoan}

\begin{baitoan}[\cite{TLCT_BT_dai_so_giai_tich_11}, 7., p. 49]
	Tính $f'(x)$ với
	\begin{equation}
		f(x) = \left\{\begin{split}
			&2x + 1&&\mbox{if } x < 1,\\
			&x^2 + 2&&\mbox{if } 1\le x\le2,\\
			&x^3 - x^2 - 8x + 10&&\mbox{if } x > 2.
		\end{split}\right.
	\end{equation}
\end{baitoan}

%------------------------------------------------------------------------------%

\section{Differentiation Rules -- Các Quy Tắc Tính Đạo Hàm}

\begin{baitoan}[\cite{TLCT_BT_dai_so_giai_tich_11}, 8., p. 50]
	Tính đạo hàm của hàm số: (a) $y = x^4 - 3x^3 + 5x^2 - 7x + 9$. (b) $y = (x - 1)^5(x + 1)^7$. (c) $y = \dfrac{x^2 + 1}{x^4 + 1}$. (d) $y = (x + 1)^3(x + 2)^4(x + 3)^5$.
\end{baitoan}

\begin{baitoan}[\cite{TLCT_BT_dai_so_giai_tich_11}, 9., p. 50]
	Tính đạo hàm của hàm số: (a) $y = \sqrt{\dfrac{1 - x}{1 + x}}$. (b) $y = \sin x^2 + x\cos x^2$. (c) $y = \ln(x + \sqrt{x^2 + 1})$. (d) $y = (x^3 + x^2 + x + 1)e^{x^2 + x}$.
\end{baitoan}

\begin{baitoan}[\cite{TLCT_BT_dai_so_giai_tich_11}, 10., p. 50]
	Tính đạo hàm của hàm số: (a) $y = \dfrac{\sin x - \cos x}{\sin x + \cos x}$. (b) $y = \dfrac{\sin x - 1}{\sin x + \cos x}$.
\end{baitoan}

\begin{baitoan}[\cite{TLCT_BT_dai_so_giai_tich_11}, 11., p. 50]
	Viết phương trình tiếp tuyến của đồ thị hàm số: (a) $y = \dfrac{x}{x^2 + 1}$ biết hoành độ tiếp điểm là $x_0 = \frac{1}{2}$. (b) $y = \sqrt{x + 2}$ biết tung độ tiếp điểm là $y_0 = 2$.
\end{baitoan}

\begin{baitoan}[\cite{TLCT_BT_dai_so_giai_tich_11}, 12., p. 50]
	Chứng minh hàm số $y = \sin^6x + \cos^6x + 3\sin^2x\cos^2x$ có đạo hàm bằng $0$.
\end{baitoan}

\begin{baitoan}[\cite{TLCT_BT_dai_so_giai_tich_11}, 13., p. 50]
	Viết phương trình tiếp tuyến của parabol $y = x^2$ biết tiếp tuyến đó đi qua điểm $A(0,-1)$.
\end{baitoan}

\begin{baitoan}[\cite{TLCT_BT_dai_so_giai_tich_11}, 14., p. 50]
	1 viên đạn được bắn lên từ mặt đất theo phương thẳng đứng với tốc độ ban đầu $v_0 = 196$ {\rm m{\tt/}s} (bỏ qua sức cản của không khí). Tìm thời điểm tại đó tốc độ của viên đạn bằng $0$. Khi đó viên đạn cách mặt đất bao nhiêu {\rm m}?
\end{baitoan}

%------------------------------------------------------------------------------%

\section{Các định lý giá trị trung bình}

\begin{baitoan}[\cite{TLCT_BT_dai_so_giai_tich_11}, 15., p. 50]
	Cho $a,b,c\in\mathbb{R},2a + 3b + 6c = 0$. Chứng minh phương trình $ax^2 + bx + c = 0$ có ít nhất 1 nghiệm thuộc $(0,1)$.
\end{baitoan}

\begin{baitoan}[\cite{TLCT_BT_dai_so_giai_tich_11}, 16., p. 50]
	Cho $f(x) = x(x - 1)(x - 2)(x - 3)(x - 4)(x - 5)(x - 6)$. Đếm số nghiệm của phương trình $f'(x) = 0$.
\end{baitoan}

\begin{baitoan}[\cite{TLCT_BT_dai_so_giai_tich_11}, 17., p. 51]
	Xét hàm số $f(x)$ liên tục trên đoạn $[a,b]$ có đạo hàm trên $(a,b)$. Giả sử phương trình $f(x) = 0$ có đúng 2 nghiệm $x_1,x_2$ với $x_1\ne x_2$. Chứng minh phương trình $f'(x) = 0$ có nghiệm, hơn nữa biểu thức $f'(x)$ phải đổi dấu.
\end{baitoan}

\begin{baitoan}[\cite{TLCT_BT_dai_so_giai_tich_11}, 18., p. 51]
	Chứng minh $2(\sqrt{n + 1} - \sqrt{n}) < \frac{1}{\sqrt{n}} < 2(\sqrt{n} - \sqrt{n - 1})$, $\forall n\in\mathbb{N}^\star$.
\end{baitoan}

\begin{baitoan}[\cite{TLCT_BT_dai_so_giai_tich_11}, 19., p. 51]
	Cho $0 < a < b$ \& $f$ là 1 hàm liên tục trên $[a,b]$, có đạo hàm trên $(a,b)$. Chứng minh tồn tại $c\in(a,b)$ thỏa $\dfrac{af(b) - bf(a)}{a - b} = f(c) - f'(c)$.
\end{baitoan}

\begin{baitoan}[\cite{TLCT_BT_dai_so_giai_tich_11}, 20., p. 51]
	Tính giới hạn: (a) $\lim_{x\to0} \dfrac{\tan x - \sin x}{x^3}$. (b) $\lim_{x\to0} \dfrac{\sqrt[m]{1 + x} - 1}{\sqrt[n]{1 + x} - 1}$. (c) $\lim_{x\to0} \dfrac{1 - \cos x}{x\sin x}$.
\end{baitoan}

\begin{baitoan}[\cite{TLCT_BT_dai_so_giai_tich_11}, 21., p. 51]
	Tính giới hạn: (a) $\lim_{x\to1} \left(\dfrac{1}{x - 1} - \dfrac{1}{\ln x}\right)$. (b) $\lim_{x\to0} (1 + x)^{\cot x}$.
\end{baitoan}

%------------------------------------------------------------------------------%

\section{2nd-Order Derivative -- Đạo Hàm Cấp 2}

%------------------------------------------------------------------------------%

\section{Vi Phân \& Đạo Hàm Cấp Cao}

\begin{baitoan}[\cite{TLCT_BT_dai_so_giai_tich_11}, 22., p. 51]
	Tính vi phân của hàm số: (a) $y = \sqrt{x^2 + a^2}$. (b) $y = x\sin x$. (c) $y = x^2 + \sin^2x$. (d) $y = e^x\ln x$.
\end{baitoan}

\begin{baitoan}[\cite{TLCT_BT_dai_so_giai_tich_11}, 23., p. 51]
	Làm tròn đến hàng phần nghìn: (a) $\dfrac{1}{0.9995}$. (b) $\ln1.001$. (c) $\cos61^\circ$.
\end{baitoan}

\begin{baitoan}[\cite{TLCT_BT_dai_so_giai_tich_11}, 24., p. 51]
	Chứng minh nếu $f,g$ là 2 hàm số có đạo hàm đến cấp 2 thì $fg$ cũng có đạo hàm đến cấp 2 \& có công thức $(f(x)g(x))'' = f''(x)g(x) + 2f'(x)g'(x) + g''(x)$.
\end{baitoan}

\begin{baitoan}[\cite{TLCT_BT_dai_so_giai_tich_11}, 25., p. 51]
	Tính đạo hàm: (a) $f(x) = x^4 - \cos2x$, tính $f^{(4)}(x)$. (b) $f(x) = \cos^2x$, tính $f^{(5)}(x)$. (c) $f(x) = (x + 10)^6$, tính $f^{(n)}(x)$.
\end{baitoan}

\begin{baitoan}[\cite{TLCT_BT_dai_so_giai_tich_11}, 26., p. 52]
	Vận tốc của 1 chất điểm chuyển động được biểu thị bởi công thức $v(t) = 8t + 3t^2$, với $t > 0$, $t$ được tính bằng giây {\rm s} \& $v(t)$ tính bằng {\rm m{\tt/}s}. Tính gia tốc của chất điểm: (a) Lúc $t = 4$. (b) Lúc vận tốc chuyển động bằng $11$.
\end{baitoan}

\begin{baitoan}[\cite{TLCT_BT_dai_so_giai_tich_11}, 27., p. 52]
	Chứng minh $\forall n\ge1$: (a) Nếu $f(x) = \frac{1}{x}$ thì $f^{(n)}(x) = \dfrac{(-1)^nn!}{x^{n+1}}$. (b) Nếu $f(x) = \cos x$ thì $f^{(n)}(x) = \cos\left(x + \frac{n\pi}{2}\right)$.
\end{baitoan}

\begin{baitoan}[\cite{TLCT_BT_dai_so_giai_tich_11}, 28., p. 52]
	Cho $f(x) = \sqrt{x}$. Tính $f^{(n)}(x)$.
\end{baitoan}

%------------------------------------------------------------------------------%

\section{Miscellaneous}

\begin{baitoan}[\cite{TLCT_BT_dai_so_giai_tich_11}, 29., p. 52]
	Tính $f'(x)$ với
	\begin{equation}
		f(x) = \left\{\begin{split}
			&2x + 1&&\mbox{if } x < 1,\\
			&x^2 + 1&&\mbox{if } 1\le x\le2,\\
			&x^3 - x^2 - 4x + 10&&\mbox{if } x > 2.
		\end{split}\right.
	\end{equation}
\end{baitoan}

\begin{baitoan}[\cite{TLCT_BT_dai_so_giai_tich_11}, 30., p. 52]
	Tính $f'(x) + f(x) + 2$ nếu $f(x) = x\sin2x$.
\end{baitoan}

\begin{baitoan}[\cite{TLCT_BT_dai_so_giai_tich_11}, 31., p. 52]
	Chứng minh nếu $f(x) = 3e^{x^2}$ thì $f'(x) - 2xf(x) + \frac{1}{3}f(0) - f'(0) = 1$.
\end{baitoan}

\begin{baitoan}[\cite{TLCT_BT_dai_so_giai_tich_11}, 32., p. 52]
	Viết phương trình tiếp tuyến của đường cong $y = 4x - x^2$ tại các điểm mà đường cong cắt trục hoành.
\end{baitoan}

\begin{baitoan}[\cite{TLCT_BT_dai_so_giai_tich_11}, 33., p. 52]
	Cho đa thức bậc 4 $P(x)$ thỏa mãn điều kiện $P(x)\ge0$, $\forall x\in\mathbb{R}$. Chứng minh $P(x) + P'(x) + P''(x) + P^{(3)}(x) + P^{(4)}(x)\ge0$, $\forall x\in\mathbb{R}$.
\end{baitoan}

\begin{baitoan}[\cite{TLCT_BT_dai_so_giai_tich_11}, 34., p. 53]
	Áp dụng định lý Rolle cho hàm số $f(x) = e^xP(x)$ để chứng minh nếu đa thức $P(x)$ bậc $n$ có $n$ nghiệm thực phân biệt thì đa thức $P(x) + P'(x)$ cũng có $n$ nghiệm thực phân biệt.
\end{baitoan}

\begin{baitoan}[\cite{TLCT_BT_dai_so_giai_tich_11}, 35., p. 53]
	Cho hàm số $f(x)$ khả vi trên đoạn $[0,1]$ \& $f'(0)f'(1) < 0$. Chứng minh tồn tại $c\in(0,1)$ thỏa $f'(c) = 0$.
\end{baitoan}

\begin{baitoan}[\cite{TLCT_BT_dai_so_giai_tich_11}, 36., p. 53]
	Giả sử $f(x)$ là 1 hàm số lẻ \& khả vi trên $\mathbb{R}$. Chứng minh $f'(x)$ là 1 hàm số chẵn.
\end{baitoan}

\begin{baitoan}[\cite{TLCT_BT_dai_so_giai_tich_11}, 37., p. 53]
	Tính đạo hàm cấp $100$ của hàm số $f(x) = \dfrac{x}{x^2 - 1}$.
\end{baitoan}

\begin{baitoan}[\cite{TLCT_BT_dai_so_giai_tich_11}, 38., p. 53]
	Tính giới hạn: (a) $\lim_{x\to0} \cos^{\frac{1}{2x^2}} x$. (b) $\lim_{x\to0} \cos^{\frac{5}{x}} 3x$.
\end{baitoan}

\begin{baitoan}[\cite{TLCT_BT_dai_so_giai_tich_11}, 39., p. 53]
	Chứng minh: (a) {\rm (Phương trình dao động điều hòa)} Nếu $y = A\sin(\omega t + \varphi) + B\cos(\omega t + \varphi)$ với $A,B,\omega,\varphi$ là 4 hằng số thì $y'' + \omega^2y = 0$. (b) Nếu $y = \sqrt{2x - x^2}$ thì $y^3y'' + 1 = 0$.
\end{baitoan}

\begin{baitoan}[\cite{TLCT_BT_dai_so_giai_tich_11}, 40., p. 53, công thức Newton--Leibnitz]
	Cho $f,g$ là 2 hàm số có đạo hàm đến cấp $n$, chứng minh công thức: $(f(x)g(x))^{(n)} = \sum_{k=0}^n C_n^kf^{(k)}(x)g^{(n-k)}(x)$.
\end{baitoan}

\begin{baitoan}[\cite{TLCT_BT_dai_so_giai_tich_11}, 41., p. 53]
	Cho hàm số $f(x) = \dfrac{x}{x^2 + 1}$. Tính $f^{(100)}(0),f^{(101)}(0)$.
\end{baitoan}

%------------------------------------------------------------------------------%

\printbibliography[heading=bibintoc]
	
\end{document}