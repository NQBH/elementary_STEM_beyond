\documentclass{article}
\usepackage[backend=biber,natbib=true,style=alphabetic,maxbibnames=50]{biblatex}
\addbibresource{/home/nqbh/reference/bib.bib}
\usepackage[utf8]{vietnam}
\usepackage{tocloft}
\renewcommand{\cftsecleader}{\cftdotfill{\cftdotsep}}
\usepackage[colorlinks=true,linkcolor=blue,urlcolor=red,citecolor=magenta]{hyperref}
\usepackage{amsmath,amssymb,amsthm,float,graphicx,mathtools,tikz}
\usetikzlibrary{angles,calc,intersections,matrix,patterns,quotes,shadings}
\allowdisplaybreaks
\newtheorem{assumption}{Assumption}
\newtheorem{baitoan}{}
\newtheorem{cauhoi}{Câu hỏi}
\newtheorem{conjecture}{Conjecture}
\newtheorem{corollary}{Corollary}
\newtheorem{dangtoan}{Dạng toán}
\newtheorem{definition}{Definition}
\newtheorem{dinhluat}{Định luật}
\newtheorem{dinhly}{Định lý}
\newtheorem{dinhnghia}{Định nghĩa}
\newtheorem{example}{Example}
\newtheorem{ghichu}{Ghi chú}
\newtheorem{hequa}{Hệ quả}
\newtheorem{hypothesis}{Hypothesis}
\newtheorem{lemma}{Lemma}
\newtheorem{luuy}{Lưu ý}
\newtheorem{nhanxet}{Nhận xét}
\newtheorem{notation}{Notation}
\newtheorem{note}{Note}
\newtheorem{principle}{Principle}
\newtheorem{problem}{Problem}
\newtheorem{proposition}{Proposition}
\newtheorem{question}{Question}
\newtheorem{remark}{Remark}
\newtheorem{theorem}{Theorem}
\newtheorem{vidu}{Ví dụ}
\usepackage[left=1cm,right=1cm,top=5mm,bottom=5mm,footskip=4mm]{geometry}
\def\labelitemii{$\circ$}
\DeclareRobustCommand{\divby}{%
	\mathrel{\vbox{\baselineskip.65ex\lineskiplimit0pt\hbox{.}\hbox{.}\hbox{.}}}%
}
\def\labelitemii{$\circ$}

\title{Problem: Equation {\it\&} Inequation -- Bài Tập: Phương Trình {\it\&} Bất Phương Trình}
\author{Nguyễn Quản Bá Hồng\footnote{A Scientist {\it\&} Creative Artist Wannabe. E-mail: {\tt nguyenquanbahong@gmail.com}. Bến Tre City, Việt Nam.}}
\date{\today}

\begin{document}
\maketitle
\begin{abstract}
	This text is a part of the series {\it Some Topics in Elementary STEM \& Beyond}:
	
	{\sc url}: \url{https://nqbh.github.io/elementary_STEM}.
	
	Latest version:
	\begin{itemize}
		\item {\it Problem: Equation \& Inequation -- Bài Tập: Phương Trình \& Bất Phương Trình}.
		
		PDF: {\sc url}: \url{.pdf}.
		
		\TeX: {\sc url}: \url{.tex}.
		\item {\it Problem \& Solution: Equation \& Inequation -- Bài Tập \& Lời Giải: Phương Trình \& Bất Phương Trình}.
		
		PDF: {\sc url}: \url{.pdf}.
		
		\TeX: {\sc url}: \url{.tex}.
	\end{itemize}
\end{abstract}
\tableofcontents

%------------------------------------------------------------------------------%

\section{Equation -- Phương Trình}

\begin{dinhnghia}[Graph -- đồ thị]
	{\rm Đồ thị} của hàm số $f:D\to\mathbb{R}$ là tập hợp $G(f)\coloneqq\{(x,f(x))\in\mathbb{R}^2|x\in D\}$.
\end{dinhnghia}

\begin{dinhnghia}[Equation, inequation -- phương trình, bất phương trình]
	Cho 2 hàm số $y = f(x),y = g(x)$ có tập xác định lần lượt là $D_f,D_g\subset\mathbb{R}$. Đặt $D\coloneqq D_f\cap D_g$. Mệnh đề chứa biến ``$f(x) = g(x),f(x) > g(x),f(x)\ge g(x)$'' lần lượt được gọi là {\rm phương trình 1 ẩn, bất phương trình 1 ẩn}, x được gọi là {\rm ẩn số} (hay {\rm ẩn}) \& D được gọi là {\rm tập xác định (TXĐ)} của phương trình, bất phương trình. $x_0\in D$ gọi là 1 {\rm nghiệm} của phương trình $f(x) = g(x)$ nếu ``$f(x_0) = g(x_0)$'' là mệnh đề đúng. $x_0\in D$ lần lượt gọi là 1 {\rm nghiệm} của bất phương trình $f(x) > g(x),f(x)\ge g(x)$ nếu ``$f(x_0) > g(x_0),f(x_0)\ge g(x_0)$'' là mệnh đề đúng.
	
	{\rm Giải} 1 phương trình, 1 bất phương trình là đi tìm tất cả các nghiệm của nó, i.e., 3 tập hợp $S\coloneqq\{x\in D|f(x) = g(x)\},S\coloneqq\{x\in D|f(x) > g(x)\},S\coloneqq\{x\in D|f(x)\ge g(x)\}$ lần lượt được gọi là {\rm tập nghiệm} của phương trình $f(x) = g(x)$, bất phương trình $f(x) > g(x),f(x)\ge g(x)$. Khi $S = \emptyset$, ta nói (bất) phương trình {\rm vô nghiệm}. Nếu $|S| = n\in\mathbb{N}^\star$, ta nói (bất) phương trình có n nghiệm hay số nghiệm của (bất) phương trình bằng n. Nếu $|S| = \infty$, ta nói (bất) phương trình có vô số nghiệm.
\end{dinhnghia}
Xét phương trình cấu tạo bởi các hàm $f(x),f^n(x),\sqrt{f(x)},\sqrt[3]{f(x)},\sqrt[n]{f(x)},|f(x)|$ với $f(x) = ax + b,f(x) = ax^2 + bx + c,f(x) = (dx + e)(ax^2 + bx + c),f(x) = \prod_{i=1}^m (d_ix + e_i)\prod_{i=1}^n (a_ix^2 + b_ix + c_i)$.

\subsection{Phương trình đa thức -- Polynomial equation}
Giải \& biện luận phương trình theo các tham số thực:

\begin{baitoan}[Phương trình bậc nhất 1 ẩn]
	(a) $ax + b = 0$. (b) $ax + b = cx + d$. (c) $\sum_{i=1}^n a_ix + b_i = a_1x + b_1 + a_2x + b_2 + \cdots + a_nx + b_n = 0$. (d) $\sum_{i=1}^n a_ix + b_i = \sum_{i=1}^n c_ix + d_i$, i.e., $a_1x + b_1 + a_2x + b_2 + \cdots + a_nx + b_n = c_1x + d_1 + c_2x + d_2 + \cdots + c_nx + d_n$.
\end{baitoan}

\begin{itemize}
	\item \textit{Problem: 1st-Order Function -- Bài Tập: Hàm Số Bậc Nhất $y = ax + b,a\ne0$}.\\{\sc url}: \url{https://github.com/NQBH/elementary_STEM_beyond/blob/main/elementary_mathematics/grade_8/1st_order_function/problem/NQBH_1st_order_function_problem.pdf}.
	\item \textit{Problem \& Solution: 1st-Order Function -- Bài Tập \& Lời Giải: Hàm Số Bậc Nhất $y = ax + b,a\ne0$}.\\{\sc url}: \url{https://github.com/NQBH/elementary_STEM_beyond/blob/main/elementary_mathematics/grade_8/1st_order_function/solution/NQBH_1st_order_function_solution.pdf}.
\end{itemize}

\begin{baitoan}[Phương trình bậc nhất 1 ẩn với trị tuyệt đối]
	(a) $|ax + b| = c$. (b) $|ax + b| = |cx + d|$. (c) $\sum_{i=1}^n |a_ix + b_i| = |a_1x + b_1| + |a_2x + b_2| + \cdots + |a_nx + b_n| = a$. (d) $\sum_{i=1}^n |a_ix + b_i| = \sum_{i=1}^n |c_ix + d_i|$, i.e., $|a_1x + b_1| + |a_2x + b_2| + \cdots + |a_nx + b_n| = |c_1x + d_1| + |c_2x + d_2| + \cdots + |c_nx + d_n|$.
\end{baitoan}

\begin{baitoan}[Phương trình bậc 2 1 ẩn]
	(a) $ax^2 + bx + c = 0$. (b) $a_1x^2 + b_1x + c_1 = a_2x^2 + b_2x + c_2$. (c) $\sum_{i=1}^n a_ix^2 + b_ix + c_i = a_1x^2 + b_1x + c_1 + a_2x^2 + b_2x + c_2 + \cdots + a_nx^2 + b_nx + c_n = 0$. (d) $\sum_{i=1}^n a_ix^2 + b_ix + c_i = \sum_{i=1}^n d_ix^2 + e_ix + f_i$.
\end{baitoan}

\begin{itemize}
	\item \textit{Problem: 2nd-Order Function. Quadratic Equation -- Bài Tập: Hàm Số Bậc 2 $y = ax^2$. Phương Trình Bậc 2 1 Ẩn $ax^2 + bx + c = 0$}.\\{\sc url}: \url{https://github.com/NQBH/elementary_STEM_beyond/blob/main/elementary_mathematics/grade_9/2nd_order_function/problem/NQBH_2nd_order_function_problem.pdf}.
	\item \textit{Problem \& Solution: 2nd-Order Function. Quadratic Equation -- Bài Tập \& Lời Giải: Hàm Số Bậc 2 $y = ax^2$. Phương Trình Bậc 2 1 Ẩn $ax^2 + bx + c = 0$}.\\{\sc url}: \url{https://github.com/NQBH/elementary_STEM_beyond/blob/main/elementary_mathematics/grade_9/2nd_order_function/solution/NQBH_2nd_order_function_solution.pdf}.
\end{itemize}

\begin{baitoan}[Phương trình bậc 2 1 ẩn với trị tuyệt đối]
	(a) $|ax^2 + bx + c| = d$. (b) $|a_1x^2 + b_1x + c_1| = |a_2x^2 + b_2x + c_2|$. (c) $\sum_{i=1}^n |a_ix^2 + b_ix + c_i| = |a_1x^2 + b_1x + c_1| + |a_2x^2 + b_2x + c_2| + \cdots + |a_nx^2 + b_nx + c_n| = 0$. (d) $\sum_{i=1}^n |a_ix^2 + b_ix + c_i| = \sum_{i=1}^n |d_ix^2 + e_ix + f_i|$.
\end{baitoan}

\subsection{Phương trình phân thức -- algebraic rational fraction equation}

\subsection{Phương trình vô tỷ}

\begin{baitoan}[Phương trình vô tỷ với phương trình bậc nhất 1 ẩn]
	(a) $\sqrt{ax + b} = 0$. (b) $\sqrt{ax + b} = \sqrt{cx + d}$. (c) $\sum_{i=1}^n \sqrt{a_ix + b_i} = \sqrt{a_1x + b_1} + \sqrt{a_2x + b_2} + \cdots + \sqrt{a_nx + b_n} = 0$. (d) $\sum_{i=1}^n \sqrt{a_ix + b_i} = \sum_{i=1}^n \sqrt{c_ix + d_i}$, i.e., $\sqrt{a_1x + b_1} + \sqrt{a_2x + b_2} + \cdots + \sqrt{a_nx + b_n} = \sqrt{c_1x + d_1} + \sqrt{c_2x + d_2} + \cdots + \sqrt{c_nx + d_n}$.
\end{baitoan}

\begin{baitoan}[Phương trình vô tỷ với phương trình bậc 2 1 ẩn]
	(a) $\sqrt{ax^2 + bx + c} = 0$. (b) $\sqrt{a_1x^2 + b_1x + c_1} = \sqrt{a_2x^2 + b_2x + c_2}$. (c) $\sum_{i=1}^n \sqrt{a_ix^2 + b_ix + c_i} = \sqrt{a_1x^2 + b_1x + c_1} + \sqrt{a_2x^2 + b_2x + c_2} + \cdots + \sqrt{a_nx^2 + b_nx + c_n} = 0$. (d) $\sum_{i=1}^n \sqrt{a_ix^2 + b_ix + c_i} = \sum_{i=1}^n \sqrt{d_ix^2 + e_ix + f_i}$.
\end{baitoan}

%------------------------------------------------------------------------------%

\section{Inequation -- Bất Phương Trình}
Gọi $\mathcal{R}$ là 1 trong 4 quan hệ thứ tự $>,<,\ge,\le$.

\begin{baitoan}
	Giải \& biện luận bất phương trình $ax + b\mathcal{R} 0$ theo 2 tham số $a,b\in\mathbb{R}$.
\end{baitoan}

\begin{baitoan}
	Giải \& biện luận bất phương trình $|ax + b|\mathcal{R} c$ theo 2 tham số $a,b,c\in\mathbb{R}$.
\end{baitoan}

\begin{baitoan}
	Giải \& biện luận bất phương trình $ax^2 + bx + c\mathcal{R} 0$ theo 2 tham số $a,b,c\in\mathbb{R}$.
\end{baitoan}

%------------------------------------------------------------------------------%

\section{Miscellaneous}

%------------------------------------------------------------------------------%

\printbibliography[heading=bibintoc]
	
\end{document}