\documentclass{article}
\usepackage[backend=biber,natbib=true,style=alphabetic,maxbibnames=50]{biblatex}
\addbibresource{/home/nqbh/reference/bib.bib}
\usepackage[utf8]{vietnam}
\usepackage{tocloft}
\renewcommand{\cftsecleader}{\cftdotfill{\cftdotsep}}
\usepackage[colorlinks=true,linkcolor=blue,urlcolor=red,citecolor=magenta]{hyperref}
\usepackage{amsmath,amssymb,amsthm,float,graphicx,mathtools,tikz}
\usetikzlibrary{angles,calc,intersections,matrix,patterns,quotes,shadings}
\allowdisplaybreaks
\newtheorem{assumption}{Assumption}
\newtheorem{baitoan}{}
\newtheorem{cauhoi}{Câu hỏi}
\newtheorem{conjecture}{Conjecture}
\newtheorem{corollary}{Corollary}
\newtheorem{dangtoan}{Dạng toán}
\newtheorem{definition}{Definition}
\newtheorem{dinhly}{Định lý}
\newtheorem{dinhnghia}{Định nghĩa}
\newtheorem{example}{Example}
\newtheorem{ghichu}{Ghi chú}
\newtheorem{hequa}{Hệ quả}
\newtheorem{hypothesis}{Hypothesis}
\newtheorem{lemma}{Lemma}
\newtheorem{luuy}{Lưu ý}
\newtheorem{nhanxet}{Nhận xét}
\newtheorem{notation}{Notation}
\newtheorem{note}{Note}
\newtheorem{principle}{Principle}
\newtheorem{problem}{Problem}
\newtheorem{proposition}{Proposition}
\newtheorem{question}{Question}
\newtheorem{remark}{Remark}
\newtheorem{theorem}{Theorem}
\newtheorem{vidu}{Ví dụ}
\usepackage[left=1cm,right=1cm,top=5mm,bottom=5mm,footskip=4mm]{geometry}
\def\labelitemii{$\circ$}
\DeclareRobustCommand{\divby}{%
	\mathrel{\vbox{\baselineskip.65ex\lineskiplimit0pt\hbox{.}\hbox{.}\hbox{.}}}%
}
\def\labelitemii{$\circ$}

\title{Problem: Function {\it\&} Graph  -- Bài Tập: Hàm Số {\it\&} Đồ Thị}
\author{Nguyễn Quản Bá Hồng\footnote{A Scientist {\it\&} Creative Artist Wannabe. E-mail: {\tt nguyenquanbahong@gmail.com}. Bến Tre City, Việt Nam.}}
\date{\today}

\begin{document}
\maketitle
\begin{abstract}
	This text is a part of the series {\it Some Topics in Elementary STEM \& Beyond}:
	
	{\sc url}: \url{https://nqbh.github.io/elementary_STEM}.
	
	Latest version:
	\begin{itemize}
		\item {\it Problem: Function \& Graph  -- Bài Tập: Hàm Số \& Đồ Thị}.
		
		PDF: {\sc url}: \url{https://github.com/NQBH/elementary_STEM_beyond/blob/main/elementary_mathematics/grade_10/function_graph/problem/NQBH_function_graph_problem.pdf}.
		
		\TeX: {\sc url}: \url{https://github.com/NQBH/elementary_STEM_beyond/blob/main/elementary_mathematics/grade_10/function_graph/problem/NQBH_function_graph_problem.tex}.
		\item {\it Problem \& Solution: Function \& Graph  -- Bài Tập \& Lời Giải: Hàm Số \& Đồ Thị}.
		
		PDF: {\sc url}: \url{https://github.com/NQBH/elementary_STEM_beyond/blob/main/elementary_mathematics/grade_10/function_graph/solution/NQBH_function_graph_solution.pdf}.
		
		\TeX: {\sc url}: \url{https://github.com/NQBH/elementary_STEM_beyond/blob/main/elementary_mathematics/grade_10/function_graph/solution/NQBH_function_graph_solution.tex}.
	\end{itemize}
\end{abstract}
\tableofcontents

%------------------------------------------------------------------------------%

\textbf{\textsf{Resources -- Tài nguyên.}}
\begin{enumerate}
	\item \cite{Hai_Hung_Thu_Tung_ncpt_Toan_10_tap_2}. {\sc Phan Việt Hải, Trần Quang Hùng, Ninh Văn Thu, Phạm Đình Tùng}. {\it Nâng Cao \& Phát Triển Toán 10. Tập 2}.
\end{enumerate}

\section{General Function -- Đại Cương Về Hàm Số}

\subsection*{Abbreviations -- Viết tắt}

\begin{enumerate}
	\item TXĐ: Tập xác định.
\end{enumerate}

\begin{baitoan}[\cite{Hai_Hung_Thu_Tung_ncpt_Toan_10_tap_2}, VD1, p. 5]
	Công thức tính chu vi \& diện tích hình tròn $P = 2\pi r,S = \pi r^2$ có là hàm số không?
\end{baitoan}

\begin{baitoan}[\cite{Hai_Hung_Thu_Tung_ncpt_Toan_10_tap_2}, VD2, p. 6]
	Tìm {\rm TXĐ} của hàm số $f(x) = \sqrt{x + \sqrt{x}}$.
\end{baitoan}

\begin{baitoan}
	Biện luận theo 4 tham số $a,b,c,d\in\mathbb{R}$ {\rm TXĐ} của hàm số $f(x) = \sqrt{ax + \sqrt{bx + c} + d}$.
\end{baitoan}

\begin{baitoan}[\cite{Hai_Hung_Thu_Tung_ncpt_Toan_10_tap_2}, VD3, p. 6]
	Chứng minh hàm số $f(x) = x^2$ đồng biến trên $[0,+\infty)$ \& nghịch biến trên $(-\infty,0]$.
\end{baitoan}

\begin{baitoan}
	Biện luận theo 3 tham số $a,b,c\in\mathbb{R}$ các khoảng đồng biến, nghịch biến của hàm số $y = f(x) = ax^2 + bx + c$.
\end{baitoan}

\begin{baitoan}[\cite{Hai_Hung_Thu_Tung_ncpt_Toan_10_tap_2}, VD4, p. 7]
	Chứng minh hàm $f(x) = \sqrt{2 - x} + \sqrt{2 + x}$ là hàm chẵn trên {\rm TXĐ} của nó.
\end{baitoan}

\begin{baitoan}[\cite{Hai_Hung_Thu_Tung_ncpt_Toan_10_tap_2}, VD5, p. 7]
	Chứng minh hàm $f(x) = (e^x + e^{-x})\cos x$ là hàm chẵn trên {\rm TXĐ} của nó.
\end{baitoan}

\begin{baitoan}[\cite{Hai_Hung_Thu_Tung_ncpt_Toan_10_tap_2}, VD6, p. 7]
	Chứng minh hàm $f(x) = \cos x$ có chu kỳ cơ sở là $2\pi$.
\end{baitoan}
Tồn tại các hàm tuần hoàn nhưng không có chù kỳ cơ sở.

\begin{baitoan}[\cite{Hai_Hung_Thu_Tung_ncpt_Toan_10_tap_2}, VD7, p. 7]
	Tìm chu kỳ cơ sở của hàm Dirichlet
	\begin{equation}
		f(x) = \chi_\mathbb{Q} = \left\{\begin{split}
			&1&&\mbox{if } x\in\mathbb{Q},\\
			&0&&\mbox{if } x\in\mathbb{R}\backslash\mathbb{Q}.
		\end{split}\right.
	\end{equation}
\end{baitoan}

\begin{baitoan}[\cite{Hai_Hung_Thu_Tung_ncpt_Toan_10_tap_2}, VD8, p. 7]
	Cho $a,b,c,d\in\mathbb{R}^\star$. Chứng minh hàm số $f(x) = a\sin cx + b\cos dx$ tuần hoàn trên $\mathbb{R}$ khi \& chỉ khi $\frac{c}{d}\in\mathbb{Q}$.
\end{baitoan}

\begin{baitoan}[\cite{Hai_Hung_Thu_Tung_ncpt_Toan_10_tap_2}, VD9, p. 7]
	Chứng minh hàm số $f(x) = \cos x + \cos x\sqrt{2}$ không tuần hoàn trên $\mathbb{R}$.
\end{baitoan}

\begin{baitoan}[\cite{Hai_Hung_Thu_Tung_ncpt_Toan_10_tap_2}, VD10, p. 8]
	Cho 2 hàm số $f(x) = x^2 + 5,g(x) = x^3 + 2x^2 + 1$. Tính $f(g(x))$.
\end{baitoan}

\begin{baitoan}[\cite{Hai_Hung_Thu_Tung_ncpt_Toan_10_tap_2}, 17.1., p. 8]
	Tìm {\rm TXĐ} của hàm số: $f(x) = \dfrac{|x + 1|}{(x - 3)\sqrt{2x - 1}},g(x) = \dfrac{\sqrt{5 - 3|x|}}{x^2 + 4x + 3},h(x) = \dfrac{x + 4}{\sqrt{x^2 - 16}}$.
\end{baitoan}

\begin{baitoan}[\cite{Hai_Hung_Thu_Tung_ncpt_Toan_10_tap_2}, 17.2., p. 8]
	2 hàm số $f(x) = \dfrac{|x|}{x},g(x) = 1$ có bằng nhau không?
\end{baitoan}

\begin{baitoan}[\cite{Hai_Hung_Thu_Tung_ncpt_Toan_10_tap_2}, 17.3., p. 8]
	Cho hàm số $f(x) = \dfrac{x}{\sqrt{x^2 + 1}}$. Tính $f_n(x)$ với $f_1(x)\coloneqq f(x),f_n(x)\coloneqq f(f_{n-1}(x))$.
\end{baitoan}

\begin{baitoan}[\cite{Hai_Hung_Thu_Tung_ncpt_Toan_10_tap_2}, 17.4., p. 8]
	Cho $f(x)$ là 1 hàm bất kỳ với {\rm TXĐ} $\mathbb{R}$. Chứng minh $f(x)$ luôn biểu diễn được 1 cách duy nhất dưới dạng tổng của 1 hàm số chẵn \& 1 hàm số lẻ.
\end{baitoan}

\begin{baitoan}[\cite{Hai_Hung_Thu_Tung_ncpt_Toan_10_tap_2}, 17.5., p. 8]
	Cho $f(x)$ là 1 hàm tuần hoàn bất kỳ với {\rm TXĐ} $\mathbb{R}$ \& chu kỳ cơ sở là $T$. Tìm chu kỳ cơ sở của hàm số $y(x) = f(ax + b)$, $a,b\in\mathbb{R}$, $a > 0$.
\end{baitoan}

\begin{baitoan}[\cite{Hai_Hung_Thu_Tung_ncpt_Toan_10_tap_2}, 17.6., p. 8]
	Cho $f(x)$ là 1 hàm bất kỳ với {\rm TXĐ} $D$. Giả sử tồn tại $a\in\mathbb{R}^\star$ thỏa $f(x + a) = \dfrac{f(x) - 1}{f(x) + 1}$. Chứng minh $f(x)$ là hàm tuần hoàn.
\end{baitoan}

\begin{baitoan}[\cite{Hai_Hung_Thu_Tung_ncpt_Toan_10_tap_2}, 17.7., p. 8]
	Cho $a\in\mathbb{R}^\star$, $f:\mathbb{R}_+\to\mathbb{R}$ thỏa $f(x + a) = \frac{1}{2} + \sqrt{f(x) - f(x)^2}$, $\forall x > 0$. Chứng minh $f(x)$ là hàm tuần hoàn.
\end{baitoan}

\begin{baitoan}[\cite{Hai_Hung_Thu_Tung_ncpt_Toan_10_tap_2}, 17.8., p. 8]
	Cho hàm số $f(x)$ xác định trên $\mathbb{R}$, thỏa $f(x + 3)\le f(x) + 3$, $f(x + 2)\ge f(x) + 2$, $\forall x\in\mathbb{R}$. Chứng minh $g(x)\coloneqq f(x) - x$ là hàm tuần hoàn.
\end{baitoan}

%------------------------------------------------------------------------------%

\section{2nd-Order Function -- Hàm Số Bậc 2}
\fbox{1} {\sf Định nghĩa.} {\it Hàm số bậc 2} là hàm số có dạng $y = ax^2 + bx + c$ với $a,b,c\in\mathbb{R}$, $a\ne0$: 3 hệ số, có TXĐ $D = \mathbb{R}$. \fbox{2} Bảng biến thiên của hàm số bậc 2: Khi $a > 0$, $x:-\infty\to-\dfrac{b}{2a}\to+\infty$, $y:+\infty\searrow-\dfrac{\Delta}{4a}\nearrow+\infty$. Khi $a < 0$, $x:-\infty\to-\dfrac{b}{2a}\to+\infty$, $y:-\infty\nearrow-\dfrac{\Delta}{4a}\searrow-\infty$. \fbox{3} Tính chất của đồ thị của hàm số bậc 2: (i) có đỉnh $I\left(-\dfrac{b}{2a},-\dfrac{\Delta}{4a}\right)$. (ii) Quay bề lõm lên trên khi $a > 0$, quay bề bõm xuống dưới khi $a < 0$. (iii) Có trục đối xứng là đường thẳng $x = -\dfrac{b}{2a}$ đi qua đỉnh $I$ \& song song với trục tung $Oy$. \fbox{4} {\sf GTLN, GTNN.} 

\begin{baitoan}[\cite{Hai_Hung_Thu_Tung_ncpt_Toan_10_tap_2}, VD1, p. 10]
	Đếm số giá trị $m\in\mathbb{N}^\star$ để hàm số $y = x^2 - 2(m + 1)x - 3$ đồng biến trên khoảng $(4,2018)$.
\end{baitoan}

\begin{baitoan}[\cite{Hai_Hung_Thu_Tung_ncpt_Toan_10_tap_2}, VD2, p. 11]
	Cho parabol $(P)$ đi qua $A(-1,4),B(3,4)$. Tìm phương trình trục đối xứng của $(P)$.
\end{baitoan}

\begin{baitoan}[\cite{Hai_Hung_Thu_Tung_ncpt_Toan_10_tap_2}, VD3, p. 11]
	Tìm {\rm GTLN, GTNN} của hàm số $y = 5x^2 + 2x + 1$ trên đoạn $[-2,2]$.
\end{baitoan}

\begin{baitoan}[\cite{Hai_Hung_Thu_Tung_ncpt_Toan_10_tap_2}, VD4, p. 11]
	Cho 2 parabol có phương trình $y = x^2 + x + 1,y = 2x^2 - x - 2$. Biết 2 parabol cắt nhau tại 2 điểm $A,B$ với $x_A < x_B$. Tính $AB$.
\end{baitoan}

\begin{baitoan}[\cite{Hai_Hung_Thu_Tung_ncpt_Toan_10_tap_2}, VD5, p. 11]
	Đếm số giá trị $m\in\mathbb{Z}$ trong nửa khoảng $[-10,-4)$ để đường thẳng $d:y = -(m + 1)x + m + 2$ cắt parabol $(P):y = x^2 + x - 2$ tại 2 điểm phân biệt nằm về cùng 1 phía đối với trục tung.
\end{baitoan}

\begin{baitoan}[\cite{Hai_Hung_Thu_Tung_ncpt_Toan_10_tap_2}, VD6, p. 12]
	Đếm số giá trị $m\in\mathbb{Z}$ để phương trình $x^2 - 2|x| + 1 - m = 0$ có 4 nghiệm phân biệt.
\end{baitoan}

\begin{baitoan}[\cite{Hai_Hung_Thu_Tung_ncpt_Toan_10_tap_2}, VD7, p. 13]
	Biết $S = (a,b)$ là tập hợp tất cả các giá trị của tham số $m$ để đường thẳng $y = m$ cắt đồ thị hàm số $y = |x^2 - 4x + 3|$ tại 4 điểm phân biệt. Tìm $a + b$.
\end{baitoan}

\begin{baitoan}[\cite{Hai_Hung_Thu_Tung_ncpt_Toan_10_tap_2}, VD8, p. 13]
	Tìm $m\in\mathbb{R}$ để hàm số $f(x) = x^2 + (2m - 1)x + m^2$ luôn nhận giá trị dương.
\end{baitoan}

\begin{baitoan}[\cite{Hai_Hung_Thu_Tung_ncpt_Toan_10_tap_2}, VD9, p. 13]
	Tìm $m\in\mathbb{R}$ để hàm số $f(x) = (m - 1)x^2 + 2x + 1$ luôn nhận giá trị âm.
\end{baitoan}

\begin{baitoan}[\cite{Hai_Hung_Thu_Tung_ncpt_Toan_10_tap_2}, VD10, p. 14]
	Tìm $m\in\mathbb{R}$ để hàm số $f(x) = (m - 1)x^2 + (2m + 1)x + m + 1\le0$, $\forall x\in\mathbb{R}$.
\end{baitoan}

\begin{baitoan}[\cite{Hai_Hung_Thu_Tung_ncpt_Toan_10_tap_2}, VD11, p. 14]
	Tìm $m\in\mathbb{R}$ để $\dfrac{(m + 2)x^2 - 2(m - 1)x + 4m + 1}{2x^2 + 1} > 1$, $\forall x\in\mathbb{R}$.
\end{baitoan}

\begin{baitoan}[\cite{Hai_Hung_Thu_Tung_ncpt_Toan_10_tap_2}, VD12, p. 14]
	Tìm nghiệm nguyên của hệ bất phương trình
	\begin{equation*}
		\left\{\begin{split}
			x^2 - 4 &< 0,\\
			(x - 1)(x^2 + 5x + 4)&\ge0.
		\end{split}\right.
	\end{equation*}
\end{baitoan}

\begin{baitoan}[\cite{Hai_Hung_Thu_Tung_ncpt_Toan_10_tap_2}, 18.1., p. 15]
	Gọi $M$ là điểm cố định mà parabol $(P_m):y = x^2 + 3mx + 6m + 1$ luôn đi qua với mọi giá trị của tham số $m\in\mathbb{R}$. Tính tổng khoảng cách từ $M$ đến 2 trục tọa độ.
\end{baitoan}

\begin{baitoan}[\cite{Hai_Hung_Thu_Tung_ncpt_Toan_10_tap_2}, 18.2., p. 15]
	Cho parabol $(P):y = x^2 - 2(m - 1)x - 2$ với tham số $m\in\mathbb{R}$. Tìm quỹ tích đỉnh của $(P)$ khi $m$ thay đổi.
\end{baitoan}

\begin{baitoan}[\cite{Hai_Hung_Thu_Tung_ncpt_Toan_10_tap_2}, 18.3., p. 15]
	Cho parabol $(P):y = x^2 - mx$ \& đường thẳng $(d):y = (m + 2)x + 1$ với tham số $m\in\mathbb{R}$. Khi $(P),(d)$ cắt nhau tại 2 điểm $M\ne N$, tìm tập hợp trung điểm $I$ của đoạn thẳng $MN$.
\end{baitoan}

\begin{baitoan}[\cite{Hai_Hung_Thu_Tung_ncpt_Toan_10_tap_2}, 18.4., p. 15]
	1 chiếc ăng-ten chảo parabol có chiều cao $h = 0.5$ {\rm m} \& đường kính miệng $d = 4$ {\rm m}. Mặt cắt qua trục là 1 parabol dạng $y = ax^2$. Biết $a = \frac{m}{n}$ với $m,n\in\mathbb{N}^\star$ nguyên tố cùng nhau. Tính $m - n$.
\end{baitoan}

\begin{baitoan}[\cite{Hai_Hung_Thu_Tung_ncpt_Toan_10_tap_2}, 18.5., p. 15]
	Khi 1 quả bóng được đá lên, nó sẽ đạt đến độ cao nào đó rồi rơi xuống. Biết quỹ đạo của quả bóng là 1 cung parabol trong mặt phẳng với hệ tọa độ $Oth$ với $t$ là thời gian, tính bằng giây, kể từ khi quả bóng được đá lên; $h$ là độ cao, tính bằng mét, của quả bóng. Giả thiết quả bóng được đá lên từ độ cao {\rm1.2 m}. Sau đó {\rm1 s}, nó đạt độ cao {\rm8.5 m} \& {\rm2 s} sau khi đá lên, nó đạt độ cao {\rm6 m}. Sau bao lâu thì quả bóng sẽ chạm đất kể từ khi được đá lên?
\end{baitoan}

\begin{baitoan}[\cite{Hai_Hung_Thu_Tung_ncpt_Toan_10_tap_2}, 18.6., p. 15]
	Cho parabol $(P):y = x^2 - 3mx + m^2 + 1$ \& đường thẳng $(d):y = mx + m^2$, $m$ là tham số. Đếm số giá trị $m\in\mathbb{Z}$ để đường thẳng $(d)$ cắt parabol $(P)$ tại 2 điểm phân biệt có hoành độ $x_1,x_2$ thỏa mãn $|\sqrt{x_1} - \sqrt{x_2}| = 1$.
\end{baitoan}

\begin{baitoan}[\cite{Hai_Hung_Thu_Tung_ncpt_Toan_10_tap_2}, 18.7., p. 15]
	Đếm số giá trị $m\in\mathbb{R}$ để {\rm GTNN} của hàm số $f(x) = x^2 + (2m + 1)x + m^2 - 1$ trên đoạn $[0,1]$ là $1$.
\end{baitoan}

\begin{baitoan}[\cite{Hai_Hung_Thu_Tung_ncpt_Toan_10_tap_2}, 18.8., p. 16]
	Cho hàm số $f(x) = x^2 - 2\left(m + \frac{1}{m}\right)x + m$. Đặt $m\coloneqq\min_{x\in[-1,1]} f(x)$, $M\coloneqq\max_{x\in[-1,1]} f(x)$, $S$ là tập hợp tất cả các giá trị $m\in\mathbb{R}$ để $M - m = 8$. Tính tổng bình phương của các phần tử thuộc $S$.
\end{baitoan}

\begin{baitoan}[\cite{Hai_Hung_Thu_Tung_ncpt_Toan_10_tap_2}, 18.9., p. 16]
	Đếm số giá trị $m\in\mathbb{Z}$ thuộc $[1,2018]$ để bất phương trình $x^2 + 2x|x + 2| - 2\le m$ thỏa mãn $\forall x\in[-4,1]$.
\end{baitoan}

\begin{baitoan}[\cite{Hai_Hung_Thu_Tung_ncpt_Toan_10_tap_2}, 18.10., p. 16]
	Biết tập hợp tất cả các giá trị của tham số $m\in\mathbb{R}$ để phương trình $|x|\sqrt{x^2  4|x| + 4} = m$ có 6 nghiệm phân biệt là khoảng $(a,b)$. Tính $a + b$.
\end{baitoan}

\begin{baitoan}[\cite{Hai_Hung_Thu_Tung_ncpt_Toan_10_tap_2}, 18.11., p. 16]
	Cho hàm số $f(x) = \sqrt{(m + 4)x^2 - (m - 4)x - 2m + 1}$. Tìm tất cả $m\in\mathbb{R}$ để {\rm TXĐ} của $f(x)$ là $\mathbb{R}$.
\end{baitoan}

\begin{baitoan}[\cite{Hai_Hung_Thu_Tung_ncpt_Toan_10_tap_2}, 18.12., p. 16]
	Đếm số giá trị $m\in\mathbb{Z}$ để hàm số $y = \sqrt{x^2 - 2mx - 2m + 3}$ có {\rm TXĐ} là $\mathbb{R}$.
\end{baitoan}

\begin{baitoan}[\cite{Hai_Hung_Thu_Tung_ncpt_Toan_10_tap_2}, 18.13., p. 16]
	Tìm $m\in\mathbb{R}$ để bất phương trình $\dfrac{(m - 3)x^2 - 2(m - 1)x + 4m - 1}{3x^2 + 1}\le-1$ vô nghiệm.
\end{baitoan}

\begin{baitoan}[\cite{Hai_Hung_Thu_Tung_ncpt_Toan_10_tap_2}, 18.14., p. 16]
	Tìm $m\in\mathbb{R}$ để bất phương trình nghiệm đúng $\forall x\in\mathbb{R}$: (a) $\dfrac{x^2 + mx - 1}{2x^2 - 2x + 3} < 1$.\\(b) $\dfrac{-x^2 + 8x - 20}{mx^2 + 2(m + 1)x + 9m + 4} > 0$.
\end{baitoan}

%------------------------------------------------------------------------------%

\section{Solvable Equations via Quadratic Equations -- Phương Trình Quy Về Phương Trình Bậc 2}

\begin{baitoan}[\cite{Hai_Hung_Thu_Tung_ncpt_Toan_10_tap_2}, VD2, p. 19]
	Giải \& biện luận phương trình $x^2 - mx + 1 = 0$.
\end{baitoan}

\begin{baitoan}[\cite{Hai_Hung_Thu_Tung_ncpt_Toan_10_tap_2}, VD3, p. 19]
	Giải \& biện luận phương trình $mx^2 - 2(m + 1)x + 2 = 0$.
\end{baitoan}

\begin{baitoan}[\cite{Hai_Hung_Thu_Tung_ncpt_Toan_10_tap_2}, VD4, p. 19]
	Tìm $m\in\mathbb{R}$ để phương trình $x^2 + 2mx + m(m - 1) = 0$ có: (a) 2 nghiệm trái dấu. (b) 2 nghiệm cùng dấu.
\end{baitoan}

\begin{baitoan}[\cite{Hai_Hung_Thu_Tung_ncpt_Toan_10_tap_2}, VD5, p. 19]
	Giả sử phương trình bậc 2 $x^2 - Sx + P = 0$ có 2 nghiệm $x_1,x_2$. Biểu diễn biểu thức qua hệ số $S,P$: (a) $x_1^2 + x_2^2$. (b) $x_1^3 + x_2^3$. (c) $S_n(x_1,x_2)\coloneqq x_1^n + x_2^n$ với $n\in\mathbb{N}$.
\end{baitoan}

\begin{baitoan}[\cite{Hai_Hung_Thu_Tung_ncpt_Toan_10_tap_2}, VD6, p. 19]
	Giải phương trình trùng phương $x^4 - 6x^2 + 8 = 0$.
\end{baitoan}

\begin{baitoan}[\cite{Hai_Hung_Thu_Tung_ncpt_Toan_10_tap_2}, VD7, p. 20]
	Giải \& biện luận phương trình $x^4 - 2(m + 1)x^2 + (m - 1)^2 = 0$.
\end{baitoan}

\begin{baitoan}[\cite{Hai_Hung_Thu_Tung_ncpt_Toan_10_tap_2}, 19.1., p. 20]
	Giải \& biện luận phương trình: (a) $mx^2 + 3x - 1 = 0$. (b) $x^2 - 2x + m - 1 = 0$. (c) $mx^2 + 2(m - 1)x + m + 1 = 0$. (d) $x^2 + mx + 2 = 0$.
\end{baitoan}

\begin{baitoan}[\cite{Hai_Hung_Thu_Tung_ncpt_Toan_10_tap_2}, 19.2., p. 20]
	Cho phương trình $x^2 + mx - 8 = 0$. Tìm $m\in\mathbb{R}$ để phương trình có 2 nghiệm $x_1,x_2$ thỏa: (a) $x_1^2 + x_2^2$ đạt {\rm GTNN}. (b) $(x_1^2 - 1)(x_2^2 - 1)$ đạt {\rm GTLN}.
\end{baitoan}

\begin{baitoan}[\cite{Hai_Hung_Thu_Tung_ncpt_Toan_10_tap_2}, 19.3., p. 20]
	Biện luận theo tham số $m\in\mathbb{R}$ số nghiệm của phương trình trùng phương: (a) $mx^4 - 2x^2 + 1 = 0$. (b) $x^4 - 2x^2 + m = 0$.
\end{baitoan}

\begin{baitoan}[\cite{Hai_Hung_Thu_Tung_ncpt_Toan_10_tap_2}, 19.4., p. 20]
	Giải \& biện luận phương trình: (a) $mx^4 - 2x^2 + 1 = 0$. (b) $x^4 - 2x^2 + m = 0$.
\end{baitoan}

\begin{baitoan}[\cite{Hai_Hung_Thu_Tung_ncpt_Toan_10_tap_2}, 19.5., p. 21]
	Giải \& biện luận phương trình: (a) $x^4 - 2mx^2 + 2m - 1 = 0$. (b) $(m - 3)x^4 - 2(m - 1)x^2 + m = 0$. (c) $x^4 - 2(m - 4)x^2 + m^2 - 8 = 0$.
\end{baitoan}

\begin{baitoan}[\cite{Hai_Hung_Thu_Tung_ncpt_Toan_10_tap_2}, 19.6., p. 21]
	Giải phương trình: (a) $(x + 1)(x + 2)(x + 3)(x + 4) = 3$. (b) $x^4 + 3x^3 - 2x^2 + 3x + 1 = 0$.
\end{baitoan}

%------------------------------------------------------------------------------%

\section{Ứng Dụng của Hàm Số Trong Chứng Minh Bất Đẳng Thức \& Tìm GTLN, GTNN}

\begin{baitoan}[\cite{Hai_Hung_Thu_Tung_ncpt_Toan_10_tap_2}, VD1, p. 21]
	Cho $x,y,z\in[0,2]$. Chứng minh $2(x + y + z)\le xy + yz + zx + 4$.
\end{baitoan}

\begin{baitoan}[\cite{Hai_Hung_Thu_Tung_ncpt_Toan_10_tap_2}, VD2, p. 22]
	Cho $x,y,z\ge0$ thỏa $x + y + z = 1$. Chứng minh $xy + yz + zx - 2xyz\le\frac{7}{27}$.
\end{baitoan}

\begin{baitoan}[\cite{Hai_Hung_Thu_Tung_ncpt_Toan_10_tap_2}, VD3, p. 22]
	Cho $a,b,c\ge0$ thỏa $a + b + c = 1$. Chứng minh $5(a^2 + b^2 + c^2) - 6(a^3 + b^3 + c^3)\le1$.
\end{baitoan}

\begin{baitoan}[\cite{Hai_Hung_Thu_Tung_ncpt_Toan_10_tap_2}, VD4, p. 23]
	Cho $x,y\in\mathbb{R}$ thỏa $x^2 + y^2 + xy - 6(x + y) + 5 = 0$. Tìm {\rm GTLN, GTNN} cảu biểu thức $A = 2x + y$.
\end{baitoan}

\begin{baitoan}[\cite{Hai_Hung_Thu_Tung_ncpt_Toan_10_tap_2}, VD5, p. 23]
	Cho $x,y,z\ge0$ thỏa $x + y + z = 1$. Tìm {\rm GTLN} của biểu thức $A = 9xy + 10yz + 11zx$.
\end{baitoan}

\begin{baitoan}[\cite{Hai_Hung_Thu_Tung_ncpt_Toan_10_tap_2}, VD6, p. 23]
	Chứng minh $\sin\dfrac{A}{2} + \sin\dfrac{B}{2} + \sin\dfrac{C}{2}\le\dfrac{3}{2}$, $\forall\Delta ABC$.
\end{baitoan}

\begin{baitoan}[\cite{Hai_Hung_Thu_Tung_ncpt_Toan_10_tap_2}, VD7, p. 24]
	Chứng minh $\dfrac{\cos A}{x} + \dfrac{\cos B}{y} + \dfrac{\cos C}{z}\le\dfrac{x^2 + y^2 + z^2}{2xyz}$, $\forall\Delta ABC$, $\forall x,y,z > 0$.
\end{baitoan}

\begin{baitoan}[\cite{Hai_Hung_Thu_Tung_ncpt_Toan_10_tap_2}, VD8, p. 24]
	Cho $a,b,c > 0$ thỏa $abc + a + c = b$. Tìm {\rm GTLN} của $A = \dfrac{2}{a^2 + 1} - \dfrac{2}{b^2 + 1} + \dfrac{3}{c^2 + 1}$.
\end{baitoan}

\begin{baitoan}[\cite{Hai_Hung_Thu_Tung_ncpt_Toan_10_tap_2}, VD9, p. 24, HSG TpHCM 2006--2007]
	Tìm $x,y,z\in\mathbb{R}$ thỏa $x + y + z = 1$ \& $x^2 + 2y^2 + 3z^2 = 4$ sao cho $x$ đạt {\rm GTLN}.
\end{baitoan}

\begin{baitoan}[\cite{Hai_Hung_Thu_Tung_ncpt_Toan_10_tap_2}, VD10, p. 25, TS ĐH khối B 2008--2009]
	Cho $x,y\in\mathbb{R}$ thỏa $x^2 + y^2 = 1$. Tìm {\rm GTNN, GTLN} của biểu thức $A = \dfrac{2(x^2 + 6xy)}{1 + 2xy + 2y^2}$.
\end{baitoan}

\begin{baitoan}[\cite{Hai_Hung_Thu_Tung_ncpt_Toan_10_tap_2}, VD11, p. 25, HSG TpHCM 2005--2006]
	Cho $n\in\mathbb{N}^\star$, $a_1,\ldots,a_n\in[0,1]$. Chứng minh $(1 + a_1 + a_2 + \cdot + a_n)^2\ge4(a_1^2 + a_2^2 + \cdots + a_n^2)$ bằng cách xét tam thức bậc 2 $f(x) = x^2 - \left(1 + \sum_{i=1}^n a_i\right)x + \sum_{i=1}^n a_i^2$.
\end{baitoan}

\begin{baitoan}[\cite{Hai_Hung_Thu_Tung_ncpt_Toan_10_tap_2}, VD12, p. 25, bất đẳng thức Cauchy--Schwarz]
	Chứng minh $\left(\sum_{i=1}^n a_ib_i\right)^2\le\left(\sum_{i=1}^n a_i^2\right)\left(\sum_{i=1}^n b_i^2\right)$ bằng cách xét tam thức bậc 2 $f(x) = \sum_{i=1}^n (a_ix - b_i)^2$.
\end{baitoan}

\begin{baitoan}[\cite{Hai_Hung_Thu_Tung_ncpt_Toan_10_tap_2}, VD13, p. 26, bất đẳng thức Acz\'el]
	Cho $n\in\mathbb{N}^\star$, $a_1,\ldots,a_n\in\mathbb{R}$ thỏa $a_1^2 - a_2^2 - \cdots - a_n^2 > 0$. Chứng minh $(a_1^2 - a_2^2 - \cdots - a_n^2)(b_1^2 - b_2^2 - \cdots - b_n^2)\le(a_1b_1 - a_2b_2 - \cdots - a_nb_n)^2$.
\end{baitoan}

\begin{baitoan}[\cite{Hai_Hung_Thu_Tung_ncpt_Toan_10_tap_2}, VD14, p. 26, bất đẳng thức Vasile-Cirtoaje]
	Chứng minh $(a^2 + b^2 + c^2)^2\ge3(a^3b + b^3c + c^3a)$, $\forall a,b,c\in\mathbb{R}$.
\end{baitoan}

\begin{baitoan}[\cite{Hai_Hung_Thu_Tung_ncpt_Toan_10_tap_2}, 20.1., p. 27]
	Cho $x,y > 0$ thỏa $x^2y = 1$. Tìm {\rm GTNN} của biểu thức $A = x\sqrt{x^2 + y^2} + x^2$.
\end{baitoan}

\begin{baitoan}[\cite{Hai_Hung_Thu_Tung_ncpt_Toan_10_tap_2}, 20.2., p. 27]
	Cho $x,y,z > 0$ thỏa $x + y + z = 3$. Chứng minh $x + xy + 2xyz\le\frac{9}{2}$.
\end{baitoan}

\begin{baitoan}[\cite{Hai_Hung_Thu_Tung_ncpt_Toan_10_tap_2}, 20.3., p. 27]
	Chứng minh $(a + b + c + d)^2\le3(a^2 + b^2 + c^2 + d^2) + 6ab$, $\forall a,b,c,d\in\mathbb{R}$.
\end{baitoan}

\begin{baitoan}[\cite{Hai_Hung_Thu_Tung_ncpt_Toan_10_tap_2}, 20.4., p. 27]
	Cho $a,b,c,d,p,q\in\mathbb{R}$ thỏa $p^2 + q^2 - a^2 - b^2 - c^2 - d^2 > 0$. Chứng minh $(p^2 - a^2 - b^2)(q^2 - c^2 - d^2)\le(pq - ac - bd)^2$.
\end{baitoan}

\begin{baitoan}[\cite{Hai_Hung_Thu_Tung_ncpt_Toan_10_tap_2}, 20.5., p. 27]
	Cho $a,b,c\ge0$ thỏa $a^2 + b^2 + c^2 = 2$. Chứng minh $ab + bc + ca\le1 + 2abc$.
\end{baitoan}

\begin{baitoan}[\cite{Hai_Hung_Thu_Tung_ncpt_Toan_10_tap_2}, 20.6., p. 27]
	Chứng minh $\left(\dfrac{a}{a + b}\right)^2 + \left(\dfrac{b}{b + c}\right)^2 + \left(\dfrac{c}{c + a}\right)^2 + \dfrac{abc}{abc + a^2b + b^2c + c^2a}\ge1$, $\forall a,b,c > 0$.
\end{baitoan}

\begin{baitoan}[\cite{Hai_Hung_Thu_Tung_ncpt_Toan_10_tap_2}, 20.7., p. 27]
	Chứng minh $\dfrac{a^2 + b^2 + c^2}{2}\ge\min\{(a - b)^2,(b - c)^2,(c - a)^2\}$, $\forall a,b,c\in\mathbb{R}$.
\end{baitoan}

\begin{baitoan}[\cite{Hai_Hung_Thu_Tung_ncpt_Toan_10_tap_2}, 20.8., p. 27]
	Tìm {\rm GTNN} của biểu thức $A = 19x^2 + 54y^2 + 16z^2 + 36xy - 24yz - 16zx$.
\end{baitoan}

%------------------------------------------------------------------------------%

\section{Miscellaneous}

%------------------------------------------------------------------------------%

\printbibliography[heading=bibintoc]
	
\end{document}