\documentclass{article}
\usepackage[backend=biber,natbib=true,style=alphabetic,maxbibnames=50]{biblatex}
\addbibresource{/home/nqbh/reference/bib.bib}
\usepackage[utf8]{vietnam}
\usepackage{tocloft}
\renewcommand{\cftsecleader}{\cftdotfill{\cftdotsep}}
\usepackage[colorlinks=true,linkcolor=blue,urlcolor=red,citecolor=magenta]{hyperref}
\usepackage{amsmath,amssymb,amsthm,float,graphicx,mathtools,tikz}
\usetikzlibrary{angles,calc,intersections,matrix,patterns,quotes,shadings}
\allowdisplaybreaks
\newtheorem{assumption}{Assumption}
\newtheorem{baitoan}{}
\newtheorem{cauhoi}{Câu hỏi}
\newtheorem{conjecture}{Conjecture}
\newtheorem{corollary}{Corollary}
\newtheorem{dangtoan}{Dạng toán}
\newtheorem{definition}{Definition}
\newtheorem{dinhly}{Định lý}
\newtheorem{dinhnghia}{Định nghĩa}
\newtheorem{example}{Example}
\newtheorem{ghichu}{Ghi chú}
\newtheorem{hequa}{Hệ quả}
\newtheorem{hypothesis}{Hypothesis}
\newtheorem{lemma}{Lemma}
\newtheorem{luuy}{Lưu ý}
\newtheorem{nhanxet}{Nhận xét}
\newtheorem{notation}{Notation}
\newtheorem{note}{Note}
\newtheorem{principle}{Principle}
\newtheorem{problem}{Problem}
\newtheorem{proposition}{Proposition}
\newtheorem{question}{Question}
\newtheorem{remark}{Remark}
\newtheorem{theorem}{Theorem}
\newtheorem{vidu}{Ví dụ}
\usepackage[left=1cm,right=1cm,top=5mm,bottom=5mm,footskip=4mm]{geometry}
\def\labelitemii{$\circ$}
\DeclareRobustCommand{\divby}{%
	\mathrel{\vbox{\baselineskip.65ex\lineskiplimit0pt\hbox{.}\hbox{.}\hbox{.}}}%
}
\def\labelitemii{$\circ$}

\title{Problem: Function {\it\&} Graph  -- Bài Tập: Hàm Số {\it\&} Đồ Thị}
\author{Nguyễn Quản Bá Hồng\footnote{A Scientist {\it\&} Creative Artist Wannabe. E-mail: {\tt nguyenquanbahong@gmail.com}. Bến Tre City, Việt Nam.}}
\date{\today}

\begin{document}
\maketitle
\begin{abstract}
	This text is a part of the series {\it Some Topics in Elementary STEM \& Beyond}:
	
	{\sc url}: \url{https://nqbh.github.io/elementary_STEM}.
	
	Latest version:
	\begin{itemize}
		\item {\it Problem: Function \& Graph  -- Bài Tập: Hàm Số \& Đồ Thị}.
		
		PDF: {\sc url}: \url{https://github.com/NQBH/elementary_STEM_beyond/blob/main/elementary_mathematics/grade_10/function_graph/problem/NQBH_function_graph_problem.pdf}.
		
		\TeX: {\sc url}: \url{https://github.com/NQBH/elementary_STEM_beyond/blob/main/elementary_mathematics/grade_10/function_graph/problem/NQBH_function_graph_problem.tex}.
		\item {\it Problem \& Solution: Function \& Graph  -- Bài Tập \& Lời Giải: Hàm Số \& Đồ Thị}.
		
		PDF: {\sc url}: \url{https://github.com/NQBH/elementary_STEM_beyond/blob/main/elementary_mathematics/grade_10/function_graph/solution/NQBH_function_graph_solution.pdf}.
		
		\TeX: {\sc url}: \url{https://github.com/NQBH/elementary_STEM_beyond/blob/main/elementary_mathematics/grade_10/function_graph/solution/NQBH_function_graph_solution.tex}.
	\end{itemize}
\end{abstract}
\tableofcontents

%------------------------------------------------------------------------------%

\textbf{\textsf{Resources -- Tài nguyên.}}
\begin{enumerate}
	\item \cite{Hai_Hung_Thu_Tung_ncpt_Toan_10_tap_2}. {\sc Phan Việt Hải, Trần Quang Hùng, Ninh Văn Thu, Phạm Đình Tùng}. {\it Nâng Cao \& Phát Triển Toán 10. Tập 2}.
\end{enumerate}

\section{General Function -- Đại Cương Về Hàm Số}

\subsection*{Abbreviations -- Viết tắt}

\begin{enumerate}
	\item TXĐ: Tập xác định.
\end{enumerate}

\begin{baitoan}[\cite{Hai_Hung_Thu_Tung_ncpt_Toan_10_tap_2}, VD1, p. 5]
	Công thức tính chu vi \& diện tích hình tròn $P = 2\pi r,S = \pi r^2$ có là hàm số không?
\end{baitoan}

\begin{baitoan}[\cite{Hai_Hung_Thu_Tung_ncpt_Toan_10_tap_2}, VD2, p. 6]
	Tìm {\rm TXĐ} của hàm số $f(x) = \sqrt{x + \sqrt{x}}$.
\end{baitoan}

\begin{baitoan}
	Biện luận theo 4 tham số $a,b,c,d\in\mathbb{R}$ {\rm TXĐ} của hàm số $f(x) = \sqrt{ax + \sqrt{bx + c} + d}$.
\end{baitoan}

\begin{baitoan}[\cite{Hai_Hung_Thu_Tung_ncpt_Toan_10_tap_2}, VD3, p. 6]
	Chứng minh hàm số $f(x) = x^2$ đồng biến trên $[0,+\infty)$ \& nghịch biến trên $(-\infty,0]$.
\end{baitoan}

\begin{baitoan}
	Biện luận theo 3 tham số $a,b,c\in\mathbb{R}$ các khoảng đồng biến, nghịch biến của hàm số $y = f(x) = ax^2 + bx + c$.
\end{baitoan}

\begin{baitoan}[\cite{Hai_Hung_Thu_Tung_ncpt_Toan_10_tap_2}, VD4, p. 7]
	Chứng minh hàm $f(x) = \sqrt{2 - x} + \sqrt{2 + x}$ là hàm chẵn trên {\rm TXĐ} của nó.
\end{baitoan}

\begin{baitoan}[\cite{Hai_Hung_Thu_Tung_ncpt_Toan_10_tap_2}, VD5, p. 7]
	Chứng minh hàm $f(x) = (e^x + e^{-x})\cos x$ là hàm chẵn trên {\rm TXĐ} của nó.
\end{baitoan}

\begin{baitoan}[\cite{Hai_Hung_Thu_Tung_ncpt_Toan_10_tap_2}, VD6, p. 7]
	Chứng minh hàm $f(x) = \cos x$ có chu kỳ cơ sở là $2\pi$.
\end{baitoan}
Tồn tại các hàm tuần hoàn nhưng không có chù kỳ cơ sở.

\begin{baitoan}[\cite{Hai_Hung_Thu_Tung_ncpt_Toan_10_tap_2}, VD7, p. 7]
	Tìm chu kỳ cơ sở của hàm Dirichlet
	\begin{equation}
		f(x) = \chi_\mathbb{Q} = \left\{\begin{split}
			&1&&\mbox{if } x\in\mathbb{Q},\\
			&0&&\mbox{if } x\in\mathbb{R}\backslash\mathbb{Q}.
		\end{split}\right.
	\end{equation}
\end{baitoan}

\begin{baitoan}[\cite{Hai_Hung_Thu_Tung_ncpt_Toan_10_tap_2}, VD8, p. 7]
	Cho $a,b,c,d\in\mathbb{R}^\star$. Chứng minh hàm số $f(x) = a\sin cx + b\cos dx$ tuần hoàn trên $\mathbb{R}$ khi \& chỉ khi $\frac{c}{d}\in\mathbb{Q}$.
\end{baitoan}

\begin{baitoan}[\cite{Hai_Hung_Thu_Tung_ncpt_Toan_10_tap_2}, VD9, p. 7]
	Chứng minh hàm số $f(x) = \cos x + \cos x\sqrt{2}$ không tuần hoàn trên $\mathbb{R}$.
\end{baitoan}

\begin{baitoan}[\cite{Hai_Hung_Thu_Tung_ncpt_Toan_10_tap_2}, VD10, p. 8]
	Cho 2 hàm số $f(x) = x^2 + 5,g(x) = x^3 + 2x^2 + 1$. Tính $f(g(x))$.
\end{baitoan}

\begin{baitoan}[\cite{Hai_Hung_Thu_Tung_ncpt_Toan_10_tap_2}, 17.1., p. 8]
	Tìm {\rm TXĐ} của hàm số: $f(x) = \dfrac{|x + 1|}{(x - 3)\sqrt{2x - 1}},g(x) = \dfrac{\sqrt{5 - 3|x|}}{x^2 + 4x + 3},h(x) = \dfrac{x + 4}{\sqrt{x^2 - 16}}$.
\end{baitoan}

\begin{baitoan}[\cite{Hai_Hung_Thu_Tung_ncpt_Toan_10_tap_2}, 17.2., p. 8]
	2 hàm số $f(x) = \dfrac{|x|}{x},g(x) = 1$ có bằng nhau không?
\end{baitoan}

\begin{baitoan}[\cite{Hai_Hung_Thu_Tung_ncpt_Toan_10_tap_2}, 17.3., p. 8]
	Cho hàm số $f(x) = \dfrac{x}{\sqrt{x^2 + 1}}$. Tính $f_n(x)$ với $f_1(x)\coloneqq f(x),f_n(x)\coloneqq f(f_{n-1}(x))$.
\end{baitoan}

\begin{baitoan}[\cite{Hai_Hung_Thu_Tung_ncpt_Toan_10_tap_2}, 17.4., p. 8]
	Cho $f(x)$ là 1 hàm bất kỳ với {\rm TXĐ} $\mathbb{R}$. Chứng minh $f(x)$ luôn biểu diễn được 1 cách duy nhất dưới dạng tổng của 1 hàm số chẵn \& 1 hàm số lẻ.
\end{baitoan}

\begin{baitoan}[\cite{Hai_Hung_Thu_Tung_ncpt_Toan_10_tap_2}, 17.5., p. 8]
	Cho $f(x)$ là 1 hàm tuần hoàn bất kỳ với {\rm TXĐ} $\mathbb{R}$ \& chu kỳ cơ sở là $T$. Tìm chu kỳ cơ sở của hàm số $y(x) = f(ax + b)$, $a,b\in\mathbb{R}$, $a > 0$.
\end{baitoan}

\begin{baitoan}[\cite{Hai_Hung_Thu_Tung_ncpt_Toan_10_tap_2}, 17.6., p. 8]
	Cho $f(x)$ là 1 hàm bất kỳ với {\rm TXĐ} $D$. Giả sử tồn tại $a\in\mathbb{R}^\star$ thỏa $f(x + a) = \dfrac{f(x) - 1}{f(x) + 1}$. Chứng minh $f(x)$ là hàm tuần hoàn.
\end{baitoan}

\begin{baitoan}[\cite{Hai_Hung_Thu_Tung_ncpt_Toan_10_tap_2}, 17.7., p. 8]
	Cho $a\in\mathbb{R}^\star$, $f:\mathbb{R}_+\to\mathbb{R}$ thỏa $f(x + a) = \frac{1}{2} + \sqrt{f(x) - f(x)^2}$, $\forall x > 0$. Chứng minh $f(x)$ là hàm tuần hoàn.
\end{baitoan}

\begin{baitoan}[\cite{Hai_Hung_Thu_Tung_ncpt_Toan_10_tap_2}, 17.8., p. 8]
	Cho hàm số $f(x)$ xác định trên $\mathbb{R}$, thỏa $f(x + 3)\le f(x) + 3$, $f(x + 2)\ge f(x) + 2$, $\forall x\in\mathbb{R}$. Chứng minh $g(x)\coloneqq f(x) - x$ là hàm tuần hoàn.
\end{baitoan}

%------------------------------------------------------------------------------%

\section{2nd-Order Function -- Hàm Số Bậc 2}

%------------------------------------------------------------------------------%

\section{Solvable Equations via Quadratic Equations -- Phương Trình Quy Về Phương Trình Bậc 2}

%------------------------------------------------------------------------------%

\section{Ứng Dụng của Hàm Số Trong Chứng Minh Bất Đẳng Thức \& Tìm GTLN, GTNN}

%------------------------------------------------------------------------------%

\section{Miscellaneous}

%------------------------------------------------------------------------------%

\printbibliography[heading=bibintoc]
	
\end{document}