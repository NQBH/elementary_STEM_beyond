\documentclass{article}
\usepackage[backend=biber,natbib=true,style=alphabetic,maxbibnames=50]{biblatex}
\addbibresource{/home/nqbh/reference/bib.bib}
\usepackage[utf8]{vietnam}
\usepackage{tocloft}
\renewcommand{\cftsecleader}{\cftdotfill{\cftdotsep}}
\usepackage[colorlinks=true,linkcolor=blue,urlcolor=red,citecolor=magenta]{hyperref}
\usepackage{amsmath,amssymb,amsthm,float,graphicx,mathtools,tikz}
\usetikzlibrary{angles,calc,intersections,matrix,patterns,quotes,shadings}
\allowdisplaybreaks
\newtheorem{assumption}{Assumption}
\newtheorem{baitoan}{}
\newtheorem{cauhoi}{Câu hỏi}
\newtheorem{conjecture}{Conjecture}
\newtheorem{corollary}{Corollary}
\newtheorem{dangtoan}{Dạng toán}
\newtheorem{definition}{Definition}
\newtheorem{dinhly}{Định lý}
\newtheorem{dinhnghia}{Định nghĩa}
\newtheorem{example}{Example}
\newtheorem{ghichu}{Ghi chú}
\newtheorem{hequa}{Hệ quả}
\newtheorem{hypothesis}{Hypothesis}
\newtheorem{lemma}{Lemma}
\newtheorem{luuy}{Lưu ý}
\newtheorem{nhanxet}{Nhận xét}
\newtheorem{notation}{Notation}
\newtheorem{note}{Note}
\newtheorem{principle}{Principle}
\newtheorem{problem}{Problem}
\newtheorem{proposition}{Proposition}
\newtheorem{question}{Question}
\newtheorem{remark}{Remark}
\newtheorem{theorem}{Theorem}
\newtheorem{vidu}{Ví dụ}
\usepackage[left=1cm,right=1cm,top=5mm,bottom=5mm,footskip=4mm]{geometry}
\def\labelitemii{$\circ$}
\DeclareRobustCommand{\divby}{%
	\mathrel{\vbox{\baselineskip.65ex\lineskiplimit0pt\hbox{.}\hbox{.}\hbox{.}}}%
}

\title{Problem: Proposition {\it\&} Set -- Bài Tập: Mệnh Đề {\it\&} Tập Hợp}
\author{Nguyễn Quản Bá Hồng\footnote{A Scientist {\it\&} Creative Artist Wannabe. E-mail: {\tt nguyenquanbahong@gmail.com}. Bến Tre City, Việt Nam.}}
\date{\today}

\begin{document}
\maketitle
\begin{abstract}
	This text is a part of the series {\it Some Topics in Elementary STEM \& Beyond}:
	
	{\sc url}: \url{https://nqbh.github.io/elementary_STEM}.
	
	Latest version:
	\begin{itemize}
		\item {\it Problem: Proposition \& Set -- Bài Tập: Mệnh Đề \& Tập Hợp}.
		
		PDF: {\sc url}: \url{https://github.com/NQBH/elementary_STEM_beyond/blob/main/elementary_mathematics/grade_10/proposition_set/problem/NQBH_proposition_set_problem.pdf}.
		
		\TeX: {\sc url}: \url{https://github.com/NQBH/elementary_STEM_beyond/blob/main/elementary_mathematics/grade_10/proposition_set/problem/NQBH_proposition_set_problem.tex}.
		\item {\it Problem \& Solution: Proposition \& Set -- Bài Tập \& Lời Giải: Mệnh Đề \& Tập Hợp}.
		
		PDF: {\sc url}: \url{https://github.com/NQBH/elementary_STEM_beyond/blob/main/elementary_mathematics/grade_10/proposition_set/solution/NQBH_proposition_set_solution.pdf}.
		
		\TeX: {\sc url}: \url{https://github.com/NQBH/elementary_STEM_beyond/blob/main/elementary_mathematics/grade_10/proposition_set/solution/NQBH_proposition_set_solution.tex}.
	\end{itemize}
\end{abstract}
\tableofcontents

%------------------------------------------------------------------------------%

\section{Mathematical Proposition -- Mệnh Đề Toán Học}

\begin{baitoan}[\cite{Hai_Hung_Thu_Tung2022_tap_1}, p. ]
	
\end{baitoan}

\begin{baitoan}
	{\rm Đ{\tt/}S?} (a) $\forall x\in\mathbb{R}$, $ax^2 + bx + c\ge0$.
\end{baitoan}


%------------------------------------------------------------------------------%

\section{Set -- Tập Hợp}

\begin{baitoan}
	Viết tập nghiệm của phương trình $ax + b = 0$ với $a,b\in\mathbb{R}$.
\end{baitoan}

\begin{baitoan}
	Viết tập nghiệm của bất phương trình bậc nhất $ax + b > 0$, $ax + b\ge0$ với $a,b\in\mathbb{R}$.
\end{baitoan}

\begin{baitoan}
	Viết tập nghiệm của phương trình bậc 2 $ax^2 + bx + c = 0$ với $a,b,c\in\mathbb{R}$, $a\ne0$.
\end{baitoan}

\begin{baitoan}
	Viết tập nghiệm của bất phương trình bậc 2: $ax^2 + bx + c > 0$, $ax^2 + bx + c\ge0$ với $a,b,c\in\mathbb{R}$, $a\ne0$.
\end{baitoan}

%------------------------------------------------------------------------------%

\section{Miscellaneous}

%------------------------------------------------------------------------------%

\printbibliography[heading=bibintoc]
	
\end{document}