\documentclass{article}
\usepackage[backend=biber,natbib=true,style=alphabetic,maxbibnames=50]{biblatex}
\addbibresource{/home/nqbh/reference/bib.bib}
\usepackage[utf8]{vietnam}
\usepackage{tocloft}
\renewcommand{\cftsecleader}{\cftdotfill{\cftdotsep}}
\usepackage[colorlinks=true,linkcolor=blue,urlcolor=red,citecolor=magenta]{hyperref}
\usepackage{amsmath,amssymb,amsthm,enumitem,float,graphicx,mathtools,tikz}
\usetikzlibrary{angles,calc,intersections,matrix,patterns,quotes,shadings}
\allowdisplaybreaks
\newtheorem{assumption}{Assumption}
\newtheorem{baitoan}{}
\newtheorem{cauhoi}{Câu hỏi}
\newtheorem{conjecture}{Conjecture}
\newtheorem{corollary}{Corollary}
\newtheorem{dangtoan}{Dạng toán}
\newtheorem{definition}{Definition}
\newtheorem{dinhly}{Định lý}
\newtheorem{dinhnghia}{Định nghĩa}
\newtheorem{example}{Example}
\newtheorem{ghichu}{Ghi chú}
\newtheorem{hequa}{Hệ quả}
\newtheorem{hypothesis}{Hypothesis}
\newtheorem{lemma}{Lemma}
\newtheorem{luuy}{Lưu ý}
\newtheorem{nhanxet}{Nhận xét}
\newtheorem{notation}{Notation}
\newtheorem{note}{Note}
\newtheorem{principle}{Principle}
\newtheorem{problem}{Problem}
\newtheorem{proposition}{Proposition}
\newtheorem{question}{Question}
\newtheorem{remark}{Remark}
\newtheorem{theorem}{Theorem}
\newtheorem{vidu}{Ví dụ}
\usepackage[left=1cm,right=1cm,top=5mm,bottom=5mm,footskip=4mm]{geometry}
\def\labelitemii{$\circ$}
\DeclareRobustCommand{\divby}{%
	\mathrel{\vbox{\baselineskip.65ex\lineskiplimit0pt\hbox{.}\hbox{.}\hbox{.}}}%
}

\title{Problem: Vector -- Bài Tập: Vector}
\author{Nguyễn Quản Bá Hồng\footnote{A Scientist {\it\&} Creative Artist Wannabe. E-mail: {\tt nguyenquanbahong@gmail.com}. Bến Tre City, Việt Nam.}}
\date{\today}
\setlist[itemize]{leftmargin=*}
\setlist[enumerate]{leftmargin=*}

\begin{document}
\maketitle
\begin{abstract}
	This text is a part of the series {\it Some Topics in Elementary STEM \& Beyond}:
	
	{\sc url}: \url{https://nqbh.github.io/elementary_STEM}.
	
	Latest version:
	\begin{itemize}
		\item {\it Problem: Vector -- Bài Tập: Vector}.
		
		PDF: {\sc url}: \url{https://github.com/NQBH/elementary_STEM_beyond/blob/main/elementary_mathematics/grade_10/vector/problem/NQBH_vector_problem.pdf}.
		
		\TeX: {\sc url}: \url{https://github.com/NQBH/elementary_STEM_beyond/blob/main/elementary_mathematics/grade_10/vector/problem/NQBH_vector_problem.tex}.
		\item {\it Problem \& Solution: Vector -- Bài Tập \& Lời Giải: Vector}.
		
		PDF: {\sc url}: \url{https://github.com/NQBH/elementary_STEM_beyond/blob/main/elementary_mathematics/grade_10/vector/solution/NQBH_vector_solution.pdf}.
		
		\TeX: {\sc url}: \url{https://github.com/NQBH/elementary_STEM_beyond/blob/main/elementary_mathematics/grade_10/vector/solution/NQBH_vector_solution.tex}.
	\end{itemize}
\end{abstract}
\tableofcontents

%------------------------------------------------------------------------------%

\section{Vector \& Các Phép Toán Trên Vector}
\textbf{\textsf{Resources -- Tài nguyên.}}
\begin{enumerate}
	\item \cite[Chuyên đề 1: {\it Vector Phẳng}]{Pho_Dung_chuyen_de_Toan_PT_1}. {\sc Lê Hoành Phò, Trần Nam Dũng}. {\it Tuyển Chọn Các Chuyên Đề Toán Phổ Thông Tập 1}.
	\item \cite{TLCT_hinh_hoc_10}. {\it Tài Liệu Chuyên Toán Hình Học 10}.
\end{enumerate}
\fbox{1} {\it Quy tắc 3 điểm}: $\overrightarrow{AC} = \overrightarrow{AB} + \overrightarrow{CC}$, $\forall A,B,C\in\mathbb{R}^d$, $d\in\mathbb{N}^\star$. \fbox{2} {\sf Trung điểm.} $I$ là trung điểm của đoạn thẳng $AB$, $M$ là 1 điểm bất kỳ $\Rightarrow\overrightarrow{IA} + \overrightarrow{IB} = \vec{0}$, $\overrightarrow{MA} + \overrightarrow{MB} = 2\overrightarrow{MI}$. \fbox{3} {\sf Trọng tâm.} $G$ là trọng tâm $\Delta ABC$, $M$ là 1 điểm bất kỳ $\Rightarrow\overrightarrow{GA} + \overrightarrow{GB} + \overrightarrow{GC} = \vec{0}$, $\overrightarrow{MA} + \overrightarrow{MB} + \overrightarrow{MC} = 3\overrightarrow{MG}$. \fbox{4} {\sf Hình bình hành.} $ABCD$ là hình bình hành $\Rightarrow\overrightarrow{AB} + \overrightarrow{AD} = \overrightarrow{AC}$, $\overrightarrow{AB} - \overrightarrow{AD} = \overrightarrow{DB}$. \fbox{5} {\sf Chia tỷ lệ.} Điểm $M$ được gọi là chia đoạn $AB$ theo tỷ số $k\in\mathbb{R}\backslash\{1\}$ nếu $\overrightarrow{MA} = k\overrightarrow{MB}\Leftrightarrow\overrightarrow{OM} = \dfrac{\overrightarrow{OA} - k\overrightarrow{OB}}{1 - k}$. \fbox{6} {\sf Phân tích 2 vector không cùng phương.} Nếu 2 vector $\vec{a},\vec{b}$ không cùng phương thì với mọi vector $\vec{c}$, tồn tại duy nhất $m,n\in\mathbb{R}$ thỏa $\vec{c} = m\vec{a} + n\vec{b}$. \fbox{7} {\sf Tâm tỷ cự.} Với $n\in\mathbb{N}^\star$ điểm $A_i$ \& $a_i\in\mathbb{R}$, $i = 1,\ldots,n$ có tổng khác 0, i.e., $\sum_{i=1}^n a_i\ne0$, tồn tại duy nhất điểm $I$ thỏa $\sum_{i=1}^n a_i\overrightarrow{IA_i} = \vec{0}$. Điểm $I$ được gọi là {\it tâm tỷ cự} của hệ điểm $(A_i)_{i=1}^n$ với bộ trọng số $(a_i)_{i=1}^n$ tương ứng. \fbox{8} {\sf Điều kiện vuông góc.} $AB\bot CD\Leftrightarrow\overrightarrow{AB}\cdot\overrightarrow{CD} = 0$. \fbox{9} {\sf Điều kiện thẳng hàng.} $A,B,C$ thẳng hàng $\Leftrightarrow\overrightarrow{AB} = k\overrightarrow{AC}$. \fbox{10} $\Delta ABC$ có $\overrightarrow{OA} + \overrightarrow{OB} + \overrightarrow{OC} = \overrightarrow{OH} = 3\overrightarrow{OG}$, $a\overrightarrow{IA} + b\overrightarrow{IB} + c\overrightarrow{IC} = \vec{0}$.

\begin{baitoan}[\cite{Pho_Dung_chuyen_de_Toan_PT_1}, VD1.1, p. 9]
	Tứ giác $ABCD$, $M,N$ chia $AD,BC$ theo tỷ số $k\in(-\infty,0)$. (a) Tìm tập hợp trung điểm $I$ của $MN$ khi $k$ thay đổi. (b) Biểu diễn $\overrightarrow{MN}$ theo $\overrightarrow{AB},\overrightarrow{DC},k$ để suy ra $MN\le\max\{AB,CD\}$.
\end{baitoan}

\begin{baitoan}[\cite{Pho_Dung_chuyen_de_Toan_PT_1}, VD1.2, p. 10]
	Cho $n$-giác đều $A_1A_2\ldots A_n$ nội tiếp trong đường tròn $(O,R)$ \& điểm $M\in(O,R)$. Chứng minh: (a) $\sum_{i=1}^n \overrightarrow{OA_i} = \vec{0}$. (b) $\sum_{i=1}^n \cos\widehat{MOA_i} = 0$, $\sum_{i=1}^n MA_i^2 = 2nR^2$.
\end{baitoan}

\begin{baitoan}[\cite{Pho_Dung_chuyen_de_Toan_PT_1}, VD1.3, p. 11]
	$\Delta ABC$. Chứng minh: (a) $c^2 CM^2 = a^2AM^2 + b^2BM^2 + (a^2 + b^2 - c^2)AM\cdot BM$ với mọi điểm $M$ thuộc cạnh $AB$. (b) Độ dài đường phân giác $l_a^2 = \dfrac{4bc}{(a + b)^2}p(p - a)$.
\end{baitoan}

\begin{baitoan}[\cite{Pho_Dung_chuyen_de_Toan_PT_1}, VD1.4, p. 13]
	Tứ giác $ABCD$ có 2 đường chéo $AC,BD$ cắt nhau tại $O$. $H,K$ là trực tâm $\Delta ABO,\Delta CDO$, $I,J$ lần lượt là trung điểm $AD,BC$. Chứng minh $HK\bot IJ$.
\end{baitoan}

\begin{baitoan}[\cite{Pho_Dung_chuyen_de_Toan_PT_1}, VD1.5, p. 14]
	Cho 3 dây cung song song $AA',BB',CC'$ của đường tròn $(O)$. Chứng minh 3 trực tâm của $\Delta ABC',\Delta BCA',CAB'$ thẳng hàng.
\end{baitoan}

\begin{baitoan}[\cite{Hai_Hung_Thu_Tung2022_tap_1}, VD1, p. 59]
	Cho đoạn thẳng $AB$ \& $I$ là trung điểm của $AB$. Chứng minh: (a) $\overrightarrow{IA} + \overrightarrow{IB} = \vec{0}$. (b)  $\overrightarrow{MA} + \overrightarrow{MB} = 2\overrightarrow{MI}$ với mọi điểm $M$.
\end{baitoan}

\begin{baitoan}[\cite{Hai_Hung_Thu_Tung2022_tap_1}, VD2, p. 59]
	Cho $\Delta ABC$ \& điểm $M$ nằm giữa B,C. Chứng minh:
	\begin{align*}
		\overrightarrow{AM} = \frac{MB}{BC}\overrightarrow{AC} + \frac{MC}{BC}\overrightarrow{AB}.
	\end{align*}
\end{baitoan}

\begin{baitoan}[\cite{Hai_Hung_Thu_Tung2022_tap_1}, VD3, p. 60]
	Cho $\Delta ABC$. Chứng minh: (a) 3 đường trung tuyến đồng quy tại 1 điểm $G$. (b) $\overrightarrow{GA} + \overrightarrow{GB} + \overrightarrow{GC} = \vec{0}$. (c) $\overrightarrow{MA} + \overrightarrow{MB} + \overrightarrow{MC} = 3\overrightarrow{MG}$ với mọi điểm $M$.
\end{baitoan}

\begin{baitoan}[\cite{Hai_Hung_Thu_Tung2022_tap_1}, VD4, p. 60]
	Cho $\Delta ABC$ \& 1 điểm $M$ bất kỳ trong tam giác. Đặt $S_{MBC} = S_a$, $S_{MCA} = S_b$, $S_{MAB} = S_c$. Chứng minh: $S_a\overrightarrow{MA} + S_b\overrightarrow{MB} + S_c\overrightarrow{MC} = \vec{0}$.
\end{baitoan}

\begin{baitoan}[\cite{Hai_Hung_Thu_Tung2022_tap_1}, VD5, p. 61]
	Cho $\Delta ABC$. Đường tròn nội tiếp $(I)$ tiếp xúc với cạnh $BC$ tại $D$. Gọi $M$ là trung điểm của $BC$. Chứng minh: $a\overrightarrow{MD} + b\overrightarrow{MC} + c\overrightarrow{MB} = \vec{0}$ (với $a,b,c$ là độ dài các cạnh $BC,AC,AB$).
\end{baitoan}

\begin{baitoan}[\cite{Hai_Hung_Thu_Tung2022_tap_1}, VD6, p. 61]
	Cho $\Delta ABC$ \& điểm $P$ bất kỳ. Gọi $A_1,B_1,C_1$ lần lượt là trung điểm của $BC,CA,AB$. Trên các tia $PA_1,PB_1,PC_1$ lần lượt lấy các điểm $X,Y,Z$ sao cho $\dfrac{PX}{PA_1} = \dfrac{PY}{PB_1} = \dfrac{PZ}{PC_1} = k$. Chứng minh: (a) $AX,BY,CZ$ đồng quy tại $T$. (b) $P,T,G$ thẳng hàng \& $\dfrac{TG}{PG} = \left|\dfrac{3k}{2 + k}\right|$.
\end{baitoan}

\begin{baitoan}[\cite{Hai_Hung_Thu_Tung2022_tap_1}, VD7, p. 62]
	Đường đối trung trong tam giác là đường đối xứng với trung tuyến qua phân giác. Chứng minh: 3 đường đối trung đồng quy tại điểm $L$ thỏa mãn $a^2\overrightarrow{LA} + b^2\overrightarrow{LB} + c^2\overrightarrow{LC} = \vec{0}$. Điểm $L$ như vậy gọi là \emph{điểm Lemoine} của $\Delta ABC$.
\end{baitoan}

\begin{baitoan}[\cite{Hai_Hung_Thu_Tung2022_tap_1}, VD8, p. 62]
	Cho $\Delta ABC$ \& điểm $P$ bất kỳ. $PA,PB,PC$ cắt các cạnh $BC,CA,AB$ tương ứng tại các điểm $A_1,B_1,C_1$. Gọi $A_2,B_2,C_2$ lần lượt là trung điểm của $BC,CA,AB$. Gọi $A_3,B_3,C_3$ lần lượt là trung điểm của $AA_1,BB_1,CC_1$. (a) Chứng minh: $A_2A_3,B_2B_3,C_2C_3$ đồng quy. (b) Lấy điểm $A_4$ thuộc $BC$ sao cho $QA_4$ song song với $PA$. Xác định các điểm $B_4$ \& $C_4$ tương tự $A_4$. Chứng minh: $Q$ là trọng tâm của $\Delta A_4B_4C_4$.
\end{baitoan}

\begin{baitoan}[\cite{Hai_Hung_Thu_Tung2022_tap_1}, VD9, p. 64]
	Cho $\Delta ABC$. Đường tròn nội tiếp $(I)$ tiếp xúc với $BC,CA,AB$ lần lượt tại $D,E,F$. Chứng minh: $a\overrightarrow{ID} + b\overrightarrow{IE} + c\overrightarrow{IF} = \vec{0}$.
\end{baitoan}

\begin{baitoan}[\cite{Hai_Hung_Thu_Tung2022_tap_1}, VD10, p. 64]
	Cho $\Delta ABC$ có $\widehat{A} = 90^\circ$ \& các đường phân giác $BE$ \& $CF$. Đặt $\vec{u} = (AB + BC + CA)\overrightarrow{BC} + BC\overrightarrow{EF}$. Chứng minh: giá của $\vec{u}$ vuông góc với $BC$.
\end{baitoan}

\begin{baitoan}[\cite{Hai_Hung_Thu_Tung2022_tap_1}, 8.1., p. 65]
	Cho vector $\vec{u}$ có 2 phương khác nhau, chứng minh $\vec{u} = \vec{0}$.
\end{baitoan}

\begin{baitoan}[\cite{Hai_Hung_Thu_Tung2022_tap_1}, 8.2., p. 65]
	Cho $\Delta ABC$ có $M$ \& $N$ lần lượt là trung điểm của $AB$ \& $AC$. Lấy $P$ đối xứng với $M$ qua $N$. Chứng minh: $\overrightarrow{MP} = \overrightarrow{BC}$.
\end{baitoan}

\begin{baitoan}[\cite{Hai_Hung_Thu_Tung2022_tap_1}, 8.3., p. 65]
	Cho $\Delta ABC$ có tâm đường tròn ngoại tiếp $O$, trực tâm $H$. Lấy $K$ đối xứng với $O$ qua $BC$. Chứng minh: $\overrightarrow{OK} = \overrightarrow{AH}$.
\end{baitoan}

\begin{baitoan}[\cite{Hai_Hung_Thu_Tung2022_tap_1}, 8.4., p. 65]
	Cho 2 vector $\vec{a}$ \& $\vec{b}$ thỏa mãn $|\vec{a} + \vec{b}| = |\vec{a} - \vec{b}|$. Chứng minh: 2 vector $\vec{a}$ \& $\vec{b}$ có giá vuông góc.
\end{baitoan}

\begin{baitoan}[\cite{Hai_Hung_Thu_Tung2022_tap_1}, 8.5., p. 65]
	Cho $\Delta ABC$ \& $\Delta DEF$ thỏa mãn $\overrightarrow{AD} + \overrightarrow{BE} + \overrightarrow{CF} = \vec{0}$. Chứng minh: $\Delta ABC$ \& $\Delta DEF$ có cùng trọng tâm.
\end{baitoan}

\begin{baitoan}[\cite{Hai_Hung_Thu_Tung2022_tap_1}, 8.6., p. 65]
	Cho 2 vector $\vec{a}$ \& $\vec{b}$ thỏa mãn $\vec{a}$ có giá vuông góc với giá của vector $\vec{a} + \vec{b}$. Chứng minh: $|\vec{a} + \vec{b}|^2 = |\vec{b}|^2 - |\vec{a}|^2$.
\end{baitoan}

\begin{baitoan}[\cite{Hai_Hung_Thu_Tung2022_tap_1}, 8.7., p. 65]
	Cho $\Delta ABC$ \& điểm $P$ thỏa mãn $|\overrightarrow{PB} + \overrightarrow{PA} - \overrightarrow{PC}| = |\overrightarrow{PB} + \overrightarrow{PC} - \overrightarrow{PA}|$, $|\overrightarrow{PC} + \overrightarrow{PB} - \overrightarrow{PA}| = |\overrightarrow{PC} + \overrightarrow{PA} - \overrightarrow{PB}|$. Chứng minh: $|\overrightarrow{PA} + \overrightarrow{PC} - \overrightarrow{PB}| = |\overrightarrow{PA} + \overrightarrow{PB} - \overrightarrow{PC}|$.
\end{baitoan}

\begin{baitoan}[\cite{Hai_Hung_Thu_Tung2022_tap_1}, 8.8., p. 65]
	Cho $\Delta ABC$ nội tiếp đường tròn $(O)$. Cho $(O),B,C$ cố định \& $A$ di chuyển trên đường tròn $(O)$. $BE,CF$ là 2 đường cao của $\Delta ABC$. Giả sử có vector $\vec{u}$ thỏa mãn $\dfrac{|\overrightarrow{EF} - \vec{u}|^2}{EF^2} + \dfrac{|\overrightarrow{OA} - \vec{u}|^2}{OA^2} = 1$. Chứng minh $\dfrac{1}{EF^2} - \dfrac{1}{|\vec{u}|^2}$ luôn không đổi khi A thay đổi.
\end{baitoan}

\begin{baitoan}[\cite{Hai_Hung_Thu_Tung2022_tap_1}, 8.9., p. 65]
	Cho $\Delta ABC$ có các phân giác trong $AD,BE,CF$. Gọi $X,Y,Z$ lần lượt là trung điểm của $EF,FD,DE$. (a) Chứng minh: $AX,BY,CZ$ đồng quy tại điểm $P$ thỏa mãn hệ thức: $a(b + c)\overrightarrow{PA} + b(c + a)\overrightarrow{PB} + c(a + b)\overrightarrow{PC} = \vec{0}$. (b) Gọi $N$ là tâm đường tròn Euler của $\Delta ABC$. Dựng vector $\vec{u}$ thỏa mãn $\vec{u} = \dfrac{\overrightarrow{NA}}{a} + \dfrac{\overrightarrow{NB}}{b} + \dfrac{\overrightarrow{NC}}{c}$. Gọi $Q$ là trung điểm $ON$, trong đó $O$ là tâm đường tròn ngoại tiếp $\Delta ABC$. Chứng minh: $PQ$ song song hoặc trùng với giá của vector $\vec{u}$.
\end{baitoan}

%------------------------------------------------------------------------------%

\section{Scalar Product -- Tích Vô Hướng}

\begin{baitoan}[\cite{Hai_Hung_Thu_Tung2022_tap_1}, VD1, p. 75]
	(a) Cho đoạn AB \& điểm M. Chứng minh $\overrightarrow{MA}\cdot\overrightarrow{MB} = \frac{1}{2}(MA^2 + MB^2 - AB^2)$. (b) Cho đoạn thẳng AB,CD. Chứng minh $\overrightarrow{AB}\cdot\overrightarrow{CD} = \frac{1}{2}(AD^2 - AC^2 + BD^2 - BC^2)$. (c) Chứng minh $AB\bot CD\Leftrightarrow AD^2 - AC^2 = BD^2 - BC^2$.
\end{baitoan}

\begin{baitoan}[\cite{Hai_Hung_Thu_Tung2022_tap_1}, VD2, p. 76]
	Cho $\Delta ABC$. Lấy I thỏa $\alpha\overrightarrow{IA} + \beta\overrightarrow{IB} + \gamma\overrightarrow{IC} = \vec{0}$ với $\alpha + \beta + \gamma = 0$. Chứng minh: (a) $\alpha IA^2 + \beta IB^2 + \gamma IC^2 = \dfrac{\beta\gamma BC^2 + \gamma\alpha CA^2 + \alpha\beta AB^2}{\alpha + \beta + \gamma} = \dfrac{\beta\gamma a^2 + \gamma\alpha b^2 + \alpha\beta c^2}{\alpha + \beta + \gamma}$. (b) $\alpha PA^2 + \beta PB^2 + \gamma PC^2 = (\alpha + \beta + \gamma)PI^2 + \alpha IA^2 + \beta IB^2 + \gamma IC^2$ với mọi điểm P. (c) $PI^2 = \dfrac{\alpha PA^2 + \beta PB^2 + \gamma PC^2}{\alpha + \beta + \gamma} - \dfrac{\beta\gamma a^2 + \gamma\alpha b^2 + \alpha\beta c^2}{(\alpha + \beta + \gamma)^2}$ với mọi điểm P.
\end{baitoan}

\begin{baitoan}[\cite{Hai_Hung_Thu_Tung2022_tap_1}, VD3, p. 77]
	Cho $\vec{a},\vec{b}$ không cùng phương. Tìm $\vec{u}$ thỏa $\vec{a}\cdot\vec{u} = \alpha,\vec{b}\cdot\vec{u} = \beta$.
\end{baitoan}

\begin{baitoan}[\cite{Hai_Hung_Thu_Tung2022_tap_1}, VD4, p. 77]
	Cho $\Delta ABC$ đều có trọng tâm O \& điểm M bất kỳ. Chứng minh: (a) $\cos\widehat{AOM} + \cos\widehat{BOM} + \cos\widehat{COM} = 0$. (b) $\cos^2\widehat{AOM} + \cos^2\widehat{BOM} + \cos^2\widehat{COM} = {\rm const}$. (c) $\cos^4\widehat{AOM} + \cos^4\widehat{BOM} + \cos^4\widehat{COM} = {\rm const}$.
\end{baitoan}

\begin{baitoan}[\cite{Hai_Hung_Thu_Tung2022_tap_1}, BĐ, p. 77]
	Cho $\Delta ABC$ đều. (a) Điểm N nằm trên đường tròn $(O)$ ngoại tiếp $\Delta ABC$. Chứng minh $AN^4 + BN^4 + CN^4$ không đổi. (b) Chứng minh $AN^4 + BN^4 + CN^4 = 18R^4 + 3(ON^2 - R^2)(ON^2 + 5R^2)$ với mọi điểm N. (c) Từ đó suy ra $AN^4 + BN^4 + CN^4 < 18R^4\Leftarrow N$ nằm trong $(O)$, $AN^4 + BN^4 + CN^4 = 18R^4\Leftarrow N\in(O)$, $AN^4 + BN^4 + CN^4 > 18R^4\Leftarrow N$ nằm ngoài $(O)$.
\end{baitoan}

\begin{baitoan}[\cite{Hai_Hung_Thu_Tung2022_tap_1}, VD5, p. 79]
	Cho $\Delta ABC$ đều nội tiếp đường tròn $(O)$. Đường thẳng d đi qua O \& cắt BC,CA,AB lần lượt tại D,E,F. Chứng minh $\dfrac{1}{OD^4} + \dfrac{1}{OE^4} + \dfrac{1}{OF^4} = {\rm const}$.
\end{baitoan}

\begin{baitoan}[\cite{Hai_Hung_Thu_Tung2022_tap_1}, VD6, p. 79]
	Cho 3 vector $\vec{a},\vec{b},\vec{c}$ thỏa $\vec{a} + \vec{b} + \vec{c} = \vec{0},|\vec{a}| = |\vec{b}| = |\vec{c}|$ (mô hình vector của tam giác đều) \& $\vec{u}$ là vector bất kỳ. Chứng minh: (a) $\cos(\vec{u},\vec{a}) + \cos(\vec{u},\vec{b}) + \cos(\vec{u},\vec{c}) = 0$. (b) $\cos^2(\vec{u},\vec{a}) + \cos^2(\vec{u},\vec{b}) + \cos^2(\vec{u},\vec{c}) = \frac{3}{2}$. (c) $\cos^4(\vec{u},\vec{a}) + \cos^4(\vec{u},\vec{b}) + \cos^4(\vec{u},\vec{c}) = \frac{9}{8}$. (d) Tính $\cos^{2^n}(\vec{u},\vec{a}) + \cos^{2^n}(\vec{u},\vec{b}) + \cos^{2^n}(\vec{u},\vec{c})$ với $n\in\mathbb{N}$.
\end{baitoan}

\begin{baitoan}[\cite{Hai_Hung_Thu_Tung2022_tap_1}, VD7, p. 79]
	Cho $\Delta ABC$ đều \& M,N bất kỳ. $M_a,M_b,M_c$ lần lượt là hình chiếu của M lên BC,CA,AB. $N_a,N_b,N_c$ lần lượt là hình chiếu của N lên BC,CA,AB. Chứng minh $M_aN_a^2 + M_bN_b^2 + M_cN_c^2 = \frac{3}{2}MN^2$. 
\end{baitoan}

\begin{baitoan}[\cite{Hai_Hung_Thu_Tung2022_tap_1}, 10.1., p. 79]
	Cho $\Delta ABC$, trọng tâm G. E,F nằm trên đường thẳng GC,GB sao cho $EF\parallel BC$, AG cắt $(ABF),(ACE)$ tại N,M. Chứng minh $FM = EN$.
\end{baitoan}

\begin{baitoan}[\cite{Hai_Hung_Thu_Tung2022_tap_1}, 10.3., p. 80]
	Cho $\Delta ABC$ có $DEF$ là tam giác Ceva của điểm P bất kỳ. L,K là tâm đường tròn ngoại tiếp $\Delta PCA,\Delta PAB$. Lấy $S\in KL$ thỏa $DS\bot EF$. Đường trung trực của BC cắt KL tại T. Chứng minh $S,T$ đối xứng qua trung điểm KL.
\end{baitoan}

\begin{baitoan}[\cite{Hai_Hung_Thu_Tung2022_tap_1}, 10.4., p. 80]
	Cho $\Delta ABC$, đường tròn nội tiếp $(I)$ tiếp xúc CA,AB tại E,F. Điểm P di chuyển trên EF,PB cắt CA tại M, MI cắt đường thẳng qua C vuông góc AC tại N. Chứng minh đường thẳng qua N vuông góc PC luôn đi qua 1 điểm cố định khi P di chuyển.
\end{baitoan}

\begin{baitoan}[\cite{Hai_Hung_Thu_Tung2022_tap_1}, 10.5., p. 80]
	Cho $\Delta ABC$ \& điểm $I(\alpha,\beta,\gamma)$ ở trong tam giác với mọi điểm P trong mặt phẳng. Chứng minh $\alpha PA\cdot IA + \beta PB\cdot IB + \gamma PC\cdot IC\ge\alpha IA^2 + \beta IB^2 + \gamma IC^2$.
\end{baitoan}

\begin{baitoan}[\cite{Hai_Hung_Thu_Tung2022_tap_1}, 10.6., p. 80]
	Cho $\Delta ABC$ \& điểm P bất kỳ nằm trong tam giác. $A',B',C'$ lần lượt là hình chiếu của P xuống đoạn BC,CA,AB \& $(I,r0$ là đường tròn nội tiếp $\Delta ABC$. Tìm {\rm GTNN} của biểu thức $PA' + PB' + PC' + \dfrac{PI^2}{2r}$.
\end{baitoan}

\begin{baitoan}[\cite{Hai_Hung_Thu_Tung2022_tap_1}, 10.7., p. 80]
	Cho $\Delta ABC$ với 3 trung tuyến $m_a,m_b,m_c$. $A',B',C'$ di chuyển trên 3 đường thẳng BC,CA,AB. Tìm cực trị của $\dfrac{B'C'^3}{m_a} + \dfrac{C'A'^3}{m_b} + \dfrac{A'B'^3}{m_c}$.
\end{baitoan}

\begin{baitoan}[\cite{Hai_Hung_Thu_Tung2022_tap_1}, 10.8., p. 80]
	Cho $\Delta ABC$ nội tiếp đường tròn $(O)$, I là tâm đường tròn nội tiếp, M là điểm bất kỳ trên cung nhỏ BC. Chứng minh $MA + 2OI\ge MB + MC\ge MA - 2OI$.
\end{baitoan}

\begin{baitoan}[\cite{Hai_Hung_Thu_Tung2022_tap_1}, 10.9., p. 80]
	Cho $\Delta ABC$, trực tâm H, bán kính đường tròn ngoại tiếp R. Với mọi M trên mặt phẳng, tìm {\rm GTNN} của biểu thức $MA^3 + MB^3 + MC^3 - \frac{3}{2}R\cdot MH^2$.
\end{baitoan}

%------------------------------------------------------------------------------%

\printbibliography[heading=bibintoc]
	
\end{document}