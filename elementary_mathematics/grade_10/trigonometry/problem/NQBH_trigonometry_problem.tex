\documentclass{article}
\usepackage[backend=biber,natbib=true,style=alphabetic,maxbibnames=50]{biblatex}
\addbibresource{/home/nqbh/reference/bib.bib}
\usepackage[utf8]{vietnam}
\usepackage{tocloft}
\renewcommand{\cftsecleader}{\cftdotfill{\cftdotsep}}
\usepackage[colorlinks=true,linkcolor=blue,urlcolor=red,citecolor=magenta]{hyperref}
\usepackage{amsmath,amssymb,amsthm,float,graphicx,mathtools,tikz}
\usetikzlibrary{angles,calc,intersections,matrix,patterns,quotes,shadings}
\allowdisplaybreaks
\newtheorem{assumption}{Assumption}
\newtheorem{baitoan}{}
\newtheorem{cauhoi}{Câu hỏi}
\newtheorem{conjecture}{Conjecture}
\newtheorem{corollary}{Corollary}
\newtheorem{dangtoan}{Dạng toán}
\newtheorem{definition}{Definition}
\newtheorem{dinhly}{Định lý}
\newtheorem{dinhnghia}{Định nghĩa}
\newtheorem{example}{Example}
\newtheorem{ghichu}{Ghi chú}
\newtheorem{hequa}{Hệ quả}
\newtheorem{hypothesis}{Hypothesis}
\newtheorem{lemma}{Lemma}
\newtheorem{luuy}{Lưu ý}
\newtheorem{nhanxet}{Nhận xét}
\newtheorem{notation}{Notation}
\newtheorem{note}{Note}
\newtheorem{principle}{Principle}
\newtheorem{problem}{Problem}
\newtheorem{proposition}{Proposition}
\newtheorem{question}{Question}
\newtheorem{remark}{Remark}
\newtheorem{theorem}{Theorem}
\newtheorem{vidu}{Ví dụ}
\usepackage[left=1cm,right=1cm,top=5mm,bottom=5mm,footskip=4mm]{geometry}
\def\labelitemii{$\circ$}
\DeclareRobustCommand{\divby}{%
	\mathrel{\vbox{\baselineskip.65ex\lineskiplimit0pt\hbox{.}\hbox{.}\hbox{.}}}%
}

\title{Problem: Trigonometrical Identities in Triangles -- Bài Tập: Hệ Thức Lượng Trong Tam Giác}
\author{Nguyễn Quản Bá Hồng\footnote{e-mail: {\sf nguyenquanbahong@gmail.com}, website: \url{https://nqbh.github.io}, Bến Tre, Việt Nam.}}
\date{\today}

\begin{document}
\maketitle
\tableofcontents

%------------------------------------------------------------------------------%

\section{Giá Trị Lượng Giác Của 1 Góc \& Hệ Thức Lượng Trong Tam Giác}
\fbox{1} $\forall\alpha\in[0^\circ;180^\circ]$, $\sin\alpha\in[-1;1]$, $\cos\alpha\in[-1;1]$. \fbox{2} $\cos\alpha > 0\Leftrightarrow\alpha\in(0^\circ;90^\circ)\Leftrightarrow\alpha$ nhọn. $\cos\alpha < 0\Leftrightarrow\alpha\in(90^\circ;180^\circ)\Leftrightarrow\alpha$ tù. \fbox{3} Định lý cosin: $a^2 = b^2 + c^2 - 2bc\cos A,b^2 = c^2 + a^2 - 2ca\cos B,c^2 = a^2 + b^2 - 2ab\cos C$ hay $\cos A = \dfrac{b^2 + c^2 - a^2}{2bc},\cos B = \dfrac{c^2 + a^2 - b^2}{2ca},\cos C = \dfrac{a^2 + b^2 - c^2}{2ab}$. \fbox{4} Định lý sin: $\dfrac{a}{\sin A} = \dfrac{b}{\sin B} = \dfrac{c}{\sin C} = 2R$ hay $a = 2R\sin A,b = 2R\sin B,c = 2R\sin C$. \fbox{5} Công thức tính diện tích tam giác: $S = \frac{1}{2}bc\sin A = \frac{1}{2}ca\sin B = \frac{1}{2}ab\sin C = \sqrt{p(p - a)(p - b)(p - c)}$ với $p = \dfrac{a + b + c}{2}$.

\begin{baitoan}
	Cho $\alpha\in[0^\circ;360^\circ)$. Tìm các khoảng giá trị của $\alpha$ để các hàm $\sin\alpha,\cos\alpha,\tan\alpha,\cot\alpha$ lần lượt bằng $0$, âm, dương.
\end{baitoan}

\begin{baitoan}
	Dùng định lý sin, giải thích vì sao trong 1 tam giác, cạnh đối diện với góc lớn hơn là cạnh lớn hơn \& góc đối diện với cạnh lớn hơn là góc lớn hơn.
\end{baitoan}

\begin{baitoan}[\cite{Hai_Hung_Thu_Tung2022_tap_1}, BĐ1, p. 22]
	Cho $\Delta ABC$, đường phân giác AD. Chứng minh $AD^2 < bc$.
\end{baitoan}

\begin{baitoan}[\cite{Hai_Hung_Thu_Tung2022_tap_1}, VD1, p. 22]
	Cho $\Delta ABC$ vuông tại A, 2 phân giác trong BE,CF cắt đường cao AH lần lượt tại P,Q. M là trung điểm BC. Chứng minh $PE + QF < AM$.
\end{baitoan}

\begin{baitoan}[\cite{Hai_Hung_Thu_Tung2022_tap_1}, VD2, p. 22]
	Cho $\Delta ABC$ vuông tại A, đường cao AH, $D\in AB$ thỏa $BH = BD = CD$. Chứng minh $\dfrac{AD}{BD} = \sqrt[3]{2} - 1$.
\end{baitoan}

\begin{baitoan}[\cite{Hai_Hung_Thu_Tung2022_tap_1}, VD3, p. 23]
	Cho $\Delta ABC$. Chứng minh $\widehat{A} = 90^\circ\Leftrightarrow(\sqrt{a + b} + \sqrt{a - b})(\sqrt{a + c} + \sqrt{a - c}) = \sqrt{2}(a + b + c)$.
\end{baitoan}

\begin{baitoan}[\cite{Hai_Hung_Thu_Tung2022_tap_1}, BĐ1, p. 23]
	$\Delta ABC$ có $\widehat{A} = 2\widehat{B}$. Chứng minh $a^2 = b^2 + bc$.
\end{baitoan}

\begin{baitoan}[\cite{Hai_Hung_Thu_Tung2022_tap_1}, VD4, p. 23]
	Cho $\Delta ABC$ vuông tại A. Lấy $D\in AC$ thỏa $\widehat{C} = 2\widehat{CBD}$. Chứng minh $AB + AD = BC\Leftrightarrow\widehat{C} = 30^\circ$ hoặc $\widehat{C} = 45^\circ$.
\end{baitoan}

\begin{baitoan}[\cite{Hai_Hung_Thu_Tung2022_tap_1}, VD5, p. 23]
	Cho $\Delta ABC$, trung tuyến AM. Giả sử $\widehat{B} + \widehat{AMC} = 90^\circ$. Chứng minh $\Delta ABC$ vuông hoặc cân.
\end{baitoan}

\begin{baitoan}[\cite{Hai_Hung_Thu_Tung2022_tap_1}, VD6, p. 24]
	Cho $\Delta ABC$, tâm đường tròn nội tiếp I. IA,IB,IC cắt $(ABC)$ lần lượt tại D,E,F. Chứng minh $\dfrac{1}{S_{DBC}} + \dfrac{1}{S_{EAC}} + \dfrac{1}{S_{FAB}}\ge\dfrac{9}{S_{ABC}}$.
\end{baitoan}

%------------------------------------------------------------------------------%

\section{Giải Tam Giác}
\fbox{1} Cho $a,b,c$: áp dụng định lý cosin: $\widehat{A} = \arccos\dfrac{b^2 + c^2 - a^2}{2bc},\widehat{B} = \arccos\dfrac{c^2 + a^2 - b^2}{2ca},\widehat{C} = \arccos\dfrac{a^2 + b^2 - c^2}{2ab}$. \fbox{2} Cho $b,c,\widehat{A}$: áp dụng định lý cosin: $a = \sqrt{b^2 + c^2 - 2bc\cos A}$, $\widehat{B} = \arccos\dfrac{c^2 + a^2 - b^2}{2ca},\widehat{C} = \arccos\dfrac{a^2 + b^2 - c^2}{2ab}$. \fbox{3} Cho $a,\widehat{B},\widehat{C}$: $\widehat{A} = 180^\circ - \widehat{B} - \widehat{C}$, áp dụng định lý sin: $b = \dfrac{a\sin B}{\sin A},c = \dfrac{b\sin C}{\sin B}$.

\begin{baitoan}
	Chứng minh công thức Heron.
\end{baitoan}

\begin{baitoan}
	Cho $\Delta ABC$. Tính $\sin A,\sin B,\sin C,\tan A,\tan B,\tan C,\cot A,\cot B,\cot C$ theo $a,b,c$.
\end{baitoan}

\begin{baitoan}
	Nếu chỉ cho số đo 3 góc của 1 tam giác, có thể giải tam giác đó không? Nếu có thì mô tả tập nghiệm các tam giác thỏa mãn.
\end{baitoan}

\begin{baitoan}
	Nếu cho trước độ dài 2 cạnh \& số đo 1 góc không nằm giữa 2 cạnh đó của 1 tam giác thì có giải tam giác đó được không?
\end{baitoan}

\begin{baitoan}
	Nếu cho trước độ dài 1 cạnh \& số đo 2 góc không cùng kề với cạnh đó của 1 tam giác thì có giải tam giác đó được không?
\end{baitoan}

\begin{baitoan}[{\sf Program: Solve triangle}]
	(a) Nêu các bộ 3 yếu tố cần cho trước về cạnh \& góc của 1 tam giác để tam giác đó có thể giải được. (b) Viết chương trình {\sf Pascal, Python, C{\tt/}C++} để minh họa.
\end{baitoan}

\begin{baitoan}
	Cho độ dài 3 cạnh của 1 tam giác. Tính độ dài 3 đường trung tuyến \& 6 góc tạo bởi 3 đường trung tuyến đó.
\end{baitoan}

\begin{baitoan}
	Cho độ dài 3 cạnh của 1 tam giác. (a) Tính độ dài 3 đường phân giác \& 6 đoạn tạo thành trên 3 cạnh. (b) Tính khoảng cách từ tâm đường tròn nội tiếp I đến 3 đỉnh \& 3 cạnh.
\end{baitoan}

\begin{baitoan}
	Cho độ dài 3 cạnh của 1 tam giác. (a) Tính độ dài 3 đường cao \& 6 góc tạo bởi 3 đường cao đó \& 6 đoạn tạo thành trên 3 cạnh. (b) Tính khoảng cách từ trực tâm đến 3 đỉnh \& 3 cạnh.
\end{baitoan}

\begin{baitoan}
	Cho độ dài 3 cạnh của 1 tam giác. Tính khoảng cách từ tâm đường tròn ngoại tiếp O đến 3 cạnh của tam giác đó.
\end{baitoan}

%------------------------------------------------------------------------------%

\section{Miscellaneous}

\begin{baitoan}
	Cần cho trước bao nhiêu yếu tố về cạnh, góc, đường chéo để giải 1 đa giác lồi đều n cạnh?
\end{baitoan}

\begin{baitoan}
	Cho độ dài 4 cạnh của 1 tứ giác lồi, liệu có thể giải được tứ giác đó không?
\end{baitoan}

\begin{baitoan}
	Cần cho trước bao nhiêu yếu tố về cạnh, góc, đường chéo của 1 tứ giác lồi để có thể giải được tứ giác đó?
\end{baitoan}

\begin{baitoan}
	Đặt 3 điện tích $q_1,q_2,q_3$ tại 3 đỉnh của $\Delta ABC$. Tính các lực điện từ.
\end{baitoan}

%------------------------------------------------------------------------------%

\printbibliography[heading=bibintoc]
	
\end{document}