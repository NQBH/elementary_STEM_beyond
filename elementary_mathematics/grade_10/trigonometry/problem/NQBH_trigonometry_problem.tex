\documentclass{article}
\usepackage[backend=biber,natbib=true,style=alphabetic,maxbibnames=50]{biblatex}
\addbibresource{/home/nqbh/reference/bib.bib}
\usepackage[utf8]{vietnam}
\usepackage{tocloft}
\renewcommand{\cftsecleader}{\cftdotfill{\cftdotsep}}
\usepackage[colorlinks=true,linkcolor=blue,urlcolor=red,citecolor=magenta]{hyperref}
\usepackage{amsmath,amssymb,amsthm,float,graphicx,mathtools,tikz}
\usetikzlibrary{angles,calc,intersections,matrix,patterns,quotes,shadings}
\allowdisplaybreaks
\newtheorem{assumption}{Assumption}
\newtheorem{baitoan}{}
\newtheorem{cauhoi}{Câu hỏi}
\newtheorem{conjecture}{Conjecture}
\newtheorem{corollary}{Corollary}
\newtheorem{dangtoan}{Dạng toán}
\newtheorem{definition}{Definition}
\newtheorem{dinhly}{Định lý}
\newtheorem{dinhnghia}{Định nghĩa}
\newtheorem{example}{Example}
\newtheorem{ghichu}{Ghi chú}
\newtheorem{hequa}{Hệ quả}
\newtheorem{hypothesis}{Hypothesis}
\newtheorem{lemma}{Lemma}
\newtheorem{luuy}{Lưu ý}
\newtheorem{nhanxet}{Nhận xét}
\newtheorem{notation}{Notation}
\newtheorem{note}{Note}
\newtheorem{principle}{Principle}
\newtheorem{problem}{Problem}
\newtheorem{proposition}{Proposition}
\newtheorem{question}{Question}
\newtheorem{remark}{Remark}
\newtheorem{theorem}{Theorem}
\newtheorem{vidu}{Ví dụ}
\usepackage[left=1cm,right=1cm,top=5mm,bottom=5mm,footskip=4mm]{geometry}
\def\labelitemii{$\circ$}
\DeclareRobustCommand{\divby}{%
	\mathrel{\vbox{\baselineskip.65ex\lineskiplimit0pt\hbox{.}\hbox{.}\hbox{.}}}%
}

\title{Problem: Trigonometrical Identities in Triangles -- Bài Tập: Hệ Thức Lượng Trong Tam Giác}
\author{Nguyễn Quản Bá Hồng\footnote{e-mail: {\sf nguyenquanbahong@gmail.com}, website: \url{https://nqbh.github.io}, Bến Tre, Việt Nam.}}
\date{\today}

\begin{document}
\maketitle
\tableofcontents

%------------------------------------------------------------------------------%

\section{Giá Trị Lượng Giác Của 1 Góc \& Hệ Thức Lượng Trong Tam Giác}
\fbox{1} $\forall\alpha\in[0^\circ;180^\circ]$, $\sin\alpha\in[-1;1]$, $\cos\alpha\in[-1;1]$. \fbox{2} $\cos\alpha > 0\Leftrightarrow\alpha\in(0^\circ;90^\circ)\Leftrightarrow\alpha$ nhọn. $\cos\alpha < 0\Leftrightarrow\alpha\in(90^\circ;180^\circ)\Leftrightarrow\alpha$ tù. \fbox{3} Định lý cosin: $a^2 = b^2 + c^2 - 2bc\cos A,b^2 = c^2 + a^2 - 2ca\cos B,c^2 = a^2 + b^2 - 2ab\cos C$ hay $\cos A = \dfrac{b^2 + c^2 - a^2}{2bc},\cos B = \dfrac{c^2 + a^2 - b^2}{2ca},\cos C = \dfrac{a^2 + b^2 - c^2}{2ab}$. \fbox{4} Định lý sin: $\dfrac{a}{\sin A} = \dfrac{b}{\sin B} = \dfrac{c}{\sin C} = 2R$ hay $a = 2R\sin A,b = 2R\sin B,c = 2R\sin C$. \fbox{5} Công thức tính diện tích tam giác: $S = \frac{1}{2}ah_a = \frac{1}{2}bh_b = \frac{1}{2}ch_c = \frac{1}{2}bc\sin A = \frac{1}{2}ca\sin B = \frac{1}{2}ab\sin C = \sqrt{p(p - a)(p - b)(p - c)}$ với $p = \dfrac{a + b + c}{2}$.

\begin{baitoan}
	$\alpha\in[0^\circ;360^\circ)$. Tìm các khoảng giá trị của $\alpha$ để các hàm $\sin\alpha,\cos\alpha,\tan\alpha,\cot\alpha$ lần lượt bằng $0$, âm, dương.
\end{baitoan}

\begin{baitoan}
	Dùng định lý sin, giải thích vì sao trong 1 tam giác, cạnh đối diện với góc lớn hơn là cạnh lớn hơn \& góc đối diện với cạnh lớn hơn là góc lớn hơn.
\end{baitoan}

\begin{baitoan}[\cite{Hai_Hung_Thu_Tung2022_tap_1}, BĐ1, p. 22]
	$\Delta ABC$, đường phân giác AD. Chứng minh $AD^2 < bc$.
\end{baitoan}

\begin{baitoan}[\cite{Hai_Hung_Thu_Tung2022_tap_1}, VD1, p. 22]
	$\Delta ABC$ vuông tại A, 2 phân giác trong BE,CF cắt đường cao AH lần lượt tại P,Q. M là trung điểm BC. Chứng minh $PE + QF < AM$.
\end{baitoan}

\begin{baitoan}[\cite{Hai_Hung_Thu_Tung2022_tap_1}, VD2, p. 22]
	$\Delta ABC$ vuông tại A, đường cao AH, $D\in AB$ thỏa $BH = BD = CD$. Chứng minh $\dfrac{AD}{BD} = \sqrt[3]{2} - 1$.
\end{baitoan}

\begin{baitoan}[\cite{Hai_Hung_Thu_Tung2022_tap_1}, VD3, p. 23]
	$\Delta ABC$. Chứng minh $\widehat{A} = 90^\circ\Leftrightarrow(\sqrt{a + b} + \sqrt{a - b})(\sqrt{a + c} + \sqrt{a - c}) = \sqrt{2}(a + b + c)$.
\end{baitoan}

\begin{baitoan}[\cite{Hai_Hung_Thu_Tung2022_tap_1}, BĐ1, p. 23]
	$\Delta ABC$ có $\widehat{A} = 2\widehat{B}$. Chứng minh $a^2 = b^2 + bc$.
\end{baitoan}

\begin{baitoan}[\cite{Hai_Hung_Thu_Tung2022_tap_1}, VD4, p. 23]
	$\Delta ABC$ vuông tại A. Lấy $D\in AC$ thỏa $\widehat{C} = 2\widehat{CBD}$. Chứng minh $AB + AD = BC\Leftrightarrow\widehat{C} = 30^\circ$ hoặc $\widehat{C} = 45^\circ$.
\end{baitoan}

\begin{baitoan}[\cite{Hai_Hung_Thu_Tung2022_tap_1}, VD5, p. 23]
	$\Delta ABC$, trung tuyến AM. Giả sử $\widehat{B} + \widehat{AMC} = 90^\circ$. Chứng minh $\Delta ABC$ vuông hoặc cân.
\end{baitoan}

\begin{baitoan}[\cite{Hai_Hung_Thu_Tung2022_tap_1}, VD6, p. 24]
	$\Delta ABC$, tâm đường tròn nội tiếp I. IA,IB,IC cắt $(ABC)$ lần lượt tại D,E,F. Chứng minh $\dfrac{1}{S_{DBC}} + \dfrac{1}{S_{EAC}} + \dfrac{1}{S_{FAB}}\ge\dfrac{9}{S_{ABC}}$.
\end{baitoan}

%------------------------------------------------------------------------------%

\section{Giải Tam Giác}
\fbox{1} Cho $a,b,c$: áp dụng định lý cosin: $\widehat{A} = \arccos\dfrac{b^2 + c^2 - a^2}{2bc},\widehat{B} = \arccos\dfrac{c^2 + a^2 - b^2}{2ca},\widehat{C} = \arccos\dfrac{a^2 + b^2 - c^2}{2ab}$. \fbox{2} Cho $b,c,\widehat{A}$: áp dụng định lý cosin: $a = \sqrt{b^2 + c^2 - 2bc\cos A}$, $\widehat{B} = \arccos\dfrac{c^2 + a^2 - b^2}{2ca},\widehat{C} = \arccos\dfrac{a^2 + b^2 - c^2}{2ab}$. \fbox{3} Cho $a,\widehat{B},\widehat{C}$: $\widehat{A} = 180^\circ - \widehat{B} - \widehat{C}$, áp dụng định lý sin: $b = \dfrac{a\sin B}{\sin A},c = \dfrac{b\sin C}{\sin B}$.

\begin{baitoan}[\cite{TLCT_hinh_hoc_10}, VD1, p. 124]
	$\Delta ABC$ có đường cao $AH = h$, $\widehat{B} = \beta$, K trên cạnh BC thỏa $BK = 2CK$, $AK = AB$. Giải $\Delta ABC$.
\end{baitoan}

\begin{baitoan}[\cite{TLCT_hinh_hoc_10}, VD2, p. 124]
	Cho $x,y\in[1;+\infty)$. Đặt $a = x^2 + 1,b = y^2 + 1,c = x^2 + y^2 + 1$. Chứng minh tồn tại 1 tam giác có độ dài 3 cạnh là $a,b,c$ \& tam giác đó là tam giác tù.
\end{baitoan}

\begin{baitoan}[\cite{TLCT_hinh_hoc_10}, VD3, p. 126]
	$\Delta ABC$ có $BC = a,\widehat{A} = \alpha,\widehat{B} = \beta$, tâm đường tròn nội tiếp. Tính bán kính $(IBC),(ICA),(IAB)$.
\end{baitoan}

\begin{baitoan}[\cite{TLCT_hinh_hoc_10}, p. 124, hệ thức về bán kính các đường tròn nội tiếp \& bàng tiếp]
	$\Delta ABC$ có bán kính đường tròn nội tiếp r, bán kính 3 đường tròn bàng tiếp góc A,B,C lần lượt là $r_a,r_b,r_c$. Chứng minh: (a) $(p - a)\tan\dfrac{A}{2} = (p - b)\tan\dfrac{B}{2} = (p - c)\tan\dfrac{C}{2} = r$. (b) $r_a\cot\dfrac{A}{2} = r_b\cot\dfrac{B}{2} = r_c\cot\dfrac{C}{2} = p$.
\end{baitoan}

\begin{baitoan}[\cite{TLCT_hinh_hoc_10}, p. 128]
	Chứng minh $S = \dfrac{abc}{4R} = pr = (p - a)r_a = (p - b)r_b = (p - c)r_c = \sqrt{p(p - a)(p - b)(p - c)}$.
\end{baitoan}

\begin{baitoan}[\cite{TLCT_hinh_hoc_10}, VD4, p. 129]
	$\Delta ABC$, $\widehat{A} = 60^\circ,R = 8,r = 3$. Tính S.
\end{baitoan}

\begin{baitoan}[\cite{TLCT_hinh_hoc_10}, VD5, p. 129]
	Tính r theo $a,b,c$.
\end{baitoan}

\begin{baitoan}[\cite{TLCT_hinh_hoc_10}, VD6, p. 130]
	Chứng minh $\dfrac{1}{r} = \dfrac{1}{r_a} + \dfrac{1}{r_b} + \dfrac{1}{r_c} = \dfrac{1}{h_a} + \dfrac{1}{h_b} + \dfrac{1}{h_c}$.
\end{baitoan}

\begin{baitoan}[\cite{TLCT_hinh_hoc_10}, VD7, p. 130, công thức độ dài phân giác]
	Gọi $l_a,l_b,l_c$ lần lượt là độ dài 3 đường phân giác trong góc A,B,C. Chứng minh $l_a = \dfrac{2bc\cos\dfrac{A}{2}}{b + c},l_b = \dfrac{2ca\cos\dfrac{B}{2}}{c + a},l_c = \dfrac{2ab\cos\dfrac{C}{2}}{a + b}$.
\end{baitoan}

\begin{baitoan}[\cite{TLCT_hinh_hoc_10}, VD8, p. 131, tứ giác điều hòa]
	$\Delta ABC$ nội tiếp đường tròn $(O)$ có AM là trung tuyến đỉnh A. Đường thẳng qua A \& đối xứng với AM qua phân giác trong góc A cắt $(O)$ tại N. Chứng minh $AB\cdot CN = AC\cdot BN$.
\end{baitoan}

\begin{baitoan}[\cite{TLCT_hinh_hoc_10}, 37., p. 131]
	$\Delta ABC$ có 2 trung tuyến BM,CN cắt nhau tại G, $BM = \frac{3}{2},CN = 3,\widehat{BGC} = 120^\circ$. Giải $\Delta ABC$.
\end{baitoan}

\begin{baitoan}[\cite{TLCT_hinh_hoc_10}, 38., p. 132]
	$\Delta ABC$ có $AC = b,AB = c,\widehat{A} = \alpha$, M là trung điểm BC, N trên cạnh AB thỏa $\dfrac{NA}{NB} = \dfrac{3}{2}$. Tính MN.
\end{baitoan}

\begin{baitoan}[\cite{TLCT_hinh_hoc_10}, 39., p. 132]
	$\Delta ABC$ có $BC = 10$, $(I)$ là đường tròn có tâm I thuộc cạnh BC \& tiếp xúc với 2 cạnh AB,AC. (a) Biết $IA = 3,2IB = 3IC$, tính AB,AC. (b) Biết $(I)$ có bán kính bằng $3$ \& $2IB = 3IC$, tính $R,AB,AC$.
\end{baitoan}

\begin{baitoan}[\cite{TLCT_hinh_hoc_10}, 40., p. 132]
	Hình thang ABCD ngoại tiếp được có 2 đáy $BC = b,AD = d > b$, góc giữa 2 cạnh bên bằng $\alpha$. Tính bán kính đường tròn nội tiếp.
\end{baitoan}

\begin{baitoan}[\cite{TLCT_hinh_hoc_10}, 41., p. 132]
	Hình thang cân ABCD với đáy lớn AB ngoại tiếp 1 đường tròn bán kính r. (a) Đặt $\widehat{BAD} = \alpha$. Tính độ dài 2 đáy \& đường chéo theo $r,\alpha$. (b) R là bán kính đường tròn ngoại tiếp hình thang. Biết $\dfrac{R}{r} = \dfrac{2}{3}\sqrt{7}$, tính $\widehat{BAD}$.
\end{baitoan}

\begin{baitoan}[\cite{TLCT_hinh_hoc_10}, 42., p. 132]
	Chứng minh $\cos\dfrac{A}{2} = \sqrt{\dfrac{p(p - a)}{bc}},\tan\dfrac{A}{2} = \sqrt{\dfrac{(p - b)(p - c)}{p(p - a)}}$.
\end{baitoan}

\begin{dinhnghia}
	$a,b,c\in\mathbb{R}$ được gọi là lập thành {\rm cấp số cộng} nếu $a + c = 2b$. Lúc đó giá trị $d = b - a = c - b$ được gọi là {\rm công sai} của cấp số cộng.
\end{dinhnghia}

\begin{baitoan}[\cite{TLCT_hinh_hoc_10}, 43., p. 132]
	Chứng minh 3 cạnh $a,b,c$ của $\Delta ABC$ lập thành cấp số cộng khi \& chỉ khi $\tan\dfrac{A}{2}\tan\dfrac{C}{2} = \dfrac{1}{3}$. Chứng minh khi đó công sai của cấp số cộng này là $d = \dfrac{3}{2}r\left(\tan\dfrac{C}{2} - \tan\dfrac{A}{2}\right)$.
\end{baitoan}

\begin{baitoan}[\cite{TLCT_hinh_hoc_10}, 44., p. 132]
	Chứng minh: (a) $\sin A,\sin B,\sin C$ lập thành cấp số cộng khi \& chỉ khi $\cot\dfrac{A}{2},\cot\dfrac{B}{2},\cot\dfrac{C}{2}$ lập thành cấp số cộng. (b) $\cos A,\cos B,\cos C$ lập thành cấp số cộng khi \& chỉ khi $\tan\dfrac{A}{2},\tan\dfrac{B}{2},\tan\dfrac{C}{2}$ lập thành cấp số cộng.
\end{baitoan}

\begin{baitoan}[\cite{TLCT_hinh_hoc_10}, 45., p. 133]
	$\Delta ABC$ \& điểm M thay đổi trên cạnh BC, $r_1,r_2$ là bán kính đường tròn nội tiếp \& $\rho_1,\rho_2$ là bán kính đường tròn bàng tiếp góc A của $\Delta ABM,\Delta ACM$. Chứng minh $\dfrac{r_1r_2}{\rho_1\rho_2}$ không đổi.
\end{baitoan}

\begin{baitoan}[\cite{TLCT_hinh_hoc_10}, 46., p. 133]
	$\Delta ABC$, M,N trên cạnh BC thỏa $\widehat{BAM} = \widehat{CAN}$, P,Q là tiếp điểm của đường tròn nội tiếp $\Delta BAM,\Delta CAN$ với cạnh BC. Chứng minh $\dfrac{1}{PB} + \dfrac{1}{PM} = \dfrac{1}{QC} + \dfrac{1}{QN}$.
\end{baitoan}

\begin{baitoan}[\cite{TLCT_hinh_hoc_10}, 47., p. 133]
	Đường tròn $(O;R)$ \& A bên ngoài $(O)$. 1 đường thẳng thay đổi qua A cắt $(O)$ tại B,C. Đặt $\widehat{AOB} = \alpha,\widehat{AOC} = \beta$. Chứng minh $\tan\dfrac{\alpha}{2}\tan\dfrac{\beta}{2}$ không đổi.
\end{baitoan}

\begin{baitoan}[\cite{TLCT_hinh_hoc_10}, 48., p. 133]
	$\Delta ABC$ có diện tích S. (a) Chứng minh $\cot A = \dfrac{b^2 + c^2 - a^2}{4S}$. (b) M là trung điểm BC, đặt $\widehat{AMB} = \varphi$. Chứng minh $\cot C - \cot B = 2\cot\varphi$. (c) G là trọng tâm $\Delta ABC$. Đặt $\widehat{BGC} = \alpha$. Chứng minh $\cot\alpha = \dfrac{5bc\cos A - 2(b^2 + c^2)}{3bc\sin A}$.
\end{baitoan}

\begin{baitoan}[\cite{TLCT_hinh_hoc_10}, 49., p. 133]
	Chứng minh: (a) $b^2 + c^2 = 2a^2\Leftrightarrow\cot B + \cot C = 2\cot A$. (b) $b^4 + c^4 = a^4\Leftrightarrow\tan B\tan C = 2\sin^2A$.
\end{baitoan}

\begin{baitoan}[\cite{TLCT_hinh_hoc_10}, 50., p. 133]
	$\Delta ABC$ có 2 trung tuyến BM,CN. (a) Chứng minh $BM\bot CN\Leftrightarrow\cot B + \cot C = \frac{1}{2}\cot A$. (b) Chứng minh nếu $\Delta ABC$ không cân tại A thì $\dfrac{BM}{CN} = \dfrac{AB}{AC}\Leftrightarrow\cot B + \cot C = 2\cot A$.
\end{baitoan}

\begin{baitoan}[\cite{TLCT_hinh_hoc_10}, 51., p. 133]
	Hình chữ nhật ABCD \& 1 điểm M tùy ý. Chứng minh $\dfrac{\tan\widehat{AMC}}{\tan\widehat{BMD}} = \dfrac{S_{AMC}}{S_{BMD}}$.
\end{baitoan}

\begin{baitoan}[\cite{TLCT_hinh_hoc_10}, 52., p. 134]
	$\Delta ABC$, $\widehat{B} > \widehat{C}$m $O,I,O_1$ lần lượt là tâm đường tròn ngoại tiếp, đường tròn nội tiếp, \& đường tròn bàng tiếp góc A. Chứng minh $\tan\widehat{IOO_1} = \dfrac{2(\sin B - \sin C)}{2\cos A - 1}$.
\end{baitoan}

\begin{baitoan}[\cite{TLCT_hinh_hoc_10}, 53., p. 134]
	Hình bình hành có $a,b$ là độ dài các cạnh, $m,n$ là độ dài 2 đường chéo. $a > b,m > n$. $\alpha$ là góc nhọn của hình bình hành \& $\varphi$ là góc giữa 2 đường chéo. Chứng minh: (a) $\cos\alpha\cos\varphi = \dfrac{(a^2 - b^2)(m^2 - n^2)}{4abmn}$. (b) $\tan\varphi = \dfrac{2ab\sin\alpha}{a^2 - b^2}$.
\end{baitoan}

\begin{baitoan}[\cite{TLCT_hinh_hoc_10}, 54., p. 134]
	$\Delta ABC$, trực tâm H, 2 đường cao $BB' = \sqrt{5},CC' = 2$. Tính S biết: (a) $\cos\widehat{BHC} = -\dfrac{2}{3}$. (b) $\cos\widehat{CBB'} = \dfrac{2}{\sqrt{5}}$.
\end{baitoan}

\begin{baitoan}[\cite{TLCT_hinh_hoc_10}, 55., p. 134]
	Hình bình hành ABCD, M,N là 2 giao điểm của $(ABC)$ với AD,CD. (a) Biết khoảng cách từ M đến B,C,D là $b,c,d$. Tính $S_{BMN}$. (b) Biết góc nhọn của hình bình hành bằng $\alpha$ \& bán kính $(ABC)$ bằng R. Tính $S_{BMN}$.
\end{baitoan}

\begin{baitoan}[\cite{TLCT_hinh_hoc_10}, 56., p. 134]
	$\Delta ABC$ có O,I,G lần lượt là tâm đường tròn ngoại tiếp, tâm đường tròn nội tiếp \& trọng tâm. (a) Chứng minh $IA\bot IG\Leftrightarrow\dfrac{a + b + c}{3} = \dfrac{2bc}{b + c}$. (b) Chứng minh $IA\bot IO\Leftrightarrow b + c = 2a$. (c) M,N là trung điểm AB,AC. Chứng minh A,I,M,N đồng viên $\Leftrightarrow b + c = 2a$.
\end{baitoan}

\begin{baitoan}[\cite{TLCT_hinh_hoc_10}, 57., p. 134]
	$\Delta ABC$, D,E trên BC thỏa $BD = DE = EC = \frac{1}{3}BC$. Đặt $\widehat{BAD} = \alpha,\widehat{DAE} = \beta,\widehat{EAC} = \gamma$. Chứng minh $(\cot\alpha + \cot\beta)(\cot\beta + \cot\gamma) = 4(1 + \cot^2\beta)$.
\end{baitoan}

\begin{baitoan}[\cite{TLCT_hinh_hoc_10}, 58., p. 135]
	$\Delta ABC$, $\widehat{B} > \widehat{C}$, AM,AD lần lượt là trung tuyến \& phân giác trong góc A. Đặt $\widehat{DAM} = \alpha$. (a) Chứng minh $\tan\alpha = \tan^2\dfrac{A}{2}\tan\dfrac{B - C}{2}$. (b) Đặt $\dfrac{AD}{AM} = k$. Chứng minh $\cos\alpha = \dfrac{k}{2}\sin^2\dfrac{A}{2} + \sqrt{\dfrac{k^2}{4}\sin^4\dfrac{A}{2} + \cos^2\dfrac{A}{2}}$.
\end{baitoan}

\begin{baitoan}
	Cho $\Delta ABC$ vuông tại $A$. (a) Cho trước 2 trong 6 số $a,b,c,b',c',h$. Tính 4 số còn lại theo 2 số đã cho. (c) Cho trước 2 trong 8 số $a,b,c,b',c',h,p,S$. Tính 6 số còn lại theo 2 số đã cho. (b) Cho trước 2 trong 14 số $a,b,c,b',c',h$, $m_a,m_b,m_c,d_a,d_b,d_c,p,S$ với $d_a,d_b,d_c$ lần lượt là 3 đường phân giác ứng với $BC,CA,AB$. Tính 12 số còn lại theo 2 số đã cho. Viết các chương trình {\sf Pascal, Python, C{\tt/}C++} để mô phỏng.
\end{baitoan}

\begin{baitoan}
	Cho $\Delta ABC$ Cho trước 3 trong 14 số $a,b,c,b',c',h$, $m_a,m_b,m_c,d_a,d_b,d_c,p,S$ với $d_a,d_b,d_c$ lần lượt là 3 đường phân giác ứng với $BC,CA,AB$. Tính 12 số còn lại theo 2 số đã cho. Viết các chương trình {\sf Pascal, Python, C{\tt/}C++} để mô phỏng.
\end{baitoan}

\begin{baitoan}
	Cho $\Delta ABC$. Tính $\sin A,\sin B,\sin C,\tan A,\tan B,\tan C,\cot A,\cot B,\cot C$ theo $a,b,c$.
\end{baitoan}

\begin{baitoan}
	Nếu chỉ cho số đo 3 góc của 1 tam giác, có thể giải tam giác đó không? Nếu có thì mô tả tập nghiệm các tam giác thỏa mãn.
\end{baitoan}

\begin{baitoan}
	Nếu cho trước độ dài 2 cạnh \& số đo 1 góc không nằm giữa 2 cạnh đó của 1 tam giác thì có giải tam giác đó được không?
\end{baitoan}

\begin{baitoan}
	Nếu cho trước độ dài 1 cạnh \& số đo 2 góc không cùng kề với cạnh đó của 1 tam giác thì có giải tam giác đó được không?
\end{baitoan}

\begin{baitoan}[{\sf Program: Solve triangle}]
	(a) Nêu các bộ 3 yếu tố cần cho trước về cạnh \& góc của 1 tam giác để tam giác đó có thể giải được. (b) Viết chương trình {\sf Pascal, Python, C{\tt/}C++} để minh họa.
\end{baitoan}

\begin{baitoan}
	Cho độ dài 3 cạnh của 1 tam giác. Tính độ dài 3 đường trung tuyến \& 6 góc tạo bởi 3 đường trung tuyến đó.
\end{baitoan}

\begin{baitoan}
	Cho độ dài 3 cạnh của 1 tam giác. (a) Tính độ dài 3 đường phân giác \& 6 đoạn tạo thành trên 3 cạnh. (b) Tính khoảng cách từ tâm đường tròn nội tiếp I đến 3 đỉnh \& 3 cạnh.
\end{baitoan}

\begin{baitoan}
	Cho độ dài 3 cạnh của 1 tam giác. (a) Tính độ dài 3 đường cao \& 6 góc tạo bởi 3 đường cao đó \& 6 đoạn tạo thành trên 3 cạnh. (b) Tính khoảng cách từ trực tâm đến 3 đỉnh \& 3 cạnh.
\end{baitoan}

\begin{baitoan}
	Cho độ dài 3 cạnh của 1 tam giác. Tính khoảng cách từ tâm đường tròn ngoại tiếp O đến 3 cạnh của tam giác đó.
\end{baitoan}

%------------------------------------------------------------------------------%

\section{Miscellaneous}

\begin{baitoan}
	Cần cho trước bao nhiêu yếu tố về cạnh, góc, đường chéo để giải 1 đa giác lồi đều n cạnh?
\end{baitoan}

\begin{baitoan}
	Cho độ dài 4 cạnh của 1 tứ giác lồi, liệu có thể giải được tứ giác đó không?
\end{baitoan}

\begin{baitoan}
	Cần cho trước bao nhiêu yếu tố về cạnh, góc, đường chéo của 1 tứ giác lồi để có thể giải được tứ giác đó?
\end{baitoan}

\begin{baitoan}
	Đặt 3 điện tích $q_1,q_2,q_3$ tại 3 đỉnh của $\Delta ABC$. Tính các lực điện từ.
\end{baitoan}

%------------------------------------------------------------------------------%

\printbibliography[heading=bibintoc]
	
\end{document}