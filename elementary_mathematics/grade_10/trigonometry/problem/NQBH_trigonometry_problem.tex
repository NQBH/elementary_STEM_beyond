\documentclass{article}
\usepackage[backend=biber,natbib=true,style=alphabetic,maxbibnames=50]{biblatex}
\addbibresource{/home/nqbh/reference/bib.bib}
\usepackage[utf8]{vietnam}
\usepackage{tocloft}
\renewcommand{\cftsecleader}{\cftdotfill{\cftdotsep}}
\usepackage[colorlinks=true,linkcolor=blue,urlcolor=red,citecolor=magenta]{hyperref}
\usepackage{amsmath,amssymb,amsthm,float,graphicx,mathtools,tikz}
\usetikzlibrary{angles,calc,intersections,matrix,patterns,quotes,shadings}
\allowdisplaybreaks
\newtheorem{assumption}{Assumption}
\newtheorem{baitoan}{}
\newtheorem{cauhoi}{Câu hỏi}
\newtheorem{conjecture}{Conjecture}
\newtheorem{corollary}{Corollary}
\newtheorem{dangtoan}{Dạng toán}
\newtheorem{definition}{Definition}
\newtheorem{dinhly}{Định lý}
\newtheorem{dinhnghia}{Định nghĩa}
\newtheorem{example}{Example}
\newtheorem{ghichu}{Ghi chú}
\newtheorem{hequa}{Hệ quả}
\newtheorem{hypothesis}{Hypothesis}
\newtheorem{lemma}{Lemma}
\newtheorem{luuy}{Lưu ý}
\newtheorem{nhanxet}{Nhận xét}
\newtheorem{notation}{Notation}
\newtheorem{note}{Note}
\newtheorem{principle}{Principle}
\newtheorem{problem}{Problem}
\newtheorem{proposition}{Proposition}
\newtheorem{question}{Question}
\newtheorem{remark}{Remark}
\newtheorem{theorem}{Theorem}
\newtheorem{vidu}{Ví dụ}
\usepackage[left=1cm,right=1cm,top=5mm,bottom=5mm,footskip=4mm]{geometry}
\def\labelitemii{$\circ$}
\DeclareRobustCommand{\divby}{%
	\mathrel{\vbox{\baselineskip.65ex\lineskiplimit0pt\hbox{.}\hbox{.}\hbox{.}}}%
}

\title{Problem: Trigonometrical Identities in Triangles -- Bài Tập: Hệ Thức Lượng Trong Tam Giác}
\author{Nguyễn Quản Bá Hồng\footnote{e-mail: {\sf nguyenquanbahong@gmail.com}, website: \url{https://nqbh.github.io}, Bến Tre, Việt Nam.}}
\date{\today}

\begin{document}
\maketitle
\tableofcontents

%------------------------------------------------------------------------------%

\section{Giá Trị Lượng Giác Của 1 Góc \& Hệ Thức Lượng Trong Tam Giác}
\fbox{1} $\forall\alpha\in[0^\circ;180^\circ]$, $\sin\alpha\in[-1;1]$, $\cos\alpha\in[-1;1]$. \fbox{2} $\cos\alpha > 0\Leftrightarrow\alpha\in(0^\circ;90^\circ)\Leftrightarrow\alpha$ nhọn. $\cos\alpha < 0\Leftrightarrow\alpha\in(90^\circ;180^\circ)\Leftrightarrow\alpha$ tù.

\begin{baitoan}[\cite{Hai_Hung_Thu_Tung2022_tap_1}, VD2, p. 22]
	Cho $\Delta ABC$ vuông tại A, đường cao AH, $D\in AB$ thỏa $BH = BD = CD$. Chứng minh $\dfrac{AD}{BD} = \sqrt[3]{2} - 1$.
\end{baitoan}

\begin{baitoan}[\cite{Hai_Hung_Thu_Tung2022_tap_1}, VD3, p. 23]
	Cho $\Delta ABC$. Chứng minh $\widehat{A} = 90^\circ\Leftrightarrow(\sqrt{a + b} + \sqrt{a - b})(\sqrt{a + c} + \sqrt{a - c}) = \sqrt{2}(a + b + c)$.
\end{baitoan}

%------------------------------------------------------------------------------%

\section{Giải Tam Giác}

%------------------------------------------------------------------------------%

\section{Miscellaneous}

%------------------------------------------------------------------------------%

\printbibliography[heading=bibintoc]
	
\end{document}