\documentclass{article}
\usepackage[backend=biber,natbib=true,style=alphabetic,maxbibnames=50]{biblatex}
\addbibresource{/home/nqbh/reference/bib.bib}
\usepackage[utf8]{vietnam}
\usepackage{tocloft}
\renewcommand{\cftsecleader}{\cftdotfill{\cftdotsep}}
\usepackage[colorlinks=true,linkcolor=blue,urlcolor=red,citecolor=magenta]{hyperref}
\usepackage{amsmath,amssymb,amsthm,float,graphicx,mathtools,tikz}
\usetikzlibrary{angles,calc,intersections,matrix,patterns,quotes,shadings}
\allowdisplaybreaks
\newtheorem{assumption}{Assumption}
\newtheorem{baitoan}{}
\newtheorem{cauhoi}{Câu hỏi}
\newtheorem{conjecture}{Conjecture}
\newtheorem{corollary}{Corollary}
\newtheorem{dangtoan}{Dạng toán}
\newtheorem{definition}{Definition}
\newtheorem{dinhly}{Định lý}
\newtheorem{dinhnghia}{Định nghĩa}
\newtheorem{example}{Example}
\newtheorem{ghichu}{Ghi chú}
\newtheorem{hequa}{Hệ quả}
\newtheorem{hypothesis}{Hypothesis}
\newtheorem{lemma}{Lemma}
\newtheorem{luuy}{Lưu ý}
\newtheorem{nhanxet}{Nhận xét}
\newtheorem{notation}{Notation}
\newtheorem{note}{Note}
\newtheorem{principle}{Principle}
\newtheorem{problem}{Problem}
\newtheorem{proposition}{Proposition}
\newtheorem{question}{Question}
\newtheorem{remark}{Remark}
\newtheorem{theorem}{Theorem}
\newtheorem{vidu}{Ví dụ}
\usepackage[left=1cm,right=1cm,top=5mm,bottom=5mm,footskip=4mm]{geometry}
\def\labelitemii{$\circ$}
\DeclareRobustCommand{\divby}{%
	\mathrel{\vbox{\baselineskip.65ex\lineskiplimit0pt\hbox{.}\hbox{.}\hbox{.}}}%
}

\title{Problem: 2D Method of Cartesian Coordinates\\Bài Tập: Phương Pháp Tọa Độ Cartesian Trong Mặt Phẳng}
\author{Nguyễn Quản Bá Hồng\footnote{A Scientist {\it\&} Creative Artist Wannabe. E-mail: {\tt nguyenquanbahong@gmail.com}. Bến Tre City, Việt Nam.}}
\date{\today}

\begin{document}
\maketitle
\begin{abstract}
	This text is a part of the series {\it Some Topics in Elementary STEM \& Beyond}:
	
	{\sc url}: \url{https://nqbh.github.io/elementary_STEM}.
	
	Latest version:
	\begin{itemize}
		\item {\it Problem: 2D Method of Cartesian Coordinates -- Bài Tập: Phương Pháp Tọa Độ Cartesian Trong Mặt Phẳng}.
		
		PDF: {\sc url}: \url{https://github.com/NQBH/elementary_STEM_beyond/blob/main/elementary_mathematics/grade_10/2D_method_coordinate/problem/NQBH_2D_method_coordinate_problem.pdf}.
		
		\TeX: {\sc url}: \url{https://github.com/NQBH/elementary_STEM_beyond/blob/main/elementary_mathematics/grade_10/2D_method_coordinate/problem/NQBH_2D_method_coordinate_problem.tex}.
		\item {\it Problem \& Solution: 2D Method of Cartesian Coordinates -- Bài Tập \& Lời Giải: Phương Pháp Tọa Độ Cartesian Trong Mặt Phẳng}.
		
		PDF: {\sc url}: \url{https://github.com/NQBH/elementary_STEM_beyond/blob/main/elementary_mathematics/grade_10/2D_method_coordinate/solution/NQBH_2D_method_coordinate_solution.pdf}.
		
		\TeX: {\sc url}: \url{https://github.com/NQBH/elementary_STEM_beyond/blob/main/elementary_mathematics/grade_10/2D_method_coordinate/solution/NQBH_2D_method_coordinate_solution.tex}.
	\end{itemize}
\end{abstract}
\tableofcontents

%------------------------------------------------------------------------------%

\section{2D Coordinate Vector -- Tọa Độ của Vector Trong Mặt Phẳng}
\fbox{1} {\sf Tọa độ của 1 điểm}: Ký hiệu $M(x_M,y_M)$ với $x_M,y_M\in\mathbb{R}$ lần lượt là hoành độ, tung độ của điểm $M\in\mathbb{R}^2$, cặp số $(x_M,y_M)$ được gọi là {\it tọa độ} của điểm $M$ trong mặt phẳng tọa độ $Oxy$. \fbox{2} {\sf Tọa độ của 1 vector}: $\overrightarrow{OM} = (a,b)\Leftrightarrow M(a,b)$. $\vec{i}(1,0),\vec{j}(0,1)$ lần lượt là {\it vector đơn vị} trên trục $Ox,Oy$. Với mỗi vector $\vec{u}$ trong mặt phẳng tọa độ $Oxy$, tọa độ của vector $\vec{u}$ là tọa độ của điểm $A(x_A,y_A)$ thỏa $\overrightarrow{OA} = \vec{u}$, khi đó $x_A,y_A$ lần lượt là hoành độ, tung độ của vector $\vec{u}$. \fbox{3} $\vec{u} = (a,b)\Leftrightarrow\vec{u} = a\vec{i} + b\vec{j}$. \fbox{4} $\vec{a} = \vec{b}\Leftrightarrow x_{\vec{a}} = x_{\vec{b}}\land y_{\vec{a}} = y_{\vec{b}}$. \fbox{4} $A(x_A,y_A),B(x_B,y_B)\Rightarrow\overrightarrow{AB} = (x_B - x_A,y_B - y_A)$. \fbox{5} Các điểm đối xứng với điểm $A(x_A,y_A)$ trong mặt phẳng tọa độ $Oxy$ qua gốc $O$, trục $Ox$, trục $Oy$ lần lượt là $(-x_A,-y_A),(x_A,-y_A),(-x_A,y_A)$.

\begin{baitoan}[Cf. segment vs. vector]
	(a) So sánh 2 khái niệm `2 đoạn thẳng bằng nhau' \& `2 vectors bằng nhau'. (b) So sánh 2 khái niệm `2 đoạn thẳng song song' \& `2 vectors song song'. (c) So sánh 2 khái niệm `2 tia đối nhau' \& `2 vectors đối nhau'.
\end{baitoan}

\begin{baitoan}[Mở rộng \cite{SGK_Toan_10_Canh_Dieu_tap_2}, 5., p. 65]
	Trong mặt phẳng tọa độ $Oxy$, cho trước tọa độ của 3 trong 5 điểm gồm 4 đỉnh của hình bình hành ABCD \& tâm đối xứng I của nó. (a) Tìm tọa độ các điểm còn lại. (b) Tính chu vi, độ dài 2 đường cao, \& diện tích của hình bình hành ABCD theo tọa độ của 3 điểm cho trước đó.
\end{baitoan}

\begin{baitoan}[Mở rộng \cite{SGK_Toan_10_Canh_Dieu_tap_2}, 6., p. 65]
	Trong mặt phẳng tọa độ $Oxy$, cho tứ giác ABCD có $A(x_A,y_A),B(x_B,y_B),C(x_C,y_C),D(x_D,y_D)$. Tìm điều kiện cần \& đủ để tứ giác ABCD là: (a) hình bình hành. (b) hình thang. (c) hình thang cân. (d) hình thoi. (e) hình chữ nhật. (f) hình vuông. (g) tứ giác nội tiếp. (h) tứ giác ngoại tiếp.
\end{baitoan}
{\sf Hint.} Tứ giác $ABCD$ là hình bình hành $\Leftrightarrow x_A + x_C = x_B + x_D\land y_A + y_C = y_B + y_D$.

\begin{baitoan}[Mở rộng \cite{SGK_Toan_10_Canh_Dieu_tap_2}, 7., p. 65: Tọa độ 3 đỉnh tam giác]
	Trong mặt phẳng tọa độ $Oxy$. (a) Cho biết tọa độ trung điểm 3 cạnh $\Delta ABC$. Tìm tọa độ 3 đỉnh $A,B,C$.
\end{baitoan}
{\sf Hint.} Hoặc giải 2 hệ phương trình ``khuyết'' 3 ẩn $(x_A,x_B,x_C)$ \& $(y_A,y_B,y_C)$. Hoặc dùng nhận xét nếu $M,N,P$ lần lượt là trung điểm của $BC,CA,AB$ thì $ANMP,BMNP,CMPN$ là 3 hình bình hành.

\begin{baitoan}[Tọa độ 4 đỉnh tứ giác]
	Trong mặt phẳng tọa độ $Oxy$. (a) Liệu nếu chỉ biết tọa độ 4 trung điểm của 4 cạnh tứ giác ABCD thì có thể tìm được tọa độ của 4 đỉnh $A,B,C,D$ không? (b) Nếu cho thêm tọa độ của trung điểm của 1 trong 2 đường chéo của tứ giác ABCD thì có thể tìm được tọa độ của 4 đỉnh $A,B,C,D$ không? (c) Nếu cho thêm tọa độ của 2 trung điểm của 2 đường chéo của tứ giác ABCD thì có thể tìm được tọa độ của 4 đỉnh $A,B,C,D$ không? Nếu được thì 6 tọa độ này phải thỏa điều kiện gì để bài toán có nghiệm? (d) Mở rộng bài toán cho đa giác $n$ cạnh $A_1A_2\ldots A_n$.
\end{baitoan}

\begin{baitoan}[Điều kiện cần \& đủ để 2 vectors bằng nhau]
	Trong mặt phẳng tọa độ $Oxy$, cho 2 vector $\vec{u}_1(a_1m + b_1n,c_1m + d_1n),\vec{u}_2(a_2m + b_2n,c_2m + d_2n)$. Tìm điều kiện cần \& đủ của $m,n$ theo $a_i,b_i,c_i,d_i\in\mathbb{R}$, $i = 1,2$ cho trước để: (a) $\vec{u}_1 = \vec{u}_2$. (b) $\vec{u}_1 + \vec{u}_2 = \vec{0}$. (c) $\vec{u}_1\parallel\vec{u}_2$, i.e., $\vec{u}_1,\vec{u}_2$ cùng phương. (d) $\vec{u}_1\uparrow\uparrow\vec{u}_2$, i.e., $\vec{u}_1,\vec{u}_2$ cùng phương, cùng hướng. (e) $\vec{u}_1\uparrow\downarrow\vec{u}_2$, i.e., $\vec{u}_1,\vec{u}_2$ cùng phương, ngược hướng. (f) $\vec{u}_1 = k\vec{u}_2$ với $k\in\mathbb{R}^\star$ cho trước. (g) $\vec{u}_1\bot\vec{u}_2$.
\end{baitoan}

\begin{baitoan}[Mở rộng \cite{SBT_Toan_10_Canh_Dieu_tap_2}, 11., p. 62]
		Trong mặt phẳng tọa độ $Oxy$, cho tọa độ 3 điểm không thẳng hàng $A(x_A,y_A),B(x_B,y_B)$, $C(x_C,y_C)$. Tìm tọa độ điểm D sao cho tứ giác ABCD là hình thang có $AB\parallel CD$ \& $CD = aAB$ với $a > 0$ cho trước.
\end{baitoan}
See also \href{https://en.wikipedia.org/wiki/Coordinate_vector}{Wikipedia{\tt/}coordinate vector}.

%------------------------------------------------------------------------------%

\section{Vector Calculus in Cartesian Coordinates -- Biểu Thức Tọa Độ của Các Phép Toán Vector}

%------------------------------------------------------------------------------%

\section{Line Equation -- Phương Trình Đường Thẳng}

%------------------------------------------------------------------------------%

\section{Relative Position -- Vị Trí Tương Đối \& Góc Giữa 2 Đường Thẳng. Khoảng Cách Từ 1 Điểm Đến 1 Đường Thẳng}

%------------------------------------------------------------------------------%

\section{Circle Equation -- Phương Trình Đường Tròn}

%------------------------------------------------------------------------------%

\section{3 Conics -- 3 Đường Conic}

%------------------------------------------------------------------------------%

\section{Miscellaneous}

%------------------------------------------------------------------------------%

\printbibliography[heading=bibintoc]
	
\end{document}