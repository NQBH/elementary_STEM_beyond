\documentclass{article}
\usepackage[backend=biber,natbib=true,style=alphabetic,maxbibnames=50]{biblatex}
\addbibresource{/home/nqbh/reference/bib.bib}
\usepackage[utf8]{vietnam}
\usepackage{tocloft}
\renewcommand{\cftsecleader}{\cftdotfill{\cftdotsep}}
\usepackage[colorlinks=true,linkcolor=blue,urlcolor=red,citecolor=magenta]{hyperref}
\usepackage{amsmath,amssymb,amsthm,float,graphicx,mathtools,tikz}
\usetikzlibrary{angles,calc,intersections,matrix,patterns,quotes,shadings}
\allowdisplaybreaks
\newtheorem{assumption}{Assumption}
\newtheorem{baitoan}{}
\newtheorem{cauhoi}{Câu hỏi}
\newtheorem{conjecture}{Conjecture}
\newtheorem{corollary}{Corollary}
\newtheorem{dangtoan}{Dạng toán}
\newtheorem{definition}{Definition}
\newtheorem{dinhly}{Định lý}
\newtheorem{dinhnghia}{Định nghĩa}
\newtheorem{example}{Example}
\newtheorem{ghichu}{Ghi chú}
\newtheorem{hequa}{Hệ quả}
\newtheorem{hypothesis}{Hypothesis}
\newtheorem{lemma}{Lemma}
\newtheorem{luuy}{Lưu ý}
\newtheorem{nhanxet}{Nhận xét}
\newtheorem{notation}{Notation}
\newtheorem{note}{Note}
\newtheorem{principle}{Principle}
\newtheorem{problem}{Problem}
\newtheorem{proposition}{Proposition}
\newtheorem{question}{Question}
\newtheorem{remark}{Remark}
\newtheorem{theorem}{Theorem}
\newtheorem{vidu}{Ví dụ}
\usepackage[left=1cm,right=1cm,top=5mm,bottom=5mm,footskip=4mm]{geometry}
\def\labelitemii{$\circ$}
\DeclareRobustCommand{\divby}{%
	\mathrel{\vbox{\baselineskip.65ex\lineskiplimit0pt\hbox{.}\hbox{.}\hbox{.}}}%
}

\title{Problem: 2D Method of Cartesian Coordinates\\Bài Tập: Phương Pháp Tọa Độ Cartesian Trong Mặt Phẳng}
\author{Nguyễn Quản Bá Hồng\footnote{A Scientist {\it\&} Creative Artist Wannabe. E-mail: {\tt nguyenquanbahong@gmail.com}. Bến Tre City, Việt Nam.}}
\date{\today}

\begin{document}
\maketitle
\begin{abstract}
	This text is a part of the series {\it Some Topics in Elementary STEM \& Beyond}:
	
	{\sc url}: \url{https://nqbh.github.io/elementary_STEM}.
	
	Latest version:
	\begin{itemize}
		\item {\it Problem: 2D Method of Cartesian Coordinates -- Bài Tập: Phương Pháp Tọa Độ Cartesian Trong Mặt Phẳng}.
		
		PDF: {\sc url}: \url{https://github.com/NQBH/elementary_STEM_beyond/blob/main/elementary_mathematics/grade_10/2D_method_coordinate/problem/NQBH_2D_method_coordinate_problem.pdf}.
		
		\TeX: {\sc url}: \url{https://github.com/NQBH/elementary_STEM_beyond/blob/main/elementary_mathematics/grade_10/2D_method_coordinate/problem/NQBH_2D_method_coordinate_problem.tex}.
		\item {\it Problem \& Solution: 2D Method of Cartesian Coordinates -- Bài Tập \& Lời Giải: Phương Pháp Tọa Độ Cartesian Trong Mặt Phẳng}.
		
		PDF: {\sc url}: \url{https://github.com/NQBH/elementary_STEM_beyond/blob/main/elementary_mathematics/grade_10/2D_method_coordinate/solution/NQBH_2D_method_coordinate_solution.pdf}.
		
		\TeX: {\sc url}: \url{https://github.com/NQBH/elementary_STEM_beyond/blob/main/elementary_mathematics/grade_10/2D_method_coordinate/solution/NQBH_2D_method_coordinate_solution.tex}.
	\end{itemize}
\end{abstract}
\tableofcontents

%------------------------------------------------------------------------------%

\noindent\textbf{\textsf{Resources -- Tài nguyên.}}
\begin{enumerate}
	\item \cite{Hai_Hung_Thu_Tung_ncpt_Toan_10_tap_2}. {\sc Phan Việt Hải, Trần Quang Hùng, Ninh Văn Thu, Phạm Đình Tùng}. {\it Nâng Cao \& Phát Triển Toán 10. Tập 2}.
\end{enumerate}

\section{2D Coordinate Vector -- Tọa Độ của Vector Trong Mặt Phẳng}
\fbox{1} {\sf Tọa độ của 1 điểm}: Ký hiệu $M(x_M,y_M)$ với $x_M,y_M\in\mathbb{R}$ lần lượt là hoành độ, tung độ của điểm $M\in\mathbb{R}^2$, cặp số $(x_M,y_M)$ được gọi là {\it tọa độ} của điểm $M$ trong mặt phẳng tọa độ $Oxy$. \fbox{2} {\sf Tọa độ của 1 vector}: $\overrightarrow{OM} = (a,b)\Leftrightarrow M(a,b)$. $\vec{i}(1,0),\vec{j}(0,1)$ lần lượt là {\it vector đơn vị} trên trục $Ox,Oy$. Với mỗi vector $\vec{u}$ trong mặt phẳng tọa độ $Oxy$, tọa độ của vector $\vec{u}$ là tọa độ của điểm $A(x_A,y_A)$ thỏa $\overrightarrow{OA} = \vec{u}$, khi đó $x_A,y_A$ lần lượt là hoành độ, tung độ của vector $\vec{u}$. \fbox{3} $\vec{u} = (a,b)\Leftrightarrow\vec{u} = a\vec{i} + b\vec{j}$. \fbox{4} $\vec{a} = \vec{b}\Leftrightarrow x_{\vec{a}} = x_{\vec{b}}\land y_{\vec{a}} = y_{\vec{b}}$. \fbox{4} $A(x_A,y_A),B(x_B,y_B)\Rightarrow\overrightarrow{AB} = (x_B - x_A,y_B - y_A)$. \fbox{5} Các điểm đối xứng với điểm $A(x_A,y_A)$ trong mặt phẳng tọa độ $Oxy$ qua gốc $O$, trục $Ox$, trục $Oy$ lần lượt là $(-x_A,-y_A),(x_A,-y_A),(-x_A,y_A)$. See \href{https://en.wikipedia.org/wiki/Coordinate_vector}{Wikipedia{\tt/}coordinate vector}.

\begin{baitoan}[Cf. segment vs. vector]
	(a) So sánh 2 khái niệm `2 đoạn thẳng bằng nhau' \& `2 vectors bằng nhau'. (b) So sánh 2 khái niệm `2 đoạn thẳng song song' \& `2 vectors song song'. (c) So sánh 2 khái niệm `2 tia đối nhau' \& `2 vectors đối nhau'.
\end{baitoan}

\begin{baitoan}[Mở rộng \cite{SGK_Toan_10_Canh_Dieu_tap_2}, 5., p. 65]
	Trong mặt phẳng tọa độ $Oxy$, cho trước tọa độ của 3 trong 5 điểm gồm 4 đỉnh của hình bình hành ABCD \& tâm đối xứng I của nó. (a) Tìm tọa độ các điểm còn lại. (b) Tính chu vi, độ dài 2 đường cao, \& diện tích của hình bình hành ABCD theo tọa độ của 3 điểm cho trước đó.
\end{baitoan}

\begin{baitoan}[Mở rộng \cite{SGK_Toan_10_Canh_Dieu_tap_2}, 6., p. 65]
	Trong mặt phẳng tọa độ $Oxy$, cho tứ giác ABCD có $A(x_A,y_A),B(x_B,y_B),C(x_C,y_C),D(x_D,y_D)$. Tìm điều kiện cần \& đủ để tứ giác ABCD là: (a) hình bình hành. (b) hình thang. (c) hình thang cân. (d) hình thoi. (e) hình chữ nhật. (f) hình vuông. (g) tứ giác nội tiếp. (h) tứ giác ngoại tiếp.
\end{baitoan}
{\sf Hint.} Tứ giác $ABCD$ là hình bình hành $\Leftrightarrow x_A + x_C = x_B + x_D\land y_A + y_C = y_B + y_D$.

\begin{baitoan}[Mở rộng \cite{SGK_Toan_10_Canh_Dieu_tap_2}, 7., p. 65: Tọa độ 3 đỉnh tam giác]
	Trong mặt phẳng tọa độ $Oxy$. (a) Cho biết tọa độ trung điểm 3 cạnh $\Delta ABC$. Tìm tọa độ 3 đỉnh $A,B,C$.
\end{baitoan}
{\sf Hint.} Hoặc giải 2 hệ phương trình ``khuyết'' 3 ẩn $(x_A,x_B,x_C)$ \& $(y_A,y_B,y_C)$. Hoặc dùng nhận xét nếu $M,N,P$ lần lượt là trung điểm của $BC,CA,AB$ thì $ANMP,BMNP,CMPN$ là 3 hình bình hành.

\begin{baitoan}[Tọa độ 4 đỉnh tứ giác]
	Trong mặt phẳng tọa độ $Oxy$. (a) Liệu nếu chỉ biết tọa độ 4 trung điểm của 4 cạnh tứ giác ABCD thì có thể tìm được tọa độ của 4 đỉnh $A,B,C,D$ không? (b) Nếu cho thêm tọa độ của trung điểm của 1 trong 2 đường chéo của tứ giác ABCD thì có thể tìm được tọa độ của 4 đỉnh $A,B,C,D$ không? (c) Nếu cho thêm tọa độ của 2 trung điểm của 2 đường chéo của tứ giác ABCD thì có thể tìm được tọa độ của 4 đỉnh $A,B,C,D$ không? Nếu được thì 6 tọa độ này phải thỏa điều kiện gì để bài toán có nghiệm? (d) Mở rộng bài toán cho đa giác $n$ cạnh $A_1A_2\ldots A_n$.
\end{baitoan}

\begin{baitoan}[Điều kiện cần \& đủ để 2 vectors bằng nhau]
	Trong mặt phẳng tọa độ $Oxy$, cho 2 vector $\vec{u}_1(a_1m + b_1n,c_1m + d_1n),\vec{u}_2(a_2m + b_2n,c_2m + d_2n)$. Tìm điều kiện cần \& đủ của $m,n$ theo $a_i,b_i,c_i,d_i\in\mathbb{R}$, $i = 1,2$ cho trước để: (a) $\vec{u}_1 = \vec{u}_2$. (b) $\vec{u}_1 + \vec{u}_2 = \vec{0}$. (c) $\vec{u}_1\parallel\vec{u}_2$, i.e., $\vec{u}_1,\vec{u}_2$ cùng phương. (d) $\vec{u}_1\uparrow\uparrow\vec{u}_2$, i.e., $\vec{u}_1,\vec{u}_2$ cùng phương, cùng hướng. (e) $\vec{u}_1\uparrow\downarrow\vec{u}_2$, i.e., $\vec{u}_1,\vec{u}_2$ cùng phương, ngược hướng. (f) $\vec{u}_1 = k\vec{u}_2$ với $k\in\mathbb{R}^\star$ cho trước. (g) $\vec{u}_1\bot\vec{u}_2$.
\end{baitoan}

\begin{baitoan}[Mở rộng \cite{SBT_Toan_10_Canh_Dieu_tap_2}, 11., p. 62]
		Trong mặt phẳng tọa độ $Oxy$, cho tọa độ 3 điểm không thẳng hàng $A(x_A,y_A),B(x_B,y_B)$, $C(x_C,y_C)$. Tìm tọa độ điểm D sao cho tứ giác ABCD là hình thang có $AB\parallel CD$ \& $CD = aAB$ với $a > 0$ cho trước.
\end{baitoan}

%------------------------------------------------------------------------------%

\section{Vector Calculus in Cartesian Coordinates -- Biểu Thức Tọa Độ của Các Phép Toán Vector}
\fbox{1} {\sf Biểu thức tọa độ của phép $\pm$ vectors, phép nhân vô hướng của vector}: Nếu $\vec{u} = (x_1,y_1),\vec{v} = (x_2,y_2)$ thì $\vec{u} + \vec{v} = (x_1 + x_2,y_1 + y_2)$, $\vec{u} - \vec{v} = (x_1 - x_2,y_1 - y_2)$ hay viết gộp chung thành $\vec{u}\pm\vec{v} = (x_1\pm x_2,y_1\pm y_2)$, $k\vec{u} = (kx_1,ky_1)$, $\forall k\in\mathbb{R}$; $\vec{u}\parallel\vec{v}\ne\vec{0}$, i.e., $\vec{u}\parallel\vec{v}$ cùng phương $\Leftrightarrow\exists k\in\mathbb{R}$ sao cho $x_1 = kx_2\land y_1 = ky_2$. \fbox{2} {\it Tọa độ trung điểm} của đoạn thẳng $AB$ với $A(x_A,y_A),B(x_B,y_B)$ là $M(x_M,y_M)$ với $x_M\coloneqq\frac{1}{2}(x_A + x_B),y_M\coloneqq\frac{1}{2}(y_A + y_B)$. \fbox{3} {\it Tọa độ trọng tâm} của tam giác $\Delta ABC$ với $A(x_A,y_A),B(x_B,y_B),C(x_C,y_C)$ là $G(x_G,y_G)$ với $x_G\coloneqq\frac{1}{3}(x_A + x_B + x_C),y_G\coloneqq\frac{1}{3}(y_A + y_B + y_C)$. \fbox{4} {\sf Biểu thức tọa độ của tích vô hướng}: Với $\vec{u} = (x_1,y_1),\vec{v} = (x_2,y_2)$, $\vec{u}\cdot\vec{v} = x_1x_2 + y_1y_2$. $\vec{i}^2 = |\vec{i}|^2 = \vec{j}^2 = |\vec{j}|^2 = 1$, $\vec{i}\cdot\vec{j} = 0$. Nếu $\vec{a} = (x,y)$ thì $|\vec{a}| = \sqrt{\vec{a}\cdot\vec{a}} = \sqrt{\vec{a}^2} = \sqrt{x^2 + y^2}$. Nếu $A(x_1,y_1),B(x_2,y_2)$ thì $AB = |\overrightarrow{AB}| = \sqrt{(x_2 - x_1)^2 + (y_2 - y_1)^2}$. \fbox{5} Với $\vec{u} = (x_1,y_1),\vec{v} = (x_2,y_2)\ne\vec{0}$, $\vec{u}\bot\vec{v}\Leftrightarrow\vec{u}\cdot\vec{v} = x_1x_2 + y_1y_2 = 0$ \&
\begin{equation}
	\cos(\vec{u},\vec{v}) = \frac{\vec{u}\cdot\vec{v}}{|\vec{u}||\vec{v}|} = \frac{x_1x_2 + y_1y_2}{\sqrt{x_1^2 + y_1^2}\sqrt{x_2^2 + y_2^2}},\ (\vec{u},\vec{v}) = \arccos\frac{\vec{u}\cdot\vec{v}}{|\vec{u}||\vec{v}|} = \arccos\frac{x_1x_2 + y_1y_2}{\sqrt{x_1^2 + y_1^2}\sqrt{x_2^2 + y_2^2}}.
\end{equation}

\begin{baitoan}[Coordinate of linear combination of vectors -- Tọa độ của tổ hợp tuyến tính các vectors]
	Cho $n\in\mathbb{N}^\star$ số thực $a_i$ \& $n$ vectors $\vec{u}_i(x_i,y_i)$, $i = 1,2,\ldots,n$. Tìm tọa độ của vector $\sum_{i=1}^n a_i\vec{u}_i$.
\end{baitoan}

\begin{baitoan}[Điều kiện cần \& đủ để hệ điểm thẳng hàng]
	(a) Tìm điều kiện cần \& đủ theo tọa độ để 3 điểm $A,B,C$ thẳng hàng, không thẳng hàng. (b) Tìm điều kiện cần \& đủ theo tọa độ để $n\in\mathbb{N}^\star$ điểm $A_i$, $i = 1,\ldots,n$ thẳng hàng, không thẳng hàng.
\end{baitoan}

\begin{baitoan}
	Cho 2 điểm $A(x_A,y_A),M(x_M,y_M)$. Tìm tọa độ điểm $B(x_B,y_B)$ sao cho M là trung điểm đoạn thẳng AB.
\end{baitoan}

\begin{baitoan}
	(a) Cho 3 điểm $A(x_A,y_A),B(x_B,y_B),G(x_G,y_G)$ không thẳng hàng. Tìm tọa độ điểm $C(x_C,y_C)$ để G là trọng tâm $\Delta ABC$. (b) $M,N,P$ lần lượt là trung điểm $BC,CA,AB$. Tìm biểu thức tọa độ của tính chất hình học $AG = 2GM,BG = 2GN,CG = 2GP$.
\end{baitoan}

\begin{baitoan}[Giải tam giác trên mặt phẳng tọa độ]
	Trong mặt phẳng tọa độ $Oxy$, cho $\Delta ABC$ có $A(x_A,y_A),B(x_B,y_B),C(x_C,y_C)$. (a) Viết định lý sin \& định lý cos phiên bản tọa độ. (b) Giải $\Delta ABC$.
\end{baitoan}

\begin{baitoan}[Tổng hợp lực]
	(a) Tính lực tổng hợp $\vec{F}$ của 2 lực $\overrightarrow{F_1},\overrightarrow{F_2}$, i.e., $\overrightarrow{F}\coloneqq\overrightarrow{F_1} + \overrightarrow{F_2}$ biết số đo $( \overrightarrow{F_1},\overrightarrow{F_2})$. (b) Tính lực tổng hợp $\vec{F}$ của$n\in\mathbb{N}^\star$ lực $\overrightarrow{F_i}$ với $i = 1,2,\ldots,n$, i.e., $\overrightarrow{F}\coloneqq\sum_{i=1}^n \overrightarrow{F_i} = \overrightarrow{F_1} + \cdots + \overrightarrow{F_n}$ biết số đo các góc $( \overrightarrow{F_i},\overrightarrow{F_j})$ với $1\le i < j\le n$.
\end{baitoan}

\begin{baitoan}[Tọa độ tâm tỷ cự]
	(a) Tìm tọa độ của tâm tỷ cự của hệ $n\in\mathbb{N}^\star$ điểm $A_i(x_i,y_i)$ với trọng số $\alpha_i\in\mathbb{R}$, $i = 1,2,\ldots,n$, i.e., $\sum_{i=1}^n \alpha_i\overrightarrow{AA_i} = \vec{0}$. (b) Viết biểu thức vector $n\overrightarrow{MA} = \sum_{i=1}^n \alpha_i\overrightarrow{MA_i}$ phiên bản tọa độ.
\end{baitoan}

\begin{baitoan}
	Cho $\Delta ABC,\Delta MNP$ với tọa độ 6 đỉnh. Tìm điều kiện cần \& đủ của 6 tọa độ này để $\Delta ABC,\Delta MNP$ có cùng: (a) trọng tâm. (b) trực tâm. (c) tâm đường tròn nội tiếp. (d) tâm đường tròn ngoại tiếp. (e) Mở rộng cho các tâm tỷ cự khác.
\end{baitoan}

%\begin{baitoan}
%	
%\end{baitoan}
%
%\begin{baitoan}
%	
%\end{baitoan}
%
%\begin{baitoan}
%	
%\end{baitoan}

%------------------------------------------------------------------------------%

\section{2D Line Equation -- Phương Trình Đường Thẳng Trong Mặt Phẳng}

\begin{baitoan}[\cite{Hai_Hung_Thu_Tung_ncpt_Toan_10_tap_2}, VD1, p. 48]
	Cho 3 điểm $A(0,2),B(2,2),C(4,-2)$. (a) Viết phương trình 3 cạnh $\Delta ABC$. (b) Viết phương trình 3 đường cao của $\Delta ABC$. Từ đó chứng minh chúng đồng quy tại trực tâm $H$. (c) Viết phương trình 3 đường trung tuyến của $\Delta ABC$. Từ đó chứng minh chúng đồng quy tại trọng tâm $G$. (d) Viết phương trình 3 đường trung trực của $\Delta ABC$. Từ đó chứng minh chúng đồng quy tại tâm đường tròn ngoại tiếp $E$. Tính bán kính đường tròn ngoại tiếp $\Delta ABC$. (e) Kiểm tra lại đường thẳng Euler theo tọa độ. Từ đó viết phương trình đường thẳng Euler. (f) Viết phương trình các đường trung bình của $\Delta ABC$.
\end{baitoan}

\begin{baitoan}[\cite{Hai_Hung_Thu_Tung_ncpt_Toan_10_tap_2}, VD2, p. 49]
	Cho $\Delta ABC$ biết $A(0,2)$, phương trình 2 đường cao $BB_2:x - y = 0,CC_2: x = 4$. Tìm tọa độ $B,C$.
\end{baitoan}

\begin{baitoan}[\cite{Hai_Hung_Thu_Tung_ncpt_Toan_10_tap_2}, VD3, p. 50]
	Cho $\Delta ABC$ biết $A(0,2)$, phương trình đường cao $BB_2:x - y = 0$ \& phương trình đường trung tuyến $CC_1:4x + 3y - 10 = 0$. Tìm tọa độ $B,C$.
\end{baitoan}

\begin{baitoan}[\cite{Hai_Hung_Thu_Tung_ncpt_Toan_10_tap_2}, VD4, p. 50]
	Cho điểm $A(1,1)$, điểm $C$ nằm trên trục hoành \& điểm $B$ thuộc đường thẳng $d:y - 3 = 0$. Tìm tọa độ $B,C$ để $\Delta ABC$ đều.
\end{baitoan}

\begin{baitoan}[\cite{Hai_Hung_Thu_Tung_ncpt_Toan_10_tap_2}, VD5, p. 51]
	Cho $\Delta ABC$ vuông tại $A$, $AB = c,AC = b$, $b,c > 0$ không đổi. $B,C$ lần lượt chuyển động trên $(x'Ox),(y'Oy)$. Tìm quỹ tích của điểm $A$.
\end{baitoan}

\begin{baitoan}[\cite{Hai_Hung_Thu_Tung_ncpt_Toan_10_tap_2}, VD6, p. 52]
	Cho $A(a,0),B(0,b)$ với $a,b\in\mathbb{R}^\star$: hằng số. $M,N$ lần lượt thuộc trục hoành, trục tung thỏa mãn $\dfrac{\overline{OM}}{\overline{OA}} + \dfrac{\overline{ON}}{\overline{OB}} = 2$, $(AN)\cap(BM) = E$. Tìm quỹ tích của $E$.
\end{baitoan}

\begin{baitoan}[\cite{Hai_Hung_Thu_Tung_ncpt_Toan_10_tap_2}, VD7, p. 52]
	Cho điểm $M(1,2)$. Xét đường thẳng $d$ đi qua $M$ cắt 2 tia $Ox,Oy$ lần lượt tại $A,B\ne O$. Viết phương trình đường thẳng $d$ nếu: (a) $OA = OB$. (b) $OA = 2OB$, $OA = aOB$ với $a\in(0,\infty)$. (c) $S_{OAB}$ nhỏ nhất. (d) $OA + OB$ nhỏ nhất. (e) $\left(\dfrac{1}{OA^2} + \dfrac{1}{OB^2}\right)$ nhỏ nhất.
\end{baitoan}

\begin{baitoan}[\cite{Hai_Hung_Thu_Tung_ncpt_Toan_10_tap_2}, VD8, p. 53]
	Trên mặt phẳng tọa độ cho $A(1,3),B(4,2)$. Viết phương trình đường thẳng $d$ biết: (a) $d$ đi qua $A$ \& cách $B$ $3$ đơn vị. (b) $d$ đi qua $A$ \& cách $B$ 1 khoảng nhỏ nhất. (c) $d$ đi qua $A$ \& cách $B$ 1 khoảng lớn nhất. (d) $d$ cách $A$ 1 đơn vị \& cách $B$ $2$ đơn vị.
\end{baitoan}

\begin{baitoan}[\cite{Hai_Hung_Thu_Tung_ncpt_Toan_10_tap_2}, VD9, p. 54]
	Cho đường thẳng $d$ \& 2 điểm $M_1,M_2\notin d$. Tìm thuật toán để xác định điểm $M\in d$ thỏa: (a) $MM_1 + MM_2$ nhỏ nhất. (b) $|MM_1 - MM_2|$ lớn nhất.
\end{baitoan}

\begin{baitoan}[\cite{Hai_Hung_Thu_Tung_ncpt_Toan_10_tap_2}, VD10, p. 54]
	Cho $M_1(1,2),M_2(0,3),d:3x - 4y + 6 = 0$. Tìm điểm $M\in d$ thỏa: (a) $MM_1 + MM_2$ nhỏ nhất. (b) $|MM_1 - MM_2|$ lớn nhất.
\end{baitoan}

\begin{baitoan}[\cite{Hai_Hung_Thu_Tung_ncpt_Toan_10_tap_2}, VD11, p. 55]
	Cho 2 đường thẳng $d_1:a_1x + b_1y + c_1 = 0$, $d_2:a_2x + b_2y + c_2 = 0$. Tìm \& lập phương trình quỹ tích các điểm $M$ sao cho $M$ cách đều 2 đường thẳng $d_1,d_2$.
\end{baitoan}

\begin{baitoan}[\cite{Hai_Hung_Thu_Tung_ncpt_Toan_10_tap_2}, VD12, p. 55]
	Cho 2 đường thẳng $d_1:x + y + 2 = 0,d_2:7x - y + 5 = 0$. Tìm quỹ tích của điểm $M$ sao cho $M$ cách đều 2 đường thẳng $d_1,d_2$.
\end{baitoan}

\begin{baitoan}[\cite{Hai_Hung_Thu_Tung_ncpt_Toan_10_tap_2}, VD13, p. 56]
	Cho $\Delta ABC$, biết tọa độ 3 đỉnh. Tìm thuật toán xác định tọa độ tâm đường tròn nội tiếp, tâm 3 đường tròn bàng tiếp \& tính 4 bán kính tương ứng.
\end{baitoan}

\begin{baitoan}[\cite{Hai_Hung_Thu_Tung_ncpt_Toan_10_tap_2}, VD14, p. 56]
	Cho điểm $A(\frac{28}{3},\frac{22}{3}),B(1,1),C(0,-6)$. Xác định tọa độ tâm đường tròn nội tiếp, tâm 3 đường tròn bàng tiếp \& tính 4 bán kính tương ứng.
\end{baitoan}

\begin{baitoan}[\cite{Hai_Hung_Thu_Tung_ncpt_Toan_10_tap_2}, 25.1., p. 57]
	Cho $\Delta ABC$ biết $A(0,2)$, phương trình đường cao $BB_2:x - y = 0$, phương trình đường trung tuyến $BB_1:x = 2$. Tìm tọa độ $B,C$.
\end{baitoan}

\begin{baitoan}[\cite{Hai_Hung_Thu_Tung_ncpt_Toan_10_tap_2}, 25.2., p. 57]
	Cho $\Delta ABC$ biết $A(0,2)$, phương trình 2 đường trung tuyến $BB_1:x = 2$, $CC_1:4x + 3y - 10 = 0$. Tìm tọa độ $B,C$.
\end{baitoan}

\begin{baitoan}[\cite{Hai_Hung_Thu_Tung_ncpt_Toan_10_tap_2}, 25.3., p. 57]
	Cho đường thẳng $d:x - 2y + 1 = 0$. (a) Viết phương trình đường thẳng $\Delta\parallel d$ \& cách $d$ $3$ đơn vị. (b) Trong các đường thẳng thu được, đường thẳng nào \& gốc tọa độ nằm về 2 phía của $d$?
\end{baitoan}

%------------------------------------------------------------------------------%

\section{Relative Position -- Vị Trí Tương Đối \& Góc Giữa 2 Đường Thẳng. Khoảng Cách Từ 1 Điểm Đến 1 Đường Thẳng}

%------------------------------------------------------------------------------%

\section{2D Circle Equation -- Phương Trình Đường Tròn Trong Mặt Phẳng}

\begin{baitoan}[\cite{Hai_Hung_Thu_Tung_ncpt_Toan_10_tap_2}, VD1, p. 59]
	Cho họ đường tròn $(C_m):x^2 + y^2 - 2mx + 2(m + 1)y - 2m - 1 = 0$, $m\in\mathbb{R}$. (a) Tìm $m\in\mathbb{R}$ để $(C_m)$ là phương trình đường tròn. (b) Có tồn tại $R_m$ nhỏ nhất hoặc lớn nhất không? (c) Tìm quỹ tích tâm $I_m$. (d) Đường tròn có đi qua điểm cố định không?
\end{baitoan}

\begin{baitoan}[\cite{Hai_Hung_Thu_Tung_ncpt_Toan_10_tap_2}, VD2, p. 59]
	Cho họ đường tròn $(C_m):x^2 + y^2 - 2(\cos\alpha - 1)x - 2(\sin\alpha - 1)y - 1 = 0$ với tham số $\alpha\in\mathbb{R}$. (a) Tìm $m\in\mathbb{R}$ để $(C_m)$ là phương trình đường tròn. (b) Có tồn tại $R_m$ nhỏ nhất hoặc lớn nhất không? (c) Tìm quỹ tích tâm $I_m$. (d) Đường tròn có đi qua điểm cố định không?
\end{baitoan}

\begin{baitoan}[\cite{Hai_Hung_Thu_Tung_ncpt_Toan_10_tap_2}, VD3, p. 60]
	Cho 2 điểm $A,B$. Viết phương trình đường tròn $(C)$ đi qua 2 điểm $A,B$ thỏa: (a) Đường tròn bé nhất. (b) Bán kính cho trước. (c) Có tâm $I$ thuộc đường thẳng $d$ cho trước. (d) Tiếp xúc với đường thẳng $d_1$ cho trước. (e) Tiếp xúc với đường tròn $(C_1)$ cho trước.
\end{baitoan}

\begin{baitoan}[\cite{Hai_Hung_Thu_Tung_ncpt_Toan_10_tap_2}, VD4, p. 61]
	Cho 3 điểm $A(0,2),B(-2,2),C(4,-2)$. Viết phương trình: (a) đường tròn ngoại tiếp $\Delta ABC$. (b) đường tròn bé nhất chứa $\Delta ABC$.
\end{baitoan}

\begin{baitoan}[\cite{Hai_Hung_Thu_Tung_ncpt_Toan_10_tap_2}, VD5, p. 62]
	Cho $\Delta ABC$ có $AB = AC$. Đường tròn $(C)$ tiếp xúc với $AB$ tại $B$, tiếp xúc với $AC$ tại $C$ \& điểm $M\in(C)$ tùy ý. $A_1,B_1,C_1$ lần lượt là hình chiếu của $M$ lên $BC,CA,AB$. Chứng minh $MA_1^2 = MB_1\cdot MC_1$.
\end{baitoan}

\begin{baitoan}[\cite{Hai_Hung_Thu_Tung_ncpt_Toan_10_tap_2}, VD6, p. 63]
	Cho đường tròn $(O)$, điểm $A$ cố định nằm ngoài đường tròn \& điểm $M$ thuộc đường tròn. Xét đường tròn $(M;MA)$. Chứng minh trục đẳng phương của 2 đường tròn luôn tiếp xúc với 1 đường tròn cố định.
\end{baitoan}

\begin{baitoan}[\cite{Hai_Hung_Thu_Tung_ncpt_Toan_10_tap_2}, 26.1., p. 64]
	Cho họ đường tròn $(C_m):x^2 + y^2 - 2(m + 1)x + 4my - 5 = 0$, $m\in\mathbb{R}$, \& đường tròn $(C):x^2 + y^2 = 1$. (a) Chứng minh có 2 đường tròn thuộc họ $(C_m)$ tiếp xúc với $(C)$. (b) Viết phương trình các tiếp tuyến chung của 2 đường tròn thu được. (c) Tìm $m\in\mathbb{R}$ để $(C_m)$ cắt $(C)$ theo 1 dây cung lớn nhất.
\end{baitoan}

\begin{baitoan}[\cite{Hai_Hung_Thu_Tung_ncpt_Toan_10_tap_2}, 26.2., p. 64]
	Cho $\Delta ABC$ có $A(1,5),B(4,-1),C(-4,-5)$. Viết phương trình các đường tròn nội tiếp \& bàng tiếp của $\Delta ABC$.
\end{baitoan}

\begin{baitoan}[\cite{Hai_Hung_Thu_Tung_ncpt_Toan_10_tap_2}, 26.3., p. 64]
	Cho đường tròn $(C):x^2 + y^2 = 4$ \& điểm $M(1,-1)$. Viết phương trình đường tròn $(C_1)$ có bán kính $R_1 = 3$ \& cắt đường tròn $(C)$ theo dây cung bé nhất qua $M$.
\end{baitoan}

\begin{baitoan}[\cite{Hai_Hung_Thu_Tung_ncpt_Toan_10_tap_2}, 26.4., p. 64]
	Cho đường thẳng $\Delta_1:mx - (m + 1)y + 2m - 1 = 0$, $\Delta_2:(m + 1)x + my - m + 2 = 0$, $m\in\mathbb{R}$. Tìm quỹ tích giao điểm của $\Delta_1,\Delta_2$.
\end{baitoan}

\begin{baitoan}[\cite{Hai_Hung_Thu_Tung_ncpt_Toan_10_tap_2}, 26.5., p. 64]
	Cho đường tròn tâm $O$ đường kính $AB$ \& điểm $M$ chuyển động trên đường tròn. Kẻ $MH\bot AB$, $H\in AB$. Vẽ đường tròn $(M;MH)$ cắt đường tròn $(O)$ tại $C,D$. Chứng minh đường thẳng $CD$ là tiếp tuyến chung của 2 đường tròn đường kính $HA,HB$.
\end{baitoan}

%------------------------------------------------------------------------------%

\section{3 Conics -- 3 Đường Conic}

\begin{baitoan}[\cite{Hai_Hung_Thu_Tung_ncpt_Toan_10_tap_2}, VD1, p. 68]
	Cho đường tròn $(O)$ \& điểm $A\ne O$ cố định nằm trong đường tròn. Vẽ đường tròn tâm $M$ tùy ý vừa qua $A$ vừa tiếp xúc với đường tròn $(O)$. Tìm quỹ tích điểm $M$.
\end{baitoan}

\begin{baitoan}[\cite{Hai_Hung_Thu_Tung_ncpt_Toan_10_tap_2}, VD2, p. 68]
	Cho 3 điểm $A,B,C$ cố định \& thẳng hàng theo thứ tự đó. Đường tròn $(O)$ tùy ý tiếp xúc với đường thẳng $ABC$ tại $A$. Từ $B,C$ kẻ 2 tiếp tuyến với đường tròn $(O)$, chúng cắt nhau tại $M\ne A$. Tìm quỹ tích điểm $M$.
\end{baitoan}

\begin{baitoan}[\cite{Hai_Hung_Thu_Tung_ncpt_Toan_10_tap_2}, VD3, p. 68]
	Cho $\Delta ABC$ có $BC$ cố định \& điểm $A$ chuyển động sao cho đường thẳng Euler của $\Delta ABC$ song song với $BC$. Tìm quỹ tích điểm $A$.
\end{baitoan}

\begin{baitoan}[\cite{Hai_Hung_Thu_Tung_ncpt_Toan_10_tap_2}, VD4, p. 69]
	(a) Cho hyperbol $(H):\dfrac{x^2}{9} - \dfrac{y^2}{4} = 1$. Tìm các yếu tố của $(H)$. Tìm các đường tiệm cận, góc giữa 2 đường tiệm cận. Tìm diện tích, chu vi hình chữ nhật cơ sở, hình thoi $A_1B_1A_2B_2$. Tìm bán kính $R_1$ của đường tròn ngoại tiếp $\Delta A_1B_2A_2$; bán kính $R_2$ của đường tròn ngoại tiếp $\Delta B_1A_2B_2$. (b) Viết phương trình chính tắc của $(H)$ với 2 điều kiện xác định $e$ \& $R_1$, $R_1$ \& $R_2$, $d_1$ \& $r$.
\end{baitoan}

\begin{baitoan}[\cite{Hai_Hung_Thu_Tung_ncpt_Toan_10_tap_2}, VD5, p. 70]
	Cho $\Delta ABC$ có $B,C$ cố định, $A$ chuyển động sao cho tâm đường tròn Euler của $\Delta ABC$ nằm trên $BC$. Tìm quỹ tích điểm $A$.
\end{baitoan}

\begin{baitoan}[\cite{Hai_Hung_Thu_Tung_ncpt_Toan_10_tap_2}, VD6, p. 71]
	Cho hyperbol $(H):\dfrac{x^2}{a^2} - \dfrac{y^2}{b^2} = 1$. 1 đường thẳng $\Delta$ tùy ý cắt 2 đường tiệm cận tại $E,F$, cắt $(H)$ tại $M,N$. Chứng minh $EF,MN$ có chung trung điểm.
\end{baitoan}

%------------------------------------------------------------------------------%

\section{Miscellaneous}

%------------------------------------------------------------------------------%

\printbibliography[heading=bibintoc]
	
\end{document}