\documentclass{article}
\usepackage[backend=biber,natbib=true,style=alphabetic,maxbibnames=50]{biblatex}
\addbibresource{/home/nqbh/reference/bib.bib}
\usepackage[utf8]{vietnam}
\usepackage{tocloft}
\renewcommand{\cftsecleader}{\cftdotfill{\cftdotsep}}
\usepackage[colorlinks=true,linkcolor=blue,urlcolor=red,citecolor=magenta]{hyperref}
\usepackage{amsmath,amssymb,amsthm,float,graphicx,mathtools,tikz}
\usetikzlibrary{angles,calc,intersections,matrix,patterns,quotes,shadings}
\allowdisplaybreaks
\newtheorem{assumption}{Assumption}
\newtheorem{baitoan}{}
\newtheorem{cauhoi}{Câu hỏi}
\newtheorem{conjecture}{Conjecture}
\newtheorem{corollary}{Corollary}
\newtheorem{dangtoan}{Dạng toán}
\newtheorem{definition}{Definition}
\newtheorem{dinhly}{Định lý}
\newtheorem{dinhnghia}{Định nghĩa}
\newtheorem{example}{Example}
\newtheorem{ghichu}{Ghi chú}
\newtheorem{hequa}{Hệ quả}
\newtheorem{hypothesis}{Hypothesis}
\newtheorem{lemma}{Lemma}
\newtheorem{luuy}{Lưu ý}
\newtheorem{nhanxet}{Nhận xét}
\newtheorem{notation}{Notation}
\newtheorem{note}{Note}
\newtheorem{principle}{Principle}
\newtheorem{problem}{Problem}
\newtheorem{proposition}{Proposition}
\newtheorem{question}{Question}
\newtheorem{remark}{Remark}
\newtheorem{theorem}{Theorem}
\newtheorem{vidu}{Ví dụ}
\usepackage[left=1cm,right=1cm,top=5mm,bottom=5mm,footskip=4mm]{geometry}
\def\labelitemii{$\circ$}
\DeclareRobustCommand{\divby}{%
	\mathrel{\vbox{\baselineskip.65ex\lineskiplimit0pt\hbox{.}\hbox{.}\hbox{.}}}%
}
\def\labelitemii{$\circ$}

\title{Problem: Combinatorics -- Bài Tập: Đại Số Tổ Hợp}
\author{Nguyễn Quản Bá Hồng\footnote{A Scientist {\it\&} Creative Artist Wannabe. E-mail: {\tt nguyenquanbahong@gmail.com}. Bến Tre City, Việt Nam.}}
\date{\today}

\begin{document}
\maketitle
\begin{abstract}
	This text is a part of the series {\it Some Topics in Elementary STEM \& Beyond}:
	
	{\sc url}: \url{https://nqbh.github.io/elementary_STEM}.
	
	Latest version:
	\begin{itemize}
		\item {\it Problem: Combinatorics -- Bài Tập: Đại Số Tổ Hợp}.
		
		PDF: {\sc url}: \url{https://github.com/NQBH/elementary_STEM_beyond/blob/main/elementary_mathematics/grade_10/combinatorics/problem/NQBH_combinatorics_problem.pdf}.
		
		\TeX: {\sc url}: \url{https://github.com/NQBH/elementary_STEM_beyond/blob/main/elementary_mathematics/grade_10/combinatorics/problem/NQBH_combinatorics_problem.tex}.
		\item {\it Problem \& Solution: Combinatorics -- Bài Tập \& Lời Giải: Đại Số Tổ Hợp}.
		
		PDF: {\sc url}: \url{https://github.com/NQBH/elementary_STEM_beyond/blob/main/elementary_mathematics/grade_10/combinatorics/solution/NQBH_combinatorics_solution.pdf}.
		
		\TeX: {\sc url}: \url{https://github.com/NQBH/elementary_STEM_beyond/blob/main/elementary_mathematics/grade_10/combinatorics/solution/NQBH_combinatorics_solution.tex}.
	\end{itemize}
\end{abstract}
\tableofcontents

%------------------------------------------------------------------------------%

\noindent\textbf{\textsf{Resources -- Tài nguyên.}}
\begin{enumerate}
	\item \cite{Andreescu_Feng_102_combinatorics}. {\sc Titu Andreescu, Zuming Feng}. {\it102 Combinatorial Problems}.
	\item \cite{Andreescu_Feng_path_combinatorics}. {\sc Titu Andreescu, Zuming Feng}. {\it A Path to Combinatorics for Undergraduates: Counting Strategies}.
	\item \cite{Hai_Hung_Thu_Tung_ncpt_Toan_10_tap_2}. {\sc Phan Việt Hải, Trần Quang Hùng, Ninh Văn Thu, Phạm Đình Tùng}. {\it Nâng Cao \& Phát Triển Toán 10. Tập 2}.
\end{enumerate}

\section{Quy Tắc Đếm}

\begin{baitoan}[\cite{Hai_Hung_Thu_Tung_ncpt_Toan_10_tap_2}, VD1, p. 28]
	Có 4 con đường nối 2 thành phố $X$ \& $Y$, có 5 con đường nối 2 thành phố $Y$ \& $Z$. Muốn đi từ $X$ đến $Z$ thì phải qua $Y$. Đếm số cách chọn: (a) đường đi từ $X$ đến $Z$. (b) đường đi từ $X$ đến $Z$ rồi về lại $X$ mà không đi qua đoạn đường đã đi.
\end{baitoan}

\begin{baitoan}[\cite{Hai_Hung_Thu_Tung_ncpt_Toan_10_tap_2}, VD2, p. 29]
	1 hội học sinh gồm $5$ học sinh lớp 10, $6$ học sinh lớp 11, $4$ học sinh lớp 12. Đếm số cách thành lập ban cán sự của hội gồm $4$ học sinh trong 2 trường hợp: (a) Lớp 10 có $1$ học sinh, lớp 12 có $1$ học sinh. (b) Mỗi khối lớp phải có ít nhất $1$ học sinh.
\end{baitoan}

\begin{baitoan}[\cite{Hai_Hung_Thu_Tung_ncpt_Toan_10_tap_2}, VD3, p. 29]
	Có 2 hộp chứa các quả bóng, hộp thứ 1 có $3$ quả bóng đỏ \& $2$ quả bóng xanh, hộp thứ 2 có $2$ quả bóng đỏ \& $3$ quả bóng xanh. Lấy ra $1$ quả bóng từ hộp thứ nhất \& lấy $1$ quả bóng từ hộp thứ 2. (a) Đếm số khả năng có thể xảy ra với 2 quả bóng lấy ra. (b) Đếm số cách để lấy ra được 2 quả bóng khác màu.
\end{baitoan}

\begin{baitoan}[\cite{Hai_Hung_Thu_Tung_ncpt_Toan_10_tap_2}, VD4, p. 30]
	Mỗi người sử dụng hệ thống máy tính đều có mật khẩu dài từ $6$--$8$ ký tự, trong đó mỗi ký tự là 1 chữ hoa chọn từ $26$ chữ cái tiếng Anh hay chữ số $0,1,2,\ldots,9$. Mỗi mật khẩu phải chứa ít nhất 1 chữ số. Đếm tổng số cách tạo ra mật khẩu.
\end{baitoan}

\begin{baitoan}[\cite{Hai_Hung_Thu_Tung_ncpt_Toan_10_tap_2}, VD5, p. 30]
	1 chủ tịch, 1 thủ quỹ, \& 1 thư ký sẽ được chọn từ $4$ người trong cơ quan là An, Bình, Chi, Dân. Biết An không thể là chủ tịch \& Chi hoặc Dân phải là thư ký. Đếm số cách chọn ra được $3$ vị trí trên bằng cách lập sơ đồ hình cây. Quy tắc nhân có thể thực hiện được ở sơ đồ hình cây này không?
\end{baitoan}

\begin{baitoan}[\cite{Hai_Hung_Thu_Tung_ncpt_Toan_10_tap_2}, 21.1., p. 30]
	Có 2 hộp, hộp thứ 1 có $3$ quả bóng đỏ \& $2$ quả bóng xanh, hộp 2 có $2$ quả bóng đỏ \& $3$ quả bóng xanh. Lấy lần lượt $2$ quả bóng từ hộp 1 \& tiếp tục lấy lần lượt $2$ quả bóng từ hộp 2 ra. (a) Đếm số khả năng có thể xảy ra với $4$ quả bóng lấy ra. (b) Đếm số cách để lấy được $4$ quả bóng có $2$ màu xanh \& $2$ màu đỏ.
\end{baitoan}

\begin{baitoan}[\cite{Hai_Hung_Thu_Tung_ncpt_Toan_10_tap_2}, 21.3., p. 31]
	Cho 1 khung dây có dạng hình hộp chữ nhật với kích thức dài {\rm5 cm}, rộng {\rm4 cm}, cao {\rm3 cm}. 1 con kiến bò dọc theo dây dẫn từ A đến B. Đếm số con đường khác nhau có chiều dài ngắn nhất dần từ A đến B.
\end{baitoan}

\begin{baitoan}[\cite{Hai_Hung_Thu_Tung_ncpt_Toan_10_tap_2}, 21.4., p. 31]
	1 ngôi nhà có $4$ tầng được thiết kế như sau: tầng 1 làm phòng khách \& bếp, tầng 2 có $2$ phòng ngủ, tầng 3 có $3$ phòng ngủ, tầng 4 có $2$ phòng ngủ \& sân chơi. Đếm số cách sắp xếp phòng ngủ cho 1 gia đình có $8$ người gồm ông, bà, bố, mẹ, $4$ người con sao cho: (a) Luôn có 1 phòng trống để cho khách ở tầng 3, bố \& mẹ ở 1 phòng, ông \& bà ở 1 phòng, mỗi người con 1 phòng. (b) Thỏa mãn điều kiện (a) \& ông bà ở tần 3 trở xuống.
\end{baitoan}

\begin{baitoan}[\cite{Hai_Hung_Thu_Tung_ncpt_Toan_10_tap_2}, 21.5., p. 31]
	Biển số xe ôtô ở Hà Nội là 1 dãy các ký tự lần lượt gồm 3 phần: phần 1 gồm 1 trong 3 số $29,30,31$, phần 2 gồm 1 chữ cái in hoa trong số $26$ chữ cái tiếng Anh, phần 3 là $5$ chữ số chọn ra từ $0,1,2,3,\ldots,9$, e.g., $29A12345$. Đếm số cách lập được biển số xe ôtô.
\end{baitoan}

\begin{baitoan}[\cite{Hai_Hung_Thu_Tung_ncpt_Toan_10_tap_2}, 21.8., p. 31]
	Trong 1 trận đá bóng giữa 2 đội A \& B, đội nào đầu tiên giành thắng lợi $3$ trận hoặc thắng liên tiếp $2$ trận sẽ là đội chiến thắng \& khi đó trận đấu kết thúc. Sử dụng sơ đồ hình cây để: (a) Đếm số phương án để trận đấu kết thúc. (b) Đếm số phương án để trận đấu kết thúc với giả sử đội A thắng trận đầu tiên.
\end{baitoan}

%------------------------------------------------------------------------------%

\section{Permutation, Arrangement, \& Combinations -- Hoán Vị, Chỉnh Hợp, \& Tổ Hợp}

\begin{baitoan}[\cite{Hai_Hung_Thu_Tung_ncpt_Toan_10_tap_2}, VD1, p. 32]
	Tính số kết quả: (a) Lập số có $3$ chữ số, các chữ số đôi một khác nhau, từ $5$ số $1,2,3,4,5$. (b) Lập số có $5$ chữ số, các chữ số đôi một khác nhau, từ $5$ số $1,2,3,4,5$. (c) Lập số có $3$ chữ số từ $5$ số $1,2,3,4,5$. (d) Lập tập hợp gồm $3$ phần tử từ $5$ số $1,2,3,4,5$.
\end{baitoan}

\begin{baitoan}[\cite{Hai_Hung_Thu_Tung_ncpt_Toan_10_tap_2}, VD2, p. 33]
	Cho tập $A = \{3,4,5,6,7\}$. (a) Từ $A$ có thể lập được bao nhiêu số gồm $5$ chữ số đôi một khác nhau? (b) Tính tổng $S$ của tất cả các số được tạo ra từ (a).
\end{baitoan}

\begin{baitoan}[\cite{Hai_Hung_Thu_Tung_ncpt_Toan_10_tap_2}, VD3, p. 33]
	Cho tập $A = \{0,1,2,3,4,5,6,7\}$. (a) Từ tập $A$ có thể lập được bao nhiêu số gồm $5$ chữ số khác nhau mà mỗi số luôn có mặt 2 chữ số $1,7$. (b) Trong các số vừa tìm được, có bao nhiêu số mà 2 chữ số $1,7$ đứng kề nhau, chữ số $1$ bên trái chữ số $7$?
\end{baitoan}

\begin{baitoan}[\cite{Hai_Hung_Thu_Tung_ncpt_Toan_10_tap_2}, VD4, p. 34]
	Cho lục giác lồi $ABCDEF$. (a) Đếm số tam giác có đỉnh là đỉnh của lục giác đã cho. Trong số các tam giác này, đếm số tam giác có cạnh không phải là cạnh của lục giác. (b) Tổng quát cho bát giác lồi. (c) Cho $n\in\mathbb{N}^\star$, $n\ge3$. Tổng quát cho $n$-giác lồi.
\end{baitoan}

\begin{baitoan}[\cite{Hai_Hung_Thu_Tung_ncpt_Toan_10_tap_2}, VD5, p. 35]
	Đếm số đường đi từ góc dưới bên trái đến góc trên bên phải của lưới $3\times3$ nếu chỉ có thể di chuyển sang phải hoặc lên trên. (b) Cho $n\in\mathbb{N}^\star$. Mở rộng ra cho lưới $n\times n$. (c) Cho $m,n\in\mathbb{N}^\star$. Mở rộng ra cho lưới $m\times n$.
\end{baitoan}

\begin{baitoan}[\cite{Hai_Hung_Thu_Tung_ncpt_Toan_10_tap_2}, 22.1., p. 35]
	Cho 6 chữ số $0,1,2,3,4,5$. (a) Đếm số số gồm $8$ chữ số trong đó chữ số $1$ có mặt đúng $3$ lần, chữ số $2$ có mặt đúng $2$ lần \& mỗi chữ số còn lại có mặt đúng $1$ lần. (b) Đếm số số gồm $5$ chữ số khác nhau trong đó 2 chữ số $1,2$ không đứng cạnh nhau. (c) Đếm số số gồm 5 chữ số khác nhau chia hết cho $5$.
\end{baitoan}

\begin{baitoan}[\cite{Hai_Hung_Thu_Tung_ncpt_Toan_10_tap_2}, 22.2., p. 35]
	Cho tập $A$ gồm các số tự nhiên từ $0$ đến $100$. $B\subset A$, $n(B) = 2$. (a) Đếm số tập $B$ thỏa tổng của 2 số trong $B$ là 1 số chẵn. (b) Đếm số tập $B$ thỏa tích của 2 số trong $B$ là 1 số chẵn.
\end{baitoan}

\begin{baitoan}[\cite{Hai_Hung_Thu_Tung_ncpt_Toan_10_tap_2}, 22.3., p. 35]
	$6$ người cùng đến rạp hát \& ngồi cùng hàng có đúng $6$ chỗ. (a) Đếm số cách họ xếp trong 1 hàng. (b) Giả sử $1$ trong $6$ người là bác sĩ \& người đó phải ngồi đầu lối đi vào đề phòng trường hợp có sự cố cần di chuyển nhanh. Đếm số cách xếp vị trí cho $6$ người. (c) Giả sử $6$ người gồm $3$ cặp vợ chồng đã kết hôn \& vợ luôn muốn ngồi bên trái của chồng. Đếm số cách xếp vị trí cho $6$ người.
\end{baitoan}

\begin{baitoan}[\cite{Hai_Hung_Thu_Tung_ncpt_Toan_10_tap_2}, 22.4., p. 36]
	Trên hình tròn có $12$ điểm phân biệt $A,B,C,D,E,F,G,H,I,J,K,L$. Vẽ đoạn thẳng nối 2 điểm. (a) Đếm số đoạn thẳng tạo được. (b) Đếm số đoạn thẳng tạo được đi qua $B$. (c) Đếm số tam giác tạo được từ các đoạn thẳng này. (d) Đếm số tam giác tạo được từ các đoạn thẳng này \& có 1 đỉnh là $B$.
\end{baitoan}

\begin{baitoan}[\cite{Hai_Hung_Thu_Tung_ncpt_Toan_10_tap_2}, 22.5., p. 36]
	Có $n\in\mathbb{N}^\star$ quả cầu trắng \& $n$ quả cầu đen, được đánh dấu theo các số $1,2,3,\ldots,n$. Đếm số cách sắp xếp các quả cầu này thành 1 dãy sao cho $2$ quả cầu cùng màu không nằm cạnh nhau.
\end{baitoan}

\begin{baitoan}[\cite{Hai_Hung_Thu_Tung_ncpt_Toan_10_tap_2}, 22.6., p. 36]
	1 hàm số $y = f(x)$ từ tập $A$ sang tập $B$ được gọi là {\rm toàn ánh} nếu mỗi phần tử $b\in B$ luôn tồn tại ít nhất 1 phần tử $a\in A$ sao cho $b = f(a)$. Số phần tử của $A,B$: $n(A) = n,n(B) = m$. (a) Đếm số cách lập 1 hàm số từ tập $A$ sang tập $B$. (b) Đếm số cách để lập được 1 hàm số toàn ánh từ tập $A$ sang tập $B$.
\end{baitoan}

\begin{baitoan}[\cite{Hai_Hung_Thu_Tung_ncpt_Toan_10_tap_2}, 22.7., p. 36]
	Cho 1 lưới $n\times n$. (a) Chứng minh tổng số đường đi từ góc dưới bên trái đến góc trên bên phải của lưới là $C{2n}^n$, nếu chỉ có thể di chuyển sang bên phải hoặc lên trên. (b) Chứng minh $C_{2n}^n = \sum_{i=0}^n (C_n^i)^2$.
\end{baitoan}

\begin{baitoan}[\cite{Hai_Hung_Thu_Tung_ncpt_Toan_10_tap_2}, 22.8., p. 36]
	Cho 1 lưới $m\times n$. Tính tổng số đường đi từ góc dưới bên trái đến góc trên bên phải của lưới nếu ta chỉ có thể di chuyển sang bên phải hoặc lên trên.
\end{baitoan}

%------------------------------------------------------------------------------%

\section{Newton Bionomial -- Nhị Thức Newton}

\begin{baitoan}[\cite{Hai_Hung_Thu_Tung_ncpt_Toan_10_tap_2}, VD1, p. 37]
	Khai triển \& đếm số các số hạng có trong khai triển: (a) $\left(x + \dfrac{1}{2x}\right)^8$. (b) $(1 - x + x^2)^4$.
\end{baitoan}

\begin{baitoan}[\cite{Hai_Hung_Thu_Tung_ncpt_Toan_10_tap_2}, VD2, p. 38]
	Tìm hệ số lớn nhất trong khai triển $(1 + 3x)^{21}$.
\end{baitoan}

\begin{baitoan}[\cite{Hai_Hung_Thu_Tung_ncpt_Toan_10_tap_2}, VD3, p. 38]
	Tính: (a) $A = 2^8\cdot3^8C_8^0 + 2^7\cdot3^7C_8^1 + \cdots + C_8^8$. (b) $B = 2^9\cdot5^9C_9^0 - 2^8\cdot5^8\cdot3C_9^1 + \cdots + 3^9C_9^9$.
\end{baitoan}

\begin{baitoan}[\cite{Hai_Hung_Thu_Tung_ncpt_Toan_10_tap_2}, VD4, p. 38]
	Chứng minh $a^n + b^n\ge\dfrac{1}{2^{n-1}}$, $\forall a,b\in\mathbb{R}$, $a + b = 1$, $\forall n\in\mathbb{N}$.
\end{baitoan}

\begin{baitoan}[\cite{Hai_Hung_Thu_Tung_ncpt_Toan_10_tap_2}, VD5, p. 39]
	Sử dụng công thức khai triển nhị thức Newton, chứng minh tổng các hệ số bình phương ở dòng thứ $n$ của tam giác Pascal bằng phần tử đứng giữa của dòng thứ $2n$.
\end{baitoan}

\begin{baitoan}[\cite{Hai_Hung_Thu_Tung_ncpt_Toan_10_tap_2}, 23.1., p. 39]
	Khai triển: (a) $(x^2 - \sqrt{1 - x^2})^4 + (x^2 + \sqrt{1 - x^2})^4$. (b) $(1 - 2x + 3x^2)^4$.
\end{baitoan}

\begin{baitoan}[\cite{Hai_Hung_Thu_Tung_ncpt_Toan_10_tap_2}, 23.2., p. 39]
	Tìm hệ số của số hạng không chứa $x$ trong khai triển: (a) $\left(\dfrac{\sqrt{x}}{\sqrt{3}} + \dfrac{\sqrt{3}}{2x^2}\right)^{10}$. (b) $\left(x^2 + \dfrac{2}{x}\right)^{15}$.
\end{baitoan}

\begin{baitoan}[\cite{Hai_Hung_Thu_Tung_ncpt_Toan_10_tap_2}, 23.3., p. 39]
	Tìm hệ số lớn nhất trong khai triển: (a) $(2 + 3x)^9$. (b) $(2x + 3y)^{15}$.
\end{baitoan}

\begin{baitoan}[\cite{Hai_Hung_Thu_Tung_ncpt_Toan_10_tap_2}, 23.4., p. 39]
	(a) Cho $n\in\mathbb{N}^\star$. Tìm hệ số của $x^{-1}$ trong khai triển $(1 + x)^n\left(1 + \dfrac{1}{x}\right)^n$. (b) Tìm hệ số của $x^{50}$ trong khai triển $(1 + x)^{1000} + x(1 + x)^{999} + x^2(1 + x)^{998} + \cdots + x^{1000}$.
\end{baitoan}

\begin{baitoan}[\cite{Hai_Hung_Thu_Tung_ncpt_Toan_10_tap_2}, 23.5., p. 39]
	Chứng minh trong mỗi dòng của tam giác Pascal, tổng của các số hạng ở vị trí lẻ bằng tổng của các số hạng ở vị trí chẵn.
\end{baitoan}

\begin{baitoan}[\cite{Hai_Hung_Thu_Tung_ncpt_Toan_10_tap_2}, 23.6., p. 39]
	Cho $(1 - x + x^2)^n = a_0 + a_1x + a_2x^2 + \cdots + a_{2n}x^{2n}$. (a) Tính tổng $a_0 + a_2 + \cdots + a_{2n}$. (b) Tính tổng $a_1 + a_3 + \cdots + a_{2n - 1}$.
\end{baitoan}

\begin{baitoan}[\cite{Hai_Hung_Thu_Tung_ncpt_Toan_10_tap_2}, 23.7., p. 39]
	Chứng minh nếu $a_1,a_2,a_3,a_4$ là 4 hệ số của 4 số hạng liên tiếp trong khai triển $(1 + x)^n$ thì $\dfrac{a_1}{a_1 + a_2} + \dfrac{a_2}{a_1 + a_2} = \dfrac{2a_2}{a_2 + a_3}$.
\end{baitoan}

\begin{baitoan}[\cite{Hai_Hung_Thu_Tung_ncpt_Toan_10_tap_2}, 23.8., p. 39]
	Chứng minh trong khai triển $(a + b + c)^n$ với $n\in\mathbb{N}$ gồm $C_{n + 2}^2$ số hạng.
\end{baitoan}

%------------------------------------------------------------------------------%

\section{Algebraic Combinatorics -- Đại Số Tổ Hợp}
See \href{https://en.wikipedia.org/wiki/Algebraic_combinatorics}{Wikipedia{\tt/}algebraic combinatorics}.

\begin{baitoan}[\cite{Hai_Hung_Thu_Tung_ncpt_Toan_10_tap_2}, VD1, p. 40]
	Đếm số tự nhiên không lớn hơn $1000$ chia hết cho $7$ hoặc $11$.
\end{baitoan}

\begin{baitoan}[\cite{Hai_Hung_Thu_Tung_ncpt_Toan_10_tap_2}, VD2, p. 41]
	Đếm số cách để sắp xếp các ký tự trong xâu MISSISSIPPI để tạo thành các xâu khác nhau.
\end{baitoan}

\begin{baitoan}[\cite{Hai_Hung_Thu_Tung_ncpt_Toan_10_tap_2}, VD3, p. 41]
	Sử dụng khai triển nhị thức Newton để so sánh $99^{50} + 100^{50}$ với $101^{50}$.
\end{baitoan}

\begin{baitoan}[\cite{Hai_Hung_Thu_Tung_ncpt_Toan_10_tap_2}, VD4, p. 41]
	Dãy số các số nguyên $0,1,1,2,3,5,8,13,21,34,\ldots$ trong đó mỗi số trong dãy bằng tổng 2 số đứng liền trước số đó được gọi là {\rm dãy Fibonacci}. Cho dãy số $F(n)$ được xác định bởi $F(n) = C_n^0 + C_{n-1}^1 + C_{n-2}^2 + \cdots + C_{n-k}^k$ với $k\coloneqq\left\lfloor\dfrac{n}{2}\right\rfloor$. Chứng minh đây là dãy Fibonacci nếu thêm số $0$ vào đầu dãy.
\end{baitoan}

\begin{baitoan}[\cite{Hai_Hung_Thu_Tung_ncpt_Toan_10_tap_2}, VD5, p. 42]
	Chứng minh $2^{4n + 4} - 15n - 16\divby225$, $\forall n\in\mathbb{N}$.
\end{baitoan}

\begin{baitoan}[\cite{Hai_Hung_Thu_Tung_ncpt_Toan_10_tap_2}, 24.1., p. 42]
	Đếm số số tự nhiên nhỏ hơn $1000$ \& không có ước chung $\ne1$ với $1000$.
\end{baitoan}

\begin{baitoan}[\cite{Hai_Hung_Thu_Tung_ncpt_Toan_10_tap_2}, 24.2., p. 42]
	Trong cuộc thi học sinh giỏi môn Toán, Vật Lý, Hóa Học của trường có $50$ em tham gia. Mỗi học sinh đều có thể tham gia thi tối đa 3 môn. Kết quả thi xác định: $21$ em đạt giải môn Toán, $21$ em đạt giải môn Vật Lý, $25$ em đạt giải môn Hóa Học, $9$ em đạt giải cả 2 môn Toán \& Vật Lý, $10$ em đạt giải cả 2 môn Toán \& Hóa Học, $11$ em đạt giải cả 2 môn Vật Lý \& Hóa Học, $5$ em đạt giải cả 3 môn. Đếm số học sinh đạt giải ít nhất 1 môn.
\end{baitoan}

\begin{baitoan}[\cite{Hai_Hung_Thu_Tung_ncpt_Toan_10_tap_2}, 24.3., p. 43]
	2 loại thuốc $A,B$ được thử nghiệm trong phòng thí nghiệm. $60$ con chuột bách được sử dụng như sau: $20$ con chuột được tiêm thuốc $A$, $20$ con chuột được tiêm thuốc $B$, $20$ con chuột còn lại thuộc nhóm đối chứng (không tiêm thuốc). Mỗi 1 phương án lập ra để tiêm cho chuột sẽ xác định rõ mỗi con chuột sẽ tiêm thuốc A hay tiêm thuốc B hay không tiêm. Đếm số cách lập phương án tiêm cho chuột.
\end{baitoan}

\begin{baitoan}[\cite{Hai_Hung_Thu_Tung_ncpt_Toan_10_tap_2}, 24.4., p. 43]
	(a) Đếm số cách để tạo ra các xâu ký tự khác nhau từ các ký tự trong xâu {\rm HULLABALOO}. (b) Đếm số cách để tạo ra các xâu ký tự khác nhau bắt đầu bằng ký tự H \& kết thúc bằng ký tự U từ các ký tự trong xâu {\rm HULLABALOO}.
\end{baitoan}

\begin{baitoan}[\cite{Hai_Hung_Thu_Tung_ncpt_Toan_10_tap_2}, 24.5., p. 43]
	An có 3 chuồng để nhốt thỏ, mỗi chuồng nhốt tối đa được $3$ con. (a) An mua về $9$ con thỏ. Đếm số cách để An nhốt hết thỏ vào chuồng. (b) An mua về $8$ con thỏ. Đếm số cách để An nhốt hết thỏ vào chuồng. (c) An mua về $6$ con thỏ. Đếm số cách để An nhốt hết thỏ vào chuồng sao cho không có chuồng nào bị trống.
\end{baitoan}

\begin{baitoan}[\cite{Hai_Hung_Thu_Tung_ncpt_Toan_10_tap_2}, 24.6., p. 43]
	Chứng minh 2 chữ số cuối của $3^{400}$ là $01$.
\end{baitoan}

\begin{baitoan}[\cite{Hai_Hung_Thu_Tung_ncpt_Toan_10_tap_2}, 24.7., p. 43]
	Chứng minh $7^9 + 9^7\divby64$.
\end{baitoan}

\begin{baitoan}[\cite{Hai_Hung_Thu_Tung_ncpt_Toan_10_tap_2}, 24.8., p. 43]
	Tìm phần dư của phép chia $25^{15}$ cho $13$.
\end{baitoan}

%------------------------------------------------------------------------------%

\section{Miscellaneous}

\begin{baitoan}[\cite{TLCT_BT_dai_so_giai_tich_11}, 1., p. 60]
	Ký hiệu $S$ là tập hợp tất cả các số nguyên dương $s$ có tính chất: (i) Các chữ số của $s$ khác nhau. (ii) Các chữ số của $s$ thuộc tập hợp $\{1,3,5,7\}$. Tính: (a) Số phần tử của $S$. (b) Tổng tất cả các số có 3 chữ số của $S$. (c) Tổng tất cả các số của $S$.
\end{baitoan}

\begin{baitoan}[\cite{TLCT_BT_dai_so_giai_tich_11}, 2., p. 60]
	(a) Đếm số cách xếp $6$ cặp vợ chồng ngồi xung quanh 1 chiếc bàn tròn sao cho nam \& nữ ngồi xen kẽ nhau. (b) Đếm số cách xếp $6$ cặp vợ chồng ngồi xung quanh 1 chiếc bàn tròn sao cho mỗi bà đều ngồi cạnh chồng của mình.
\end{baitoan}

\begin{baitoan}[\cite{TLCT_BT_dai_so_giai_tich_11}, 3., p. 60]
	1 nhóm $6$ người vào 1 quán ăn. Đếm số cách xếp $6$ người này ngồi vào 3 bàn tròn giống hệt nhau trong đó yêu cầu mỗi bàn phải có ít nhất 1 người.
\end{baitoan}

\begin{baitoan}[\cite{TLCT_BT_dai_so_giai_tich_11}, 4., p. 61]
	1 dãy tam phân độ dài $n$ là 1 dãy gồm $n$ chữ số mà mỗi chữ số chỉ nhận 1 trong 3 giá trị $\{0,1,2\}$. Tìm số các dãy tam phân độ dài $10$ mà trong đó có 2 chữ số $0$, 3 chữ số $1$, \& 5 chữ số $2$.
\end{baitoan}

\begin{baitoan}[\cite{TLCT_BT_dai_so_giai_tich_11}, 5., p. 61]
	1 đoàn khách du lịch gồm $n$ người được xếp vào $r$ khách sạn $S_1,S_2,\ldots,S_r$. Yêu cầu: đưa $n_i$ khách ở tại khách sạn $S_i$, $i = 1,2,\ldots,r$, trong đó $n_1,n_2,\ldots,n_r\in\mathbb{N}$ thỏa $\sum_{i=1}^r = n$. Chứng minh số cách phân phối khách thỏa mãn yêu cầu là $P = \dfrac{n!}{n_1!n_2!\cdots n_r!}$. Nhờ công thức này, suy ra $(4m)!\divby2^{3m}\cdot3^m$, $\forall m\in\mathbb{N}^\star$.
\end{baitoan}

\begin{baitoan}[\cite{TLCT_BT_dai_so_giai_tich_11}, 6., p. 61]
	Cho $n\in\mathbb{N},n > 3$. (a) Tìm số các bộ 3 $(a,b,c)$ với $a,b,c\in\mathbb{N}$ thỏa $a + b + c = n$. (b) Tìm số các bộ 3 $(a,b,c)$ với $a,b,c\in\mathbb{N}^\star$ thỏa $a + b + c = n$.
\end{baitoan}

\begin{baitoan}[\cite{TLCT_BT_dai_so_giai_tich_11}, 7., p. 61]
	Tìm số các bộ 3 $(a,b,c)$ trong đó $a,b,c\in\mathbb{N}$ thỏa $a\le5,b\le6,c\le7,a + b + c = 15$.
\end{baitoan}

\begin{baitoan}[\cite{TLCT_BT_dai_so_giai_tich_11}, 8., p. 61]
	Tìm số các bộ 3 $(a,b,c)$ với $a,b,c\in\mathbb{N}$ thỏa đồng thời: (i) Tổng của chúng bằng $15$. (ii) Trong 3 số $a,b,c$ có ít nhất 1 số $\ge7$.
\end{baitoan}

\begin{baitoan}[\cite{TLCT_BT_dai_so_giai_tich_11}, 9., p. 61]
	Đếm số số tự nhiên có $9$ chữ số trong đó có $3$ chữ số lẻ khác nhau \& $3$ chữ số chẵn khác nhau, mỗi chữ số chẵn có mặt đúng 2 lần.
\end{baitoan}

\begin{baitoan}[\cite{TLCT_BT_dai_so_giai_tich_11}, 10., p. 61]
	Cho $n\in\mathbb{N},n > 3$. Chứng minh $\sum_{k=1}^n k^3C_n^k = n^2(n + 3)2^{n - 3}$.
\end{baitoan}

\begin{baitoan}[Bài toán chia kẹo Euler]
	See, e.g., \href{https://viblo.asia/p/bai-toan-chia-keo-euler-L4x5xqvqKBM}{Viblo{\tt/}bài toán chia kẹo Euler}.
\end{baitoan}

%------------------------------------------------------------------------------%

\printbibliography[heading=bibintoc]
	
\end{document}