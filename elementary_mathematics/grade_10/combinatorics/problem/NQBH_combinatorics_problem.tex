\documentclass{article}
\usepackage[backend=biber,natbib=true,style=alphabetic,maxbibnames=50]{biblatex}
\addbibresource{/home/nqbh/reference/bib.bib}
\usepackage[utf8]{vietnam}
\usepackage{tocloft}
\renewcommand{\cftsecleader}{\cftdotfill{\cftdotsep}}
\usepackage[colorlinks=true,linkcolor=blue,urlcolor=red,citecolor=magenta]{hyperref}
\usepackage{amsmath,amssymb,amsthm,float,graphicx,mathtools,tikz}
\usetikzlibrary{angles,calc,intersections,matrix,patterns,quotes,shadings}
\allowdisplaybreaks
\newtheorem{assumption}{Assumption}
\newtheorem{baitoan}{}
\newtheorem{cauhoi}{Câu hỏi}
\newtheorem{conjecture}{Conjecture}
\newtheorem{corollary}{Corollary}
\newtheorem{dangtoan}{Dạng toán}
\newtheorem{definition}{Definition}
\newtheorem{dinhly}{Định lý}
\newtheorem{dinhnghia}{Định nghĩa}
\newtheorem{example}{Example}
\newtheorem{ghichu}{Ghi chú}
\newtheorem{hequa}{Hệ quả}
\newtheorem{hypothesis}{Hypothesis}
\newtheorem{lemma}{Lemma}
\newtheorem{luuy}{Lưu ý}
\newtheorem{nhanxet}{Nhận xét}
\newtheorem{notation}{Notation}
\newtheorem{note}{Note}
\newtheorem{principle}{Principle}
\newtheorem{problem}{Problem}
\newtheorem{proposition}{Proposition}
\newtheorem{question}{Question}
\newtheorem{remark}{Remark}
\newtheorem{theorem}{Theorem}
\newtheorem{vidu}{Ví dụ}
\usepackage[left=1cm,right=1cm,top=5mm,bottom=5mm,footskip=4mm]{geometry}
\def\labelitemii{$\circ$}
\DeclareRobustCommand{\divby}{%
	\mathrel{\vbox{\baselineskip.65ex\lineskiplimit0pt\hbox{.}\hbox{.}\hbox{.}}}%
}
\def\labelitemii{$\circ$}

\title{Problem: Combinatorics -- Bài Tập: Đại Số Tổ Hợp}
\author{Nguyễn Quản Bá Hồng\footnote{A Scientist {\it\&} Creative Artist Wannabe. E-mail: {\tt nguyenquanbahong@gmail.com}. Bến Tre City, Việt Nam.}}
\date{\today}

\begin{document}
\maketitle
\begin{abstract}
	This text is a part of the series {\it Some Topics in Elementary STEM \& Beyond}:
	
	{\sc url}: \url{https://nqbh.github.io/elementary_STEM}.
	
	Latest version:
	\begin{itemize}
		\item {\it Problem: Combinatorics -- Bài Tập: Đại Số Tổ Hợp}.
		
		PDF: {\sc url}: \url{https://github.com/NQBH/elementary_STEM_beyond/blob/main/elementary_mathematics/grade_10/combinatorics/problem/NQBH_combinatorics_problem.pdf}.
		
		\TeX: {\sc url}: \url{https://github.com/NQBH/elementary_STEM_beyond/blob/main/elementary_mathematics/grade_10/combinatorics/problem/NQBH_combinatorics_problem.tex}.
		\item {\it Problem \& Solution: Combinatorics -- Bài Tập \& Lời Giải: Đại Số Tổ Hợp}.
		
		PDF: {\sc url}: \url{https://github.com/NQBH/elementary_STEM_beyond/blob/main/elementary_mathematics/grade_10/combinatorics/solution/NQBH_combinatorics_solution.pdf}.
		
		\TeX: {\sc url}: \url{https://github.com/NQBH/elementary_STEM_beyond/blob/main/elementary_mathematics/grade_10/combinatorics/solution/NQBH_combinatorics_solution.tex}.
	\end{itemize}
\end{abstract}
\tableofcontents

%------------------------------------------------------------------------------%

\noindent\textbf{\textsf{Resources -- Tài nguyên.}}
\begin{enumerate}
	\item \cite{Andreescu_Feng_102_combinatorics}. {\sc Titu Andreescu, Zuming Feng}. {\it102 Combinatorial Problems}.
	\item \cite{Andreescu_Feng_path_combinatorics}. {\sc Titu Andreescu, Zuming Feng}. {\it A Path to Combinatorics for Undergraduates: Counting Strategies}.
	\item \cite{Hai_Hung_Thu_Tung_ncpt_Toan_10_tap_2}. {\sc Phan Việt Hải, Trần Quang Hùng, Ninh Văn Thu, Phạm Đình Tùng}. {\it Nâng Cao \& Phát Triển Toán 10. Tập 2}.
\end{enumerate}

\section{Quy Tắc Đếm}

\begin{baitoan}[\cite{Hai_Hung_Thu_Tung_ncpt_Toan_10_tap_2}, VD1, p. 28]
	Có 4 con đường nối 2 thành phố $X$ \& $Y$, có 5 con đường nối 2 thành phố $Y$ \& $Z$. Muốn đi từ $X$ đến $Z$ thì phải qua $Y$. Đếm số cách chọn: (a) đường đi từ $X$ đến $Z$. (b) đường đi từ $X$ đến $Z$ rồi về lại $X$ mà không đi qua đoạn đường đã đi.
\end{baitoan}

\begin{baitoan}[\cite{Hai_Hung_Thu_Tung_ncpt_Toan_10_tap_2}, VD2, p. 29]
	1 hội học sinh gồm $5$ học sinh lớp 10, $6$ học sinh lớp 11, $4$ học sinh lớp 12. Đếm số cách thành lập ban cán sự của hội gồm $4$ học sinh trong 2 trường hợp: (a) Lớp 10 có $1$ học sinh, lớp 12 có $1$ học sinh. (b) Mỗi khối lớp phải có ít nhất $1$ học sinh.
\end{baitoan}

\begin{baitoan}[\cite{Hai_Hung_Thu_Tung_ncpt_Toan_10_tap_2}, VD3, p. 29]
	Có 2 hộp chứa các quả bóng, hộp thứ 1 có $3$ quả bóng đỏ \& $2$ quả bóng xanh, hộp thứ 2 có $2$ quả bóng đỏ \& $3$ quả bóng xanh. Lấy ra $1$ quả bóng từ hộp thứ nhất \& lấy $1$ quả bóng từ hộp thứ 2. (a) Đếm số khả năng có thể xảy ra với 2 quả bóng lấy ra. (b) Đếm số cách để lấy ra được 2 quả bóng khác màu.
\end{baitoan}

\begin{baitoan}[\cite{Hai_Hung_Thu_Tung_ncpt_Toan_10_tap_2}, VD4, p. 30]
	Mỗi người sử dụng hệ thống máy tính đều có mật khẩu dài từ $6$--$8$ ký tự, trong đó mỗi ký tự là 1 chữ hoa chọn từ $26$ chữ cái tiếng Anh hay chữ số $0,1,2,\ldots,9$. Mỗi mật khẩu phải chứa ít nhất 1 chữ số. Đếm tổng số cách tạo ra mật khẩu.
\end{baitoan}

\begin{baitoan}[\cite{Hai_Hung_Thu_Tung_ncpt_Toan_10_tap_2}, VD5, p. 30]
	1 chủ tịch, 1 thủ quỹ, \& 1 thư ký sẽ được chọn từ $4$ người trong cơ quan là An, Bình, Chi, Dân. Biết An không thể là chủ tịch \& Chi hoặc Dân phải là thư ký. Đếm số cách chọn ra được $3$ vị trí trên bằng cách lập sơ đồ hình cây. Quy tắc nhân có thể thực hiện được ở sơ đồ hình cây này không?
\end{baitoan}

\begin{baitoan}[\cite{Hai_Hung_Thu_Tung_ncpt_Toan_10_tap_2}, 21.1., p. 30]
	Có 2 hộp, hộp thứ 1 có $3$ quả bóng đỏ \& $2$ quả bóng xanh, hộp 2 có $2$ quả bóng đỏ \& $3$ quả bóng xanh. Lấy lần lượt $2$ quả bóng từ hộp 1 \& tiếp tục lấy lần lượt $2$ quả bóng từ hộp 2 ra. (a) Đếm số khả năng có thể xảy ra với $4$ quả bóng lấy ra. (b) Đếm số cách để lấy được $4$ quả bóng có $2$ màu xanh \& $2$ màu đỏ.
\end{baitoan}

\begin{baitoan}[\cite{Hai_Hung_Thu_Tung_ncpt_Toan_10_tap_2}, 21.3., p. 31]
	Cho 1 khung dây có dạng hình hộp chữ nhật với kích thức dài {\rm5 cm}, rộng {\rm4 cm}, cao {\rm3 cm}. 1 con kiến bò dọc theo dây dẫn từ A đến B. Đếm số con đường khác nhau có chiều dài ngắn nhất dần từ A đến B.
\end{baitoan}

\begin{baitoan}[\cite{Hai_Hung_Thu_Tung_ncpt_Toan_10_tap_2}, 21.4., p. 31]
	1 ngôi nhà có $4$ tầng được thiết kế như sau: tầng 1 làm phòng khách \& bếp, tầng 2 có $2$ phòng ngủ, tầng 3 có $3$ phòng ngủ, tầng 4 có $2$ phòng ngủ \& sân chơi. Đếm số cách sắp xếp phòng ngủ cho 1 gia đình có $8$ người gồm ông, bà, bố, mẹ, $4$ người con sao cho: (a) Luôn có 1 phòng trống để cho khách ở tầng 3, bố \& mẹ ở 1 phòng, ông \& bà ở 1 phòng, mỗi người con 1 phòng. (b) Thỏa mãn điều kiện (a) \& ông bà ở tần 3 trở xuống.
\end{baitoan}

\begin{baitoan}[\cite{Hai_Hung_Thu_Tung_ncpt_Toan_10_tap_2}, 21.5., p. 31]
	Biển số xe ôtô ở Hà Nội là 1 dãy các ký tự lần lượt gồm 3 phần: phần 1 gồm 1 trong 3 số $29,30,31$, phần 2 gồm 1 chữ cái in hoa trong số $26$ chữ cái tiếng Anh, phần 3 là $5$ chữ số chọn ra từ $0,1,2,3,\ldots,9$, e.g., $29A12345$. Đếm số cách lập được biển số xe ôtô.
\end{baitoan}

\begin{baitoan}[\cite{Hai_Hung_Thu_Tung_ncpt_Toan_10_tap_2}, 21.8., p. 31]
	Trong 1 trận đá bóng giữa 2 đội A \& B, đội nào đầu tiên giành thắng lợi $3$ trận hoặc thắng liên tiếp $2$ trận sẽ là đội chiến thắng \& khi đó trận đấu kết thúc. Sử dụng sơ đồ hình cây để: (a) Đếm số phương án để trận đấu kết thúc. (b) Đếm số phương án để trận đấu kết thúc với giả sử đội A thắng trận đầu tiên.
\end{baitoan}

%------------------------------------------------------------------------------%

\section{Permutation, Arrangement, \& Combinations -- Hoán Vị, Chỉnh Hợp, \& Tổ Hợp}

%------------------------------------------------------------------------------%

\section{Newton Bionomial -- Nhị Thức Newton}

%------------------------------------------------------------------------------%

\section{Algebraic Combinatorics -- Đại Số Tổ Hợp}

%------------------------------------------------------------------------------%

\section{Miscellaneous}

\begin{baitoan}[Bài toán chia kẹo Euler]
	
\end{baitoan}

%------------------------------------------------------------------------------%

\printbibliography[heading=bibintoc]
	
\end{document}