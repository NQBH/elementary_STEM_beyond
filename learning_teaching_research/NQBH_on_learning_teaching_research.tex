\documentclass[12pt]{article}
\usepackage[backend=biber,natbib=true,style=alphabetic,maxbibnames=10]{biblatex}
\usepackage[utf8]{vietnam}
\addbibresource{/home/nqbh/reference/bib.bib}
\usepackage{tocloft}
\renewcommand{\cftsecleader}{\cftdotfill{\cftdotsep}}
\usepackage[colorlinks=true,linkcolor=blue,urlcolor=red,citecolor=magenta]{hyperref}
\usepackage{amsmath,amssymb,amsthm,float,graphicx,mathtools}
\allowdisplaybreaks
\newtheorem{assumption}{Assumption}
\newtheorem{conjecture}{Conjecture}
\newtheorem{corollary}{Corollary}
\newtheorem{definition}{Definition}
\newtheorem{example}{Example}
\newtheorem{hypothesis}{Hypothesis}
\newtheorem{lemma}{Lemma}
\newtheorem{notation}{Notation}
\newtheorem{note}{Note}
\newtheorem{principle}{Principle}
\newtheorem{problem}{Problem}
\newtheorem{proposition}{Proposition}
\newtheorem{question}{Question}
\newtheorem{remark}{Remark}
\newtheorem{Rule}{Rule}
\newtheorem{theorem}{Theorem}
\usepackage[margin=2cm]{geometry}

\title{Some Thoughts On Learning, Teaching, {\it\&} Research\\Vài Suy Nghĩ Về Học, Dạy, {\it\&} Nghiên Cứu}
\author{Nguyễn Quản Bá Hồng\footnote{E-mail: {\tt nguyenquanbahong@gmail.com}. Bến Tre City, Việt Nam.}}
\date{\today}

\begin{document}
\maketitle
\begin{abstract}
	This note consists of some pieces of my writing which are able to share and I am willing to share, in order to sharpen my flow of thoughts, to balance my scientific work via various aesthetic forms, \& to track psychologically and mentally my transitions from boyhood to manhood if there is any.
\end{abstract}
\setcounter{secnumdepth}{4}
\setcounter{tocdepth}{3}
\tableofcontents

%------------------------------------------------------------------------------%

\section{An Initial Configuration}

\subsection{Rules}
Mỗi người $P$ (abbr., person) có một xuất phát điểm $\{P(t)\}_{0\le t\le t_0}$ khác nhau, được{\tt/}bị hưởng các nền tảng giáo dục khác nhau, sự tương tác với những người, những chuyện họ gặp trong suốt 1 cuộc đời hoàn toàn khác nhau, thành ra nền tảng nhận thức \& xu hướng phát triển nhận thức, cùng sự hình thành các cấu trúc niềm tin \& các hệ giá trị cơ bản cùng thế giới quan của mỗi người hoàn toàn khác nhau.

Quy tắc đầu tiên ở đây là không phán xét, công kích (e.g., dí trên mạng xã hội) bất cứ ai. Cũng không áp đặt ai, thậm chí cả việc áp đặt ai đó không được áp đặt người khác. Tạo cho người khác 1 cảm giác thoải mái tối thiểu khi tiếp xúc.

Quy tắc thứ 2 là không quá tò mò vào cuộc sống cá nhân của người khác, e.g., stalk in social media -- rình mò trên các nền tảng mạng xã hội, xâm phạm tài khoản  riêng tư cá nhân bất hợp pháp. Keep healthy boundaries for both.

\begin{Rule}[Reset]
	Một phản tư xa hơn trong tương lai có lẽ là chẳng có hành trình phát triển tự thân hiệu quả nào mà đủ sức chống chọi 1 cách hiệu quả với các tương tác xã hội cả, đặc biệt là các tương tác xấu \& các mối quan hệ độc hại (toxic relationships) cả. Khi đó thì tất cả các ghi chú ở đây sẽ bị xóa. Mọi thứ trở về cấu hình sống nhiều mặt để che giấu bản thân.
\end{Rule} 

\subsection{Goals}
This writing is one of many ways, which likely become the main one, to balance between my scientific work \& personal life. I believe some art will be the tool.

Việc viết lách, theo mình nghĩ, bằng cách này hay cách khác, một lúc nào đó \& theo 1 cách tự nhiên nào đó, cũng sẽ tìm tới những kẻ thích suy nghĩ, những kẻ hay nghĩ nhiều, \& những kẻ mệt mỏi vì cái tật đó, e.g., nhà nghiên cứu, nghiên cứu sinh, các học giả, nói chung là những người làm trong mảng học thuật hoặc phải tiếp xúc nhiều với chữ. Tật hay tài thì chưa biết nhưng ắt hẳn việc viết dùng để sắp xếp mọi thứ trong đầu cho ngăn nắp thì không thể tránh khỏi đối với những người làm việc đầu óc nhiều.

%------------------------------------------------------------------------------%

\section{On Learning -- Bàn Về Việc Học}

%------------------------------------------------------------------------------%

\section{On Teaching -- Bàn Về Việc Dạy}

%------------------------------------------------------------------------------%

\section{On Research -- Bàn Về Nghiên Cứu}
It is kind of funny, ironic, and sarcastic that the author of this writing is a dropout PhD student from one of the best research institutes of applied mathematics in Germany.

Khi phải đối đầu với những thứ thật sự khó nhằn, hoàn toàn nằm ngoài hiểu biết hiện tại của 1 cá nhân, thì 1 cách khá đơn giản là bám víu vào những thứ đã biết rõ, dù có thể lặp đi lặp lại 1 cách đơn điệu \& nhàm chán, nhưng lại có trật tự để cân bằng với hỗn loạn -- tượng trưng cho những điều chưa biết \cite{Peterson2018,Peterson2022a,Peterson2022b}.

\begin{example}
	Dạy học bậc phổ thông trở xuống thì ``nhàn'', theo nghĩa là không cần phải nạp quá nhiều kiến thức mới, nhưng phải chú trọng về phương pháp dạy \& truyền đạt kiến thức 1 cách hiệu quả tới các học sinh. Nếu học sinh giỏi, tiếp thu nhanh thì khỏe. Gặp học sinh dốt hoặc đầu gấu thì mệt, đâm ra chán chường, cảm thấy phí phạm thời gian \& nguồn sức lực hạn chế của bản thân.
	
	Nghiên cứu thì lại khác. Trách nhiệm của nghiên cứu là phải đọc thật nhiều, nạp thật nhiều kiến thức để trau dồi bản thân mỗi ngày.
\end{example}

%------------------------------------------------------------------------------%

\section{Miscellaneous}

%------------------------------------------------------------------------------%

\appendix

%------------------------------------------------------------------------------%

\printbibliography[heading=bibintoc]
	
\end{document}