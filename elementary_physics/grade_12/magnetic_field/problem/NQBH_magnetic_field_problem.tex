\documentclass{article}
\usepackage[backend=biber,natbib=true,style=alphabetic,maxbibnames=50]{biblatex}
\addbibresource{/home/nqbh/reference/bib.bib}
\usepackage[utf8]{vietnam}
\usepackage{tocloft}
\renewcommand{\cftsecleader}{\cftdotfill{\cftdotsep}}
\usepackage[colorlinks=true,linkcolor=blue,urlcolor=red,citecolor=magenta]{hyperref}
\usepackage{amsmath,amssymb,amsthm,enumitem,float,graphicx,mathtools,tikz}
\usetikzlibrary{angles,calc,intersections,matrix,patterns,quotes,shadings}
\allowdisplaybreaks
\newtheorem{assumption}{Assumption}
\newtheorem{baitoan}{}
\newtheorem{cauhoi}{Câu hỏi}
\newtheorem{conjecture}{Conjecture}
\newtheorem{corollary}{Corollary}
\newtheorem{dangtoan}{Dạng toán}
\newtheorem{definition}{Definition}
\newtheorem{dinhly}{Định lý}
\newtheorem{dinhnghia}{Định nghĩa}
\newtheorem{example}{Example}
\newtheorem{ghichu}{Ghi chú}
\newtheorem{goal}{Goal}
\newtheorem{hequa}{Hệ quả}
\newtheorem{hypothesis}{Hypothesis}
\newtheorem{lemma}{Lemma}
\newtheorem{luuy}{Lưu ý}
\newtheorem{nhanxet}{Nhận xét}
\newtheorem{notation}{Notation}
\newtheorem{note}{Note}
\newtheorem{principle}{Principle}
\newtheorem{problem}{Problem}
\newtheorem{proposition}{Proposition}
\newtheorem{question}{Question}
\newtheorem{remark}{Remark}
\newtheorem{theorem}{Theorem}
\newtheorem{vidu}{Ví dụ}
\usepackage[left=1cm,right=1cm,top=5mm,bottom=5mm,footskip=4mm]{geometry}
\def\labelitemii{$\circ$}
\DeclareRobustCommand{\divby}{%
	\mathrel{\vbox{\baselineskip.65ex\lineskiplimit0pt\hbox{.}\hbox{.}\hbox{.}}}%
}
\def\labelitemii{$\circ$}
\setlist[itemize]{leftmargin=*}
\setlist[enumerate]{leftmargin=*}

\title{Problem: Magnetic Field -- Bài Tập: Từ Trường}
\author{Nguyễn Quản Bá Hồng\footnote{A Scientist {\it\&} Creative Artist Wannabe. E-mail: {\tt nguyenquanbahong@gmail.com}. Bến Tre City, Việt Nam.}}
\date{\today}

\begin{document}
\maketitle
\begin{abstract}
	This text is a part of the series {\it Some Topics in Elementary STEM \& Beyond}:
	
	{\sc url}: \url{https://nqbh.github.io/elementary_STEM}.
	
	Latest version:
	\begin{itemize}
		\item {\it Problem: Magnetic Field -- Bài Tập: Từ Trường}.
		
		PDF: {\sc url}: \url{https://github.com/NQBH/elementary_STEM_beyond/blob/main/elementary_physics/grade_12/magnetic_field/problem/NQBH_magnetic_field_problem.pdf}.
		
		\TeX: {\sc url}: \url{https://github.com/NQBH/elementary_STEM_beyond/blob/main/elementary_physics/grade_12/magnetic_field/problem/NQBH_magnetic_field_problem.tex}.
		\item {\it Problem \& Solution: Magnetic Field -- Bài Tập \& Lời Giải: Từ Trường}.
		
		PDF: {\sc url}: \url{https://github.com/NQBH/elementary_STEM_beyond/blob/main/elementary_physics/grade_12/magnetic_field/solution/NQBH_magnetic_field_solution.pdf}.
		
		\TeX: {\sc url}: \url{https://github.com/NQBH/elementary_STEM_beyond/blob/main/elementary_physics/grade_12/magnetic_field/solution/NQBH_magnetic_field_solution.tex}.
	\end{itemize}
\end{abstract}
\tableofcontents

%------------------------------------------------------------------------------%

\section{Basic}
Tính chất từ là cơ sở cho nhiều công nghệ khác nhau, e.g., trong công nghệ tàu điện từ, máy biến thế, $\ldots$ Target: Những tri thức mở đầu về từ trường \& ứng dụng điện từ trong thực tiễn.

\noindent\textbf{\textsf{Resources -- Tài nguyên.}}
\begin{enumerate}
	\item \cite[Chủ đề 3: {\it Từ Trường}]{SGK_Vat_Ly_12_CD}. {\sc Nguyễn Văn Khánh, Phạm Thùy Giang, Đoàn Thị Hải Quỳnh, Trần Bá Trình, Trương Anh Tuấn}. {\it Vật Lý 12 Cánh Diều}.
	
	\item \cite[Chủ đề 3: {\it Từ Trường}]{SBT_Vat_Ly_12_CD}. {\sc Nguyễn Văn Khánh, Phạm Thùy Giang, Đoàn Thị Hải Quỳnh, Đỗ Hương Trà, Mai Văn Túc, Trương Anh Tuấn}. {\it Bài Tập Vật Lý 12 Cánh Diều}.
\end{enumerate}
See also, e.g., \href{https://vi.wikipedia.org/wiki/T%E1%BB%AB_tr%C6%B0%E1%BB%9Dng}{Wikipedia{\tt/}từ trường}, \href{https://en.wikipedia.org/wiki/Magnetic_field}{Wikipedia{\tt/}magnetic field}.

%------------------------------------------------------------------------------%

\section{Magnetic Field -- Từ Trường}

\begin{goal}
	Thực hiện thí nghiệm để vẽ được các đường sức từ bằng các dụng cụ đơn giản. Từ trường là trường lực gây ra bởi dòng điện hoặc nam châm, là 1 dạng của vật chất tồn tại xung quanh dòng điện hoặc nam châm mà biểu hiện cụ thể là sự xuất hiện của lực từ tác dụng lên 1 dòng điện hay 1 nam châm khác đặt trong đó.
\end{goal}

\subsection{Tóm tắt kiến thức}

\begin{itemize}
	\item Từ trường là trường lực gây ra bởi dòng điện hoặc nam châm, là 1 dạng của vật chất tồn tại xung quanh dòng điện hoặc nam châm mà biểu hiện cụ thể là sự xuất hiện của lực từ tác dụng lên dòng điện hay nam châm khác đặt trong đó.
	\item Đường sức từ là những đường vẽ trong không gian có từ trường, sao cho tiếp tuyến với nó tại mỗi điểm có phương trùng với phương của kim nam châm nhỏ nằm cân bằng tại điểm đó.
	\item Chiều của đường sức từ tại 1 điểm là chiều từ cực từ nam đến cực từ bắc của kim nam châm nhỏ nằm cân bằng tại điểm đó.
\end{itemize}

%------------------------------------------------------------------------------%

\section{Lực Từ Tác Dụng Lên Đoạn Dây Dẫn Mang Dòng Điện. Cảm Ứng Từ}

\begin{goal}
	Thực hiện thí nghiệm để mô tả được hướng của lực từ tác dụng lên đoạn dây dẫn mang dòng điện đặt trong từ trường. Xác định được độ lớn \& hướng của lực từ tác dụng lên đoạn dây dẫn mang dòng điện đặt trong từ trường. Định nghĩa được cảm ứng từ $B$ \& đơn vị tesla. Đơn vị cơ bản \& dẫn xuất để đo các đại lượng từ. Thiết kế, lựa chọn, thực hiện phương án, đo được (hoặc mô tả được phương pháp đo) cảm ứng từ bằng ``cân dòng điện''. Vận dụng được công thức tính lực $F = BIl\sin\theta$.
\end{goal}

\subsection{Tóm tắt kiến thức}

\begin{itemize}
	\item Cảm ứng từ $\overline{B}$ là 1 đại lượng vector đặc trưng cho từ trường về mặt tác dụng lực: Có phương trùng với phương của kim nam châm nằm cân bằng tại điểm đang xét, có chiều từ cực nam sang cực bắt của kim nam châm. Có độ lớn $B = \dfrac{F}{Il\sin\theta}$ với $F$: độ lớn của lực tương tác giữa từ trường \& đoạn dây dẫn có chiều dài $l$ mang dòng điện có cường độ $I$, $\theta$: góc hợp bởi dòng điện \& đường sức từ.
	\item Lực từ tác dụng lên đoạn dòng điện có chiều dài $l$ \& cường độ $I$ ở trong từ trường đều có cảm ứng từ $\overline{B}$: Có điểm đặt tại trung điểm của đoạn dòng điện. Có phương vuông góc với đoạn dòng điện \& cảm ứng từ $\overline{B}$. Có chiều tuân theo quy tắc bàn tay trái: Đặt bàn tay trái sao cho các đường sức từ đâm xuyên vào lòng bàn tay, chiều từ cổ tay đến các ngón tay trùng với chiều dòng điện, thì ngón cái choãi ra $90^\circ$ chỉ chiều của lực từ tác dụng lên dòng điện. Có độ lớn $F = BIl\sin\theta$.
\end{itemize}

%------------------------------------------------------------------------------%

\section{Cảm Ứng Điện Từ}

\begin{goal}
	Định nghĩa được từ thông \& đơn vị weber. Tiến hành các thí nghiệm đơn giản minh họa được hiện tượng cảm ứng điện từ. Vận dụng được định luật Faraday \& định luật Lenz về cảm ứng điện từ. Giải thích được vài ứng dụng đơn giản của hiện tượng cảm ứng điện từ. Mô tả được mô hình sóng điện từ \& ứng dụng để giải thích sự tạo thành \& lan truyền của các sóng điện từ trong thang sóng điện từ.
\end{goal}

\subsection{Tóm tắt kiến thức}

\begin{itemize}
	\item Từ thông qua diện tích $S$: $\Phi = BS\cos\alpha$.
	\item Suất điện động cảm ứng trong mạch điện kín: $e_{\rm C} = -\dfrac{\Delta\Phi}{\Delta t}$.
	\item Độ lớn suất điện động cảm ứng trong 1 đoạn dây dẫn chuyển động trong từ trường: $|e_{\rm C}| = Blv\sin\theta$.
	\item Hiện tượng cảm ứng điện từ có nhiều ứng dụng, e.g., hãm chuyển động bằng điện từ, chế tạo máy biến áp, $\ldots$
	\item Trường điện từ lan truyền trong không gian được gọi là {\it sóng điện từ}.
\end{itemize}

%------------------------------------------------------------------------------%

\section{Đại Cương Về Dòng Điện Xoay Chiều}

\begin{goal}
	Chu kỳ, tần số, giá trị cực đại, giá trị hiệu dụng của cường độ dòng điện, \& điện áp xoay chiều. Thiết kế phương án (hoặc mô tả được phương pháp) tạo ra dòng điện xoay chiều. Nêu vài ứng dụng của dòng điện xoay chiều trong cuộc sống, tầm quan trọng của việc tuân thủ quy tắc an toàn khi sử dụng dòng điện xoay chiều trong cuộc sống.
\end{goal}

\subsection{Tóm tắt kiến thức}

\begin{itemize}
	\item Nguyên tắc hoạt động của các loại máy phát điện xoay chiều dựa trên hiện tượng cảm ứng điện từ: Khi từ thông qua 1 khung dây biến thiên điều hòa, trong khung dây xuất hiện 1 suất điện động cảm ứng xoay chiều.
	\item Các máy phát điện xoay chiều đều có 2 bộ phận chính là {\it phần ứng \& phần cảm}.
	\item Công thức tổng quát của điện áp xoay chiều giữa 2 đầu 1 đoạn mạch: $u = U_0\cos(\omega t + \varphi_u)$. Công thức tổng quát của cường độ dòng điện xoay chiều trong đoạn mạch: $i = I_0\cos(\omega t + \varphi_i)$. Độ lệch pha của điện áp so với cường độ dòng điện: $\Delta\varphi = \varphi_u - \varphi_i$.
	\item {\it Cường độ hiệu dụng} của dòng điện xoay chiều: $I = \dfrac{I_0}{\sqrt{2}}$.
	\item {\it Giá trị hiệu dụng} của điện áp xoay chiều: $U = \dfrac{U_0}{\sqrt{2}}$.U
	\item Để đảm bảo an toàn, cần tuân thủ các quy tắc an toàn điện.
\end{itemize}

%------------------------------------------------------------------------------%

\section{Miscellaneous}

%------------------------------------------------------------------------------%

\printbibliography[heading=bibintoc]
	
\end{document}