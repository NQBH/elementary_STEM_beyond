\documentclass{article}
\usepackage[backend=biber,natbib=true,style=alphabetic,maxbibnames=50]{biblatex}
\addbibresource{/home/nqbh/reference/bib.bib}
\usepackage[utf8]{vietnam}
\usepackage{tocloft}
\renewcommand{\cftsecleader}{\cftdotfill{\cftdotsep}}
\usepackage[colorlinks=true,linkcolor=blue,urlcolor=red,citecolor=magenta]{hyperref}
\usepackage{amsmath,amssymb,amsthm,enumitem,float,graphicx,mathtools,tikz}
\usetikzlibrary{angles,calc,intersections,matrix,patterns,quotes,shadings}
\allowdisplaybreaks
\newtheorem{assumption}{Assumption}
\newtheorem{baitoan}{}
\newtheorem{cauhoi}{Câu hỏi}
\newtheorem{conjecture}{Conjecture}
\newtheorem{corollary}{Corollary}
\newtheorem{dangtoan}{Dạng toán}
\newtheorem{definition}{Definition}
\newtheorem{dinhly}{Định lý}
\newtheorem{dinhnghia}{Định nghĩa}
\newtheorem{example}{Example}
\newtheorem{ghichu}{Ghi chú}
\newtheorem{goal}{Goal}
\newtheorem{hequa}{Hệ quả}
\newtheorem{hypothesis}{Hypothesis}
\newtheorem{lemma}{Lemma}
\newtheorem{luuy}{Lưu ý}
\newtheorem{nhanxet}{Nhận xét}
\newtheorem{notation}{Notation}
\newtheorem{note}{Note}
\newtheorem{principle}{Principle}
\newtheorem{problem}{Problem}
\newtheorem{proposition}{Proposition}
\newtheorem{question}{Question}
\newtheorem{remark}{Remark}
\newtheorem{theorem}{Theorem}
\newtheorem{vidu}{Ví dụ}
\usepackage[left=1cm,right=1cm,top=5mm,bottom=5mm,footskip=4mm]{geometry}
\def\labelitemii{$\circ$}
\DeclareRobustCommand{\divby}{%
	\mathrel{\vbox{\baselineskip.65ex\lineskiplimit0pt\hbox{.}\hbox{.}\hbox{.}}}%
}
\def\labelitemii{$\circ$}
\setlist[itemize]{leftmargin=*}
\setlist[enumerate]{leftmargin=*}

\title{Problem: Ideal Gas -- Bài Tập: Khí Lý Tưởng}
\author{Nguyễn Quản Bá Hồng\footnote{A Scientist {\it\&} Creative Artist Wannabe. E-mail: {\tt nguyenquanbahong@gmail.com}. Bến Tre City, Việt Nam.}}
\date{\today}

\begin{document}
\maketitle
\begin{abstract}
	This text is a part of the series {\it Some Topics in Elementary STEM \& Beyond}:
	
	{\sc url}: \url{https://nqbh.github.io/elementary_STEM}.
	
	Latest version:
	\begin{itemize}
		\item {\it Problem: Ideal Gas -- Bài Tập: Khí Lý Tưởng}.
		
		PDF: {\sc url}: \url{.pdf}.
		
		\TeX: {\sc url}: \url{.tex}.
		\item {\it Problem \& Solution: Ideal Gas -- Bài Tập \& Lời Giải: Khí Lý Tưởng}.
		
		PDF: {\sc url}: \url{.pdf}.
		
		\TeX: {\sc url}: \url{.tex}.
	\end{itemize}
\end{abstract}
\tableofcontents

%------------------------------------------------------------------------------%

\section{Basic}
Vì chất khí tồn tại xung quanh ta \& luôn ảnh hưởng đến các vật, việc hiểu rõ về chất khí là nhu cầu của rất nhiều lĩnh vực khoa học \& đời sống. Việc nghiên cứu về chất khí trước hết được thực hiện với khí lý tưởng -- 1 mô hình lý thuyết gần đúng với hầu hết các chất khí ở điều kiện nhiệt độ \& áp suất thông thường. Từ mô hình động học phân tử, chủ đề khí lý tưởng tập trung nghiên cứu các đặc điểm chuyển động của các phân tử chất khí, từ đó hoàn thiện dần các giả thuyết của thuyết động học phân tử chất khí.

\noindent\textbf{\textsf{Resources -- Tài nguyên.}}
\begin{enumerate}
	\item \cite[Chủ đề 2: {\it Khí Lý Tưởng}]{SGK_Vat_Ly_12_CD}. {\sc Nguyễn Văn Khánh, Phạm Thùy Giang, Đoàn Thị Hải Quỳnh, Trần Bá Trình, Trương Anh Tuấn}. {\it Vật Lý 12 Cánh Diều}.
	
	\item \cite[Chủ đề 1: {\it Khí Lý Tưởng}]{SBT_Vat_Ly_12_CD}. {\sc Nguyễn Văn Khánh, Phạm Thùy Giang, Đoàn Thị Hải Quỳnh, Đỗ Hương Trà, Mai Văn Túc, Trương Anh Tuấn}. {\it Bài Tập Vật Lý 12 Cánh Diều}.
\end{enumerate}
See also, e.g., \href{https://vi.wikipedia.org/wiki/Kh%C3%AD_l%C3%BD_t%C6%B0%E1%BB%9Fng}{Wikipedia{\tt/}khí lý tưởng}, \href{https://en.wikipedia.org/wiki/Ideal_gas}{Wikipedia{\tt/}ideal gas}.

%------------------------------------------------------------------------------%

\section{Mô hình động học phân tử chất khí}

\begin{goal}
	Phân tích mô hình chuyển động Brown, nêu được các phân tử trong chất khí chuyển động hỗn loạn. Từ các kết quả thực nghiệm hoặc mô hình, thảo luận để nêu được các giả thuyết của thuyết động lực phân tử chất khí.
\end{goal}

\subsection{Tóm tắt kiến thức}

\begin{itemize}
	\item {\it Mô hình động học phân tử chất khí}: Chất khí được cấu tạo từ các phân tử có kích thước rất nhỏ so với khoảng cách giữa chúng. Các phân tử khí chuyển động hỗn loạn, không ngừng; các phần tử khí chuyển động càng nhanh thì nhiệt độ chất khí càng cao. Khi chuyển động hỗn loạn, các phân tử khí va chạm vào thành bình gây áp suất lên thành bình.
	\item {\it Khí lý tưởng} là chất khí gồm các phần tử có thể bỏ qua kích thước của chúng, chỉ tương tác khi va chạm. Giữa 2 va chạm liên tiếp, chúng chuyển động thẳng đều. Va chạm của các phân tử khí lý tưởng với nhau \& với thành bình là va chạm hoàn toàn đàn hồi.
\end{itemize}

%------------------------------------------------------------------------------%

\section{Phương trình trạng thái khí lý tưởng}

\begin{goal}
	Thực hiện thí nghiệm khảo sát được định luật Boyle: Khi giữ không đổi nhiệt độ của 1 khối lượng khí xác định thì áp suất gây ra bởi khí tỷ lệ nghịch với thể tích của nó. Thực hiện thí nghiệm minh họa được đinh luật Charles: Khi giữ không đổi áp suất của 1 khối lượng khí xác định thì thể tích của khí tỷ lệ với nhiệt độ tuyệt đối của nó. Sử dụng định luật Boyle \& định luật Charles rút ra được phương trình trạng thái của khí lý tưởng. Vận dụng được phương trình trạng thái của khí lý tưởng.
\end{goal}

\subsection{Tóm tắt kiến thức}

\begin{itemize}
	\item Định luật Boyle: $pV =$ const ($T$ không đổi).
	\item Định luật Charles: $\dfrac{V}{T} =$ const ($p$ không đổi).
	\item Phương trình trạng thái của 1 lượng khí lý tưởng: $pV = nRT$ với $n$: số mol khí.
\end{itemize}

%------------------------------------------------------------------------------%

\section{Áp suất \& động năng phân tử chất khí}

\begin{goal}
	Giải thích được chuyển động của các phân tử ảnh hưởng như thế nào đến áp suất tác dụng lên thành bình \& từ đó rút ra được hệ thức $p = \frac{1}{3}\mu m\overline{v^2}$ với $\mu$: số phân tử khí trong 1 đơn vị thể tích (sử dụng mô hình va chạm 1 chiều đơn giản rồi mở rộng ra cho trường hợp 3D bằng cách sử dụng $\frac{1}{3}\overline{v^2} = \overline{v_x^2}$, không yêu cầu chứng minh 1 cách chính xác \& chi tiết). Nêu được {\it công thức hằng số Boltzmann} $k = \frac{R}{N_A}$. So sánh $pV = \frac{1}{3}Nm\overline{v^2}$ với $pV = nRT$, rút ra được động năng tịnh tiến trung bình của phân tử tỷ lệ với nhiệt độ $T$.
\end{goal}

\subsection{Tóm tắt kiến thức}

\begin{itemize}
	\item Áp suất khí lý tưởng: $p = \dfrac{1}{3}\dfrac{Nm\overline{v^2}}{V} = \dfrac{1}{3}\rho\overline{v^2}$. Hằng số Boltzmann: $k = \dfrac{R}{N_A}$.
	\item Động năng tịnh tiến trung bình của phân tử khí lý tưởng tỷ lệ thuận với nhiệt độ $T$: $\frac{1}{2}m\overline{v^2} = \frac{3}{2}kT$.
\end{itemize}

%------------------------------------------------------------------------------%

\section{Miscellaneous}

%------------------------------------------------------------------------------%

\printbibliography[heading=bibintoc]
	
\end{document}