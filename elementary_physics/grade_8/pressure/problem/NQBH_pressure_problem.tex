\documentclass{article}
\usepackage[backend=biber,natbib=true,style=alphabetic,maxbibnames=50]{biblatex}
\addbibresource{/home/nqbh/reference/bib.bib}
\usepackage[utf8]{vietnam}
\usepackage{tocloft}
\renewcommand{\cftsecleader}{\cftdotfill{\cftdotsep}}
\usepackage[colorlinks=true,linkcolor=blue,urlcolor=red,citecolor=magenta]{hyperref}
\usepackage{amsmath,amssymb,amsthm,float,graphicx,mathtools,tikz}
\usetikzlibrary{angles,calc,intersections,matrix,patterns,quotes,shadings}
\allowdisplaybreaks
\newtheorem{assumption}{Assumption}
\newtheorem{baitoan}{}
\newtheorem{cauhoi}{Câu hỏi}
\newtheorem{conjecture}{Conjecture}
\newtheorem{corollary}{Corollary}
\newtheorem{dangtoan}{Dạng toán}
\newtheorem{definition}{Definition}
\newtheorem{dinhly}{Định lý}
\newtheorem{dinhnghia}{Định nghĩa}
\newtheorem{example}{Example}
\newtheorem{ghichu}{Ghi chú}
\newtheorem{hequa}{Hệ quả}
\newtheorem{hypothesis}{Hypothesis}
\newtheorem{lemma}{Lemma}
\newtheorem{luuy}{Lưu ý}
\newtheorem{nhanxet}{Nhận xét}
\newtheorem{notation}{Notation}
\newtheorem{note}{Note}
\newtheorem{principle}{Principle}
\newtheorem{problem}{Problem}
\newtheorem{proposition}{Proposition}
\newtheorem{question}{Question}
\newtheorem{remark}{Remark}
\newtheorem{theorem}{Theorem}
\newtheorem{vidu}{Ví dụ}
\usepackage[left=1cm,right=1cm,top=5mm,bottom=5mm,footskip=4mm]{geometry}
\def\labelitemii{$\circ$}
\DeclareRobustCommand{\divby}{%
	\mathrel{\vbox{\baselineskip.65ex\lineskiplimit0pt\hbox{.}\hbox{.}\hbox{.}}}%
}

\title{Problem: Pressure -- Bài Tập: Áp Suất}
\author{Nguyễn Quản Bá Hồng\footnote{Independent Researcher, Ben Tre City, Vietnam\\e-mail: \texttt{nguyenquanbahong@gmail.com}; website: \url{https://nqbh.github.io}.}}
\date{\today}

\begin{document}
\maketitle
\tableofcontents

%------------------------------------------------------------------------------%

\section{Pressure -- Áp Suất}

\begin{baitoan}[\cite{Van_Quyen_Hanh_Nhu_10_chuyen_Ly}, VD1, p. 49]
	Đối với việc làm nhà thì vấn đề xây dựng móng nhà vô cùng quan trọng để tránh sự sụp lún. Tính chiều cao lớn nhất của 1 tường gạch nếu áp suất lớn nhất mà móng có thể chịu được là $\rm1170000\ N{\tt/}m^2$. Biết trọng lượng riêng trung bình của gạch \& vữa là $\rm18000\ N{\tt/}m^3$. Giả sử nếu tường dày {\rm20 cm}, dài {\rm8 m}, \& chiều cao như trên thì áp lực của tường đã tác dụng lên móng là bao nhiêu?
\end{baitoan}

\begin{baitoan}[\cite{Van_Quyen_Hanh_Nhu_10_chuyen_Ly}, VD2, p. 50]
	1 ôtô có trọng lượng {\rm20000 N}, mỗi bánh xe có diện tích tiếp xúc với mặt đường là $\rm10\ cm^2$. (a) Tính áp suất của xe lên mặt đường khi xe dừng lại. (b) Giả sử ôtô đó đi với 1 xe tăng có trọng lượng {\rm40000 N} trên 1 vũng lầy, xe nào dễ bị lún sâu vào vũng lầy hơn, biết diện tích tiếp xúc của các bản xích với mặt đất là $\rm1.6\ m^2$.
\end{baitoan}

\begin{baitoan}[\cite{Van_Quyen_Hanh_Nhu_10_chuyen_Ly}, VD1, pp. 50--51]
	1 người thợ lặn mặc 1 bộ áo lặn chịu được 1 áp suất tối đa là $\rm3\cdot10^5\ N{\tt/}m^2$. Biết trọng lượng riêng của nước là $\rm10^4\ N{\tt/}m^3$. (a) Thợ lặn đó lặn sâu nhất bao nhiêu {\rm m}? (b) Tính áp lực của nước tác dụng lên cửa kính quan sát của áo lặn có diện tích $\rm400\ cm^2$ khi lặn sâu {\rm20 m}.
\end{baitoan}

\begin{baitoan}[\cite{Van_Quyen_Hanh_Nhu_10_chuyen_Ly}, VD1, p. 52]
	Trong bình thông nhau, nhánh lớn có tiết diện lớn gấp đôi nhánh nhỏ. Khi chưa mở khóa T ở chính giữa, chiều cao của cột nước ở nhánh lớn là {\rm30 cm}. Tìm chiều cao của cột nước ở 2 nhánh sau khi đã mở khóa T \& khi nước đã đứng yên. Bỏ qua thể tích của ống nối 2 nhánh.
\end{baitoan}

\begin{baitoan}[\cite{Van_Quyen_Hanh_Nhu_10_chuyen_Ly}, VD2, p. 52]
	Bình thông nhau gồm 2 nhánh hình trụ có tiết diện lần lượt là $S_1,S_2$ \& có chứa nước. Trên mặt nước có đặt 2 piston mỏng, khối lượng $m_1,m_2$. Mực nước 2 bên chênh nhanh 1 đoạn $h$. (a) Tìm khối lượng $m$ của quả cân đặt lên piston lớn để mực nước ở 2 bên ngang nhau. (b) Nếu đặt quả cân trên sang piston nhỏ thì mực nước bây giờ sẽ chênh nhau 1 đoạn $h$ bao nhiêu?
\end{baitoan}

\begin{baitoan}[\cite{Van_Quyen_Hanh_Nhu_10_chuyen_Ly}, VD1, p. 53]
	Đường kính piston nhỏ của 1 máy dùng chất lỏng là {\rm2 cm}. Hỏi diện tích tối thiểu của piston lớn là bao nhiêu để tác dụng 1 lực {\rm120 N} lên piston nhỏ có thể nâng được 1 ôtô có trọng lượng {\rm24000 N}.
\end{baitoan}

\begin{baitoan}[\cite{Van_Quyen_Hanh_Nhu_10_chuyen_Ly}, VD2, p. 54]
	1 máy ép dùng dầu có 2 xilanh $A,B$ thẳng đứng nối với nhau bằng 1 ống nhỏ. Tiết diện thẳng của xilanh A là $\rm200\ cm^2$ \& của xilanh B là $\rm4\ cm^2$. Trọng lượng riêng của dầu là $\rm8000\ N{\tt/}m^3$. Đầu tiên mực dầu ở trong 2 xilanh ở cùng 1 độ cao. (a) Đặt lên mặt dầu trong A 1 piston có trọng lượng {\rm40 N}. Tính độ chênh lệch giữa 2 mặt chất lỏng trong 2 xilanh sau khi cân bằng. (b) Cần phải đặt lên mặt chất lỏng trong B 1 piston có trọng lượng bao nhiêu để 2 mặt dưới của 2 piston nằm trên cùng 1 mặt phẳng.
\end{baitoan}

%------------------------------------------------------------------------------%

\section{Lực Đẩy Acsimet}

%------------------------------------------------------------------------------%

\section{Miscellaneous}

%------------------------------------------------------------------------------%

\printbibliography[heading=bibintoc]
	
\end{document}