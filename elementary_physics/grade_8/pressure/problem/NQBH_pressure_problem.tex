\documentclass{article}
\usepackage[backend=biber,natbib=true,style=alphabetic,maxbibnames=50]{biblatex}
\addbibresource{/home/nqbh/reference/bib.bib}
\usepackage[utf8]{vietnam}
\usepackage{tocloft}
\renewcommand{\cftsecleader}{\cftdotfill{\cftdotsep}}
\usepackage[colorlinks=true,linkcolor=blue,urlcolor=red,citecolor=magenta]{hyperref}
\usepackage{amsmath,amssymb,amsthm,float,graphicx,mathtools,tikz}
\usetikzlibrary{angles,calc,intersections,matrix,patterns,quotes,shadings}
\allowdisplaybreaks
\newtheorem{assumption}{Assumption}
\newtheorem{baitoan}{}
\newtheorem{cauhoi}{Câu hỏi}
\newtheorem{conjecture}{Conjecture}
\newtheorem{corollary}{Corollary}
\newtheorem{dangtoan}{Dạng toán}
\newtheorem{definition}{Definition}
\newtheorem{dinhly}{Định lý}
\newtheorem{dinhnghia}{Định nghĩa}
\newtheorem{example}{Example}
\newtheorem{ghichu}{Ghi chú}
\newtheorem{hequa}{Hệ quả}
\newtheorem{hypothesis}{Hypothesis}
\newtheorem{lemma}{Lemma}
\newtheorem{luuy}{Lưu ý}
\newtheorem{nhanxet}{Nhận xét}
\newtheorem{notation}{Notation}
\newtheorem{note}{Note}
\newtheorem{principle}{Principle}
\newtheorem{problem}{Problem}
\newtheorem{proposition}{Proposition}
\newtheorem{question}{Question}
\newtheorem{remark}{Remark}
\newtheorem{theorem}{Theorem}
\newtheorem{vidu}{Ví dụ}
\usepackage[left=1cm,right=1cm,top=5mm,bottom=5mm,footskip=4mm]{geometry}
\def\labelitemii{$\circ$}
\DeclareRobustCommand{\divby}{%
	\mathrel{\vbox{\baselineskip.65ex\lineskiplimit0pt\hbox{.}\hbox{.}\hbox{.}}}%
}

\title{Problem: Pressure -- Bài Tập: Áp Suất}
\author{Nguyễn Quản Bá Hồng\footnote{Independent Researcher, Ben Tre City, Vietnam\\e-mail: \texttt{nguyenquanbahong@gmail.com}; website: \url{https://nqbh.github.io}.}}
\date{\today}

\begin{document}
\maketitle
\tableofcontents

%------------------------------------------------------------------------------%

\section{Pressure -- Áp Suất}

\begin{baitoan}[\cite{Van_Quyen_Hanh_Nhu_10_chuyen_Ly}, VD1, p. 49]
	Đối với việc làm nhà thì vấn đề xây dựng móng nhà vô cùng quan trọng để tránh sự sụp lún. Tính chiều cao lớn nhất của 1 tường gạch nếu áp suất lớn nhất mà móng có thể chịu được là $\rm1170000\ N{\tt/}m^2$. Biết trọng lượng riêng trung bình của gạch \& vữa là $\rm18000\ N{\tt/}m^3$. Giả sử nếu tường dày {\rm20 cm}, dài {\rm8 m}, \& chiều cao như trên thì áp lực của tường đã tác dụng lên móng là bao nhiêu?
\end{baitoan}

\begin{baitoan}[\cite{Van_Quyen_Hanh_Nhu_10_chuyen_Ly}, VD2, p. 50]
	1 ôtô có trọng lượng {\rm20000 N}, mỗi bánh xe có diện tích tiếp xúc với mặt đường là $\rm10\ cm^2$. (a) Tính áp suất của xe lên mặt đường khi xe dừng lại. (b) Giả sử ôtô đó đi với 1 xe tăng có trọng lượng {\rm40000 N} trên 1 vũng lầy, xe nào dễ bị lún sâu vào vũng lầy hơn, biết diện tích tiếp xúc của các bản xích với mặt đất là $\rm1.6\ m^2$.
\end{baitoan}

\begin{baitoan}[\cite{Van_Quyen_Hanh_Nhu_10_chuyen_Ly}, VD1, pp. 50--51]
	1 người thợ lặn mặc 1 bộ áo lặn chịu được 1 áp suất tối đa là $\rm3\cdot10^5\ N{\tt/}m^2$. Biết trọng lượng riêng của nước là $\rm10^4\ N{\tt/}m^3$. (a) Thợ lặn đó lặn sâu nhất bao nhiêu {\rm m}? (b) Tính áp lực của nước tác dụng lên cửa kính quan sát của áo lặn có diện tích $\rm400\ cm^2$ khi lặn sâu {\rm20 m}.
\end{baitoan}

\begin{baitoan}[\cite{Van_Quyen_Hanh_Nhu_10_chuyen_Ly}, VD1, p. 52]
	Trong bình thông nhau, nhánh lớn có tiết diện lớn gấp đôi nhánh nhỏ. Khi chưa mở khóa T ở chính giữa, chiều cao của cột nước ở nhánh lớn là {\rm30 cm}. Tìm chiều cao của cột nước ở 2 nhánh sau khi đã mở khóa T \& khi nước đã đứng yên. Bỏ qua thể tích của ống nối 2 nhánh.
\end{baitoan}

\begin{baitoan}[\cite{Van_Quyen_Hanh_Nhu_10_chuyen_Ly}, VD2, p. 52]
	Bình thông nhau gồm 2 nhánh hình trụ có tiết diện lần lượt là $S_1,S_2$ \& có chứa nước. Trên mặt nước có đặt 2 piston mỏng, khối lượng $m_1,m_2$. Mực nước 2 bên chênh nhanh 1 đoạn $h$. (a) Tìm khối lượng $m$ của quả cân đặt lên piston lớn để mực nước ở 2 bên ngang nhau. (b) Nếu đặt quả cân trên sang piston nhỏ thì mực nước bây giờ sẽ chênh nhau 1 đoạn $h$ bao nhiêu?
\end{baitoan}

\begin{baitoan}[\cite{Van_Quyen_Hanh_Nhu_10_chuyen_Ly}, VD1, p. 53]
	Đường kính piston nhỏ của 1 máy dùng chất lỏng là {\rm2 cm}. Hỏi diện tích tối thiểu của piston lớn là bao nhiêu để tác dụng 1 lực {\rm120 N} lên piston nhỏ có thể nâng được 1 ôtô có trọng lượng {\rm24000 N}.
\end{baitoan}

\begin{baitoan}[\cite{Van_Quyen_Hanh_Nhu_10_chuyen_Ly}, VD2, p. 54]
	1 máy ép dùng dầu có 2 xilanh $A,B$ thẳng đứng nối với nhau bằng 1 ống nhỏ. Tiết diện thẳng của xilanh A là $\rm200\ cm^2$ \& của xilanh B là $\rm4\ cm^2$. Trọng lượng riêng của dầu là $\rm8000\ N{\tt/}m^3$. Đầu tiên mực dầu ở trong 2 xilanh ở cùng 1 độ cao. (a) Đặt lên mặt dầu trong A 1 piston có trọng lượng {\rm40 N}. Tính độ chênh lệch giữa 2 mặt chất lỏng trong 2 xilanh sau khi cân bằng. (b) Cần phải đặt lên mặt chất lỏng trong B 1 piston có trọng lượng bao nhiêu để 2 mặt dưới của 2 piston nằm trên cùng 1 mặt phẳng.
\end{baitoan}

\begin{baitoan}[\cite{Van_Quyen_Hanh_Nhu_10_chuyen_Ly}, 1., p. 61]
	1 bình thông nhau chứa nước biển, đổ thêm xăng vào 1 nhánh. Mặt thoáng ở 2 nhánh chênh lệch nhau {\rm18 mm}. Tính độ cao của cột xăng, cho biết trọng lượng riêng của nước biển là $\rm10300\ N{\tt/}m^3$, của xăng là $\rm7000\ N{\tt/}m^3$.
\end{baitoan}

\begin{baitoan}[\cite{Van_Quyen_Hanh_Nhu_10_chuyen_Ly}, 2., p. 61]
	1 chậu chứa nước \& 1 chậu chứa dầu, mực nước \& dầu trong 2 chậu ngang nhau. Lấy 1 ống xiphông bên trong đựng dầy nước nhúng 1 đầu vào chậu nước, đầu kia vào chậu đựng dầu. Mức chất lỏng trong 2 chậu ngang nhau. Hỏi nước trong ống có chảy không, nếu có chảy thì chảy theo hướng nào? Cho biết khối lượng riêng của nước lớn hơn của dầu.
\end{baitoan}

\begin{baitoan}[\cite{Van_Quyen_Hanh_Nhu_10_chuyen_Ly}, 3., p. 61]
	Cho 2 bình hình trụ thông với nhau bằng 1 ống nhỏ có khóa thể tích không đáng kể. Bán kính đáy của bình A là $r_1$, của bình B là $r_2 = 0.5r_1$ (khóa K đóng). Đổ vào bình A 1 lượng nước đến chiều cao $h_1 = 18$ {\rm cm}, sau đó đổ lên trên mặt nước 1 lớp chất lỏng cao $h_2 = 4$ {\rm cm} có trọng lượng riêng $d_2 = 9000$ {\rm N{\tt/}$\rm m^3$} \& đổ vào bình B chất lỏng thứ 3 có chiều cao $h_3 = 6$ {\rm cm} trọng lượng riêng $d_3 = 8000$ {\rm N{\tt/}$\rm m^3$} (trọng lượng riêng của nước là $d_1 = 10000$ {\rm N{\tt/}$\rm m^3$}, các chất lỏng không hòa lẫn vào nhau). Mở khóa K để 2 bình thông nhau. Tính: (a) Độ chênh lệch chiều cao của mặt thoáng chất lỏng ở 2 bình. (b) Tính thể tích nước chảy qua khóa K. Biết diện tích đáy của bình A là $\rm12\ cm^2$.
\end{baitoan}

\begin{baitoan}[\cite{Van_Quyen_Hanh_Nhu_10_chuyen_Ly}, 4., p. 61]
	2 xilanh có tiết diện $S_1,S_2$ thông với nhau \& có chứa nước. Trên mặt nước có đặt các piston mỏng có khối lượng riêng khác nhau nên mực nước ở 2 bên chênh nhau 1 đoạn $h$. Đổ 1 lớp dầu lên piston $S_1$ sao cho mực nước ở 2 bên ngang nhau. Tính độ chênh lệch $x$ của mực nước ở 2 xilanh (theo $S_1,S_2,h$). Nếu lấy lượng dầu đó từ bên $S_1$ đổ lên piston $S_2$.
\end{baitoan}

\begin{baitoan}[\cite{Van_Quyen_Hanh_Nhu_10_chuyen_Ly}, 5., p. 61]
	1 ống chữ U có 2 nhánh hình trụ tiết diện khác nhau \& chứa thủy ngân. Đổ nước vào nhánh nhỏ đến khi cân bằng thì thấy mực thủy ngân ở 2 nhánh chênh lệch nhau $h = 4$ {\rm cm}. Tính chiều cao cột nước cho biết trọng lượng riêng của thủy ngân là $d_1 = 136000$ {\rm N{\tt/}$\rm m^3$}, của nước là $d_2 = 10000$ {\rm N{\tt/}$\rm m^3$}. Kết quả có thay đổi không nếu đổ nước vào nhánh to?
\end{baitoan}

\begin{baitoan}[\cite{Van_Quyen_Hanh_Nhu_10_chuyen_Ly}, 6., p. 62]
	2 hình trụ $A,B$ đặt thẳng đứng có tiết diện lần lượt là $\rm100\ cm^2,200\ cm^2$ được nối thông đáy bằng 1 ống nhỏ qua khóa k ở chính giữa. Lúc đầu khóa k để ngăn cách 2 bình, sau đó đổ {\rm3 l} dầu vào bình A, đổ {\rm5.4 l} nước vào bình B. Sau đó mở khóa k để tạo thành 1 bình thông nhau. Tính độ cao mực chất lỏng ở mỗi bình. Cho biết trọng lượng riêng của dầu \& của nước lần lượt là $d_1 = 8000$ {\rm N{\tt/}$\rm m^3$}, $d_2 = 10000$ {\rm N{\tt/}$\rm m^3$}.
\end{baitoan}

\begin{baitoan}[\cite{Van_Quyen_Hanh_Nhu_10_chuyen_Ly}, 7., p. 62]
	1 ống thủy tinh hình trụ 1 đầu kín, 1 đầu hở có diện tích đáy $\rm5\ cm^2$ chứa đầy dầu. Biết thể tích dầu trong ống là $\rm60\ cm^3$, trọng lượng riêng của dầu là $\rm9000$ {\rm N{\tt/}$\rm m^3$}, áp suất khí quyển là $p_0 = 10^5\ {\rm Pa}$. Tính: (a) Áp suất tại đáy ống khi đặt ống thẳng đứng trong không khí, miệng ống hướng lên. (b) Áp suất tại đáy ống khi dìm ống thẳng đứng trong nước, miệng ống hướng xuống, cách mặt thoáng {\rm70 cm}, biết trọng lượng riêng của nước là $\rm10000$ {\rm N{\tt/}$\rm m^3$}.
\end{baitoan}

\begin{baitoan}[\cite{Van_Quyen_Hanh_Nhu_10_chuyen_Ly}, 1., p. 62]
	Trong 1 trận chiến có 1 xe tăng nặng $33$ tấn có diện tích tiếp xúc của các bản xích với mặt đất là $\rm1.5\ m^2$. 1 ôtô nặng $2$ tấn có diện tích tiếp xúc 2 bánh với mặt đất là $\rm250\ cm^2$. Cả ôtô \& xe tăng cùng đi vào 1 vùng đất mềm. Biết áp suất tối đa mà vùng đất chịu được để khi vật đi vào mà không bị lún là $2\cdot10^5$ {\rm Pa}. Xe tăng \& ôtô khi đi vào vùng đất này, xe nào dễ bị xa lầy hơn?
\end{baitoan}

\begin{baitoan}[\cite{Van_Quyen_Hanh_Nhu_10_chuyen_Ly}, 2., p. 62]
	1 chiếc tàu bị thủng 1 lỗ ở độ sâu {\rm2.8 m}. Đặt 1 miếng vá áp vào lỗ thủng từ phía trong. Hỏi cần 1 lực tối thiểu bằng bao nhiêu để giữ miếng vá nếu lỗ thủng rộng $\rm150\ cm^2$ \& trọng lượng riêng của nước là $\rm10000$ {\rm N{\tt/}$\rm m^3$}.
\end{baitoan}

\begin{baitoan}[\cite{Van_Quyen_Hanh_Nhu_10_chuyen_Ly}, 3., p. 62]
	Móng nhà là 1 phần vô cùng quan trọng để xây nhà. Do đó việc tính toán xây dựng 1 nền móng vững chắc để tránh sạt lở là vô cùng quan trọng, giả sử 1 ngôi nhà có khối lượng là $140$ tấn. Mặt đất ở nơi xây nhà chỉ chịu được áp suất tối đa là $\rm16\ N{\tt/}cm^2$. Tính diện tích tối thiểu của móng.
\end{baitoan}

\begin{baitoan}[\cite{Van_Quyen_Hanh_Nhu_10_chuyen_Ly}, 4., p. 62]
	1 máy ép dùng dầu có 2 xilanh $A,B$ thẳng đứng nối với nhau bằng 1 ống nhỏ. Tiết diện thẳng của xilanh A là $\rm200\ cm^2$ \& của xilanh B là $\rm4\ cm^2$. Trọng lượng riêng của dầu là $\rm8000$ {\rm N{\tt/}$\rm m^3$}. Đầu tiên mực dầu ở trong 2 xilanh ở cùng 1 độ cao. Đặt lên mặt dầu trong A 1 piston có trọng lượng {\rm40 N}. Tính độ chênh lệch giữa 2 mặt chất lỏng trong 2 xilanh sau khi cân bằng.
\end{baitoan}

\begin{baitoan}[\cite{Van_Quyen_Hanh_Nhu_10_chuyen_Ly}, 5., p. 62]
	Đặt 1 bao gạo khối lượng {\rm50 kg} lên 1 cái ghế 4 chân có khối lượng {\rm4 kg}. Diện tích tiếp xúc với mặt đất của mỗi chân ghế là $\rm8\ cm^2$. Tính áp suất các chân ghế tác dụng lên mặt đất.
\end{baitoan}

\begin{baitoan}[\cite{Van_Quyen_Hanh_Nhu_10_chuyen_Ly}, 6., p. 63]
	Bố mẹ Hoa muốn có 1 mảnh đất dài {\rm10 m}, rộng {\rm40 m}. Gia đình Hoa muốn xây 1 ngôi nhà để chung sống, tuy nhiên để tiết kiệm chi phí \& xây 1 ngôi nhà theo sở thích của mình nên bố mẹ Hoa quyết định tự thiết căn nhà. Nhưng do thiếu kinh nghiệm nên bố mẹ Hoa có rất nhiều vấn đề băn khoăn lo lắng, đầu tiên về tường thì muốn xây 1 bức tường dài {\rm 8 m}, dày {\rm22 cm}, nhưng không biết chiều cao giới hạn của bức tường là bao nhiêu, bố mẹ Hoa lo sợ nếu xây quá cao thì bức tường dễ bị đổ. Vì vậy bố mẹ Hoa đã đến công ty xây dựng để được tư vấn, các nhân viên ở công ty này đã tư vấn cho bố mẹ Hoa chiều cao giới hạn mà bức tường chịu được là bao nhiêu. Biết áp suất tối đa mà nền đất chịu được là $\rm100000$ {\rm N{\tt/}$\rm m^2$} \& khối lượng riêng trung bình của bức tường là $2000$ {\rm kg{\tt/}$\rm m^3$}.
\end{baitoan}

\begin{baitoan}[\cite{Van_Quyen_Hanh_Nhu_10_chuyen_Ly}, 7., p. 63]
	Tháp Eiffel là 1 công trình kiến trúc bằng thép nằm trên công viên ChampdeMars, cạnh sông Seine, thành phố Paris. Trở thành biểu tượng của ``kinh đô ánh sáng,'' tháp Eiffen là 1 trong các công trình kiến trúc nổi tiếng nhất toàn cầu. Để đo độ cao của tháp Eiffel 1 kỹ sư đã sử dụng khí áp kế. Khi ở chân tháp, áp kế chỉ {\rm76 cmHg}, khi đặt áp kế ở đỉnh tháp, áp kế chỉ {\rm733 cmHg}. Biết trọng lượng riêng của không khí \& thủy ngân lần lượt là $12.5$ {\rm N{\tt/}$\rm m^3$}, $136000$ {\rm N{\tt/}$\rm m^3$}. Xác định chiều cao tháp Eiffen.
\end{baitoan}

\begin{baitoan}[\cite{Van_Quyen_Hanh_Nhu_10_chuyen_Ly}, 8., p. 63]
	1 bình hình trụ tiết diện $\rm10\ cm^2$ chứa nước tới độ cao {\rm30 cm}. 1 bình hình trụ khác có tiết diện $\rm12\ cm^2$ chứa nước tới độ cao {\rm50 cm}. Tính độ cao cột nước ở mỗi bình nếu nối chúng thông đáy bằng 1 ống nằm ngang nhỏ có tiết diện không đáng kể.
\end{baitoan}

%------------------------------------------------------------------------------%

\section{Lực Đẩy Acsimet}


\begin{baitoan}[\cite{Van_Quyen_Hanh_Nhu_10_chuyen_Ly}, VD1, p. 55]
	Cho 1 khối gỗ hình hộp lập phương cạnh $a = 10$ {\rm cm} có trọng lượng riêng $d = 6000$ {\rm N{\tt/}$\rm m^3$} được thả vào trong nước sao cho 1 mặt đáy song song với mặt thoáng của nước. Trọng lượng riêng của nước là $d_n = 10000$ {\rm N{\tt/}$\rm m^3$}. (a) Tính lực đẩy Acsimet của nước tác dụng lên khối gỗ. (b) Tính chiều cao phần khối gỗ ngập trong nước.
\end{baitoan}

\begin{baitoan}[\cite{Van_Quyen_Hanh_Nhu_10_chuyen_Ly}, VD2, p. 56]
	1 quả cầu bằng đồng có khối lượng {\rm200 g} thể tích $\rm40\ cm^3$. Biết khối lượng riêng của đồng là $\rm8900\ kg{\tt/}m^3$, trọng lượng riêng của nước là $10^4\ N{\tt/}m^3$. (a) Quả cầu rỗng hay đặc? (b) Quả cầu khi thả vào nước nổi hay chìm?
\end{baitoan}

\begin{baitoan}[\cite{Van_Quyen_Hanh_Nhu_10_chuyen_Ly}, VD3, p. 56]
	1 miếng thép có 1 lỗ hổng ở bên trong. Dùng lực kế đo trọng lượng của miếng thép trong không khí thấy lực kế chỉ {\rm370 N}, khi miếng thép ở hoàn toàn trong nước lực kế chỉ {\rm320 N}. Xác định thể tích của lỗ hổng. Trọng lượng riêng của nước là $\rm10^4\ N{\tt/}m^3$, của thép là $\rm78\cdot 10^3\ N{\tt/}m^3$. Bỏ qua lực đẩy Acsimet do không khí tác dụng lên miếng thép.
\end{baitoan}

\begin{baitoan}[\cite{Van_Quyen_Hanh_Nhu_10_chuyen_Ly}, VD4, p. 57]
	(a) 1 khí cầu có thể tích $\rm10\ m^3$ chứa khí hydrogen, có thể kéo lên trên không 1 vật nặng bằng bao nhiêu? Biết khối lượng của vỏ khí cầu là {\rm10 kg}. Khối lượng riêng của không khí $D_{kk} = 1.29$ {\rm kg{\tt/}$\rm m^3$}, của hydrogen $D_{\rm H_2} = 0.09$ {\rm kg{\tt/}$\rm m^3$}. (b) Muốn kéo 1 người nặng {\rm60 kg} bay lên thì khí cầu phải có thể tích bằng bao nhiêu?
\end{baitoan}

\begin{baitoan}[\cite{Van_Quyen_Hanh_Nhu_10_chuyen_Ly}, VD5, pp. 57--58]
	1 chiếc vòng bằng hợp kim vàng \& bạc, khi cân trong không khí có trọng lượng $P_0 = 3$ {\rm N}. Khi cân trong nước, vòng có trọng lượng $P = 2.74$ {\rm N}. Xác định khối lượng phần vàng \& khối lượng phần bạc trong chiếc vòng nếu xem thể tích $V$ của vòng đúng bằng tổng thể tích ban đầu $V_1$ của vàng \& thể tích ban đầu $V_2$ của bạc. Khối lượng riêng của vàng là $\rm19300\ kg{\tt/}m^3$, của bạc là $\rm10500\ kg{\tt/}m^3$, của nước là $\rm1000\ kg{\tt/}m^3$.
\end{baitoan}

\begin{baitoan}[\cite{Van_Quyen_Hanh_Nhu_10_chuyen_Ly}, VD6, p. 58]
	1 khối gỗ hình trụ tiết diện $S = 200\ {\rm cm^2}$, chiều cao $h = 25$ {\rm cm} có trọng lượng riêng $d_0 = 9000$ {\rm N{\tt/}$\rm m^3$} được thả nổi thẳng đứng trong nước sao cho đáy song song với mặt thoáng. Trọng lượng riêng của nước là $d_1 = 10000$ {\rm N{\tt/}$\rm m^3$}. (a) Tính chiều cao của khối gỗ ngập trong nước. (b) Đổ vào phía trên 1 lớp dầu sao cho dầu vừa ngập khối gỗ. Tính chiều cao lớp dầu \& chiều cao phần gỗ ngập trong nước lúc này. Biết trọng lượng riêng của dầu là $d_2 = 8000$ {\rm N{\tt/}$\rm m^3$}.
\end{baitoan}

\begin{baitoan}[\cite{Van_Quyen_Hanh_Nhu_10_chuyen_Ly}, VD7, p. 59]
	1 khối gỗ đặc hình trụ, tiết diện đáy $S = 300\ {\rm cm^2}$, chiều cao $h = 40$ {\rm cm}, có trọng lượng riêng $d = 6000$ {\rm N{\tt/}$\rm m^3$} được giữ ngập trong 1 bể nước đến độ sâu $x = 40$ {\rm cm} bằng 1 sợi dây mảnh nhẹ, không giãn (mặt đáy song song với mặt thoáng nước). Cho biết trọng lượng riêng của nước là $d_{\rm H_2O} = 10^4$ {\rm N{\tt/}$\rm m^3$}. (a) Tính lực căng sợi dây. (b) Nếu dây bị đứt khối gỗ sẽ chuyển động thế nào?
\end{baitoan}

\begin{baitoan}[\cite{Van_Quyen_Hanh_Nhu_10_chuyen_Ly}, VD8, p. 60]
	Thả 1 khối sắt hình lập phương, cạnh $a = 10$ {\rm cm} vào 1 bể hình hộp chữ nhật, đáy nằm ngang, vật chìm hoàn toàn trong bể. Tính lực khối sắt đè lên đáy bể. Cho trọng lượng riêng của sắt là $d_1 = 78000$ {\rm N{\tt/}$\rm m^3$}, của nước là $d_2 = 10^4$ {\rm N{\tt/}$\rm m^3$}. Bỏ qua sự thay đổi của mực nước trong bể.
\end{baitoan}

\begin{baitoan}[\cite{Van_Quyen_Hanh_Nhu_10_chuyen_Ly}, 1., p. 63]
	1 khối gỗ hình hộp chữ nhật tiết diện $S = 40\ {\rm cm}^2$ cao $h = 10$ {\rm cm}. Có khối lượng $m = 160$ {\rm g}. (a) Thả khối gỗ vào nước. Tìm chiều cao của phần gỗ nổi trên mặt nước. Cho khối lượng riêng của nước là $D_{\rm H_2O} = 1000$ {\rm kg{\tt/}$\rm m^3$}. (b) Bây giờ khối gỗ được khoét 1 lỗ hình trụ ở giữa có tiết diện $\Delta S = 4\ {\rm cm}^2$, sâu $\Delta h$ \& lấp đầy chì có khối lượng riêng $D_{\rm Pb} = 11300$ {\rm kg{\tt/}$\rm m^3$} khi thả vào trong nước, thấy mực nước bằng với mặt trên của khối gỗ. Tìm độ sâu $\Delta h$ của lỗ.
\end{baitoan}

\begin{baitoan}[\cite{Van_Quyen_Hanh_Nhu_10_chuyen_Ly}, 2., p. 63]
	2 quả cầu đặc có thể tích mỗi quả là $V = 100\ {\rm m}^3$ được nối với nhau bằng 1 sợi dây nhẹ không co dãn thả trong nước. Khối lượng quả cầu bên dưới gấp $4$ lần khối lượng quả cầu bên trên. Khi cân bằng thì $\frac{1}{2}$ thể tích quả cầu bên trên bị ngập trong nước. Cho biết khối lượng riêng của nước là $D_{\rm H_2O} = 1000$ {\rm kg{\tt/}$\rm m^3$}. Tính: (a) Khối lượng riêng của các quả cầu. (b) Lực căng của sợi dây.
\end{baitoan}

\begin{baitoan}[\cite{Van_Quyen_Hanh_Nhu_10_chuyen_Ly}, 3., p. 64]
	Trong bình hình trụ tiết diện $S_0$ chứa nước, mực nước trong bình có chiều cao $H = 20$ {\rm cm}. Thả vào trong bình 1 thanh đồng chất, tiết diện đều sao cho nó nổi thẳng đứng trong bình thì mực nước dâng lên 1 đoạn $\Delta h = 4$ {\rm cm}. (a) Nếu nhấn chiumf thanh trong nước hoàn toàn thì mực nước sẽ dâng cao bao nhiêu so với đáy? Cho khối lượng riêng của thanh \& nước lần lượt là $D = 0.8$ {\rm g{\tt/}$\rm cm^3$}, $D_{\rm H_2O} = 1$ {\rm g{\tt/}$\rm cm^3$}. (b) Nếu dùng lực kế để đo trọng lượng của thanh, khi thanh chìm hoàn toán trong nước thì lực kế chỉ bao nhiêu. Cho thể tích thanh là $50\ {\rm cm}^3$.
\end{baitoan}

\begin{baitoan}[\cite{Van_Quyen_Hanh_Nhu_10_chuyen_Ly}, 4., p. 64]
	Trên đĩa cân bên trái có 1 bình chứa nước, bên phải là giá đỡ có treo vật A bằng sợi dây mảnh nhẹ. Khi quả nặng chưa chạm nước cân ở vị trí cân bằng. Nối dài sợi dây để vật A chìm hoàn toàn trong nước, trạng thái cân bằng của vật bị phá vỡ. Hỏi phải đặt 1 quả cân có trọng lượng bao nhiêu vào đĩa cân nào, để 2 đĩa cân bằng được trở lại. Cho thể tích vật A bằng $V = 100\ {\rm cm}^3$. Trọng lượng riêng của nước bằng $d = 10^4$ {\rm N{\tt/}$\rm m^3$}.
\end{baitoan}

\begin{baitoan}[\cite{Van_Quyen_Hanh_Nhu_10_chuyen_Ly}, 5., p. 64]
	1 khối nước đá hình lập phương mỗi cạnh {\rm10 cm} nổi trên mặt nước trong 1 bình thủy tinh. Phần nho lên mặt nước có chiều cao {\rm 1cm}. Biết trọng lượng riêng của nước là $d = 10^4$ {\rm N{\tt/}$\rm m^3$}. (a) Tính khối lượng riêng của nước đá. (b) Nếu nước đá tan hết thành nước thì mực nước trong bình có thay đổi không?
\end{baitoan}

\begin{baitoan}[\cite{Van_Quyen_Hanh_Nhu_10_chuyen_Ly}, 6., p. 64]
	1 cốc nhje có đặt quả cầu nhỏ trong bình chứa nước. Mực nước có độ cao $h$ thay đổi ra sao nếu lấy quả cầu ra thả vào bình nước. Khảo sát 2 trường hợp: (a) Quả cầu bằng gỗ có khối lượng riêng bé hơn của nước. (b) Quả cầu bằng sắt có khối lượng riêng lớn hơn của nước.
\end{baitoan}

\begin{baitoan}[\cite{Van_Quyen_Hanh_Nhu_10_chuyen_Ly}, 7., p. 64]
	Có 1 tảng băng trôi trên biển. Phần nhô lên của tảng băng ước tính $25\cdot10^4\ {\rm m}^3$. Vậy thể tích phần chìm dưới nước biển là bao nhiêu? Cho biết khối lượng riêng của băng là {\rm909 kg{\tt/}$\rm m^3$} \& khối lượng riêng của nước biển là {\rm1050 kg{\tt/}$\rm m^3$}.
\end{baitoan}

\begin{baitoan}[\cite{Van_Quyen_Hanh_Nhu_10_chuyen_Ly}, 1., p. 64]
	Treo 1 vật nhỏ vào 1 lực kế \& đặt chúng trong không khí thấy lực kế chỉ $F = 12$ {\rm N}, nhưng khi nhúng chìm hoàn toàn vật trong nước thì lực kế chỉ $F' = 7$ {\rm N}. Cho khối lượng riêng của nước là {\rm1000 kg$\rm m^3$}. Tính thể tích \& trọng lượng riêng của vật.
\end{baitoan}

\begin{baitoan}[\cite{Van_Quyen_Hanh_Nhu_10_chuyen_Ly}, 2., p. 64]
	1 vật được móc vào lực kế để đo lực theo phương thẳng đứng. Khi vật ở trong không khí, lực kế chỉ {\rm4.8 N}. Khi vật chìm trong nước, lực kế chỉ {\rm3.6 N}. Biết trọng lượng riêng của nước là $10^4$ {\rm N{\tt/}$\rm m^3$}. Bỏ qua lực đẩy Acsimet của không khí. Tính thể tích của vật nặng.
\end{baitoan}

\begin{baitoan}[\cite{Van_Quyen_Hanh_Nhu_10_chuyen_Ly}, 3., p. 65]
	1 quả cầu bằng nhôm, ở ngoài không khí có trọng lượng là {\rm1.458 N}. Hỏi phải khoét bớt lõi quả cầu 1 thể tích bằng bao nhiêu rồi hàn kín lại, để khi thả vào nước quả cầu nằm lơ lửng trong nước? Biết trọng lượng riêng của nước \& nhôm lần lượt là {\rm10000 N{\tt/}$\rm m^3$, 27000 N{\tt/}$\rm m^3$}.
\end{baitoan}

\begin{baitoan}[\cite{Van_Quyen_Hanh_Nhu_10_chuyen_Ly}, 4., p. 65]
	Treo 1 vật ở ngoài không khí vào lực kế, lực kế chỉ {\rm2.1 N}. Nhúng chìm vật đó vào nước thì chỉ số của lực kế giảm {\rm0.2 N}. Hỏi chất làm vật đó có trọng lượng riêng gấp bao nhiêu lần trọng lượng riêng của nước? Biết trọng lượng riêng của nước là {\rm10000 N{\tt/}$\rm m^3$}.
\end{baitoan}

\begin{baitoan}[\cite{Van_Quyen_Hanh_Nhu_10_chuyen_Ly}, 5., p. 65]
	$\rm1\ cm^3$ nhôm có trọng lượng riêng {\rm27000 N{\tt/}$\rm m^3$} \& $\rm1\ cm^3$ chì có trọng lượng riêng {\rm130000 N{\tt/}$\rm m^3$} được thả vào 1 bể nước. Lực đẩy tác dụng lên khối nào lớn hơn?
\end{baitoan}

\begin{baitoan}[\cite{Van_Quyen_Hanh_Nhu_10_chuyen_Ly}, 6., p. 65]
	{\rm 1 kg} nhôm có trọng lượng riêng {\rm27000 N{\tt/}$\rm m^3$} \& {\rm1 kg} đồng có trọng lượng riêng {\rm89000 N{\tt/}$\rm m^3$} được thả vào 1 bể nước. Lực đẩy tác dụng lên khối nào lớn hơn?
\end{baitoan}

\begin{baitoan}[\cite{Van_Quyen_Hanh_Nhu_10_chuyen_Ly}, 7., p. 65]
	1 quả cầu bằng sắt treo vào 1 lực kế ở ngoài không khí lực kế chỉ {\rm1.7 N}. Nhúng chìm quả cầu vào nước thì lực kế chỉ {\rm1.2 N}. Tính độ lớn của lực đẩy Acsimet.
\end{baitoan}

\begin{baitoan}[\cite{Van_Quyen_Hanh_Nhu_10_chuyen_Ly}, 8., p. 65]
	1 quả cầu bằng đồng được treo vào lực kế ở ngoài không khí thì lực kế chỉ {\rm4.45 N}. Nhúng chìm quả cầu vào rượu thì lực kế chỉ bao nhiêu? Biết $d_{\rm alcohol} = 8000$ {\rm N{\tt/}$\rm m^3$}, $d_{\rm Cu} = 89000$ {\rm N{\tt/}$\rm m^3$}.
\end{baitoan}

\begin{baitoan}[\cite{Van_Quyen_Hanh_Nhu_10_chuyen_Ly}, 9., p. 65]
	1 quả càu bằng sắt có thể tích $\rm4\ dm^3$ được nhúng chìm trong nước, biết khối lượng riêng của nước là {\rm1000 kg{\tt/}$\rm m^3$}. Tính độ lớn lực đẩy Acsimet tác dụng lên quả cầu.
\end{baitoan}

%------------------------------------------------------------------------------%

\section{Miscellaneous}

%------------------------------------------------------------------------------%

\printbibliography[heading=bibintoc]
	
\end{document}