\documentclass{article}
\usepackage[backend=biber,natbib=true,style=alphabetic,maxbibnames=50]{biblatex}
\addbibresource{/home/nqbh/reference/bib.bib}
\usepackage[utf8]{vietnam}
\usepackage{tocloft}
\renewcommand{\cftsecleader}{\cftdotfill{\cftdotsep}}
\usepackage[colorlinks=true,linkcolor=blue,urlcolor=red,citecolor=magenta]{hyperref}
\usepackage{amsmath,amssymb,amsthm,float,graphicx,mathtools,tikz}
\usetikzlibrary{angles,calc,intersections,matrix,patterns,quotes,shadings}
\allowdisplaybreaks
\newtheorem{assumption}{Assumption}
\newtheorem{baitoan}{}
\newtheorem{cauhoi}{Câu hỏi}
\newtheorem{conjecture}{Conjecture}
\newtheorem{corollary}{Corollary}
\newtheorem{dangtoan}{Dạng toán}
\newtheorem{definition}{Definition}
\newtheorem{dinhly}{Định lý}
\newtheorem{dinhnghia}{Định nghĩa}
\newtheorem{example}{Example}
\newtheorem{ghichu}{Ghi chú}
\newtheorem{hequa}{Hệ quả}
\newtheorem{hypothesis}{Hypothesis}
\newtheorem{lemma}{Lemma}
\newtheorem{luuy}{Lưu ý}
\newtheorem{nhanxet}{Nhận xét}
\newtheorem{notation}{Notation}
\newtheorem{note}{Note}
\newtheorem{principle}{Principle}
\newtheorem{problem}{Problem}
\newtheorem{proposition}{Proposition}
\newtheorem{question}{Question}
\newtheorem{remark}{Remark}
\newtheorem{theorem}{Theorem}
\newtheorem{vidu}{Ví dụ}
\usepackage[left=1cm,right=1cm,top=5mm,bottom=5mm,footskip=4mm]{geometry}
\def\labelitemii{$\circ$}
\DeclareRobustCommand{\divby}{%
	\mathrel{\vbox{\baselineskip.65ex\lineskiplimit0pt\hbox{.}\hbox{.}\hbox{.}}}%
}
\def\labelitemii{$\circ$}

\title{Problem: Circular Motion -- Bài Tập: Chuyển Động Tròn}
\author{Nguyễn Quản Bá Hồng\footnote{A Scientist {\it\&} Creative Artist Wannabe. E-mail: {\tt nguyenquanbahong@gmail.com}. Bến Tre City, Việt Nam.}}
\date{\today}

\begin{document}
\maketitle
\begin{abstract}
	This text is a part of the series {\it Some Topics in Elementary STEM \& Beyond}:
	
	{\sc url}: \url{https://nqbh.github.io/elementary_STEM}.
	
	Latest version:
	\begin{itemize}
		\item {\it Problem: Circular Motion -- Bài Tập: Chuyển Động Tròn}.
		
		PDF: {\sc url}: \url{https://github.com/NQBH/elementary_STEM_beyond/blob/main/elementary_physics/grade_10/circular_motion/problem/NQBH_circular_motion_problem.pdf}.
		
		\TeX: {\sc url}: \url{https://github.com/NQBH/elementary_STEM_beyond/blob/main/elementary_physics/grade_10/circular_motion/problem/NQBH_circular_motion_problem.tex}.
		\item {\it Problem \& : Circular Motion -- Bài Tập \& Lời Giải: Chuyển Động Tròn}.
		
		PDF: {\sc url}: \url{https://github.com/NQBH/elementary_STEM_beyond/blob/main/elementary_physics/grade_10/circular_motion/solution/NQBH_circular_motion_solution.pdf}.
		
		\TeX: {\sc url}: \url{https://github.com/NQBH/elementary_STEM_beyond/blob/main/elementary_physics/grade_10/circular_motion/solution/NQBH_circular_motion_solution.tex}.
	\end{itemize}
\end{abstract}
\tableofcontents

%------------------------------------------------------------------------------%

\section{Basic}
\fbox{1} 1 vật chuyển động trên cung tròn có chiều dài $s$ m, trong thời gian $t$ s, với $r$ m là bán kính của cung tròn thì độ dịch chuyển góc $\theta$ rad: $\theta = \frac{s}{r}$ với {\it tốc độ dài} $v = \frac{s}{t}$ m{\tt/}s, {\it tốc độ góc} $\omega = \frac{\theta}{t} = \frac{v}{r}$ rad{\tt/}s, {\it chu kỳ} $T$ s là khoảng thời gian để vật đi được hết 1 vòng quỹ đạo, {\it tần số} $f$ Hz là số vòng quỹ đạo mà vật đi được trong 1 s: $T = \frac{1}{f} = \frac{2\pi}{\omega},f = \frac{1}{T}  =\frac{\omega}{2\pi}$. \fbox{2} {\it Chuyển động tròn đều} là chuyển động có quỹ đạo là đường tròn \& tốc độ không đổi theo thời gian. Vận tốc tức thời có độ lớn không đổi, tại mỗi điểm, vector vận tốc có phương tiếp tuyến với quỹ đạo tại điểm đó. \fbox{3} 1 vật có khối lượng $m$ kg chuyển động tròn đều trên quỹ đạo có bán kính $r$ m, tốc độ dài $v$ m{\tt/}s thì {\it gia tốc hướng tâm} $a_{\rm ht} = \frac{v^2}{r}\ {\rm m{\tt/}s^2}$, {\it lực hướng tâm} $F_{\rm ht} = ma_{\rm ht} = \frac{mv^2}{r}$.

%------------------------------------------------------------------------------%

\section{Miscellaneous}

%------------------------------------------------------------------------------%

\printbibliography[heading=bibintoc]
	
\end{document}