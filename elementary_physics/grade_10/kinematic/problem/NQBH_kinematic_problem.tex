\documentclass{article}
\usepackage[backend=biber,natbib=true,style=alphabetic,maxbibnames=50]{biblatex}
\addbibresource{/home/nqbh/reference/bib.bib}
\usepackage[utf8]{vietnam}
\usepackage{tocloft}
\renewcommand{\cftsecleader}{\cftdotfill{\cftdotsep}}
\usepackage[colorlinks=true,linkcolor=blue,urlcolor=red,citecolor=magenta]{hyperref}
\usepackage{amsmath,amssymb,amsthm,float,graphicx,mathtools,tikz}
\usetikzlibrary{angles,calc,intersections,matrix,patterns,quotes,shadings}
\allowdisplaybreaks
\newtheorem{assumption}{Assumption}
\newtheorem{baitoan}{}
\newtheorem{cauhoi}{Câu hỏi}
\newtheorem{conjecture}{Conjecture}
\newtheorem{corollary}{Corollary}
\newtheorem{dangtoan}{Dạng toán}
\newtheorem{definition}{Definition}
\newtheorem{dinhly}{Định lý}
\newtheorem{dinhnghia}{Định nghĩa}
\newtheorem{example}{Example}
\newtheorem{ghichu}{Ghi chú}
\newtheorem{hequa}{Hệ quả}
\newtheorem{hypothesis}{Hypothesis}
\newtheorem{lemma}{Lemma}
\newtheorem{luuy}{Lưu ý}
\newtheorem{nhanxet}{Nhận xét}
\newtheorem{notation}{Notation}
\newtheorem{note}{Note}
\newtheorem{principle}{Principle}
\newtheorem{problem}{Problem}
\newtheorem{proposition}{Proposition}
\newtheorem{question}{Question}
\newtheorem{remark}{Remark}
\newtheorem{theorem}{Theorem}
\newtheorem{vidu}{Ví dụ}
\usepackage[left=1cm,right=1cm,top=5mm,bottom=5mm,footskip=4mm]{geometry}
\def\labelitemii{$\circ$}
\DeclareRobustCommand{\divby}{%
	\mathrel{\vbox{\baselineskip.65ex\lineskiplimit0pt\hbox{.}\hbox{.}\hbox{.}}}%
}

\title{Problem: Kinematic -- Bài Tập: Chuyển Động Học}
\author{Nguyễn Quản Bá Hồng\footnote{A Scientist {\it\&} Creative Artist Wannabe. E-mail: {\tt nguyenquanbahong@gmail.com}. Bến Tre City, Việt Nam.}}
\date{\today}

\begin{document}
\maketitle
\begin{abstract}
	This text is a part of the series {\it Some Topics in Elementary STEM \& Beyond}:
	
	{\sc url}: \url{https://nqbh.github.io/elementary_STEM}.
	
	Latest version:
	\begin{itemize}
		\item {\it Problem: Kinematic -- Bài Tập: Chuyển Động Học}.
		
		PDF: {\sc url}: \url{https://github.com/NQBH/elementary_STEM_beyond/blob/main/elementary_physics/grade_10/kinematic/problem/NQBH_kinematic_problem.pdf}.
		
		\TeX: {\sc url}: \url{https://github.com/NQBH/elementary_STEM_beyond/blob/main/elementary_physics/grade_10/kinematic/problem/NQBH_kinematic_problem.tex}.
		\item {\it Problem \& Solution: Kinematic -- Bài Tập \& Lời Giải: Chuyển Động Học}.
		
		PDF: {\sc url}: \url{https://github.com/NQBH/elementary_STEM_beyond/blob/main/elementary_physics/grade_10/kinematic/solution/NQBH_kinematic_solution.pdf}.
		
		\TeX: {\sc url}: \url{https://github.com/NQBH/elementary_STEM_beyond/blob/main/elementary_physics/grade_10/kinematic/solution/NQBH_kinematic_solution.tex}.
	\end{itemize}
\end{abstract}
\tableofcontents

%------------------------------------------------------------------------------%

\section{Basic}
\textbf{\textsf{Resources -- Tài nguyên.}}
\begin{itemize}
	\item \cite{Giang_Hang_Trung_ncpt_Vat_Ly_10}. {\sc Tô Giang, Trần Thúy Hằng, Lê Minh Trung}. {\it Nâng \& Phát Triển Vật Lý 10}.
\end{itemize}
\fbox{1} {\sf Độ dịch chuyển \& quãng đường đi được.} {\it Độ dịch chuyển} là 1 đại lượng vector, cho biết độ dài \& hướng của sự thay đổi vị trí của vật. Khi vật chuyển động thẳng, không đổi chiều thì độ lớn của độ dịch chuyển \& quãng đường đi được bằng nhau. Khi vật chuyển động thẳng, có đổi chiều thì quãng đường đi được \& độ dịch chuyển có độ lớn không bằng nhau. Tổng hợp các độ dịch chuyển bằng các tổng hợp vector. \fbox{2} {\sf Tốc độ \& vận tốc.} 1 vật chuyển động từ điểm A đến điểm B với $s$: quãng đường đi được, $\vec{d}$: độ dịch chuyển, $t$: khoảng thời gian vật đi từ A đến B, thì {\it tốc độ trung bình} $v = \frac{s}{t}$, {\it vận tốc trung bình} $\vec{v} = \frac{\vec{d}}{t}$. \fbox{3} {\sf Đồ thị dịch chuyển--thời gian.} Dùng đồ thị dịch chuyển--thời gian của chuyển động thẳng có thể mô tả được chuyển động: biết khi nào vật chuyển động, khi nào vật dừng lại, khi nào vật chuyển động nhanh, khi nào vật chuyển động chậm, khi nào vật đổi chiều chuyển động, $\ldots$ Vận tốc có giá trị bằng hệ số góc (độ dốc) của đường biểu diễn trong đồ thị độ dịch chuyển--thời gian của chuyển động thẳng. \fbox{4} {\sf Chuyển động biến đổi. Gia tốc.} {\it Gia tốc} là đại lượng cho biết mức độ nhanh hay chậm của sự thay đổi vận tốc: $\vec{a} = \frac{\overrightarrow{\Delta v}}{\Delta t}$. Khi $\vec{a}$ cùng chiều với $\vec{v}$, i.e., $\vec{a}\cdot\vec{v} > 0$: chuyển động nhanh dần; Khi $\vec{a}$ ngược chiều với $\vec{v}$, i.e., $\vec{a}\cdot\vec{v} < 0$: chuyển động chậm dần. Đơn vị của gia tốc trong hệ SI là $\rm m{\tt/}s^2$. \fbox{5} {\sf Chuyển động thẳng biến đổi đều} là chuyển động thẳng có gia tốc không đổi theo thời gian, i.e., $\vec{a} = \vec{a}_0 = {\rm const}$. Chuyển động thẳng nhanh dần đều có $\vec{a}\cdot\vec{v} > 0$, chuyển động thẳng chậm dần đều có $\vec{a}\cdot\vec{v} < 0$. Hệ số góc của đồ thị vận tốc--thời gian của chuyển động thẳng biến đổi đều cho biết giá trị của gia tốc. Các công thức của chuyển động thẳng biến đổi đều: $v = v_0 + at$, $d = v_0t + \frac{1}{2}at^2$, $v^2 - v_0^2 = 2ad$. \fbox{6} {\sf Sự rơi tự do.} {\it Chuyển động tự do} là chuyển động rơi dưới tác dụng của trọng lực. Quãng đường rơi tỷ lệ thuận với bình phương thời gian rơi.  \fbox{7} {\sf Chuyển động ném ngang.} 1 vật được ném theo phương ngang từ vị trí có độ cao $H$ \& vận tốc ban đầu $v_0$. Gia tốc rơi tự do tại nơi ném vật là $g$. Thời gian chuyển động $\Delta t = \sqrt{\frac{2H}{g}}$. {\it Tầm bay xa} là khoảng cách từ vị trí chân đường vuông góc hạ từ điểm ném xuống mặt đất, tới vị trí vật chạm đất $L = v_0t = v_0\sqrt{\frac{2H}{g}}$. Chọn mốc thời gian là lúc ném, độ dịch chuyển của vật tại thời điểm $t$ có: độ lớn $d = \sqrt{d_x^2 + d_y^2}$ trong đó $d_x = v_0t$: độ dịch chuyển theo phương ngang, $d_y = \frac{1}{2}gt^2$: độ dịch chuyển theo phương thẳng đứng, $\tan\alpha = \frac{d_y}{d_x} = \frac{g}{2v}t$: hướng lệch so với hướng ném góc $\alpha$. Vận tốc tại thời điểm $t$ có độ lớn $v = \sqrt{v_x^2 + v_y^2}$ trong đó $v_x = v$: thành phần theo phương ngang, $v_y = gt$: thành phần theo phương thẳng đứng, $\tan\beta = \frac{v_y}{v_x} = \frac{g}{v_0}t$: hướng lệch so với hướng ném góc $\beta$. \fbox{7} {\sf Chuyển động ném xiên.} Vật được ném lên từ mặt đất với vận tốc $v_0$, hướng ném hợp với phương ngang 1 góc $\alpha$. Bỏ qua sức cản của không khí, gia tốc trọng trường là $g$ thì độ cao cực đại: $H = \frac{v_0^2}{2g}\sin^2\alpha$, tầm bay xa $L = v_0t\cos\alpha = \frac{v_0^2}{g}\sin2\alpha$, thời gian chuyển động $\Delta t = \frac{2v_0\sin\alpha}{g}$. Chọn mốc thời gian là lúc ném, độ dịch chuyển tại thời điểm $t$ được xác định với độ lớn $d = \sqrt{d_x^2 + d_y^2}$ theo phương ngang $d_x = v_0t\cos\alpha$, theo phương thẳng đứng $d_y = v_0t\sin\alpha - \frac{1}{2}gt^2$, hướng của độ dịch chuyển hợp với phương ngang $\tan\beta = \frac{d_y}{d_x}$. Vận tốc của vật tại thời điểm $t$ có độ lớn $v = \sqrt{v_x^2 + v_y^2}$ theo phương ngang $v_x = v_0\cos\alpha$, theo phương thẳng đứng $v_y = v_0\sin\alpha - gt$, hướng của độ dịch chuyển hợp với phương ngang $\tan\gamma = \frac{v_y}{v_x}$, thời điểm vật đạt độ cao cực đại là thời điểm $v_y = 0$: $t_H = \frac{v_0\sin\alpha}{g}$.

\begin{baitoan}[\cite{Giang_Hang_Trung_ncpt_Vat_Ly_10}, 1.1., p. 8]
	(a) 1 con kiến bò trên miệng của 1 chiếc bát ăn cơm, bán kính $r$, từ điểm $A(0,-r)$ đến điểm $B(r,0)$. Xác định độ dịch chuyển \& quãng đường đi được của con kiến. (b) Mở rộng bài toán cho 2 điểm bất kỳ $A,B\in S_2((x_0,y_0),r)\subset\mathbb{R}^2$ với $S_2((x_0,y_0),r)$ là đường tròn tâm $(x_0,y_0)\in\mathbb{R}^2$ bán kính $r > 0$ trong mặt phẳng tọa độ $Oxy$. (c) Mở rộng hình tròn thành 3 đường conic: elipse, hyperbol, parabol.
\end{baitoan}

\begin{baitoan}[\cite{Giang_Hang_Trung_ncpt_Vat_Ly_10}, 1.2., p. 8]
	1 vật chuyển động dọc theo 3 cạnh của $\Delta ABC$ vuông tại B với $AB = 3$ {\rm m}, $AC = 6$ {\rm m}, bắt đầu từ điểm A tới điểm B tại thời điểm $t_1 = 10$ {\rm s}, tới điểm C tại thời điểm $t_2 = 30$ {\rm s}. Xác định độ dịch chuyển \& quãng đường đi được của vật tại 2 thời điểm $t_1,t_2$ \& trong khoảng thời gian từ điểm $t_1$ đến thời điểm $t_2$.
\end{baitoan}

%\begin{baitoan}[\cite{Giang_Hang_Trung_ncpt_Vat_Ly_10}, 1.., p. 8]
%	
%\end{baitoan}

%------------------------------------------------------------------------------%

\section{Miscellaneous}

%------------------------------------------------------------------------------%

\printbibliography[heading=bibintoc]
	
\end{document}