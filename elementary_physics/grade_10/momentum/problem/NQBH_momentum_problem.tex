\documentclass{article}
\usepackage[backend=biber,natbib=true,style=alphabetic,maxbibnames=50]{biblatex}
\addbibresource{/home/nqbh/reference/bib.bib}
\usepackage[utf8]{vietnam}
\usepackage{tocloft}
\renewcommand{\cftsecleader}{\cftdotfill{\cftdotsep}}
\usepackage[colorlinks=true,linkcolor=blue,urlcolor=red,citecolor=magenta]{hyperref}
\usepackage{amsmath,amssymb,amsthm,float,graphicx,mathtools,tikz}
\usetikzlibrary{angles,calc,intersections,matrix,patterns,quotes,shadings}
\allowdisplaybreaks
\newtheorem{assumption}{Assumption}
\newtheorem{baitoan}{}
\newtheorem{cauhoi}{Câu hỏi}
\newtheorem{conjecture}{Conjecture}
\newtheorem{corollary}{Corollary}
\newtheorem{dangtoan}{Dạng toán}
\newtheorem{definition}{Definition}
\newtheorem{dinhly}{Định lý}
\newtheorem{dinhnghia}{Định nghĩa}
\newtheorem{example}{Example}
\newtheorem{ghichu}{Ghi chú}
\newtheorem{hequa}{Hệ quả}
\newtheorem{hypothesis}{Hypothesis}
\newtheorem{lemma}{Lemma}
\newtheorem{luuy}{Lưu ý}
\newtheorem{nhanxet}{Nhận xét}
\newtheorem{notation}{Notation}
\newtheorem{note}{Note}
\newtheorem{principle}{Principle}
\newtheorem{problem}{Problem}
\newtheorem{proposition}{Proposition}
\newtheorem{question}{Question}
\newtheorem{remark}{Remark}
\newtheorem{theorem}{Theorem}
\newtheorem{vidu}{Ví dụ}
\usepackage[left=1cm,right=1cm,top=5mm,bottom=5mm,footskip=4mm]{geometry}
\def\labelitemii{$\circ$}
\DeclareRobustCommand{\divby}{%
	\mathrel{\vbox{\baselineskip.65ex\lineskiplimit0pt\hbox{.}\hbox{.}\hbox{.}}}%
}
\def\labelitemii{$\circ$}

\title{Problem: Momentum -- Bài Tập: Động Lượng}
\author{Nguyễn Quản Bá Hồng\footnote{A Scientist {\it\&} Creative Artist Wannabe. E-mail: {\tt nguyenquanbahong@gmail.com}. Bến Tre City, Việt Nam.}}
\date{\today}

\begin{document}
\maketitle
\begin{abstract}
	This text is a part of the series {\it Some Topics in Elementary STEM \& Beyond}:
	
	{\sc url}: \url{https://nqbh.github.io/elementary_STEM}.
	
	Latest version:
	\begin{itemize}
		\item {\it Problem: Momentum -- Bài Tập: Động Lượng}.
		
		PDF: {\sc url}: \url{https://github.com/NQBH/elementary_STEM_beyond/blob/main/elementary_physics/grade_10/momentum/problem/NQBH_momentum_problem.pdf}.
		
		\TeX: {\sc url}: \url{https://github.com/NQBH/elementary_STEM_beyond/blob/main/elementary_physics/grade_10/momentum/problem/NQBH_momentum_problem.tex}.
		\item {\it Problem \& Solution: Momentum -- Bài Tập \& Lời Giải: Động Lượng}.
		
		PDF: {\sc url}: \url{https://github.com/NQBH/elementary_STEM_beyond/blob/main/elementary_physics/grade_10/momentum/solution/NQBH_momentum_solution.pdf}.
		
		\TeX: {\sc url}: \url{https://github.com/NQBH/elementary_STEM_beyond/blob/main/elementary_physics/grade_10/momentum/solution/NQBH_momentum_solution.tex}.
	\end{itemize}
\end{abstract}
\tableofcontents

%------------------------------------------------------------------------------%

\section{Basic}
\fbox{1} {\it Động lượng} của 1 vật có khối lượng $m$ đang chuyển động với vận tốc $\vec{v}$ là đại lượng được xác định bởi công thức $\vec{p} = m\vec{v}$. Động lượng là 1 vector có cùng hướng với vận tốc của vật, chỉ sai khác tỷ lệ độ dài giữa 2 vector đúng bằng $m$. Động lượng có đơn vị đo là $\rm kg\cdot m{\tt/}s$. Động lượng là đại lượng đặc trưng cho sự truyền tương tác giữa các vật. Tích $\vec{F}\Delta t$ được gọi là {\it xung lượng} của lực tác dụng trong khoảng thời gian ngắn $\Delta t$ \& bằng độ biến thiên động lượng của vật trong thời gian đó $\vec{F}\Delta t = \Delta\vec{p}$. \fbox{2} {\it Hệ kín} là hệ không chịu tác dụng của các ngoại lực từ phía các lực ở ngoài hệ (hoặc nếu có thì các lực này phải triệt tiêu nhau): $\sum \vec{F}_i = \vec{0}$. Hệ coi gần đúng là kín (trong thời gian có nội lực tương tác) nếu $F_{\footnotesize\mbox{\rm ngoại}}\ll F_{\footnotesize\mbox{\rm nội}}$ or $F_{\rm ext}\ll F_{\rm int}$ (external forces vs. internal forces -- các ngoại lực vs. các nội lực). \fbox{3} {\it Định luật bảo toàn động lượng}: Động lượng toàn phần của hệ kín là 1 đại lượng bảo toàn: $\sum \vec{p} = \sum \vec{p}'$ với $\sum \vec{p}$: động lượng của hệ lúc đầu, $\sum \vec{p}'$: động lượng của hệ lúc sau. Nếu $\vec{F}_{\footnotesize\mbox{\rm ngoại}}\ne\vec{0}$ nhưng hình chiếu của $\vec{F}_{\footnotesize\mbox{\rm ngoại}}$ trên 1 phương nào đó triệt tiêu thì động lượng của hệ được bảo toàn trên phương đó. \fbox{4} {\sf Tương tác của 2 vật trong hệ kín.} Đặt $m_i,\vec{v}_i,\vec{v}_i'$ lần lượt là khối lượng, vận tốc trước tương tác, vận tốc sau tương tác của vật thứ $i$, $i = 1,2$ thì $m_1\vec{v}_1 + m_2\vec{v}_2 = m_1\vec{v}_1' + m_2\vec{v}_2'$. 2 kiểu va chạm thường gặp: {\it va chạm đàn hồi} ( sau va chạm 2 vật chuyển động với vận tốc khác nhau); {\it va chạm mềm} (sau va chạm 2 vật dính vào nhau \& chuyển động với cùng vận tốc). Trong hệ kín gồm nhiều vật, các vật của hệ có thể chuyển động có gia tốc nhưng khối tâm của hệ đứng yên hoặc chuyển động thẳng đều. \fbox{5} {\sf Chuyển động bằng phản lực.} Trong 1 hệ kín đứng yên, nếu có 1 phần của hệ chuyển động theo 1 hướng thì theo định luật bảo toàn động lượng, phần còn lại của hệ phải chuyển động theo hướng ngược lại. Chuyển động này được gọi là {\it chuyển động bằng phản lực}.

%------------------------------------------------------------------------------%

\section{Miscellaneous}

%------------------------------------------------------------------------------%

\printbibliography[heading=bibintoc]
	
\end{document}