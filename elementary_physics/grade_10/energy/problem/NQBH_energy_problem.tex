\documentclass{article}
\usepackage[backend=biber,natbib=true,style=alphabetic,maxbibnames=50]{biblatex}
\addbibresource{/home/nqbh/reference/bib.bib}
\usepackage[utf8]{vietnam}
\usepackage{tocloft}
\renewcommand{\cftsecleader}{\cftdotfill{\cftdotsep}}
\usepackage[colorlinks=true,linkcolor=blue,urlcolor=red,citecolor=magenta]{hyperref}
\usepackage{amsmath,amssymb,amsthm,float,graphicx,mathtools,tikz}
\usetikzlibrary{angles,calc,intersections,matrix,patterns,quotes,shadings}
\allowdisplaybreaks
\newtheorem{assumption}{Assumption}
\newtheorem{baitoan}{}
\newtheorem{cauhoi}{Câu hỏi}
\newtheorem{conjecture}{Conjecture}
\newtheorem{corollary}{Corollary}
\newtheorem{dangtoan}{Dạng toán}
\newtheorem{definition}{Definition}
\newtheorem{dinhly}{Định lý}
\newtheorem{dinhnghia}{Định nghĩa}
\newtheorem{example}{Example}
\newtheorem{ghichu}{Ghi chú}
\newtheorem{hequa}{Hệ quả}
\newtheorem{hypothesis}{Hypothesis}
\newtheorem{lemma}{Lemma}
\newtheorem{luuy}{Lưu ý}
\newtheorem{nhanxet}{Nhận xét}
\newtheorem{notation}{Notation}
\newtheorem{note}{Note}
\newtheorem{principle}{Principle}
\newtheorem{problem}{Problem}
\newtheorem{proposition}{Proposition}
\newtheorem{question}{Question}
\newtheorem{remark}{Remark}
\newtheorem{theorem}{Theorem}
\newtheorem{vidu}{Ví dụ}
\usepackage[left=1cm,right=1cm,top=5mm,bottom=5mm,footskip=4mm]{geometry}
\def\labelitemii{$\circ$}
\DeclareRobustCommand{\divby}{%
	\mathrel{\vbox{\baselineskip.65ex\lineskiplimit0pt\hbox{.}\hbox{.}\hbox{.}}}%
}
\def\labelitemii{$\circ$}

\title{Problem: Energy, Work, {\it\&} Productivity\\Bài Tập: Năng Lượng, Công, {\it\&} Công Suất}
\author{Nguyễn Quản Bá Hồng\footnote{A Scientist {\it\&} Creative Artist Wannabe. E-mail: {\tt nguyenquanbahong@gmail.com}. Bến Tre City, Việt Nam.}}
\date{\today}

\begin{document}
\maketitle
\begin{abstract}
	This text is a part of the series {\it Some Topics in Elementary STEM \& Beyond}:
	
	{\sc url}: \url{https://nqbh.github.io/elementary_STEM}.
	
	Latest version:
	\begin{itemize}
		\item {\it Problem: Energy, Work, {\it\&} Productivity -- Bài Tập: Năng Lượng, Công, {\it\&} Công Suất}.
		
		PDF: {\sc url}: \url{https://github.com/NQBH/elementary_STEM_beyond/blob/main/elementary_physics/grade_10/energy/problem/NQBH_energy_problem.pdf}.
		
		\TeX: {\sc url}: \url{https://github.com/NQBH/elementary_STEM_beyond/blob/main/elementary_physics/grade_10/energy/problem/NQBH_energy_problem.tex}.
		\item {\it Problem \& Solution: Energy, Work, {\it\&} Productivity -- Bài Tập \& Lời Giải: Năng Lượng, Công, {\it\&} Công Suất}.
		
		PDF: {\sc url}: \url{https://github.com/NQBH/elementary_STEM_beyond/blob/main/elementary_physics/grade_10/energy/solution/NQBH_energy_solution.pdf}.
		
		\TeX: {\sc url}: \url{https://github.com/NQBH/elementary_STEM_beyond/blob/main/elementary_physics/grade_10/energy/solution/NQBH_energy_solution.tex}.
	\end{itemize}
\end{abstract}
\tableofcontents

%------------------------------------------------------------------------------%

\section{Basic}
\fbox{1} {\sf Năng lượng. Công cơ học.} {\it Năng lượng} là đại lượng đặc trưng cho khả năng thực hiện công của 1 vật hoặc 1 hệ vật. Giá trị năng lượng của 1 vật (hoặc hệ vật) ở 1 trạng thái xác định nào đó bằng công lớn nhất mà vật (hoặc hệ vật) thực hiện được trong những điều kiện nhất định. Đơn vị của năng lượng là đơn vị của công. Lực $\vec{F}$ thực hiện 1 công cơ học khi nó tác dụng lên 1 vật \& làm vật di chuyển. Công thức $A = Fs\cos\alpha$. \fbox{2} {\sf Công suất. Hiệu suất.} Công suất $\mathcal{P} = \frac{A}{t}$. {\it Hiệu suất của động cơ} $H$ là tỷ số giữa công suất có ích \& công suất toàn phần của động cơ, đặc trưng cho hiệu quả làm việc của động cơ $H = \frac{\mathcal{P}'}{\mathcal{P}}\cdot100\%$ với $\Delta\mathcal{P} = \mathcal{P} - \mathcal{P}'$ được gọi là {\it công suất hao phí} của động cơ. Hiệu suất của động cơ còn có thể được tính theo công thức $H = \frac{A'}{A}\cdot100\%$ với $A,A'$ lần lượt là công có ích \& công toàn phần của động cơ. $\Delta A = A' - A$ được gọi là {\it công hao phí của động cơ}. Hệ thức liên hệ giữa lực \& công suất $\mathcal{P} = Fv$. \fbox{3} {\it Động năng} $W_{\footnotesize\mbox{\rm đ}} = \frac{1}{2}mv^2$. {\it Định lý về động năng}: Biến thiên động năng của 1 vật bằng công $A$ của ngoại lực tác dụng lên vật: $\Delta W_{\footnotesize\mbox{\rm đ}} = W_{\footnotesize\mbox{\rm đ}2} - W_{\footnotesize\mbox{\rm đ}1} = A$. \fbox{4} {\sf Thế năng.} {\it Thế năng hấp dẫn} $W_{\rm t} = mgh$. {\it Thế năng đàn hồi} $W_{\rm t} = \frac{1}{2}kx^2$. \fbox{5} {\sf Định luật bảo toàn cơ năng}: Hệ kín, nội lực ma sát không sinh công thì cơ năng hệ bảo toàn: $W = W_{\footnotesize\mbox{\rm đ}} + W_{\rm t} = {\rm const}$. Trong trường hợp có ngoại lực tác dụng lên hệ (sinh công $A_{\rm ngl}$ hay $A_{\rm extf}$, external force) hoặc nội lực ma sát sinh công (chuyển hóa thành nhiệt năng $Q$) thì cơ năng hệ vật sẽ biến thiên 1 lượng $\Delta W = W' - W = A_{\rm ngl} + Q$. {\sf 6} {\sf Sự va chạm của các vật.} {\it Va chạm đàn hồi}: Động lượng bảo toàn, động năng bảo toàn, độ lớn của vận tốc tương đối giữa 2 vật ngay sau va chạm bằng độ lớn của vận tốc tương đối giữa 2 vật ngay trước va chạm. {\it Va chạm mềm}: Động lượng bảo toàn, cơ năng giảm (chuyển hóa thành nhiệt năng), độ lớn của vận tốc tương đối giữa 2 vật sau va chạm bằng $0$.

%------------------------------------------------------------------------------%

\section{Miscellaneous}

%------------------------------------------------------------------------------%

\printbibliography[heading=bibintoc]
	
\end{document}