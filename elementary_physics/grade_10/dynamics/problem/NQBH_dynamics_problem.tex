\documentclass{article}
\usepackage[backend=biber,natbib=true,style=alphabetic,maxbibnames=50]{biblatex}
\addbibresource{/home/nqbh/reference/bib.bib}
\usepackage[utf8]{vietnam}
\usepackage{tocloft}
\renewcommand{\cftsecleader}{\cftdotfill{\cftdotsep}}
\usepackage[colorlinks=true,linkcolor=blue,urlcolor=red,citecolor=magenta]{hyperref}
\usepackage{amsmath,amssymb,amsthm,float,graphicx,mathtools,tikz}
\usetikzlibrary{angles,calc,intersections,matrix,patterns,quotes,shadings}
\allowdisplaybreaks
\newtheorem{assumption}{Assumption}
\newtheorem{baitoan}{}
\newtheorem{cauhoi}{Câu hỏi}
\newtheorem{conjecture}{Conjecture}
\newtheorem{corollary}{Corollary}
\newtheorem{dangtoan}{Dạng toán}
\newtheorem{definition}{Definition}
\newtheorem{dinhluat}{Định luật}
\newtheorem{dinhly}{Định lý}
\newtheorem{dinhnghia}{Định nghĩa}
\newtheorem{example}{Example}
\newtheorem{ghichu}{Ghi chú}
\newtheorem{hequa}{Hệ quả}
\newtheorem{hypothesis}{Hypothesis}
\newtheorem{lemma}{Lemma}
\newtheorem{luuy}{Lưu ý}
\newtheorem{nhanxet}{Nhận xét}
\newtheorem{notation}{Notation}
\newtheorem{note}{Note}
\newtheorem{principle}{Principle}
\newtheorem{problem}{Problem}
\newtheorem{proposition}{Proposition}
\newtheorem{question}{Question}
\newtheorem{remark}{Remark}
\newtheorem{theorem}{Theorem}
\newtheorem{vidu}{Ví dụ}
\usepackage[left=1cm,right=1cm,top=5mm,bottom=5mm,footskip=4mm]{geometry}
\def\labelitemii{$\circ$}
\DeclareRobustCommand{\divby}{%
	\mathrel{\vbox{\baselineskip.65ex\lineskiplimit0pt\hbox{.}\hbox{.}\hbox{.}}}%
}
\def\labelitemii{$\circ$}

\title{Problem: Dynamics -- Bài Tập: Động Lực Học}
\author{Nguyễn Quản Bá Hồng\footnote{A Scientist {\it\&} Creative Artist Wannabe. E-mail: {\tt nguyenquanbahong@gmail.com}. Bến Tre City, Việt Nam.}}
\date{\today}

\begin{document}
\maketitle
\begin{abstract}
	This text is a part of the series {\it Some Topics in Elementary STEM \& Beyond}:
	
	{\sc url}: \url{https://nqbh.github.io/elementary_STEM}.
	
	Latest version:
	\begin{itemize}
		\item {\it Problem: Dynamics -- Bài Tập: Động Lực Học}.
		
		PDF: {\sc url}: \url{https://github.com/NQBH/elementary_STEM_beyond/blob/main/elementary_physics/grade_10/dynamics/problem/NQBH_dynamics_problem.pdf}.
		
		\TeX: {\sc url}: \url{https://github.com/NQBH/elementary_STEM_beyond/blob/main/elementary_physics/grade_10/dynamics/problem/NQBH_dynamics_problem.tex}.
		\item {\it Problem \& Solution: Dynamics -- Bài Tập \& Lời Giải: Động Lực Học}.
		
		PDF: {\sc url}: \url{https://github.com/NQBH/elementary_STEM_beyond/blob/main/elementary_physics/grade_10/dynamics/solution/NQBH_dynamics_solution.pdf}.
		
		\TeX: {\sc url}: \url{https://github.com/NQBH/elementary_STEM_beyond/blob/main/elementary_physics/grade_10/dynamics/solution/NQBH_dynamics_solution.tex}.
	\end{itemize}
\end{abstract}
\tableofcontents

%------------------------------------------------------------------------------%

\section{\href{https://en.wikipedia.org/wiki/Dynamics_(mechanics)}{Wikipedia\texttt{/}Dynamics (Mechanics)}}
``{\it Dynamics} is the \href{https://en.wikipedia.org/wiki/Branch_(academia)#Physics}{branch} of \href{https://en.wikipedia.org/wiki/Classical_mechanics}{classical mechanics} that is concerned with the study of \href{https://en.wikipedia.org/wiki/Force_(physics)}{force} \& their effects on \href{https://en.wikipedia.org/wiki/Motion_(physics)}{motion}. \href{https://en.wikipedia.org/wiki/Isaac_Newton}{Isaac Newton} was the 1st to formulate the fundamental \href{https://en.wikipedia.org/wiki/Physical_law}{physical laws} that govern dynamics in classical non-relativistic physics, especially his \href{https://en.wikipedia.org/wiki/Second_law_of_motion}{2nd law of motion}.'' -- \href{https://en.wikipedia.org/wiki/Dynamics_(mechanics)}{Wikipedia\texttt{/}dynammics (mechanics)}

\subsection{Principles}
``Generally speaking, researchers involved in dynamics study how a physical system might develop or alter over time \& study the causes of those changes. In addition, Newton established the fundamental physical laws which govern dynamics in physics. By studying his system of mechanics, dynamics can be understood. In particular, dynamics is mostly related to Newton's 2nd law of motion. However, all 3 laws of motion are taken into account because these are interrelated in any given observation or experiment.'' -- \href{https://en.wikipedia.org/wiki/Dynamics_(mechanics)#Principles}{Wikipedia\texttt{/}dynamics (mechanics)\texttt{/}principles}

\subsection{Linear \& rotational dynamics}
``The study of dynamics falls under 2 categories: linear \& rotational. Linear dynamics pertains to objects moving in a line \& involves such quantities as \href{https://en.wikipedia.org/wiki/Force}{force}, \href{https://en.wikipedia.org/wiki/Mass}{mass}\texttt{/}\href{https://en.wikipedia.org/wiki/Inertia#Mass_and_inertia}{inertia}, \href{https://en.wikipedia.org/wiki/Displacement_(vector)}{displacement} (in units of distance), \href{https://en.wikipedia.org/wiki/Velocity}{velocity} (distance per unit time), \href{https://en.wikipedia.org/wiki/Acceleration}{acceleration} (distance per unit of time squared) \& \href{https://en.wikipedia.org/wiki/Momentum}{momentum} (mass times unit of velocity). Rotational dynamics pertains to obejcts that are rotating or moving in a curved path \& involves such quantities as \href{https://en.wikipedia.org/wiki/Torque}{torque}, \href{https://en.wikipedia.org/wiki/Moment_of_inertia}{moment of inertia}\texttt{/}\href{https://en.wikipedia.org/wiki/Rotational_inertia}{rotational inertia}, \href{https://en.wikipedia.org/wiki/Angular_displacement}{angular displacement} (in radians or less often, degrees), \href{https://en.wikipedia.org/wiki/Angular_velocity}{angular velocity} (radians per unit time), \href{https://en.wikipedia.org/wiki/Angular_acceleration}{angular acceleration} (radians per unit of time squared) \& \href{https://en.wikipedia.org/wiki/Angular_momentum}{angular momentum} (moment of inertia times unit of angular velocity). Very often, objects exhibit linear \& rotational motion.

For classical \href{https://en.wikipedia.org/wiki/Electromagnetism}{electromagnetism}, \href{https://en.wikipedia.org/wiki/Maxwell%27s_equations}{Maxwell's equations} describe the kinematics. The dynamics of classical systems involving both mechanics \& electromagnetism are described by the combination of Newton's laws, Maxwell's equations, \& the \href{https://en.wikipedia.org/wiki/Lorentz_force}{Lorentz force}.'' -- \href{https://en.wikipedia.org/wiki/Dynamics_(mechanics)#Linear_and_rotational_dynamics}{Wikipedia\texttt{/}dynamics (mechanics)\texttt{/}linear \& rotational dynamics}

\subsection{Force}
``Main article: \href{https://en.wikipedia.org/wiki/Force}{Wikipedia\texttt{/}force}. From Newton, force can be defined as an exertion or \href{https://en.wikipedia.org/wiki/Pressure}{pressure} which can cause an object to \href{https://en.wikipedia.org/wiki/Accelerate}{accelerate}. The concept of force is used to describe an influence which causes a \href{https://en.wikipedia.org/wiki/Free_body}{free body} (object) to accelerate. It can be a push or a pull, which causes an object to change direction, have new \href{https://en.wikipedia.org/wiki/Velocity}{velocity}, or to \href{https://en.wikipedia.org/wiki/Deformation_(mechanics)}{deform} temporarily or permanently. Generally speaking, force causes an object's \href{https://en.wikipedia.org/wiki/Motion_(physics)}{state of motion} to change.'' -- \href{https://en.wikipedia.org/wiki/Dynamics_(mechanics)#Force}{Wikipedia\texttt{/}dynamics (mechanics)\texttt{/}force}

\subsection{Newton's laws}
``Main article: \href{https://en.wikipedia.org/wiki/Newton%27s_laws_of_motion}{Wikipedia\texttt{/}Newton's laws of motion}. Newton described force as the ability to cause a mass to accelerate. His 3 laws can be summarized as follows:
\begin{itemize}
	\item {\it 1st law}: if there is no net force on an object, then its velocity is constant: either the object is at rest (if its velocity is equal to zero), or it moves with constant speed in a single direction.
	\item {\it 2nd law}: The rate of change of linear momentum ${\bf P}$ of an object is equal to the net force ${\bf F}_{\rm net}$, i.e., $\frac{{\rm d}{\bf P}}{{\rm d}t} = {\bf F}_{\rm net}$.
	\item {\it 3rd law}: When a 1st body exerts a force ${\bf F}_1$ on a 2nd body, the 2nd body simultaneously exerts a force ${\bf F}_2 = -{\bf F}_1$ on the 1st body. I.e., ${\bf F}_1$ \& ${\bf F}_2$ are equal in magnitude \& opposite in direction.
\end{itemize}
Newton's laws of motion are valid only in an \href{https://en.wikipedia.org/wiki/Inertial_frame_of_reference}{inertial frame of reference}.'' -- \href{https://en.wikipedia.org/wiki/Dynamics_(mechanics)#Newton's_laws}{Wikipedia\texttt{/}dynamics (mechanics)\texttt{/}Newton's laws}

%------------------------------------------------------------------------------%

\section{Tổng Hợp Lực \& Phân Tích Lực. Cân Bằng Lực}

\subsection{Tổng Hợp Lực}
\begin{itemize}
	\item ``{\it Tổng hợp lực} là phép thay thế các lực tác dụng đồng thời vào cùng 1 vật bằng 1 lực có tác dụng giống hệt như các lực ấy. Lực thay thế này gọi là {\it hợp lực}.
	\item Tổng hợp 2 lực cùng phương \& đồng quy đều tuân theo {\it quy tắc cộng vector}.
	\item Nếu các lực tác dụng lên 1 vật cân bằng nhau thì hợp lực tác dụng lên vật bằng $0$.
	\item Nếu các lực tác dụng lên 1 vật không cân bằng thì hợp lực tác dụng lên vật đó khác $0$. Khi đó, vận tốc của vật thay đổi (độ lớn, hướng).'' -- \cite[Chủ đề II: {\it Động Lực Học}, p. 19]{Giang_Hang_Trung_ncpt_Vat_Ly_10}
\end{itemize}

\subsection{Phân Tích Lực}
\begin{enumerate}
	\item ``{\it Phân tích lực} là phép thay thế 1 lực bằng 2 lực thành phần có tác dụng giống hệt lực đó.
	\item Khi phân tích lực thì 2 lực thành phần phải vuông góc với nhau để lực thành phần này không có tác dụng nào theo phương vuông góc của lực thành phần kia.'' -- \cite[p. 19]{Giang_Hang_Trung_ncpt_Vat_Ly_10}
\end{enumerate}

%------------------------------------------------------------------------------%

\section{Newton Laws -- Các Định Luật Newton}

\begin{dinhluat}[Định luật 1 Newton]
	``Nếu 1 vật không chịu tác dụng của lực nào hoặc chịu tác dụng của các lực có hợp lực bằng $0$, thì vật đang đứng yên sẽ tiếp tục đứng yên, đang chuyển động sẽ tiếp tục chuyển động thẳng đều: $\vec{F}_{\rm hl} = \vec{0}\Rightarrow\vec{a} = \vec{0}$.
\end{dinhluat}
Quán tính của vật là tính chất bảo toàn trạng thái đứng yên hay chuyển động.'' -- \cite[p. 19]{Giang_Hang_Trung_ncpt_Vat_Ly_10}

\begin{dinhluat}[Định luật 2 Newton]
	``Gia tốc của 1 vật cùng hướng với lực tác dụng lên vật. Độ lớn của gia tốc tỷ lệ thuận với độ lớn của lực \& tỷ lệ nghịch với khối lượng của vật. $\vec{a} = \frac{\vec{F}}{m}$ hay $\vec{a} = \frac{\vec{F}_{\rm hl}}{m}$.
\end{dinhluat}
Xét về mặt toán học, định luật 2 Newton có thể viết là $\vec{F} = m\vec{a}$. Khối lượng là đại lượng đặc trưng cho mức quán tính của vật.'' -- \cite[p. 19]{Giang_Hang_Trung_ncpt_Vat_Ly_10}

\begin{dinhluat}[Định luật 3 Newton]
	``Trong mọi trường hợp, khi vật A tác dụng lên vật B 1 lực, thì vật B cùng tác dụng trở lại vật A 1 lực. 2 lực này tác dụng theo cùng 1 phương, cùng độ lớn, nhưng ngược chiều, điểm đặt lên 2 vật khác nhau: $\vec{F}_{AB} = -\vec{F}_{BA}$.'' -- \cite[p. 20]{Giang_Hang_Trung_ncpt_Vat_Ly_10}
\end{dinhluat}

\section{Problem}

\begin{baitoan}[\cite{Giang_Hang_Trung_ncpt_Vat_Ly_10}, \textbf{2.1.}, p. 21]
	Cho 2 lực khác phương, có độ lớn bằng $9$\emph{N} \& $12$\emph{N}. Độ lớn của hợp lực có thể nhận giá trị nào sau đây?
	\begin{enumerate}
		\item[{\rm\sf A.}] $15$\emph{N}.
		\item[{\rm\sf B.}] $1$\emph{N}.
		\item[{\rm\sf C.}] $2$\emph{N}.
		\item[{\rm\sf D.}] $25$\emph{N}.
	\end{enumerate}
\end{baitoan}

\begin{baitoan}[\cite{Giang_Hang_Trung_ncpt_Vat_Ly_10}, \textbf{2.2.}, p. 21]
	Chất điểm chịu tác dụng của lực có độ lớn là $F_1$ \& $F_2 = 6$\emph{N}. Biết 2 lực này hợp với nhau góc $150^\circ$ \& hợp lực của chúng có giá trị nhỏ nhất. Giá trị của $F_1$ là: 
	\begin{enumerate}
		\item[{\rm\sf A.}] $2$\emph{N}.
		\item[{\rm\sf B.}] $3\sqrt{3}$\emph{N}.
		\item[{\rm\sf C.}] $3$\emph{N}.
		\item[{\rm\sf D.}] $5$\emph{N}.
	\end{enumerate}
\end{baitoan}

%------------------------------------------------------------------------------%

\printbibliography[heading=bibintoc]
	
\end{document}