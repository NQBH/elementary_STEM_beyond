\documentclass{article}
\usepackage[backend=biber,natbib=true,style=alphabetic,maxbibnames=50]{biblatex}
\addbibresource{/home/nqbh/reference/bib.bib}
\usepackage[utf8]{vietnam}
\usepackage{tocloft}
\renewcommand{\cftsecleader}{\cftdotfill{\cftdotsep}}
\usepackage[colorlinks=true,linkcolor=blue,urlcolor=red,citecolor=magenta]{hyperref}
\usepackage{amsmath,amssymb,amsthm,float,graphicx,mathtools,tikz}
\usetikzlibrary{angles,calc,intersections,matrix,patterns,quotes,shadings}
\allowdisplaybreaks
\newtheorem{assumption}{Assumption}
\newtheorem{baitoan}{}
\newtheorem{cauhoi}{Câu hỏi}
\newtheorem{conjecture}{Conjecture}
\newtheorem{corollary}{Corollary}
\newtheorem{dangtoan}{Dạng toán}
\newtheorem{definition}{Definition}
\newtheorem{dinhly}{Định lý}
\newtheorem{dinhnghia}{Định nghĩa}
\newtheorem{example}{Example}
\newtheorem{ghichu}{Ghi chú}
\newtheorem{hequa}{Hệ quả}
\newtheorem{hypothesis}{Hypothesis}
\newtheorem{lemma}{Lemma}
\newtheorem{luuy}{Lưu ý}
\newtheorem{nhanxet}{Nhận xét}
\newtheorem{notation}{Notation}
\newtheorem{note}{Note}
\newtheorem{principle}{Principle}
\newtheorem{problem}{Problem}
\newtheorem{proposition}{Proposition}
\newtheorem{question}{Question}
\newtheorem{remark}{Remark}
\newtheorem{theorem}{Theorem}
\newtheorem{vidu}{Ví dụ}
\usepackage[left=1cm,right=1cm,top=5mm,bottom=5mm,footskip=4mm]{geometry}
\def\labelitemii{$\circ$}
\DeclareRobustCommand{\divby}{%
	\mathrel{\vbox{\baselineskip.65ex\lineskiplimit0pt\hbox{.}\hbox{.}\hbox{.}}}%
}
\def\labelitemii{$\circ$}

\title{Problem: Deformation of Solids. Pressure of Fluids\\Bài Tập: Biến Dạng Của Vật Rắn. Áp Suất Của Chất Lỏng}
\author{Nguyễn Quản Bá Hồng\footnote{A Scientist {\it\&} Creative Artist Wannabe. E-mail: {\tt nguyenquanbahong@gmail.com}. Bến Tre City, Việt Nam.}}
\date{\today}

\begin{document}
\maketitle
\begin{abstract}
	This text is a part of the series {\it Some Topics in Elementary STEM \& Beyond}:
	
	{\sc url}: \url{https://nqbh.github.io/elementary_STEM}.
	
	Latest version:
	\begin{itemize}
		\item {\it Problem: Deformation of Solids. Pressure of Fluids -- Bài Tập: Biến Dạng Của Vật Rắn. Áp Suất Của Chất Lỏng}.
		
		PDF: {\sc url}: \url{https://github.com/NQBH/elementary_STEM_beyond/blob/main/elementary_physics/grade_10/deformation/problem/NQBH_deformation_solid_pressure_fluid_problem.pdf}.
		
		\TeX: {\sc url}: \url{https://github.com/NQBH/elementary_STEM_beyond/blob/main/elementary_physics/grade_10/deformation/problem/NQBH_deformation_solid_pressure_fluid_problem.tex}.
		\item {\it Problem \& Solution: Deformation of Solids. Pressure of Fluids -- Bài Tập \& Lời Giải: Biến Dạng Của Vật Rắn. Áp Suất Của Chất Lỏng}.
		
		PDF: {\sc url}: \url{https://github.com/NQBH/elementary_STEM_beyond/blob/main/elementary_physics/grade_10/deformation/solution/NQBH_deformation_solid_pressure_fluid_solution.pdf}.
		
		\TeX: {\sc url}: \url{https://github.com/NQBH/elementary_STEM_beyond/blob/main/elementary_physics/grade_10/deformation/solution/NQBH_deformation_solid_pressure_fluid_solution.tex}.
	\end{itemize}
\end{abstract}
\tableofcontents

%------------------------------------------------------------------------------%

\section{Basic}
\fbox{1} {\sf Biến dạng của vật rắn.} Khi chịu tác dụng bởi ngoại lực, hình dạng \& kích thước của vật thay đổi được gọi là {\it vật bị biến dạng}. Khi ngoại lực ngừng tác dụng, vật lấy lại hình dạng \& kích thước ban đầu, biến dạng của vật được gọi là {\it biến dạng đàn hồi}. Giới hạn để vật còn giữ được tính đàn hồi được gọi là {\it giới hạn đàn hồi}. \fbox{2} {\it Biến dạng kéo} xảy ra khi cặp lực tác dụng lên vật hướng ra ngoài vật, tác dụng của cặp lực này làm cho kích thước của vật tăng (dãn). \fbox{3} {\it Biến dạng nén} xảy ra khi cặp lực tác dụng lên vật hướng vào trong vật, tác dụng của cặp lực này làm cho kích thước của vật giảm (nén). \fbox{4} {\it Biến dạng đàn hồi} của 1 lò xo (hình trụ), có thể là biến dạng kéo hoặc biến dạng nén. Với 1 lò xo xác định, độ biến dạng của lò xo tỷ lệ thuận với độ cứng của lò xo. \fbox{5} {\it Định luật Hooke}\footnote{Hooke ăn cắp thành quả nghiên cứu khoa học của {\sc Isaac Newton}. Điều đó giải thích tại sao Newton có chữ `new' trong tên ông vì ông thích sáng tạo những cái mới, còn trong tên của Hooke thì lại có chữ `hook', i.e., kéo. Fucking hooker wannabe?}: $F_{\footnotesize\mbox{\rm đh}} = k|\Delta l|$ với $F_{\footnotesize\mbox{\rm đh}}$ N: độ lớn của lực đàn hồi, $k$ N{\tt/}m: độ cứng của lò xo, $\Delta l$ m: độ biến dạng của lò xo, nếu lò xo biến dạng kéo $\Delta l > 0$, nếu lò xo biến dạng nén $\Delta l < 0$. \fbox{6} {\sf Áp suất chất lỏng.} 1 vật có khối lượng $m$ kg, thể tích $V\ {\rm m}^3$ thì khối lượng riêng của vật $\rho\coloneqq\frac{m}{V}\ {\rm kg{\tt/}m^3}$. Áp lực tác dụng lên 1 bề mặt có diện tích $S\ {\rm m}^2$ là $F_N$ N thì áp suất gây ra trên bề mặt đó là $p = \frac{F_N}{S}$ Pa. Trong chất lỏng có khối lượng riêng $\rho\ {\rm kg{\tt/}m^3}$, áp suất khí quyển $p_{\rm a}$ hay $p_{\rm atm}$ (atm, abbr., atmosphere -- khi quyển), gia tốc rơi tự do $g\ {\rm m{\tt/}s^2}$ thì tại vị trí nằm cách mặt thoáng 1 khoảng $h$ m, áp suất của chất lỏng tại đó được tính bằng công thức $p = p_{\rm a} + \rho gh$; 2 điểm $M,N$ nằm cách mặt thoáng lần lượt là $h_1,h_2$ thì độ chênh lệch áp suất chất lỏng tại 2 điểm đó là $\Delta p = \rho g\Delta h$.

%------------------------------------------------------------------------------%

\section{Miscellaneous}

%------------------------------------------------------------------------------%

\printbibliography[heading=bibintoc]
	
\end{document}